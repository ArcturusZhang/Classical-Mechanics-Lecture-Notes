\chapter{刚体}

任意两点之间距离在运动过程中保持不变的体系即可称为{\heiti 刚体}。刚体是一个理想模型,在研究问题的过程中,形变可忽略的体系都可以视为刚体。

\section{刚体运动学}

\subsection{刚体的自由度}

首先将刚体看作多质点系,其中的质点满足约束
\begin{equation*}
	|\mbf{r}_i-\mbf{r}_j| = C_{ij}\,\text{(常数)},\quad i,j=1,2,\cdots,n\,(i\neq j)
\end{equation*}
因此,其中的任意不共线的三个质点的位置确定,其余各质点的位置也相应地确定。此三个质点之间有三个完整约束,即$n=3,k=3$,因此自由度$s=3n-k=6$。每增加一个质点,增加三个独立约束,因此自由度不变。

由此可得,仅有内部刚性约束而无外部附加约束的刚体的自由度$s=6$。

\subsection{刚体的运动}
\label{节:刚体的运动}
为了描述刚体的运动,需要引入三个坐标系:
\begin{itemize}
	\item {\heiti 空间系}:即实验室坐标系,为惯性系。
	\item {\heiti 本体系}:以刚体上任选的一点(称为{\heiti 基点})$O$为原点,并固定在刚体上的坐标系。刚体上各点相对本体系静止。
	\item {\heiti 平动系}:以基点$O$为原点,相对空间系平动的坐标系。
\end{itemize}

\begin{figure}[htb]
\centering
\begin{asy}
	size(250);
	//刚体的运动
	pair O,OO,r,i,j,k;
	picture tmp;
	O = (0,0);
	i = 0.5*dir(-135);
	j = dir(0);
	k = dir(90);
	draw(Label("$X_0$",EndPoint),O--i,Arrow);
	draw(Label("$Y_0$",EndPoint),O--j,Arrow);
	draw(Label("$Z_0$",EndPoint),O--k,Arrow);
	label("$O_0$",O,SSE);
	OO = (0.7,0.5);
	draw(tmp,Label("$X$",EndPoint,SE),O--i,Arrow);
	draw(tmp,Label("$Y$",EndPoint),O--j,Arrow);
	draw(tmp,Label("$Z$",EndPoint),O--k,Arrow);
	label(tmp,"$O$",O,SSE);
	add(shift(OO)*scale(0.9)*tmp);
	erase(tmp);
	real f(real x){
		return x**2-0.3;
	}
	draw(tmp,graph(f,-1.5,1.5),linewidth(1bp));
	draw(tmp,shift((0,f(1.5)))*yscale(1/3)*scale(1.5)*unitcircle,linewidth(1bp));
	add(shift(OO)*rotate(-25)*scale(0.4)*tmp);
	erase(tmp);
	draw(tmp,Label("$x$",EndPoint,S),rotate(-25)*(O--i),Arrow);
	draw(tmp,Label("$y$",EndPoint),rotate(-25)*(O--j),Arrow);
	draw(tmp,Label("$z$",EndPoint,W),rotate(-25)*(O--k),Arrow);
	add(shift(OO)*scale(0.9)*tmp);
	r = (-0.1,0.6);
	//dot(OO+r);
	draw(Label("$\boldsymbol{r}_O$",Relative(0.3),Relative(E),black),O--OO,red,Arrow);
	draw(Label("$\boldsymbol{r}'$",Relative(0.3),Relative(W),black),OO--OO+r,red,Arrow);
	draw(Label("$\boldsymbol{r}$",Relative(0.5),Relative(W),black),O--OO+r,red,Arrow);
	//draw(O--(1.8,1.55),invisible);
\end{asy}
\caption{刚体的运动}
\label{刚体的运动}
\end{figure}

则刚体在这三个坐标系中的运动为(见图\ref{刚体的运动}):
\begin{itemize}
	\item {\heiti 本体系}:刚体静止。
	\item {\heiti 平动系}:刚体做定点运动(基点固定)。
	\item {\heiti 空间系}:刚体随基点平动的同时,绕基点做定点运动。
\end{itemize}

如果基点在空间系内不动,则平动系与空间系合一,刚体无平动,只有定点运动。即刚体平动自由度$s_t = 3$,定点运动自由度$s_r=3$。

\subsection{定点运动的矩阵表示}

\begin{figure}[htb]
\centering
\begin{asy}
	size(250);
	//坐标系的定点旋转
	pair O,i,j,k;
	O = (0,0);
	i = 0.5*dir(-135);
	j = dir(0);
	k = dir(90);
	pair xyz(triple coff){
		real x,y,z,r,theta,phi;
		r = coff.x;
		theta = coff.y;
		phi = coff.z;
		x = r*sin(theta)*cos(phi);
		y = r*sin(theta)*sin(phi);
		z = r*cos(theta);
		return x*i+y*j+z*k;
	}
	pair t2t(triple xyz){
		return xyz.x*i+xyz.y*j+xyz.z*k;
	}
	draw(Label("$X_1$",EndPoint,S),O--i,Arrow);
	draw(Label("$X_2$",EndPoint),O--j,Arrow);
	draw(Label("$X_3$",EndPoint),O--k,Arrow);
	label("$O$",O,SSE);
	//dot(xyz((1,pi/4,pi/3)),red);
	draw(Label("$\boldsymbol{R}$",MidPoint,Relative(E),black),O--xyz((1,pi/4,pi/3)),red,Arrow);
	draw(Label("$x_3$",EndPoint,black),O--xyz((1,pi/6,pi/3)),blue,Arrow);
	draw(Label("$x_2$",EndPoint,black),O--xyz((1,pi/6+pi/2,pi/3)),blue,Arrow);
	draw(Label("$x_1$",EndPoint,black),O--xyz((1,pi/2,pi/3-pi/2)),blue,Arrow);
	//draw(O--(1.15,0),invisible);
\end{asy}
\caption{坐标系的定点旋转}
\label{坐标系的定点旋转}
\end{figure}

设有矢量$\mbf{R}$,则其在坐标系$OX_1X_2X_3$和$Ox_1x_2x_3$中可以分别表示为
\begin{equation*}
	\mbf{R} = \sum_{i=1}^3 X_i \mbf{E}_i = \sum_{i=1}^3 x_j \mbf{e}_j
\end{equation*}
同理可有
\begin{equation*}
	\mbf{E}_i = \sum_{j=1}^3 (\mbf{E}_i\cdot \mbf{e}_j) \mbf{e}_j = \sum_{j=1}^3 u_{ij} \mbf{e}_j
\end{equation*}
此处$u_{ij} = \mbf{E}_j\cdot \mbf{e}_j$。因此有
\begin{equation}
	X_i = \mbf{E}_i \cdot \mbf{R} = \sum_{j=1}^3 u_{ij} (\mbf{e}_j \cdot \mbf{R}) = \sum_{j=1}^3 u_{ij} x_j
	\label{坐标转换关系}
\end{equation}
式\eqref{坐标转换关系}用矩阵可以表示为
\begin{equation}
	\mbf{X} = \mbf{U}\mbf{x}
\end{equation}
其中
\begin{equation*}
	\mbf{X} = \begin{pmatrix} X_1 \\ X_2 \\ X_3 \end{pmatrix},\quad \mbf{x} = \begin{pmatrix} x_1 \\ x_2 \\ x_3 \end{pmatrix},\quad \mbf{U} = \begin{pmatrix} u_{11} & u_{12} & u_{13} \\ u_{21} & u_{22} & u_{23} \\ u_{31} & u_{32} & u_{33} \end{pmatrix}
\end{equation*}
坐标系的定点运动需要满足矢量的长度不变,即
\begin{equation*}
	\mbf{R}^{\mathrm{T}} \mbf{R} = \mbf{X}^{\mathrm{T}} \mbf{X} = \mbf{x}^{\mathrm{T}} \mbf{x}
\end{equation*}
所以有
\begin{equation*}
	\mbf{X}^{\mathrm{T}} \mbf{X} = \mbf{x}^{\mathrm{T}} \mbf{U}^{\mathrm{T}} \mbf{U} \mbf{x} = \mbf{x}^{\mathrm{T}} \mbf{x}
\end{equation*}
即
\begin{equation*}
	\mbf{U}^{\mathrm{T}} \mbf{U} = \mbf{I}
\end{equation*}
因此,系数矩阵$\mbf{U}$是正交矩阵,其独立矩阵元有3个。

对于刚体的定点运动,本体系内刚体的各点坐标不变,将某参考点的坐标记作$\mbf{x}$。设在$t_1,t_2$时刻,刚体上该参考点在平动系的坐标分别为$\mbf{X}_1,\mbf{X}_2$,则有
\begin{equation*}
	\mbf{X}_1 = \mbf{U}_1 \mbf{x},\quad \mbf{X}_2 = \mbf{U}_2 \mbf{x}
\end{equation*}
此处,$\mbf{U}_1,\mbf{U}_2$都是正交矩阵,所以有
\begin{equation*}
	\mbf{X}_2 = \mbf{U}_2 \mbf{U}_1^{-1} \mbf{X}_1 = (\mbf{U}_2 \mbf{U}_1^{\mathrm{T}}) \mbf{X}_1 = \mbf{U} \mbf{X}_1
\end{equation*}
其中矩阵$\mbf{U}$满足
\begin{equation*}
	\mbf{U}^{\mathrm{T}} \mbf{U} = \mbf{U}_1 \mbf{U}_2^{\mathrm{T}} \mbf{U}_2 \mbf{U}_1^{\mathrm{T}} = \mbf{I}
\end{equation*}
因此,矩阵$\mbf{U}$也是正交矩阵。由此可得,刚体的定点运动可用正交矩阵表示。

\iffalse
\begin{example}[有限角位移的非对易性]

\begin{figure}[htb]
\centering
\begin{asy}
	size(200);
	//刚体的定轴转动
	pair O,n,R,RR;
	real theta,l,r,r1,r2;
	path cir;
	picture tmp;
	O = (0,0);
	theta = 70;
	n = dir(theta);
	l = 2;
	r = 0.8;
	r1 = -0.2;
	r2 = -0.05;
	draw(Label("$\boldsymbol{n}$",EndPoint),O--O+l*n,Arrow);
	cir = scale(r)*unitcircle;
	draw(tmp,cir);
	R = relpoint(cir,r1);
	RR = relpoint(cir,r2);
	draw(tmp,R--O--RR);
	draw(tmp,R--RR,red,Arrow);
	draw(tmp,Label(rotate(90-theta)*"$\phi$",EndPoint,Relative(E)),arc(O,0.2,degrees(R),degrees(RR)),Arrow);
	add(shift(2/3*l*n)*rotate(theta-90)*yscale(1/3)*tmp);
	R = shift(2/3*l*n)*rotate(theta-90)*yscale(1/3)*R;
	RR = shift(2/3*l*n)*rotate(theta-90)*yscale(1/3)*RR;
	//draw(R--2/3*l*n--RR);
	//draw(R--RR,red,Arrow);
	draw(Label("$\boldsymbol{R}$",MidPoint,Relative(E)),O--R,Arrow);
	draw(Label("$\boldsymbol{R}'$",MidPoint,Relative(E)),O--RR,Arrow);
	//draw(O--(1.3,0),invisible);
\end{asy}
\caption{刚体的定轴转动}
\label{刚体的定轴转动}
\end{figure}

考虑刚体绕指向为$\mbf{n}$的转动轴旋转$\theta$,即{\heiti 定轴转动},如图\ref{刚体的定轴转动}所示。定轴转动是定点运动的特例,因此可用矩阵表示为
\begin{equation*}
	\mbf{X}' = \mbf{U}\mbf{X}
\end{equation*}
如果进行两次定轴转动(转轴指向可以不同),那么可有
\begin{equation*}
	\mbf{X}'' = \mbf{U}' \mbf{X}' = \mbf{U}' \mbf{U} \mbf{X}
\end{equation*}
由于矩阵的乘积具有不可交换性,因此有限角位移一般不具有对易性,因而不能用矢量表示。
\end{example}\fi

\begin{example}[有限定点运动的非对易性]

刚体的定点运动可用矩阵表示为
\begin{equation*}
	\mbf{X}' = \mbf{U}\mbf{X}
\end{equation*}
如果再进行一次定轴运动,那么可有
\begin{equation*}
	\mbf{X}'' = \mbf{U}' \mbf{X}' = \mbf{U}' \mbf{U} \mbf{X}
\end{equation*}
由于矩阵的乘积具有不可交换性,因此有限定点运动一般不具有对易性,从而有限定点转动不能用矢量表示。
\end{example}

\subsection{无限小定点运动的矢量表示}

有限定点运动可以由连续的无限小定点运动合成。考虑
\begin{equation*}
	\mbf{X}' = \mbf{U}\mbf{X}
\end{equation*}
满足
\begin{equation*}
	\mathrm{d} \mbf{X}' = \mbf{X}' - \mbf{X} = (\mbf{U}-\mbf{I}) \mbf{X}
\end{equation*}
是一阶无穷小量,即有
\begin{equation*}
	\mbf{U} = \mbf{I}+\mbf{u}
\end{equation*}
此处$\mbf{u}$为一阶无穷小量。$\mbf{U}$应为正交矩阵,即有
\begin{equation*}
	\mbf{I} = (\mbf{I}+\mbf{u}^{\mathrm{T}})(\mbf{I}+\mbf{u}) = \mbf{I} + \mbf{u}^{\mathrm{T}} + \mbf{u} + \mbf{u}^{\mathrm{T}} \mbf{u}
\end{equation*}
略去上式中的高阶无穷小量,可以得到
\begin{equation}
	\mbf{u}^{\mathrm{T}} = -\mbf{u}
\end{equation}
即无穷小定点运动的变换矩阵为反对称矩阵,将其记作
\begin{equation*}
	\mbf{u} = \begin{pmatrix} 0 & -\eps_3 & \eps_2 \\ \eps_3 & 0 & -\eps_1 \\ -\eps_2 & \eps_1 & 0 \end{pmatrix}
\end{equation*}
所以有
\begin{equation*}
	\mathrm{d} \mbf{X} = \mbf{u} \mbf{X} = \begin{pmatrix} \eps_2 X_3 - \eps_3 X_2 \\ \eps_3 X_1 - \eps_1 X_3 \\ \eps_1 X_2 - \eps_2 X_1 \end{pmatrix} = \mbf{\eps} \times \mbf{R}
\end{equation*}
此处$\mbf{\eps} = \begin{pmatrix} \eps_1 \\ \eps_2 \\ \eps_3 \end{pmatrix}$称为{\heiti 无限小角位移矢量}。

将无限小角位移矢量表示为$\mbf{\eps} = \eps \mbf{n}$,其中$\mbf{n}$是单位矢量,取位矢$\mbf{R} = R\mbf{n}$,则在此无限小定点运动下的位移为
\begin{equation*}
	\mathrm{d} \mbf{R} = \eps \mbf{n} \times R\mbf{n} = \mbf{0}
\end{equation*}
上式说明,此定点运动下,所有位于过定点方向为$\mbf{n}$的直线上的点都是固定的,即$\mbf{n}$为转轴。因此无限小定点运动是绕瞬时轴的无限小转动,定点运动即为{\heiti 定点转动}。

\subsection{刚体运动的矢量-矩阵描述}

\label{节:刚体运动的矢量-矩阵描述}
\ref{节:刚体的运动}节已指出,刚体的运动可由平动系相对于空间坐标系的平动与本体系相对于平动系的定点转动,即刚体中任意一点的位矢为
\begin{equation}
	\mbf{r} = \mbf{r}_O + \mbf{r}'
	\label{刚体中任意一点的位矢的矢量表示}
\end{equation}
设矢量$\mbf{r}'$在平动系中的坐标为$\mbf{X}$,在本体系中的坐标为$\mbf{x}$\footnote{需要注意到,平动系中的坐标$\mbf{X}$是与时间有关的,而本体系中的坐标$\mbf{x}$则与时间无关。},则它们之间通过旋转矩阵可以表示为
\begin{equation*}
	\mbf{X} = \mbf{U} \mbf{x}
\end{equation*}
此处矩阵$\mbf{U} = \mbf{U}(t)$是正交矩阵,即矩阵$\mbf{U}$满足
\begin{equation}
	\mbf{U}^{\mathrm{T}} \mbf{U} = \mbf{I}
	\label{正交矩阵的条件}
\end{equation}
记平动系的基为$\mbf{E}_1,\mbf{E}_2,\mbf{E}_3$,本体系的基为$\mbf{e}_1,\mbf{e}_2,\mbf{e}_3$\footnote{需要注意到,平动系中的基$\mbf{E}_1,\mbf{E}_2,\mbf{E}_3$是与时间无关的,而本体系中的基$\mbf{e}_1,\mbf{e}_2,\mbf{e}_3$则与时间有关。},则有
\begin{align*}
	\mbf{r}' & = \begin{pmatrix} \mbf{E}_1 & \mbf{E}_2 & \mbf{E}_3 \end{pmatrix} \begin{pmatrix} X_1 \\ X_2 \\ X_3 \end{pmatrix} = \mbf{E} \mbf{X} = \begin{pmatrix} \mbf{e}_1 & \mbf{e}_2 & \mbf{e}_3 \end{pmatrix} \begin{pmatrix} x_1 \\ x_2 \\ x_3 \end{pmatrix} = \mbf{e} \mbf{x}
\end{align*}
由此可以得到
\begin{equation*}
	\mbf{e} = \mbf{E} \mbf{U}
\end{equation*}
综上,式\eqref{刚体中任意一点的位矢的矢量表示}可以用矩阵表示为\footnote{在用矩阵表示矢量表达式时,如不同时乘以该坐标系基构成的矩阵,则容易造成谬误。}
\begin{align}
	\mbf{r} & = \mbf{r}_O + \mbf{E} \mbf{X} = \mbf{r}_O + \mbf{E} \mbf{U} \mbf{x}
	\label{刚体中任意一点的位矢}
\end{align}

\begin{figure}[htb]
\centering
\begin{asy}
	size(250);
	//刚体的运动
	pair O,OO,r,i,j,k;
	picture tmp;
	O = (0,0);
	i = 0.5*dir(-135);
	j = dir(0);
	k = dir(90);
	draw(Label("$X_0$",EndPoint),O--i,Arrow);
	draw(Label("$Y_0$",EndPoint),O--j,Arrow);
	draw(Label("$Z_0$",EndPoint),O--k,Arrow);
	label("$O_0$",O,SSE);
	OO = (0.7,0.5);
	draw(tmp,Label("$X$",EndPoint,SE),O--i,Arrow);
	draw(tmp,Label("$Y$",EndPoint),O--j,Arrow);
	draw(tmp,Label("$Z$",EndPoint),O--k,Arrow);
	label(tmp,"$O$",O,SSE);
	add(shift(OO)*scale(0.9)*tmp);
	erase(tmp);
	real f(real x){
		return x**2-0.3;
	}
	draw(tmp,graph(f,-1.5,1.5),linewidth(1bp));
	draw(tmp,shift((0,f(1.5)))*yscale(1/3)*scale(1.5)*unitcircle,linewidth(1bp));
	add(shift(OO)*rotate(-25)*scale(0.4)*tmp);
	erase(tmp);
	draw(tmp,Label("$x$",EndPoint,S),rotate(-25)*(O--i),Arrow);
	draw(tmp,Label("$y$",EndPoint),rotate(-25)*(O--j),Arrow);
	draw(tmp,Label("$z$",EndPoint,W),rotate(-25)*(O--k),Arrow);
	add(shift(OO)*scale(0.9)*tmp);
	r = (-0.1,0.6);
	//dot(OO+r);
	draw(Label("$\boldsymbol{r}_O$",Relative(0.3),Relative(E),black),O--OO,red,Arrow);
	draw(Label("$\boldsymbol{r}'$",Relative(0.3),Relative(W),black),OO--OO+r,red,Arrow);
	draw(Label("$\boldsymbol{r}$",Relative(0.5),Relative(W),black),O--OO+r,red,Arrow);
	//draw(O--(1.8,1.55),invisible);
\end{asy}
\caption{刚体的运动}
\label{刚体的运动2}
\end{figure}

\label{定点转动矩阵与基点的选取无关}考虑刚体上的另外一点$O'$,则有
\begin{equation*}
	\mbf{r}_{O'} = \mbf{r}_O + \mbf{E}\mbf{U}\mbf{x}_{O'}
\end{equation*}
于是可有
\begin{equation}
	\mbf{r} = \mbf{r}_O + \mbf{E}\mbf{U}\mbf{x} = \mbf{r}_{O'} - \mbf{E}\mbf{U} \mbf{x}_{O'} + \mbf{E}\mbf{U} \mbf{x} = \mbf{r}_{O'} + \mbf{E} \mbf{U} (\mbf{x}-\mbf{x}_{O'})
	\label{定点转动矩阵与基点的选取无关1}
\end{equation}
将$O'$取为基点,各坐标系坐标轴的方向不变,则同样有刚体中任意一点的位矢可以表示为
\begin{equation}
	\mbf{r} = \mbf{r}_{O'} + \mbf{E} \mbf{U}' \mbf{x}'
	\label{定点转动矩阵与基点的选取无关2}
\end{equation}
对比式\eqref{定点转动矩阵与基点的选取无关1}和\eqref{定点转动矩阵与基点的选取无关2},并考虑到$\mbf{x}'=\mbf{x}-\mbf{x}_{O'}$,可有
\begin{equation*}
	\mbf{U} =\mbf{U}'
\end{equation*}
即,{\heiti 刚体定点转动矩阵与基点的选取无关}。

\subsection{刚体上各点的速度与加速度}
\label{节:刚体上各点的速度与加速度}
将式\eqref{刚体中任意一点的位矢}两端对时间$t$求导数,得
\begin{equation*}
	\mbf{v} = \mbf{v}_O + \mbf{E} \dot{\mbf{U}} \mbf{x} = \mbf{v}_O + \mbf{e} \mbf{U}^{\mathrm{T}} \dot{\mbf{U}} \mbf{x}
\end{equation*}
将式\eqref{正交矩阵的条件}两端对时间$t$求导数,得
\begin{equation*}
	\dot{\mbf{U}}^{\mathrm{T}} \mbf{U} + \mbf{U}^{\mathrm{T}} \dot{\mbf{U}} = \left(\mbf{U}^{\mathrm{T}} \dot{\mbf{U}}\right)^{\mathrm{T}} + \mbf{U}^{\mathrm{T}} \dot{\mbf{U}} = \mbf{0}
\end{equation*}
即矩阵$\mbf{U}^{\mathrm{T}} \dot{\mbf{U}}$是反对称矩阵,因此可以将其表示为
\begin{equation*}
	\mbf{U}^{\mathrm{T}} \dot{\mbf{U}} = \begin{pmatrix} 0 & -\omega_3 & \omega_2 \\ \omega_3 & 0 & -\omega_1 \\ -\omega_2 & \omega_1 & 0 \end{pmatrix}
\end{equation*}
由此,记矢量$\mbf{\omega} = \omega_1 \mbf{e}_1 + \omega_2 \mbf{e}_2 + \omega_3 \mbf{e}_3$称为刚体的{\heiti 角速度},则有
\begin{equation*}
	\mbf{e} \mbf{U}^{\mathrm{T}} \dot{\mbf{U}} \mbf{x} = \begin{pmatrix} \mbf{e}_1 & \mbf{e}_2 & \mbf{e}_3 \end{pmatrix} \begin{pmatrix} \omega_2 x_3 - \omega_3 x_2 \\ \omega_3 x_1 - \omega_1 x_3 \\ \omega_1 x_2 - \omega_2 x_1 \end{pmatrix} = \mbf{\omega} \times \mbf{r}'
\end{equation*}
所以刚体上任意一点的速度可以表示为
\begin{equation}
	\mbf{v} = \mbf{v}_O + \mbf{\omega} \times \mbf{r}' = \mbf{v}_O + \mbf{\omega} \times (\mbf{r}-\mbf{r}_O)
	\label{刚体上任意一点的速度}
\end{equation}
另取基点$O'$,则当以$O$为基点时,$O'$的速度为
\begin{equation*}
	\mbf{v}_{O'} = \mbf{v}_O + \mbf{\omega} \times (\mbf{r}_{O'} - \mbf{r}_O)
\end{equation*}
于是,以$O'$为基点表示刚体上各点的速度可有
\begin{align*}
	\mbf{v} = \mbf{v}_O + \mbf{\omega} \times (\mbf{r}-\mbf{r}_O) = \mbf{v}_{O'} + \mbf{\omega} \times (\mbf{r}-\mbf{r}_{O'})
\end{align*}
上式说明{\heiti 刚体的角速度与基点的选取无关}。

下面考虑刚体上各点的加速度。首先定义{\heiti 角加速度}为
\begin{equation}
	\mbf{\alpha} = \frac{\mathrm{d} \mbf{\omega}}{\mathrm{d} t} = \dot{\mbf{\omega}}
\end{equation}
将式\eqref{刚体上任意一点的速度}两端对时间$t$求导数,得
\begin{equation*}
	\mbf{a} = \mbf{a}_O + \dot{\mbf{\omega}} \times (\mbf{r}-\mbf{r}_O) + \mbf{\omega} \times (\mbf{v}-\mbf{v}_O)
\end{equation*}
由此即有刚体上任意一点的加速度
\begin{equation}
	\mbf{a} = \mbf{a}_O + \mbf{\alpha} \times (\mbf{r}-\mbf{r}_O) + \mbf{\omega} \times \left[\mbf{\omega} \times (\mbf{r}-\mbf{r}_O)\right]
	\label{刚体上任意一点的加速度}
\end{equation}

\subsection{定轴转动}

\begin{figure}[htb]
\centering
\begin{minipage}[t]{0.45\textwidth}
\centering
\begin{asy}
	size(200);
	//刚体的定轴转动
	pair O,i,j,k,ii,jj,P;
	real a,r,alpha;
	O = (0,0);
	a = 1;
	alpha = 3;
	i = (-a/sqrt(1+alpha**2),-a/sqrt(1+alpha**2));
	j = (a,0);
	k = (0,a);
	draw(Label("$X$",EndPoint),O--2*i,Arrow);
	draw(Label("$Y$",EndPoint),O--2*j,Arrow);
	draw(Label("$Z$",EndPoint),O--2*k,Arrow);
	label("$O$",O,W);
	path cir;
	cir = yscale(1/alpha)*scale(a)*unitcircle;
	draw(cir,dashed);
	ii = relpoint(cir,-0.16);
	jj = relpoint(cir,0.13);
	draw(Label("$x$",EndPoint),O--2*ii,Arrow);
	draw(Label("$y$",EndPoint),O--2*jj,Arrow);
	dot(1.0*k);
	label("$O_1$",1.0*k,W);
	draw(Label("$z$",EndPoint,W),O--1.7*k,Arrow);
	r = 0.3;
	real d2tan(pair de){
		return atan(alpha*de.y/de.x)/pi*180;
	}
	P = a*cos(pi/3)*i+a*sin(pi/3)*j+1.2*k;
	draw(Label("$\boldsymbol{r}$",MidPoint,Relative(W)),O--P,red,Arrow);
	draw(P--(P.x,-a/alpha*sqrt(1-P.x*P.x/a/a))--O,red+dashed);
	draw(P--(P-(P.x,-a/alpha*sqrt(1-P.x*P.x/a/a))),red+dashed);
	label("$P$",P,N);
	draw(Label("$\phi$",MidPoint,Relative(E)),yscale(1/alpha)*arc(O,r,d2tan(i)-180,d2tan(ii)),Arrow);
	draw(Label("$\phi$",MidPoint,E),yscale(1/alpha)*arc(O,r,d2tan(j),d2tan(jj)),Arrow);
\end{asy}
\caption{刚体的定轴转动}
\label{刚体的定轴转动}
\end{minipage}
\hspace{0.5cm}
\begin{minipage}[t]{0.45\textwidth}
\centering
\begin{asy}
	size(200);
	//向心加速度和切向加速度
	pair O,OO,i,j,k,P,tau,n;
	real r,phi;
	O = (0,0);
	i = (-sqrt(2)/4,-sqrt(14)/12);
	j = (sqrt(14)/4,-sqrt(2)/12);
	k = (0,2*sqrt(2)/3);
	pair cir(real theta){
		real xx,yy;
		xx = r*cos(theta);
		yy = r*sin(theta);
		return xx*i+yy*j;
	}
	r = 1;
	draw(graph(cir,0,2*pi));
	draw(Label("$z$",EndPoint),O--1.5*k,Arrow);
	draw(Label("$d$",MidPoint,Relative(W)),O--(-1*j),dashed);
	OO = -1.5*k;
	label("$O$",OO,W);
	draw(Label("$\boldsymbol{\omega}$",EndPoint,W),OO--k,linewidth(0.6bp),Arrow);
	draw(Label("$\boldsymbol{\alpha}$",EndPoint,W),OO--(-0.7*k),linewidth(0.6bp),Arrow);
	phi = 65*pi/180;
	P = r*cos(phi)*i+r*sin(phi)*j;
	tau = -sin(phi)*i+cos(phi)*j;
	n = -0.5*P;
	label("$P$",P,SE);
	draw(O--P,dashed);
	draw(Label("$\boldsymbol{r}$",MidPoint,Relative(E)),OO--P,linewidth(0.6bp),Arrow);
	draw(Label("$\boldsymbol{a}_\tau$",EndPoint,Relative(E)),P--P+0.7*tau,linewidth(0.6bp),Arrow);
	draw(Label("$\boldsymbol{v}$",EndPoint),P--P+tau,linewidth(0.6bp),Arrow);
	draw(Label("$\boldsymbol{a}_n$",EndPoint,Relative(W)),P--P+n,linewidth(0.6bp),Arrow);
	draw(Label("$\boldsymbol{a}$",EndPoint,Relative(W)),P--P+n+0.7*tau,linewidth(0.6bp),Arrow);
	draw(P+0.7*tau--P+0.7*tau+n--P+n,dashed);
\end{asy}
\caption{向心加速度和切向加速度}
\label{向心加速度和切向加速度}
\end{minipage}
\end{figure}

设刚体上有两个不动点$O$和$O_1$,则过$O$和$O_1$的直线即为转轴,平动系的$OZ$轴和本体系的$Oz$都沿着转轴。刚体在平动系中的位形由$OX$轴和$Ox$轴之间的夹角$\phi$确定,刚体上不在转动轴上的点沿着以转轴为圆心的圆周运动,设此时刚体上任意一点的矢径在本体系中的坐标为$\mbf{x}$,在平动系的坐标为$\mbf{X}$,则有
\begin{equation*}
	\mbf{X} = \mbf{U} \mbf{x}
\end{equation*}
此处转动矩阵为
\begin{equation}
	\mbf{U} = \begin{pmatrix} \cos \phi & -\sin \phi & 0 \\ \sin \phi & \cos \phi & 0 \\ 0 & 0 & 1 \end{pmatrix}
	\label{定轴转动的转动矩阵}
\end{equation}
直接计算可以验证
\begin{equation*}
	\mbf{U}^{\mathrm{T}} \dot{\mbf{U}} = \begin{pmatrix} 0 & -\dot{\phi} & 0 \\ \dot{\phi} & 0 & 0 \\ 0 & 0 & 0 \end{pmatrix}
\end{equation*}
因此角速度和角加速度为
\begin{equation*}
	\mbf{\omega} = \begin{pmatrix} 0 \\ 0 \\ \dot{\phi} \end{pmatrix} = \dot{\phi} \mbf{e}_3,\quad \mbf{\alpha} = \dot{\mbf{\omega}} = \begin{pmatrix} 0 \\ 0 \\ \ddot{\phi} \end{pmatrix} = \ddot{\phi} \mbf{e}_3
\end{equation*}
即角速度和角加速度都沿着转轴方向。由此根据式\eqref{刚体上任意一点的速度}可得刚体上任意一点的速度
\begin{equation*}
	\mbf{v} = \mbf{\omega} \times \mbf{r}
\end{equation*}
根据式\eqref{刚体上任意一点的加速度}可得刚体上任意一点的加速度
\begin{equation*}
	\mbf{a} = \mbf{\alpha} \times \mbf{r} + \mbf{\omega} \times (\mbf{\omega} \times \mbf{r})
\end{equation*}
此处的加速度$\mbf{a}$可以分解为向心加速度$\mbf{a}_n = \mbf{\omega} \times (\mbf{\omega} \times \mbf{r})$和切向加速度$\mbf{a}_\tau = \mbf{\alpha} \times \mbf{r}$。

\begin{figure}[htb]
\centering
\begin{asy}
	size(300);
	//定轴转动公式
	real r,theta,a,h;
	r = 1;
	a = 0.5;
	theta = 20;
	h = 3;
	path p;
	p = rotate(-theta)*shift((0,h))*scale(r)*yscale(a)*unitcircle;
	draw(p,linewidth(0.8bp));
	draw(Label("$N$",EndPoint,Relative(W)),(0,0)--h*dir(90-theta),Arrow);
	draw(h*dir(90-theta)--(h+1)*dir(90-theta));
	draw(Label("$\boldsymbol{n}$",EndPoint,Relative(W)),(0,0)--dir(90-theta),Arrow);
	label("$(\boldsymbol{n}\cdot\boldsymbol{r})\boldsymbol{n}$",0.6*h*dir(90-theta),dir(180-theta));
	pair NP,P,Q;
	NP = h*dir(90-theta);
	P = relpoint(p,0);
	Q = relpoint(p,0.9);
	draw(Label("$P$",EndPoint),NP--P,Arrow);
	draw(Label("$Q$",EndPoint),NP--Q,Arrow);
	draw(Label("$\boldsymbol{r}$",Relative(0.7),Relative(E)),(0,0)--P,Arrow);
	draw(Label("$\boldsymbol{r}'$",Relative(0.7),Relative(W)),(0,0)--Q,Arrow);
	real R = 0.3;
	draw(Label("$\phi$",MidPoint,Relative(W)),arc(NP,R,degrees(P-NP),degrees(Q-NP)),Arrow);
	draw(Label("$V$",BeginPoint),NP+0.7*(P-NP)--Q,Arrow);
	picture pic;
	r = 1.3;
	path q;
	q = scale(r)*unitcircle;
	pair P,Q;
	draw(pic,q,linewidth(0.8bp));
	P = relpoint(q,0);
	Q = relpoint(q,0.86);
	draw(pic,Label("$P$",EndPoint),(0,0)--P,Arrow);
	draw(pic,Label("$Q$",EndPoint),(0,0)--Q,Arrow);
	draw(pic,Label("$\boldsymbol{r}\times\boldsymbol{n}$",MidPoint,Relative(E)),(0,0)--r*dir(-90),Arrow);
	draw(pic,Label("$\phi$",MidPoint,Relative(W)),arc((0,0),R,0,degrees(Q)-360),Arrow);
	draw(pic,Label("$V$",BeginPoint),(Q.x,0)--Q,Arrow);
	label(pic,"$N$",(0,0),NW);
	add(shift((2.6+r,r+0.5))*pic);
\end{asy}
\caption{定轴转动公式}
\label{定轴转动公式}
\end{figure}

设转轴方向为$\mbf{n}$($\mbf{n}$是单位向量),转动角为$\phi$(顺时针方向),则根据图\ref{定轴转动公式}中的几何关系可有
\begin{align*}
	\mbf{r}' & = \overrightarrow{ON} + \overrightarrow{NV} + \overrightarrow{VQ} \\
	& = (\mbf{n}\cdot \mbf{r})\mbf{n} + \big[\mbf{r}-(\mbf{n}\cdot\mbf{r})\mbf{n}\big] \cos \phi + (\mbf{r}\times \mbf{n})\sin \phi
\end{align*}
由此可得{\heiti Rodrigues公式}
\begin{equation}
	\mbf{r}' = \mbf{r}\cos \phi + (\mbf{n}\cdot \mbf{r})\mbf{n}(1-\cos \phi) + (\mbf{r}\times \mbf{n})\sin \phi
	\label{Rodrigues公式}
\end{equation}

\subsection{刚体定点转动的Euler角}

刚体定点转动的自由度$s_r = 3$,可用$3$个广义坐标确定本体系相对平动系的取向。

\begin{figure}[htb]
\centering
\begin{asy}
	size(350);
	//刚体定点转动的Euler角
	real covert(pair x,real a){
		if (x.x==0) return 90;
		else return atan(a*x.y/x.x)*180/pi;
	}
	picture elpic1,tmp;
	pair O1,O,x[],y[],z[];
	real a,alpha1,alpha2,beta,gamma;
	a = 2;
	O1 = (0,0);
	O = (0,0);
	x[1] = (-a/sqrt(5),-a/sqrt(5));
	y[1] = (a,0);
	z[1] = (0,a);
	//第一层
	draw(elpic1,Label("$X$",EndPoint),O--x[1],blue,Arrow);
	draw(elpic1,Label("$Y$",EndPoint),O--y[1],blue,Arrow);
	draw(elpic1,Label("$Z$",EndPoint),O--z[1],blue,Arrow);
	label(elpic1,"$O$",O,W);
	path cir[],clp;
	cir[1] = yscale(1/2)*scale(a)*unitcircle;
	draw(elpic1,cir[1],blue);
	alpha1 = 70;
	alpha2 = 10;
	real k;
	k = tan(-alpha1*pi/180);
	x[2] = (a/sqrt(1+4*k*k),a*k/sqrt(1+4*k*k));
	k = tan(alpha2*pi/180);
	y[2] = (a/sqrt(1+4*k*k),a*k/sqrt(1+4*k*k));
	z[2] = z[1];
	draw(elpic1,Label("$N$",EndPoint),-x[2]--x[2],Arrow);
	draw(elpic1,Label("$L$",EndPoint),O--y[2],Arrow);
	//第二层
	picture elpic2;
	beta = 35;
	x[3] = x[2];
	y[3] = rotate(beta)*y[1];
	z[3] = rotate(0.9*beta)*z[1];
	draw(elpic2,Label("$M$",EndPoint),O--y[3],Arrow);
	draw(elpic2,Label("$z$",EndPoint),O--z[3],red,Arrow);
	real k1,k2,ra,b;
	k1 = -tan(alpha1*pi/180);
	k2 = -tan((alpha1+beta)*pi/180);
	ra = sqrt((1+4*k1^2)*(1+k2^2)/(1+k1^2)-1);
	b = k2*a/ra;
	cir[2] = rotate(beta)*yscale(b)*xscale(a)*unitcircle;
	clp = 1.1*x[3]--1.1*x[3]+1.1*y[3]--(-1.1*x[3]+1.1*y[3])--(-1.1*x[3])--cycle;
	draw(tmp,cir[2],red);
	clip(tmp,clp);
	add(elpic2,tmp);
	erase(tmp);
	draw(tmp,cir[2],red+dashed);
	unfill(tmp,clp);
	add(elpic2,tmp);
	erase(tmp);
	//第一层被第二层遮住的部分
	unfill(elpic1,cir[2]);
	draw(tmp,Label("$X$",EndPoint),O--x[1],blue+dashed,Arrow);
	draw(tmp,Label("$Y$",EndPoint),O--y[1],blue+dashed,Arrow);
	draw(tmp,Label("$Z$",EndPoint),O--z[1],blue,Arrow);
	draw(tmp,Label("$N$",EndPoint),-x[2]--x[2],Arrow);
	draw(tmp,Label("$L$",EndPoint),O--y[2],dashed,Arrow);
	label(tmp,"$O$",O,W);
	draw(tmp,cir[1],blue+dashed);
	clip(tmp,cir[2]);
	clip(tmp,clp);
	add(elpic1,tmp);
	erase(tmp);
	draw(tmp,Label("$X$",EndPoint),O--x[1],blue,Arrow);
	draw(tmp,Label("$Y$",EndPoint),O--y[1],blue,Arrow);
	draw(tmp,Label("$Z$",EndPoint),O--z[1],blue,Arrow);
	draw(tmp,Label("$N$",EndPoint),-x[2]--x[2],Arrow);
	draw(tmp,Label("$L$",EndPoint),O--y[2],dashed,Arrow);
	draw(tmp,Label("$Z$",EndPoint),O--z[2],dashed,Arrow);
	label(tmp,"$O$",O,W);
	draw(tmp,cir[1],blue);
	clip(tmp,cir[2]);
	unfill(tmp,clp);
	add(elpic1,tmp);
	erase(tmp);
	//画自转
	x[4] = relpoint(cir[2],0.78);
	y[4] = relpoint(cir[2],0.05);
	z[4] = z[3];
	draw(elpic2,Label("$x$",EndPoint),O--x[4],red,Arrow);
	draw(elpic2,Label("$y$",EndPoint),O--y[4],red,Arrow);
	add(shift(O1)*elpic1);
	add(shift(O1)*elpic2);
	picture elpic4;
	//画第一层上的角度
	real theta1,theta2,r;
	r = 0.4;
	theta1 = covert(x[1],2)+180;
	theta2 = covert(x[2],2)+360;
	draw(elpic4,Label("$\phi$",MidPoint,Relative(E)),yscale(1/2)*arc(O,r,theta1,theta2),Arrow);
	theta1 = covert(y[1],2);
	theta2 = covert(y[2],2);
	r = 0.7;
	draw(elpic4,Label("$\phi$",MidPoint,E),yscale(1/2)*arc(O,r,theta1,theta2),Arrow);
	//画第二层上的角度
	r = 1.3;
	theta1 = covert(y[2],0.9);
	theta2 = covert(y[3],0.9);
	draw(elpic4,Label("$\theta$",MidPoint,Relative(E)),xscale(0.9)*arc(O,r,theta1,theta2),Arrow);
	theta1 = covert(z[2],0.9);
	theta2 = covert(z[3],0.9);
	draw(elpic4,Label("$\theta$",MidPoint,Relative(E)),xscale(0.9)*arc(O,r,theta1,180+theta2),Arrow);
	//画自转角
	r = 0.3;
	//theta1 = covert(x[2],a/b);
	//theta2 = covert(x[4],a/b);
	draw(elpic4,Label("$\psi$",MidPoint,Relative(E)),arc(O,r,degrees(x[3]),degrees(x[4])),Arrow);
	r = 0.4;
	draw(elpic4,Label("$\psi$",MidPoint,Relative(E)),arc(O,r,degrees(y[3]),degrees(y[4])),Arrow);
	add(shift(O1)*elpic4);
\end{asy}
\caption{刚体定点转动的Euler角}
\label{刚体定点转动的Euler角}
\end{figure}

如图\ref{刚体定点转动的Euler角}所示,其中$XOY$平面与$xOy$平面的交线$ON$称为{\heiti 节线}。三个Euler角分别为{\heiti 进动(precession)角}$\,\,\phi$,{\heiti 章动(nutation)角}$\,\,\theta$和{\heiti 自转(spin)角}$\,\,\psi$。

三个Euler角是相互独立的,可以任意取值。如果给定三个角的值,则唯一地确定了刚体定点转动位形,通常假设$0 \leqslant \phi < 2\pi, \,0 \leqslant \theta \leqslant \pi,\, 0 \leqslant \psi < 2\pi$。刚体的连续转动表现为Euler角随时间连续变化。

从平动系$OXYZ$到本体系$Oxyz$的转换可以通过下面的三个按顺序的转动实现:
\begin{enumerate}
	\item 绕$OZ$轴旋转$\phi$角,称为{\heiti 进动};
	\item 绕$ON$旋转$\theta$角,称为{\heiti 章动};
	\item 绕$Oz$轴旋转$\psi$角,称为{\heiti 自转}。
\end{enumerate}
如果从转轴的顶端看,所有的转动都是逆时针方向。即定点转动可以由上面的进动、章动和自转运动来合成。由于刚体的有限定轴转动是不可交换的,因此上面的三种转动的顺序也是不可交换的。

根据定轴转动的转动矩阵,可以得到进动、章动和自转的旋转变换矩阵分别为
\begin{equation*}
	\mbf{U}_1 = \begin{pmatrix} \cos \phi & -\sin \phi & 0 \\ \sin \phi & \cos \phi & 0 \\ 0 & 0 & 1 \end{pmatrix},\quad \mbf{U}_2 = \begin{pmatrix} 1 & 0 & 0 \\ 0 & \cos \theta & -\sin \theta \\ 0 & \sin \theta & \cos \theta \end{pmatrix},\quad \mbf{U}_3 = \begin{pmatrix} \cos \psi & -\sin \psi & 0 \\ \sin \psi & \cos \psi & 0 \\ 0 & 0 & 1 \end{pmatrix}
\end{equation*}
因此,定点转动的旋转变换矩阵为
\begin{align*}
	\mbf{U} & = \mbf{U}_1 \mbf{U}_2 \mbf{U}_3 \\
	& = \begin{pmatrix}
		\cos \phi \cos \psi - \sin \phi \cos \theta \sin \psi & -\cos \phi \sin \psi - \sin \phi \cos \theta \cos \psi & \sin \phi \sin \theta \\
		\sin \phi \cos \psi + \cos \phi \cos \theta \sin \psi & -\sin \phi \sin \psi + \cos \phi \cos \theta \cos \psi & -\cos \phi \sin \theta \\
		\sin \theta \sin \psi & \sin \theta \cos \psi & \cos \theta
	\end{pmatrix}
\end{align*}
直接求$\mbf{U}^{\mathrm{T}} \dot{\mbf{U}}$的运算十分复杂,考虑到$\mbf{U}_1,\mbf{U}_2,\mbf{U}_3$都是正交矩阵,因此有
\begin{equation*}
	\mbf{U}_1^{\mathrm{T}} \dot{\mbf{U}}_1 = \begin{pmatrix} 0 & -\dot{\phi} & 0 \\ \dot{\phi} & 0 & 0 \\ 0 & 0 & 0 \end{pmatrix},\quad \mbf{U}_2^{\mathrm{T}} \dot{\mbf{U}}_2 = \begin{pmatrix} 0 & 0 & 0 \\ 0 & 0 & -\dot{\theta} \\ 0 & \dot{\theta} & 0 \end{pmatrix},\quad \mbf{U}_3^{\mathrm{T}} \dot{\mbf{U}}_3 = \begin{pmatrix} 0 & -\dot{\psi} & 0 \\ \dot{\psi} & 0 & 0 \\ 0 & 0 & 0 \end{pmatrix}
\end{equation*}
故有
\begin{align*}
	\mbf{U}^{\mathrm{T}} \dot{\mbf{U}} & = \mbf{U}_3^{\mathrm{T}} \mbf{U}_2^{\mathrm{T}} \mbf{U}_1^{\mathrm{T}} (\dot{\mbf{U}}_1 \mbf{U}_2 \mbf{U}_3 + \mbf{U}_1 \dot{\mbf{U}}_2 \mbf{U}_3 + \mbf{U}_1 \mbf{U}_2 \dot{\mbf{U}}_3) \\
	& = \mbf{U}_3^{\mathrm{T}} \mbf{U}_2^{\mathrm{T}} \mbf{U}_1^{\mathrm{T}} \dot{\mbf{U}}_1 \mbf{U}_2 \mbf{U}_3 + \mbf{U}_3^{\mathrm{T}} \mbf{U}_2^{\mathrm{T}} \dot{\mbf{U}}_2 \mbf{U}_3 + \mbf{U}_1^{\mathrm{T}} \dot{\mbf{U}}_1 \\
	& = \begin{pmatrix}
		0 & -\dot{\phi} \cos \theta - \dot{\psi} & \dot{\phi} \sin \theta \cos \psi - \dot{\theta} \sin \psi \\
		\dot{\phi} \cos \theta + \dot{\psi} & 0 & -\dot{\phi} \sin \theta \sin \psi - \dot{\theta} \cos \psi \\
		- \dot{\phi} \sin \theta \cos \psi + \dot{\theta} \sin \psi & \dot{\phi} \sin \theta \sin \psi + \dot{\theta} \cos \psi & 0
	\end{pmatrix}
\end{align*}
因此有
\begin{equation}
	\begin{cases}
		\omega_1 = \dot{\phi} \sin \theta \sin \psi + \dot{\theta} \cos \psi \\
		\omega_2 = \dot{\phi} \sin \theta \cos \psi - \dot{\theta} \sin \psi \\
		\omega_3 = \dot{\phi} \cos \theta + \dot{\psi}
	\end{cases}
	\label{Euler运动学方程}
\end{equation}
式\eqref{Euler运动学方程}称为{\heiti Euler运动学方程},它给出了刚体在平动系中的角速度在本体系内的分量与广义坐标(即Euler角)、广义速度(即Euler角的导数)之间的关系。

\begin{example}
半径为$a$的圆盘垂直于地面作纯滚动,圆盘中心$C$以速率$v_C = \omega_1 R$沿着半径为$R$的圆周运动,求圆盘边缘上任意一点$P$的速度。
\end{example}
\begin{solution}
\begin{figure}[htb]
\centering
\begin{asy}
	size(350);
	//第六章例2图
	picture tmp;
	pair O,i,j,k;
	real R,a,phi;
	O = (0,0);
	i = (-sqrt(2)/4,-sqrt(14)/12);
	j = (sqrt(14)/4,-sqrt(2)/12);
	k = (0,2*sqrt(2)/3);
	R = 2.5;
	a = 1;
	phi = 5/6*pi;
	pair cir(real theta){
		return R*cos(theta)*i+R*sin(theta)*j;
	}
	pair pln(real theta){
		real x,y,z;
		x = R*cos(phi)+a*cos(theta)*sin(phi);
		y = R*sin(phi)-a*cos(theta)*cos(phi);
		z = a+a*sin(theta);
		return x*i+y*j+z*k;
	}
	real inner(pair v1,pair v2){
		return v1.x*v2.x+v1.y*v2.y;
	}
	path plate;
	draw(Label("$Z$",EndPoint),O--3*a*k,Arrow);
	draw(graph(cir,0,2*pi),dashed);
	plate = graph(pln,0,2*pi)--cycle;
	draw(plate,linewidth(0.8bp));
	pair OO,C,RC,P,yy,dpsi,dphi,omega;
	real x1,x2;
	OO = a*k;
	C = R*cos(phi)*i+R*sin(phi)*j+a*k;
	RC = R*cos(phi)*i+R*sin(phi)*j;
	P = pln(2/3*pi);
	yy = pln(2/3*pi-pi/2);
	dot(O);
	label("$O$",OO,SE);
	dot(C);
	label("$C$",C,S);
	draw(OO--C);
	//draw(Label("$y$",EndPoint),C--yy,Arrow);
	draw(Label("$z$",EndPoint),C--interp(OO,C,-0.8),Arrow);
	dot(P);
	label("$P$",P,NW);
	draw(Label("$a$",MidPoint,Relative(W)),C--P,dashed);
	draw(Label("$x$",EndPoint),OO--OO+(P-C),Arrow);
	draw(OO+(P-C)--P,dashed);
	draw(Label("$\boldsymbol{r}_P$",MidPoint,Relative(E),black),OO--P,blue,Arrow);
	dphi = k;
	dpsi = dir(OO-C);
	omega = dir(RC-OO);
	x1 = -(inner(dpsi,dpsi)*inner(omega,dphi)-inner(dphi,dpsi)*inner(omega,dpsi))/(inner(dphi,dphi)*inner(dpsi,dpsi)-inner(dphi,dpsi)**2);
	x2 = -(inner(dphi,dphi)*inner(omega,dpsi)-inner(dphi,dpsi)*inner(omega,dphi))/(inner(dphi,dphi)*inner(dpsi,dpsi)-inner(dphi,dpsi)**2);
	dphi = x1*dphi;
	dpsi = x2*dpsi;
	draw(RC--OO,dashed);
	draw(Label("$\dot{\boldsymbol{\psi}}$",EndPoint,Relative(W),black),OO--OO+dpsi,red,Arrow);
	draw(Label("$\dot{\boldsymbol{\phi}}$",EndPoint,Relative(E),black),OO--OO+dphi,red,Arrow);
	draw(Label("$\boldsymbol{\omega}$",EndPoint,black),OO--OO+dphi+dpsi,red,Arrow);
	draw(OO+dpsi--OO+dpsi+dphi--OO+dphi,dashed);
	draw(Label("$R$",MidPoint,Relative(E)),O--RC,dashed);
	label("$Q$",RC,dir(70));
	label("$O'$",O,W);
	draw(C--pln(0),dashed);
	a = 0.3;
	draw(Label("$\psi$",MidPoint,Relative(E)),shift((1-a)*k)*graph(pln,0,2/3*pi),Arrow);
	draw(Label("$X$",EndPoint),O--3*i,Arrow);
	draw(Label("$Y$",EndPoint),O--3*j,Arrow);
	R = 0.4;
	draw(Label("$\phi$",MidPoint,Relative(E)),graph(cir,0,phi),Arrow);
\end{asy}
\caption{例\theexample}
\label{第六章例2图}
\end{figure}

{\heiti 方法一}:如图\ref{第六章例2图}所示,圆盘在转动过程中,盘上各点和$O$点的相对距离保持不变,因此圆盘的运动可以看成是绕$O$点的定点转动。而纯滚动条件要求$Q$点的瞬时速度为零,因此$OQ$为瞬时转轴,即角速度沿此方向。由此可得圆盘绕$O$的角速度(进动)$\omega_1$与圆盘自转的角速度$\omega_2$之间需要满足
\begin{equation*}
	\frac{\omega_1}{\omega_2} = \frac{a}{R}
\end{equation*}
由此可有
\begin{equation*}
	\omega_2 = \frac{R}{a} \omega_1
\end{equation*}
由此即有
\begin{equation*}
	\mbf{\omega} = \omega_1 (\sin \psi \mbf{e}_x + \cos \psi \mbf{e}_y) - \omega_2 \mbf{e}_z = \omega_1 (\sin \psi \mbf{e}_x + \cos \psi \mbf{e}_y) + \frac{R}{a} \omega_1 \mbf{e}_z
\end{equation*}
再考虑到$\mbf{r}_P = a\mbf{e}_x+R\mbf{e}_z$,此处的$\mbf{e}_x,\mbf{e}_y,\mbf{e}_z$为本体系中的基\footnote{此处$y$轴的方向需通过右手规则来确定以保证本体系为右手系。}。综上,即有
\begin{equation*}
	\mbf{v}_P = \mbf{\omega} \times \mbf{r}_P = \omega_1 R \cos \psi \mbf{e}_x - \omega_1 R(1+\sin \psi) \mbf{e}_y + \omega_1 a \cos \psi \mbf{e}_z
\end{equation*}

\noindent {\heiti 方法二}:也可以用确定Euler角的方法通过Euler运动学方程直接得到角速度。进动角$\phi$即为$O'Q$与平动系(此时与空间系重合)$X$轴之间的夹角,章动角$\theta = \dfrac{\pi}{2}$恒定不变,自转角$\psi$即为圆盘上水平线到本体系$Ox$轴的角。据此,根据Euler运动学方程\eqref{Euler运动学方程}可得圆盘的角速度$\mbf{\omega}$在本体系中的坐标为
\begin{equation*}
	\begin{cases}
		\omega_1 = \dot{\phi} \sin \psi \\
		\omega_2 = \dot{\phi} \cos \psi \\
		\omega_3 = \dot{\psi}
	\end{cases}
\end{equation*}
即有
\begin{equation*}
	\mbf{\omega} = \dot{\phi} \sin \psi \mbf{e}_x + \dot{\phi} \cos \psi \mbf{e}_y + \dot{\psi} \mbf{e}_z
\end{equation*}
现在需要得到$\dot{\phi},\dot{\psi}$与$\omega_1$的关系,考虑$\mbf{r}_C = -R\mbf{e}_z$,则有
\begin{equation*}
	\mbf{v}_C = \mbf{\omega} \times \mbf{r}_C = -R\dot{\phi} \cos \psi \mbf{e}_x + R\dot{\phi} \sin \psi \mbf{e}_y
\end{equation*}
易知$\mbf{v}_C$的方向沿$Q$点处的切线方向,并可得$\dot{\phi}$,其大小为
\begin{equation*}
	R\dot{\phi} = \omega_1 R
\end{equation*}
即有$\dot{\phi} = \omega_1$。再考虑$\mbf{r}_Q = -a\sin \psi \mbf{e}_x - a\cos \psi \mbf{e}_y - R\mbf{e}_z$,则有
\begin{equation*}
	\mbf{v}_Q = \mbf{\omega} \times \mbf{r}_Q = (a\dot{\psi}-R\dot{\phi}) \cos \psi \mbf{e}_x - (a\dot{\psi}-R\dot{\phi}) \sin \psi \mbf{e}_y
\end{equation*}
纯滚动条件要求$\mbf{v}_Q = \mbf{0}$,即可得
\begin{equation*}
	\dot{\psi} = \frac{R}{a} \dot{\phi} = \frac{R}{a} \omega_1
\end{equation*}
由此可得圆盘边缘上任意一点的速度
\begin{equation*}
	\mbf{v}_P = \mbf{\omega} \times \mbf{r}_P = \omega_1 R \cos \psi \mbf{e}_x - \omega_1 R(1+\sin \psi) \mbf{e}_y + \omega_1 a \cos \psi \mbf{e}_z
\end{equation*}
\end{solution}

\section{惯量矩阵}

\subsection{刚体定点转动的角动量、动能与转动惯量}

在平动系中,刚体绕$O$点定点转动。设刚体任意一点的位矢为$\mbf{r}$,则其在平动系中的速度为$\dot{\mbf{r}} = \mbf{\omega} \times \mbf{r}$,由此角动量可计算为\footnote{此式可通过先将刚体分割为$n$块,作为质点系处理,其角动量为\begin{equation*} \mbf{L} = \sum_{i=1}^n \mbf{r}_i \times \Delta m_i \dot{\mbf{r}}_i = \sum_{i=1}^n \mbf{r}_i \times \rho_i \dot{\mbf{r}}_i \Delta V_i \end{equation*}然后令$n$趋于无穷,根据重积分的定义即得。}
\begin{align*}
	\mbf{L} = \int_V \mbf{r} \times \rho \dot{\mbf{r}} \mathrm{d} V = \int_V \rho \mbf{r} \times (\mbf{\omega} \times \mbf{r}) \mathrm{d} V = \int_V \rho \big[r^2 \mbf{\omega} - (\mbf{r} \cdot \mbf{\omega}) \mbf{r} \big] \mathrm{d} V
\end{align*}
此处积分遍及刚体所在的空间,$\rho$为刚体的密度。将上式向任意直角座标系投影,即
\begin{align*}
	L_1 & = \mbf{L} \cdot \mbf{e}_1 = \int_V \rho \big[r^2 \mbf{\omega} \cdot \mbf{e}_1 - (\mbf{r} \cdot \mbf{\omega}) (\mbf{r} \cdot \mbf{e}_1) \big] \mathrm{d} V \\
	& = \int_V \rho \big[(x^2+y^2+z^2) \omega_1 - (x \omega_1 + y\omega_2 + z\omega_3) x \big] \mathrm{d} V \\
	& = \int_V \rho \big[(y^2+z^2)\omega_1 - xy \omega_2 - xz \omega_3\big] \mathrm{d} V \\
	L_2 & = \int_V \rho \big[-xy \omega_1 + (x^2+z^2)\omega_2 - yz \omega_3\big] \mathrm{d} V \\
	L_3 & = \int_V \rho \big[-xz \omega_1 - yz \omega_2 + (x^2+y^2)\omega_3\big] \mathrm{d} V
\end{align*}
由此,角动量矢量的分量可以表示为
\begin{equation}
	\begin{pmatrix} L_1 \\ L_2 \\ L_3 \end{pmatrix} = \begin{pmatrix} I_{11} & I_{12} & I_{13} \\ I_{21} & I_{22} & I_{23} \\ I_{31} & I_{32} & I_{33} \end{pmatrix} \begin{pmatrix} \omega_1 \\ \omega_2 \\ \omega_3 \end{pmatrix}
\end{equation}
其中
\begin{equation}
	I_{ij} = \int_V \rho \left(\delta_{ij}\sum_{k=1}^3 x_k^2-x_ix_j\right) \mathrm{d}V,\quad x_1 \equiv x,\,x_2 \equiv y,\,x_3 \equiv z
	\label{chapter6:惯量元素的定义式}
\end{equation}
称为刚体对$O$的{\heiti 惯量元素},矩阵
\begin{equation}
	\mbf{I} = \begin{pmatrix} I_{11} & I_{12} & I_{13} \\ I_{21} & I_{22} & I_{23} \\ I_{31} & I_{32} & I_{33} \end{pmatrix}
\end{equation}
称为刚体对$O$的{\heiti 惯量矩阵}。由此,角动量可以表示为
\begin{equation}
	\mbf{L} = \mbf{I} \mbf{\omega}
	\label{惯量矩阵与角动量}
\end{equation}

刚体的动能可以表示为\footnote{对于刚体,角动量矢量是与空间位置无关的,否则刚体将发生变形。}
\begin{equation*}
	T = \frac12 \int_V \rho \dot{\mbf{r}} \cdot \dot{\mbf{r}} \mathrm{d} V = \frac12 \int_V \rho \dot{\mbf{r}} \cdot (\mbf{\omega} \times \mbf{r}) \mathrm{d} V = \frac12 \int_V \mbf{\omega} \cdot (\mbf{r} \times \rho \dot{\mbf{r}}) \mathrm{d} V = \frac12 \mbf{\omega} \cdot \mbf{L}
\end{equation*}
考虑到式\eqref{惯量矩阵与角动量}可有
\begin{equation}
	T = \frac12 \mbf{\omega}^{\mathrm{T}} \mbf{I} \mbf{\omega}
\end{equation}
动能的正定性表明惯量矩阵$\mbf{I}$是正定矩阵。再考虑到惯量矩阵的定义,可知惯量矩阵$\mbf{I}$是对称正定矩阵。

在本体系中,刚体静止,因此在本体系中惯量元素为常量。

\subsection{平行轴定理}

设对$O$点,惯量元素为
\begin{equation*}
	I_{ij} = \int_V \rho \left(\delta_{ij}\sum_{k=1}^3 x_k^2-x_ix_j\right) \mathrm{d} V
\end{equation*}
对质心$C$,惯量元素为
\begin{equation*}
	I'_{ij} = \int_V \rho \left(\delta_{ij}\sum_{k=1}^3 x'_k{}^2-x'_ix'_j\right) \mathrm{d} V
\end{equation*}
设刚体质心的坐标为$X_i$,则有
\begin{equation*}
	x_i = X_i + x'_i
\end{equation*}
再考虑到质心坐标满足\footnote{此式可通过先将刚体分割为$n$块,作为质点系处理,然后令$n$趋于无穷,根据重积分的定义即得。}
\begin{equation*}
	\int_V \rho x_i \mathrm{d} V = MX_i
\end{equation*}
由此可得{\heiti 平行轴定理}
\begin{equation}
	\mbf{I} = \mbf{I}_C + \mbf{I}'
\end{equation}
其中$\mbf{I}$为刚体对$O$点的惯量矩阵,$\mbf{I}'$为刚体对质心的惯量矩阵,而
\begin{equation}
	I_{Cij} = M\left(\delta_{ij}\sum_{k=1}^3 X_k^2-X_iX_j\right)
\end{equation}
即为总质量集中于质心的质点对$O$的惯量元素。

平行轴定理给出了对质心$C$点的惯量矩阵与对任意点$O$的惯量矩阵之间的关系。

\subsection{会聚轴定理}

角速度表示可以为$\mbf{\omega} = \omega\mbf{n}$,此处$\mbf{n}$为瞬时转轴方向单位矢量。由此可得角动量轴向投影分量为
\begin{equation*}
	L_n = \mbf{n} \cdot \mbf{L} = \int_V \rho \mbf{r} \times (\mbf{\omega} \times \mbf{r}) \cdot \mbf{n} \mathrm{d} V = \int_V \rho (\mbf{n} \times \mbf{r}) \cdot (\mbf{n} \times \mbf{r}) \omega \mathrm{d} V
\end{equation*}
即有
\begin{equation}
	L_n = I_n \omega
\end{equation}
此处
\begin{equation}
	I_n = \int_V \rho |\mbf{n} \times \mbf{r}|^2 \mathrm{d} V
\end{equation}
为对$\mbf{n}$轴的转动惯量,$|\mbf{n} \times \mbf{r}|$即为矢径$\mbf{r}$末端点到转轴的距离。

设在某坐标系下,$\mbf{n}$和$\mbf{L}$的分量可以表示为
\begin{equation*}
	\mbf{n} = \begin{pmatrix} n_1 \\ n_2 \\ n_3 \end{pmatrix},\quad \mbf{L} = \begin{pmatrix} L_1 \\ L_2 \\ L_3 \end{pmatrix}
\end{equation*}
因此有
\begin{equation*}
	L_n = \mbf{n}^{\mathrm{T}} \mbf{L} = \mbf{n}^{\mathrm{T}} \mbf{I} \mbf{\omega} = \mbf{n}^{\mathrm{T}} \mbf{I} \mbf{n} \omega
\end{equation*}
即有
\begin{equation}
	I_n = \mbf{n}^{\mathrm{T}} \mbf{I} \mbf{n}
	\label{会聚轴定理}
\end{equation}
式\eqref{会聚轴定理}表示的关系即称为{\heiti 会聚轴定理}。此时,刚体的动能也可以简单的表示为
\begin{equation}
	T = \frac12 \mbf{\omega}^{\mathrm{T}} \mbf{I} \mbf{\omega} = \frac12 I_n \omega^2
\end{equation}

对轴的转动惯量一般与取向有关,是各向异性的。

\subsection{惯量主轴与主轴系}

刚体的动能是标量,因此是坐标变换的不变量。在不同的坐标系中,角动量的坐标之间将满足
\begin{equation*}
	\mbf{\omega} = \mbf{U} \mbf{\omega}'
\end{equation*}
因此,动能的形式将变为
\begin{equation*}
	T = \frac12 \mbf{\omega}^{\mathrm{T}} \mbf{I} \mbf{\omega} = \frac12 \mbf{\omega}'^{\mathrm{T}} \mbf{U}^{\mathrm{T}} \mbf{I} \mbf{U} \mbf{\omega}' = T' = \frac12 \mbf{\omega}'^{\mathrm{T}} \mbf{I}' \mbf{\omega}'
\end{equation*}
因此,在不同的坐标系中,惯量矩阵将满足
\begin{equation*}
	\mbf{I}' = \mbf{U}^{\mathrm{T}} \mbf{I} \mbf{U}
\end{equation*}
即,在不同的坐标系中,惯量矩阵是不同的。而能够使得惯量矩阵$\mbf{I}$为对角矩阵的坐标系称为{\heiti 主轴系},其坐标轴称为{\heiti 惯量主轴},对相应主轴的转动惯量称为{\heiti 主惯量}。

今后,我们约定本体系一律取为主轴系。在主轴系中,惯量矩阵为
\begin{equation*}
	\mbf{I} = \begin{pmatrix} I_1 & & \\ & I_2 & \\ & & I_3 \end{pmatrix}
\end{equation*}
此时,角动量和动能分别为
\begin{equation*}
	\mbf{L} = \sum_{i=1}^3 I_i \omega_i \mbf{e}_i,\quad T = \frac12 \sum_{i=1}^3 I_i \omega_i^2
\end{equation*}
惯量矩阵的正定性保证主惯量都是正值,即$I_i>0,\,\,i=1,2,3$。

\subsection{主轴系的求法}

考虑角速度沿主轴方向,即$\mbf{\omega} = \omega \mbf{e}_j$时,则
\begin{equation*}
	\mbf{L} = \mbf{I} \mbf{\omega} = \mbf{I} \omega \mbf{e}_j = I_j \omega \mbf{e}_j
\end{equation*}
由此可得
\begin{equation*}
	\mbf{I} \mbf{e}_j = I_j \mbf{e}_j
\end{equation*}
即惯量主轴为惯量矩阵的本征矢量,相应的本征值为主惯量\footnote{实际上,根据矩阵正交相似对角化的条件,可以直接得出此结论。}。

\begin{example}
求边长为$a$质量为$M$的匀质立方体对顶点的惯量矩阵,并求出其惯量主轴和主惯量。

\begin{figure}[htb]
\centering
\begin{asy}
	size(200);
	//第六章例3图
	pair O,i,j,k;
	picture tmp;
	O = (0,0);
	i = (-sqrt(2)/4,-sqrt(14)/12);
	j = (sqrt(14)/4,-sqrt(2)/12);
	k = (0,2*sqrt(2)/3);
	draw(Label("$x$",EndPoint),O--1.5*i,Arrow);
	draw(Label("$y$",EndPoint),O--1.5*j,Arrow);
	draw(Label("$z$",EndPoint),O--1.5*k,Arrow);
	label("$O$",O,W);
	path clp;
	clp = i--i+j--j--j+k--k--i+k--cycle;
	unfill(clp);
	draw(tmp,Label("$x$",EndPoint),O--1.5*i,dashed,Arrow);
	draw(tmp,Label("$y$",EndPoint),O--1.5*j,dashed,Arrow);
	draw(tmp,Label("$z$",EndPoint),O--1.5*k,dashed,Arrow);
	label(tmp,"$O$",O,W);
	clip(tmp,clp);
	add(tmp);
	erase(tmp);
	draw(clp);
	draw(i+k--i+j+k--j+k);
	draw(i+j--i+j+k);
	//draw(O--(1.6,0),invisible);
\end{asy}
\caption{例\theexample}
\label{第六章例3图}
\end{figure}
\end{example}
\begin{solution}
根据惯量矩阵中惯量元素的定义
\begin{equation*}
	I_{ij} = \int_V \rho \left(\delta_{ij} \sum_{k=1}^3 x_k^2 - x_ix_j\right) \mathrm{d} V
\end{equation*}
可有
\begin{align*}
	I_{11} & = \frac{M}{a^3} \int_V (y^2+z^2) \mathrm{d}x\mathrm{d}y\mathrm{d}z = \frac{M}{a^3} \int_0^a \mathrm{d} x \int_0^a \mathrm{d} y \int_0^a (y^2+z^2) \mathrm{d} z = \frac23 Ma^2 \\
	I_{22} & = \frac{M}{a^3} \int_V (x^2+z^2) \mathrm{d}x\mathrm{d}y\mathrm{d}z = \frac{M}{a^3} \int_0^a \mathrm{d} x \int_0^a \mathrm{d} y \int_0^a (x^2+z^2) \mathrm{d} z = \frac23 Ma^2 \\
	I_{33} & = \frac{M}{a^3} \int_V (x^2+y^2) \mathrm{d}x\mathrm{d}y\mathrm{d}z = \frac{M}{a^3} \int_0^a \mathrm{d} x \int_0^a \mathrm{d} y \int_0^a (x^2+y^2) \mathrm{d} z = \frac23 Ma^2 \\
	I_{12} & = I_{21} = \frac{M}{a^3} \int_V (-xy) \mathrm{d}x\mathrm{d}y\mathrm{d}z = -\frac{M}{a^3} \int_0^a x \mathrm{d} x \int_0^a y \mathrm{d} y \int_0^a \mathrm{d} z = -\frac14 Ma^2 \\
	I_{23} & = I_{32} = \frac{M}{a^3} \int_V (-yz) \mathrm{d}x\mathrm{d}y\mathrm{d}z = -\frac{M}{a^3} \int_0^a \mathrm{d} x \int_0^a y \mathrm{d} y \int_0^a z \mathrm{d} z = -\frac14 Ma^2 \\
	I_{13} & = I_{31} = \frac{M}{a^3} \int_V (-xz) \mathrm{d}x\mathrm{d}y\mathrm{d}z = -\frac{M}{a^3} \int_0^a x \mathrm{d} x \int_0^a \mathrm{d} y \int_0^a z \mathrm{d} z = -\frac14 Ma^2
\end{align*}
因此,惯量矩阵为
\begin{equation*}
	\mbf{I} = \frac{Ma^2}{12} \begin{pmatrix} 8 & -3 & -3 \\ -3 & 8 & -3 \\ -3 & -3 & 8 \end{pmatrix}
\end{equation*}
设矩阵$\mbf{I}$的特征值为$\dfrac{Ma^2}{12} \lambda$,则有
\begin{equation*}
	\begin{vmatrix} 8-\lambda & -3 & -3 \\ -3 & 8-\lambda & -3 \\ -3 & -3 & 8-\lambda \end{vmatrix} = 0
\end{equation*}
解得$\lambda_1=\lambda_2=11,\lambda_3=2$,因此有主惯量
\begin{equation*}
	I_1 = I_2 = \frac{11}{12} Ma^2,\quad I_3 =\frac16 Ma^2
\end{equation*}
相应的本征矢量可取为\footnote{其中$\mbf{e}_1$和$\mbf{e}_2$的取法不唯一。}
\begin{equation*}
	\mbf{e}_1 = \frac{1}{\sqrt{6}} \begin{pmatrix} 1 \\ 1 \\ -2 \end{pmatrix},\quad \mbf{e}_2 = \frac{1}{\sqrt{2}} \begin{pmatrix} -1 \\ 1 \\ 0 \end{pmatrix},\quad \mbf{e}_3 = \frac{1}{\sqrt{3}} \begin{pmatrix} 1 \\ 1 \\ 1 \end{pmatrix}
\end{equation*}

\begin{figure}[htb]
\centering
\begin{asy}
	size(300);
	//第六章例3惯量主轴
	pair O,i,j,k;
	picture tmp;
	O = (0,0);
	i = (-sqrt(2)/4,-sqrt(14)/12);
	j = (sqrt(14)/4,-sqrt(2)/12);
	k = (0,2*sqrt(2)/3);
	draw(Label("$x$",EndPoint),O--1.5*i,Arrow);
	draw(Label("$y$",EndPoint),O--1.5*j,Arrow);
	draw(Label("$z$",EndPoint),O--1.5*k,Arrow);
	label("$O$",O,W);
	path clp;
	clp = i--i+j--j--j+k--k--i+k--cycle;
	unfill(clp);
	draw(tmp,Label("$x$",EndPoint),O--1.5*i,dashed,Arrow);
	draw(tmp,Label("$y$",EndPoint),O--1.5*j,dashed,Arrow);
	draw(tmp,Label("$z$",EndPoint),O--1.5*k,dashed,Arrow);
	label(tmp,"$O$",O,W);
	clip(tmp,clp);
	add(tmp);
	erase(tmp);
	draw(clp);
	draw(i+k--i+j+k--j+k);
	draw(i+j--i+j+k);
	draw(-i--(-i+k),dashed);
	draw(-i--(j-i),dashed);
	draw(i--(i-2*k)--(i+j-2*k)--(j-2*k)--j,dashed);
	draw(O--(-2*k)--(j-2*k),dashed);
	draw((-2*k)--(i-2*k),dashed);
	draw(i+j--(i+j-2*k),dashed);
	draw(j--(j-i)--(j-i+k)--j+k,dashed);
	draw(k--(k-i)--(k-i+j),dashed);
	draw(O--(i+j-2*k),dashed);
	draw(O--(j-i),dashed);
	draw(O--i+j+k,dashed);
	draw(Label("$x'$",EndPoint,Relative(E),black),O--(i+j-2*k)/sqrt(3),red,Arrow);
	draw(Label("$y'$",EndPoint,black),O--(-i+j),red,Arrow);
	draw(Label("$z'$",EndPoint,N,black),O--(i+j+k),red,Arrow);
	//draw(O--(1.7,0),invisible);
\end{asy}
\caption{例\theexample:立方体对顶点的惯量主轴}
\label{第六章例3惯量主轴}
\end{figure}

质心的坐标为
\begin{equation*}
	\mbf{X} = \frac{a}{2} \begin{pmatrix} 1 \\ 1 \\ 1 \end{pmatrix}
\end{equation*}
故对质量集中与质心对顶点的惯量元素为
\begin{equation*}
	I_{Cij} = M\left(\delta_{ij} \sum_{k=1}^3 X_k^2 -X_iX_j\right)
\end{equation*}
由此可得质量集中与质心对顶点的惯量矩阵为
\begin{equation*}
	\mbf{I}_C = \frac{Ma^2}{4} \begin{pmatrix} 2 & -1 & -1 \\ -1 & 2 & -1 \\ -1 & -1 & 2 \end{pmatrix}
\end{equation*}
故有对质心$C$的惯量矩阵为
\begin{equation*}
	\mbf{I}' = \mbf{I} - \mbf{I}_C = \frac{Ma^2}{6} \begin{pmatrix} 1 & 0 & 0 \\ 0 & 1 & 0 \\ 0 & 0 & 1 \end{pmatrix}
\end{equation*}
故对质心$C$可选任意三个相互正交的方向为惯量主轴,主惯量为
\begin{equation*}
	I_1 = I_2 = I_3 = \frac16 Ma^2
\end{equation*}
即,匀质立方体对质心的转动惯性是球对称的。
\end{solution}

\subsection{惯量椭球}

在刚体的转轴上取点$P$,使得其坐标满足
\begin{equation*}
	\mbf{x} = \frac{1}{\sqrt{I_n}} \mbf{n}
\end{equation*}
所以有
\begin{equation*}
	\mbf{x}^{\mathrm{T}} \mbf{I} \mbf{x} = \frac{1}{I_n} \mbf{n}^{\mathrm{T}} \mbf{I} \mbf{n} = 1
\end{equation*}
由于惯量矩阵$\mbf{I}$对称正定,因此点$P$的集合为中心在原点的椭球面,此椭球面称为{\heiti 惯量椭球}。它是刚体转动惯量的几何表示。

在主轴系中,惯量椭球为标准形式
\begin{equation*}
	I_1 x'^2 + I_2 y'^2 + I_3 z'^2 = 1
\end{equation*}
主轴是惯量椭球的对称轴。

当$I_1=I_2=I_3$时,惯量椭球退化为球,其主轴可以任选,物体呈转动各项同性。当$I_1=I_2\neq I_3$时,惯量椭球退化为旋转椭球面,两主轴可以在垂直于第三个主轴的平面内任选。

\begin{figure}[htb]
\centering
\begin{asy}
	size(250);
	//惯量椭球和主轴系
	pair O,i,j,k;
	picture tmp,clp1,clp2;
	O = (0,0);
	i = (-sqrt(2)/4,-sqrt(14)/12);
	j = (sqrt(14)/4,-sqrt(2)/12);
	k = (0,2*sqrt(2)/3);
	label("$O$",O,W);
	real a,b,alpha,beta;
	a = 3;
	b = 5;
	pair ell1(real t){
		real x,y,z;
		x = a*cos(t)*cos(alpha);
		y = b*sin(t);
		z = a*cos(t)*sin(alpha);
		return x*i+y*j+z*k;
	}
	pair ell2(real t){
		real x,y,z;
		x = a*cos(t);
		y = 0;
		z = a*sin(t);
		return x*i+y*j+z*k;
	}
	alpha = 0;
	beta = 20;
	draw(tmp,Label(rotate(-beta)*"$x'$",EndPoint,black),O--4.5*i,red,Arrow);
	draw(tmp,Label(rotate(-beta)*"$y'$",EndPoint,black),O--6*j,red,Arrow);
	draw(tmp,Label(rotate(-beta)*"$z'$",EndPoint,black),O--4.5*k,red,Arrow);
	path cir1,cir2;
	cir1 = graph(ell1,0,2*pi)--cycle;
	cir2 = graph(ell2,0,2*pi)--cycle;
	unfill(tmp,cir1);
	unfill(tmp,cir2);
	draw(clp1,Label(rotate(-beta)*"$x'$",EndPoint,black),O--4.5*i,red+dashed,Arrow);
	draw(clp1,Label(rotate(-beta)*"$y'$",EndPoint,black),O--6*j,red+dashed,Arrow);
	draw(clp1,Label(rotate(-beta)*"$z'$",EndPoint,black),O--4.5*k,red+dashed,Arrow);
	add(clp2,clp1);
	clip(clp1,cir1);
	clip(clp2,cir2);
	add(tmp,clp1);
	add(tmp,clp2);
	draw(tmp,cir1,dashed);
	draw(tmp,cir2,dashed);
	draw(tmp,xscale(0.97*b)*yscale(a)*unitcircle);
	add(rotate(beta)*tmp);
	erase(tmp);
	draw(Label("$x$",EndPoint,black),O--6*i,blue,Arrow);
	draw(Label("$y$",EndPoint,black),O--6*j,blue,Arrow);
	draw(Label("$z$",EndPoint,black),O--4.5*k,blue,Arrow);
\end{asy}
\caption{惯量椭球和主轴系(红色坐标系为主轴系)}
\label{惯量椭球和主轴系}
\end{figure}

惯量椭球的几何意义为:
\begin{enumerate}
	\item 给出对过定点的任意轴的转动惯量;
	\item 给出主轴系;
	\item 已知瞬时转动轴(即角速度$\mbf{\omega}$的方向),确定角动量方向。
\end{enumerate}

\begin{figure}[htb]
\centering
\begin{asy}
	size(250);
	//角速度的方向与角动量的方向
	pair O,a,b,L,omega;
	real rx,ry,theta,lL,lomega;
	path rect,cir;
	picture tmp1,tmp2;
	O = (0,0);
	a = (2,0);
	b = (0.7,0.8);
	L = (0,-0.9);
	omega = L+0.7*dir(190);
	lL = 2.2;
	lomega = lL/cos((degrees(omega)-degrees(L))*pi/180);
	rx = 1.5;
	ry = 1;
	theta = 50;
	rect = L+a+b--L+a-b--L-a-b--L-a+b--cycle;
	cir = rotate(theta)*xscale(rx)*yscale(ry)*unitcircle;
	dot(O);
	label("$O$",O,N);
	draw(cir);
	draw(tmp1,rect);
	unfill(tmp1,cir);
	add(tmp1);
	erase(tmp1);
	draw(tmp1,O--lL*L,blue+dashed,Arrow);
	draw(tmp1,O--lomega*omega,red+dashed,Arrow);
	add(tmp2,tmp1);
	clip(tmp1,rect);
	add(tmp1);
	clip(tmp2,cir);
	unfill(tmp2,rect);
	add(tmp2);
	erase(tmp1);
	erase(tmp2);
	draw(tmp1,Label("$\boldsymbol{L}$",EndPoint,black),O--lL*dir(L),blue,Arrow);
	draw(tmp1,Label("$\boldsymbol{\omega}$",EndPoint,black),O--lomega*dir(omega),red,Arrow);
	unfill(tmp1,rect);
	unfill(tmp1,cir);
	add(tmp1);
	erase(tmp1);
	dot(omega);
	dot(L);
	label("$N$",omega,N);
	label("$L$",L,NE);
	draw(L--omega,dashed);
\end{asy}
\caption{角速度的方向与角动量的方向}
\label{角速度的方向与角动量的方向}
\end{figure}

下面考虑已知角速度$\mbf{\omega}$的方向,来求解角动量$\mbf{L}$的方向。首先刚体的动能可以表示为
\begin{equation*}
	T = \frac12 I_n \omega^2
\end{equation*}
此处$I_n$为刚体绕瞬时转轴的转动惯量。则转轴与惯量椭球的交点$N$(称为{\heiti 极点})为
\begin{equation*}
	\mbf{x}_N = \frac{1}{\sqrt{I_n}} \frac{\mbf{\omega}}{\omega} = \frac{\mbf{\omega}}{\sqrt{2T}}
\end{equation*}
则极点处,惯量椭球的切平面法向矢量可以表示为
\begin{equation*}
	\mbf{n} = \bnb (\mbf{x}^{\mathrm{T}} \mbf{I} \mbf{x}-1)\big|_N = 2\mbf{I} \mbf{x}_N = \frac{2\mbf{I}\mbf{\omega}}{\sqrt{2T}} = \frac{2\mbf{L}}{\sqrt{2T}}
\end{equation*}
由此即可得角动量的方向沿惯量椭球过极点切平面的法向。

惯量椭球固定于刚体,其转动代表刚体的定点转动。

\subsection{主惯量的性质}

当坐标系取为主轴系时,根据惯量矩阵的定义式\eqref{chapter6:惯量元素的定义式}同样也有
\begin{equation*}
	I_1=\int_V \rho(y^2+z^2)\mathd V,\quad I_2=\int_V \rho(z^2+x^2)\mathd V,\quad I_3=\int_V \rho(x^2+y^2)\mathd V
\end{equation*}
由此可得
\begin{align*}
	I_1+I_2=\int_V \rho(x^2+y^2+2z^2)\mathd V = I_3+2\int_V \rho z^2\mathd V \geqslant I_3
\end{align*}
其中等号成立的条件为刚体仅位于$Oxy$平面内。即主惯量满足三角不等式
\begin{equation}
	I_1+I_2\geqslant I_3,\quad I_3+I_1\geqslant I_2,\quad I_2+I_3\geqslant I_1
	\label{chapter6:主惯量的三角不等式}
\end{equation}
对于一般的三维刚体,等号是无法取得的。

利用参数$\theta_1=\dfrac{I_1}{I_3},\theta_2=\dfrac{I_2}{I_3}$对主惯量进行无量纲化,此时不等式\eqref{chapter6:主惯量的三角不等式}可以改写为
\begin{equation*}
	\theta_1+\theta_2\geqslant 1,\quad \theta_1+1\geqslant \theta_2,\quad 1+\theta_2\geqslant \theta_1
\end{equation*}
参数允许的取值如图\ref{chapter6:figure-主惯量的允许取值区域}中的阴影区域所示,它是位于平行直线$\theta_1+1=\theta_2$和$1+\theta_2=\theta_1$之间以及直线$\theta_1+\theta_2=1$右上方的无穷大带状区域。参数允许取值区域的边界$\theta_1+\theta_2=1,\theta_1+1=\theta_2$和$\theta_2+1=\theta_1$分别对应位于$Oxy, Oxz$和$Oyz$平面内的二维刚体(或质点系)。而点$(0,1)$、点$(1,0)$和无穷远点则对应位于$Ox$轴、$Oy$轴和$Oz$轴上的一维刚体(或质点系)。

\begin{figure}[htb]
\centering
\begin{asy}
	size(250);
	//主惯量的允许取值区域
	picture pic;
	real x,y;
	x = 3;
	y = 3;
	draw(Label("$\theta_1$",EndPoint),(0,0)--(x,0),Arrow);
	draw(Label("$\theta_2$",EndPoint),(0,0)--(0,y),Arrow);
	label("$O$",(0,0),SW);
	label("$1$",(1,0),S);
	label("$1$",(0,1),W);
	path p = (1,0)--(0,1);
	draw(pic,Label(rotate(-45)*"$\theta_1+\theta_2=1$",MidPoint,Relative(W)),p,linewidth(0.8bp));
	pair t = dir(45);
	real maxl = sqrt(2)*x;
	draw(pic,(1,0)--(1,0)+maxl*t,linewidth(0.8bp));
	draw(pic,(0,1)--(0,1)+maxl*t,linewidth(0.8bp));
	real delta = 1/12;
	for(real r=delta;r<1;r=r+delta){
		pair P=relpoint(p,r);
		draw(pic,P--P+maxl*t);
	}
	clip(pic,box((-0.1,-0.1),(maxl,y-0.2)));
	add(pic);
	real l=1.5;
	label(rotate(45)*"$1+\theta_2=\theta_1$",(1,0)+l*t,SE);
	label(rotate(45)*"$\theta_1+1=\theta_2$",(0,1)+l*t,NW);
\end{asy}
\caption{主惯量的允许取值区域}
\label{chapter6:figure-主惯量的允许取值区域}
\end{figure}

\section{Euler动力学方程}

\subsection{刚体运动方程}

当刚体在空间系中作定点转动时(此时平动系与空间系重合),取定点为基点$O$,根据角动量定理可有
\begin{equation*}
	\frac{\mathrm{d} \mbf{L}}{\mathrm{d} t} = \mbf{M}^{(e)}
\end{equation*}
此处$\mbf{M}^{(e)}$为刚体对$O$点的合外力矩,定义为\footnote{此式同样可以通过将刚体分割为$n$个质点,然后令$n$趋于无穷大根据重积分的定义获得。}
\begin{equation*}
	\mbf{M}^{(e)} = \int_V \mbf{r} \times \mbf{f}^{(e)} \mathrm{d} V
\end{equation*}
式中$\mbf{f}^{(e)}$为刚体所受的外力分布(单位体积刚体所受外力)。刚体的动能为
\begin{equation*}
	T = \frac12 \mbf{\omega} \cdot \mbf{L}
\end{equation*}
由于刚体的性质,刚体在运动过程中其内部任意两个质点间都没有相对位移,因此内约束力不做功。根据动能定理可有
\begin{equation*}
	\frac{\mathrm{d} T}{\mathrm{d} t} = \int_V \mbf{f}^{(e)} \cdot \dot{\mbf{r}} \mathrm{d} V = \int_V \mbf{f}^{(e)} \cdot (\mbf{\omega} \times \mbf{r}) \mathrm{d} V = \int_V \mbf{\omega} \cdot (\mbf{r} \times \mbf{f}^{(e)}) \mathrm{d} V = \mbf{\omega} \cdot \mbf{M}^{(e)}
\end{equation*}

对于刚体在空间系的一般运动,有如下定理。

\begin{theorem}[Chasles定理\footnote{中文可作“沙勒定理”或“夏莱定理”。}]
\label{Chasles定理}
刚体最一般的位移可以分解为随任选基点的平动位移和绕该基点的定点转动。
\end{theorem}

由此,可取质心为基点,刚体的运动则分解为随质心的平动与绕质心的定点转动,而且前面已说明刚体定点转动的角速度与基点的选取无关。对于刚体质心的平动运动,根据动量定理可有
\begin{equation}
	M \dot{\mbf{V}} = \mbf{F}^{(e)}
\end{equation}
式中$\mbf{F}^{(e)}$为刚体所受合外力,即
\begin{equation*}
	\mbf{F}^{(e)} = \int_V \mbf{f}^{(e)} \mathrm{d} V
\end{equation*}
再考虑到刚体在运动过程中其内部任意两个质点间都没有相对位移,根据动能定理可有
\begin{equation}
	\frac{\mathrm{d}}{\mathrm{d} t}\left(\frac12 MV^2\right) = \mbf{F}^{(e)} \cdot \mbf{V}
\end{equation}
对于刚体绕质心的定点转动,根据质心系的角动量定理可有
\begin{equation}
	\frac{\mathrm{d} \mbf{L}'}{\mathrm{d} t} = \mbf{M}'^{(e)}
\end{equation}
式中$\mbf{M}'^{(e)}$为对质心的合外力矩,即
\begin{equation*}
	\mbf{M}'^{(e)} = \int_V \mbf{r}' \times \mbf{f}^{(e)} \mathrm{d} V
\end{equation*}
根据质心系的动能定理可有
\begin{equation}
	\frac{\mathrm{d} T'}{\mathrm{d} t} = \frac{\mathrm{d}}{\mathrm{d} t} \left(\frac12 \mbf{\omega} \cdot \mbf{L}'\right) = \mbf{\omega} \cdot \mbf{M}'^{(e)}
\end{equation}

在空间系中,刚体对原点的角动量为
\begin{equation}
	\mbf{L} = \mbf{R} \times M \mbf{V} + \mbf{L}'
\end{equation}
在空间系中,刚体的动能为
\begin{equation}
	T = \frac12 MV^2 + T' = \frac12 \mbf{V} \cdot \mbf{P} + \frac12 \mbf{\omega} \cdot \mbf{L}'
\end{equation}

\subsection{Euler动力学方程}

刚体质心的运动规律与质点是相同的,现在只考虑刚体绕质心的定点转动。为了更具一般性,我们考虑刚体对任意基点的定点运动方程。即考虑角动量定理
\begin{equation}
	\frac{\mathrm{d} \mbf{L}}{\mathrm{d} t} = \mbf{M}
	\label{Euler动力学方程——角动量定理}
\end{equation}
的分量形式。式中$\mbf{L}$为刚体在平动系(可以不是质心系)中的角动量,$\mbf{M}$为刚体对基点的外力矩。在本体系中,刚体的惯量矩阵是常数矩阵,因此下面考虑求得方程\eqref{Euler动力学方程——角动量定理}在本体系中的分量形式。本体系的基矩阵$\mbf{e}$与平动系的基矩阵$\mbf{E}$之间满足
\begin{equation*}
	\mbf{e} = \mbf{E} \mbf{U}
\end{equation*}
在本体系中可有
\begin{equation*}
	\mbf{L} = \sum_{i=1}^3 L_i \mbf{e}_i = \begin{pmatrix} \mbf{e}_1 & \mbf{e}_2 & \mbf{e}_3 \end{pmatrix} \begin{pmatrix} L_1 \\ L_2 \\ L_3 \end{pmatrix} = \mbf{e} \bar{\mbf{L}}
\end{equation*}
式中$\bar{\mbf{L}} = \begin{pmatrix} L_1 \\ L_2 \\ L_3 \end{pmatrix}$。由此可有
\begin{align}
	\frac{\mathrm{d} \mbf{L}}{\mathrm{d} t} & = \mbf{e} \dot{\bar{\mbf{L}}} + \dot{\mbf{e}} \bar{\mbf{L}} = \mbf{e} \dot{\bar{\mbf{L}}} + \mbf{E} \dot{\mbf{U}} \bar{\mbf{L}} = \mbf{e} \dot{\bar{\mbf{L}}} + \mbf{e} \mbf{U}^{\mathrm{T}} \dot{\mbf{U}} \bar{\mbf{L}} \nonumber \\
	& = \frac{\tilde{\mathrm{d}} \mbf{L}}{\mathrm{d} t} + \mbf{\omega} \times \mbf{L}
\end{align}
式中$\displaystyle \frac{\tilde{\mathrm{d}} \mbf{L}}{\mathrm{d} t} = \sum_{i=1}^3 \dot{L}_i \mbf{e}_i$称为{\heiti 相对变化率},$\mbf{\omega} \times \mbf{L}$称为{\heiti 转动牵连变化率}。再利用角动量定理,可得
\begin{equation}
	\frac{\tilde{\mathrm{d}} \mbf{L}}{\mathrm{d} t} + \mbf{\omega} \times \mbf{L} = \mbf{M}
	\label{Euler动力学方程——角动量定理的矢量形式}
\end{equation}
式\eqref{Euler动力学方程——角动量定理的矢量形式}称为{\bf Euler动力学方程}(矢量形式)。考虑到$\mbf{L} = \mbf{I} \mbf{\omega}$,可以将式\eqref{Euler动力学方程——角动量定理的矢量形式}写作分量形式如下:
\begin{equation}
\begin{cases}
	I_{11}\dot{\omega}_1+I_{12}\dot{\omega}_2+I_{13}\dot{\omega}_3-(I_{22}-I_{33})\omega_2\omega_3+I_{23}(\omega_2^2-\omega_3^2)-\omega_1(I_{12}\omega_3-I_{13}\omega_2)=M_1 \\
	I_{21}\dot{\omega}_1+I_{22}\dot{\omega}_2+I_{23}\dot{\omega}_3-(I_{33}-I_{11})\omega_3\omega_1+I_{31}(\omega_3^2-\omega_1^2)-\omega_2(I_{23}\omega_1-I_{12}\omega_3)=M_2 \\
	I_{31}\dot{\omega}_1+I_{32}\dot{\omega}_2+I_{33}\dot{\omega}_3-(I_{11}-I_{22})\omega_1\omega_2+I_{13}(\omega_1^2-\omega_2^2)-\omega_3(I_{13}\omega_2-I_{23}\omega_1)=M_3
\end{cases}
\label{Euler动力学方程——角动量定理的分量形式}
\end{equation}

事实上,对于任何矢量$\mbf{A}$,都有类似式\eqref{Euler动力学方程——角动量定理的矢量形式}的式子成立,即
\begin{equation}
	\frac{\mathrm{d} \mbf{A}}{\mathrm{d} t} = \frac{\tilde{\mathrm{d}} \mbf{A}}{\mathrm{d} t} + \mbf{\omega} \times \mbf{A}
	\label{chapter6:相对变化率和绝对变化率之间的关系}
\end{equation}
其中$\ds \dfrac{\mathd \mbf{A}}{\mathd t} = \dfrac{\mathd}{\mathd t}\sum_{i=1}^3A_i\mbf{e}_i$称为{\bf 绝对变化率},$\ds \frac{\tilde{\mathrm{d}} \mbf{A}}{\mathrm{d} t} = \sum_{i=1}^3 \dot{A}_i \mbf{e}_i$称为{\bf 相对变化率},$\mbf{\omega} \times \mbf{A}$称为{\bf 转动牵连变化率}。

如果本体系取为主轴系,则有
\begin{equation*}
	L_i = I_i \omega_i,\quad i = 1,2,3
\end{equation*}
此时可使式\eqref{Euler动力学方程——角动量定理的矢量形式}的分量形式大为简化,即
\begin{equation}
\begin{cases}
	I_1 \dot{\omega}_1 - (I_2-I_3)\omega_2 \omega_3 = M_1\\
	I_2 \dot{\omega}_2 - (I_3-I_1)\omega_3 \omega_1 = M_2\\
	I_3 \dot{\omega}_3 - (I_1-I_2)\omega_1 \omega_2 = M_3
\end{cases}
\label{Euler动力学方程}
\end{equation}
式\eqref{Euler动力学方程}即为主轴系中的{\heiti Euler动力学方程},它是对角速度的一阶非线性常微分方程组。将Euler运动学方程\eqref{Euler运动学方程}代入,可得对Euler角的二阶非线性常微分方程组。

\section{刚体的定轴转动}

\subsection{运动方程与约束反力}

考虑有两个点$O$和$O_1$固定不动的刚体,如图\ref{chapter6:刚体的定轴转动与约束反力}所示。设点$O$和$O_1$受到的约束反力分别为$\mbf{R}$和$\mbf{R}_1$,刚体受到的主动力的合力为$\mbf{F}$,主动力对$O$点的力矩为$\mbf{M}_O$。

\begin{figure}[htb]
\centering
\begin{asy}
	size(200);
	//刚体的定轴转动
	pair O,i,j,k;
	real x,y,z,x0,y0,z0;
	O = (0,0);
	x = 4.3;
	y = 2.5;
	z = 4.5;
	x0 = 3;
	y0 = 1.8;
	z0 = 3.7;
	i = (-sqrt(2)/4,-sqrt(14)/12);
	j = (sqrt(14)/4,-sqrt(2)/12);
	k = (0,2*sqrt(2)/3);
	draw(Label("$X$",EndPoint),x0*i--x*i,Arrow);
	draw(Label("$Y$",EndPoint),y0*j--y*j,Arrow);
	draw(Label("$Z$",EndPoint),z0*k--z*k,Arrow);
	draw(O--x0*i,dashed);
	draw(O--y0*j,dashed);
	draw(O--z0*k,dashed);
	pair A[];
	A[0] = (-1.8,1);
	A[1] = (-2,0);
	A[2] = (-1,-2);
	A[3] = (0,-2.2);
	A[4] = (1.5,-1.8);
	A[5] = (2,0);
	A[6] = (1.8,1);
	A[7] = (1,2.3);
	A[8] = (0,2.5);
	A[9] = (-1,2);
	guide p;
	for(pair P:A){
		p = p..P;
	}
	p = shift((0,1))*(p..cycle);
	draw(p,linewidth(0.8bp));
	pair O1;
	O1 = 3*k;
	pair F,F1;
	F = 1.5*dir(130);
	F1 = 1.8*dir(35);
	draw(Label("$\boldsymbol{F}$",EndPoint),O--O+F,Arrow);
	draw(Label("$\boldsymbol{F}_1$",EndPoint),O1--O1+F1,Arrow);
	pair ii,jj;
	real psi = -pi/6;
	real xx,yy,xx0,yy0;
	xx = x;
	yy = y;
	xx0 = 3.2;
	ii = cos(psi)*i-sin(psi)*j;
	jj = sin(psi)*i+cos(psi)*j;
	draw(Label("$x$",EndPoint),xx0*ii--xx*ii,Arrow);
	draw(O--xx0*ii,dashed);
	pair P = intersectionpoint(p,O--yy*jj);
	draw(Label("$y$",EndPoint),P--yy*jj,Arrow);
	draw(O--P,dashed);
	pair rc = (0.8,1.5);
	draw(Label("$\boldsymbol{r}_C$",MidPoint,Relative(E)),O--O+rc,Arrow);
	label("$C$",O+rc,E);
	label("$h$",(O+O1)/2,E);
	dot("$O$",O,W,UnFill);
	dot("$O_1$",O1,W,UnFill);
	real r = 0.8;
	pair f(real theta){
		return r*cos(theta)*i+r*sin(theta)*j;
	}
	draw(Label("$\psi$",MidPoint,Relative(E)),graph(f,0,-psi),Arrow);
\end{asy}
\caption{刚体的定轴转动与约束反力}
\label{chapter6:刚体的定轴转动与约束反力}
\end{figure}

以点$O$为平动系$OXYZ$的原点,$OZ$轴沿着$OO_1$方向,本体系$Oxyz$的$Oz$轴也沿着$OO_1$方向。定轴转动的刚体有一个自由度,取坐标轴$OX$和$Ox$的夹角$\psi$(自转角)为广义坐标。

以$O$点为基点,利用矢量形式的Euler动理学方程\eqref{Euler动力学方程——角动量定理的矢量形式}可得
\begin{equation}
	\frac{\tilde{\mathrm{d}} \mbf{L}_O}{\mathrm{d} t} + \mbf{\omega} \times \mbf{L}_O = \mbf{M}_O+\vec{OO_1}\times \mbf{F}_1
	\label{chapter6:定轴转动刚体的角动量方程}
\end{equation}
再根据质心系的动量定理,并将其投影至本体系中\footnote{实际上此处可以看作是在非惯性系(即本体系)中应用Newton第二定律,更多具体的内容参见第\ref{chapter7:非惯性系章}章。},过程类似于前面得到式\eqref{Euler动力学方程——角动量定理的矢量形式}的过程,可得
\begin{equation}
	m\frac{\tilde{\mathrm{d}} \mbf{v}_C}{\mathrm{d} t} + m\mbf{\omega} \times \mbf{v}_C = \mbf{F}+\mbf{R}+\mbf{R}_1
	\label{chapter6:定轴转动刚体的动量方程}
\end{equation}
其中$m$是刚体的质量,$\mbf{\omega}$是刚体的角速度,$\mbf{v}_C$是刚体质心的速度。设在本体系中这些矢量的分量分别如下
\begin{align*}
	\mbf{F}=\begin{pmatrix} F_x \\ F_y \\ F_z \end{pmatrix},\quad \mbf{M}_O=\begin{pmatrix} M_x \\ M_y \\ M_z \end{pmatrix},\quad \mbf{R}=\begin{pmatrix} R_x \\ R_y \\ R_z \end{pmatrix},\quad \mbf{R}_1=\begin{pmatrix} R_{1x} \\ R_{1y} \\ R_{1z} \end{pmatrix},\quad \mbf{L}_O=\begin{pmatrix} L_x \\ L_y \\ L_z \end{pmatrix}
\end{align*}
由于角速度$\mbf{\omega}=\begin{pmatrix} 0 \\ 0 \\ \dot{\psi} \end{pmatrix}$,所以$\mbf{L}_O=\begin{pmatrix} I_{13}\dot{\psi} \\ I_{23}\dot{\psi} \\ I_{33}\dot{\psi} \end{pmatrix}$。由于$\mbf{v}_C=\mbf{\omega}\times\vec{OC}$,用$h$表示刚体上$O$和$O_1$的距离,方程\eqref{chapter6:定轴转动刚体的角动量方程}和\eqref{chapter6:定轴转动刚体的动量方程}的分量形式为
\begin{subnumcases}{\label{chapter6:定轴转动刚体动力学方程的分量形式}}
	-my_C\ddot{\psi}-mx_C\dot{\psi}^2=F_x+R_x+R_{1x} \label{chapter6:定轴转动刚体动力学方程的分量形式-1} \\
	mx_C\ddot{\psi}-my_C\dot{\psi}^2= F_y+R_y+R_{1y} \label{chapter6:定轴转动刚体动力学方程的分量形式-2} \\
	0=F_z+R_z+R_{1z} \label{chapter6:定轴转动刚体动力学方程的分量形式-3} \\
	I_{13}\ddot{\psi}-I_{23}\dot{\psi}^2=M_x-hR_{1y} \label{chapter6:定轴转动刚体动力学方程的分量形式-4} \\
	I_{23}\ddot{\psi}+I_{13}\dot{\psi}^2=M_y+hR_{1x} \label{chapter6:定轴转动刚体动力学方程的分量形式-5} \\
	I_{33}\ddot{\psi}=M_z \label{chapter6:定轴转动刚体动力学方程的分量形式-6}
\end{subnumcases}
方程\eqref{chapter6:定轴转动刚体动力学方程的分量形式-6}中不包含约束反力,是刚体定轴转动的运动微分方程,其余五个方程中包含了所有待求的约束反力。由方程\eqref{chapter6:定轴转动刚体动力学方程的分量形式-3}不能单独求出轴向约束反力分量$R_z$和$R_{1z}$,而仅能求出它们的和,这个和并不依赖于刚体的转动。而侧向约束反力分量$R_x,R_y,R_{1x},R_{1y}$可以有方程\eqref{chapter6:定轴转动刚体动力学方程的分量形式-1}、\eqref{chapter6:定轴转动刚体动力学方程的分量形式-2}、\eqref{chapter6:定轴转动刚体动力学方程的分量形式-4}和\eqref{chapter6:定轴转动刚体动力学方程的分量形式-5}求出,它们依赖于刚体的转动。

\begin{example}[偏斜的飞轮]
设有半径为$R$,质量为$m$的圆盘状飞轮安装在两个间距为$l$的轴承正中,但飞轮的法线方向与轴承之间有一夹角$\alpha$,求飞轮以$\omega$的角速度匀速绕轴旋转时,轴承上受到的额外负荷。
\end{example}
\begin{solution}
\begin{figure}[htb]
\centering
\begin{asy}
	size(250);
	//第六章例4图
	picture tmp;
	pair O,l,co,dash,P;
	real alpha,d,R,f,ff,F,omega,r;
	O = (0,0);
	l = (2,0);
	co = (1.8,0);
	d = 0.1;
	R = 1;
	f = 0.2;
	ff = 0.1;
	alpha = -20;
	F = 1.1;
	dash = 0.07*dir(60);
	omega = 1;
	draw(-l--l);
	draw(tmp,co+(-f,ff)--co+(f,ff));
	draw(tmp,co+(-f,-ff)--co+(f,-ff));
	for(int i=1;i<=5;i=i+1){
		P = co+(-f,ff)+(i-1)*2*f/4.5;
		draw(tmp,P--P+dash);
		P = co+(f,-ff)-(i-1)*2*f/5;
		draw(tmp,P--P-dash);
	}
	add(tmp);
	add(shift(-2*co)*tmp);
	erase(tmp);
	draw(Label("$\boldsymbol{F}_e$",EndPoint,black),co--co+F*dir(-90),red,Arrow);
	draw(Label("$\boldsymbol{F}_e$",EndPoint,black),-co--(-co+F*dir(90)),red,Arrow);
	label("$O$",O,NW);
	fill(rotate(alpha)*box((-d/2,-R),(d/2,R)),0.6white);
	draw(Label("$x$",EndPoint),O--1.3*dir(90+alpha),Arrow);
	draw(Label("$z$",EndPoint),O--1.3*dir(alpha),Arrow);
	draw(Label("$\boldsymbol{\omega}$",EndPoint,Relative(W)),O--omega*dir(0),red,Arrow);
	draw(omega*cos(alpha*pi/180)*dir(alpha)--omega*dir(0)--omega*sin(-alpha*pi/180)*dir(90+alpha),dashed);
	label("$R$",0.5*R*dir(alpha-90),dir(alpha+180));
	r = 0.4;
	draw(Label("$\alpha$",MidPoint,Relative(W)),arc(O,r,0,alpha),Arrow);
\end{asy}
\caption{例\theexample}
\label{第六章例4图}
\end{figure}

圆盘的主惯量为
\begin{equation*}
	I_1 = I_2 = \frac14 mR^2,\quad I_3 = \frac12 mR^2
\end{equation*}
在本体系中,角速度为
\begin{equation*}
	\omega_1 = \omega \sin \alpha,\quad \omega_2 = 0,\quad \omega_3 = \omega \cos \alpha
\end{equation*}
由于是匀速转动,可有
\begin{equation*}
	\dot{\omega}_1 = \dot{\omega}_2 = \dot{\omega}_3 = 0
\end{equation*}
Euler动力学方程为
\begin{equation*}
	\begin{cases}
		I_1 \dot{\omega}_1 - (I_2-I_3)\omega_2 \omega_3 = M_1 \\
		I_2 \dot{\omega}_2 - (I_3-I_1)\omega_3 \omega_1 = M_2 \\
		I_3 \dot{\omega}_3 - (I_1-I_2)\omega_1 \omega_2 = M_3
	\end{cases}
\end{equation*}
由此可得
\begin{equation*}
	\begin{cases}
		M_1 = 0 \\
		M_2 = -\dfrac18 mR^2 \omega^2 \sin 2\alpha \\ 
		M_3 = 0
	\end{cases}
\end{equation*}
因此轴承上受到的力(动约束反力)为
\begin{equation*}
	F_e = \frac{|M_2|}{l} = \frac{mR^2 \omega^2 \sin 2\alpha}{8l}
\end{equation*}
而轴承受到的静约束反力(重力负荷)为
\begin{equation*}
	F_g = \frac12 mg
\end{equation*}
当角速度$\omega$很大时,轴承受到的动约束反力可以远大于静约束反力。
\end{solution}

\subsection{动约束反力等于静约束反力的条件}

如果在方程\eqref{chapter6:定轴转动刚体动力学方程的分量形式-1}、\eqref{chapter6:定轴转动刚体动力学方程的分量形式-2}、\eqref{chapter6:定轴转动刚体动力学方程的分量形式-4}和\eqref{chapter6:定轴转动刚体动力学方程的分量形式-5}中令$\dot{\psi}=0,\ddot{\psi}=0$则可得到静约束反力的方程。如果刚体转动则$\dot{\psi}$和$\ddot{\psi}$中至少有一个非零,在一般情况下这些方程的左端将非零,即动约束反力与静约束反力不相等。

当动约束反力等于静约束反力时,需要满足
\begin{equation}
\begin{cases}
	y_C\ddot{\psi}+x_C\dot{\psi}^2=0 \\
	x_C\ddot{\psi}-y_C\dot{\psi}^2=0
\end{cases}
\label{chapter6:动反力等于静反力的条件1}
\end{equation}
和
\begin{equation}
\begin{cases}
	I_{13}\ddot{\psi}-I_{23}\dot{\psi}^2=0 \\
	I_{23}\ddot{\psi}+I_{13}\dot{\psi}^2=0
\end{cases}
\label{chapter6:动反力等于静反力的条件2}
\end{equation}
方程组\eqref{chapter6:动反力等于静反力的条件1}和\eqref{chapter6:动反力等于静反力的条件2}可以看作关于$\ddot{\psi}$和$\dot{\psi}^2$的齐次线性方程组,其有任意解的条件应为
\begin{equation*}
\begin{cases}
	x_C^2+y_C^2=0 \\
	I_{13}^2+I_{23}^2=0 \\
\end{cases}
\end{equation*}
即
\begin{equation}
	x_C=y_C=0,\quad I_{13}=I_{23}=0
	\label{chapter6:动反力等于静反力的条件}
\end{equation}
式\eqref{chapter6:动反力等于静反力的条件}表明,当转动轴过质心且与其中一个惯性主轴重合时,动约束反力才等于静约束反力。

\section{刚体定点运动的Euler情形}\label{chapter6:section-刚体定点运动的Euler情形}

\subsection{运动方程与守恒量}

当做定点运动的刚体不受外力矩作用时,称为刚体定点运动的{\heiti Euler情形},即$\mbf{M} = \mbf{0}$,由Euler动力学方程\eqref{Euler动力学方程}可得刚体的运动方程为
\begin{equation}
	\begin{cases}
		I_1 \dot{\omega}_1 - (I_2-I_3)\omega_2 \omega_3 = 0 \\
		I_2 \dot{\omega}_2 - (I_3-I_1)\omega_3 \omega_1 = 0 \\
		I_3 \dot{\omega}_3 - (I_1-I_2)\omega_1 \omega_2 = 0
	\end{cases}
	\label{chapter6:刚体定点运动的Euler情形下的运动方程}
\end{equation}
将\eqref{chapter6:刚体定点运动的Euler情形下的运动方程}各式分别乘以$I_1\omega_1$、$I_2\omega_2$、$I_3\omega_3$相加,得
\begin{equation*}
	I_1^2 \omega_1 \dot{\omega}_1 + I_2^2 \omega_2 \dot{\omega}_2 + I_3^2 \omega_3 \dot{\omega}_3 = 0
\end{equation*}
即有角动量守恒\footnote{也可同角动量定理得到。}
\begin{equation}
	\sum_{i=1}^3 I_i^2 \omega_i^2 = L^2 \quad\text{(常数)}
	\label{chapter6:刚体定点运动的Euler情形下的角动量守恒}
\end{equation}
将\eqref{chapter6:刚体定点运动的Euler情形下的运动方程}各式分别乘以$\omega_1$、$\omega_2$、$\omega_3$相加,得
\begin{equation*}
	I_1 \omega_1 \dot{\omega}_1 + I_2 \omega_2 \dot{\omega}_2 + I_3 \omega_3 \dot{\omega}_3 = 0
\end{equation*}
即有能量守恒\footnote{也可通过动能定理得到。}
\begin{equation}
	\frac12 \sum_{i=1}^3 I_i \omega_i^2 = T \quad\text{(常数)}
	\label{chapter6:刚体定点运动的Euler情形下的能量守恒}
\end{equation}

\subsection{Euler情形下的永久转动}\label{chapter6:subsection-Euler情形下的永久转动}

如果刚体的角速度保持不变,这种刚体的定点运动称为{\bf 永久转动},这时$\omega_1,\omega_2,\omega_3$都是常数。由方程\eqref{chapter6:刚体定点运动的Euler情形下的运动方程}可得
\begin{equation}
	(I_2-I_3)\omega_2\omega_3=0,\quad (I_3-I_1)\omega_3\omega_1=0,\quad (I_1-I_2)\omega_1\omega_2=0
	\label{chapter6:Euler情形下永久转动的条件}
\end{equation}
由此可以看出,只要$\omega_1,\omega_2,\omega_3$中有两个为零,第三个为任意值,则方程\eqref{chapter6:Euler情形下永久转动的条件}对于主惯量为任意值的刚体都满足。即刚体的永久转动只能绕着刚体对定点的惯性主轴,并且其角速度大小可以是任意的。

特别地,如果$I_1=I_2=I_3$,则方程\eqref{chapter6:Euler情形下永久转动的条件}对任意的$\omega_1,\omega_2,\omega_3$都成立,即刚体的转动轴可以是任意方向。事实上,当$I_1=I_2=I_3$时,刚体的惯性椭球成为球体,所以过定点的任意轴都是惯性主轴。

如果两个主惯量相等,例如$I_1=I_2$,则方程对$\omega_1=\omega_2=0$和任意的$\omega_3$都成立(绕惯性主轴$Oz$转动),同样对$\omega_3=0$和任意的$\omega_1,\omega_2$都成立(转动轴为通过定点且位于惯性主轴赤道面内的任意转动轴\footnote{此情形下,惯性椭球为旋转椭球,任何与$Oz$垂直的轴都是惯性主轴。})。

如果$I_1,I_2,I_3$各不相同,则方程\eqref{chapter6:Euler情形下永久转动的条件}的解只能是$\omega_1,\omega_2,\omega_3$中有两个为零,第三个为任意值,即转动绕惯性主轴。

\subsection{Euler情形下动力学对称刚体的运动}\label{chapter6:subsection-Euler情形下动力学对称刚体的运动}

如果刚体对定点的某两个主惯量相等,不妨设有$I_1=I_2$,则称刚体{\bf 动力学对称},轴$Oz$称为{\bf 动力学对称轴}。此时Euler动力学方程化为
\begin{equation}
	\begin{cases}
		I_1 \dot{\omega}_1 - (I_1-I_3)\omega_2 \omega_3 = 0 \\
		I_1 \dot{\omega}_2 - (I_3-I_1)\omega_3 \omega_1 = 0 \\
		I_3 \dot{\omega}_3 = 0
	\end{cases}
	\label{chapter6:动力学对称刚体的Euler情形运动方程}
\end{equation}
由方程\eqref{chapter6:动力学对称刚体的Euler情形运动方程}中的第三式可得
\begin{equation}
	\omega_3 = \varOmega \quad \text{(常数)}
	\label{对称Euler陀螺的解-1}
\end{equation}
记$n = \dfrac{I_3-I_1}{I_1} \varOmega$,则方程\eqref{chapter6:动力学对称刚体的Euler情形运动方程}的第一二式化为
\begin{equation*}
	\begin{cases}
		\dot{\omega}_1 + n\omega_2 = 0 \\
		\dot{\omega}_2 - n\omega_1 = 0
	\end{cases}
\end{equation*}
从中消去$\omega_2$可得
\begin{equation*}
	\ddot{\omega}_1 + n^2\omega_1 = 0
\end{equation*}
由此解得
\begin{equation}
\begin{cases}
	\omega_1 = \omega_0 \cos(nt+\eps) \\
	\omega_2 = \omega_0 \sin(nt+\eps)
\end{cases}
\label{对称Euler陀螺的解-2}
\end{equation}
综上可得刚体的角速度为
\begin{equation}
	\mbf{\omega} = \omega_1 \mbf{e}_1 + \omega_2 \mbf{e}_2 + \omega_3 \mbf{e}_3 = \omega_0 \cos(nt+\eps) \mbf{e}_1 + \omega_0 \sin(nt+\eps) \mbf{e}_2 + \varOmega \mbf{e}_3
	\label{对称Euler陀螺的解-3}
\end{equation}
角速度的大小为
\begin{equation*}
	\omega = \sqrt{\omega_1^2 +\omega_2^2 +\omega_3^2} = \sqrt{\omega_0^2 + \varOmega^2}\quad \text{(常数)}
\end{equation*}
该角速度与主轴系$z$轴夹角是固定的$\alpha = \arctan \dfrac{\omega_0}{\varOmega}$。

Euler情形下,动力学对称刚体的角速度位于主轴系的一个半顶角为$\alpha$,$z$轴为对称轴的锥面(称为{\heiti 本体极锥})上,绕$z$轴以匀角速度$n$旋转。

\begin{figure}[htb]
\centering
\begin{minipage}[t]{0.45\textwidth}
\centering
\begin{asy}
	size(200);
	//对称Euler陀螺的角速度矢量
	picture tmp;
	pair O,i,j,k;
	real r,h,rr,pos;
	O = (0,0);
	i = (-sqrt(2)/4,-sqrt(14)/12);
	j = (sqrt(14)/4,-sqrt(2)/12);
	k = (0,2*sqrt(2)/3);
	pos = 1.6;
	draw(Label("$x$",EndPoint),O--2*i,Arrow);
	draw(Label("$y$",EndPoint),O--2*j,Arrow);
	draw(Label("$z$",EndPoint),O--2*k,Arrow);
	label("$O$",O,W);
	r = 1;
	h = 1.5;
	rr = 0.4;
	pair cir(real theta){
		return r*cos(theta)*i+r*sin(theta)*j+h*k;
	}
	pair bot(real theta){
		return rr*cos(theta)*i+rr*sin(theta)*j;
	}
	draw(graph(cir,0,2*pi),dashed);
	draw(O--cir(pos),dashed);
	draw(xscale(-1)*(O--cir(pos)),dashed);
	draw(Label("$\alpha$",MidPoint,Relative(W)),arc(O,rr,90,degrees(cir(pos))),Arrow);
	draw(Label("$\boldsymbol{\omega}$",EndPoint,black),O--cir(pi/3),red,Arrow);
	draw(Label("$\boldsymbol{\omega}_0$",EndPoint,black),O--cir(pi/3)-h*k,red,Arrow);
	draw(Label("$\boldsymbol{\varOmega}$",EndPoint,Relative(W),black),O--h*k,red,Arrow);
	draw(h*k--cir(pi/3)--cir(pi/3)-h*k,dashed);
	draw(Label("$nt+\varepsilon$",MidPoint,Relative(E)),graph(bot,0,pi/3),Arrow);
\end{asy}
\caption{Euler情形下动力学对称刚体的角速度矢量}
\label{对称Euler陀螺的角速度矢量}
\end{minipage}
\hspace{0.5cm}
\begin{minipage}[t]{0.45\textwidth}
\centering
\begin{asy}
	size(200);
	//对称Euler陀螺的角动量矢量
	picture tmp;
	pair O,i,j,k;
	real r,h,rr,pos;
	O = (0,0);
	i = (-sqrt(2)/4,-sqrt(14)/12);
	j = (sqrt(14)/4,-sqrt(2)/12);
	k = (0,2*sqrt(2)/3);
	pos = 1.8;
	draw(Label("$x$",EndPoint),O--2*i,Arrow);
	draw(Label("$y$",EndPoint),O--2*j,Arrow);
	draw(Label("$z$",EndPoint),O--2*k,Arrow);
	label("$O$",O,W);
	r = 0.8;
	h = 1.6;
	rr = 0.4;
	pair cir(real theta){
		return r*cos(theta)*i+r*sin(theta)*j+h*k;
	}
	pair bot(real theta){
		return rr*cos(theta)*i+rr*sin(theta)*j;
	}
	draw(graph(cir,0,2*pi),dashed);
	draw(O--cir(pos),dashed);
	draw(xscale(-1)*(O--cir(pos)),dashed);
	draw(Label("$\theta$",MidPoint,Relative(W)),arc(O,rr,90,degrees(cir(pos))),Arrow);
	draw(Label("$Z$",EndPoint,black),O--1.4*cir(pi/3),Arrow);
	draw(Label("$\boldsymbol{L}$",EndPoint,NW,black),O--cir(pi/3),blue,Arrow);
	draw(Label("$I_1\boldsymbol{\omega}_0$",EndPoint,black),O--cir(pi/3)-h*k,blue,Arrow);
	draw(Label("$I_3\boldsymbol{\varOmega}$",EndPoint,Relative(W),black),O--h*k,blue,Arrow);
	draw(h*k--cir(pi/3)--cir(pi/3)-h*k,dashed);
	draw(Label("$nt+\varepsilon$",MidPoint,Relative(E)),graph(bot,0,pi/3),Arrow);
\end{asy}
\caption{Euler情形下动力学对称刚体的角动量矢量}
\label{对称Euler陀螺的角动量矢量}
\end{minipage}
\end{figure}

下面考虑角动量矢量$\mbf{L}$。根据角动量守恒,$\mbf{L}$为常矢量,方向不变,可取平动系基矢量为$\mbf{E}_Z = \dfrac{\mbf{L}}{L}$。具体可有
\begin{equation*}
	\mbf{L} = I_1(\omega_1 \mbf{e}_1+\omega_2 \mbf{e}_2) + I_3 \omega_3 \mbf{e}_3 = I_1 \omega_0 \big[\cos(nt+\eps)\mbf{e}_1 + \sin(nt+\eps)\mbf{e}_2\big] + I_3 \varOmega \mbf{e}_3
\end{equation*}
由此可以看出矢量$\mbf{e}_3$、$\mbf{\omega}$和$\mbf{L}$共面,角动量的大小
\begin{equation*}
	L = \sqrt{I_1^2 \omega_1^2+ I_2^2 \omega_2^2+ I_3^2 \omega_3^2} = \sqrt{I_1^2 \omega_0^2 + I_3^2 \varOmega^2}\quad \text{(常数)}
\end{equation*}
章动角$\theta$满足
\begin{equation*}
	\tan \theta = \frac{I_1 \omega_0}{I_3 \varOmega} = \frac{I_1}{I_3} \tan \alpha \quad \text{(常数)}
\end{equation*}
即有$\dot{\theta} = 0$,无章动。

再由Euler运动学方程
\begin{equation*}
	\begin{cases}
		\omega_1 = \dot{\phi} \sin \theta \sin \psi \\
		\omega_2 = \dot{\phi} \sin \theta \cos \psi \\
		\omega_3 = \dot{\phi} \cos \theta + \dot{\psi}
	\end{cases}
\end{equation*}
由此可得自转角$\psi$满足
\begin{equation*}
	\tan \psi = \frac{\omega_1}{\omega_2} = \cot (nt+\eps)
\end{equation*}
即有自转角$\psi = -nt+ \dfrac{\pi}{2} - \eps$。由此可得自转角速度以及进动角速度
\begin{equation*}
	\begin{cases}
		\displaystyle \dot{\psi} = -n = \frac{I_1-I_3}{I_1} \varOmega \quad \text{(常数)} \\[1.5ex]
		\displaystyle \dot{\phi} = \frac{\omega_3 - \dot{\psi}}{\cos \theta} = \frac{L}{I_1}>0 \quad \text{(常数)}
	\end{cases}
\end{equation*}
即有第三主轴绕角动量矢量以匀角速度逆时针进动。章动角恒定,进动和自转角速度恒定的定点运动称为{\heiti 规则进动}。考虑到$\mbf{e}_z$、$\mbf{\omega}$和$\mbf{L}$共面,故角速度矢量绕角动量矢量以匀角速度逆时针旋转,由于
\begin{equation*}
	\mbf{\omega} \cdot \frac{\mbf{L}}{L} = \frac{2T}{L}\quad \text{(常数)}
\end{equation*}
故角速度矢量位于空间系锥面(称为{\heiti 空间极锥})上。角$\alpha$和$\theta$满足
\begin{equation}
	\frac{\tan \theta}{\tan \alpha} = \frac{I_1}{I_3}
	\label{chapter6:Euler情形下动力学对称刚体的章动角}
\end{equation}
它们决定了空间极锥与本体极锥之间的相对位置。Euler情形下动力学对称刚体可能的三种空间极锥与本体极锥之间的相对位置关系如图\ref{空间极锥与本体极锥情形1}、\ref{空间极锥与本体极锥情形2}和\ref{空间极锥与本体极锥情形3}所示。

\begin{figure}[htbp]
\centering
\begin{minipage}[t]{0.45\textwidth}
\centering
\begin{asy}
	size(200);
	//空间极锥与本体极锥情形1
	picture tmp;
	pair O,i,j,k;
	real r,r1,r2,h,h1,h2,theta,alpha,pos,L;
	O = (0,0);
	i = (-sqrt(2)/4,-sqrt(14)/12);
	j = (sqrt(14)/4,-sqrt(2)/12);
	k = (0,2*sqrt(2)/3);
	pos = 1.88;
	r1 = 0.5;
	r2 = 0.8;
	h1 = 1.6;
	h2 = sqrt(r1*r1+h1*h1-r2*r2);
	L = 2;
	pair cir(real theta){
	return r*cos(theta)*i+r*sin(theta)*j+h*k;
	}
	r = r1;
	h = h1;
	draw(graph(cir,0,2*pi),blue+dashed);
	draw(O--cir(pos),blue+dashed);
	draw(xscale(-1)*(O--cir(pos)),blue+dashed);
	theta = 90-degrees(cir(pos));
	r = r2;
	h = h2;
	pos = 1.80;
	draw(tmp,graph(cir,0,2*pi),red+dashed);
	draw(tmp,O--cir(pos),red+dashed);
	draw(tmp,xscale(-1)*(O--cir(pos)),red+dashed);
	alpha = 90-degrees(cir(pos));
	add(rotate(alpha+theta)*tmp);
	draw(Label("$\boldsymbol{L}$",EndPoint,black),O--L*dir(90),blue,Arrow);
	draw(Label("$\boldsymbol{e}_3$",EndPoint),O--L*dir(90+alpha+theta),Arrow);
	//画角速度
	real omega,phi,psi,ralpha,rtheta,r0;
	ralpha = alpha*pi/180;
	rtheta = theta*pi/180;
	omega = 1.2;
	phi = sin(ralpha)/sin(ralpha+rtheta)*omega;
	psi = sin(rtheta)/sin(ralpha+rtheta)*omega;
	draw(Label("$\boldsymbol{\omega}$",EndPoint,Relative(W),black),O--omega*dir(90+theta),red,Arrow);
	draw(Label("$\dot{\phi}$",EndPoint,Relative(E),black),O--phi*dir(90),red,Arrow);
	draw(Label("$\dot{\psi}$",EndPoint,Relative(W),black),O--psi*dir(90+alpha+theta),red,Arrow);
	draw(psi*dir(90+alpha+theta)--omega*dir(90+theta)--phi*dir(90),dashed);
	r0 = 0.2;
	draw(Label("$\alpha$",MidPoint,Relative(E)),arc(O,r0,90+theta,90+alpha+theta),Arrow);
	r0 = 0.35;
	draw(Label("$\theta$",MidPoint,Relative(E)),arc(O,r0,90,90+alpha+theta),Arrow);
\end{asy}
\caption{$I_1=I_2>I_3$,$\theta>\alpha$}
\label{空间极锥与本体极锥情形1}
\end{minipage}
\hspace{0.5cm}
\begin{minipage}[t]{0.45\textwidth}
\centering
\begin{asy}
	size(200);
	//空间极锥与本体极锥情形2
	picture tmp;
	pair O,i,j,k;
	real r,r1,r2,h,h1,h2,theta,alpha,pos,L;
	O = (0,0);
	i = (-sqrt(2)/4,-sqrt(14)/12);
	j = (sqrt(14)/4,-sqrt(2)/12);
	k = (0,2*sqrt(2)/3);
	pos = 1.88;
	r1 = 0.5;
	r2 = 0.8;
	h1 = 1.6;
	h2 = sqrt(r1*r1+h1*h1-r2*r2);
	L = 2;
	pair cir(real theta){
		return r*cos(theta)*i+r*sin(theta)*j+h*k;
	}
	theta = 0;
	r = r2;
	h = h2;
	pos = 1.80;
	draw(tmp,graph(cir,0,2*pi),red+dashed);
	draw(tmp,O--cir(pos),red+dashed);
	draw(tmp,xscale(-1)*(O--cir(pos)),red+dashed);
	alpha = 90-degrees(cir(pos));
	add(rotate(alpha+theta)*tmp);
	draw(Label("$\boldsymbol{L}$",EndPoint,black),O--L*dir(90),blue,Arrow);
	draw(Label("$\boldsymbol{e}_3$",EndPoint),O--L*dir(90+alpha+theta),Arrow);
	//画角速度
	real omega,r0;
	omega = 1;
	draw(Label("$\boldsymbol{\omega}$",EndPoint,Relative(E),black),O--omega*dir(90+theta),red,Arrow);
	label("$\dot{\phi}$",0.8*omega*dir(90),E);
	r0 = 0.3;
	draw(Label("$\theta$",MidPoint,Relative(E)),arc(O,r0,90,90+alpha),Arrow);
	r0 = 0.3;
	draw(Label("$\alpha$",MidPoint,Relative(E)),arc(O,r0,90+alpha,90+2*alpha),Arrow);
\end{asy}
\caption{$I_1=I_2=I_3$,$\theta=\alpha$}
\label{空间极锥与本体极锥情形2}
\end{minipage}
\\
\vspace{0.2cm}
\begin{minipage}{0.45\textwidth}
\centering
\begin{asy}
	size(250);
	//空间极锥与本体极锥情形3
	picture tmp;
	pair O,i,j,k;
	real r,r1,r2,h,h1,h2,theta,alpha,pos,L;
	O = (0,0);
	i = (-sqrt(2)/4,-sqrt(14)/12);
	j = (sqrt(14)/4,-sqrt(2)/12);
	k = (0,2*sqrt(2)/3);
	pos = 1.88;
	r1 = 0.5;
	r2 = 0.8;
	h1 = 1.6;
	h2 = sqrt(r1*r1+h1*h1-r2*r2);
	L = 2;
	pair cir(real theta){
	return r*cos(theta)*i+r*sin(theta)*j+h*k;
	}
	r = r1;
	h = h1;
	draw(graph(cir,0,2*pi),blue+dashed);
	draw(O--cir(pos),blue+dashed);
	draw(xscale(-1)*(O--cir(pos)),blue+dashed);
	theta = 90-degrees(cir(pos));
	r = r2;
	h = h2;
	pos = 1.80;
	draw(tmp,graph(cir,0,2*pi),red+dashed);
	draw(tmp,O--cir(pos),red+dashed);
	draw(tmp,xscale(-1)*(O--cir(pos)),red+dashed);
	alpha = 90-degrees(cir(pos));
	add(rotate(alpha-theta)*tmp);
	draw(Label("$\boldsymbol{L}$",EndPoint,black),O--L*dir(90),blue,Arrow);
	draw(Label("$\boldsymbol{e}_3$",EndPoint),O--L*dir(90+alpha-theta),Arrow);
	//画角速度
	real omega,phi,psi,ralpha,rtheta,r0;
	theta = -theta;
	ralpha = alpha*pi/180;
	rtheta = theta*pi/180;
	omega = 0.3;
	phi = sin(ralpha)/sin(ralpha+rtheta)*omega;
	psi = sin(rtheta)/sin(ralpha+rtheta)*omega;
	draw(Label("$\boldsymbol{\omega}$",EndPoint,Relative(E),black),O--omega*dir(90+theta),red,Arrow);
	draw(Label("$\dot{\phi}$",EndPoint,Relative(E),black),O--phi*dir(90),red,Arrow);
	draw(Label("$\dot{\psi}$",EndPoint,Relative(W),black),O--psi*dir(90+alpha+theta),red,Arrow);
	draw(psi*dir(90+alpha+theta)--omega*dir(90+theta)--phi*dir(90),dashed);
	r0 = 0.2;
	draw(Label("$\alpha$",MidPoint,Relative(E)),arc(O,r0,90+alpha+theta,90+2*alpha+theta),Arrow);
	r0 = 0.35;
	draw(Label("$\theta$",MidPoint,Relative(E)),arc(O,r0,90,90+alpha+theta),Arrow);
\end{asy}
\caption{$I_1=I_2<I_3$,$\theta<\alpha$}
\label{空间极锥与本体极锥情形3}
\end{minipage}
\end{figure}

由于本体极锥相对本体系静止,故本体极锥的运动即代表了刚体的运动,角速度线上的各点在平动系中瞬时静止,即为瞬时转轴方向。Euler情形下动力学对称刚体运动的几何图像即为本体极锥贴着空间极锥作纯滚动。

\begin{example}
试证明,在Euler情形下,当主惯量满足$I_1=I_2<I_3$时,刚体空间极锥的轴和母线的夹角不超过$\arctan\dfrac{\sqrt{2}}{4}$。
\end{example}
\begin{solution}
当$I_1=I_2<I_3$时,空间极锥与本体极锥之间的关系见图{空间极锥与本体极锥情形3}。由式\eqref{chapter6:Euler情形下动力学对称刚体的章动角}可得
\begin{equation*}
	\tan\alpha=\frac{I_3}{I_1}\tan\theta
\end{equation*}
记$\gamma=\dfrac{I_3}{I_1}$,根据主惯量的三角不等式\eqref{chapter6:主惯量的三角不等式}可得此时有$2I_1\geqslant I_3$,即$1<\gamma\leqslant 2$。

记空间极锥的轴和母线的夹角为$\beta$,则有
\begin{align*}
	\tan\beta & = \tan(\alpha-\theta) = \frac{\tan\alpha-\tan\theta}{1+\tan\alpha\tan\theta} = \frac{(\gamma-1)\tan\theta}{1+\gamma\tan^2\theta} = \frac{\gamma-1}{2\sqrt{\gamma}}\frac{2\sqrt{\gamma}\tan\theta}{1-(\sqrt{\gamma}\tan\theta)^2} \\
	& \leqslant \frac{\gamma-1}{2\sqrt{\gamma}} = \frac12 \left(\sqrt{\gamma}-\frac{1}{\sqrt{\gamma}}\right) \leqslant \frac12 \left(\sqrt{2}-\frac{1}{\sqrt{2}}\right) = \frac{\sqrt{2}}{4}
\end{align*}
由此即有$\beta\leqslant \arctan\dfrac{\sqrt{2}}{4}$。
\end{solution}

\subsection{Euler情形下一般刚体的运动*}

\subsubsection{Euler情形下一般刚体的守恒量}

现在利用Euler动力学方程研究三个主惯量各不相等的非对称刚体的自由转动。为方便起见,假定
\begin{equation*}
	I_3>I_2>I_1
\end{equation*}
由角动量守恒和能量守恒可得
\begin{subnumcases}{\label{非对称Euler陀螺的守恒律方程}}
	I_1^2 \omega_1^2+I_2^2 \omega_2^2+I_3^2 \omega_3^2 = L^2 \\
	I_1 \omega_1^2+I_2 \omega_2^2+I_3 \omega_3^2 = 2T
\end{subnumcases}
其中$L$和$T$为常数。这两个等式可以用$\mbf{L}$的三个分量表示为
\begin{subnumcases}{}
	L_1^2+L_2^2+L_3^2 = L^2 \label{非对称Euler陀螺-角动量守恒} \\
	\frac{L_1^2}{I_1}+\frac{L_2^2}{I_2}+\frac{L_3^2}{I_3} = 2T \label{非对称Euler陀螺-能量守恒} 
\end{subnumcases}
在以$L_1,L_2,L_3$为轴的坐标系中,方程\eqref{非对称Euler陀螺-能量守恒}是半轴分别
\begin{equation*}
	\sqrt{2I_1T},\sqrt{2I_2T},\sqrt{2I_3T}
\end{equation*}
的椭球面,而方程\eqref{非对称Euler陀螺-角动量守恒}则为半径为$L$的球面方程。当矢量$\mbf{L}$相对刚体的主轴移动时,其端点将沿着这两个曲面的交线运动。交线存在的条件由不等式
\begin{equation}
	2I_1T\leqslant L^2\leqslant 2I_3T
\end{equation}
给出,即表示球\eqref{非对称Euler陀螺-角动量守恒}的半径介于椭球\eqref{非对称Euler陀螺-能量守恒}的最长半轴和最短半轴之间。图\ref{不同情况下角动量矢量末端的轨迹}画出了椭球与不同半径的球面的一系列这样的交线。

\begin{figure}[htb]
\centering
\begin{asy}
	size(300);
	//不同情况下角动量矢量末端的轨迹
	pair O,i,j,k;
	real a,b,c,L;
	O = (0,0);
	i = (-sqrt(2)/4,-sqrt(14)/12);
	j = (sqrt(14)/4,-sqrt(2)/12);
	k = (0,2*sqrt(2)/3);
	a = 3;
	b = 2;
	c = 1;
	pair ell1(real t){
		return a*cos(t)*i+b*sin(t)*j;
	}
	pair ell2(real t){
		return b*cos(t)*j+c*sin(t)*k;
	}
	pair ell3(real t){
		return c*cos(t)*k+a*sin(t)*i;
	}
	real la,lb,theta;
	la = 2.175;
	lb = 1.315;
	theta = 10.5;
	draw(rotate(theta)*xscale(la)*yscale(lb)*unitcircle,linewidth(0.8bp));
	real tmp1,tmp2;
	tmp1 = -1.15;
	draw(graph(ell1,tmp1,tmp1+pi),linewidth(0.5bp));
	draw(graph(ell1,tmp1+pi,tmp1+2*pi),dashed);
	tmp1 = -0.5;
	draw(graph(ell2,tmp1,tmp1+pi),linewidth(0.5bp));
	draw(graph(ell2,tmp1+pi,tmp1+2*pi),dashed);
	tmp1 = -0.9;
	draw(graph(ell3,tmp1,tmp1+pi),linewidth(0.5bp));
	draw(graph(ell3,tmp1+pi,tmp1+2*pi),dashed);
	pair ell_cur1(real t){
		real x,y,z;
		x = a*cos(t)/sqrt((a^2-c^2)/(L^2-c^2));
		y = b*sin(t)/sqrt((b^2-c^2)/(L^2-c^2));
		z = sqrt(L^2-x^2-y^2);
		return x*i+y*j+z*k;
	}
	pair ell_cur2(real t){
		real x,y,z;
		x = a*cos(t)/sqrt((a^2-c^2)/(L^2-c^2));
		y = b*sin(t)/sqrt((b^2-c^2)/(L^2-c^2));
		z = -sqrt(L^2-x^2-y^2);
		return x*i+y*j+z*k;
	}
	L = 1.2;
	draw(graph(ell_cur1,0,2*pi),linewidth(0.8bp)+red);
	draw(graph(ell_cur2,0,2*pi),dashed+red);
	L = 1.5;
	draw(graph(ell_cur1,0,2*pi),linewidth(0.8bp)+0.7red+0.3blue);
	draw(graph(ell_cur2,0,2*pi),dashed+0.7red+0.3blue);
	L = 1.8;
	draw(graph(ell_cur1,0,2*pi),linewidth(0.8bp)+0.3red+0.7blue);
	draw(graph(ell_cur2,0,2*pi),dashed+0.3red+0.7blue);
	L = 2;
	tmp1 = -1.15;
	tmp2 = 2*pi-2.8;
	draw(graph(ell_cur1,tmp1,tmp2),linewidth(0.8bp)+blue);
	draw(graph(ell_cur1,tmp2,2*pi+tmp1),dashed+blue);
	draw(graph(ell_cur2,pi+tmp1,pi+tmp2),dashed+blue);
	draw(graph(ell_cur2,pi+tmp2,3*pi+tmp1),linewidth(0.8bp)+blue);
	pair ell_cur3(real t){
		real x,y,z;
		y = b*cos(t)/sqrt((a^2-b^2)/(a^2-L^2));
		z = c*sin(t)/sqrt((a^2-c^2)/(a^2-L^2));
		x = sqrt(L^2-y^2-z^2);
		return x*i+y*j+z*k;
	}
	pair ell_cur4(real t){
		real x,y,z;
		y = b*cos(t)/sqrt((a^2-b^2)/(a^2-L^2));
		z = c*sin(t)/sqrt((a^2-c^2)/(a^2-L^2));
		x = -sqrt(L^2-y^2-z^2);
		return x*i+y*j+z*k;
	}
	L = 2.2;
	draw(graph(ell_cur3,0,2*pi),linewidth(0.8bp)+0.8blue+0.2green);
	draw(graph(ell_cur4,0,2*pi),dashed+0.8blue+0.2green);
	L = 2.5;
	draw(graph(ell_cur3,0,2*pi),linewidth(0.8bp)+0.5blue+0.5green);
	draw(graph(ell_cur4,0,2*pi),dashed+0.5blue+0.5green);
	L = 2.8;
	draw(graph(ell_cur3,0,2*pi),linewidth(0.8bp)+green);
	draw(graph(ell_cur4,0,2*pi),dashed+green);
	// draw(O--a*i,dashed);
	// draw(O--b*j,dashed);
	// draw(O--c*k,dashed);
	draw(Label("$x_3$",EndPoint),a*i--(a+2)*i,Arrow);
	draw(Label("$x_2$",EndPoint),b*j--(b+0.6)*j,Arrow);
	draw(Label("$x_1$",EndPoint),c*k--(c+1)*k,Arrow);
\end{asy}
\caption{不同情况下角动量矢量末端的轨迹}
\label{不同情况下角动量矢量末端的轨迹}
\end{figure}

现在考虑在给定能量$T$时,角动量大小$L$的变化引起矢量$\mbf{L}$的端点的轨迹性质的变化。当$L^2$略大于$2I_1T$时,球和椭球相交于椭球极点附近围绕$x_1$轴的两条很小的封闭曲线(图\ref{不同情况下角动量矢量末端的轨迹}中红色线)。当$L^2\to 2I_1T$时,两条曲线分别收缩到极点。随着$L^2$的增大,曲线也随之扩大,当$L^2=2I_2T$时,曲线变成两条平面曲线(椭圆,图\ref{不同情况下角动量矢量末端的轨迹}中蓝色线),并相交于椭球在$x_2$轴上的极点。$L^2$再继续增大,将再次出现两条分离的封闭曲线,分别围绕$x_3$轴的两个极点(图\ref{不同情况下角动量矢量末端的轨迹}中绿色线)。而当$L^2\to 2I_3T$时,这两条曲线再次收缩到两个极点。

首先,轨迹的封闭性意味着矢量$\mbf{L}$在本体系中的运动是周期性的。其次,在椭球不同极点附近的轨迹性质具有本质的区别。在$x_1$轴和$x_3$轴附近,轨迹完全位于相应的极点周围,而在$x_2$轴极点附近的轨迹将远离这个极点。这种差别即相应于刚体绕三个惯量主轴转动时有不同的稳定性。绕$x_1$轴和$x_3$轴(对应于该刚体三个主惯量中的最小值和最大值)的转动是稳定的,即如果使刚体稍微偏离这些状态时,刚体将继续在初始状态附近运动。而绕$x_2$轴的转动是不稳定的,即任意小的偏离都将导致刚体远离其初始位置运动。关于稳定性的更多讨论请见第\ref{chapter6:subsubsection-Euler情形下刚体永久转动的稳定性}节。

为了得到$\mbf{\omega}$分量对时间的依赖关系,利用方程\eqref{非对称Euler陀螺的守恒律方程}将$\omega_1$和$\omega_3$用$\omega_2$表示,可得
\begin{subnumcases}{}
	\omega_1^2 = \frac{(2I_3T-L^2)-I_2(I_3-I_2)\omega_2^2}{I_1(I_3-I_1)} \\
	\omega_3^2 = \frac{(L^2-2I_1T)-I_2(I_2-I_1)\omega_2^2}{I_3(I_3-I_1)}
\end{subnumcases}
将上述两个关系代入Euler动力学方程\eqref{Euler动力学方程}的第二式中,即有
\begin{align}
	\frac{\mathrm{d}\omega_2}{\mathrm{d}t} & = \frac{\sqrt{\big[(2I_3T-L^2)-I_2(I_3-I_2)\omega_2^2\big] \big[(L^2-2I_1T)-I_2(I_2-I_1)\omega_2^2\big]}}{I_2\sqrt{I_1I_3}}
	\label{非对称Euler陀螺的运动方程}
\end{align}
方程\eqref{非对称Euler陀螺的运动方程}可以分离变量,将其积分即可得到关系$t(\omega_2)$。为了将其化为标准形式,根据$L^2$的取值分为如下几种情况:
\begin{enumerate}
\item 当$2I_2T<L^2\leqslant 2I_3T$时,作变量代换
\begin{equation}
	\tau = \sqrt{\frac{(I_3-I_2)(L^2-2I_1T)}{I_1I_2I_3}}t,\quad s = \sqrt{\frac{I_2(I_3-I_2)}{2I_3T-L^2}}\omega_2
\end{equation}
并引入参数
\begin{equation}
	k^2 = \frac{(I_2-I_1)(2I_3T-L^2)}{(I_3-I_2)(L^2-2I_1T)}<1
\end{equation}
可将方程\eqref{非对称Euler陀螺的运动方程}化为
\begin{equation*}
	\frac{\mathrm{d}s}{\mathrm{d}\tau} = \sqrt{(1-s^2)(1-k^2s^2)}
\end{equation*}
将其分离变量并积分,可得
\begin{equation}
	\tau = \int_0^s \frac{\mathrm{d}\xi}{\sqrt{(1-\xi^2)(1-k^2\xi^2)}} = K(k,s)
	\label{非对称Euler陀螺的运动方程的解1}
\end{equation}
此处选择$\omega_2=0$的时刻为时间起点。式\eqref{非对称Euler陀螺的运动方程的解1}中右端的函数为{\bf 第一类不完全椭圆积分}\footnote{关于椭圆积分和Jacobi椭圆函数的相关知识,请见附录\ref{椭圆积分与Jacobi椭圆函数}。}。求其反函数可得Jacobi椭圆函数
\begin{equation*}
	s = \sn \tau
\end{equation*}
即
\begin{equation}
	\omega_2 = \sqrt{\frac{2I_3T-L^2}{I_2(I_3-I_2)}}\sn \tau = \sqrt{\frac{2I_3T-L^2}{I_2(I_3-I_2)}}\sn \left(\sqrt{\frac{(I_3-I_2)(L^2-2I_1T)}{I_1I_2I_3}}t\right)
\end{equation}
由此可得如下公式
\begin{equation}
\begin{cases}
	\displaystyle \omega_1 = \sqrt{\frac{2I_3T-L^2}{I_1(I_3-I_1)}}\cn \tau = \sqrt{\frac{2I_3T-L^2}{I_1(I_3-I_1)}}\cn \left(\sqrt{\frac{(I_3-I_2)(L^2-2I_1T)}{I_1I_2I_3}}t\right) \\
	\displaystyle \omega_2 = \sqrt{\frac{2I_3T-L^2}{I_2(I_3-I_2)}}\sn \tau = \sqrt{\frac{2I_3T-L^2}{I_2(I_3-I_2)}}\sn \left(\sqrt{\frac{(I_3-I_2)(L^2-2I_1T)}{I_1I_2I_3}}t\right) \\
	\displaystyle \omega_3 = \sqrt{\frac{L^2-2I_1T}{I_3(I_3-I_1)}}\dn \tau = \sqrt{\frac{L^2-2I_1T}{I_3(I_3-I_1)}}\dn \left(\sqrt{\frac{(I_3-I_2)(L^2-2I_1T)}{I_1I_2I_3}}t\right)
\end{cases}
\label{非对称Euler陀螺的运动方程的解2}
\end{equation}
函数$\sn \tau$是周期的,对变量$\tau$的周期为$4K(k)$,其中$K(k)$为第一类完全椭圆积分
\begin{equation*}
	K(k) = \int_0^1 \frac{\mathrm{d}s}{\sqrt{(1-s^2)(1-k^2s^2)}} = \int_0^{\frac{\pi}{2}} \frac{\mathrm{d}\phi}{\sqrt{1-k^2\sin^2\phi}}
\end{equation*}
因而,解\eqref{非对称Euler陀螺的运动方程的解2}对时间$t$的周期为
\begin{equation}
	T_0 = 4K(k)\sqrt{\frac{I_1I_2I_3}{(I_3-I_2)(L^2-2I_1T)}}
	\label{chapter6:非对称Euler陀螺的运动方程解的周期}
\end{equation}
即经过时间$T_0$后,矢量$\mbf{\omega}$在本体系中回到原位置,然而在空间系中,刚体自身的取向还需进一步讨论。

特别地,当$L^2=2I_3T$时,有
\begin{equation*}
	\omega_1 = \omega_2 = 0,\quad \omega_3 = \sqrt{\frac{2T}{I_3}}\,\,\text{(常数)}
\end{equation*}
即矢量$\mbf{\omega}$的方向总是沿着对称轴$x_3$,即刚体绕$x_3$轴匀速转动,此种情况即为第\ref{chapter6:subsection-Euler情形下的永久转动}节讨论过的永久转动。

\item 当$2I_1T\leqslant L^2<2I_2T$时,作变量代换
\begin{equation}
	\tau = \sqrt{\frac{(I_2-I_1)(2I_3T-L^2)}{I_1I_2I_3}}t,\quad s = \sqrt{\frac{I_2(I_2-I_1)}{L^2-2I_1T}}\omega_2
\end{equation}
并引入参数
\begin{equation}
	k^2 = \frac{(I_3-I_2)(L^2-2I_1T)}{(I_2-I_1)(2I_3T-L^2)}<1
\end{equation}
可将方程\eqref{非对称Euler陀螺的运动方程}化为
\begin{equation*}
	\frac{\mathrm{d}s}{\mathrm{d}\tau} = \sqrt{(1-s^2)(1-k^2s^2)}
\end{equation*}
这个方程的解为
\begin{equation*}
	s = \sn \tau
\end{equation*}
此时的解为
\begin{equation}
\begin{cases}
	\displaystyle \omega_1 = \sqrt{\frac{2I_3T-L^2}{I_1(I_3-I_1)}}\dn \tau = \sqrt{\frac{2I_3T-L^2}{I_1(I_3-I_1)}}\dn \left(\sqrt{\frac{(I_2-I_1)(2I_3T-L^2)}{I_1I_2I_3}}t\right) \\
	\displaystyle \omega_2 = \sqrt{\frac{L^2-2I_1T}{I_2(I_2-I_1)}}\sn \tau = \sqrt{\frac{L^2-2I_1T}{I_2(I_2-I_1)}}\sn \left(\sqrt{\frac{(I_2-I_1)(2I_3T-L^2)}{I_1I_2I_3}}t\right) \\
	\displaystyle \omega_3 = \sqrt{\frac{L^2-2I_1T}{I_3(I_3-I_1)}}\cn \tau = \sqrt{\frac{L^2-2I_1T}{I_3(I_3-I_1)}}\cn \left(\sqrt{\frac{(I_2-I_1)(2I_3T-L^2)}{I_1I_2I_3}}t\right)
\end{cases}
\label{非对称Euler陀螺的运动方程的解3}
\end{equation}

当$I_1=I_2$时,考虑当$I_1\to I_2$时,参数$k^2\to 0$,椭圆函数退化为三角函数,级
\begin{equation*}
	\sn \tau \to \sin \tau,\quad \cn\tau \to \cos \tau,\quad \dn \tau \to 1
\end{equation*}
于是,解\eqref{非对称Euler陀螺的运动方程的解3}就退化到对称情形的解,即式\eqref{对称Euler陀螺的解-1}和式\eqref{对称Euler陀螺的解-2}的结果。

类似于上一种情况,当$L^2=2I_1T$时,有
\begin{equation*}
	\omega_1 = \sqrt{\frac{2T}{I_1}}\,\,\text{(常数)},\quad \omega_2 = \omega_3 = 0
\end{equation*}
即矢量$\mbf{\omega}$的方向总是沿着对称轴$x_1$,即刚体绕$x_1$轴匀速转动,此种情况即为第\ref{chapter6:subsection-Euler情形下的永久转动}节讨论过的永久转动。

\item 当$L^2=2I_2T$时,式\eqref{非对称Euler陀螺的运动方程}变为
\begin{equation}
	\frac{\mathrm{d}\omega_2}{\mathrm{d}t} = \sqrt{\left(\frac{I_2}{I_1}-1\right)\left(1-\frac{I_2}{I_3}\right)}\left(\frac{2T}{I_2}-\omega_2^2\right)
	\label{Euler陀螺的稳定性-5}
\end{equation}
记$\varOmega = \sqrt{\dfrac{2T}{I_2}}$,作变量代换
\begin{equation}
	\tau = \varOmega\sqrt{\left(\frac{I_2}{I_1}-1\right)\left(1-\frac{I_2}{I_3}\right)} t,\quad s = \frac{\omega_2}{\varOmega}
	\label{Euler陀螺的稳定性-6}
\end{equation}
可将方程\eqref{Euler陀螺的稳定性-5}化为
\begin{equation*}
	\frac{\mathrm{d}s}{\mathrm{d}\tau} = 1-s^2
\end{equation*}
这个方程的解为
\begin{equation}
	s = \tanh \tau
\end{equation}
由此可得
\begin{equation}
\begin{cases}
	\displaystyle \omega_1 = \sqrt{\frac{I_2(I_3-I_2)}{I_1(I_3-I_1)}}\varOmega \frac{1}{\cosh \tau} \\
	\displaystyle \omega_2 = \varOmega \tanh \tau \\
	\displaystyle \omega_3 = \sqrt{\frac{I_2(I_2-I_1)}{I_3(I_3-I_1)}}\varOmega \frac{1}{\cosh \tau}
	\label{Euler陀螺的稳定性-7}
\end{cases}
\end{equation}
其中$\displaystyle \varOmega = \sqrt{\frac{2T}{I_2}},\tau = \varOmega\sqrt{\left(\frac{I_2}{I_1}-1\right)\left(1-\frac{I_2}{I_3}\right)} t$。如果我们令$\mbf{L}$的方向为$Z$轴,$\theta$为$Z$轴与$x_2$轴之间的夹角,则式\eqref{非对称Euler陀螺的Euler角-2}和式\eqref{非对称Euler陀螺的Euler角-3}中只要对下标做置换$123\to 312$即有
\begin{equation}
\begin{cases}
	\displaystyle \cos\theta = \frac{I_2\omega_2}{L} \\[1.5ex]
	\displaystyle \tan\psi = \frac{I_3\omega_3}{I_1\omega_1} \\[1.5ex]
	\displaystyle \dot{\phi} = L\frac{I_1\omega_1^2+I_3\omega_3^2}{I_1^2\omega_1^2+I_3^2\omega_3^2}
\end{cases}
\label{非对称Euler陀螺自由转动特殊情形-1}
\end{equation}
将式\eqref{Euler陀螺的稳定性-7}带入即有
\begin{equation}
\begin{cases}
	\displaystyle \cos\theta = \tanh \tau \\
	\displaystyle \tan\psi = \sqrt{\frac{I_3(I_2-I_1)}{I_1(I_3-I_2)}} \\
	\displaystyle \phi = \varOmega t+\phi_0
\end{cases}
\end{equation}
由上面各式可以看出,当$t\to \infty$时,矢量$\mbf{\omega}$渐近地趋于$x_2$轴,同时$x_2$轴也渐近地趋于固定轴$Z$。但是对于初始角速度沿$x_2$轴的转动\footnote{相当于将前面所述之运动从无穷大的时间开始进行反演。}也属于第\ref{chapter6:subsection-Euler情形下的永久转动}节所述的永久转动,但是这个永久转动是不稳定的。
\end{enumerate}

\subsubsection{Euler情形下刚体在空间中方向的确定}\label{chapter6:subsubsection-Euler情形下刚体在空间中方向的确定}

下面考虑Euler情形下,刚体在空间系中的绝对运动,即相对于固定坐标系$OXYZ$的运动。令$Z$轴沿着矢量$\mbf{L}$的方向,由于方向$Z$相对$x_1,x_2,x_3$轴的极角和方位角分别为$\theta$和$\dfrac{\pi}{2}-\psi$\label{chapter6:footnote-Euler角和球坐标系之间的关系}\footnote{可见图\ref{刚体定点转动的Euler角}。此处“极角”和“方位角”即为球坐标系
\begin{equation*}
\begin{cases}
	x = r\sin\theta\cos\phi\\
	y = r\sin\theta\sin\phi\\
	z = r\cos\theta
\end{cases}
\end{equation*}
中的$\theta$和$\phi$。},则矢量$\mbf{L}$沿$x_1,x_2,x_3$的分量分别为
\begin{subnumcases}{\label{非对称Euler陀螺的Euler角-1}}
	I_1\omega_1 = L\sin\theta\sin\psi \\
	I_2\omega_2 = L\sin\theta\cos\psi \\
	I_3\omega_3 = L\cos\theta
\end{subnumcases}
由此可得
\begin{equation}
	\cos \theta = \frac{I_3\omega_3}{L},\quad \tan\psi = \frac{I_1\omega_1}{I_2\omega_2}
	\label{非对称Euler陀螺的Euler角-2}
\end{equation}
将式\eqref{非对称Euler陀螺的运动方程的解2}的结果代入,即有
\begin{equation}
\begin{cases}
	\displaystyle \cos \theta = \sqrt{\frac{I_3(L^2-2I_1T)}{L^2(I_3-I_1)}}\dn \tau \\
	\displaystyle \tan \psi = \sqrt{\frac{I_1(I_3-I_2)}{I_2(I_3-I_1)}}\frac{\cn \tau}{\sn \tau}
\end{cases}
\end{equation}
由此获得角$\theta$和$\psi$随时间的关系,它们都是周期为$T_0$的函数,其中$T_0$由式\eqref{chapter6:非对称Euler陀螺的运动方程解的周期}确定。为了获得角$\phi$随时间的关系,根据Euler运动学方程\eqref{Euler运动学方程}可有
\begin{equation*}
\begin{cases}
	\omega_1 = \dot{\phi}\sin \theta\sin \psi+\dot{\theta}\cos \psi \\
	\omega_2 = \dot{\phi}\sin \theta\cos \psi-\dot{\theta}\sin \psi
\end{cases}
\end{equation*}
从中解得
\begin{equation*}
	\dot{\phi} = \frac{\omega_1\sin\psi+\omega_2\cos\psi}{\sin \theta}
\end{equation*}
将式\eqref{非对称Euler陀螺的Euler角-1}代入,即有
\begin{equation}
	\frac{\mathrm{d}\phi}{\mathrm{d}t} = L\frac{I_1\omega_1^2+I_2\omega_2^2}{L^2-I_3^2\omega_3^2} = L\frac{I_1\omega_1^2+I_2\omega_2^2}{I_1^2\omega_1^2+I_2^2\omega_2^2}
	\label{非对称Euler陀螺的Euler角-3}
\end{equation}
由此便可确定$\phi$与时间的关系$\phi(t)$。注意到由于$\dfrac{\mathrm{d}\phi}{\mathrm{d}t} > 0$,因此不论在何种情况下,$\phi$角始终是单调递增的,即刚体的进动永远都向着同一个方向。

由于$\omega_1,\omega_2,\omega_3$都是周期为$T_0$的函数,因此$\dfrac{\mathrm{d}\phi}{\mathrm{d}t}$也是$T_0$的周期函数,即有
\begin{equation*}
	\dfrac{\mathrm{d}\phi}{\mathrm{d}t}(t+T_0)=\dfrac{\mathrm{d}\phi}{\mathrm{d}t}(t)
\end{equation*}
积分之后可得
\begin{equation}
	\phi(t+T_0)=\phi(t)+\Delta\phi
\end{equation}
其中$\Delta\phi$是积分常数。如果$\dfrac{\Delta\phi}{2\pi}$不是有理数,则刚体永远都并不会回到初始方向。而如果
\begin{equation*}
	\dfrac{\Delta\phi}{2\pi} = \frac mn
\end{equation*}
其中$m,n$是整数($n\neq 0$),则刚体的运动是周期性的,其周期等于$nT_0$。

%这个解可以用$\varTheta$-函数表示,而且是非周期的,因此,陀螺在经过$T_0$时间后,其角速度矢量$\mbf{\omega}$相对本体系回到原位,但陀螺本身在空间系中却不能回到它的初始位置。

\subsubsection{Euler情形下刚体永久转动的稳定性}\label{chapter6:subsubsection-Euler情形下刚体永久转动的稳定性}

当刚体的角动量$\mbf{L}$与$x_1$轴有微小偏离时\footnote{此时即$\sqrt{2I_3T-L^2}$为一阶小量。},那么$L_2,L_3$是一阶小量。根据Euler动力学方程\eqref{chapter6:刚体定点运动的Euler情形下的运动方程}可得
\begin{subnumcases}{}
	I_1\dot{\omega}_1-(I_2-I_3)\omega_2\omega_3 = 0 \label{Euler陀螺的稳定性-1.1} \\
	I_2\dot{\omega}_2-(I_3-I_1)\omega_1\omega_3 = 0 \label{Euler陀螺的稳定性-1.2} \\
	I_3\dot{\omega}_3-(I_1-I_2)\omega_1\omega_2 = 0 \label{Euler陀螺的稳定性-1.3}
\end{subnumcases}
由于$L_2,L_3$是一阶小量,故$\omega_2,\omega_3$也是一阶小量,故由方程\eqref{Euler陀螺的稳定性-1.1}可得
\begin{equation*}
	I_1\dot{\omega}_1 = 0
\end{equation*}
即有
\begin{equation*}
	\omega_1 = \varOmega \quad \text{(常数)}
\end{equation*}
而式\eqref{Euler陀螺的稳定性-1.2}和\eqref{Euler陀螺的稳定性-1.3}则可写作
\begin{equation*}
\begin{cases}
	\dot{\omega}_2 = \dfrac{I_3-I_1}{I_2}\varOmega\omega_3 \\
	\dot{\omega}_3 = \dfrac{I_1-I_2}{I_3}\varOmega\omega_2
\end{cases}
\end{equation*}
由此可以解得
\begin{equation*}
\begin{cases}
	\displaystyle \omega_2 = \frac{La}{I_2} \sqrt{1-\frac{I_1}{I_2}} \sin (\omega_0t+\phi_0) \\
	\displaystyle \omega_3 = \frac{La}{I_2} \sqrt{1-\frac{I_1}{I_3}} \cos (\omega_0t+\phi_0)
\end{cases}
\end{equation*}
其中$a$是一个小常数(一阶小量),$\omega_0 = \sqrt{\left(1-\dfrac{I_1}{I_2}\right)\left(1-\dfrac{I_1}{I_3}\right)}\varOmega$。由于
\begin{equation*}
	\mbf{L} = I_1\omega_1\mbf{e}_1+I_2\omega_2\mbf{e}_2+I_3\omega_3\mbf{e}_3
\end{equation*}
则有
\begin{equation}
\begin{cases}
	\displaystyle L_1 = I_1\varOmega \approx L \\
	\displaystyle L_2 = La\sqrt{1-\frac{I_1}{I_2}} \sin (\omega_0t+\phi_0) \\
	\displaystyle L_3 = La\sqrt{1-\frac{I_1}{I_3}} \cos (\omega_0t+\phi_0)
\end{cases}
\label{Euler陀螺的稳定性-2}
\end{equation}
式\eqref{Euler陀螺的稳定性-2}表明,矢量$\mbf{L}$的端点,以圆频率$\omega_0$绕着$x_1$轴上的极点画出小椭圆(即图\ref{不同情况下角动量矢量末端的轨迹}中的红色线)。

对于刚体的角动量$\mbf{L}$与$x_3$轴有微小偏离的情形,根据类似的过程可以得到
\begin{equation}
\begin{cases}
	\displaystyle \omega_1 = \frac{La}{I_1}\sqrt{\frac{I_3}{I_2}-1} \sin (\omega_0t+\phi_0) \\
	\displaystyle \omega_2 = \frac{La}{I_2}\sqrt{\frac{I_3}{I_1}-1} \sin (\omega_0t+\phi_0) \\
	\displaystyle \omega_3 = \varOmega
\end{cases}
\label{Euler陀螺的稳定性-3}
\end{equation}
其中$a$是一个小常数(一阶小量),$\omega_0 = \sqrt{\left(\dfrac{I_3}{I_1}-1\right)\left(\dfrac{I_3}{I_2}-1\right)}\varOmega$,对角动量则有
\begin{equation}
\begin{cases}
	\displaystyle L_1 = La\sqrt{\frac{I_3}{I_2}-1} \sin (\omega_0t+\phi_0) \\
	\displaystyle L_2 = La\sqrt{\frac{I_3}{I_1}-1} \sin (\omega_0t+\phi_0) \\
	\displaystyle L_3 = I_3\varOmega \approx L
\end{cases}
\label{Euler陀螺的稳定性-4}
\end{equation}
式\eqref{Euler陀螺的稳定性-4}表明,矢量$\mbf{L}$的端点,以圆频率$\omega_0$绕着$x_3$轴上的极点画出小椭圆(即图\ref{不同情况下角动量矢量末端的轨迹}中的绿色线)。

以上的讨论表明,Euler情形下,刚体绕其主惯量最大或最小的轴的永久转动是稳定的。

而对于刚体的角动量$\mbf{L}$与$x_2$轴有微小偏离的情形,精确到一阶小量可得
\begin{subnumcases}{}
	\dot{\omega}_1 = \frac{I_2-I_3}{I_1}\varOmega\omega_3 \\
	\dot{\omega}_3 = \frac{I_1-I_2}{I_3}\varOmega\omega_1
\end{subnumcases}
从中消去其中$\omega_1$或$\omega_3$可得
\begin{subnumcases}{}
	\ddot{\omega}_1 - \frac{(I_2-I_1)(I_3-I_2)}{I_1I_3}\varOmega^2\omega_1 = 0 \\
	\ddot{\omega}_3 - \frac{(I_2-I_1)(I_3-I_2)}{I_1I_3}\varOmega^2\omega_3 = 0
\end{subnumcases}
由于$I_1<I_2<I_3$,故有
\begin{equation*}
	- \frac{(I_2-I_1)(I_3-I_2)}{I_1I_3}\varOmega^2 < 0
\end{equation*}
因此,Euler情形下,刚体绕其主惯量取中间值的轴的永久转动是不稳定的。

\subsection[Poinsot几何图像]{Poinsot\footnote{Loius Poinsot,潘索,法国数学家、力学家。}几何图像}

\begin{figure}[htb]
\centering
\begin{asy}
	size(250);
	//角速度的方向与角动量的方向
	pair O,a,b,L,omega;
	real rx,ry,theta,lL,lomega;
	path rect,cir;
	picture tmp1,tmp2;
	O = (0,0);
	a = (2,0);
	b = (0.7,0.8);
	L = (0,-0.9);
	omega = L+0.7*dir(190);
	lL = 2.2;
	lomega = lL/cos((degrees(omega)-degrees(L))*pi/180);
	rx = 1.5;
	ry = 1;
	theta = 50;
	rect = L+a+b--L+a-b--L-a-b--L-a+b--cycle;
	cir = rotate(theta)*xscale(rx)*yscale(ry)*unitcircle;
	dot(O);
	label("$O$",O,N);
	draw(cir);
	draw(tmp1,rect);
	unfill(tmp1,cir);
	add(tmp1);
	erase(tmp1);
	draw(tmp1,O--lL*L,blue+dashed,Arrow);
	draw(tmp1,O--lomega*omega,red+dashed,Arrow);
	add(tmp2,tmp1);
	clip(tmp1,rect);
	add(tmp1);
	clip(tmp2,cir);
	unfill(tmp2,rect);
	add(tmp2);
	erase(tmp1);
	erase(tmp2);
	draw(tmp1,O--lL*dir(L),blue,Arrow);
	draw(tmp1,Label("$\boldsymbol{\omega}$",EndPoint,S,black),O--lomega*dir(omega),red,Arrow);
	label(tmp1,"$\boldsymbol{L}$(常矢量)",lL*dir(L)+(0.6,0),S);
	unfill(tmp1,rect);
	unfill(tmp1,cir);
	add(tmp1);
	erase(tmp1);
	dot(omega);
	dot(L);
	label("$N$",omega,N);
	label("$L$",L,NE);
	draw(L--omega,dashed);
\end{asy}
\caption{Poinsot几何图像}
\label{Poinsot几何图像示意}
\end{figure}

根据惯量椭球的几何意义,角动量$\mbf{L}$沿角速度$\mbf{\omega}$与惯量椭球交点处切平面的法线方向。对于Euler情形,刚体的角动量守恒,即$\mbf{L}$为常矢量,因此在Euler情形中,切平面在各个时刻是相互平行的。再考虑到惯量椭球中心到切平面的距离为
\begin{equation*}
	x_N = \mbf{x}_N \cdot \frac{\mbf{L}}{L} = \frac{\mbf{\omega}}{\sqrt{2T}} \cdot \frac{\mbf{L}}{L} = \frac{\sqrt{2T}}{L}\quad \text{(常数)}
\end{equation*}
因此,在Euler情形下,惯量椭球极点处的切平面固定,称为{\heiti Poinsot平面}。

于是得到Euler情形下刚体运动的Poinsot几何解释:惯量椭球随着刚体的运动在Poinsot平面上纯滚动,极点为瞬时接触点,其瞬时速度为零\footnote{由于极点在角速度矢量$\mbf{\omega}$方向上,故极点速度为零。}。

在刚体运动时,极点在惯性椭球上画出的曲线称为{\heiti 本体极迹},相应地极点在Poinsot平面上画出的曲线称为{\heiti 空间极迹}。显然,本体极迹是刚体定点运动中本体极锥的准线,而空间极迹是刚体定点运动的空间极锥的准线。

\section{重刚体的定点运动}

\subsection{重刚体定点运动方程及其守恒量}\label{chapter6:subsection-重刚体定点运动方程及其守恒量}

下面研究刚体在重力场中绕固定点$O$的运动。空间系(此时与平动系重合)的$OZ$轴竖直向上,本体系记作$Oxyz$,其坐标轴沿着刚体对$O$点的三个惯性主轴。刚体质心$C$的坐标在本体系中记作$x_C,y_C,z_C$,而刚体相对空间系的方向则借助Euler角$\phi,\theta,\psi$来确定(如图\ref{chapter6:figure-重刚体的定点运动}所示)。

\begin{figure}[htb]
\centering
\begin{asy}
	size(300);
	//重刚体的定点运动
	real covert(pair x,real a){
		if (x.x==0) return 90;
		else return atan(a*x.y/x.x)*180/pi;
	}
	picture elpic1,tmp;
	pair O1,O,x[],y[],z[];
	real a,alpha1,alpha2,beta,gamma;
	a = 2;
	O1 = (0,0);
	O = (0,0);
	x[1] = (-a/sqrt(5),-a/sqrt(5));
	y[1] = (a,0);
	z[1] = (0,a);
	//第一层
	draw(elpic1,Label("$X$",EndPoint),O--x[1],blue,Arrow);
	draw(elpic1,Label("$Y$",EndPoint),O--y[1],blue,Arrow);
	draw(elpic1,Label("$Z$",EndPoint),O--1.2*z[1],blue,Arrow);
	label(elpic1,"$O$",O,W);
	path cir[],clp;
	cir[1] = yscale(1/2)*scale(a)*unitcircle;
	draw(elpic1,cir[1],blue);
	alpha1 = 70;
	alpha2 = 10;
	real k;
	k = tan(-alpha1*pi/180);
	x[2] = (a/sqrt(1+4*k*k),a*k/sqrt(1+4*k*k));
	k = tan(alpha2*pi/180);
	y[2] = (a/sqrt(1+4*k*k),a*k/sqrt(1+4*k*k));
	z[2] = z[1];
	draw(elpic1,x[2]--O,dashed);
	draw(elpic1,Label("$N$",EndPoint),O--x[2],Arrow);
	draw(elpic1,Label("$L$",EndPoint),O--y[2],Arrow);
	//第二层
	picture elpic2;
	beta = 35;
	x[3] = x[2];
	y[3] = rotate(beta)*y[1];
	z[3] = rotate(0.9*beta)*z[1];
	draw(elpic2,Label("$M$",EndPoint),O--y[3],Arrow);
	draw(elpic2,Label("$z$",EndPoint),O--1.2*z[3],red,Arrow);
	real k1,k2,ra,b;
	k1 = -tan(alpha1*pi/180);
	k2 = -tan((alpha1+beta)*pi/180);
	ra = sqrt((1+4*k1^2)*(1+k2^2)/(1+k1^2)-1);
	b = k2*a/ra;
	cir[2] = rotate(beta)*yscale(b)*xscale(a)*unitcircle;
	clp = 1.1*x[3]--1.1*x[3]+1.1*y[3]--(-1.1*x[3]+1.1*y[3])--(-1.1*x[3])--cycle;
	draw(tmp,cir[2],red);
	clip(tmp,clp);
	add(elpic2,tmp);
	erase(tmp);
	draw(tmp,cir[2],red+dashed);
	unfill(tmp,clp);
	add(elpic2,tmp);
	erase(tmp);
	//第一层被第二层遮住的部分
	unfill(elpic1,cir[2]);
	draw(tmp,Label("$X$",EndPoint),O--x[1],blue+dashed,Arrow);
	draw(tmp,Label("$Y$",EndPoint),O--y[1],blue+dashed,Arrow);
	draw(tmp,Label("$Z$",EndPoint),O--z[1],blue,Arrow);
	draw(tmp,-x[2]--O,dashed);
	draw(tmp,Label("$N$",EndPoint),O--x[2],Arrow);
	draw(tmp,Label("$L$",EndPoint),O--y[2],dashed,Arrow);
	label(tmp,"$O$",O,W);
	draw(tmp,cir[1],blue+dashed);
	clip(tmp,cir[2]);
	clip(tmp,clp);
	add(elpic1,tmp);
	erase(tmp);
	draw(tmp,Label("$X$",EndPoint),O--x[1],blue,Arrow);
	draw(tmp,Label("$Y$",EndPoint),O--y[1],blue,Arrow);
	draw(tmp,Label("$Z$",EndPoint),O--z[1],blue,Arrow);
	draw(tmp,-x[2]--O,dashed);
	draw(tmp,Label("$N$",EndPoint),O--x[2],Arrow);
	draw(tmp,Label("$L$",EndPoint),O--y[2],dashed,Arrow);
	draw(tmp,Label("$Z$",EndPoint),O--z[2],dashed,Arrow);
	label(tmp,"$O$",O,W);
	draw(tmp,cir[1],blue);
	clip(tmp,cir[2]);
	unfill(tmp,clp);
	add(elpic1,tmp);
	erase(tmp);
	//画自转
	x[4] = relpoint(cir[2],0.78);
	y[4] = relpoint(cir[2],0.05);
	z[4] = z[3];
	draw(elpic2,Label("$x$",EndPoint),O--x[4],red,Arrow);
	draw(elpic2,Label("$y$",EndPoint),O--y[4],red,Arrow);
	add(shift(O1)*elpic1);
	add(shift(O1)*elpic2);
	picture elpic4;
	//画第一层上的角度
	real theta1,theta2,r;
	r = 0.4;
	theta1 = covert(x[1],2)+180;
	theta2 = covert(x[2],2)+360;
	draw(elpic4,Label("$\phi$",MidPoint,Relative(E)),yscale(1/2)*arc(O,r,theta1,theta2),Arrow);
	theta1 = covert(y[1],2);
	theta2 = covert(y[2],2);
	r = 0.7;
	draw(elpic4,Label("$\phi$",MidPoint,E),yscale(1/2)*arc(O,r,theta1,theta2),Arrow);
	//画第二层上的角度
	r = 1.3;
	theta1 = covert(y[2],0.9);
	theta2 = covert(y[3],0.9);
	draw(elpic4,Label("$\theta$",MidPoint,Relative(E)),xscale(0.9)*arc(O,r,theta1,theta2),Arrow);
	theta1 = covert(z[2],0.9);
	theta2 = covert(z[3],0.9);
	draw(elpic4,Label("$\theta$",MidPoint,Relative(E)),xscale(0.9)*arc(O,r,theta1,180+theta2),Arrow);
	//画自转角
	r = 0.3;
	//theta1 = covert(x[2],a/b);
	//theta2 = covert(x[4],a/b);
	draw(elpic4,Label("$\psi$",MidPoint,Relative(E)),arc(O,r,degrees(x[3]),degrees(x[4])),Arrow);
	r = 0.4;
	draw(elpic4,Label("$\psi$",MidPoint,Relative(E)),arc(O,r,degrees(y[3]),degrees(y[4])),Arrow);
	add(shift(O1)*elpic4);
	real f(real x){
		return x**2;
	}
	draw(tmp,graph(f,-1.5,1.5),linewidth(1bp));
	draw(tmp,shift((0,f(1.5)))*yscale(1/3)*scale(1.5)*unitcircle,linewidth(1bp));
	add(rotate(35)*xscale(0.8)*scale(.85)*shift((0,-0.12))*tmp);
	pair C = 1.15*dir(90+53);
	dot("$C$",C,N);
	real G = 1;
	draw(Label("$M\boldsymbol{g}$",EndPoint),C--C+G*dir(-90),Arrow);
	pair P = yscale(1/2)*(2*dir(-40));
	real n = 0.35;
	draw(Label("$\boldsymbol{n}$",EndPoint),P--P+n*dir(90),Arrow);
\end{asy}
\caption{重刚体的定点运动}
\label{chapter6:figure-重刚体的定点运动}
\end{figure}

由于一般情形下,重刚体定点运动的Lagrange函数形式比较复杂,因此本节将直接利用主轴系中的Euler动力学方程\eqref{Euler动力学方程}来得到重刚体的定点运动方程,并通过一些物理定律来得到运动过程中的守恒量。

设竖直轴$OZ$的单位矢量$\mbf{n}$在本体系$Oxyz$中的分量分别为$\gamma_1,\gamma_2,\gamma_3$,则矢量$\mbf{n}$在本体系$Oxyz$中的极角和方位角分别为$\theta$和$\dfrac\pi2-\psi$\footnote{见第\pageref{chapter6:footnote-Euler角和球坐标系之间的关系}页脚注。},即有
\begin{equation}
	\gamma_1=\sin\theta\sin\psi,\quad \gamma_2=\sin\theta\cos\psi,\quad \gamma_3=\cos\theta
	\label{chapter6:重刚体定点运动中Z方向固定矢量分量与Euler角的关系}
\end{equation}
矢量$\mbf{n}$在空间系中是常矢量,即有$\dfrac{\mathd \mbf{n}}{\mathd t}=\mbf{0}$,根据式\eqref{chapter6:相对变化率和绝对变化率之间的关系}的关系将其投影至本体系,可有
\begin{equation}
	\frac{\tilde{\mathd} \mbf{n}}{\mathd t}+\mbf{\omega}\times \mbf{n}=\mbf{0}
	\label{chapter6:重刚体定点运动的Possion方程}
\end{equation}
其中$\mbf{\omega}$是刚体的角速度,方程\eqref{chapter6:重刚体定点运动的Possion方程}称为{\bf Possion方程}。如果用$\omega_1,\omega_2,\omega_3$来表示角速度矢量在本体系中的分量,则Possion方程\eqref{chapter6:重刚体定点运动的Possion方程}可以写作如下三个分量方程:
\begin{equation}
\begin{cases}
	\dfrac{\mathd \gamma_1}{\mathd t}=\gamma_2\omega_3-\gamma_3\omega_2 \\[1.5ex]
	\dfrac{\mathd \gamma_2}{\mathd t}=\gamma_3\omega_1-\gamma_1\omega_3 \\[1.5ex]
	\dfrac{\mathd \gamma_3}{\mathd t}=\gamma_1\omega_2-\gamma_2\omega_1
\end{cases}
\label{chapter6:重刚体定点运动的Possion方程的分量形式}
\end{equation}

作用在刚体上的外力为重力和$O$点的约束反力,其中约束反力对$O$点没有力矩,所以刚体的合外力矩就等于重力矩,即
\begin{equation}
	\mbf{M} = \vec{OC}\times (-Mg\mbf{n}) = Mg\mbf{n}\times \vec{OC}
	\label{chapter6:重刚体定点运动的外力矩}
\end{equation}
此处非黑体的$M$表示刚体的质量,该外力矩的分量形式为
\begin{equation}
	M_1 = Mg(\gamma_2z_C-\gamma_3y_C),\quad M_2 = Mg(\gamma_3x_C-\gamma_1z_C),\quad M_3 = Mg(\gamma_1y_C-\gamma_2x_C)
	\label{chapter6:重刚体定点运动的外力矩-分量形式}
\end{equation}
于是,此种情况下,Euler动力学方程\eqref{Euler动力学方程}具有如下形式
\begin{equation}
\begin{cases}
	I_1 \dfrac{\mathd \omega_1}{\mathd t} - (I_2-I_3)\omega_2 \omega_3 = Mg(\gamma_2z_C-\gamma_3y_C) \\[1.5ex]
	I_2 \dfrac{\mathd \omega_2}{\mathd t} - (I_3-I_1)\omega_3 \omega_1 = Mg(\gamma_3x_C-\gamma_1z_C) \\[1.5ex]
	I_3 \dfrac{\mathd \omega_3}{\mathd t} - (I_1-I_2)\omega_1 \omega_2 = Mg(\gamma_1y_C-\gamma_2x_C)
\end{cases}
\label{chapter6:重刚体定点运动的微分方程}
\end{equation}
方程组\eqref{chapter6:重刚体定点运动的Possion方程的分量形式}和\eqref{chapter6:重刚体定点运动的微分方程}构成了封闭方程组,包含了描述重刚体定点运动的全部6个微分方程。

如果从方程组\eqref{chapter6:重刚体定点运动的Possion方程的分量形式}和\eqref{chapter6:重刚体定点运动的微分方程}中求出了$\omega_1,\omega_2,\omega_3,\gamma_1,\gamma_2,\gamma_3$作为时间的函数,则根据关系\eqref{chapter6:重刚体定点运动中Z方向固定矢量分量与Euler角的关系}可以确定Euler角$\theta(t)$和$\psi(t)$,而确定$\phi(t)$还需要利用Euler运动学方程\eqref{Euler运动学方程}中的一个。

下面将给出方程组\eqref{chapter6:重刚体定点运动的Possion方程的分量形式}和\eqref{chapter6:重刚体定点运动的微分方程}的三个守恒量(运动积分)。第一个是向量$\mbf{n}$的长度等于$1$,即
\begin{equation}
	\gamma_1^2+\gamma_2^2+\gamma_3^2=1
	\label{chapter6:重刚体定点运动的运动积分-1}
\end{equation}

还有一个守恒量可以通过角动量定理得到,由于外力矩对$OZ$轴的分量为零,即$\mbf{M}\cdot\mbf{n}=0$,所以有
\begin{equation*}
	\mbf{L}\cdot \mbf{n}=L_Z\quad \text{(常数)}
\end{equation*}
在本体系中上式可用分量表示为
\begin{equation}
	I_1\omega_1\gamma_1+I_2\omega_2\gamma_2+I_3\omega_3\gamma_3=L_Z\quad \text{(常数)}
	\label{chapter6:重刚体定点运动的运动积分-2}
\end{equation}

再考虑到$O$点的约束反力不做功,而重力有势且不显含时间,因此运动过程中机械能守恒,即
\begin{equation}
	\frac12 (I_1\omega_1^2+I_2\omega_2^2+I_3\omega_3^2)+Mg(x_C\gamma_1+y_C\gamma_2+z_C\gamma_3) = E\quad \text{(常数)}
	\label{chapter6:重刚体定点运动的运动积分-3}
\end{equation}

如果想要在任何初始条件下将方程组\eqref{chapter6:重刚体定点运动的Possion方程的分量形式}和\eqref{chapter6:重刚体定点运动的微分方程}积分,除了上面的三个运动积分\eqref{chapter6:重刚体定点运动的运动积分-1}、\eqref{chapter6:重刚体定点运动的运动积分-2}和\eqref{chapter6:重刚体定点运动的运动积分-3}之外,还需要一个与它们相互独立的守恒量。

事实上,在任何初始条件下,对于$\omega_1,\omega_2,\omega_3,\gamma_1,\gamma_2,\gamma_3$的第四个守恒量只有在下面三种情况下存在,分别就是Euler情形、Lagrange情形和Kovalevskaya\footnote{索菲娅·柯瓦列夫斯卡娅,俄国数学家。}情形。

\begin{enumerate}
\item Euler情形:在Euler情形下,刚体可以是任意的,但其重心与固定点$O$重合,即$x_C=y_C=z_C=0$,这种情形已经在第\ref{chapter6:section-刚体定点运动的Euler情形}节做了详细讨论。

\item Lagrange情形:在Lagrange情形下,刚体对固定点$O$是动力学对称的,而重心位于对称轴上,例如$I_1=I_2, x_C=y_C=0$。由\eqref{chapter6:重刚体定点运动的微分方程}的最后一个方程可知,刚体角动量在本体系$Oz$轴上的投影即为第四个守恒量
\begin{equation}
	I_3\omega_3=L_z\quad \text{(常数)}
\end{equation}
这种情形将在第\ref{chapter6:section-刚体定点运动的Lagrange情形}节中进行详细讨论。

\item Kovalevskaya情形:在Kovalevskaya情形下,刚体对固定点$O$是动力学对称的,且其主惯量满足关系式$I_1=I_2=2I_3$,而重心位于其惯性椭球的赤道面上,即$z_C=0$。对于动力学对称刚体,通过$O$点且与对称轴垂直的任何方向都是其惯性主轴,因此简便起见,不妨设重心在$Ox$轴上,即$y_C=0$。在这种情况下Euler动力学方程\eqref{chapter6:重刚体定点运动的微分方程}可以写为
\begin{equation}
\begin{cases}
	2\dfrac{\mathd \omega_1}{\mathd t} - \omega_2 \omega_3 = 0 \\[1.5ex]
	2\dfrac{\mathd \omega_2}{\mathd t} + \omega_3 \omega_1 = \alpha\gamma_3 \\[1.5ex]
	\dfrac{\mathd \omega_3}{\mathd t} = -\alpha\gamma_2
\end{cases}\left(\alpha=\frac{Mgx_C}{I_3}\right)
\label{chapter6:重刚体定点运动的微分方程-Kovalevskaya情形}
\end{equation}
通过方程组\eqref{chapter6:重刚体定点运动的Possion方程的分量形式}和\eqref{chapter6:重刚体定点运动的微分方程-Kovalevskaya情形}不难直接验证,Kovalevaskaya情形下的第四个守恒量为
\begin{equation}
	(\omega_1^2-\omega_2^2-\alpha\gamma_1)^2+(2\omega_1\omega_2-\alpha\gamma_2)^2=h\quad \text{(常数)}
	\label{chapter6:Kovalevskaya情形的第四个守恒量}
\end{equation}
\end{enumerate}

\subsection{陀螺}\label{chapter6:subsection-陀螺}

%\subsubsection{陀螺基本公式}

动力学对称的定点运动刚体一般称为{\bf 陀螺}\footnote{有时也将第\ref{chapter6:subsection-重刚体定点运动方程及其守恒量}节所述重刚体定点运动的三种情形分别称为Euler陀螺、Lagrange陀螺和Kovalevskaya陀螺。}。在第\ref{chapter6:subsection-Euler情形下动力学对称刚体的运动}节中已经详细讨论过,如果对固定点$O$的合外力矩为零,则陀螺绕不变的角动量矢量$\mbf{L}$作规则进动。

但是为了使陀螺作规则进动,并不一定要对固定点的合外力矩为零。设陀螺的主惯量满足$I_1=I_2$,则此时Euler动力学方程\eqref{Euler动力学方程}可以写为
\begin{equation}
\begin{cases}
	I_1\dfrac{\mathd \omega_1}{\mathd t}+(I_3-I_1)\omega_2\omega_3=M_1 \\[1.5ex]
	I_1\dfrac{\mathd \omega_2}{\mathd t}-(I_3-I_1)\omega_3\omega_1=M_2 \\[1.5ex]
	I_3\dfrac{\mathd \omega_3}{\mathd t}=M_z
\end{cases}
\label{chapter6:一般情况下陀螺的Euler运动学方程}
\end{equation}

设陀螺绕空间系$OZ$轴规则进动,则章动角保持常数$\theta=\theta_0$,进动角速度$\dot{\phi}=\omega_p$和自转角速度$\dot{\psi}=\omega_s$都是常数,根据Euler运动学方程\eqref{Euler运动学方程}可得
\begin{equation}
\begin{cases}
	\omega_1=\omega_p\sin\theta_0\sin\psi \\
	\omega_2=\omega_p\sin\theta_0\cos\psi \\
	\omega_3=\omega_p\cos\theta_0+\omega_s
\end{cases}
\label{chapter6:规则进动时,陀螺的角速度在本体系中的分量}
\end{equation}
由式\eqref{chapter6:规则进动时,陀螺的角速度在本体系中的分量}的最后一式可知,$\omega_3$为常数,因此根据\eqref{chapter6:一般情况下陀螺的Euler运动学方程}的第三个方程可得
\begin{equation}
	M_3=0
	\label{chapter6:规则进动时,陀螺所受合外力矩的z分量}
\end{equation}
将式\eqref{chapter6:规则进动时,陀螺的角速度在本体系中的分量}代入到\eqref{chapter6:一般情况下陀螺的Euler运动学方程}的前两式中,并注意到$\dot{\psi}=\omega_s$,可得
\begin{equation}
\begin{cases}
	M_1=\omega_p\omega_s\sin\theta_0\cos\psi\left[I_3+(I_3-I_1)\dfrac{\omega_p}{\omega_s}\cos\theta_0\right] \\[1.5ex]
	M_2=-\omega_p\omega_s\sin\theta_0\sin\psi\left[I_3+(I_3-I_1)\dfrac{\omega_p}{\omega_s}\cos\theta_0\right]
\end{cases}
\label{chapter6:规则进动时,陀螺所受合外力矩的x和y分量}
\end{equation}

由于在坐标系$Oxyz$中有
\begin{equation}
	\mbf{\omega}_s = \omega_s\mbf{e}_3,\quad \mbf{\omega}_p = \omega_p(\sin\theta_0\sin\psi\mbf{e}_1+\sin\theta_0\cos\psi\mbf{e}_2+\cos\theta_0\mbf{e}_3)
\end{equation}
因此可以将分量形式\eqref{chapter6:规则进动时,陀螺所受合外力矩的x和y分量}和\eqref{chapter6:规则进动时,陀螺所受合外力矩的z分量}表示为一个矢量等式
\begin{equation}
	\mbf{M} = \mbf{\omega}_p\times\mbf{\omega}_s\left[I_3+(I_3-I_1)\dfrac{\omega_p}{\omega_s}\cos\theta_0\right]
	\label{chapter6:陀螺基本公式}
\end{equation}
由此可知,力矩$\mbf{M}$的大小为常数,方向沿着节线$ON$(见图\ref{chapter6:figure-重刚体的定点运动})。式\eqref{chapter6:陀螺基本公式}称为{\bf 陀螺基本公式},在已知主惯量$I_1,I_3$、章动角$\theta_0$、进动角速度$\mbf{\omega}_p$和自转角速度$\mbf{\omega}_s$的情况下,陀螺基本公式\eqref{chapter6:陀螺基本公式}可以给出该规则进动所需的合外力矩$\mbf{M}$。

这里的规则进动与第\ref{chapter6:subsection-Euler情形下动力学对称刚体的运动}节所述的Euler情形动力学对称刚体的规则进动不同,这里的角动量$\mbf{L}$不是常矢量,它满足
\begin{equation*}
	\frac{\mathd \mbf{L}}{\mathd t} = \mbf{M}
\end{equation*}

%\subsubsection{陀螺基本理论}

工程技术中使用的陀螺其自转角速度一般远大于其进动角速度,即$\omega_s\gg\omega_p$,因此可忽略陀螺基本公式\eqref{chapter6:陀螺基本公式}方括号中的二次项,即有
\begin{equation}
	\mbf{M} = I_3\mbf{\omega}_p\times \mbf{\omega}_s
	\label{chapter6:陀螺近似公式}
\end{equation}
式\eqref{chapter6:陀螺近似公式}是陀螺近似理论的基础,称为{\bf 陀螺近似公式}\footnote{如果章动角$\theta_0=\dfrac\pi2$,则公式\eqref{chapter6:陀螺近似公式}不再是近似的,而是精确的。}。对于高速自转的陀螺,其任意时刻的瞬时角速度和角动量都沿着动力学对称轴,而且满足\footnote{注意,这是近似的。}
\begin{equation*}
	\mbf{L}=I_3\mbf{\omega}_s
\end{equation*}

\section{刚体定点运动的Lagrange情形}\label{chapter6:section-刚体定点运动的Lagrange情形}

根据第\ref{chapter6:subsection-重刚体定点运动方程及其守恒量}节的相关论述,在Lagrange情形下,刚体应满足如下条件:
\begin{enumerate}
\item 对定点的两个主惯量相等,如$I_1=I_2$;
\item 质心在动力学对称轴上,且与定点不重合,即$l = OC \neq 0$;
\item 刚体仅受重力矩作用。
\end{enumerate}
这种定点转动刚体称为{\heiti 对称重陀螺},又称为{\heiti Lagrange陀螺}。

\begin{figure}[htb]
\centering
\begin{asy}
	size(200);
	//Lagrange陀螺的运动
	pair O,r,i,j,k,C;
	picture tmp;
	O = (0,0);
	i = 0.5*dir(-135);
	j = dir(0);
	k = dir(90);
	draw(tmp,Label("$X$",EndPoint,SE,black),O--i,blue,Arrow);
	draw(tmp,Label("$Y$",EndPoint,black),O--j,blue,Arrow);
	draw(tmp,Label("$Z$",EndPoint,black),O--1.6*k,blue,Arrow);
	label(tmp,"$O$",O,W);
	add(tmp);
	erase(tmp);
	real f(real x){
		return x**2;
	}
	draw(tmp,graph(f,-1.5,1.5),linewidth(1bp));
	draw(tmp,shift((0,f(1.5)))*yscale(1/3)*scale(1.5)*unitcircle,linewidth(1bp));
	add(rotate(25)*xscale(0.8)*scale(0.5)*shift((0,-0.15))*tmp);
	erase(tmp);
	draw(tmp,Label("$x$",EndPoint,S,black),rotate(25)*(O--i),red,Arrow);
	draw(tmp,Label("$y$",EndPoint,black),rotate(25)*(O--j),red,Arrow);
	draw(tmp,Label("$z$",EndPoint,W,black),rotate(25)*(O--1.6*k),red,Arrow);
	add(tmp);
	real r;
	r = 0.3;
	C = rotate(25)*(0.75*k);
	draw(Label("$\theta$",MidPoint,Relative(E)),arc(O,r,90,90+25),Arrow);
	dot(C);
	label("$C$",C,E);
	draw(Label("$Mg$",EndPoint,W),C--C+0.8*dir(-90),Arrow);
\end{asy}
\caption{Lagrange陀螺的运动}
\label{Lagrange陀螺的运动}
\end{figure}

\subsection{Lagrange函数与运动积分}

Lagrange情形下刚体的运动方程和运动积分已经在第\ref{chapter6:subsection-重刚体定点运动方程及其守恒量}节中列出,但在Lagrange情形下,刚体的Lagrange函数形式相对简单,因此此处用刚体的Euler作为广义坐标,使用Lagrange方程再重新推导一次Lagrange陀螺的运动积分。

在Lagrange情形下,刚体的动能和势能分别为
\begin{equation*}
	T = \frac12 \sum_{i=1}^3 I_i \omega_i^2,\quad V = Mgl\cos \theta
\end{equation*}
其中角速度$\omega_1,\omega_2,\omega_3$与广义坐标和广义速度的关系由Euler运动学方程\eqref{Euler运动学方程}给出
\begin{equation*}
\begin{cases}
	\omega_1 = \dot{\phi} \sin \theta \sin \psi + \dot{\theta} \cos \psi \\
	\omega_2 = \dot{\phi} \sin \theta \cos \psi - \dot{\theta} \sin \psi \\
	\omega_3 = \dot{\phi} \cos \theta + \dot{\psi}
\end{cases}
\end{equation*}
因此Lagrange陀螺的Lagrange函数\footnote{Lagrange函数和角动量大小的符号都是$L$,但在Lagrange情形下,刚体的角动量大小并不是运动积分,因此下文将避免使用角动量大小$L$,而仅使用角动量的分量(带有脚标的$L$),以免产生混淆。}为
\begin{equation}
	L = T-V = \frac12 I_1 (\dot{\theta}^2 + \dot{\phi}^2 \sin^2 \theta) + \frac12 I_3 (\dot{\psi} + \dot{\phi} \cos \theta)^2 - Mgl\cos \theta
	\label{Lagrange陀螺的Lagrange函数}
\end{equation}
由$\dfrac{\pl L}{\pl \phi} = 0$可得
\begin{equation}
	p_\phi = \frac{\pl L}{\pl \dot{\phi}} = (I_1 \sin^2 \theta + I_3 \cos^2 \theta) \dot{\phi} + I_3 \dot{\psi} \cos \theta = L_Z \quad \text{(常数)}
	\label{Lagrange陀螺守恒量——Z轴角动量}
\end{equation}
由$\dfrac{\pl L}{\pl \psi} = 0$可得
\begin{equation}
	p_\psi = \frac{\pl L}{\pl \dot{\psi}} = I_3(\dot{\psi}+\dot{\phi}\cos \theta) = L_3 \quad \text{(常数)}
	\label{Lagrange陀螺守恒量——3轴角动量}
\end{equation}
由$\dfrac{\pl L}{\pl t} = 0$可得
\begin{equation}
	H = \frac12 I_1 (\dot{\theta}^2 + \dot{\phi}^2 \sin^2 \theta) + \frac12 I_3 (\dot{\psi} + \dot{\phi} \cos \theta)^2 + Mgl\cos \theta = E \quad \text{(常数)}
	\label{Lagrange陀螺守恒量——能量}
\end{equation}

\subsection{Lagrange情形下的运动求解}

记$\varOmega=\dfrac{L_3}{I_3}, a = \dfrac{2Mgl}{I_1}, b = \dfrac{I_3}{I_1}, \alpha = \dfrac{2E}{I_1}-b\varOmega^2, \beta = \dfrac{L_Z}{I_1}$,则Lagrange情形下的三个运动积分\eqref{Lagrange陀螺守恒量——Z轴角动量}、\eqref{Lagrange陀螺守恒量——3轴角动量}和\eqref{Lagrange陀螺守恒量——能量}可以用这些参数写为
\begin{subnumcases}{}
	\sin^2\theta \dot{\phi} = \beta-b\varOmega\cos\theta \label{chapter6-Lagrange情形运动积分-1} \\
	\dot{\psi}+\dot{\phi}\cos\theta = \varOmega \label{chapter6-Lagrange情形运动积分-2} \\
	\dot{\theta}^2+\dot{\phi}^2\sin^2\theta = \alpha - a\cos\theta \label{chapter6-Lagrange情形运动积分-3}
\end{subnumcases}
作变量代换$u=\cos\theta$,则式\eqref{chapter6-Lagrange情形运动积分-1}和\eqref{chapter6-Lagrange情形运动积分-2}可以写作
\begin{subnumcases}{}
	\dot{\phi} = \frac{\beta-b\varOmega u}{1-u^2} \label{chapter6:进动角的方程} \\
	\dot{\psi} = \varOmega - u\frac{\beta-b\varOmega u}{1-u^2} \label{chapter6:自转角的方程}
\end{subnumcases}
而式\eqref{chapter6-Lagrange情形运动积分-3}则化为
\begin{equation}
	\dot{u}^2 = (\alpha-au)(1-u^2)-(\beta-b\varOmega u)^2 = G(u)
	\label{chapter6:章动角的控制方程}
\end{equation}
式中函数$G(u)$为
\begin{equation}
	G(u) = (\alpha-au)(1-u^2)-(\beta-b\varOmega u)^2
	\label{chapter6:函数G(u)的表达式}
\end{equation}
求解出\eqref{chapter6:章动角的控制方程}的解之后可得函数$u=u(t)$,进而可得$\theta=\theta(t)$,再将$u=u(t)$的结果代入式\eqref{chapter6:进动角的方程}和\eqref{chapter6:自转角的方程},那么三个Euler角关于时间的关系都得到了,刚体定点运动的Lagrange情形就解决了。

首先注意到方程\eqref{chapter6:章动角的控制方程}左端应该是非负的,以及$u=\cos\theta$,所以方程\eqref{chapter6:章动角的控制方程}具有物理意义的解应该要求函数$G(u)$在$[-1,1]$区间上有非负的区域。注意到$a>0$,很容易发现
\begin{align*}
	G(\pm 1) = -(\beta \mp b\varOmega)^2 < 0,\quad G(-\infty) < 0,\quad G(+\infty) > 0
\end{align*}
因此,函数$G(u)$有三个根,它的图像如图\ref{chapter6:figure-G(u)的图像}所示。如果将它们从小到大分别记作$u_1,u_2,u_3$,则它们分布在如下区域内:
\begin{equation}
	-1 < u_1<u_2<1<u_3
\end{equation}
因此,$u$只能在$u_1\leqslant u\leqslant u_2$的区域内变动,对应的章动角$\theta$也只能在$\theta_2\leqslant \theta\leqslant \theta_1$的区域内变动,其中$u_1=\cos\theta_1, u_2=\cos\theta_2$。这个运动对应的是陀螺的对称轴与空间系的$OZ$轴之间的夹角在$\theta_2$到$\theta_1$之间往复摆动。

\begin{figure}[htb]
\centering
\begin{asy}
	size(250);
	//Lagrange情形下G(u)的图像
	real alpha,beta,a,b,Omega;
	real G(real u){
		return (alpha-a*u)*(1-u^2)-(beta-b*Omega*u)^2;
	}
	alpha = 2;
	a = 2;
	b = 1/2;
	beta = 1.2;
	Omega = 1;
	real xmin,xmax,xm,ym;
	xm = 2;
	ym = 2;
	xmin = -1.9;
	xmax = 1.9;
	draw(Label("$u$",EndPoint),(-xm,0)--(xm,0),Arrow);
	draw(Label("$G(u)$",EndPoint,W),(0,-ym)--(0,ym),Arrow);
	path p = graph(G,xmin,xmax);
	picture tmp;
	draw(tmp,p,linewidth(0.8bp));
	clip(tmp,box((xmin,-ym),(xmax,ym)));
	add(tmp);
	real l = 0.05;
	draw(Label("$-1$",BeginPoint),(-1,0)--(-1,l));
	draw(Label("$1$",BeginPoint),(1,0)--(1,l));
	pair P[];
	P = intersectionpoints(p,(-xm,0)--(xm,0));
	label("$u_1$",P[0],dir(-50));
	label("$u_2$",P[1],SSW);
	label("$u_3$",P[2],SE);
\end{asy}
\caption{函数$G(u)$的图像}
\label{chapter6:figure-G(u)的图像}
\end{figure}

注意到当$u=\dfrac{\beta}{b\varOmega}$时,进动角速度$\dot{\phi}=0$,因此Lagrange陀螺的进动运动有三种情形:
\begin{enumerate}
\item 如果$u_1<\dfrac{\beta}{b\varOmega}<u_2$,则进动角速度$\dot{\phi}$有时为正有时为负,进动的方向不断改变。这种情况下陀螺的对称轴线在单位球面上画出的轨迹形状如图\ref{chapter6:figure-Lagrange陀螺进动1}所示。

\item 如果$\dfrac{\beta}{b\varOmega}<u_1$或者$\dfrac{\beta}{b\varOmega}>u_2$,则进动角速度$\dot{\phi}$恒正或恒负,进动的方向不变。这种情况下陀螺的对称轴线在单位球面上画出的轨迹形状如图\ref{chapter6:figure-Lagrange陀螺进动2}所示。

\item 如果$\dfrac{\beta}{b\varOmega}=u_1$或者$\dfrac{\beta}{b\varOmega}=u_2$,则进动角速度$\dot{\phi}$除了在某些时刻为零之外恒正或恒负,进动的方向仍然不变,但是陀螺轴线运动到$\dot{\phi}=0$时会有一个反向折回的现象。这种情况下陀螺的对称轴线在单位球面上画出的轨迹形状如图\ref{chapter6:figure-Lagrange陀螺进动3}所示。
\end{enumerate}

\begin{figure}[htb]
\centering
\begin{minipage}[t]{0.3\textwidth}
\centering
\begin{asy}
	size(150);
	//Lagrange陀螺进动3
	pair i,j,k;
	real r,theta0,thetam,w,theta,d;
	r = 1;
	theta0 = pi/4;
	thetam = pi/10;
	w = 10;
	d = 2.5;
	i = (-sqrt(2)/4,-sqrt(14)/12);
	j = (sqrt(14)/4,-sqrt(2)/12);
	k = (0,2*sqrt(2)/3);
	pair spconvert(real r,real theta,real phi){
		return r*sin(theta)*cos(phi)*i+r*sin(theta)*sin(phi)*j+r*cos(theta)*k;
	}
	pair thetaphi(real t){
		real x,y;
		x = t-d*sin(t);
		y = cos(t);
		return spconvert(r,theta0-thetam*y,x/r/sin(theta0)/w-0.1);
	}
	pair phicir(real phi){
		return spconvert(r,theta,phi);
	}
	draw(Label("$Z$",EndPoint),r*k--1.4*r*k,Arrow);
	theta = theta0-thetam;
	draw(graph(phicir,-1.8,2.57),dashed);
	label("$\theta_2$",phicir(2.57),NE);
	theta = theta0+thetam;
	draw(graph(phicir,-1.4,2.1),dashed);
	label("$\theta_1$",phicir(2.1),NE);
	draw(graph(thetaphi,-9.3,15.5,200),linewidth(0.8bp)+red);
	draw(scale(r)*unitcircle,linewidth(0.8bp));
\end{asy}
\caption{$u_1<\dfrac{\beta}{b\varOmega}<u_2$}
\label{chapter6:figure-Lagrange陀螺进动1}
\end{minipage}
\hspace{0.2cm}
\begin{minipage}[t]{0.3\textwidth}
\centering
\begin{asy}
	size(150);
	//Lagrange陀螺进动1
	pair i,j,k;
	real r,theta0,thetam,w,theta;
	r = 1;
	theta0 = pi/4;
	thetam = pi/10;
	w = 9;
	i = (-sqrt(2)/4,-sqrt(14)/12);
	j = (sqrt(14)/4,-sqrt(2)/12);
	k = (0,2*sqrt(2)/3);
	pair spconvert(real r,real theta,real phi){
		return r*sin(theta)*cos(phi)*i+r*sin(theta)*sin(phi)*j+r*cos(theta)*k;
	}
	pair thetaphi(real phi){
		return spconvert(r,theta0-thetam*sin(w*r*sin(theta0)*phi),phi);
	}
	pair phicir(real phi){
		return spconvert(r,theta,phi);
	}
	draw(Label("$Z$",EndPoint),r*k--1.4*r*k,Arrow);
	theta = theta0-thetam;
	draw(graph(phicir,-1.8,2.57),dashed);
	label("$\theta_2$",phicir(2.57),NE);
	theta = theta0+thetam;
	draw(graph(phicir,-1.4,2.1),dashed);
	label("$\theta_1$",phicir(2.1),NE);
	draw(graph(thetaphi,-1.4,2.4,200),linewidth(0.8bp)+red);
	draw(scale(r)*unitcircle,linewidth(0.8bp));
\end{asy}
\caption{$u_1<u_2<\dfrac{\beta}{b\varOmega}$}
\label{chapter6:figure-Lagrange陀螺进动2}
\end{minipage}
\hspace{0.2cm}
\begin{minipage}[t]{0.3\textwidth}
\centering
\begin{asy}
	size(150);
	//Lagrange陀螺进动2
	pair i,j,k;
	real r,theta0,thetam,w,theta;
	r = 1;
	theta0 = pi/4;
	thetam = pi/10;
	w = 7.5;
	i = (-sqrt(2)/4,-sqrt(14)/12);
	j = (sqrt(14)/4,-sqrt(2)/12);
	k = (0,2*sqrt(2)/3);
	pair spconvert(real r,real theta,real phi){
		return r*sin(theta)*cos(phi)*i+r*sin(theta)*sin(phi)*j+r*cos(theta)*k;
	}
	pair thetaphi(real t){
		real x,y;
		x = t-sin(t);
		y = cos(t);
		return spconvert(r,theta0-thetam*y,x/r/sin(theta0)/w-0.2);
	}
	pair phicir(real phi){
		return spconvert(r,theta,phi);
	}
	draw(Label("$Z$",EndPoint),r*k--1.4*r*k,Arrow);
	theta = theta0-thetam;
	draw(graph(phicir,-1.8,2.57),dashed);
	label("$\theta_2$",phicir(2.57),NE);
	theta = theta0+thetam;
	draw(graph(phicir,-1.4,2.1),dashed);
	label("$\theta_1$",phicir(2.1),NE);
	draw(graph(thetaphi,-8,14,200),linewidth(0.8bp)+red);
	draw(scale(r)*unitcircle,linewidth(0.8bp));
\end{asy}
\caption{$u_1<u_2=\dfrac{\beta}{b\varOmega}$}
\label{chapter6:figure-Lagrange陀螺进动3}
\end{minipage}
\end{figure}

常数$\dfrac{\beta}{b\varOmega}$的值完全由初始条件决定,它与$u_1, u_2$之间的关系,则由函数值$G\left(\dfrac{\beta}{b\varOmega}\right)$的正负决定。如果$G\left(\dfrac{\beta}{b\varOmega}\right)>0$,则$\dfrac{\beta}{b\varOmega}$在$u_1,u_2$之间,进动对应的是第一种情况;相反,如果$G\left(\dfrac{\beta}{b\varOmega}\right)<0$,则$\dfrac{\beta}{b\varOmega}<u_1$或者$\dfrac{\beta}{b\varOmega}>u_2$,进动对应的是第二种情况。如果$G\left(\dfrac{\beta}{b\varOmega}\right)=0$,则进动对应的是第三种情况。

根据式\eqref{chapter6:函数G(u)的表达式}可得
\begin{equation}
	G\left(\dfrac{\beta}{b\varOmega}\right)= \left(\alpha-\frac{a\beta}{b\varOmega}\right)\left(1-\dfrac{\beta^2}{b^2\varOmega^2}\right)
\end{equation}
对其正负的讨论需要根据$\left|\dfrac{\beta}{b\varOmega}\right|=\left|\dfrac{L_Z}{L_3}\right|$与$1$的关系分为如下几类:
\begin{enumerate}
\item $\left|\dfrac{\beta}{b\varOmega}\right|=\left|\dfrac{L_Z}{L_3}\right|<1$。在这种情况下,前文所述的三种情形都有可能发生。
\begin{enumerate}
\item 当$\alpha>\dfrac{a\beta}{b\varOmega}$,即$E>\dfrac{L_3^2}{2I_3}+Mgl\dfrac{L_Z}{L_3}$时,此时有$u_1<\dfrac{\beta}{b\varOmega}<u_2$,进动运动为第一种情况,如图\ref{chapter6:figure-Lagrange陀螺进动1}所示。

\item 当$\alpha>\dfrac{a\beta}{b\varOmega}$,即$E<\dfrac{L_3^2}{2I_3}+Mgl\dfrac{L_Z}{L_3}$时,进动运动为第二种情况,如图\ref{chapter6:figure-Lagrange陀螺进动2}所示。

\item 当$\alpha=\dfrac{a\beta}{b\varOmega}$,即$E=\dfrac{L_3^2}{2I_3}+Mgl\dfrac{L_Z}{L_3}$时,进动运动为第三种情况,如图\ref{chapter6:figure-Lagrange陀螺进动3}所示。特别地,在这种情况下,一定有$\dfrac{\beta}{b\varOmega}=u_2>u_1$,因而进动角速度为零的点必然在章动区域的上边缘。为了说明这一点,将参数$\alpha=\dfrac{a\beta}{b\varOmega}$代入$G(u)$的表达式中可得
\begin{equation}
	G(u) = \left(\frac{\beta}{b\varOmega}-u\right)\left[a(1-u^2)-b^2\varOmega^2\left(\frac{\beta}{b\varOmega}-u\right)\right]
\end{equation}
$G(u)$的一个根是$\dfrac{\beta}{b\varOmega}$,另外两个根由方程
\begin{equation}
	b^2\varOmega^2\left(\frac{\beta}{b\varOmega}-u\right) = a(1-u^2)
	\label{chapter6:G(u)另一根的方程}
\end{equation}
来决定,其中一个根是$u_3>1$,而另一个根为$u_1,u_2$其中之一,不论是哪一个,都满足$u^2<1$,据此根据方程\eqref{chapter6:G(u)另一根的方程}的左端为正可得这个根小于$\dfrac{a\beta}{b\varOmega}$,由此即说明$\dfrac{a\beta}{b\varOmega}=u_2>u_1$。
\end{enumerate}

\item $\left|\dfrac{\beta}{b\varOmega}\right|=\left|\dfrac{L_Z}{L_3}\right|\geqslant 1$。在这种情况下,只有第二种运动是可能的,如图\ref{chapter6:figure-Lagrange陀螺进动2}。
\end{enumerate}

下面对方程\eqref{chapter6:章动角的控制方程}进行解析求解以分析其运动的周期性。利用函数$G(u)$的三个根,可以将方程\eqref{chapter6:章动角的控制方程}表示为
\begin{equation}
	\dot{u}^2 = a(u-u_1)(u-u_2)(u-u_3)
	\label{chapter6:针对章动角的微分方程}
\end{equation}
作变量代换
\begin{equation}
	u = u_1 + (u_2-u_1)\sin^2v
\end{equation}
可以将方程\eqref{chapter6:针对章动角的微分方程}变换为
\begin{equation}
	\dot{v} = \frac{a(u_3-u_1)}{4}\left(1-k^2\sin^2v\right)
	\label{chapter6:针对章动角的微分方程-变换后}
\end{equation}
其中
\begin{equation}
	k^2=\dfrac{u_2-u_1}{u_3-u_1} \quad (0\leqslant k^2\leqslant 1)
\end{equation}
取$u=u_1$的时刻为初始时刻,则积分方程\eqref{chapter6:针对章动角的微分方程-变换后}可得
\begin{equation}
	\frac{\sqrt{a(u_3-u_1)}}{2}t = \int_0^v \frac{\mathd w}{\sqrt{1-k^2\sin^2w}} = F(k,\sin v)
\end{equation}
由此可得
\begin{equation}
	\sin v = \sn \left(\frac{\sqrt{a(u_3-u_1)}}{2}t\right)
\end{equation}
所以
\begin{equation}
	u = u_1+(u_2-u_1)\sn^2 \left(\frac{\sqrt{a(u_3-u_1)}}{2}t\right)
\end{equation}
进而可得
\begin{equation}
	\theta = \arccos \left[u_1+(u_2-u_1)\sn^2 \left(\frac{\sqrt{a(u_3-u_1)}}{2}t\right)\right]
\end{equation}
由函数$\sn \tau$的周期为$4K(k)$可得函数$\sn^2\tau$的周期为$2K(k)$,因此函数$u$的周期为$\dfrac{4K(k)}{\sqrt{a(u_3-u_1)}}$,即章动角变化的周期为
\begin{equation}
	T = \dfrac{4K(k)}{\sqrt{a(u_3-u_1)}}
\end{equation}
再根据式\eqref{chapter6:进动角的方程}和\eqref{chapter6:自转角的方程}可得函数$\dot{\phi}$和$\dot{\psi}$的周期也为$T = \dfrac{4K(k)}{\sqrt{a(u_3-u_1)}}$,根据
\begin{equation}
	\dif{\phi}{t}(t+T) = \dif{\phi}{t}(t),\quad \dif{\psi}{t}(t+T) = \dif{\psi}{t}(t)
\end{equation}
两端积分可得
\begin{equation}
	\phi(t+T)=\phi(t)+\Delta \phi,\quad \psi(t+T)=\psi(t)+\Delta\psi
\end{equation}
其中$\Delta \phi$和$\Delta \psi$分别为进动角和自转角一周期内的改变量,这可以根据\eqref{chapter6:进动角的方程}和\eqref{chapter6:自转角的方程}直接积分得到。

Lagrange情形下运动的周期性,将取决于$\theta_2-\theta_1, \Delta\phi, \Delta\psi$这三个量之间的比值是否能够化为有理数之比。或者将条件放宽,不考虑自转运动,那么周期性的条件为$\dfrac{\theta_2-\theta_1}{\Delta\phi}$是否为有理数。如果比值$\dfrac{\theta_2-\theta_1}{\Delta\phi}$是有理数,那么图\ref{chapter6:figure-Lagrange陀螺进动1}-\ref{chapter6:figure-Lagrange陀螺进动3}上的轨迹将是闭合的;如果比值$\dfrac{\theta_2-\theta_1}{\Delta\phi}$不是有理数,那么图\ref{chapter6:figure-Lagrange陀螺进动1}-\ref{chapter6:figure-Lagrange陀螺进动3}上的轨迹在足够长的时间内将充满$\theta_1$和$\theta_2$之间的球面区域。

\subsection{Lagrange情形下刚体规则进动的条件}

作为一种特殊情况,来看一下在Lagrange情形下刚体作规则进动的条件。直接用方程\eqref{chapter6:针对章动角的微分方程}来考虑的话,需要$u_1=u_2$,这样方程\eqref{chapter6:针对章动角的微分方程}的解只有$\dot{u}=0$,则$\theta$为常数,即对应规则进动。但是函数$G(u)$满足$u_1=u_2$的条件比较复杂,因此此处利用等效势来讨论刚体在Lagrange情形下刚体规则进动的条件。

将方程\eqref{Lagrange陀螺守恒量——Z轴角动量}和\eqref{Lagrange陀螺守恒量——3轴角动量}联立可得与方程\eqref{chapter6:进动角的方程}和\eqref{chapter6:自转角的方程}相同的结果:
\begin{subnumcases}{}
	\dot{\phi} = \frac{L_Z-L_3\cos\theta}{I_1\sin^2\theta} \label{chapter6:进动角的方程-重写} \\
	\dot{\psi} = \frac{L_3}{I_3}-\frac{(L_Z-L_3\cos\theta)\cos\theta}{I_1\sin^2\theta} \label{chapter6:自转角的方程-重写}
\end{subnumcases}
将式\eqref{chapter6:进动角的方程-重写}和\eqref{chapter6:自转角的方程-重写}代入能量积分\eqref{Lagrange陀螺守恒量——能量}可得
\begin{equation}
	\frac12 I_1\dot{\theta}^2 + \frac{(L_Z-L_3\cos\theta)^2}{2I_1\sin^2\theta}+\frac{L_3^2}{2I_3}+Mgl\cos\theta = E
	\label{chapter6:Lagrange情形下的章动等效运动方程}
\end{equation}
此方程可以视为关于$\theta$的一维运动方程,令
\begin{equation}
	U(\theta) = \frac{(L_Z-L_3\cos\theta)^2}{2I_1\sin^2\theta}+\frac{L_3^2}{2I_3}+Mgl\cos\theta
\end{equation}
为重力矩与惯性力矩的{\bf 等效势},由此方程\eqref{chapter6:Lagrange情形下的章动等效运动方程}可以表示为
\begin{equation}
	\frac12 I_1\dot{\theta}^2 + U(\theta) = E
\end{equation}
等效势的图像如图\ref{Lagrange陀螺等效势能曲线}所示。

\begin{figure}[htb]
\centering
\begin{asy}
	size(250);
	//Lagrange陀螺等效势能曲线
	picture tmp;
	pair O;
	real mgl,LZ,L3,I1,I3,x,thetam,theta1,theta2,theta0,bot,top;
	path clp;
	real U(real theta){
		return mgl*cos(theta)+(LZ-L3*cos(theta))*(LZ-L3*cos(theta))/(2*I1*sin(theta)*sin(theta)) + L3*L3/(2*I3);
	}
	real dU(real theta){
		return -mgl*sin(theta)+(L3*(sin(theta))**2*(LZ-L3*cos(theta))-sin(theta)*cos(theta)*(LZ-L3*cos(theta))**2)/(I1*(sin(theta))**3);
	}
	real Utheta1(real theta){
		return mgl*cos(theta)+(LZ-L3*cos(theta))*(LZ-L3*cos(theta))/(2*I1*sin(theta)*sin(theta)) + L3*L3/(2*I3)-U(theta1);
	}
	real bis(real f(real),real x1,real x2,real eps){
		int imax;
		real a,b,fab,fa,fb;
		a = x1;
		b = x2;
		imax = 1000;
		for(int i=1;i<=imax;i=i+1){
			fa = f(a);
			fb = f(b);
			fab = f((a+b)/2);
			if(fab == 0) return (a+b)/2;
			if(fab*fa<0){
				b = (a+b)/2;
			}
			else {
				a = (a+b)/2;
			}
			if(abs(b-a)<eps){
				return (a+b)/2;
			}
		}
		return (a+b)/2;
	}
	mgl = 10;
	LZ = -.6;
	L3 = 2;
	I1 = 2;
	I3 = 1;
	x = 20;
	bot = -10;
	top = 60;
	draw(tmp,xscale(x)*graph(U,0.01,pi-0.01),linewidth(1bp));
	clp = box((0,bot),(x*pi,80));
	clip(tmp,clp);
	draw(tmp,clp);
	clp = box((-1,bot),(x*pi,top));
	clip(tmp,clp);
	add(tmp);
	thetam = bis(dU,2,3,1e-10)-0.04;
	draw((0,U(thetam))--(x*pi,U(thetam)),red);
	draw((x*thetam,bot)--(x*thetam,U(thetam)),dashed);
	theta1 = 1.7;
	theta2 = bis(Utheta1,2,3,1e-10);
	theta0 = acos(LZ/L3);
	draw((0,U(theta1))--(x*pi,U(theta1)),blue);
	draw((x*theta1,bot)--(x*theta1,U(theta1)),dashed);
	draw((x*theta2,bot)--(x*theta2,U(theta2)),dashed);
	label("$0$",(x*0,bot),S);
	label("$\theta_2$",(x*theta1,bot),S);
	label("$\theta_m$",(x*thetam,bot),S);
	label("$\theta_1$",(x*theta2,bot),S);
	label("$\pi$",(x*pi,bot),S);
	label("$\theta$",(x*pi,bot),E);
	label("$E$",(0,U(theta1)),W);
	label("$U$",(0,top),N);
\end{asy}
\caption{Lagrange情形下的等效势}
\label{Lagrange陀螺等效势能曲线}
\end{figure}

在图\ref{Lagrange陀螺等效势能曲线}中,$U(\theta)=E$所对应的$\theta$就是前文所提到的章动角转折点$\theta_1$和$\theta_2$,在这两点有$\dot{\theta}=0$。规则进动的条件$u_1=u_2$等价于$\theta_1=\theta_2$,因此只需要有$E=U(\theta_m)$即可,其中$\theta_m$为函数$U(\theta)$的极值点。因此,$\theta_m$需要满足方程
\begin{equation}
	U'(\theta) = \frac{(L_3-L_Z\cos\theta)(L_Z-L_3\cos\theta)}{I_1\sin^3\theta} - Mgl\sin \theta = 0
	\label{chapter6:Lagrange情形下规则进动的条件1}
\end{equation}
将式\eqref{chapter6:进动角的方程-重写}表示的$L_Z$代入整理,可得
\begin{equation}
	I_1\dot{\phi}^2\cos\theta_m-L_3\dot{\phi} + Mgl = 0
	\label{chapter6:Lagrange情形下规则进动的条件2}
\end{equation}
由于$\dot{\phi}$应该为实数,方程\eqref{chapter6:Lagrange情形下规则进动的条件2}有解的条件为
\begin{equation}
	L_3^2 \geqslant 4MglI_1\cos\theta_m
	\label{chapter6:Lagrange情形下规则进动的条件3}
\end{equation}
式中$\theta_m$由方程\eqref{chapter6:Lagrange情形下规则进动的条件1}决定。因此,Lagrange情形下,规则进动的条件为$E=U(\theta_m)$和式\eqref{chapter6:Lagrange情形下规则进动的条件3}。

\begin{example}
一陀螺由半径为$2r$的薄圆盘及通过圆盘中心$C$,并和盘面垂直的场为$r$的杆轴所组成,杆的质量可忽略不计。将杆的另一端$O$放在水平面上,使其作无滑动的转动。如起始时杆$OC$与铅直线的夹角为$\alpha$,起始时的总角速度为$\omega$,方向沿着$\alpha$角的平分线,证明经过
\begin{equation*}
	t = \int_0^\alpha \frac{\mathrm{d} \theta}{\omega\sqrt{1 - \cos^2 \dfrac{\alpha}{2} \sec^2 \dfrac{\theta}{2} + \dfrac{g}{r\omega^2}(\cos \alpha-\cos \theta)}}
\end{equation*}
后杆将直立起来,式中$\theta$为任意瞬时杆轴与铅直线的夹角。

\begin{figure}[htb]
\centering
\begin{asy}
	size(250);
	//第六章例6图
	pair O,dash,P;
	real alpha,r,d;
	O = (0,0);
	r = 1;
	d = 0.05;
	alpha = 30;
	fill(rotate(alpha)*box((r-d,-2*r),(r+d,2*r)),0.6white);
	draw(Label("$z$",EndPoint),O--2.5*r*dir(alpha),Arrow);
	draw(Label("$Z$",EndPoint),O--2.5*r*dir(90),Arrow);
	draw(Label("$\boldsymbol{\omega}_0$",EndPoint,black),O--2.5*r*dir(alpha+(90-alpha)/2),red,Arrow);
	draw((-r/2,0)--(r/2,0));
	dash = 2*d*dir(-120);
	for(int i=1;i<=6;i=i+1){
		P = (r/2-r/6*(i-1),0);
		draw(P--P+dash);
	}
	label("$O$",O,NW);
	label("$C$",r*dir(alpha),2*E);
	label("$r$",1/2*r*dir(alpha),dir(alpha-90));
	label("$2r$",r*dir(alpha)+r*dir(alpha-90),dir(alpha-180));
	draw(Label("$\dfrac{\alpha}{2}$",Relative(0.3),Relative(E)),arc(O,0.4*r,alpha,alpha+(90-alpha)/2),Arrow);
	draw(Label("$\dfrac{\alpha}{2}$",Relative(0.5),Relative(E)),arc(O,0.4*r,alpha+(90-alpha)/2,90),Arrow);
	//draw(O--(2.5,0),invisible);
\end{asy}
\caption{例\theexample}
\label{第六章例6图}
\end{figure}
\end{example}

\begin{solution}
圆盘对$C$的惯量矩阵为
\begin{equation*}
	\mbf{I}' = \begin{pmatrix} mr^2 & 0 & 0 \\ 0 & mr^2 & 0 \\ 0 & 0 & 2mr^2 \end{pmatrix}
\end{equation*}
质心对$O$点的坐标为$\mbf{X}_C = \begin{pmatrix} 0 \\ 0 \\ r \end{pmatrix}$,因此有质心对$O$的转动惯量为
\begin{equation*}
	\mbf{I}_C = \begin{pmatrix} mr^2 & 0 & 0 \\ 0 & mr^2 & 0 \\ 0 & 0 & 0 \end{pmatrix}
\end{equation*}
根据平行轴定理可得圆盘对$O$的转动惯量为
\begin{equation*}
	\mbf{I} = \mbf{I}_C + \mbf{I}' = \begin{pmatrix} 2mr^2 & 0 & 0 \\ 0 & 2mr^2 & 0 \\ 0 & 0 & 2mr^2 \end{pmatrix}
\end{equation*}
故圆盘为Lagrange陀螺。其转动惯量和动能分别为
\begin{equation*}
	\mbf{L} = \mbf{I} \mbf{\omega} = 2mr^2 \mbf{\omega},\quad T = \frac12 \mbf{\omega}\cdot \mbf{L} = mr^2 \omega^2
\end{equation*}
由此可得三个积分常数为
\begin{equation*}
	\begin{cases}
		L_Z = L_{Z0} = 2mr^2 \omega_{Z0} = 2mr^2 \omega \cos \dfrac{\alpha}{2} \\
		L_3 = L_{30} = 2mr^2 \omega_{30} = 2mr^2 \omega \cos \dfrac{\alpha}{2} \\
		E = E_0 = mr^2 \omega^2 + mgr\cos \alpha
	\end{cases}
\end{equation*}
此时,章动等效势为
\begin{equation*}
	U(\theta) = mr^2 \omega^2 \cos^2 \frac{\alpha}{2} \sec^2 \frac{\theta}{2} + mgr\cos \theta = mgr\left(\frac{r\omega^2}{g}\cos^2 \frac{\alpha}{2} \sec^2 \frac{\theta}{2} + \cos \theta\right)
\end{equation*}
章动方程为
\begin{equation*}
	\frac12 I_1 \dot{\theta}^2 + U(\theta) = E
\end{equation*}
初始时刻有
\begin{equation*}
	\frac12 I_1 \dot{\theta}_0^2 = E - U(\alpha) = 0
\end{equation*}
即初始章动角速度为零,章动处于极限角位置。下面按$\dfrac{r\omega^2}{g}$可能的取值分类讨论,每种情况对应的等效势能曲线如图\ref{第六章例6势能曲线}所示。

\begin{figure}[htb]
\centering
\begin{asy}
	size(400);
	//直立转动Lagrange陀螺的等效势能曲线
	picture tmp;
	pair O,P;
	real alpha,beta,top,bot,x,left,right,height;
	path clp;
	real U(real theta){
		return 10*(beta*(cos(alpha/2))**2*(sec(theta/2))**2+cos(theta));
	}
	O = (0,0);
	alpha = 1.3;
	beta = 0;
	top = 70;
	bot = U(pi);
	x = 20;
	draw(tmp,O--(pi,0));
	draw(tmp,(alpha,bot)--(alpha,top),dashed);
	beta = 0;
	draw(tmp,graph(U,0,pi),black+linewidth(1bp));
	draw(tmp,Label("$E_1$",BeginPoint),(0,U(alpha))--(pi,U(alpha)),black);
	beta = (cos(alpha/2))**2;
	draw(tmp,graph(U,0,3),red+linewidth(1bp));
	draw(tmp,Label("$E_2$",BeginPoint),(0,U(alpha))--(pi,U(alpha)),red);
	beta = 2*(cos(alpha/2))**2;
	draw(tmp,graph(U,0,3),orange+linewidth(1bp));
	draw(tmp,Label("$E_3$",BeginPoint),(0,U(alpha))--(pi,U(alpha)),orange);
	beta = (2*(cos(alpha/2))**2+2)/2;
	draw(tmp,graph(U,0,3),green+linewidth(1bp));
	draw(tmp,Label("$E_4$",BeginPoint),(0,U(alpha))--(pi,U(alpha)),green);
	beta = 2;
	draw(tmp,graph(U,0,3),blue+linewidth(1bp));
	draw(tmp,Label("$E_5$",BeginPoint),(0,U(alpha))--(pi,U(alpha)),blue);
	beta = (2+2*(sec(alpha/2))**2)/2;
	draw(tmp,graph(U,0,3),darkblue+linewidth(1bp));
	draw(tmp,Label("$E_6$",BeginPoint),(0,U(alpha))--(pi,U(alpha)),darkblue);
	beta = 2*(sec(alpha/2))**2;
	draw(tmp,graph(U,0,3),purple+linewidth(1bp));
	draw(tmp,Label("$E_7$",BeginPoint),(0,U(alpha))--(pi,U(alpha)),purple);
	beta = 3*(sec(alpha/2))**2;
	draw(tmp,graph(U,0,3),brown+linewidth(1bp));
	draw(tmp,Label("$E_8$",BeginPoint),(0,U(alpha))--(pi,U(alpha)),brown);
	clp = box((-10,bot),(pi,top));
	clip(tmp,clp);
	clp = box((0,bot),(pi,top));
	draw(tmp,clp);
	top = 60;
	clp = box((-1,bot),(pi,top));
	clip(tmp,clp);
	add(xscale(x)*tmp);
	label("$0$",(0,bot),S);
	label("$\pi$",(x*pi,bot),S);
	label("$\alpha$",(x*alpha,bot),S);
	label("$\theta$",(x*pi,bot),E);
	label("$U$",(0,top),N);
	label("$0$",O,W);
	left = x*pi+4;
	right = x*pi+30;
	height = 66;
	top = top-2;
	fill(shift((1,-1))*box((left,top),(right,top-height)),black);
	unfill(box((left,top),(right,top-height)));
	draw(box((left,top),(right,top-height)));
	P = ((left+right)/2,top-height+height/16);
	label("$\dfrac{r\omega^2}{g} = 0$",P,black);
	P = ((left+right)/2,top-height+3*height/16);
	label("$0 < \dfrac{r\omega^2}{g} < 2\cos^2 \dfrac{\alpha}{2}$",P,red);
	P = ((left+right)/2,top-height+5*height/16);
	label("$\dfrac{r\omega^2}{g} = 2\cos^2 \dfrac{\alpha}{2}$",P,orange);
	P = ((left+right)/2,top-height+7*height/16);
	label("$2\cos^2 \dfrac{\alpha}{2} < \dfrac{r\omega^2}{g} < 2$",P,green);
	P = ((left+right)/2,top-height+9*height/16);
	label("$\dfrac{r\omega^2}{g} = 2$",P,blue);
	P = ((left+right)/2,top-height+11*height/16);
	label("$2 < \dfrac{r\omega^2}{g} < 2\sec^2 \dfrac{\alpha}{2}$",P,darkblue);
	P = ((left+right)/2,top-height+13*height/16);
	label("$\dfrac{r\omega^2}{g} = 2\sec^2 \dfrac{\alpha}{2}$",P,purple);
	P = ((left+right)/2,top-height+15*height/16);
	label("$\dfrac{r\omega^2}{g} > 2\sec^2 \dfrac{\alpha}{2}$",P,brown);
	//draw(O--(right*1.05,0),invisible);
\end{asy}
\caption{例\theexample 势能曲线}
\label{第六章例6势能曲线}
\end{figure}

\begin{enumerate}
	\item $\dfrac{r\omega^2}{g} = 0$。此时$U(\theta) = mgr\cos \theta$,刚体为复摆,平衡位置在$\theta = \pi$,不可能直立。
	\item $0 < \dfrac{r\omega^2}{g} < 2\cos^2 \dfrac{\alpha}{2}$。此时$U'(\alpha)<0$,刚体进行初始向下的往复章动,不可能直立。
	\item $\dfrac{r\omega^2}{g} = 2\cos^2 \dfrac{\alpha}{2}$。此时$U'(\alpha) = 0$,刚体进行规则进动,不可能直立。
	\item $2\cos^2 \dfrac{\alpha}{2} < \dfrac{r\omega^2}{g} < 2$。此时$U'(\alpha) > 0,\,U(0) > E$,刚体进行初始向上的往复章动,不可能直立。
	\item $\dfrac{r\omega^2}{g} = 2$。此时$U(0) = E$,恰能直立,刚体将以无限小章动角速度趋于直立位置,所需时间为无穷大。
	\item $2 < \dfrac{r\omega^2}{g} < 2\sec^2 \dfrac{\alpha}{2}$。此时$U(0)<E,\,U''(0)<0$,章动角速度先增后减,渐趋匀速,故刚体可在有限时间之内直立起来。
	\item $\dfrac{r\omega^2}{g} = 2\sec^2 \dfrac{\alpha}{2}$。此时$U''(0) = 0,\,U^{(4)}(0) > 0$,章动角速度递增,较快趋于匀速,故刚体可在有限时间之内直立起来。
	\item $\dfrac{r\omega^2}{g} > 2\sec^2 \dfrac{\alpha}{2}$。此时$U''(0) > 0$,章动角速度递增,较慢趋于匀速,故刚体可在有限时间之内直立起来。
\end{enumerate}
综上,刚体能够直立的条件为$\ds\frac{r\omega^2}{g} \geqslant 2$。由章动方程$\dfrac12 I_1 \dot{\theta}^2 + U(\theta) = E$可得
\begin{equation*}
	\frac{\mathrm{d} \theta}{\mathrm{d} t} = -\sqrt{\frac{2(E-U(\theta))}{I_1}}
\end{equation*}
此处,由于直立过程中,章动角随时间减小,故开方取负。因此,从初始章动角位置至直立所需时间为
\begin{equation*}
	t = -\sqrt{\frac{I_1}{2}} \int_\alpha^0 \frac{\mathrm{d} \theta}{\sqrt{E-U(\theta)}} = \int_0^\alpha \frac{\mathrm{d} \theta}{\omega\sqrt{1 - \cos^2 \dfrac{\alpha}{2} \sec^2 \dfrac{\theta}{2} + \dfrac{g}{r\omega^2}(\cos \alpha-\cos \theta)}}
\end{equation*}
\end{solution}

\section{重刚体沿水平面的运动}

\subsection{摩擦}

设刚性曲面$S$沿着固定曲面$S_1$运动(如图\ref{chapter6:figure-曲面之间的接触}所示),这里假设曲面$S$和$S_1$都是凸的,相切于$O$点。假设在每个时刻经过$O$点只能有唯一的$S$和$S_1$的公共切平面\footnote{如果曲面$S$和$S_1$都是光滑曲面,并且又具有唯一接触点,则过该点必然只存在一个公共切平面。},而一般来说,在$S$运动时,点$O$既沿着$S$运动又沿着$S_1$运动。故显然点$O$的速度$\mbf{v}_O$应该位于过$O$点的公共切平面内。如果$\mbf{v}_O=\mbf{0}$则称运动是无滑动的,如果$\mbf{v}_O\neq \mbf{0}$则称运动是有滑动的,而$\mbf{v}_O$称为{\bf 滑动速度}。

\begin{figure}[htb]
\centering
\begin{asy}
	size(300);
	path curve(pair P1,pair P2,pair modify){
		return P1..(P1+P2)/2+modify..P2;
	}
	pair O,A,B,C;
	O = (0,0);
	A = (2.5,-1);
	B = (5.5,1);
	C = (3,2);
	draw(curve(O,A,(0,0.15)));
	draw(curve(A,B,(0,0.25)));
	draw(curve(B,C,(0,0.15)));
	draw(curve(C,O,(0,0.25)));
	pair D,F,G,H,L;
	D = (2,1.2);
	F = (3.5,1.15);
	G = (4.75,3);
	H = (2.7,3.8);
	L = (0.5,2);
	path clp,p1;
	clp = curve(F,G,(0,-0.3))--curve(G,H,(0,-0.05))--curve(H,L,(0,-0.55))--curve(L,F,(0,-0.7))--cycle;
	p1 = curve(L,D,(0,-0.2))--curve(D,F,(0,-0.15));
	clp = shift(0,-0.4)*clp;
	p1 = shift(0,-0.4)*p1;
	//curve(D,F,(0,-0.1))--
	unfill(clp);
	draw(clp,linewidth(0.8bp));
	draw(p1,linewidth(0.8bp));
	label("$S_1$",B-(0.35,0));
	label("$S$",G+(0,-0.4)+0.25*dir(-160));
	O = (2.7,1.2);
	A = O+1.5*dir(95);
	C = O+1*dir(170);
	B = A+C-O;
	label("$O$",O,S);
	draw(Label("$\boldsymbol{\omega}_{\bot}$",EndPoint,E),O--A,Arrow);
	draw(Label("$\boldsymbol{\omega}$",EndPoint),O--B,Arrow);
	draw(Label("$\boldsymbol{\omega}_{\parallel}$",EndPoint),O--C,Arrow);
	draw(A--B--C,dashed);
\end{asy}
\caption{曲面之间的接触}
\label{chapter6:figure-曲面之间的接触}
\end{figure}

取$O$点为基点,那么在每个时刻曲面$S$的运动都可以看作是速度为$\mbf{v}_O$的平动和角速度为$\mbf{\omega}$的定点转动。将角速度$\mbf{\omega}$分解为垂直于公共切平面的$\mbf{\omega}_{\bot}$和位于公共切平面内的$\mbf{\omega}_{\parallel}$,其中$\mbf{\omega}_{\bot}$称为曲面$S$的转动角速度,$\mbf{\omega}_{\parallel}$称为曲面$S$的滚动角速度。

如果$\mbf{v}_O=\mbf{0}$,则称曲面$S$在曲面$S_1$上滚动。如果$\mbf{v}_O=\mbf{0}$的同时有$\mbf{\omega}_{\bot}=\mbf{0}$但$\mbf{\omega}_{\parallel}\neq \mbf{0}$,则称曲面$S$在曲面$S_1$上纯滚动,而如果$\mbf{v}_O=\mbf{0}$的同时有$\mbf{\omega}_{\bot}\neq\mbf{0}, \mbf{\omega}_{\parallel}=\mbf{0}$则称曲面$S$在曲面$S_1$上转动。如果$\mbf{v}_O\neq\mbf{0}$而$\mbf{\omega}_{\bot}=\mbf{\omega}_{\parallel}=\mbf{0}$,则称曲面$S$在曲面$S_1$上滑动。一般来说,当$\mbf{v}_O\neq\mbf{0}, \mbf{\omega}_{\bot} \neq \mbf{0}, \mbf{\omega}_{\parallel} \neq \mbf{0}$时,曲面$S$在曲面$S_1$上既滑动又转动和滚动。

曲面$S_1$对曲面$S$的作用力也可以分解为垂直于公共切平面的$\mbf{N}$和位于公共切平面内的$\mbf{F}$。垂直于公共切平面的约束力$\mbf{N}$称为{\bf 法向约束反力},对于实际运动来说$N\geqslant 0$。位于公共切平面内的力$\mbf{F}$称为{\bf 摩擦力},一般情况下可以用{\bf Coulomb摩擦定律}来描述,摩擦力的大小不超过其最大可能值$kN$,其中$k$是摩擦系数。如果$\mbf{v}_O=\mbf{0}$,则$F<kN$,这时$\mbf{F}$称为{\bf 静摩擦力};如果$\mbf{v}_O\neq\mbf{0}$,则$F=kN$,这时$\mbf{F}$称为{\bf 滑动摩擦力}\footnote{需要注意的是,摩擦是十分复杂的现象,Coulomb摩擦定律只是近似的实验定律。}。对于实际的接触面,摩擦力总是存在的,如果$k$充分小,那么可以忽略摩擦力,认为曲面是绝对光滑的。

实际上,刚体的相互接触不是一个点,而是一个很小的面,那么曲面$S_1$对曲面$S$的作用力不能简化为一个力(即法向约束反力和摩擦力的合力),而只能简化为一个力和一个力矩。这个力与前文所述的相同,可以分解为法向约束反力和摩擦力。而力矩也可以分解为垂直于公共切平面的{\bf 转动摩擦力矩}和位于公共切平面内的{\bf 滚动摩擦力矩}。通常情况下,这两个摩擦力矩与摩擦力相比对曲面$S$的运动影响是很小的,因此在通常的问题中经常忽略转动摩擦力矩和滚动摩擦力矩而只考虑滑动摩擦力。

\subsection{存在摩擦时均匀球在平面上的运动}

设质量为$m$,半径为$a$的均质球在固定的粗糙水平面上运动。空间系$OXYZ$以固定水平面上任意一点为原点,$OZ$轴取竖直方向;平动系$Cxyz$以球的质心为原点,各坐标轴与空间系坐标轴平行。

\begin{figure}[htb]
\centering
\begin{asy}
	size(250);
	picture pic;
	pair O,i,j,k,C;
	real x,y,z,a;
	O = (0,0);
	i = dir(-135);
	j = dir(0);
	k = dir(90);
	x = 1;
	y = 2.5;
	z = 1.75;
	draw(Label("$X$",EndPoint),O--x*i,Arrow);
	draw(Label("$Y$",EndPoint),O--y*j,Arrow);
	draw(pic,Label("$Y$",EndPoint),O--y*j,dashed,Arrow);
	draw(Label("$Z$",EndPoint),O--z*k,Arrow);
	label("$O$",O,W);
	a = 0.7;
	C = (1,0.35);
	x = 1.55;
	y = y-C.x;
	z = z-C.y;
	path p = shift(C)*scale(a)*unitcircle;
	draw(Label("$x$",EndPoint),C--C+x*i,Arrow);
	draw(Label("$y$",EndPoint),C--C+y*j,Arrow);
	draw(Label("$z$",EndPoint),C--C+z*k,Arrow);
	draw(pic,Label("$z$",EndPoint),C--C+x*i,dashed,Arrow);
	draw(pic,Label("$y$",EndPoint),C--C+y*j,dashed,Arrow);
	draw(pic,Label("$z$",EndPoint),C--C+z*k,dashed,Arrow);
	unfill(p);
	clip(pic,p);
	add(pic);
	draw(p,linewidth(0.8bp));
	dot("$C$",C,W);
	pair mg,D,FN,F;
	mg = 0.3*dir(-90);
	D = intersectionpoint(p,C--C+100*dir(-90));
	FN = 0.3*dir(90);
	F = 0.8*dir(-20);
	draw(Label("$m\boldsymbol{g}$",EndPoint,E),C--C+mg,Arrow);
	draw(Label("$\boldsymbol{N}$",EndPoint,dir(-20)),D--D+FN,Arrow);
	draw(Label("$\boldsymbol{F}$",EndPoint,E),D--D+F,Arrow);
	label("$D$",D,SSW);
\end{asy}
\caption{固定粗糙水平面上运动球体的受力}
\label{chapter6:figure-固定粗糙水平面上运动球体的受力}
\end{figure}

球所受平面的约束反力$\mbf{R}$可以分解为两个力:$\mbf{R}=\mbf{N}+\mbf{F}$,其中$\mbf{N}$是平面的法向约束反力,而$\mbf{F}$是摩擦力。将球的质心速度记作$\mbf{v}_C$,角速度记作$\mbf{\omega}$,则球与平面接触点$D$的速度为
\begin{equation}
	\mbf{v}_D = \mbf{v}_C+\mbf{\omega}\times \vec{CD}
	\label{chapter6:球在平面上的运动-接触点的速度}
\end{equation}
当$\mbf{v}_D\neq \mbf{0}$时,摩擦力为滑动摩擦力,可以表示为
\begin{equation}
	\mbf{F} = -kN\mbf{u}
	\label{chapter6:球在平面上的运动-滑动摩擦力}
\end{equation}
其中$k$为摩擦系数,$\mbf{u}$为单位向量,方向沿着速度$\mbf{v}_D$的方向,即$\mbf{u}=\dfrac{\mbf{v}_D}{v_D}$。

根据质心运动定理可得
\begin{equation}
	m\dif{\mbf{v}_C}{t} = m\mbf{g}+\mbf{R}
	\label{chapter6:球在平面上的运动-质心运动方程}
\end{equation}
而球对质心的角动量为
\begin{equation}
	\mbf{L}_C = \frac25 ma^2\mbf{\omega}
\end{equation}
根据角动量定理可得
\begin{equation}
	\frac25ma^2\dif{\mbf{\omega}}{t} = \vec{CD}\times \mbf{R}
	\label{chapter6:球在平面上的运动-角动量定理}
\end{equation}

设$X_C, Y_C, Z_C$是球质心在空间系中的坐标,$F_X, F_Y$是摩擦力在空间系内的分量,则质心运动方程\eqref{chapter6:球在平面上的运动-质心运动方程}的分量形式为
\begin{subnumcases}{}
	\frac{\mathd^2X_C}{\mathd t^2} = \frac1m F_X \label{chapter6:球在平面上的运动-质心运动方程-1}\\
	\frac{\mathd^2Y_C}{\mathd t^2} = \frac1m F_Y \label{chapter6:球在平面上的运动-质心运动方程-2}\\
	\frac{\mathd^2Z_C}{\mathd t^2} = -g+\frac1m N \label{chapter6:球在平面上的运动-质心运动方程-3}
\end{subnumcases}
由于$Z_C=a$保持不变,所以根据方程\eqref{chapter6:球在平面上的运动-质心运动方程-3}可得$N=mg$。即平面的法向约束反力为常量,且不依赖于球在平面上是否滑动($\mbf{v}_D$是否为$\mbf{0}$)。

设$\omega_x, \omega_y, \omega_z$是角速度$\mbf{\omega}$在平动系中的分量,则角动量定理方程\eqref{chapter6:球在平面上的运动-角动量定理}的分量形式为
\begin{subnumcases}{}
	\dif{\omega_x}{t} = \frac{5}{2ma}F_Y \label{chapter6:球在平面上的运动-角动量定理-1}\\
	\dif{\omega_y}{t} = -\frac{5}{2ma}F_X \label{chapter6:球在平面上的运动-角动量定理-2}\\
	\dif{\omega_z}{t} = 0 \label{chapter6:球在平面上的运动-角动量定理-3}
\end{subnumcases}
方程\eqref{chapter6:球在平面上的运动-角动量定理-3}表明在球的运动过程中,球的角速度在竖直方向的分量为常数,且不依赖于球在平面上是否滑动($\mbf{v}_D$是否为$\mbf{0}$)。

首先考虑初始时刻有滑动,即$\mbf{v}_D\neq \mbf{0}$的情况。由于$N=mg$,由式\eqref{chapter6:球在平面上的运动-滑动摩擦力}可知滑动摩擦力的大小为常数,即$F=kmg$。下面我们将证明滑动摩擦力的方向也保持不变。将方程\eqref{chapter6:球在平面上的运动-接触点的速度}两端对时间求导,并利用方程\eqref{chapter6:球在平面上的运动-质心运动方程}和\eqref{chapter6:球在平面上的运动-角动量定理},并注意到$\mbf{R}=\mbf{N}+\mbf{F}=-m\mbf{g}+\mbf{F}$,可得
\begin{equation}
	\dif{\mbf{v}_D}{t} = \frac{7}{2m}\mbf{F}
	\label{chapter6:球在平面上的运动-关于v_D的方程-向量形式}
\end{equation}
由于$\mbf{v}_D=v_D\mbf{u}$,所以有
\begin{equation}
	\dif{v_D}{t}\mbf{u} + v_D\dif{\mbf{u}}{t} = -\frac72kg\mbf{u}
	\label{chapter6:球在平面上的运动-关于u的方程}
\end{equation}
在方程\eqref{chapter6:球在平面上的运动-关于u的方程}两端点乘$\mbf{u}$,并注意到$\mbf{u}$是单位向量,$\mbf{u}\cdot\dif{\mbf{u}}{t} = \dfrac12\dif{\mbf{u}^2}{t}=0$,所以可有
\begin{equation}
	\dif{v_D}{t} = -\frac72kg
	\label{chapter6:球在平面上的运动-关于v_D的方程}
\end{equation}
再将方程\eqref{chapter6:球在平面上的运动-关于v_D的方程}代入方程\eqref{chapter6:球在平面上的运动-关于u的方程}中可得
\begin{equation}
	\dif{\mbf{u}}{t} = \mbf{0}
\end{equation}
由此即说明向量$\mbf{u}$是常向量,进而得到滑动摩擦力是常向量
\begin{equation}
	\mbf{F} = -kmg\mbf{u}
\end{equation}
根据方程\eqref{chapter6:球在平面上的运动-关于v_D的方程}可以得到$D$点速度随时间的变化规律
\begin{equation}
	v_D = v_D(0)-\frac72kg
\end{equation}
如果用$\alpha$表示$D$点速度与空间系$OX$轴的夹角,则将质心运动方程\eqref{chapter6:球在平面上的运动-质心运动方程-1}和\eqref{chapter6:球在平面上的运动-质心运动方程-2}积分可得质心速度
\begin{equation}
\begin{cases}
	\ds \dot{X}_C(t) = -(kg\cos\alpha) t + \dot{X}_C(0) \\
	\ds \dot{Y}_C(t) = -(kg\sin\alpha) t + \dot{Y}_C(0)
\end{cases}
\end{equation}
和质心坐标
\begin{equation}
\begin{cases}
	\ds X_C(t) = -\frac12 (kg\cos\alpha) t^2 + \dot{X}_C(0)t + X_C(0) \\
	\ds Y_C(t) = -\frac12 (kg\sin\alpha) t^2 + \dot{Y}_C(0)t + Y_C(0)
\end{cases}
\label{chapter6:球在平面上的运动-质心运动}
\end{equation}
将角动量方程\eqref{chapter6:球在平面上的运动-角动量定理-1}和\eqref{chapter6:球在平面上的运动-角动量定理-2}积分可得
\begin{equation}
\begin{cases}
	\ds \omega_x(t) = \omega_x(0) - \frac{5kg\sin\alpha}{2a}t \\
	\ds \omega_y(t) = \omega_y(0) + \frac{5kg\cos\alpha}{2a}t
\end{cases}
\end{equation}

由质心运动规律\eqref{chapter6:球在平面上的运动-质心运动}可知,如果初始时刻质心速度和接触点速度\footnote{接触点速度是由质心速度和角速度共同决定的,见式\eqref{chapter6:球在平面上的运动-接触点的速度}。}不平行,则在滑动阶段球的质心沿着抛物线运动。这样的滑动运动一直持续到接触点速度降为$0$为止,即在时刻
\begin{equation}
	t = t_* = \frac{2v_D(0)}{7kg}
\end{equation}
时,$v_D=0$,球停止滑动开始滚动(同时也在转动)。因为$v_D=0$,根据式\eqref{chapter6:球在平面上的运动-关于v_D的方程-向量形式}可知,在滚动阶段摩擦力为零。据此,根据方程\eqref{chapter6:球在平面上的运动-质心运动方程-1}和\eqref{chapter6:球在平面上的运动-质心运动方程-2}可知滚动时质心沿直线运动。再根据方程\eqref{chapter6:球在平面上的运动-角动量定理-1}和\eqref{chapter6:球在平面上的运动-角动量定理-2},球在滚动时角速度$\mbf{\omega}$的大小和方向都不变,接触点$D$在平面上沿直线运动,而在球面上沿着垂直于角速度$\mbf{\omega}$的固定圆运动。

特别需要指出的一点是,质心运动的方向为方程\eqref{chapter6:球在平面上的运动-质心运动}所确定的抛物线的切线方向,如果该切线与球心的初始速度可以成钝角,则球可以向后返回。

