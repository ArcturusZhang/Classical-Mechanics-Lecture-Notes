\chapter{Hamilton动力学}

理想完整有势体系的Lagrange函数可以表示为
\begin{equation*}
	L = T(\mbf{q},\dot{\mbf{q}},t) - U(\mbf{q},\dot{\mbf{q}},t)
\end{equation*}
则有理想完整有势系的Lagrange方程为
\begin{equation*}
	\frac{\mathrm{d}}{\mathrm{d} t} \frac{\pl L}{\pl \dot{q}_\alpha} - \frac{\pl L}{\pl q_\alpha} = 0,\quad \alpha = 1,2,\cdots,s
\end{equation*}
利用体系的对称性可以简化求解上述常微分方程组。

Lagrange力学有如下不足:
\begin{enumerate}
	\item 运动学:速度$\dot{\mbf{q}} = \dfrac{\mathrm{d} \mbf{q}}{\mathrm{d} t}$作为坐目标衍生变量出现;
	\item 动力学:坐标、速度的初值$\mbf{q}(t_0),\dot{\mbf{q}}(t_0)$独立,不能体现坐标、速度在因果律上的独立性;
	\item 在位形空间描述运动,方程为二阶常微分方程组,同一点可能代表体系的不同状态(速度不同的状态),导致运动轨道的不同。
\end{enumerate}

1835年,Hamilton发表了两篇著名论文,将Lagrange方程化为更对称的一阶方程组,创立了Hamilton动力学,为量子力学、统计物理、量子场论奠定理论基础。

\section{正则方程}

广义动量和广义动量定理可以表示为
\begin{equation*}
	\begin{cases}
		\displaystyle p_\alpha = \frac{\pl L}{\pl \dot{q}_\alpha} \\[1.5ex]
		\displaystyle \dot{p}_\alpha = \frac{\pl L}{\pl q_\alpha}
	\end{cases}
\end{equation*}
由此定义的广义动量具有
\begin{equation*}
	p_\alpha = p_\alpha(\mbf{q},\dot{\mbf{p}},t)
\end{equation*}
的形式,则有
\begin{equation*}
	\dot{\mbf{q}} = \dot{\mbf{q}}(\mbf{q},\mbf{p},t)
\end{equation*}
由此,可作力学状态参量变换$\mbf{q},\dot{\mbf{q}} \to \mbf{q},\mbf{p}$,找到新的特征函数,通过对$\mbf{p},\mbf{q}$的偏导数生成力学方程。

\subsection{Legendre变换}

设有函数$f(x,y)$,考虑
\begin{align*}
	\mathrm{d} f & = \frac{\pl f}{\pl x} \mathrm{d}x + \frac{\pl f}{\pl y} \mathrm{d} y = u\mathrm{d} x + v\mathrm{d} y \\
	& = \mathrm{d} (ux) - x\mathrm{d} u + v\mathrm{d} y
\end{align*}
此处
\begin{equation*}
	\begin{cases}
		\displaystyle u = \frac{\pl f}{\pl x} = u(x,y) \\[1.5ex]
		\displaystyle v = \frac{\pl f}{\pl y} = v(x,y)
	\end{cases}
\end{equation*}
由此定义以$u,y$为自变量的新函数
\begin{equation}
	g = ux - f = ux(u,y) - f(x(u,y),y) = g(u,y)
\end{equation}
则有
\begin{equation*}
	\mathrm{d} g = x\mathrm{d} u - v\mathrm{d}y = \frac{\pl g}{\pl u} \mathrm{d} u + \frac{\pl g}{\pl y} \mathrm{d} y
\end{equation*}
此时有
\begin{equation}
	\begin{cases}
		\displaystyle x = \frac{\pl g}{\pl u} = x(u,y) \\[1.5ex]
		\displaystyle v = - \frac{\pl g}{\pl y} = \bar{v}(u,y)
	\end{cases}
	\label{chp3:Legendre变换}
\end{equation}
式\eqref{chp3:Legendre变换}给出的变换称为{\heiti Legendre变换}。

\subsection{Hamilton正则方程}

Lagrange函数具有形式
\begin{equation*}
	L = L(\mbf{q},\dot{\mbf{q}},t)
\end{equation*}
考虑做Legendre变换
\begin{align*}
	\mathrm{d} L & = \sum_{\alpha=1}^s \frac{\pl L}{\pl \dot{q}_\alpha} \mathrm{d} \dot{q}_\alpha + \sum_{i=1}^s \frac{\pl L}{\pl q_\alpha} + \frac{\pl L}{\pl t} \mathrm{d} t \\
	& = \sum_{\alpha=1}^s p_\alpha \mathrm{d} \dot{q}_\alpha + \sum_{\alpha=1}^s \dot{p}_\alpha \mathrm{d} q_\alpha - \dot{H} \mathrm{d} t \\
	& = \mathrm{d} \left(\sum_{\alpha=1}^s p_\alpha \dot{q}_\alpha \right) - \sum_{\alpha=1}^s \dot{q}_\alpha \mathrm{d} p_\alpha + \sum_{\alpha=1}^s \dot{p}_\alpha \mathrm{d}q_\alpha - \dot{H} \mathrm{d} t
\end{align*}
由此可有
\begin{equation}
	H = \sum_{\alpha=1}^s p_\alpha \dot{q}_\alpha - L = \sum_{\alpha=1}^s p_\alpha \dot{q}_\alpha(\mbf{q},\mbf{p},t) - L(\mbf{q},\dot{\mbf{q}}(\mbf{q},\mbf{p},t),t) = H(\mbf{q},\mbf{p},t)
\end{equation}
即
\begin{equation*}
	H = \sum_{\alpha=1}^s \frac{\pl H}{\pl p_\alpha} \mathrm{d} p_\alpha + \sum_{\alpha=1}^s \frac{\pl H}{\pl q_\alpha} \mathrm{d} q_\alpha + \frac{\pl H}{\pl t} \mathrm{d} t = \sum_{\alpha=1}^s \dot{q}_\alpha \mathrm{d} p_\alpha - \sum_{\alpha=1}^s \dot{p}_\alpha \mathrm{d} q_\alpha + \dot{H} \mathrm{d} t
\end{equation*}
由此可得
\begin{equation}
	\begin{cases}
		\dot{q}_\alpha = \dfrac{\pl H}{\pl p_\alpha} \\[1.5ex]
		\dot{p}_\alpha = -\dfrac{\pl H}{\pl q_\alpha} 
	\end{cases}
	\alpha = 1,2,\cdots,s
	\label{Hamilton正则方程}
\end{equation}
以及
\begin{equation}
	\dot{H} = \frac{\pl H}{\pl t}
\end{equation}
式\eqref{Hamilton正则方程}称为{\heiti Hamilton正则方程},$\mbf{q},\mbf{p}$称为{\heiti 正则变量}。函数$H = H(\mbf{q},\mbf{p},t)$称为{\heiti Hamilton函数},其为以正则变量表示的广义能量。

由正则变量张成了一个$2s$维空间,称为{\heiti 相空间}。相空间中的点称为{\heiti 相点},代表体系的力学状态。相点在相空间运动而描述的轨道称为{\heiti 相轨道},反映了体系状态的演化。

正则方程是由$2s$个一阶常微分方程构成的常微分方程组,决定了相点的速度。给定正则变量的$2s$个初始值,可以唯一确定正则变量的时间演化关系。

\begin{example}[一维谐振子]
写出一维谐振子的正则方程和相空间轨迹。
\end{example}
\begin{solution}
一维谐振子的Lagrange函数为
\begin{equation*}
	L = \frac12 m\dot{q}^2 - \frac12 kq^2
\end{equation*}
考虑$p = \dfrac{\pl L}{\pl \dot{q}} = m\dot{q}$,则有
\begin{equation*}
	\dot{q} = \frac{p}{m}
\end{equation*}
所以Hamilton函数为
\begin{equation*}
	H = p\dot{q} - L = \frac{p^2}{2m} + \frac12 kq^2
\end{equation*}
正则方程为
\begin{equation*}
	\begin{cases}
		\dot{q} = \dfrac{\pl H}{\pl p} = \dfrac{p}{m} \\[1.5ex]
		\dot{p} = -\dfrac{\pl H}{\pl q} = -kq
	\end{cases}
\end{equation*}
消去广义动量$p$即可得到系统的动力学方程
\begin{equation*}
	m\ddot{q} + kq = 0
\end{equation*}
由于
\begin{equation*}
	\dot{H} = \frac{\pl H}{\pl t} = 0
\end{equation*}
所以$H = E(\text{常数})$,相空间轨迹即为
\begin{equation*}
	\frac{p^2}{2mE} + \frac{q^2}{2E/k} = 1
\end{equation*}
是一个椭圆。
\end{solution}

\begin{example}[电磁场中带电粒子的Hamilton函数]
求电磁场中带电粒子的Hamilton函数。
\end{example}
\begin{solution}
电磁场中带电粒子的Lagrange函数为
\begin{equation*}
	L(\mbf{r},\dot{\mbf{r}},t) = \frac12 m \dot{\mbf{r}}^2 + e\mbf{A}(\mbf{r},t) \cdot \dot{\mbf{r}} - e\phi(\mbf{r},t)
\end{equation*}
则有
\begin{equation*}
	\mbf{p} = \frac{\pl L}{\pl \dot{\mbf{r}}} = m\dot{\mbf{r}} + e\mbf{A}
\end{equation*}
所以
\begin{equation*}
	\dot{\mbf{r}} = \frac{\mbf{p} - e\mbf{A}}{m}
\end{equation*}
由此可得Hamilton函数
\begin{equation*}
	H = \mbf{p} \cdot \dot{\mbf{r}} - L = \frac{1}{2m} (\mbf{p}-e\mbf{A})^2 + e\phi
\end{equation*}
\end{solution}

\subsection{守恒定律}

根据正则方程,如果Hamilton函数中不显含坐标$q_\alpha$,则有
\begin{equation*}
	\dot{p}_\alpha = -\frac{\pl H}{\pl q_\alpha} = 0
\end{equation*}
即有与该坐标共轭的动量$p_\alpha$守恒。如果Hamilton函数中不显含时间$t$,则有
\begin{equation*}
	\dot{H} = \frac{\pl H}{\pl t} = 0
\end{equation*}
即有广义能量守恒。

在Hamilton力学中,体系的对称性表现为$H$的变换不变性。

\subsection{正则方程的矩阵形式}

记
\begin{equation*}
	\begin{array}{l}
	\displaystyle \mbf{\xi} = \begin{pmatrix} \mbf{q} \\ \mbf{p} \end{pmatrix} = \begin{pmatrix} q_1 \\ \vdots \\ q_s \\ p_1 \\ \vdots \\ p_s \end{pmatrix},\quad \mbf{J} = \begin{pmatrix} \mbf{0} & \mbf{I} \\ -\mbf{I} & \mbf{0} \end{pmatrix} = \begin{pmatrix} 0 & \cdots & 0 & 1 & \cdots & 0 \\ \vdots & & \vdots & \vdots & & \vdots \\ 0 & \cdots & 0 & 0 & \cdots & 1 \\ -1 & \cdots & 0 & 0 & \cdots & 0 \\ \vdots & & \vdots & \vdots & & \vdots \\ 0 & \cdots & -1 & 0 & \cdots & 0 \end{pmatrix} \\
	\displaystyle \frac{\pl H}{\pl \mbf{\xi}} = \begin{pmatrix} \dfrac{\pl H}{\pl \mbf{q}} \\[1.5ex] \dfrac{\pl H}{\pl \mbf{p}} \end{pmatrix} = \begin{pmatrix} \dfrac{\pl H}{\pl q_1} \\[1.5ex] \vdots \\[1.5ex] \dfrac{\pl H}{\pl q_s} \\[1.5ex] \dfrac{\pl H}{\pl p_1} \\[1.5ex] \vdots \\[1.5ex] \dfrac{\pl H}{\pl p_s} \end{pmatrix}
	\end{array}
\end{equation*}
则正则方程可以表示为
\begin{equation}
	\dot{\mbf{\xi}} = \mbf{J} \frac{\pl H}{\pl \mbf{\xi}}
	\label{正则方程的辛形式}
\end{equation}
式\eqref{正则方程的辛形式}称为{\heiti 正则方程的矩阵形式}或{\heiti 正则方程的辛(sympletic)形式}。

可通过直接计算的方式验证矩阵$\mbf{J}$满足如下性质:
\begin{equation}
	\mbf{J}^2 = -\mbf{I}_{2s},\quad \mbf{J}^{\mathrm{T}} = \mbf{J}^{-1} = -\mbf{J},\quad \det \mbf{J} = 1
\end{equation}

\section{泛函与变分}

\subsection{泛函}

依赖于函数关系的变量,即函数之函数,称为{\heiti 泛函},一般记作$J = J[y(x)]$。常见的泛函一般具有形式
\begin{equation*}
	J[y(x)] = \int_a^b F(x,y,y') \mathrm{d} x
\end{equation*}

\subsection{变分}

自变函数发生无限小变动后的新函数与原函数之差,称为{\heiti 自变函数的变分},即
\begin{equation*}
	\delta y(x) = \tilde{y}(x) - y(x)
\end{equation*}
当$y(x)$可微时,总是假定$\delta y(x)$也是可微的。考虑自变函数导数的变分,即
\begin{equation*}
	\delta \frac{\mathrm{d}y}{\mathrm{d}x}(x) = \frac{\mathrm{d}\tilde{y}}{\mathrm{d}x}(x) - \frac{\mathrm{d}y}{\mathrm{d}x}(x) = \frac{\mathrm{d}}{\mathrm{d}x}\left[\tilde{y}(x) - y(x)\right] = \frac{\mathrm{d}}{\mathrm{d} x} \delta y(x)
\end{equation*}
因此,对任意函数$f$均有
\begin{equation}
	\delta \frac{\mathrm{d}}{\mathrm{d} x} f = \frac{\mathrm{d}}{\mathrm{d} x} \delta f
\end{equation}
下面考虑函数的变分,首先有
\begin{equation*}
	\Delta F = F(x,y+\delta y,y'+\delta y') - F(x,y,y')
\end{equation*}
其中一级变动称为函数的变分,即
\begin{equation*}
	\delta F = \frac{\pl F}{\pl y} \delta y + \frac{\pl F}{\pl y'} \delta y'
\end{equation*}
现在考虑泛函的变分,首先有
\begin{equation*}
	\Delta J = J\left[y(x)+\delta y(x)\right] - J\left[y(x)\right]
\end{equation*}
其中一级变动即为泛函的变分,即
\begin{align*}
	\delta J & = \int_a^b \delta F(x,y,y') \mathrm{d} x = \int_a^b \left(\frac{\pl F}{\pl y} \delta y + \frac{\pl F}{\pl y'} \delta y'\right) \mathrm{d} x \\
	& = \int_a^b \left[\frac{\mathrm{d}}{\mathrm{d} x} \left(\frac{\pl F}{\pl y'} \delta y\right) + \left(\frac{\pl F}{\pl y} - \frac{\mathrm{d}}{\mathrm{d} x} \frac{\pl F}{\pl y'}\right) \delta y \right] \mathrm{d} x \\
	& = \frac{\pl F}{\pl y'} \delta y \bigg|_a^b + \int_a^b \left(\frac{\pl F}{\pl y} - \frac{\mathrm{d}}{\mathrm{d} x} \frac{\pl F}{\pl y'}\right) \delta y \mathrm{d} x
\end{align*}

\subsection{变分问题(泛函极值问题)}

变分问题始于最速降线问题(1696,J. Bernoulli):给定重力场中两端点,求从一点到另一点用时最短的光滑轨道。即求自变函数$y(x)$,使泛函$\displaystyle J[y(x)] = \int_a^b F(x,y,y') \mathrm{d} x$在边界条件$y(a) = y_A,y(b) = y_B$下取得极值。

可以证明,泛函极值的必要条件即为
\begin{equation}
	\delta J = 0
\end{equation}

\subsection{Euler-Lagrange方程}

如果给定边界条件,则有
\begin{equation}
	\delta y(a) = \delta y(b) = 0
	\label{变分问题边界条件}
\end{equation}
所以有变分
\begin{equation*}
	\delta J = \frac{\pl F}{\pl y'} \delta y \bigg|_a^b + \int_a^b \left(\frac{\pl F}{\pl y} - \frac{\mathrm{d}}{\mathrm{d} x} \frac{\pl F}{\pl y'}\right) \delta y \mathrm{d} x = \int_a^b \left(\frac{\pl F}{\pl y} - \frac{\mathrm{d}}{\mathrm{d} x} \frac{\pl F}{\pl y'}\right) \delta y \mathrm{d} x
\end{equation*}
由于$\delta y$是任意的,可有
\begin{equation}
	\frac{\mathrm{d}}{\mathrm{d} x} \frac{\pl F}{\pl y'} - \frac{\pl F}{\pl y} = 0
	\label{Euler-Lagrange方程}
\end{equation}
式\eqref{Euler-Lagrange方程}称为{\heiti Euler-Lagrange方程}\footnote{可以看到,Euler-Lagrange方程与理想完整有势系的Lagrange方程\eqref{理想完整系的Lagrange方程}具有完全相同的形式。因此可以想见,理想完整有势系的Lagrange方程也是某种变分原理(即{\heiti Hamilton原理})的结果。}。

下面考虑
\begin{align*}
	\frac{\mathrm{d}}{\mathrm{d} x} \left(\frac{\pl F}{\pl y'} y' - F\right) & = \frac{\mathrm{d}}{\mathrm{d} x} \left(\frac{\pl F}{\pl y'}y'\right) - \left(\frac{\pl F}{\pl y'} y'' + \frac{\pl F}{\pl y} y' + \frac{\pl F}{\pl x}\right) \\
	& = \left(\frac{\mathrm{d}}{\mathrm{d} x} \frac{\pl F}{\pl y'} - \frac{\pl F}{\pl y}\right) y' - \frac{\pl F}{\pl x} = -\frac{\pl F}{\pl x}
\end{align*}
如果函数$F$不显含$x$,则Euler-Lagrange方程有初积分
\begin{equation}
	\frac{\pl F}{\pl y'}y' - F = C
\end{equation}

\begin{example}[最速降线问题]
给定重力场中两端点,求从一点到另一点用时最短的光滑轨道。
\end{example}
\begin{solution}
所求的时间可以表示为
\begin{equation*}
	T = \frac{1}{\sqrt{2g}} \int_a^b \sqrt{\frac{1+y'^2}{y}} \mathrm{d} x
\end{equation*}
即此时
\begin{equation*}
	F(x,y,y') = \sqrt{\frac{1+y'^2}{y}}
\end{equation*}
由于$F$不显含$x$,利用Euler-Lagrange方程的初积分
\begin{equation*}
	\frac{\pl F}{\pl y'} y' - F = -\frac{1}{\sqrt{y(1+y'^2)}} = -\frac{1}{\sqrt{2C_1}}
\end{equation*}
所以有
\begin{equation*}
	\frac{\mathrm{d} y}{\mathrm{d} x} = \sqrt{\frac{2C_1 - y}{y}}
\end{equation*}
做换元$y = C_1(1-\cos \theta)$,可得
\begin{equation*}
	\begin{cases}
		x = C_1(\theta - \sin \theta) + C_2 \\
		y = C_1(1-\cos \theta)
	\end{cases}
\end{equation*}
这是一条倒置的旋轮线(摆线),如图\ref{最速降线:摆线}所示,其中的常数$C_1,C_2$由边界条件决定。
\begin{figure}[htb]
\centering
\begin{asy}
	texpreamble("\usepackage{xeCJK}");
	texpreamble("\setCJKmainfont{SimSun}");
	usepackage("amsmath");
	usepackage("siunitx");
	import graph;
	import math;
	size(300);
	//最速降线
	pair O,P;
	real x,y,rp;
	path p;
	O = (0,0);
	x = 7.4;
	y = 2.5;
	draw(Label("$x$",EndPoint),0.1*x*dir(180)--x*dir(0),Arrow);
	draw(Label("$y$",EndPoint),0.2*y*dir(90)--y*dir(-90),Arrow);
	label("$O$",O,SW);
	pair cycloid(real t){
		return (t-sin(t),cos(t)-1);
	}
	rp = 0.7*2*pi;
	p = graph(cycloid,0,rp);
	draw(p,red+linewidth(1bp));
	label("$A$",O,NW);
	P = cycloid(rp);
	label("$B$",P,SE);
	p = graph(cycloid,rp,2*pi);
	draw(p);
	dot(O);
	dot(P);
\end{asy}
\caption{最速降线}
\label{最速降线:摆线}
\end{figure}
\end{solution}

\begin{example}[球面短程线(测地线)]
求球面上连接任意两点之间的最短曲线。
\end{example}
\begin{solution}
在球面上取球面坐标,则球面上的曲线可以表示为
\begin{equation*}
	\begin{cases}
		r = R \\
		\theta = \theta(\phi)
	\end{cases}
\end{equation*}
根据球坐标系的距离元素表达式可得球面上曲线的弧长元素为
\begin{equation*}
	\mathrm{d} s = R\sqrt{(\mathrm{d} \theta)^2 + \sin^2 \theta (\mathrm{d} \phi)^2} = R \sqrt{\theta'^2 + \sin^2 \theta} \mathrm{d} \phi
\end{equation*}
所以,球面上曲线的弧长为
\begin{equation*}
	s = R \int_{\phi_A}^{\phi_B} \sqrt{\theta'^2 + \sin^2 \theta} \mathrm{d} \phi
\end{equation*}
即此时
\begin{equation*}
	F(\phi,\theta,\theta') = \sqrt{\theta'^2 + \sin^2 \theta}
\end{equation*}
由于$F$不显含$\phi$,利用Euler-Lagrange方程的初积分
\begin{equation*}
	\frac{\pl F}{\pl \theta'} \theta' - F = -\frac{\sin^2 \theta}{\sqrt{\theta'^2 + \sin^2 \theta}} = -\cos \theta_0\text{(常数)}
\end{equation*}
由此可得
\begin{equation*}
	\frac{\mathrm{d} \theta}{\mathrm{d} \phi} = \sin^2 \theta \sqrt{\tan^2 \theta_0 - \cot^2 \theta}
\end{equation*}
解得
\begin{equation}
	\phi - \arccos(-\cot \theta_0 \cot\theta) = \phi_0\text{(常数)}
	\label{chp3:球面大圆的参数方程}
\end{equation}
整理可得
\begin{equation*}
	\cos \phi \cos \phi_0 + \sin \phi \sin \phi_0 + \cot \theta \cot \theta_0 = 0
\end{equation*}
在上式两端乘$R\sin \theta \sin \theta_0$,并考虑到球坐标系的坐标转换关系,可得
\begin{equation*}
	x \sin \theta_0 \cos \phi_0 + y\sin \theta_0 \sin \phi_0 + z\cos \theta_0 = 0
\end{equation*}
此处,记$\mbf{e}_n = \sin \theta_0 \cos \phi_0 \mbf{e}_1 + \sin \theta_0 \sin \phi_0 \mbf{e}_2 + \cos \theta_0 \mbf{e}_3$,则上式可记作
\begin{equation*}
	\mbf{r} \cdot \mbf{e}_n = 0
\end{equation*}
边界条件为
\begin{equation*}
	\mbf{r}_A \cdot \mbf{e}_n = \mbf{r}_B \cdot \mbf{e}_n = 0
\end{equation*}
由此,短程线即为$OAB$平面与球面的交线之间较短的弧,即大圆弧。此处$\mbf{e}_n$即为$OAB$平面的法向量。
\begin{figure}[htb]
\centering
\begin{asy}
	texpreamble("\usepackage{xeCJK}");
	texpreamble("\setCJKmainfont{SimSun}");
	usepackage("amsmath");
	import graph;
	import math;
	size(200);
	//球面短程线示意图
	real hd,hdd;
	pair O,i,j,k,en;
	real r,theta0,phi0;
	O = (0,0);
	r = 2;
	label("$O$",O,NW);
	draw(shift(O)*scale(r)*unitcircle);
	i = (-sqrt(2)/4,-sqrt(14)/12);
	j = (sqrt(14)/4,-sqrt(2)/12);
	k = (0,2*sqrt(2)/3);
	theta0 = pi/6;
	phi0 = 1.5;
	en = sin(theta0)*cos(phi0)*i+sin(theta0)*sin(phi0)*j+cos(theta0)*k;
	dot(r*en,red);
	draw(O--r*en,red+dashed);
	draw(Label("$\boldsymbol{e}_n$",EndPoint,W,black),r*en--r*en+en,red,Arrow);
	
	pair cir(real phi){
		real theta,u,x,y,z;
		if(phi==phi0){
			theta = pi/2;
		}
		else{
			u = -1./cos(phi-phi0)/tan(theta0);
			if(u<0){
				theta = pi-atan(-u);
			}
			else{
				theta = atan(u);
			}
		}
		x = r*sin(theta)*cos(phi);
		y = r*sin(theta)*sin(phi);
		z = r*cos(theta);
		return x*i+y*j+z*k;
	}
	draw(graph(cir,pi/2,3*pi/2),blue+dashed);
	draw(graph(cir,3*pi/2,2*pi+pi/2),blue);
	real theta,theta2;
	theta = -0.1*pi;
	theta2 = 0.3*pi;
	draw(graph(cir,theta,theta2),blue+linewidth(1.5bp));
	draw(Label("$\boldsymbol{r}_A$",MidPoint,Relative(E)),O--cir(theta),dashed,Arrow);
	label("$A$",cir(theta),SW);
	dot(cir(theta));
	draw(Label("$\boldsymbol{r}_B$",MidPoint,Relative(W)),O--cir(theta2),dashed,Arrow);
	label("$B$",cir(theta2),S);
	dot(cir(theta2));
	theta = 0.13*pi;
	draw(Label("$\boldsymbol{r}$",MidPoint,Relative(E)),O--cir(theta),dashed,Arrow);
	dot(cir(theta));
\end{asy}
\caption{球面短程线示意图}
\label{球面短程线示意图}
\end{figure}
\end{solution}

\subsection{多元泛函极值问题}

考虑多元泛函
\begin{equation*}
	J[y_1(x),y_2(x),\cdots,y_s(x)] = \int_a^b F(x;y_1,y_2,\cdots,y_s;y'_1,y'_2,\cdots,y'_s) \mathrm{d} x
\end{equation*}
在固定边界条件
\begin{equation*}
	\delta y_\alpha(a) = \delta y_\beta(b) = 0,\quad \alpha = 1,2,\cdots,s
\end{equation*}
下,泛函的变分为
\begin{align*}
	\delta J & = \delta \int_a^b F(x;y_1,y_2,\cdots,y_s;y'_1,y'_2,\cdots,y'_s) \mathrm{d} x \\
	& = \int_a^b \delta F(x;y_1,y_2,\cdots,y_s;y'_1,y'_2,\cdots,y'_s) \mathrm{d} x \\
	& = \int_a^b \left(\sum_{\alpha=1}^s \frac{\pl F}{\pl y_\alpha} \delta y_\alpha + \sum_{\alpha=1}^s \frac{\pl F}{\pl y'_\alpha} \delta y'_\alpha\right) \mathrm{d} x \\
	& = \int_a^b \left[\frac{\mathrm{d}}{\mathrm{d} x} \left(\sum_{\alpha=1}^s \frac{\pl F}{\pl y'_\alpha} \delta y_\alpha\right) + \sum_{\alpha=1}^s \left(\frac{\pl F}{\pl y_\alpha} - \frac{\mathrm{d}}{\mathrm{d} x} \frac{\pl F}{\pl y'_\alpha} \right) \delta y_\alpha\right] \mathrm{d} x \\
	& = \sum_{\alpha=1}^s \frac{\pl F}{\pl y'_\alpha} \delta y_\alpha \bigg|_a^b + \int_a^b \sum_{\alpha=1}^s \left(\frac{\pl F}{\pl y_\alpha} - \frac{\mathrm{d}}{\mathrm{d} x} \frac{\pl F}{\pl y'_\alpha} \right) \delta y_\alpha \mathrm{d} x \\
	& = \int_a^b \sum_{\alpha=1}^s \left(\frac{\pl F}{\pl y_\alpha} - \frac{\mathrm{d}}{\mathrm{d} x} \frac{\pl F}{\pl y'_\alpha} \right) \delta y_\alpha \mathrm{d} x = 0
\end{align*}
由于$\delta y_\alpha\,(\alpha=1,2,\cdots,s)$之间相互独立,可得
\begin{equation}
	\frac{\mathrm{d}}{\mathrm{d} x} \frac{\pl F}{\pl y'_\alpha} - \frac{\pl F}{\pl y_\alpha} = 0,\quad \alpha = 1,2,\cdots,s
	\label{多元泛函的Euler-Lagrange方程}
\end{equation}
当$F$不显含$x$时,式\eqref{多元泛函的Euler-Lagrange方程}同样有初积分
\begin{equation*}
	\sum_{\alpha=1}^s \frac{\pl F}{\pl y'_\alpha} y'_\alpha - F = C
\end{equation*}

\section{力学变分原理}

d'Alambert原理的形式为
\begin{equation*}
	\sum_{i=1}^n(\mbf{F}_i - m_i \ddot{\mbf{r}}) \cdot \delta \mbf{r}_i = 0
\end{equation*}
式中,$\delta \mbf{r}_i$为$t$时刻实际运动所达位形处的虚位移,即由约束限定的位矢等时变分,因而与实际运动$\mbf{r}_i(t),\,i=1,2,\cdots,n$相对,称
\begin{equation*}
	\tilde{\mbf{r}}_i(t) = \mbf{r}_i(t) + \delta \mbf{r}_i(t),\quad i=1,2,\cdots,n
\end{equation*}
为虚拟运动。力学变分原理指出实际运动与近邻虚拟运动相比所具有的极值性。

\subsection{Hamilton原理}

考虑d'Alambert原理
\begin{align*}
	\sum_{i=1}^n \left(\mbf{F}_i - m_i \frac{\mathrm{d} \dot{\mbf{r}}_i}{\mathrm{d} t}\right) \cdot \delta \mbf{r}_i & = \sum_{i=1}^n \left(\mbf{F}_i \cdot \delta \mbf{r}_i + m_i \dot{\mbf{r}}_i \cdot \delta \dot{\mbf{r}}_i\right) - \frac{\mathrm{d}}{\mathrm{d} t} \sum_{i=1}^n m_i \dot{\mbf{r}}_i \cdot \delta \mbf{r}_i \\
	& = \delta T + \sum_{i=1}^n \mbf{F}_i \cdot \delta \mbf{r}_i - \frac{\mathrm{d}}{\mathrm{d} t} \sum_{i=1}^n \frac{\pl T}{\pl \dot{\mbf{r}}_i} \cdot \delta \mbf{r}_i = 0
\end{align*}
将上式对时间积分,可得
\begin{equation}
	\int_{t_1}^{t_2} \left(\delta T + \sum_{i=1}^n \mbf{F}_i \cdot \delta \mbf{r}_i\right) \mathrm{d} t = \sum_{i=1}^n \frac{\pl T}{\pl \dot{\mbf{r}}_i} \cdot \delta \mbf{r}_i \bigg|_{t_1}^{t_2}
	\label{d'Alambert原理的积分形式}
\end{equation}
式\eqref{d'Alambert原理的积分形式}表示的关系为{\heiti d'Alambert原理的积分形式}。对于有势系,即有广义势$U = U(\mbf{r},\dot{\mbf{r}},t)$,则有
\begin{equation*}
	\mbf{F}_i = \frac{\mathrm{d}}{\mathrm{d} t} \frac{\pl U}{\pl \dot{\mbf{r}}_i} - \frac{\pl U}{\pl \mbf{r}_i},\quad i=1,2,\cdots,n
\end{equation*}
所以
\begin{align*}
	\sum_{i=1}^n \mbf{F}_i \cdot \delta \mbf{r}_i & = \frac{\mathrm{d}}{\mathrm{d} t} \left(\sum_{i=1}^n \frac{\pl U}{\pl \dot{\mbf{r}}_i} \cdot \delta \mbf{r}_i\right) - \sum_{i=1}^n \left(\frac{\pl U}{\pl \dot{\mbf{r}}_i} \cdot \delta \dot{\mbf{r}}_i + \frac{\pl U}{\pl \mbf{r}_i} \cdot \delta \mbf{r}_i \right) \\
	& = \frac{\mathrm{d}}{\mathrm{d} t} \left(\sum_{i=1}^n \frac{\pl U}{\pl \dot{\mbf{r}}_i} \cdot \delta \mbf{r}_i\right) - \delta U
\end{align*}
由此可有
\begin{equation}
	\int_{t_1}^{t_2} \delta L \mathrm{d} t = \sum_{i=1}^n \frac{\pl L}{\pl \dot{\mbf{r}}_i} \cdot \delta \mbf{r}_i \bigg|_{t_1}^{t_2}
	\label{有势系d'Alambert原理的积分形式}
\end{equation}
式\eqref{有势系d'Alambert原理的积分形式}表示的关系为{\heiti 有势系d'Alambert原理的积分形式}。

如果采用固定边界条件,即
\begin{equation*}
	\mbf{r}_i(t_1) = \mbf{r}_{i1},\mbf{r}_i(t_2) = \mbf{r}_{i2},\quad i=1,2,\cdots,n
\end{equation*}
则有
\begin{equation*}
	\delta \mbf{r}_i(t_1) = \delta \mbf{r}_i(t_2) = 0,\quad i = 1,2,\cdots,n
\end{equation*}
因此,对于有势系满足
\begin{equation*}
	\int_{t_1}^{t_2} \delta L \mathrm{d} t = 0
\end{equation*}
由此,可定义{\heiti Hamilton作用量}
\begin{equation}
	S = \int_{t_1}^{t_2} L \mathrm{d} t
	\label{Hamilton作用量}
\end{equation}
则对于有势系有
\begin{equation}
	\delta S = \delta\int_{t_1}^{t_2} L \mathrm{d} t = 0
	\label{Hamilton原理}
\end{equation}
式\eqref{Hamilton原理}表示的关系称为{\heiti Hamilton原理}\footnote{有些文献中称为{\heiti 最小作用量原理}。}。即与始末时间和位形均相同的虚拟运动相比,有势系实际运动使体系的作用量取驻值。对于一般体系,则有
\begin{equation}
	\int_{t_1}^{t_2} \left(\delta T + \sum_{i=1}^n \mbf{F}_i \cdot \delta \mbf{r}_i\right) \mathrm{d} t = 0
	\label{一般体系的Hamilton原理}
\end{equation}
称为{\heiti 一般体系的Hamilton原理}。对于完整系,可有
\begin{equation*}
	\sum_{i=1}^n \mbf{F}_i \cdot \delta \mbf{r}_i = \sum_{\alpha=1}^s Q_\alpha \delta q_\alpha
\end{equation*}
故有{\heiti 一般完整系的Hamilton原理}
\begin{equation}
	\int_{t_1}^{t_2} \left(\delta T + \sum_{\alpha=1}^s Q_\alpha \delta q_\alpha\right) \mathrm{d} t = 0
	\label{一般完整系的Hamilton原理}
\end{equation}

Hamilton原理是积分变分原理,与此相对应的d'Alambert原理是微分变分原理。变分原理具有如下特点:
\begin{enumerate}
	\item 不涉及具体坐标选取,形式统一;
	\item 作用量是标量,形式简单,高度概括;
	\item 易于推广至无限自由度体系(场)和非力学体系。
\end{enumerate}

\subsection{位形空间与Lagrange方程}

对于完整系,可取广义坐标$q_\alpha,\,\,\alpha = 1,2,\cdots,s$通过广义坐标变换
\begin{equation*}
	\mbf{r}_i = \mbf{r}_i(\mbf{q},t),\quad i=1,2,\cdots,n
\end{equation*}
使得完整约束自动满足。完整有势系下的Hamilton作用量可以表示为
\begin{equation}
	S[\mbf{q}(t)] = \int_{t_1}^{t_2} L(\mbf{q},\dot{\mbf{q}},t) \mathrm{d} t
\end{equation}
在固定边界条件$\delta \mbf{q}(t_1) = \delta \mbf{q}(t_2) = 0$下,根据Hamilton原理以及Euler-Lagrange方程,可有
\begin{equation}
	\frac{\mathrm{d}}{\mathrm{d} t} \frac{\pl L}{\pl \dot{q}_\alpha} - \frac{\pl L}{\pl q_\alpha} = 0,\quad \alpha = 1,2,\cdots,s
\end{equation}

Lagrange函数$L$具有规范不定性,即作{\heiti 定域规范变换}
\begin{equation}
	\tilde{L} = L + \frac{\mathrm{d} f}{\mathrm{d} t}(\mbf{q},t)
\end{equation}
可有
\begin{equation*}
	\delta \tilde{S} = \delta S + \delta f(\mbf{q},t)\big|_{t_1}^{t_2} = \delta S + \sum_{\alpha=1}^s \frac{\pl f}{\pl q_\alpha} \delta q_\alpha \bigg|_{t_1}^{t_2} = \delta S
\end{equation*}
即在定域规范变换下,Hamilton作用量的变分不变,进而体系的Lagrange方程也不变。因此规范变换下,体系的运动规律不变,此即为体系的{\heiti 规范不变性}。

$L$的不定性来自广义势的规范不定性,即$\tilde{U}(\mbf{q},\dot{\mbf{q}},t) = U(\mbf{q},\dot{\mbf{q}},t) - \dfrac{\mathrm{d} f}{\mathrm{d} t}(\mbf{q},t)$与$U(\mbf{q},\dot{\mbf{q}},t)$表示的是同一个势场。下面证明这一点,首先考虑$f = f(\mbf{q},t)$,则有
\begin{equation*}
	\dfrac{\mathrm{d} f}{\mathrm{d} t}(\mbf{q},t) = \sum_{\alpha=1}^n \frac{\pl f}{\pl q_\alpha} \dot{q}_\alpha + \frac{\pl f}{\pl t} = \dot{f}(\mbf{q},\dot{\mbf{q}},t)
\end{equation*}
以及
\begin{equation*}
	\mathrm{d} f = \sum_{\alpha=1}^n \frac{\pl f}{\pl q_\alpha} \mathrm{d} q_\alpha + \frac{\pl f}{\pl t} \mathrm{d} t
\end{equation*}
在上式两端求对时间$t$的导数,可有
\begin{align}
	\mathrm{d} \dot{f} & = \frac{\mathrm{d}}{\mathrm{d} t} \left(\sum_{\alpha=1}^n \frac{\pl f}{\pl q_\alpha} \mathrm{d} q_\alpha\right) + \frac{\mathrm{d}}{\mathrm{d} t} \left(\frac{\pl f}{\pl t}\right) \mathrm{d} t \nonumber \\
	& = \sum_{\alpha=1}^n \frac{\mathrm{d}}{\mathrm{d} t} \frac{\pl f}{\pl q_\alpha} \mathrm{d} q_\alpha + \sum_{\alpha=1}^n \frac{\pl f}{\pl q_\alpha} \mathrm{d} \dot{q}_\alpha + \frac{\mathrm{d}}{\mathrm{d} t} \left(\frac{\pl f}{\pl t}\right) \mathrm{d} t
	\label{推广经典Lagrange关系中间步骤1}
\end{align}
再考虑函数$\dot{f} = \dot{f}(\mbf{q},\dot{\mbf{q}},t)$,可有
\begin{equation}
	\mathrm{d} \dot{f} = \sum_{\alpha=1}^n \frac{\pl \dot{f}}{\pl q_\alpha} \mathrm{d} q_\alpha + \sum_{\alpha=1}^n \frac{\pl \dot{f}}{\pl \dot{q}_\alpha} \mathrm{d} \dot{q}_\alpha + \frac{\pl \dot{f}}{\pl t} \mathrm{d} t
	\label{推广经典Lagrange关系中间步骤2}
\end{equation}
对比式\eqref{推广经典Lagrange关系中间步骤1}和\eqref{推广经典Lagrange关系中间步骤2}可得{\heiti 推广经典Lagrange关系}
\begin{equation}
	\frac{\pl \dot{f}}{\pl \dot{q}_\alpha} = \frac{\pl f}{\pl q_\alpha},\quad \frac{\mathrm{d}}{\mathrm{d} t} \frac{\pl f}{\pl q_\alpha} = \frac{\pl \dot{f}}{\pl q_\alpha},\quad \alpha = 1,2,\cdots,s
	\label{推广经典Lagrange关系}
\end{equation}
据此考虑广义力
\begin{align*}
	\tilde{Q}_\alpha & = \frac{\mathrm{d}}{\mathrm{d} t} \frac{\pl \tilde{U}}{\pl \dot{q}_\alpha} - \frac{\pl \tilde{U}}{\pl q_\alpha} = \frac{\mathrm{d}}{\mathrm{d} t} \frac{\pl U}{\pl \dot{q}_\alpha} - \frac{\pl U}{\pl q_\alpha} - \frac{\mathrm{d}}{\mathrm{d} t} \frac{\pl \dot{f}}{\pl \dot{q}_\alpha} + \frac{\pl \dot{f}}{\pl q_\alpha} \\
	& = \frac{\mathrm{d}}{\mathrm{d} t} \frac{\pl U}{\pl \dot{q}_\alpha} - \frac{\pl U}{\pl q_\alpha} = Q_\alpha	
\end{align*}
在规范变换下体系的力场不变,故势$\tilde{U}(\mbf{q},\dot{\mbf{q}},t) = U(\mbf{q},\dot{\mbf{q}},t) - \dfrac{\mathrm{d} f}{\mathrm{d} t}(\mbf{q},t)$与$U(\mbf{q},\dot{\mbf{q}},t)$表示的是同一个势场。

在规范变换下,广义动量为
\begin{equation*}
	\tilde{p}_\alpha = \frac{\pl \tilde{L}}{\pl \dot{q}_\alpha} = \frac{\pl L}{\pl \dot{q}_\alpha} + \frac{\pl \dot{f}}{\pl \dot{q}_\alpha} = p_\alpha + \frac{\pl f}{\pl q_\alpha}
\end{equation*}
广义能量为
\begin{align*}
	\tilde{H} & = \sum_{\alpha=1}^s \frac{\pl \tilde{L}}{\pl \dot{q}_\alpha} \dot{q}_\alpha - \tilde{L} = \sum_{\alpha=1}^s \frac{\pl L}{\pl \dot{q}_\alpha} \dot{q}_\alpha - L + \sum_{\alpha=1}^s \frac{\pl \dot{f}}{\pl \dot{q}_\alpha} \dot{q}_\alpha - \frac{\mathrm{d} f}{\mathrm{d} t} \\
	& = \sum_{\alpha=1}^s \frac{\pl L}{\pl \dot{q}_\alpha} \dot{q}_\alpha - L + \sum_{\alpha=1}^s \frac{\pl f}{\pl q_\alpha} \dot{q}_\alpha - \left(\sum_{\alpha=1}^s \frac{\pl f}{\pl q_\alpha} + \frac{\pl f}{\pl t}\right) = H - \frac{\pl f}{\pl t}
\end{align*}
即,与坐标共轭的广义动量,与时间共轭的广义能量均具有不定性,取决于Lagrange函数的规范选择。

对于一般完整系的Hamilton原理
\begin{equation*}
	\int_{t_1}^{t_2} \left(\delta T + \sum_{\alpha=1}^s Q_\alpha \delta q_\alpha\right) \mathrm{d} t = 0
\end{equation*}
考虑到$T = T(\mbf{q},\dot{\mbf{q}},t)$,有
\begin{align*}
	& \int_{t_1}^{t_2} \left(\delta T + \sum_{\alpha=1}^s Q_\alpha \delta q_\alpha\right) \mathrm{d} t = \int_{t_1}^{t_2} \left(\sum_{\alpha=1}^s \frac{\pl T}{\pl q_\alpha} \delta q_\alpha + \sum_{\alpha=1}^s \frac{\pl T}{\pl \dot{q}_\alpha} \delta \dot{q}_\alpha + \sum_{\alpha=1}^s Q_\alpha \delta q_\alpha\right) \mathrm{d} t \\
	& \quad = \int_{t_1}^{t_2} \left[\sum_{\alpha=1}^s \frac{\pl T}{\pl q_\alpha} \delta q_\alpha + \frac{\mathrm{d}}{\mathrm{d} t} \left(\sum_{\alpha=1}^s \frac{\pl T}{\pl \dot{q}_\alpha} \delta q_\alpha\right) - \sum_{\alpha=1}^s \frac{\mathrm{d}}{\mathrm{d} t} \left(\frac{\pl T}{\pl \dot{q}_\alpha}\right) \delta q_\alpha + \sum_{\alpha=1}^s Q_\alpha \delta q_\alpha \right] \mathrm{d} t \\
	& \quad = \frac{\pl T}{\pl \dot{q}_\alpha} \delta q_\alpha\bigg|_{t_1}^{t_2} + \int_{t_1}^{t_2} \sum_{\alpha=1}^s \left(\frac{\pl T}{\pl q_\alpha} - \frac{\mathrm{d}}{\mathrm{d} t} \frac{\pl T}{\pl \dot{q}_\alpha} + Q_\alpha\right) \delta q_\alpha \mathrm{d} t \\
	& \quad = \int_{t_1}^{t_2} \sum_{\alpha=1}^s \left(\frac{\pl T}{\pl q_\alpha} - \frac{\mathrm{d}}{\mathrm{d} t} \frac{\pl T}{\pl \dot{q}_\alpha} + Q_\alpha\right) \delta q_\alpha \mathrm{d} t
\end{align*}
因为$\delta q_\alpha\,(\alpha=1,2,\cdots,s)$相互独立,所以可得{\heiti 一般完整系的Lagrange方程}
\begin{equation}
	\frac{\mathrm{d}}{\mathrm{d} t} \frac{\pl T}{\pl \dot{q}_\alpha} - \frac{\pl T}{\pl q_\alpha} = Q_\alpha,\quad \alpha = 1,2,\cdots,s
\end{equation}

\subsection{相空间与Hamilton方程}

在位形空间,考虑Hamilton函数的定义,有
\begin{equation*}
	L = \sum_{\alpha=1}^s \frac{\pl L}{\pl \dot{q}_\alpha} \dot{q}_\alpha - H = \sum_{\alpha=1}^s p_\alpha \dot{q}_\alpha - H(\mbf{q},\mbf{p},t)
\end{equation*}
根据Hamilton原理,可有
\begin{align*}
	\delta S & = \int_{t_1}^{t_2} \left[\sum_{\alpha=1} (\dot{q}_\alpha \delta p_\alpha + p_\alpha \delta \dot{q}_\alpha) - \sum_{\alpha=1}^s \left(\frac{\pl H}{\pl p_\alpha} \delta p_\alpha + \frac{\pl H}{\pl q_\alpha} \delta q_\alpha\right)\right] \mathrm{d} t \\
	& = \int_{t_1}^{t_2} \left[\frac{\mathrm{d}}{\mathrm{d} t} \sum_{\alpha=1}^s p_\alpha \delta q_\alpha + \sum_{\alpha=1}^s \left(\dot{q}_\alpha - \frac{\pl H}{\pl p_\alpha}\right) \delta p_\alpha - \sum_{\alpha=1}^s \left(\dot{p}_\alpha + \frac{\pl H}{\pl q_\alpha} \right) \delta q_\alpha \right] \mathrm{d} t \\
	& = \sum_{\alpha=1}^s p_\alpha \delta q_\alpha\bigg|_{t_1}^{t_2} + \int_{t_1}^{t_2} \left[\sum_{\alpha=1}^s \left(\dot{q}_\alpha - \frac{\pl H}{\pl p_\alpha}\right) \delta p_\alpha - \sum_{\alpha=1}^s \left(\dot{p}_\alpha + \frac{\pl H}{\pl q_\alpha}\right) \delta q_\alpha\right] \mathrm{d} t
\end{align*}
由于$S = S[\mbf{q}(t)]$,独立变分只有$\delta q_\alpha$,动量由坐标定义,其变分不独立。考虑
\begin{equation*}
	H = \sum_{\beta=1}^s p_\beta \dot{q}_\beta(\mbf{q},\mbf{p},t) - L(\mbf{q},\dot{\mbf{q}}(\mbf{q},\mbf{p},t),t)
\end{equation*}
所以有
\begin{equation*}
	\frac{\pl H}{\pl p_\alpha} = \dot{q}_\alpha + \sum_{\beta=1}^s p_\beta \frac{\pl \dot{q}_\beta}{\pl p_\alpha} - \sum_{\beta=1}^s \frac{\pl L}{\pl \dot{q}_\beta} \frac{\pl \dot{q}_\beta}{\pl p_\alpha} = \dot{q}_\alpha + \sum_{\beta=1}^s \left(p_\beta - \frac{\pl L}{\pl \dot{q}_\beta} \right) \frac{\pl \dot{q}_\beta}{\pl p_\alpha}
\end{equation*}
考虑到动量的定义$p_\beta = \dfrac{\pl L}{\pl \dot{q}_\beta}$,即有
\begin{equation*}
	\dot{q}_\alpha = \frac{\pl H}{\pl p_\alpha}
\end{equation*}
因此有
\begin{equation*}
	\sum_{\alpha=1}^s \left(\dot{q}_\alpha - \frac{\pl H}{\pl p_\alpha}\right) \delta p_\alpha = 0
\end{equation*}
故有
\begin{equation*}
	\delta S = -\int_{t_1}^{t_2} \sum_{\alpha=1}^s \left(\dot{p}_\alpha + \frac{\pl H}{\pl q_\alpha} \right) \delta q_\alpha \mathrm{d} t = 0
\end{equation*}
由此可有
\begin{equation*}
	\dot{p}_\alpha = - \frac{\pl H}{\pl q_\alpha}
\end{equation*}
即,在位形空间中考察Hamilton方程一半是动量定义,另一半是运动方程。

上节已经指出,与坐标共轭的动量具有规范不定性,据此,可以将动量视为与坐标地位平等、完全独立的正则变量。由此在相空间中即可将Hamilton作用量写作
\begin{equation}
	S[\mbf{q}(t),\mbf{p}(t)] = \int_{t_1}^{t_2} \left[\sum_{\alpha=1}^s p_\alpha \dot{q}_\alpha - H(\mbf{q},\mbf{p},t)\right] \mathrm{d} t
\end{equation}
在固定边界条件下,应用Hamilton原理,有
\begin{align*}
	\delta S & = \int_{t_1}^{t_2} \left(\sum_{\alpha=1}^s \dot{q}_\alpha \delta p_\alpha + \sum_{\alpha=1}^s p_\alpha \delta \dot{q}_\alpha - \sum_{\alpha=1}^s \frac{\pl H}{\pl q_\alpha} \delta q_\alpha - \sum_{\alpha=1}^s \frac{\pl H}{\pl p_\alpha} \delta p_\alpha \right) \mathrm{d} t \\
	& = \int_{t_1}^{t_2} \left[\frac{\mathrm{d}}{\mathrm{d} t} \left(\sum_{\alpha=1}^s p_\alpha \delta q_\alpha\right) - \sum_{\alpha=1}^s \dot{p}_\alpha \delta q_\alpha + \sum_{\alpha=1}^s \dot{q}_\alpha \delta p_\alpha - \sum_{\alpha=1}^s \frac{\pl H}{\pl q_\alpha} \delta q_\alpha - \sum_{\alpha=1}^s \frac{\pl H}{\pl p_\alpha} \delta p_\alpha \right] \mathrm{d} t \\
	& = \sum_{\alpha=1}^s p_\alpha \delta q_\alpha \bigg|_{t_1}^{t_2} + \int_{t_1}^{t_2} \left[\sum_{\alpha=1}^s \left(\dot{q}_\alpha - \frac{\pl H}{\pl p_\alpha}\right) \delta p_\alpha - \sum_{\alpha=1}^s \left(\dot{p}_\alpha + \frac{\pl H}{\pl q_\alpha}\right) \delta q_\alpha \right] \mathrm{d} t \\
	& = \int_{t_1}^{t_2} \left[\sum_{\alpha=1}^s \left(\dot{q}_\alpha - \frac{\pl H}{\pl p_\alpha}\right) \delta p_\alpha - \sum_{\alpha=1}^s \left(\dot{p}_\alpha + \frac{\pl H}{\pl q_\alpha}\right) \delta q_\alpha \right] \mathrm{d} t
\end{align*}
据此即可得到Hamilton正则方程
\begin{equation}
	\begin{cases}
		\dot{q}_\alpha = \dfrac{\pl H}{\pl p_\alpha} \\[1.5ex]
		\dot{p}_\alpha = -\dfrac{\pl H}{\pl q_\alpha}
	\end{cases}
\end{equation}
将Hamilton原理视为相空间力学变分原理,则Hamilton方程全体均为运动方程。坐标和动量在满足相空间Hamilton原理的前提下可以独立变换,扩大了变换的自由度。

\subsection{位形时空与推广的Lagrange方程}

作参数变换$t = t(\tau)$,则有推广的Lagrange函数为
\begin{equation*}
	\mathnormal{\Lambda} (\mbf{q},t;\mbf{q}',t') = L \left(\mbf{q}, \frac{\mbf{q}'}{t'},t\right) t'
\end{equation*}
此时Hamilton作用量变换为
\begin{equation}
	S[\mbf{q}(\tau)] = \int_{\tau_1}^{\tau_2} L\left(\mbf{q},\frac{\mbf{q}'}{t'},t\right) t' \mathrm{d} \tau = \int_{\tau_1}^{\tau_2} \mathnormal{\Lambda} (\mbf{q},t;\mbf{q}',t') \mathrm{d} \tau
\end{equation}
在固定边界条件下,即
\begin{equation*}
	\delta \mbf{q}(\tau_1) = \delta \mbf{q}(\tau_2) = 0,\quad \delta t(\tau) = 0
\end{equation*}
可有
\begin{align*}
	\delta S & = \int_{\tau_1}^{\tau_2} \sum_{\alpha=1}^s \left(\frac{\pl \mathnormal{\Lambda}}{\pl q_\alpha} \delta q_\alpha + \frac{\pl \mathnormal{\Lambda}}{\pl q'_\alpha} \delta q'_\alpha\right) \mathrm{d} \tau \\
	& = \int_{\tau_1}^{\tau_2} \left[\frac{\mathrm{d}}{\mathrm{d} \tau} \sum_{\alpha=1}^s \frac{\pl \mathnormal{\Lambda}}{\pl q'_\alpha} \delta q_\alpha + \sum_{\alpha=1}^s \left(\frac{\pl \mathnormal{\Lambda}}{\pl q_\alpha} - \frac{\mathrm{d}}{\mathrm{d} \tau} \frac{\pl \mathnormal{\Lambda}}{\pl q'_\alpha}\right) \delta q_\alpha \right] \mathrm{d} \tau \\
	& = \sum_{\alpha=1}^s \frac{\pl \mathnormal{\Lambda}}{\pl q'_\alpha} \delta q_\alpha\bigg|_{\tau_1}^{\tau_2} + \int_{\tau_1}^{\tau_2} \sum_{\alpha=1}^s \left(\frac{\pl \mathnormal{\Lambda}}{\pl q_\alpha} - \frac{\mathrm{d}}{\mathrm{d} \tau} \frac{\pl \mathnormal{\Lambda}}{\pl q'_\alpha}\right) \delta q_\alpha \mathrm{d} \tau \\
	& = \int_{\tau_1}^{\tau_2} \sum_{\alpha=1}^s \left(\frac{\pl \mathnormal{\Lambda}}{\pl q_\alpha} - \frac{\mathrm{d}}{\mathrm{d} \tau} \frac{\pl \mathnormal{\Lambda}}{\pl q'_\alpha}\right) \delta q_\alpha \mathrm{d} \tau = 0
\end{align*}
由此可得以$\tau$为参数的Lagrange方程
\begin{equation}
	\frac{\mathrm{d}}{\mathrm{d} \tau} \frac{\pl \mathnormal{\Lambda}}{\pl q'_\alpha} - \frac{\pl \mathnormal{\Lambda}}{\pl q_\alpha} = 0,\quad \alpha = 1,2,\cdots,s
\end{equation}
以上讨论限于位形空间,仅将参数$t$通过确定的对应关系变换成新参数$\tau$。

在位形时空中,$\tau$为参数,$t$视为变量,记作$q_{s+1}$。即此时$\delta \tau = 0$,但$\delta t(\tau) \neq 0$,Hamilton作用量为
\begin{equation*}
	S[\mbf{q}(\tau),t(\tau)] = \int_{\tau_1}^{\tau_2} \mathnormal{\Lambda}(\mbf{q},t;\mbf{q}',t') \mathrm{d} \tau
\end{equation*}
同样在固定边界条件下
\begin{equation*}
	\delta \mbf{q}(\tau_1) = \delta \mbf{q}(\tau_2) = 0,\quad \delta t(\tau_1) = \delta t(\tau_2) = 0
\end{equation*}
考虑Hamilton原理,即
\begin{align*}
	\delta S & = \sum_{\alpha=1}^{s+1} \frac{\pl \mathnormal{\Lambda}}{\pl q'_\alpha} \delta q_\alpha\bigg|_{\tau_1}^{\tau_2} + \int_{\tau_1}^{\tau_2} \sum_{\alpha=1}^{s+1} \left(\frac{\pl \mathnormal{\Lambda}}{\pl q_\alpha} - \frac{\mathrm{d}}{\mathrm{d} \tau} \frac{\pl \mathnormal{\Lambda}}{\pl q'_\alpha}\right) \delta q_\alpha \mathrm{d} \tau \\
	& = \int_{\tau_1}^{\tau_2} \sum_{\alpha=1}^{s+1} \left(\frac{\pl \mathnormal{\Lambda}}{\pl q_\alpha} - \frac{\mathrm{d}}{\mathrm{d} \tau} \frac{\pl \mathnormal{\Lambda}}{\pl q'_\alpha}\right) \delta q_\alpha \mathrm{d} \tau = 0
\end{align*}
由此可以得到位形时空($s+1$维空间)中的Lagrange方程
\begin{equation}
	\frac{\mathrm{d}}{\mathrm{d} \tau} \frac{\pl \mathnormal{\Lambda}}{\pl q'_\alpha} - \frac{\pl \mathnormal{\Lambda}}{\pl q_\alpha} = 0,\quad \alpha = 1,2,\cdots,s+1
\end{equation}

\subsection{其它形式的力学变分原理}

为了导出其他形式的力学变分原理,首先考虑在位形时空中,Hamilton原理可以表示为
\begin{equation*}
	\delta S = \sum_{\alpha=1}^{s+1} \frac{\pl \mathnormal{\Lambda}}{\pl q'_\alpha} \delta q_\alpha\bigg|_{\tau_1}^{\tau_2} + \int_{\tau_1}^{\tau_2} \sum_{\alpha=1}^{s+1} \left(\frac{\pl \mathnormal{\Lambda}}{\pl q_\alpha} - \frac{\mathrm{d}}{\mathrm{d} \tau} \frac{\pl \mathnormal{\Lambda}}{\pl q'_\alpha}\right) \delta q_\alpha \mathrm{d} \tau
\end{equation*}
其他形式的力学变分原理应同样满足系统的Lagrange方程,因此如果此处不附加边界条件,将Lagrange方程代入,可得力学变分原理的一种形式如下
\begin{equation*}
	\delta S = \int_{\tau_1}^{\tau_2} \delta \varLambda \mathrm{d} \tau = \sum_{\alpha=1}^{s+1} \frac{\pl \mathnormal{\Lambda}}{\pl q'_\alpha} \delta q_\alpha\bigg|_{\tau_1}^{\tau_2} = \sum_{\alpha=1}^{s+1} p_\alpha \delta q_\alpha\bigg|_{\tau_1}^{\tau_2}
\end{equation*}
或者
\begin{equation}
	\int_{\tau_1}^{\tau_2} \left(\delta \varLambda - \frac{\mathrm{d}}{\mathrm{d} \tau} \sum_{\alpha=1}^{s+1} p_\alpha \delta q_\alpha \right) \mathrm{d} \tau = 0
\end{equation}
若实际运动与虚拟运动同时同位形出发且同时同位形到达,即
\begin{equation*}
	\delta q_\alpha(\tau_1) = \delta q_\alpha(\tau_2) = 0,\quad \alpha = 1,2,\cdots,s+1
\end{equation*}
即得到位形时空($s+1$维)的Hamilton原理
\begin{equation}
	\int_{\tau_1}^{\tau_2} \delta \varLambda \mathrm{d} \tau = 0
	\label{位形时空的Hamilton原理}
\end{equation}
若实际运动与虚拟运动同位形出发且同位形到达,但出发、到达时间不同,即
\begin{equation*}
	\delta q_\alpha(\tau_1) = \delta q_\alpha(\tau_2) = 0,\quad \alpha = 1,2,\cdots,s,\quad \delta t(\tau) \neq 0
\end{equation*}
即得到位形空间($s$维)的Hamilton原理
\begin{equation}
	\int_{\tau_1}^{\tau_2} \left[\delta \varLambda + \frac{\mathrm{d}}{\mathrm{d} \tau}(H\delta t)\right] \mathrm{d} \tau = 0
	\label{位形空间的Hamilton原理}
\end{equation}

现在,考虑满足时间平移对称性的体系,此时有
\begin{equation*}
	H' = -p'_t = -\frac{\pl \varLambda}{\pl t} = 0
\end{equation*}
此时体系广义能量守恒\footnote{大部分的力学体系都能满足此条件。}。可比较具有相同能量、相同始末位形的实际运动与虚拟运动,此时变分相应地称为{\heiti 等能变分}。等能变分$\delta$满足
\begin{equation*}
	\delta H = 0,\quad \delta t \neq 0,\quad \delta \tau = 0
\end{equation*}
等能条件限制了位形轨道每一点的速度,使得实际运动与虚拟运动不能同时出发与到达。此时有
\begin{equation*}
	\frac{\mathrm{d}}{\mathrm{d} \tau}(H\delta t) = H' \delta t + H\delta t' = \delta (Ht) - t'\delta H = \delta (Ht')
\end{equation*}
由此,式\eqref{位形时空的Hamilton原理}可化为
\begin{equation*}
	\int_{\tau_1}^{\tau_2} \delta (\varLambda + Ht') \mathrm{d} \tau = \delta \int_{\tau_1}^{\tau_2} (\varLambda + Ht') \mathrm{d} \tau = 0
\end{equation*}
定义
\begin{equation*}
	W[\mbf{q}(\tau),t(\tau)] = \int_{\tau_1}^{\tau_2} (L+H)t' \mathrm{d} \tau = \int_{\tau_1}^{\tau_2} \sum_{\alpha=1}^s p_\alpha q'_\alpha \mathrm{d} \tau 
\end{equation*}
称为{\heiti Lagrange作用量}\footnote{有些文献上称为{\heiti 简约作用量}。},相应的式\eqref{位形时空的Hamilton原理}化为
\begin{equation*}
	\delta W[\mbf{q}(\tau),t(\tau)] = \delta \int_{\tau_1}^{\tau_2} \sum_{\alpha=1}^s p_\alpha q'_\alpha \mathrm{d} \tau = 0
\end{equation*}
称为{\heiti 位形时空的Lagrange原理}。在Lagrange作用量的表达式中作变量代换$t=t(\tau)$消去参数$\tau$,可有
\begin{equation}
	W[\mbf{q}(t)] = \int_{t_1}^{t_2} (L+H) \mathrm{d} t = \int_{t_1}^{t_2} \sum_{\alpha=1}^s p_\alpha \dot{q}_\alpha \mathrm{d} t
\end{equation}
此时可有{\heiti 位形空间的Lagrange原理}
\begin{equation}
	\Delta W[\mbf{q}(t)] = \Delta \int_{t_1}^{t_2} (L+H) \mathrm{d} t = \Delta \int_{t_1}^{t_2} \sum_{\alpha=1}^s p_\alpha \dot{q}_\alpha \mathrm{d} t = 0
	\label{位形空间的Lagrange原理}
\end{equation}
此处,$\Delta$为位形空间中的等时变分。由于位形空间与位形时空的参数不同,因此他们的变分规则不同,此处引入记号$\Delta$以示区别。在位形空间中,取任意函数$f(t)$,则等能变分满足
\begin{align*}
	\Delta f & = \delta f(t) + \dot{f} \Delta t \\
	\frac{\mathrm{d}}{\mathrm{d} t} \Delta f & = \delta \dot{f}(t) + \ddot{f} \Delta t + \dot{f} \frac{\mathrm{d}\Delta t}{\mathrm{d} t} = \Delta \dot{f} + \dot{f} \frac{\mathrm{d}\Delta t}{\mathrm{d} t} \\
	\Delta \int_{t_1}^{t_2} f \mathrm{d} t & = \delta \int_{\tau_1}^{\tau_2} f t' \mathrm{d} \tau = \int_{\tau_1}^{\tau_2} \left(t' \delta f + f\frac{\mathrm{d} \delta t}{\mathrm{d} \tau}\right) \mathrm{d} \tau = \int_{t_1}^{t_2} \left(\Delta f + f\frac{\mathrm{d} \Delta t}{\mathrm{d} t}\right) \mathrm{d} t
\end{align*}
Lagrange原理\eqref{位形空间的Lagrange原理}可改写为
\begin{equation}
	\Delta W = \Delta \int_{\mbf{q}_1}^{\mbf{q}_2} \sum_{\alpha=1}^s p_\alpha \mathrm{d} q_\alpha = 0
	\label{Lagrange原理的Maupertuis形式}
\end{equation}
式\eqref{Lagrange原理的Maupertuis形式}称为{\heiti Lagrange原理的Maupertuis(莫培督)形式}。

\begin{example}[由位形空间的Lagrange原理导出Lagrange方程]
由位形空间的Lagrange原理,即
\begin{equation*}
	\Delta W = \Delta \int_{t_1}^{t_2} (L+H) \mathrm{d} t = \int_{t_1}^{t_2} \left[\Delta(L+H) + (L+H) \frac{\mathrm{d} \Delta t}{\mathrm{d} t}\right] \mathrm{d} t = 0
\end{equation*}
现在来考察上面积分中的各项,首先根据等能变分的定义,有$\Delta H = 0$。考虑到体系的能量守恒,即有
\begin{equation*}
	\dot{H} = -\frac{\pl L}{\pl t} = 0
\end{equation*}
由此可有
\begin{align*}
	\Delta L + (L+H) \frac{\mathrm{d} \Delta t}{\mathrm{d} t} & = \sum_{\alpha=1}^s \frac{\pl L}{\pl q_\alpha} \Delta q_\alpha + \sum_{\alpha=1}^s \frac{\pl L}{\pl \dot{q}_\alpha} \Delta \dot{q}_\alpha + \frac{\pl L}{\pl t} \Delta t + \sum_{\alpha=1}^s p_\alpha q_\alpha \frac{\mathrm{d} \Delta t}{\mathrm{d} t}\\
	& = \sum_{\alpha=1}^s \frac{\pl L}{\pl q_\alpha} \Delta q_\alpha + \sum_{\alpha=1}^s \frac{\pl L}{\pl \dot{q}_\alpha} \left(\frac{\mathrm{d} \Delta q_\alpha}{\mathrm{d} t} - \dot{q}_\alpha \frac{\mathrm{d} \Delta t}{\mathrm{d} t}\right) + \sum_{\alpha=1}^s p_\alpha q_\alpha \frac{\mathrm{d} \Delta t}{\mathrm{d} t}\\
	& = \frac{\mathrm{d}}{\mathrm{d} t} \sum_{\alpha=1}^s \frac{\pl L}{\pl \dot{q}_\alpha} \Delta q_\alpha + \sum_{\alpha=1}^s \left(\frac{\pl L}{\pl q_\alpha} - \frac{\mathrm{d}}{\mathrm{d} t} \frac{\pl L}{\pl \dot{q}_\alpha}\right) \Delta q_\alpha
\end{align*}
综上,可得
\begin{align*}
	\Delta W & = \int_{t_1}^{t_2} \left[\frac{\mathrm{d}}{\mathrm{d} t} \sum_{\alpha=1}^s \frac{\pl L}{\pl \dot{q}_\alpha} \Delta q_\alpha + \sum_{\alpha=1}^s \left(\frac{\pl L}{\pl q_\alpha} - \frac{\mathrm{d}}{\mathrm{d} t} \frac{\pl L}{\pl \dot{q}_\alpha}\right) \Delta q_\alpha\right] \mathrm{d} t \\
	& = \sum_{\alpha=1}^s \frac{\pl L}{\pl \dot{q}_\alpha} \Delta q_\alpha\bigg|_{t_1}^{t_2} + \int_{t_1}^{t_2} \sum_{\alpha=1}^s \left(\frac{\pl L}{\pl q_\alpha} - \frac{\mathrm{d}}{\mathrm{d} t} \frac{\pl L}{\pl \dot{q}_\alpha}\right) \Delta q_\alpha \mathrm{d} t
\end{align*}
对始末位形可有
\begin{equation*}
	\Delta \mbf{q}(t_1) = \Delta \mbf{q}(t_2) = 0
\end{equation*}
因此,有
\begin{equation*}
	\Delta W = \int_{t_1}^{t_2} \sum_{\alpha=1}^s \left(\frac{\pl L}{\pl q_\alpha} - \frac{\mathrm{d}}{\mathrm{d} t} \frac{\pl L}{\pl \dot{q}_\alpha}\right) \Delta q_\alpha \mathrm{d} t = 0
\end{equation*}
根据$\Delta q_\alpha$的独立性,即可得到Lagrange方程
\begin{equation*}
	\frac{\mathrm{d}}{\mathrm{d} t} \frac{\pl L}{\pl \dot{q}_\alpha} - \frac{\pl L}{\pl q_\alpha} = 0,\quad \alpha = 1,2,\cdots,s
\end{equation*}
\end{example}

若体系的约束为完整稳定约束,根据节\ref{第二章:广义能量的数学结构}的讨论,此时有
\begin{align*}
	L & = T-V = T_2 - V \\
	H & = L_2 - L_0 = T_2 + V
\end{align*}
其中
\begin{equation*}
	T_2 = \frac12 \sum_{\alpha,\beta=1}^s M_{\alpha\beta}(\mbf{q},t) \dot{q}_\alpha \dot{q}_\beta,\quad M_{\alpha\beta}(\mbf{q},t) = \sum_{i=1}^n m_i \frac{\pl \mbf{r}_i}{\pl q_\alpha} \cdot \frac{\pl \mbf{r}_i}{\pl q_\beta}
\end{equation*}
此时Lagrange原理为
\begin{equation*}
	\Delta W = \Delta \int_{t_1}^{t_2} (L+H) \mathrm{d} t = \Delta \int_{t_1}^{t_2} \sum_{\alpha,\beta=1}^s M_{\alpha\beta}(\mbf{q},t) \dot{q}_\alpha \dot{q}_\beta \mathrm{d} t = 0
\end{equation*}
定义位形空间线元
\begin{equation*}
	\mathrm{d} s^2 = \frac{1}{M} \sum_{i=1}^n m_i \mathrm{d} \mbf{r}_i \cdot \mathrm{d} \mbf{r}_i = \frac{1}{M} \sum_{\alpha,\beta=1}^s M_{\alpha\beta} \mathrm{d} q_\alpha \mathrm{d} q_\beta
\end{equation*}
其中$\displaystyle M = \sum_{i=1}^n m_i$,此时动能可以表示为
\begin{equation*}
	T = \frac12 M \left(\frac{\mathrm{d} s}{\mathrm{d} t}\right)^2 
\end{equation*}
此处,$\dfrac{\mathrm{d} s}{\mathrm{d} t}$即为体系位形代表点沿位形轨道运动的速度,方向即为轨道的切线方向。再考虑到体系的能量守恒,将总能量记作$E$,则有
\begin{equation*}
	2T\mathrm{d} t = \sqrt{2MT}\mathrm{d} s = \sqrt{2M\big[E-V(\mbf{r})\big]} \mathrm{d}s
\end{equation*}
此时,Lagrange原理形式为
\begin{equation}
	\Delta W = \Delta \int_{\mbf{r}_1}^{\mbf{r}_2} \sqrt{2M\big[E-V(\mbf{r})\big]} \mathrm{d}s = 0
	\label{Lagrange原理的Jacobi形式}
\end{equation}
式\eqref{Lagrange原理的Jacobi形式}称为{\heiti Lagrange原理的Jacobi形式}。此原理不涉及时间,它从所有满足能量守恒的虚拟轨道中选出Lagrange作用量取极值的实际轨道。

\section{Lagrange乘子法}

对于完整体系,运动方程中不含约束力。因此对于不需要求解约束力的问题,可直接求解运动。如何求解完整体系中的约束力?

\subsection{Lagrange乘子法}

多元函数在约束下的极值问题,可以用{\heiti Lagrange乘子法}来解决。即已知多元函数
\begin{equation*}
	J = J(y_1,y_2,\cdots,y_n)
\end{equation*}
求其在约束条件
\begin{equation*}
	f_j(y_1,y_2,\cdots,y_n) = 0,\quad j = 1,2,\cdots,m<n
\end{equation*}
下的极值。根据极值条件,取极值时要求
\begin{equation}
	\mathrm{d} J = \sum_{i=1}^n \frac{\pl J}{\pl y_i} \mathrm{d}y_i = 0
\end{equation}
但此时$\mathrm{d} y_i$之间并不独立,还需满足
\begin{equation}
	\sum_{i=1}^n \frac{\pl f}{\pl y_i} \mathrm{d} y_i = 0,\quad i=1,2,\cdots,m
\end{equation}
由此,可引入$m$个不定乘子$\lambda_i$,则有
\begin{equation}
	\mathrm{d} \left(J + \sum_{j=1}^m \lambda_j f_j\right) = \sum_{i=1}^n \left(\frac{\pl J}{\pl y_i} + \sum_{j=1}^m \lambda_j \frac{\pl f_j}{\pl y_i} \right) \mathrm{d} y_i = 0
\end{equation}
即有驻点满足的方程为
\begin{equation}
	\begin{cases}
		\displaystyle \frac{\pl J}{\pl y_i} + \sum_{j=1}^m \lambda_j \frac{\pl f_j}{\pl y_i} = 0, & i=1,2,\cdots,n \\
		f_j(y_1,y_2,\cdots,y_n) = 0, & j=1,2,\cdots,m
	\end{cases}
\end{equation}
据此可以解出极值点$(y_1,y_2,\cdots,y_n)$和乘子$\lambda_j$。这种方法称为Lagrange乘子法。

一般地,Lagrange乘子法可按如下操作:首先定义Lagrange函数
\begin{equation}
	\bar{J}(y_1,y_2,\cdots,y_n;\lambda_1,\lambda_2,\cdots,\lambda_m) = J(y_1,y_2,\cdots,y_n) + \sum_{j=1}^m \lambda_j f_j(y_1,y_2,\cdots,y_n)
\end{equation}
然后按无条件极值的方法求解极值点
\begin{equation}
	\begin{cases}
		\displaystyle \frac{\pl \bar{J}}{\pl y_i} = 0, & i=1,2,\cdots,n \\[1.5ex]
		\displaystyle \frac{\pl \bar{J}}{\pl \lambda_j} = 0, & j = 1,2,\cdots,m
	\end{cases}
\end{equation}
解此方程组,即可得到条件极值点$(y_1,y_2,\cdots,y_n)$(以及$m$个Lagrange乘子)。

\subsection{Lagrange乘子法求解全部约束力}

理想完整有势体系的约束可以表示为
\begin{equation}
	f_j(\mbf{r}_1,\mbf{r}_2,\cdots,\mbf{r}_n;t) = 0,\quad j = 1,2,\cdots,k
\end{equation}
Hamilton作用量可以表示为
\begin{equation*}
	S[\mbf{r}(t)] = \int_{t_1}^{t_2} L(\mbf{r},\dot{\mbf{r}},t) \mathrm{d} t
\end{equation*}
在固定边界条件下,其变分为
\begin{align*}
	\delta S & = \int_{t_1}^{t_2} \delta L(\mbf{r},\dot{\mbf{r}},t) \mathrm{d} t = \int_{t_1}^{t_2} \left(\sum_{i=1}^n \frac{\pl L}{\pl \mbf{r}_i} \cdot \delta \mbf{r}_i + \sum_{i=1}^n \frac{\pl L}{\pl \dot{\mbf{r}}_i} \cdot \delta \dot{\mbf{r}}_i\right) \mathrm{d} t \\
	& = \int_{t_1}^{t_2} \left[\frac{\mathrm{d}}{\mathrm{d} t} \left(\sum_{i=1}^n \frac{\pl L}{\pl \dot{\mbf{r}}_i} \cdot \delta \mbf{r}_i\right) - \sum_{i=1}^n \frac{\mathrm{d}}{\mathrm{d} t} \frac{\pl L}{\pl \dot{\mbf{r}}_i} \cdot \delta \mbf{r}_i + \sum_{i=1}^n \frac{\pl L}{\pl \mbf{r}_i} \cdot \delta \mbf{r}_i\right] \mathrm{d} t \\
	& = \sum_{i=1}^n \frac{\pl L}{\pl \dot{\mbf{r}}_i} \cdot \delta \mbf{r}_i \bigg|_{t_1}^{t_2} + \int_{t_1}^{t_2} \sum_{i=1}^n \left(\frac{\pl L}{\pl \mbf{r}_i} - \frac{\mathrm{d}}{\mathrm{d} t} \frac{\pl L}{\pl \dot{\mbf{r}}_i}\right) \cdot \delta \mbf{r}_i \mathrm{d} t \\
	& = \int_{t_1}^{t_2} \sum_{i=1}^n \left(\frac{\pl L}{\pl \mbf{r}_i} - \frac{\mathrm{d}}{\mathrm{d} t} \frac{\pl L}{\pl \dot{\mbf{r}}_i}\right) \cdot \delta \mbf{r}_i \mathrm{d} t = 0
\end{align*}
再考虑到约束条件
\begin{equation*}
	\delta f_j = \sum_{i=1}^n \frac{\pl f_j}{\pl \mbf{r}_i} \cdot \delta \mbf{r}_i = 0
\end{equation*}
引入不定乘子$\lambda_j(t)$即有
\begin{equation}
	\delta \int_{t_1}^{t_2} \left(L + \sum_{j=1}^k \lambda_j(t) f_j\right) \mathrm{d} t = \int_{t_1}^{t_2} \sum_{i=1}^n \left(\frac{\pl L}{\pl \mbf{r}_i} - \frac{\mathrm{d}}{\mathrm{d} t} \frac{\pl L}{\pl \dot{\mbf{r}}_i} + \sum_{j=1}^k \lambda_j \frac{\pl f_j}{\pl \mbf{r}_i}\right) \cdot \delta \mbf{r}_i \mathrm{d} t = 0
\end{equation}
由此可有
\begin{equation}
	\begin{cases}
		\displaystyle \frac{\mathrm{d}}{\mathrm{d} t} \frac{\pl L}{\pl \dot{\mbf{r}}_i} - \frac{\pl L}{\pl \mbf{r}_i} = \sum_{j=1}^k \lambda_j \frac{\pl f_j}{\pl \mbf{r}_i},& i = 1,2,\cdots,n \\
		\displaystyle f_j(\mbf{r},t) = 0,& j=1,2,\cdots,k
	\end{cases}
	\label{完整有势系的Lagrange乘子法}
\end{equation}
据此可以解出系统的运动$\mbf{r}_i = \mbf{r}_i(t)$,以及不定乘子$\lambda_j = \lambda_j(t)$。

根据Lagrange公式\eqref{加速度的Lagrange公式}可有
\begin{equation*}
	m_i \ddot{\mbf{r}}_i = \frac{\mathrm{d}}{\mathrm{d} t} \frac{\pl T}{\pl \dot{\mbf{r}}_i} - \frac{\pl T}{\pl \mbf{r}_i}
\end{equation*}
再考虑到
\begin{equation*}
	\mbf{F}_i = \frac{\mathrm{d}}{\mathrm{d} t} \frac{\pl U}{\pl \dot{\mbf{r}}_i} - \frac{\pl U}{\pl \mbf{r}_i}
\end{equation*}
根据Newton第二定律,约束力满足
\begin{equation*}
	m_i \ddot{\mbf{r}}_i = \mbf{F}_i + \mbf{R}_i,\quad i=1,2,\cdots,n
\end{equation*}
所以有
\begin{align}
	\mbf{R}_i & = m_i\ddot{\mbf{r}}_i - \mbf{F}_i = \left(\frac{\mathrm{d}}{\mathrm{d} t} \frac{\pl T}{\pl \dot{\mbf{r}}_i} - \frac{\pl T}{\pl \mbf{r}_i}\right) - \left(\frac{\mathrm{d}}{\mathrm{d} t} \frac{\pl U}{\pl \dot{\mbf{r}}_i} - \frac{\pl U}{\pl \mbf{r}_i}\right) \nonumber = \frac{\mathrm{d}}{\mathrm{d} t} \frac{\pl L}{\pl \dot{\mbf{r}}_i} - \frac{\pl L}{\pl \mbf{r}_i} \nonumber \\
	& = \sum_{j=1}^k \lambda_j \frac{\pl f_j}{\pl \mbf{r}_i}
\end{align}
于是,记
\begin{equation}
	\mbf{R}_{ij} = \lambda_j \frac{\pl f_j}{\pl \mbf{r}_i}
\end{equation}
为第$j$个约束施于第$i$个质点的力,其方向决定于$\bnb_i f_j$,大小则与$\lambda_j$成正比。约束力可以写作
\begin{equation*}
	\mbf{R}_i = \sum_{j=1}^k \lambda_j \frac{\pl f_j}{\pl \mbf{r}_i} = -\frac{\pl}{\pl \mbf{r}_i} \left(-\sum_{j=1}^k \lambda_j f_j\right)
\end{equation*}
所以,标量函数
\begin{equation}
	W(\mbf{r},t) = -\sum_{j=1}^k \lambda_j f_j(\mbf{r},t)
\end{equation}
可称为{\heiti 约束力势}。

于是可将约束力势视作外力的势能加入Lagrange函数中,即
\begin{equation}
	\bar{L}(\mbf{r},\dot{\mbf{r}},t;\mbf{\lambda}) = T-U-W = \sum_{i=1}^n \frac12 m\dot{\mbf{r}}^2 - U(\mbf{r},\dot{\mbf{r}},t) + \sum_{j=1}^k \lambda_j f_j(\mbf{r},t)
\end{equation}
应用Hamilton原理可得
\begin{align*}
	& \delta \bar{S}[\mbf{r}(t),\mbf{\lambda}(t)] = \delta \int_{t_1}^{t_2} \bar{L} \mathrm{d} t = \int_{t_1}^{t_2} \left(\sum_{i=1}^n \frac{\pl \bar{L}}{\pl \mbf{r}_i} \cdot \delta \mbf{r}_i + \sum_{i=1}^n \frac{\pl \bar{L}}{\pl \dot{\mbf{r}}_i} \cdot \delta \dot{\mbf{r}}_i + \sum_{j=1}^k \frac{\pl \bar{L}}{\pl \lambda_j} \delta \lambda_j \right) \mathrm{d} t \\
	& \quad = \int_{t_1}^{t_2} \left[\frac{\mathrm{d}}{\mathrm{d} t} \left(\sum_{i=1}^n \frac{\pl \bar{L}}{\pl \dot{\mbf{r}}_i} \cdot \delta \mbf{r}_i\right) - \sum_{i=1}^n \frac{\mathrm{d}}{\mathrm{d} t} \frac{\pl \bar{L}}{\pl \dot{\mbf{r}}_i} \cdot \delta \mbf{r}_i + \sum_{i=1}^s \frac{\pl \bar{L}}{\pl \mbf{r}_i} \cdot \delta \mbf{r}_i + \sum_{j=1}^k \frac{\pl \bar{L}}{\pl \lambda_j} \delta \lambda_j\right] \mathrm{d} t \\
	& \quad = \sum_{i=1}^n \frac{\pl \bar{L}}{\pl \dot{\mbf{r}}_i} \cdot \delta \mbf{r}_i\bigg|_{t_1}^{t_2} + \int_{t_1}^{t_2} \left(\sum_{i=1}^s \frac{\pl \bar{L}}{\pl \mbf{r}_i} \cdot \delta \mbf{r}_i - \sum_{i=1}^n \frac{\mathrm{d}}{\mathrm{d} t} \frac{\pl \bar{L}}{\pl \dot{\mbf{r}}_i} \cdot \delta \mbf{r}_i + \sum_{j=1}^k \frac{\pl \bar{L}}{\pl \lambda_j} \delta \lambda_j\right) \mathrm{d} t \\
	& \quad = \int_{t_1}^{t_2} \left[-\sum_{i=1}^s \left(\frac{\mathrm{d}}{\mathrm{d} t} \frac{\pl \bar{L}}{\pl \dot{\mbf{r}}_i} - \frac{\pl \bar{L}}{\pl \mbf{r}_i}\right) \cdot \delta \mbf{r}_i + \sum_{j=1}^k \frac{\pl \bar{L}}{\pl \lambda_j} \delta \lambda_j\right] \mathrm{d} t
\end{align*}
由此可得
\begin{equation}
	\begin{cases}
		\displaystyle \frac{\mathrm{d}}{\mathrm{d} t} \frac{\pl \bar{L}}{\pl \dot{\mbf{r}}_i} - \frac{\pl \bar{L}}{\pl \mbf{r}_i} = 0,& i=1,2,\cdots,n \\[1.5ex]
		\displaystyle \frac{\pl \bar{L}}{\pl \lambda_j} = 0,& j = 1,2,\cdots,k
	\end{cases}
\end{equation}

对于一般理想完整体系,可构造
\begin{equation}
	\bar{L} = T- W = \sum_{i=1}^n \frac12 m_i \dot{\mbf{r}}^2 + \sum_{j=1}^k \lambda_j f_j(\mbf{r},t)
\end{equation}
由一般完整体系的Hamilton原理
\begin{align*}
	& \int_{t_1}^{t_2} \left(\delta \bar{L} + \sum_{i=1}^n \mbf{F}_i \cdot \delta \mbf{r}_i\right) \mathrm{d} t \\
	& \quad = \int_{t_1}^{t_2} \left(\sum_{i=1}^n \frac{\pl \bar{L}}{\pl \mbf{r}_i} \cdot \delta \mbf{r}_i + \sum_{i=1}^n \frac{\pl \bar{L}}{\pl \dot{\mbf{r}}_i} \cdot \delta \dot{\mbf{r}}_i + \sum_{j=1}^k \frac{\pl \bar{L}}{\pl \lambda_j} \delta \lambda_j + \sum_{i=1}^n \mbf{F}_i \cdot \delta \mbf{r}_i \right) \mathrm{d} t \\
	& \quad = \sum_{i=1}^n \frac{\pl \bar{L}}{\pl \dot{\mbf{r}}_i} \cdot \delta \mbf{r}_i \bigg|_{t_1}^{t_2} + \int_{t_1}^{t_2} \left[-\sum_{i=1}^n \left(\frac{\mathrm{d}}{\mathrm{d} t} \frac{\pl \bar{L}}{\pl \dot{\mbf{r}}_i} - \frac{\pl \bar{L}}{\pl \mbf{r}_i} - \mbf{F}_i\right) \cdot \delta \mbf{r}_i + \sum_{j=1}^k \frac{\pl \bar{L}}{\pl \lambda_j} \delta \lambda_j\right] \mathrm{d} t \\
	& \quad = \int_{t_1}^{t_2} \left[-\sum_{i=1}^n \left(\frac{\mathrm{d}}{\mathrm{d} t} \frac{\pl \bar{L}}{\pl \dot{\mbf{r}}_i} - \frac{\pl \bar{L}}{\pl \mbf{r}_i} - \mbf{F}_i\right) \cdot \delta \mbf{r}_i + \sum_{j=1}^k \frac{\pl \bar{L}}{\pl \lambda_j} \delta \lambda_j\right] \mathrm{d} t
\end{align*}
由此可得
\begin{equation}
	\begin{cases}
		\displaystyle \frac{\mathrm{d}}{\mathrm{d} t} \frac{\pl \bar{L}}{\pl \dot{\mbf{r}}_i} - \frac{\pl \bar{L}}{\pl \mbf{r}_i} = \mbf{F}_i,& i=1,2,\cdots,n \\[1.5ex]
		\displaystyle \frac{\pl \bar{L}}{\pl \lambda_j} = 0,& j = 1,2,\cdots,k
	\end{cases}
\end{equation}

\subsection{Lagrange乘子法求解部分约束力}

设有完整约束
\begin{equation*}
	\begin{cases}
		f_j(\mbf{r}_1,\mbf{r}_2,\cdots,\mbf{r}_n;t) = 0, & j = 1,2,\cdots,l \\
		f_j(\mbf{r}_1,\mbf{r}_2,\cdots,\mbf{r}_n;t) = 0, & j = l+1,l+2,\cdots,k
	\end{cases}
\end{equation*}
求前$l$个约束的约束力,可以除去约束中的前$l$个,根据后$k-l$个选择广义坐标$q_\alpha,\,\,\alpha=1,2,\cdots,r = 3n-k+l$。广义坐标使得后面$k-l$个约束自动满足,余下前$l$个约束方程变为
\begin{equation*}
	f_j(\mbf{r}(\mbf{q},t),t) = \bar{f}_j(\mbf{q},t),\quad j=1,2,\cdots,l
\end{equation*}
即
\begin{equation*}
	\delta \bar{f}_j = \sum_{\alpha=1}^r \frac{\pl \bar{f}_j}{\pl q_\alpha} \delta q_\alpha = 0
\end{equation*}
根据Hamilton原理,有
\begin{align*}
	\delta S & = \sum_{\alpha=1}^r \frac{\pl L}{\pl q_\alpha} \delta q_\alpha\bigg|_{t_1}^{t_2} + \int_{t_1}^{t_2} \left(\frac{\pl L}{\pl q_\alpha} - \frac{\mathrm{d}}{\mathrm{d} t} \frac{\pl L}{\pl \dot{q}_\alpha} \right) \delta q_\alpha \mathrm{d} t \\
	& = \int_{t_1}^{t_2} \left(\frac{\pl L}{\pl q_\alpha} - \frac{\mathrm{d}}{\mathrm{d} t} \frac{\pl L}{\pl \dot{q}_\alpha} \right) \delta q_\alpha \mathrm{d} t = 0
\end{align*}
此处$q_\alpha$不相互独立,因而不能通过上式直接得出系统方程。引入不定乘子$\lambda_j(t)$可有
\begin{equation*}
	\delta \int_{t_1}^{t_2} \left(L+\sum_{j=1}^l \lambda_j(t) \bar{f}_j\right) \mathrm{d} t = \int_{t_1}^{t_2} \sum_{\alpha=1}^r \left(\frac{\pl L}{\pl q_\alpha} - \frac{\mathrm{d}}{\mathrm{d} t} \frac{\pl L}{\pl \dot{q}_\alpha} + \sum_{j=1}^l \lambda_j \frac{\pl \bar{f}_j}{\pl q_\alpha} \right) \delta q_\alpha \mathrm{d} t = 0
\end{equation*}
由此可得
\begin{equation}
	\begin{cases}
		\displaystyle \frac{\mathrm{d}}{\mathrm{d} t} \frac{\pl L}{\pl \dot{q}_\alpha} - \frac{\pl L}{\pl q_\alpha} = \sum_{j=1}^l \lambda_j \frac{\pl \bar{f}_j}{\pl q_\alpha},& \alpha=1,2,\cdots,r \\
		\bar{f}_j(\mbf{q},t) = 0, & j=1,2,\cdots,l
	\end{cases}
\end{equation}
同样可以通过Newton第二定律得到广义约束力的表达式为
\begin{equation*}
	R_\alpha = \sum_{j=1}^l \lambda_j \frac{\pl \bar{f}_j}{\pl q_\alpha}
\end{equation*}
则有
\begin{equation}
	R_{\alpha j} = \lambda_j \frac{\pl \bar{f}}{\pl q_\alpha}
\end{equation}
为第$j$个约束条件对应约束力的第$\alpha$个广义力分量。改写约束力为
\begin{equation*}
	R_\alpha = \sum_{j=1}^l \lambda_j \frac{\pl \bar{f}_j}{\pl q_\alpha} = -\frac{\pl}{\pl q_\alpha} \left(-\sum_{j=1}^l \lambda_j \bar{f}_j\right)
\end{equation*}
故标量函数
\begin{equation}
	W(\mbf{q},t) = -\sum_{j=1}^l \lambda_j \bar{f}_j(\mbf{q},t)
\end{equation}
称为{\heiti 约束力势}。

此时也可以将约束力势当作外力的势能加入Lagrange函数中,即
\begin{equation*}
	\bar{L}(\mbf{q},\dot{\mbf{q}},t;\mbf{\lambda}) = T-U-W = L(\mbf{q},\dot{\mbf{q}},t) + \sum_{j=1}^l \lambda_j \bar{f}_j (\mbf{q},t)
\end{equation*}
于是,根据Hamilton原理,可有
\begin{align*}
	\delta S & = \int_{t_1}^{t_2} \delta \bar{L} \mathrm{d} t = \int_{t_1}^{t_2} \left(\sum_{\alpha=1}^r \frac{\pl \bar{L}}{\pl q_\alpha} \delta q_\alpha + \sum_{\alpha=1}^r \frac{\pl \bar{L}}{\pl \dot{q}_\alpha} \delta \dot{q}_\alpha + \sum_{j=1}^l \frac{\pl \bar{L}}{\pl \lambda_j} \delta \lambda_j \right) \mathrm{d} t \\
	& = \int_{t_1}^{t_2} \left[-\sum_{\alpha=1}^r \left(\frac{\mathrm{d}}{\mathrm{d} t} \frac{\pl \bar{L}}{\pl \dot{q}_\alpha} - \frac{\pl \bar{L}}{\pl q_\alpha}\right)\delta q_\alpha + \sum_{j=1}^l \frac{\pl \bar{L}}{\pl \lambda_j} \delta \lambda_j \right] \mathrm{d} t = 0
\end{align*}
由此可得
\begin{equation}
	\begin{cases}
		\displaystyle \frac{\mathrm{d}}{\mathrm{d} t} \frac{\pl \bar{L}}{\pl \dot{q}_\alpha} - \frac{\pl \bar{L}}{\pl q_\alpha} = 0, & \alpha =1,2,\cdots,r \\[1.5ex]
		\displaystyle \frac{\pl \bar{L}}{\pl \lambda_j} = 0,& j = 1,2,\cdots,l
	\end{cases}
\end{equation}

\begin{example}
质量为$m$的质点从半径为$R$的光滑半球面顶端零初速滑下,求脱离球面时的位置。
\begin{figure}[htb]
\centering
\begin{asy}
	texpreamble("\usepackage{xeCJK}");
	texpreamble("\setCJKmainfont{SimSun}");
	usepackage("amsmath");
	import graph;
	import math;
	size(200);
	//第三章例7图
	pair O;
	real R,r,theta;
	R = 3;
	theta = 60;
	r = 0.15;
	O = (0,0);
	draw(arc(O,R,0,180)--cycle);
	draw(Label("$R$",MidPoint,W),O--R*dir(90),dashed);
	draw(O--R*dir(theta),dashed);
	draw(Label("$\theta$",MidPoint,Relative(W)),arc(O,0.6,90,theta),Arrow);
	fill(shift((R+r)*dir(theta))*scale(r)*unitcircle,black);
	label("$m$",(R+r)*dir(theta),3*E);
	draw(Label("$\boldsymbol{N}$",EndPoint),(R+r)*dir(theta)--(R+r)*dir(theta)+dir(theta),Arrow);
\end{asy}
\caption{例\theexample}
\label{第三章例7图}
\end{figure}
\end{example}

\begin{solution}
本例约束为可解约束$r - R \geqslant 0$。未解时约束为$r-R=0$,约束力为$\mbf{N}$,约束解除的条件为$\mbf{N} = \mbf{0}$。

除去约束,选择广义坐标为$r,\theta$,则Lagrange函数为
\begin{equation*}
	\bar{L} = \frac12 m(\dot{r}^2+r^2 \dot{\theta}) - mgr\cos \theta + \lambda(r-R)
\end{equation*}
则有系统方程为
\begin{equation*}
	\begin{cases}
		\displaystyle \frac{\mathrm{d}}{\mathrm{d} t} \frac{\pl \bar{L}}{\pl \dot{r}} - \frac{\pl \bar{L}}{\pl r} = m(\ddot{r} - r\dot{\theta}^2) + mg\cos \theta - \lambda = 0 \\[1.5ex]
		\displaystyle \frac{\mathrm{d}}{\mathrm{d} t} \frac{\pl \bar{L}}{\pl \dot{\theta}} - \frac{\pl \bar{L}}{\pl \theta} = \frac{\mathrm{d}}{\mathrm{d} t} (mr^2\dot{\theta}) - mgr\sin \theta = 0 \\[1.5ex]
		\displaystyle \frac{\pl \bar{L}}{\pl \lambda} = r-R = 0
	\end{cases}
\end{equation*}
\end{solution}
由第三式可得$r=R$,再由前两式消去$r$,可得
\begin{equation*}
	\begin{cases}
		mR\dot{\theta}^2 - mg\cos \theta + \lambda = 0 \\
		mR^2 \ddot{\theta} - mgR\sin \theta = 0
	\end{cases}
\end{equation*}
根据$\ddot{\theta} = \dot{\theta} \dfrac{\mathrm{d} \dot{\theta}}{\mathrm{d} \theta}$,上面第二式可以化为
\begin{equation*}
	\mathrm{d} \left(\frac12 mR^2 \dot{\theta}^2 + mgR\cos \theta\right) = 0
\end{equation*}
根据初始条件$\dot{\theta} \big|_{\theta=0} = 0$可得
\begin{equation*}
	\frac12 mR^2 \dot{\theta}^2 + mgR\cos \theta = mgR
\end{equation*}
由此可解得
\begin{equation*}
	\lambda = mg(3\cos \theta-2)
\end{equation*}
约束力为
\begin{equation*}
	\begin{cases}
		N_r = \lambda \dfrac{\pl}{\pl r}(r-R) = \lambda \\[1.5ex]
		N_\theta = \lambda \dfrac{\pl}{\pl \theta}(r-R) = 0
	\end{cases}
\end{equation*}
由此脱离条件为
\begin{equation*}
	\lambda = mg(3\cos \theta-2) = 0
\end{equation*}
即$\theta = \arccos \dfrac23$时,此时高度为$h = R\cos \theta = \dfrac23 R$。

\subsection{Lagrange乘子法求解线性非完整体系的运动}

设非完整体系有完整约束
\begin{equation*}
	f_j (\mbf{r},t) = 0,\quad j = 1,2,\cdots,k
\end{equation*}
以及非完整约束
\begin{equation*}
	\sum_{i=1}^n \mbf{A}_{ji}(\mbf{r},t) \cdot \dot{\mbf{r}}_i + A_{j0}(\mbf{r},t) = 0,\quad j = 1,2,\cdots,k'
\end{equation*}
引入广义坐标$q_\alpha,\,\,\alpha = 1,2,\cdots,s=3n-k$消去完整约束,即
\begin{equation*}
	f_j(\mbf{r}(\mbf{q},t),t) \equiv 0,\quad j=1,2,\cdots,k
\end{equation*}
考虑
\begin{equation*}
	\dot{\mbf{r}}_i = \sum_{\alpha=1}^s \frac{\pl \mbf{r}_i}{\pl q_\alpha} \dot{q}_\alpha + \frac{\pl \mbf{r}_i}{\pl t}
\end{equation*}
则非完整约束变换为
\begin{align}
	\sum_{i=1}^n \mbf{A}_{ji} \cdot \dot{\mbf{r}}_i + A_{j0}(\mbf{r},t) & = \sum_{i=1}^n \mbf{A}_{ji} \cdot \left(\sum_{\alpha=1}^s \frac{\pl \mbf{r}_i}{\pl q_\alpha} \dot{q}_\alpha + \frac{\pl \mbf{r}_i}{\pl t}\right) + A_{j0}(\mbf{r},t) \nonumber \\
	& = \sum_{\alpha=1}^s \bar{A}_{j\alpha}(\mbf{q},t) \dot{q}_\alpha + \bar{A}_{j0}(\mbf{q},t) = 0,\quad j=1,2,\cdots,k'
	\label{Lagrange乘子法解非完整系统的非完整约束}
\end{align}
其中
\begin{equation*}
	\begin{cases}
		\displaystyle \bar{A}_{j\alpha}(\mbf{q},t) = \sum_{i=1}^n \mbf{A}_{ji}(\mbf{r}(\mbf{q},t),t) \cdot \frac{\pl \mbf{r}_i}{\pl q_\alpha} \\[1.5ex]
		\displaystyle \bar{A}_{j0}(\mbf{q},t) = \sum_{i=1}^n \mbf{A}_{ji}(\mbf{r}(\mbf{q},t),t) \cdot \frac{\pl \mbf{r}_i}{\pl t} + A_{j0}(\mbf{r}(\mbf{q},t),t)
	\end{cases}
\end{equation*}
式\eqref{Lagrange乘子法解非完整系统的非完整约束}所示的非完整约束可以等价的表示为
\begin{align}
	\sum_{\alpha=1}^s \bar{A}_{j\alpha}(\mbf{q},t) \mathrm{d} q_\alpha + \bar{A}_{j0}(\mbf{q},t) \mathrm{d} t = 0,\quad j=1,2,\cdots,k'
\end{align}
或者
\begin{align}
	\sum_{\alpha=1}^s \bar{A}_{j\alpha}(\mbf{q},t) \delta q_\alpha = 0,\quad j=1,2,\cdots,k'
	\label{Lagrange乘子法解非完整系统的非完整约束的变分形式}
\end{align}
Hamilton作用量可以表示为
\begin{equation*}
	S[\mbf{q}(t)] = \int_{t_1}^{t_2} L(\mbf{q},\dot{\mbf{q}},t) \mathrm{d} t
\end{equation*}
同样可以根据Hamilton原理得到
\begin{equation*}
	\delta S = \int_{t_1}^{t_2} \sum_{\alpha=1}^s \left(\frac{\pl L}{\pl q_\alpha} - \frac{\mathrm{d}}{\mathrm{d} t} \frac{\pl L}{\pl \dot{q}_\alpha}\right) \delta q_\alpha \mathrm{d} t = 0
\end{equation*}
由于此处$\delta q_\alpha$不是互相独立的,还要满足式\eqref{Lagrange乘子法解非完整系统的非完整约束的变分形式},由此可引入不定乘子$\lambda_j$,即有
\begin{align*}
	\int_{t_1}^{t_2} \left(\delta L + \sum_{j=1}^{k'} \lambda_j \sum_{\alpha=1}^s \bar{A}_{j\alpha} \delta q_\alpha\right) \mathrm{d} t = \int_{t_1}^{t_2} \sum_{\alpha=1}^s \left(\frac{\pl L}{\pl q_\alpha} - \frac{\mathrm{d}}{\mathrm{d} t} \frac{\pl L}{\pl \dot{q}_\alpha} + \sum_{j=1}^{k'} \lambda_j \bar{A}_{j\alpha} \right) \delta q_\alpha \mathrm{d} t = 0
\end{align*}
由此可有系统方程
\begin{equation}
	\begin{cases}
		\displaystyle \frac{\mathrm{d}}{\mathrm{d} t} \frac{\pl L}{\pl \dot{q}_\alpha} - \frac{\pl L}{\pl q_\alpha} = \sum_{j=1}^{k'} \lambda_j \bar{A}_{j\alpha},& \alpha = 1,2,\cdots,s \\[1.5ex]
		\displaystyle \sum_{\alpha=1}^s \bar{A}_{j\alpha} \dot{q}_\alpha + \bar{A}_{j0} = 0,& j = 1,2,\cdots,k'
	\end{cases}
\end{equation}

\begin{example}[直立圆盘在水平面上的纯滚动]
\begin{figure}[htb]
\centering
\begin{asy}
	texpreamble("\usepackage{xeCJK}");
	texpreamble("\setCJKmainfont{SimSun}");
	usepackage("amsmath");
	import graph;
	import math;
	size(200);
	//纯滚动直立圆盘
	real hd,hdd;
	pair O,x,y,z,i,j,k;
	hd = 131+25/60;
	hdd = 7+10/60;
	O = (0,0);
	x = 1.2*dir(90+hd);
	y = 1.5*dir(-hdd);
	z = 1.5*dir(90);
	draw(Label("$x$",EndPoint),O--x,Arrow);
	draw(Label("$y$",EndPoint),O--y,Arrow);
	draw(Label("$z$",EndPoint),O--z,Arrow);
	label("$O$",O,NW);
	i = (-sqrt(2)/4,-sqrt(14)/12);
	j = (sqrt(14)/4,-sqrt(2)/12);
	k = (0,2*sqrt(2)/3);
	
	real x0,y0,r,phi;
	pair P;
	pair planecur(real xx){
		real yy = y0+1*tan(xx-x0);
		return xx*i+yy*j;
	}
	pair dplanecur(real xx){
		real yy = 1/((cos(xx-x0))**2);
		return xx*i+yy*j;
	}
	pair wheel(real theta){
		real xx,yy,zz;
		xx = x0+r*sin(theta)*cos(phi);
		yy = y0+r*sin(theta)*sin(phi);
		zz = r+r*cos(theta);
		return xx*i+yy*j+zz*k;
	}
	real getphi(real xx){
		real yy = 1/((cos(xx-x0))**2);
		return atan(yy/xx);
	}
	x0 = 1;
	y0 = 0.7;
	r = 0.5;
	phi = getphi(x0);
	unfill(graph(wheel,0,2*pi)--cycle);
	draw(graph(wheel,0,2*pi),linewidth(0.8bp));
	draw(graph(planecur,x0-1,x0+1));
	P = planecur(x0);
	dot(P);
	draw((P-0.7*dplanecur(x0))--(P+dplanecur(x0)),dashed);
	draw(P--P+r*k--wheel(-1.2));
	label("$(x,y)$",P+r*k,E);
	label("$\theta$",P+r*k,WSW);
	label("$\phi$",P-0.7*dplanecur(x0),2*S);
	
	//draw(O--(1.7,0),invisible);
\end{asy}
\caption{例\theexample}
\label{第三章例8图}
\end{figure}
\end{example}
\begin{solution}
此非完整体系有完整约束
\begin{equation*}
	z-R = 0
\end{equation*}
以及非完整约束
\begin{equation*}
	\begin{cases}
		\dot{x} - R\dot{\theta} \cos \phi = 0 \\
		\dot{y} - R\dot{\theta} \sin \phi = 0
	\end{cases}
\end{equation*}
由此可将广义坐标取为$x,y,\theta,\phi$,体系的Lagrange函数为
\begin{equation*}
	L = T = \frac12 m(\dot{x}^2+\dot{y}^2) + \frac12 I_1 \dot{\theta}^2 + \frac12 I_2 \dot{\theta}^2
\end{equation*}
考虑到
\begin{align*}
	& \bar{A}_{11} = 1,\quad \bar{A}_{12} = 0,\quad \bar{A}_{13} = -R\cos \phi,\quad \bar{A}_{14} = 0 \\
	& \bar{A}_{21} = 0,\quad \bar{A}_{22} = 1,\quad \bar{A}_{23} = -R\sin \phi,\quad \bar{A}_{24} = 0
\end{align*}
由此可以写出系统方程为
\begin{subnumcases}{}
%\renewcommand{\theequation}{\theparentequation-\arabic{equation}}
	m\ddot{x} = \lambda_1 \label{方程第1式} \\
	m\ddot{y} = \lambda_2 \label{方程第2式} \\
	I_1 \ddot{\theta} = -R(\lambda_1 \cos \phi + \lambda_2 \sin \phi) \label{方程第3式} \\
	I_2 \ddot{\phi} = 0 \label{方程第4式} \\
	\dot{x} - R\dot{\theta} \cos \phi = 0 \label{方程第5式} \\
	\dot{y} - R\dot{\theta} \sin \phi = 0 \label{方程第6式} 
\end{subnumcases}
由式\eqref{方程第4式}可得
\begin{equation*}
	\phi = \omega_2 t + \phi_0
\end{equation*}
考虑$\text{式}\eqref{方程第1式} \times R\cos\phi + \text{式}\eqref{方程第2式} \times R\sin \phi + \text{式}\eqref{方程第3式}$可得
\begin{equation}
	mR(\ddot{x} \cos \phi+\ddot{y}\sin \phi) + I_1 \ddot{\theta} = 0
	\label{方程第7式}
\end{equation}
再考虑$\cos \phi \times \dfrac{\mathrm{d}}{\mathrm{d} t}\text{式}\eqref{方程第5式} + \sin \phi \times \dfrac{\mathrm{d}}{\mathrm{d} t}\text{式}\eqref{方程第6式}$可得
\begin{equation}
	\ddot{x} \cos \phi+\ddot{y}\sin \phi - R\ddot{\theta} = 0
	\label{方程第8式}
\end{equation}
由式\eqref{方程第7式}和式\eqref{方程第8式}可解得
\begin{equation*}
	\theta = \omega_1 t + \theta_0
\end{equation*}
于是根据式\eqref{方程第5式}和式\eqref{方程第6式}可得
\begin{align*}
	\begin{cases}
		\dot{x} = R\omega_1 \cos(\omega_2 t + \phi_0) \\
		\dot{y} = R\omega_1 \sin(\omega_2 t + \phi_0)
	\end{cases}
\end{align*}

下面分两种情况讨论圆盘的运动,
\begin{enumerate}
	\item $\omega_2 = 0$,系统的解为
	\begin{equation}
	\begin{cases}
		x = R\omega_1 t \cos \phi_0 + x_0 \\
		y = R\omega_1 t \sin \phi_0 + y_0 \\
		\theta = \omega_1 t + \theta_0 \\
		\phi = \phi_0 \\
		\lambda_1 = 0 \\
		\lambda_2 = 0
	\end{cases}
	\end{equation}
	可见,圆盘沿直线匀速滚动。运动过程自动保证约束条件,约束力为零。
	\item $\omega_2 \neq 0$,系统的解为
	\begin{equation}
	\begin{cases}
		\displaystyle x = \frac{R\omega_1}{\omega_2} \sin(\omega_2 t + \phi_0) - \frac{R\omega_1}{\omega_2} \sin \phi_0 + x_0 \\
		\displaystyle y = -\frac{R\omega_1}{\omega_2} \cos(\omega_2 t + \phi_0) + \frac{R\omega_1}{\omega_2} \cos \phi_0 + y_0 \\
		\theta = \omega_1 t + \theta_0 \\
		\phi = \omega_2 t + \phi_0 \\
		\lambda_1 = -R\omega_1 \omega_2 \sin \phi \\
		\lambda_2 = R\omega_1 \omega_2 \cos \phi
	\end{cases}
	\end{equation}	
	可见,圆盘沿圆周匀速滚动,纯滚动条件自动保证,约束力垂直盘面,提供向心力。
\end{enumerate}
\end{solution}

\section{Poisson括号}

\subsection{Poisson括号的定义}

对两个系统状态函数$f = f(\mbf{q},\mbf{p},t)$和$g = g(\mbf{q},\mbf{p},t)$,定义
\begin{equation}
	[f,g] = \sum_{\alpha=1}^s \left(\frac{\pl f}{\pl q_\alpha} \frac{\pl g}{\pl p_\alpha} - \frac{\pl g}{\pl q_\alpha} \frac{\pl f}{\pl p_\alpha}\right) = \sum_{\alpha=1}^s \begin{vmatrix} \dfrac{\pl f}{\pl q_\alpha} & \dfrac{\pl f}{\pl p_\alpha} \\[1.5ex] \dfrac{\pl g}{\pl q_\alpha} & \dfrac{\pl g}{\pl p_\alpha} \end{vmatrix} = \sum_{\alpha=1}^s \frac{\pl (f,g)}{\pl (q_\alpha,p_\alpha)}
\end{equation}
称为{\heiti Poisson括号}。
\begin{equation}
	[q_\alpha,q_\beta] = 0,\quad [p_\alpha,p_\beta] = 0,\quad [q_\alpha,p_\beta] = \delta_{\alpha\beta}
	\label{基本Poisson括号}
\end{equation}
称为{\heiti 基本Poisson括号}。根据Poisson括号的定义,容易得到
\begin{equation}
	[q_\alpha, f] =\frac{\pl f}{\pl p_\alpha},\quad [p_\alpha,f] = -\frac{\pl f}{\pl q_\alpha}
\end{equation}

\subsection{Poisson括号的性质}

利用Poisson括号的定义和行列式的性质很容易得到Poisson括号的以下性质:
\begin{enumerate}
	\item 反对称性:
	\begin{equation*}
		[f,g] = -[g,f]
	\end{equation*}
	特别地,可有$[f,f] = 0$;
	\item 线性性:
	\begin{equation*}
		[f,C_1 g+C_2 h] = C_1[f,g] + C_2[f,h],\quad \forall \, C_1,C_2 \in \mathbb{R}
	\end{equation*}
	\item $[f,gh] = [f,g]h+g[f,h]$;
	\item Leibniz性:
	\begin{equation*}
		\frac{\pl}{\pl x} [f,g] = \left[\frac{\pl f}{\pl x},g\right] + \left[f,\frac{\pl g}{\pl x}\right],\quad x \in \{\mbf{q},\mbf{p},t\}
	\end{equation*}
	\item Jacobi恒等式:
	\begin{equation*}
		\big[f,[g,h]\big] + \big[g,[h,f]\big] + \big[h,[f,g]\big] = 0
	\end{equation*}
\end{enumerate}

\subsection{力学的Poisson括号形式}

设有函数$f = f(\mbf{q},\mbf{p},t)$,考虑
\begin{equation*}
	\frac{\mathrm{d} f}{\mathrm{d} t} = \frac{\pl f}{\pl t} + \sum_{\alpha=1}^s \left(\frac{\pl f}{\pl q_\alpha} \dot{q}_\alpha + \frac{\pl f}{\pl p_\alpha} \dot{p}_\alpha\right) = \frac{\pl f}{\pl t} + \sum_{\alpha=1}^s \left(\frac{\pl f}{\pl q_\alpha} \frac{\pl H}{\pl p_\alpha} + \frac{\pl f}{\pl p_\alpha} \frac{\pl H}{\pl q_\alpha}\right)
\end{equation*}
此处利用了Hamilton正则方程,由此得到
\begin{equation}
	\frac{\mathrm{d} f}{\mathrm{d} t} = \frac{\pl f}{\pl t} + [f,H]
\end{equation}
如果函数$f$不显含时间,即$\dfrac{\pl f}{\pl t} = 0$,则有
\begin{equation}
	\frac{\mathrm{d} f}{\mathrm{d} t} = [f,H]
\end{equation}
分别将函数$f$取为$q_\alpha$和$p_\alpha$可得{\heiti 正则方程的Poisson括号形式}
\begin{equation}
	\begin{cases}
		\displaystyle \dot{q}_\alpha = [q_\alpha,H] \\[1.5ex]
		\displaystyle \dot{p}_\alpha = [p_\alpha,H]
	\end{cases}
\end{equation}
将函数$f$取为$H$则得到
\begin{equation}
	\dot{H} = \frac{\pl H}{\pl t}
\end{equation}

如果将$[f,H]$记作$\hat{H} f$,此处$\hat{H}$称为{\heiti 算子},它代表一种操作——即求该函数与Hamilton函数的Poisson括号。如果$f$不显含时间,可有
\begin{equation*}
	\frac{\mathrm{d} f}{\mathrm{d} t} = [f,H] = \hat{H}f
\end{equation*}
依次可有
\begin{align*}
	\frac{\mathrm{d}^2 f}{\mathrm{d} t^2} & = \frac{\mathrm{d}}{\mathrm{d} t}[f,H] = [\hat{H}f,H] = \hat{H} (\hat{H} f) = \hat{H}^2 f \\
	& \vdots \\
	\frac{\mathrm{d}^n f}{\mathrm{d} t^n} & = \frac{\mathrm{d}}{\mathrm{d} t}\left(\frac{\mathrm{d}^{n-1}f}{\mathrm{d} t^{n-1}}\right) = [\hat{H}^{n-1}f,H] = \hat{H} (\hat{H}^{n-1} f) = \hat{H}^n f
\end{align*}
如果不考虑收敛的问题,任何正则变量的函数可以用Taylor级数表示为
\begin{align*}
	f(\mbf{q}(t),\mbf{p}(t)) = \sum_{n=0}^\infty \frac{1}{n!} \frac{\mathrm{d}^n f}{\mathrm{d} t^n}(0) t^n = \sum_{n=0}^\infty \frac{t^n}{n!} \hat{H}^n f(0) = \mathrm{e}^{t\hat{H}} f(0)
\end{align*}
由此,可以得到运动的算子形式解。

\begin{example}
利用算子形式求解一维谐振子,其Hamilton函数为
\begin{equation*}
	H = \frac{p^2}{2m} + \frac12 m\omega^2 q^2
\end{equation*}
\end{example}
\begin{solution}
根据基本Poisson括号
\begin{equation*}
	[q,q] = 0,\quad [p,p] = 0,\quad [q,p] = 1
\end{equation*}
可得
\begin{align*}
	[q,H] & = \frac{1}{2m}[q,p^2] + \frac12 m\omega^2 [q,q^2] = \frac{1}{2m} \bigg([q,p]p+p[q,p]\bigg) = \frac{p}{m} \\
	[p,H] & = \frac{1}{2m}[p,p^2] + \frac12 m\omega^2 [p,q^2] = \frac12 m\omega^2 \bigg([p,q]q+q[p,q]\bigg) = -m\omega^2 q
\end{align*}
由此可以依次计算
\begin{align*}
	\hat{H}q & = [q,H] = \frac{p}{m} \\
	\hat{H}^2q & = \frac{1}{m}[p,H] = -\omega^2 q \\
	\hat{H}^3q & = -\omega^2[q,H] = -\omega^2\frac{p}{m} \\
	\hat{H}^4q & = -\frac{\omega^2}{m}[p,H] = (-\omega^2)^2 q \\
	& \vdots \\
	\hat{H}^{2m-1} q & = (-\omega^2)^{m-1} \frac{p}{m} \\
	\hat{H}^{2m} q & = (-\omega^2)^m q \\
	& \vdots 
\end{align*}
故有
\begin{align*}
	q(t) & = \left[\sum_{m=0}^\infty \frac{t^{2m}}{(2m)!} \hat{H}^{2m} + \sum_{m=1} \frac{t^{2m-1}}{(2m-1)!} \hat{H}^{2m-1}\right] q(0) \\
	& = \sum_{m=0}^\infty \frac{(-1)^m(\omega t)^{2m}}{(2m)!} q(0) + \sum_{m=1}^\infty \frac{(-1)^{m-1}(\omega t)^{2m-1}}{(2m-1)!} \frac{p(0)}{m\omega} \\
	& = q(0) \cos \omega t + \frac{p(0)}{m\omega} \sin \omega t
\end{align*}
\end{solution}

\subsection{运动积分与Poisson定理}

体系力学状态随时间演变时不变的力学量称为{\heiti 运动积分},即满足
\begin{equation}
	\frac{\mathrm{d} f}{\mathrm{d} t} = \frac{\pl f}{\pl t} + [f,H] = 0
\end{equation}

\begin{theorem}[Poisson定理]
两个运动积分的Poisson括号仍为运动积分。即如有$\dfrac{\mathrm{d} f}{\mathrm{d} t} = 0,\dfrac{\mathrm{d} g}{\mathrm{d} t} = 0$,则有
\begin{equation*}
	\frac{\mathrm{d}}{\mathrm{d} t} [f,g] = 0
\end{equation*}
\end{theorem}
\begin{proof}
由于
\begin{equation*}
	\frac{\mathrm{d} f}{\mathrm{d} t} = \frac{\pl f}{\pl t} + [f,H] = 0
\end{equation*}
所以有$\dfrac{\pl f}{\pl t} = -[f,H]$,同理可有$\dfrac{\pl g}{\pl t} = -[g,H]$。由此可有
\begin{align*}
	\frac{\mathrm{d}}{\mathrm{d} t} [f,g] & = \frac{\pl}{\pl t}[f,g] + \big[[f,g],H\big] = \left[\frac{\pl f}{\pl t},g\right] + \left[f,\frac{\pl g}{\pl t}\right] + \big[[f,g],H\big] \\
	& = \big[-[f,H],g\big] + \big[f,-[g,H]\big] + \big[[f,g],H\big] \\
	& = -\big[f,[g,H]\big] - \big[g,[H,f]\big] - \big[H,[f,g]\big] = 0 \qedhere
\end{align*}
\end{proof}

\begin{example}[角动量的Poisson括号]
角动量可以表示为
\begin{equation*}
	\mbf{L} = \mbf{r} \times \mbf{p} = \begin{vmatrix} \mbf{e}_1 & \mbf{e}_2 & \mbf{e}_3 \\ x & y & z \\ p_x & p_y & p_z \end{vmatrix}
\end{equation*}
所以
\begin{align*}
	[L_x,L_y] & = [yp_z-zp_y,zp_x-xp_z] = [yp_z,zp_x] + [zp_y,xp_z] \\
	& = y[p_z,z]p_x + x[z,p_z] p_y = xp_y-yp_x = L_z
\end{align*}
同理可有
\begin{equation*}
	[L_y,L_z] = L_x,\quad [L_z,L_x] = L_y
\end{equation*}
考虑
\begin{align*}
	[L_z,L^2] & = [L_z,L_x^2] + [L_z,L_y^2] + [L_z,L_z^2] \\
	& = [L_z,L_x]L_x + L_x[L_z,L_x] + [L_z,L_y]L_y + L_y[L_z,L_y] \\
	& = L_yL_x + L_xL_y - L_xL_y - L_yL_x = 0
\end{align*}
同理可有
\begin{equation*}
	[L_x,L^2] = [L_y,L^2] = [L_z,L^2] = 0
\end{equation*}
由此可得,若角动量任意两分量守恒,则第三分量亦守恒。角动量三分量变化不独立,这与两次异轴转动可合成绕第三轴转动的事实相照应。
\end{example}

\subsection{量子Poisson括号}

量子Poisson括号定义为
\begin{equation}
	[f,g]_{-} = fg - gf
\end{equation}
在量子力学中,将正则方程中的Possion括号$[f,g]$替换为$\dfrac{1}{\mathrm{i}\hbar} [f,g]_{-}$的过程,称为{\heiti 正则量子化}。力学量的乘积一般非对易,可用算符或矩阵表示。

基本Poisson括号表示为
\begin{equation}
	[q_i,p_j]_{-} = \mathrm{i} \hbar \delta_{ij},\quad [q_i,q_j]_{-} = 0,\quad [p_i,p_j]_{-} = 0,\quad i,j=1,2,3
\end{equation}
可有{\heiti Heisenberg方程}
\begin{equation}
	\frac{\mathrm{d} f}{\mathrm{d} t} = \frac{\pl f}{\pl t} + \frac{1}{\mathrm{i} \hbar} [f,H]_{-}
\end{equation}
取宏观极限
\begin{equation*}
	\lim_{\hbar \to 0} \frac{1}{\mathrm{i} \hbar} [f,g]_{-} = [f,g]
\end{equation*}
时,动力学过渡为经典力学。

\section{正则变换}

对称性越高,方程的求解越简便。对称性可以通过变换发现。

\subsection{点变换}

设在一组广义坐标$\mbf{q}$下,Lagrange函数形式为
\begin{equation*}
	L = L_q(\mbf{q},\dot{\mbf{q}},t)
\end{equation*}
系统方程为
\begin{equation*}
	\frac{\mathrm{d}}{\mathrm{d} t} \frac{\pl L}{\pl \dot{q}_\alpha} - \frac{\pl L}{\pl q_\alpha} = 0,\quad \alpha = 1,2,\cdots,s
\end{equation*}
做广义坐标变换
\begin{equation*}
	\mbf{q} = \mbf{q}(\mbf{Q},t)
\end{equation*}
则有
\begin{equation*}
	L = L_Q(\mbf{Q},\dot{\mbf{Q}},t) = L_q(\mbf{q}(\mbf{Q},t),\dot{\mbf{q}}(\mbf{Q},\dot{\mbf{Q}},t),t)
\end{equation*}
所以有
\begin{align*}
	\frac{\pl L}{\pl \dot{Q}_\alpha} & = \sum_{\beta=1}^s \frac{\pl L}{\pl \dot{q}_\beta} \frac{\pl \dot{q}_\beta}{\pl \dot{Q}_\alpha} \\
	\frac{\pl L}{\pl Q_\alpha} & = \sum_{\beta=1}^s \frac{\pl L}{\pl q_\beta} \frac{\pl q_\beta}{\pl Q_\alpha} + \sum_{\alpha=1}^s \frac{\pl L}{\pl \dot{q}_\beta} \frac{\pl \dot{q}_\beta}{\pl Q_\alpha} 
\end{align*}
由此可得
\begin{align*}
	\frac{\mathrm{d}}{\mathrm{d} t} \frac{\pl L}{\pl \dot{Q}_\alpha} - \frac{\pl L}{\pl Q_\alpha} & = \frac{\mathrm{d}}{\mathrm{d} t} \left(\sum_{\beta=1}^s \frac{\pl L}{\pl \dot{q}_\beta} \frac{\pl \dot{q}_\beta}{\pl \dot{Q}_\alpha}\right) - \sum_{\beta=1}^s \frac{\pl L}{\pl q_\beta} \frac{\pl q_\beta}{\pl Q_\alpha} - \sum_{\alpha=1}^s \frac{\pl L}{\pl \dot{q}_\beta} \frac{\pl \dot{q}_\beta}{\pl Q_\alpha} \\
	& = \sum_{\alpha=1}^s \left[\frac{\mathrm{d}}{\mathrm{d} t} \left(\frac{\pl L}{\pl \dot{q}_\beta}\right) \frac{\pl \dot{q}_\beta}{\pl \dot{Q}_\alpha} - \frac{\pl L}{\pl q_\beta} \frac{\pl q_\beta}{\pl Q_\alpha}\right] + \sum_{\beta=1}^s \frac{\pl L}{\pl \dot{q}_\beta} \left(\frac{\mathrm{d}}{\mathrm{d} t} \frac{\pl q_\beta}{\pl Q_\alpha} - \frac{\pl \dot{q}_\beta}{\pl Q_\alpha}\right) \\
	& = \sum_{\alpha=1}^s \left(\frac{\mathrm{d}}{\mathrm{d} t} \frac{\pl L}{\pl \dot{q}_\beta} - \frac{\pl L}{\pl q_\beta}\right) \frac{\pl q_\beta}{\pl Q_\alpha} = 0
\end{align*}
此处利用了经典Lagrange关系
\begin{equation*}
	\frac{\pl \dot{q}_\beta}{\pl \dot{Q}_\alpha} = \frac{\pl q_\beta}{\pl Q_\alpha},\quad \frac{\pl \dot{q}_\beta}{\pl Q_\alpha} = \frac{\mathrm{d}}{\mathrm{d} t} \frac{\pl q_\beta}{\pl Q_\alpha}
\end{equation*}
可见,做广义坐标变换,Lagrange方程不变。

\subsection{正则变换}

在Hamilton力学中,正则变量处于平等地位,变换的自由度更大。做正则变量变换
\begin{equation}
	\begin{cases}
		Q_\alpha = Q_\alpha(\mbf{q},\mbf{p},t), \\
		P_\alpha = P_\alpha(\mbf{q},\mbf{p},t),
	\end{cases}\quad \alpha = 1,2,\cdots,s
	\label{正则坐标变换}
\end{equation}
在变换\eqref{正则坐标变换}下,Hamilton函数将变换为
\begin{equation*}
	K = K(\mbf{Q},\mbf{P},t) = H(\mbf{q}(\mbf{Q},\mbf{P},t),\mbf{p}(\mbf{Q},\mbf{P},t),t)
\end{equation*}
一般情况下,在变换之后正则方程将不能保持原来的形式。如果在变换\eqref{正则坐标变换}下,新的Hamilton函数$K(\mbf{Q},\mbf{P},t)$仍然满足正则方程
\begin{equation*}
	\begin{cases}
		\displaystyle \dot{Q}_\alpha = \frac{\pl K}{\pl P_\alpha}, \\[1.5ex]
		\displaystyle \dot{P}_\alpha = -\frac{\pl K}{\pl Q_\alpha},
	\end{cases}\quad \alpha =1,2,\cdots,s
\end{equation*}
这种变换称为{\heiti 正则变换}。

\subsection{正则变换的条件}

由于正则方程与Hamilton原理是等价的,因此只要变换\eqref{正则坐标变换}可以满足Hamilton原理的要求,就可以保证新的正则方程形式保持不变。

在变换前后,正则变量都需要满足固定边界条件,即
\begin{align*}
	\begin{cases}
		\delta\mbf{q}(t_1) = \delta \mbf{q}(t_2) = 0, \\
		\delta\mbf{p}(t_1) = \delta \mbf{p}(t_2) = 0,
	\end{cases}\quad 
	\begin{cases}
		\delta\mbf{Q}(t_1) = \delta \mbf{Q}(t_2) = 0, \\
		\delta\mbf{P}(t_1) = \delta \mbf{P}(t_2) = 0,
	\end{cases}
\end{align*}
根据相空间Hamilton原理,有
\begin{align}
	& \delta \int_{t_1}^{t_2} \left[\sum_{\alpha=1}^s p_\alpha \dot{q}_\alpha - H(\mbf{q},\mbf{p},t)\right] \mathrm{d} t = 0 \label{正则变换前的Hamilton原理} %\\
	%& \delta \int_{t_1}^{t_2} \left[\sum_{\alpha=1}^s P_\alpha \dot{Q}_\alpha - H(\mbf{Q},\mbf{P},t)\right] \mathrm{d} t = 0 \label{正则变换后的Hamilton原理}
\end{align}
由于$L$具有规范不定性,即$L$和$L+\dfrac{\mathrm{d} f}{\mathrm{d} t}(\mbf{q},t)$是等价的,而且考虑到式\eqref{正则变换前的Hamilton原理}中的被积函数加上一项全微分之后,定积分$\displaystyle \int_{t_1}^{t_2} \mathrm{d} f$只和端点的值有关,而端点的变分恒为零,因此$f$不仅可以是$\mbf{q}$和$t$的函数,还可以将$\mbf{p}$作为参数包含在内。即正则变换前的Hamilton原理应该表达为
\begin{equation}
	\delta \int_{t_1}^{t_2} \left[\sum_{\alpha=1}^s p_\alpha \dot{q}_\alpha - H(\mbf{q},\mbf{p},t) + \frac{\mathrm{d} f_1}{\mathrm{d} t}(\mbf{q},\mbf{p},t)\right] \mathrm{d} t = 0 \label{正则变换前的Hamilton原理一般形式}
\end{equation}
因此正则变换之后的Hamilton原理应该表达为
\begin{equation}
	\delta \int_{t_1}^{t_2} \left[\sum_{\alpha=1}^s P_\alpha \dot{Q}_\alpha - K(\mbf{Q},\mbf{P},t) + \frac{\mathrm{d} f_2}{\mathrm{d} t}(\mbf{Q},\mbf{P},t)\right] \mathrm{d} t = 0 \label{正则变换前的Hamilton原理一般形式}
\end{equation}
因此,可有
\begin{equation}
	\delta \int_{t_1}^{t_2} \left[\lambda \left(\sum_{\alpha=1}^s p_\alpha \dot{q}_\alpha - H\right) - \left(\sum_{\alpha=1}^s P_\alpha \dot{Q}_\alpha - K\right) + \lambda \frac{\mathrm{d} f_1}{\mathrm{d} t} - \frac{\mathrm{d} f_2}{\mathrm{d} t}\right] \mathrm{d} t = 0
\end{equation}
记
\begin{equation}
	F_1 = f_2(\mbf{Q},\mbf{P},t) - \lambda f_1(\mbf{q},\mbf{p},t)
\end{equation}
则有
\begin{equation*}
	\delta \int_{t_1}^{t_2} \left[\lambda \left(\sum_{\alpha=1}^s p_\alpha \dot{q}_\alpha - H(\mbf{q},\mbf{p},t)\right) - \left(\sum_{\alpha=1}^s P_\alpha \dot{Q}_\alpha - K(\mbf{Q},\mbf{P},t)\right) - \frac{\mathrm{d} F_1}{\mathrm{d} t}(\mbf{Q},\mbf{P},t)\right] \mathrm{d} t = 0
\end{equation*}
要使上式成立,只要取
\begin{equation}
	\lambda \left(\sum_{\alpha=1}^s p_\alpha \dot{q}_\alpha - H(\mbf{q},\mbf{p},t)\right) = \sum_{\alpha=1}^s P_\alpha \dot{Q}_\alpha - K(\mbf{Q},\mbf{P},t) + \dot{F_1}(\mbf{Q},\mbf{P},t) = 0
	\label{正则变换需满足的一般条件1}
\end{equation}
式\eqref{正则变换需满足的一般条件1}中$\lambda$称为{\heiti 标度因子},表示不同的单位度量引起的数值放大,不影响物理实质,因此可取$\lambda=1$,则有
\begin{equation}
	\mathrm{d} F_1 = \sum_{\alpha=1}^s p_\alpha \mathrm{d} q_\alpha - \sum_{\alpha=1}^s P_\alpha \mathrm{d} Q_\alpha + (K-H) \mathrm{d} t
	\label{正则变换需满足的一般条件}
\end{equation}
满足式\eqref{正则变换需满足的一般条件}的变换即为{\heiti 正则变换}。如果$\lambda \neq 1$,则为{\heiti 拓展正则变换}。

\subsection{母函数}

\subsubsection{第一类母函数}

考虑式\eqref{正则变换需满足的一般条件}的右端,同时出现新旧两组正则变量$\mbf{q},\mbf{p}$和$\mbf{Q},\mbf{P}$,但这$4s$个变量之间由于存在$2s$个变换方程\eqref{正则坐标变换},因此独立的变量只有$2s$个。在式\eqref{正则变换需满足的一般条件}中,独立的变量是正则坐标$\mbf{q}$和$\mbf{Q}$,因此函数$F_1$也应取$\mbf{q}$和$\mbf{Q}$加上时间$t$作为自己的变量,即
\begin{equation}
	\mathrm{d} F_1(\mbf{q},\mbf{Q},t) = \sum_{\alpha=1}^s p_\alpha \mathrm{d} q_\alpha - \sum_{\alpha=1}^s P_\alpha \mathrm{d} Q_\alpha + (K-H) \mathrm{d} t
	\label{第一类母函数下的正则变换}
\end{equation}
函数$F_1 = F_1(\mbf{q},\mbf{Q},t)$称为{\heiti 第一类母函数}。考虑
\begin{equation*}
	\mathrm{d} F_1 = \sum_{\alpha=1}^s \frac{\pl F_1}{\pl q_\alpha} \mathrm{d} q_\alpha + \sum_{\alpha=1}^s \frac{\pl F_1}{\pl Q_\alpha} \mathrm{d} Q_\alpha + \frac{\pl F_1}{\pl t} \mathrm{d} t
\end{equation*}
与式\eqref{正则变换需满足的一般条件}对比,可得变换的具体形式
\begin{equation}
	\begin{cases}
		\displaystyle p_\alpha = \frac{\pl F_1}{\pl q_\alpha} = \bar{p}_\alpha(\mbf{q},\mbf{Q},t) \\[1.5ex]
		\displaystyle P_\alpha = -\frac{\pl F_1}{\pl Q_\alpha} = \bar{P}_\alpha(\mbf{q},\mbf{Q},t) \\[1.5ex]
		\displaystyle K = H+\frac{\pl F_1}{\pl t}
	\end{cases}
\end{equation}

\subsubsection{第二类母函数}

考虑
\begin{equation*}
	\sum_{\alpha=1}^s P_\alpha \mathrm{d} Q_\alpha = \mathrm{d} \left(\sum_{\alpha=1}^s P_\alpha Q_\alpha\right) - \sum_{\alpha=1}^s Q_\alpha \mathrm{d} P_\alpha
\end{equation*}
因此可以将式\eqref{正则变换需满足的一般条件}改写为
\begin{equation*}
	\mathrm{d} \left(F_1 + \sum_{\alpha=1}^s P_\alpha Q_\alpha\right) = \sum_{\alpha=1}^s p_\alpha \mathrm{d} q_\alpha + \sum_{\alpha=1}^s Q_\alpha \mathrm{d} P_\alpha + (K-H) \mathrm{d} t
\end{equation*}
定义函数
\begin{equation}
	F_2 = F_2(\mbf{q},\mbf{P},t) = F_1 + \sum_{\alpha=1}^s P_\alpha Q_\alpha
	\label{第二类母函数}
\end{equation}
于是得到
\begin{equation}
	\mathrm{d} F_2(\mbf{q},\mbf{P},t) = \sum_{\alpha=1}^s p_\alpha \mathrm{d} q_\alpha + \sum_{\alpha=1}^s Q_\alpha \mathrm{d} P_\alpha + (K-H) \mathrm{d} t
	\label{第二类母函数下的正则变换}
\end{equation}
由此可得变换的具体形式
\begin{equation}
	\begin{cases}
		\displaystyle p_\alpha = \frac{\pl F_2}{\pl q_\alpha} \\[1.5ex]
		\displaystyle Q_\alpha = \frac{\pl F_2}{\pl P_\alpha} \\[1.5ex]
		\displaystyle K = H + \frac{\pl F_2}{\pl t}
	\end{cases}
\end{equation}

\subsubsection{第三类母函数}

考虑
\begin{equation*}
	\sum_{\alpha=1}^s p_\alpha \mathrm{d} q_\alpha = \mathrm{d} \left(\sum_{\alpha=1}^s p_\alpha q_\alpha\right) - \sum_{\alpha=1}^s q_\alpha \mathrm{d} p_\alpha
\end{equation*}
因此可以将式\eqref{正则变换需满足的一般条件}改写为
\begin{equation*}
	\mathrm{d} \left(F_1 - \sum_{\alpha=1}^s p_\alpha q_\alpha\right) = -\sum_{\alpha=1}^s q_\alpha \mathrm{d} p_\alpha - \sum_{\alpha=1}^s P_\alpha \mathrm{d} Q_\alpha + (K-H) \mathrm{d} t
\end{equation*}
定义函数
\begin{equation}
	F_3 = F_3(\mbf{p},\mbf{Q},t) = F_1 - \sum_{\alpha=1}^s p_\alpha q_\alpha
	\label{第三类母函数}
\end{equation}
于是得到
\begin{equation}
	\mathrm{d} F_3(\mbf{p},\mbf{Q},t) = -\sum_{\alpha=1}^s q_\alpha \mathrm{d} p_\alpha - \sum_{\alpha=1}^s P_\alpha \mathrm{d} Q_\alpha + (K-H) \mathrm{d} t
	\label{第三类母函数下的正则变换}
\end{equation}
由此可得变换的具体形式
\begin{equation}
	\begin{cases}
		\displaystyle q_\alpha = -\frac{\pl F_3}{\pl p_\alpha} \\[1.5ex]
		\displaystyle P_\alpha = -\frac{\pl F_3}{\pl Q_\alpha} \\[1.5ex]
		\displaystyle K = H + \frac{\pl F_3}{\pl t}
	\end{cases}
\end{equation}

\subsubsection{第四类母函数}

考虑
\begin{equation*}
	\sum_{\alpha=1}^s p_\alpha \mathrm{d} q_\alpha = \mathrm{d} \left(\sum_{\alpha=1}^s p_\alpha q_\alpha\right) - \sum_{\alpha=1}^s q_\alpha \mathrm{d} p_\alpha
\end{equation*}
因此可以将式\eqref{正则变换需满足的一般条件}改写为
\begin{equation*}
	\mathrm{d} \left(F_2 - \sum_{\alpha=1}^s p_\alpha q_\alpha\right) = -\sum_{\alpha=1}^s q_\alpha \mathrm{d} p_\alpha + \sum_{\alpha=1}^s Q_\alpha \mathrm{d} P_\alpha + (K-H) \mathrm{d} t
\end{equation*}
定义函数
\begin{equation}
	F_4 = F_4(\mbf{p},\mbf{P},t) = F_2 - \sum_{\alpha=1}^s p_\alpha q_\alpha
	\label{第四类母函数}
\end{equation}
于是得到
\begin{equation}
	\mathrm{d} F_4(\mbf{p},\mbf{P},t) = -\sum_{\alpha=1}^s q_\alpha \mathrm{d} p_\alpha + \sum_{\alpha=1}^s Q_\alpha \mathrm{d} P_\alpha + (K-H) \mathrm{d} t
	\label{第四类母函数下的正则变换}
\end{equation}
由此可得变换的具体形式
\begin{equation}
	\begin{cases}
		\displaystyle q_\alpha = -\frac{\pl F_4}{\pl p_\alpha} \\[1.5ex]
		\displaystyle Q_\alpha = \frac{\pl F_4}{\pl P_\alpha} \\[1.5ex]
		\displaystyle K = H + \frac{\pl F_4}{\pl t}
	\end{cases}
\end{equation}

特别地,如果母函数$F_i$不显含时间$t$,即$\dfrac{\pl F_i}{\pl t} = 0$,则有
\begin{equation*}
	K=H
\end{equation*}

\begin{example}[恒等变换]
恒等变换可以用第二类母函数表示为
\begin{equation*}
	F_2(\mbf{q},\mbf{P},t) = \sum_{\beta=1}^s q_\beta P_\beta
\end{equation*}
则有
\begin{equation*}
	\begin{cases}
		p_\alpha = \dfrac{\pl F_2}{\pl q_\alpha} = P_\alpha \\[1.5ex]
		Q_\alpha = \dfrac{\pl F_2}{\pl P_\alpha} = q_\alpha \\[1.5ex]
		K = H + \dfrac{\pl F_2}{\pl t} = H
	\end{cases}
\end{equation*}
用第三类母函数可以表示为
\begin{equation*}
	F_3(\mbf{q},\mbf{P},t) = -\sum_{\beta=1}^s p_\beta Q_\beta
\end{equation*}
则有
\begin{equation*}
	\begin{cases}
		q_\alpha = -\dfrac{\pl F_3}{\pl p_\alpha} = Q_\alpha \\[1.5ex]
		P_\alpha = -\dfrac{\pl F_3}{\pl Q_\alpha} = p_\alpha \\[1.5ex]
		K = H + \dfrac{\pl F_3}{\pl t} = H
	\end{cases}
\end{equation*}
\end{example}

\begin{example}[平移变换]
平移变换可以用第二类母函数表示为
\begin{equation*}
	F_2(\mbf{q},\mbf{P},t) = \sum_{\beta=1}^s \left(q_\beta P_\beta + a_\beta P_\beta - b_\beta q_\beta\right)
\end{equation*}
则有
\begin{equation*}
	\begin{cases}
		p_\alpha = \dfrac{\pl F_2}{\pl q_\alpha} = P_\alpha-b_\alpha \\[1.5ex]
		Q_\alpha = \dfrac{\pl F_2}{\pl P_\alpha} = q_\alpha+a_\alpha \\[1.5ex]
		K = H + \dfrac{\pl F_2}{\pl t} = H
	\end{cases}
\end{equation*}
\end{example}

\begin{example}[对偶变换]
对偶变换可以用第一类母函数表示为
\begin{equation*}
	F_1(\mbf{q},\mbf{Q},t) = \sum_{\beta=1}^s q_\beta Q_\beta
\end{equation*}
则有
\begin{equation*}
	\begin{cases}
		p_\alpha = \dfrac{\pl F_1}{\pl q_\alpha} = Q_\alpha \\[1.5ex]
		P_\alpha = -\dfrac{\pl F_1}{\pl Q_\alpha} = -q_\alpha \\[1.5ex]
		K = H + \dfrac{\pl F_1}{\pl t} = H
	\end{cases}
\end{equation*}
即此变换可以将广义坐标变换为广义动量,将广义动量变换为广义坐标,相当于在相空间做了一个旋转。
\end{example}

\begin{example}[点变换]
点变换可以用第二类母函数表示为
\begin{equation*}
	F_2(\mbf{q},\mbf{P},t) = \sum_{\beta=1}^s Q_\beta(\mbf{q},t) P_\beta
\end{equation*}
则有
\begin{equation*}
	\begin{cases}
		\displaystyle p_\alpha = \frac{\pl F_2}{\pl q_\alpha} = \sum_{\beta=1}^s P_\beta \frac{\pl Q_\beta}{\pl q_\alpha} \\[1.5ex]
		\displaystyle Q_\alpha = \frac{\pl F_2}{\pl P_\alpha} = Q_\alpha(\mbf{q},t) \\[1.5ex]
		\displaystyle K = H + \frac{\pl F_2}{\pl t} = H + \sum_{\beta=1}^s P_\beta \frac{\pl Q_\beta}{\pl t}
	\end{cases}
\end{equation*}
\end{example}

\begin{example}[一维谐振子的正则变换求解]
一维谐振子的Hamilton函数为
\begin{equation*}
	H = \frac{p^2}{2m} + \frac12 m\omega^2 q^2
\end{equation*}
构造正则变换,使得
\begin{equation*}
	\frac{\pl K}{\pl Q} = 0,\quad \frac{\mathrm{d} P}{\mathrm{d} t} = 0
\end{equation*}
进而使得新正则方程的求解简化。构造母函数$F_1 = F_1(q,Q,t)$,则有
\begin{equation*}
	\begin{cases}
		\displaystyle p = \frac{\pl F_1}{\pl q} \\[1.5ex]
		\displaystyle P = -\frac{\pl F_1}{\pl Q} \\[1.5ex]
		\displaystyle K = H + \frac{\pl F_1}{\pl t}
	\end{cases}
\end{equation*}
考虑到$\dfrac{\mathrm{d} P}{\mathrm{d} t} = 0$,则$P$为运动积分。由于$H$不显含时间,则$H$也为运动积分(为常数),即能量守恒,于是可以令
\begin{equation}
	p = f(P) \cos Q,\quad q = \frac{f(P)}{m\omega} \sin Q
	\label{p,q}
\end{equation}
并假设母函数$F_1$不显含时间,则
\begin{equation*}
	K = H = \frac{p^2}{2m} + \frac12 m\omega^2 q^2 = \frac{[f(P)]^2}{2m}
\end{equation*}
根据变换关系式可得
\begin{equation*}
	p = \frac{\pl F_1}{\pl q}(q,Q)
\end{equation*}
根据式\eqref{p,q}可得
\begin{equation*}
	p = m\omega q \cot Q
\end{equation*}
则可得到
\begin{equation*}
	F_1 = \frac12 m\omega q^2 \cot Q 
\end{equation*}
根据变换关系式可得
\begin{equation*}
	P = -\frac{\pl F_1}{\pl Q} = \frac{m\omega q^2}{2\sin^2 Q}
\end{equation*}
所以有
\begin{align*}
	& q = \sqrt{\frac{2P}{m\omega}} \sin Q \\
	& f(P) = \sqrt{2m\omega P}
\end{align*}
此时
\begin{equation*}
	K = H = \omega P
\end{equation*}
将体系总能量记作$E$,可有
\begin{equation*}
	P = \frac{E}{\omega}
\end{equation*}
再由
\begin{equation*}
	\dot{Q} = \frac{\pl K}{\pl P} = \omega
\end{equation*}
可得
\begin{equation*}
	Q = \omega t + \phi
\end{equation*}
因此,系统的解为
\begin{equation*}
	\begin{cases}
		q = \sqrt{\dfrac{2E}{m\omega^2}} \sin (\omega t + \phi) \\[1.5ex]
		p = \sqrt{2mE} \cos(\omega t+\phi)
	\end{cases}
\end{equation*}
\end{example}

\subsection{无穷小正则变换}

考虑第二类母函数
\begin{equation}
	F_2(\mbf{q},\mbf{P},t) = \sum_{\alpha=1}^s q_\alpha P_\alpha + \eps G(\mbf{q},\mbf{P},t),\quad \eps \to 0
\end{equation}
则有
\begin{equation*}
	\begin{cases}
		\displaystyle p_\alpha = \frac{\pl F_2}{\pl q_\alpha} = P_\alpha + \eps \frac{\pl G}{\pl q_\alpha} \\[1.5ex]
		\displaystyle Q_\alpha = \frac{\pl F_2}{\pl P_\alpha} = q_\alpha + \eps \frac{\pl G}{\pl P_\alpha} \\[1.5ex]
		\displaystyle K = H+\frac{\pl F_2}{\pl t} = H + \eps \frac{\pl G}{\pl t}
	\end{cases}
\end{equation*}
其中
\begin{equation*}
	\eps \frac{\pl G}{\pl P_\alpha} = \eps \frac{\pl G}{\pl p_\alpha} \frac{\pl p_\alpha}{\pl P_\alpha} = \eps \frac{\pl G}{\pl p_\alpha} + o(\eps)
\end{equation*}
略去高阶小量,可得{\heiti 无穷小正则变换}
\begin{subnumcases}{}
	Q_\alpha - q_\alpha = \eps \frac{\pl G}{\pl p_\alpha} \\[1.5ex]
	P_\alpha - p_\alpha = -\eps \frac{\pl G}{\pl q_\alpha} \\[1.5ex]
	K - H = \eps \frac{\pl G}{\pl t}
\end{subnumcases}
式中函数$G(\mbf{q},\mbf{p},t)$则称为{\heiti 无穷小正则变换生成元}。

无穷小正则变换为对称变换,变换后正则方程结构相同,因而运动规律也是相同的。亦即,对体系操作前后,运动是完全相同的。

考虑变换
\begin{equation*}
	(\mbf{Q},\mbf{P},t) = (\mbf{q}+\delta \mbf{q},\mbf{p}+\delta \mbf{p},t)
\end{equation*}
因此有
\begin{equation*}
	\begin{cases}
		\delta q_\alpha = Q_\alpha-q_\alpha = \eps \dfrac{\pl G}{\pl p_\alpha} \\[1.5ex]
		\delta p_\alpha = P_\alpha-p_\alpha = -\eps \dfrac{\pl G}{\pl q_\alpha} \\[1.5ex]
		K(\mbf{Q},\mbf{P},t) - H(\mbf{q},\mbf{p},t) = \eps \dfrac{\pl G}{\pl t}
	\end{cases}
\end{equation*}
由第三式即有
\begin{equation*}
	K(\mbf{Q},\mbf{P},t) - H(\mbf{q},\mbf{p},t) - \eps \frac{\pl G}{\pl t} = \delta H - \eps \frac{\pl G}{\pl t} = 0
\end{equation*}
而
\begin{align*}
	\delta H & = \sum_{\alpha=1}^s \left(\frac{\pl H}{\pl q_\alpha} \delta q_\alpha + \frac{\pl H}{\pl p_\alpha} \delta p_\alpha\right) = \sum_{\alpha=1}^s \left(\frac{\pl H}{\pl q_\alpha} \eps \frac{\pl G}{\pl p_\alpha} - \frac{\pl H}{\pl p_\alpha} \eps \frac{\pl G}{\pl q_\alpha}\alpha\right) \\
	& = \eps [H,G]
\end{align*}
所以有
\begin{equation*}
	K(\mbf{Q},\mbf{P},t) - H(\mbf{q},\mbf{p},t) - \eps \frac{\pl G}{\pl t} = -\eps \left([G,H] + \frac{\pl G}{\pl t}\right) = -\eps \frac{\mathrm{d} G}{\mathrm{d} t} = 0
\end{equation*}
因此,体系对称则无穷小正则变换生成元守恒;以运动积分为生成元的无穷小正则变换为体系对称变换。

%\begin{example}[空间平移对称性与动量守恒]
%\end{example}

\subsection{运动方程的无穷小正则变换表示}

将无穷小正则变换生成元取为$G = H(\mbf{q},\mbf{p},t)$,以及$\eps = \mathrm{d} t$,则有
\begin{equation*}
	\begin{cases}
		\displaystyle Q_\alpha(t) - q_\alpha(t) = \frac{\pl H}{\pl p_\alpha} \mathrm{d} t \\[1.5ex]
		\displaystyle P_\alpha(t) - p_\alpha(t) = -\frac{\pl H}{\pl q_\alpha} \mathrm{d} t \\[1.5ex]
		\displaystyle K(\mbf{Q},\mbf{P},t) - H(\mbf{q},\mbf{p},t) = \frac{\pl H}{\pl t} \mathrm{d} t
	\end{cases}
\end{equation*}
根据正则方程,可有
\begin{equation*}
\begin{cases}
	\displaystyle Q_\alpha(t) = q_\alpha(t) + \frac{\pl H}{\pl p_\alpha} \mathrm{d} t = q_\alpha(t) + \dot{q}_\alpha \mathrm{d} t = q_\alpha(t+\mathrm{d} t) \\[1.5ex]
	\displaystyle P_\alpha(t) = p_\alpha(t) - \frac{\pl H}{\pl q_\alpha} \mathrm{d} t = p_\alpha(t) + \dot{p}_\alpha \mathrm{d} t = p_\alpha(t+\mathrm{d} t) \\[1.5ex]
	\displaystyle \hat{K}(t) = \hat{H}(t) + \frac{\pl H}{\pl t} \mathrm{d} t = \hat{H}(t) + \dot{H} \mathrm{d} t = \hat{H}(t+\mathrm{d}t)
\end{cases}
\end{equation*}
由此,体系随时间的演化可视为一系列无穷小正则变换的累积,其生成元为Hamilton函数。

\subsection{正则变换条件的辛表示}

设变换前后的正则变量分别为
\begin{equation*}
	\mbf{\xi} = \begin{pmatrix} \mbf{q} \\ \mbf{p} \end{pmatrix} = \begin{pmatrix} q_1 \\ \vdots \\ q_s \\ p_1 \\ \vdots \\ p_s \end{pmatrix},\quad \mrm{\Xi} = \begin{pmatrix} \mbf{Q} \\ \mbf{P} \end{pmatrix} = \begin{pmatrix} Q_1 \\ \vdots \\ Q_s \\ P_1 \\ \vdots \\ P_s \end{pmatrix}
\end{equation*}
设正则变换可以表示为
\begin{equation}
	\mrm{\Xi} = \mrm{\Xi}(\mbf{\xi},t),\quad \mbf{\xi} = \mbf{\xi}(\mrm{\Xi},t)
\end{equation}
则有
\begin{equation*}
	\delta \mrm{\Xi} = \mbf{M} \delta \mbf{\xi},\quad \delta \mbf{\xi} = \mbf{W} \delta \mrm{\Xi}
\end{equation*}
其中Jacobi矩阵以及逆变换的Jacobi矩阵可以表示为
\begin{equation}
	\mbf{M} = \frac{\pl \mrm{\Xi}}{\pl \mbf{\xi}} = \begin{pmatrix} \dfrac{\pl \mbf{Q}}{\pl \mbf{q}} & \dfrac{\pl \mbf{Q}}{\pl \mbf{p}} \\[1.5ex] \dfrac{\pl \mbf{P}}{\pl \mbf{q}} & \dfrac{\pl \mbf{P}}{\pl \mbf{p}} \end{pmatrix},\quad \mbf{W} = \frac{\pl \mbf{\xi}}{\pl \mrm{\Xi}} = \begin{pmatrix} \dfrac{\pl \mbf{q}}{\pl \mbf{Q}} & \dfrac{\pl \mbf{q}}{\pl \mbf{P}} \\[1.5ex] \dfrac{\pl \mbf{p}}{\pl \mbf{Q}} & \dfrac{\pl \mbf{p}}{\pl \mbf{P}} \end{pmatrix} = \mbf{M}^{-1}
\end{equation}
用第一类母函数$F_1=F_1(\mbf{q},\mbf{Q},t)$可以将正则变换的条件表示为
\begin{equation*}
	\mathrm{d} F_1 = \sum_{\gamma=1}^s p_\gamma \mathrm{d} q_\gamma - \sum_{\zeta=1}^s P_\zeta \mathrm{d} Q_\zeta + (K-H)\mathrm{d} t
\end{equation*}
考虑到$\mbf{Q} = \mbf{Q}(\mbf{q},\mbf{p},t)$,可有
\begin{equation*}
	\mathrm{d} Q_\zeta = \sum_{\gamma=1}^s \left(\frac{\pl Q_\zeta}{\pl q_\gamma} \mathrm{d} q_\gamma + \frac{\pl Q_\zeta}{\pl p_\gamma} \mathrm{d} p_\gamma\right) + \frac{\pl Q_\zeta}{\pl t} \mathrm{d} t
\end{equation*}
所以
\begin{align*}
	\mathrm{d} \bar{F}_1(\mbf{q},\mbf{p},t) & = \sum_{\gamma=1}^s \left[\left(p_\gamma - \sum_{\zeta=1}^s P_\zeta \frac{\pl Q_\zeta}{\pl q_\gamma}\right) \mathrm{d} q_\gamma - \left(\sum_{\zeta=1}^s P_\zeta \frac{\pl Q_\zeta}{\pl p_\gamma}\right) \mathrm{d} p_\gamma\right] \\
	& \quad {}+ \left(K-H - \sum_{\zeta=1}^s P_\zeta \frac{\pl Q_\zeta}{\pl t}\right) \mathrm{d} t
\end{align*}
此处,函数$\bar{F}_1(\mbf{q},\mbf{p},t) = F_1(\mbf{q},\mbf{Q}(\mbf{q},\mbf{p},t),t)$,则有
\begin{equation*}
\begin{array}{ll}
	\displaystyle \frac{\pl \bar{F}_1}{\pl q_\alpha} = p_\alpha - \sum_{\zeta=1}^s P_\zeta \frac{\pl Q_\zeta}{\pl q_\alpha}, & \displaystyle \frac{\pl \bar{F}_1}{\pl q_\beta} = p_\beta - \sum_{\zeta=1}^s P_\zeta \frac{\pl Q_\zeta}{\pl q_\beta} \\[1.5ex]
	\displaystyle \frac{\pl \bar{F}_1}{\pl p_\alpha} = -\sum_{\zeta=1}^s P_\zeta \frac{\pl Q_\zeta}{\pl p_\alpha}, & \displaystyle \frac{\pl \bar{F}_1}{\pl p_\beta} = -\sum_{\zeta=1}^s P_\zeta \frac{\pl Q_\zeta}{\pl p_\beta}
\end{array}
\end{equation*}
根据$\dfrac{\pl^2 \bar{F}_1}{\pl q_\alpha \pl q_\beta} = \dfrac{\pl^2 \bar{F}_1}{\pl q_\beta \pl q_\alpha}$可得
\begin{equation*}
	-\sum_{\zeta=1}^s \frac{\pl P_\zeta}{\pl q_\beta} \frac{\pl Q_\zeta}{\pl q_\alpha} - \sum_{\zeta=1}^s P_\zeta \frac{\pl^2 Q_\zeta}{\pl q_\alpha \pl q_\beta} = -\sum_{\zeta=1}^s \frac{\pl P_\zeta}{\pl q_\alpha} \frac{\pl Q_\zeta}{\pl q_\beta} - \sum_{\zeta=1}^s P_\zeta \frac{\pl^2 Q_\zeta}{\pl q_\beta \pl q_\alpha}
\end{equation*}
所以有
\begin{equation*}
	\sum_{\zeta=1}^s \frac{\pl P_\zeta}{\pl q_\beta} \frac{\pl Q_\zeta}{\pl q_\alpha} = \sum_{\zeta=1}^s \frac{\pl P_\zeta}{\pl q_\alpha} \frac{\pl Q_\zeta}{\pl q_\beta}
\end{equation*}
用矩阵表示即
\begin{equation}
	\left(\frac{\pl \mbf{Q}}{\pl \mbf{q}}\right)^{\mathrm{T}} \frac{\pl \mbf{P}}{\pl \mbf{q}} - \left(\frac{\pl \mbf{P}}{\pl \mbf{q}}\right)^{\mathrm{T}} \frac{\pl \mbf{Q}}{\pl \mbf{q}} = \mbf{0}
	\label{正则变换的矩阵表示1}
\end{equation}
再根据$\dfrac{\pl^2 \bar{F}_1}{\pl q_\alpha \pl p_\beta} = \dfrac{\pl^2 \bar{F}_1}{\pl p_\beta \pl q_\alpha}$可得
\begin{equation*}
	-\sum_{\zeta=1}^s \frac{\pl P_\zeta}{\pl q_\alpha} \frac{\pl Q_\zeta}{\pl p_\beta} - \sum_{\zeta=1}^s P_\zeta \frac{\pl^2 Q_\zeta}{\pl q_\alpha \pl p_\beta} = \delta_{\alpha\beta}-\sum_{\zeta=1}^s \frac{\pl P_\zeta}{\pl p_\beta} \frac{\pl Q_\zeta}{\pl q_\alpha} - \sum_{\zeta=1}^s P_\zeta \frac{\pl^2 Q_\zeta}{\pl p_\beta \pl q_\alpha}
\end{equation*}
所以有
\begin{equation*}
	\sum_{\zeta=1}^s \frac{\pl P_\zeta}{\pl q_\alpha} \frac{\pl Q_\zeta}{\pl p_\beta} = \sum_{\zeta=1}^s \frac{\pl P_\zeta}{\pl p_\beta} \frac{\pl Q_\zeta}{\pl q_\alpha} - \delta_{\alpha\beta}
\end{equation*}
用矩阵表示即
\begin{equation}
	\left(\frac{\pl \mbf{Q}}{\pl \mbf{q}}\right)^{\mathrm{T}} \frac{\pl \mbf{P}}{\pl \mbf{p}} - \left(\frac{\pl \mbf{P}}{\pl \mbf{q}}\right)^{\mathrm{T}} \frac{\pl \mbf{Q}}{\pl \mbf{p}} = \mbf{I}
	\label{正则变换的矩阵表示2}
\end{equation}
最后根据$\dfrac{\pl^2 \bar{F}_1}{\pl p_\alpha \pl p_\beta} = \dfrac{\pl^2 \bar{F}_1}{\pl p_\beta \pl p_\alpha}$可得
\begin{equation*}
	-\sum_{\zeta=1}^s \frac{\pl P_\zeta}{\pl p_\alpha} \frac{\pl Q_\zeta}{\pl p_\beta} - \sum_{\zeta=1}^s P_\zeta \frac{\pl^2 Q_\zeta}{\pl p_\alpha \pl p_\beta} = -\sum_{\zeta=1}^s \frac{\pl P_\zeta}{\pl p_\beta} \frac{\pl Q_\zeta}{\pl p_\alpha} - \sum_{\zeta=1}^s P_\zeta \frac{\pl^2 Q_\zeta}{\pl p_\beta \pl p_\alpha}
\end{equation*}
所以有
\begin{equation*}
	\sum_{\zeta=1}^s \frac{\pl P_\zeta}{\pl p_\alpha} \frac{\pl Q_\zeta}{\pl p_\beta} = \sum_{\zeta=1}^s \frac{\pl P_\zeta}{\pl p_\beta} \frac{\pl Q_\zeta}{\pl p_\alpha}
\end{equation*}
用矩阵表示即
\begin{equation}
	\left(\frac{\pl \mbf{Q}}{\pl \mbf{p}}\right)^{\mathrm{T}} \frac{\pl \mbf{P}}{\pl \mbf{p}} - \left(\frac{\pl \mbf{P}}{\pl \mbf{p}}\right)^{\mathrm{T}} \frac{\pl \mbf{Q}}{\pl \mbf{p}} = \mbf{0}
	\label{正则变换的矩阵表示3}
\end{equation}
综合式\eqref{正则变换的矩阵表示1}、式\eqref{正则变换的矩阵表示2}和式\eqref{正则变换的矩阵表示3}可得
\begin{equation*}
	\begin{pmatrix} \left(\dfrac{\pl \mbf{Q}}{\pl \mbf{q}}\right)^{\mathrm{T}} & \left(\dfrac{\pl \mbf{P}}{\pl \mbf{q}}\right)^{\mathrm{T}} \\[1.5ex] \left(\dfrac{\pl \mbf{Q}}{\pl \mbf{p}}\right)^{\mathrm{T}} & \left(\dfrac{\pl \mbf{P}}{\pl \mbf{p}}\right)^{\mathrm{T}} \end{pmatrix} \begin{pmatrix} \mbf{0} & \mbf{I}_s \\ -\mbf{I}_s & \mbf{0} \end{pmatrix} \begin{pmatrix} \dfrac{\pl \mbf{Q}}{\pl \mbf{q}} & \dfrac{\pl \mbf{Q}}{\pl \mbf{p}} \\[1.5ex] \dfrac{\pl \mbf{P}}{\pl \mbf{q}} & \dfrac{\pl \mbf{P}}{\pl \mbf{p}} \end{pmatrix} = \begin{pmatrix} \mbf{0} & \mbf{I}_s \\ -\mbf{I}_s & \mbf{0} \end{pmatrix}
\end{equation*}
或者
\begin{equation}
	\mbf{M}^{\mathrm{T}} \mbf{J} \mbf{M} = \mbf{J}
	\label{正则变换的辛条件}
\end{equation}
式\eqref{正则变换的辛条件}称为{\heiti 正则变换的辛条件},反映了正则方程的对称性。所有正则变换的Jacobi矩阵构型{\heiti 辛群}。

由式\eqref{正则变换的辛条件}可有
\begin{equation*}
	\det(\mbf{M}^{\mathrm{T}} \mbf{J} \mbf{M}) = \det \mbf{J} = 1
\end{equation*}
所以可有
\begin{equation}
	\left|\det \mbf{M}\right| = 1
\end{equation}
即,正则变换Jacobi行列式的绝对值为$1$。有限正则变换是一系列无限小正则变换的累积,即
\begin{equation*}
	\mbf{M} = \prod_{i=1}^\infty \mbf{M}_i(\eps_i,G_i),\quad \eps_i \to 0
\end{equation*}

\subsection{正则变换的不变量}

\subsubsection{相体积元不变}

在正则变换前,相体积元为
\begin{equation*}
	\prod_{i=1}^{2s} \mathrm{d} \xi_i = \prod_{\alpha=1}^s \mathrm{d} q_\alpha \mathrm{d} p_\alpha
\end{equation*}
因此,正则变换之后的相体积元为
\begin{equation*}
	\prod_{j=1}^{2s} \mathrm{d} \mathnormal{\Xi}_j = \prod_{\beta=1}^s \mathrm{d} Q_\beta \mathrm{d} P_\beta = \left|\det\frac{\pl \mrm{\Xi}}{\pl \mbf{\xi}}\right| \prod_{i=1}^{2s} \mathrm{d} \xi_i = \left|\det \mbf{M}\right| \prod_{i=1}^{2s} \mathrm{d} \xi_i = \prod_{i=1}^{2s} \mathrm{d} \xi_i
\end{equation*}
因此,相体积为不依赖于正则变量选取的绝对量。这是经典统计物理的理论基础。

\subsubsection{Poisson括号不变}

设有
\begin{align*}
	& f = f_\xi(\mbf{\xi},t) = f_{\mathnormal{\Xi}}(\mrm{\Xi},t) \\
	& g = g_\xi(\mbf{\xi},t) = g_{\mathnormal{\Xi}}(\mrm{\Xi},t)
\end{align*}
变换之前的Poisson括号为
\begin{equation*}
	[f,g]_\xi = \sum_{\alpha=1}^s \frac{\pl (f,g)}{\pl (q_\alpha,p_\alpha)} = \left(\frac{\pl f}{\pl \mbf{\xi}}\right)^{\mathrm{T}} \mbf{J} \frac{\pl g}{\pl \mbf{\xi}}
\end{equation*}
正则变换之后的Poisson括号为
\begin{align*}
	[f,g]_{\mathnormal{\Xi}} & = \left(\frac{\pl f}{\pl \mrm{\Xi}}\right)^{\mathrm{T}} \mbf{J} \frac{\pl g}{\pl \mrm{\Xi}} = \left(\frac{\pl \mbf{\xi}}{\pl \mrm{\Xi}} \frac{\pl f}{\pl \mbf{\xi}} \right)^{\mathrm{T}} \mbf{J} \frac{\pl \mbf{\xi}}{\pl \mrm{\Xi}}  \frac{\pl g}{\pl \mbf{\xi}} \\
	& = \left(\frac{\pl f}{\pl \mbf{\xi}}\right)^{\mathrm{T}} \mbf{M}^{-\mathrm{T}} \mbf{J} \mbf{M}^{-1} \frac{\pl g}{\pl \mbf{\xi}}
\end{align*}
由式\eqref{正则变换的辛条件}可有
\begin{equation*}
	\mbf{M}^{-\mathrm{T}} \mbf{J} \mbf{M}^{-1} = \mbf{J}
\end{equation*}
所以有
\begin{align*}
	[f,g]_{\mathnormal{\Xi}} & = \left(\frac{\pl f}{\pl \mbf{\xi}}\right)^{\mathrm{T}} \mbf{J} \frac{\pl g}{\pl \mbf{\xi}} = [f,g]_\xi
\end{align*}
因此,Poisson括号为不依赖于正则变量选取的绝对量。力学Poisson括号形式为不依赖于相坐标系的绝对表示。因此相应地,基本Poisson括号在正则变换下不变,此可作为正则变换的判据。

角动量分量不符合基本泊松括号,因此不可能同时选为正则变量。

\begin{example}
若新旧正则变量的变换关系为
\begin{equation*}
	Q = q^\alpha \cos \beta p,\quad P = q^\alpha \sin \beta p
\end{equation*}
当$\alpha,\beta$为何值时,这个变换是正则变换?并找出其母函数。
\end{example}
\begin{solution}
考虑基本泊松括号
\begin{equation*}
	[Q,P]_\xi = \begin{vmatrix} \dfrac{\pl Q}{\pl q} & \dfrac{\pl Q}{\pl p} \\[1.5ex] \dfrac{\pl P}{\pl q} & \dfrac{\pl P}{\pl p} \end{vmatrix} = \alpha \beta q^{2\alpha-1} = 1 = [Q,P]_{\mathnormal{\Xi}}
\end{equation*}
由此可得$\alpha=\dfrac12,\beta = 2$。所以
\begin{equation*}
	\begin{cases}
		\displaystyle Q = \sqrt{q} \cos 2p \\
		\displaystyle P = \sqrt{q} \sin 2p
	\end{cases}
\end{equation*}
因此有
\begin{equation*}
	\begin{cases}
		\displaystyle q = \frac{Q^2}{\cos^2 2p} \\
		\displaystyle P = Q\tan 2p
	\end{cases}
\end{equation*}
考虑第三类母函数$F_3 = F_3(p,Q)$,则有
\begin{equation*}
	\mathrm{d} F_3 = -q\mathrm{d} p - P\mathrm{d} Q = -\frac{Q^2}{\cos^2 2p} \mathrm{d} p - Q\tan 2p \mathrm{d}Q = \mathrm{d} \left(-\frac12 Q^2 \tan 2p\right)
\end{equation*}
即有第三类母函数
\begin{equation*}
	F_3(p,Q) = -\frac12 Q^2 \tan 2p
\end{equation*}
\end{solution}

\section{Hamilton-Jacobi方程}

通过适当的正则变换可使新的哈密顿函数$K(\mbf{Q},\mbf{P},t)$的结构简化,从而使正则方程易于求解。最理想的情况是使得$K = 0$,此时新的正则方程就变为
\begin{equation*}
	\begin{cases}
		\dot{Q}_\alpha = 0, \\
		\dot{P}_\alpha = 0,
	\end{cases}\quad \alpha = 1,2,\cdots,s
\end{equation*}
于是所有正则变量都是可遗变量,即有
\begin{equation*}
	\begin{cases}
		Q_\alpha = C_\alpha = \text{常数}, \\
		P_\alpha = D_\alpha = \text{常数},
	\end{cases}\quad \alpha=1,2,\cdots,s
\end{equation*}
然后再利用变换关系
\begin{equation*}
	\begin{cases}
		Q_\alpha = Q_\alpha(\mbf{q},\mbf{p},t), \\
		P_\alpha = P_\alpha(\mbf{q},\mbf{p},t),
	\end{cases}\quad \alpha = 1,2,\cdots,s
\end{equation*}
即可得到原来正则方程的解
\begin{equation*}
	\begin{cases}
		q_\alpha = q_\alpha(\mbf{C},\mbf{D},t) \\
		p_\alpha = p_\alpha(\mbf{C},\mbf{D},t)
	\end{cases}
\end{equation*}
对于四类正则变换母函数,新旧Hamilton函数间的关系都是
\begin{equation*}
	K(\mbf{Q},\mbf{P},t) = H(\mbf{q},\mbf{p},t) + \frac{\pl S}{\pl t}
\end{equation*}
此处将母函数记作$S$,欲使$K = 0$,则要求母函数$S$满足关系式
\begin{equation}
	H(\mbf{q},\mbf{p},t) + \frac{\pl S}{\pl t} = 0
\end{equation}
由此,力学问题便归结为求$S$。

\subsection{Hamilton-Jacobi方程}

考虑第二类母函数$F_2 = S(\mbf{q},\mbf{P},t)$,则由变换关系可有
\begin{equation*}
	p_\alpha = \frac{\pl S}{\pl q_\alpha}
\end{equation*}
并根据变换需满足的条件,令
\begin{equation*}
	P_\alpha = D_\alpha
\end{equation*}
即有
\begin{equation}
	H\left(\mbf{q},\frac{\pl S}{\pl \mbf{q}},t\right) + \frac{\pl S}{\pl t} = 0
	\label{Hamilton-Jacobi方程}
\end{equation}
式\eqref{Hamilton-Jacobi方程}称为{\heiti Hamilton-Jacobi方程}。它是函数$S(\mbf{q},\mbf{D},t)$的一阶偏微分方程,它有$s+1$个自变量($\mbf{q}$和$t$),所以应有$s+1$个独立的积分常数,其中一个为相加常数,不影响变换,可略去;其余的$s$个即为$D_\alpha$。含有$s$个任意常数(不包括相加常数)的函数$S(\mbf{q},\mbf{D},t)$称为Hamilton-Jacobi方程的{\bf 完全积分}。
%利用Hamilton-Jacobi方程\eqref{Hamilton-Jacobi方程}得到函数$S=S(\mbf{q},\mbf{D},t)$之后,便可利用关系
% \begin{equation*}
	% \frac{\pl S}{\pl D_\alpha}(\mbf{q},\mbf{D},t) = C_\alpha,\quad \alpha=1,2,\cdots,s
% \end{equation*}
% 来直接得到体系的轨道方程$\mbf{q}=\mbf{q}(t)$,其中$C_\alpha$为任意常数。

\subsection{Hamilton主函数}

在Hamilton-Jacobi方程\eqref{Hamilton-Jacobi方程}中,函数$S(\mbf{q},\mbf{D},t)$称为{\heiti Hamilton主函数}。下面来讨论Hamilton主函数的物理意义。考虑
\begin{equation*}
	\frac{\mathrm{d} S}{\mathrm{d} t} = \sum_{\alpha=1}^s \frac{\pl S}{\pl q_\alpha} \dot{q}_\alpha + \frac{\pl S}{\pl t} = \sum_{\alpha=1}^s p_\alpha \dot{q}_\alpha - H = L
\end{equation*}
由此可得
\begin{equation}
	S = \int L \mathrm{d} t
	\label{Hamilton作用函数}
\end{equation}
函数$S(\mbf{q},\mbf{D},t)$即为积分限不定的Hamilton作用量,因此也称为{\heiti Hamilton作用函数}。

当Hamilton函数不显含时间$t$,即体系广义能量守恒时\footnote{纯力学问题都属于这一类问题。},这种情况下方程\eqref{Hamilton-Jacobi方程}可以表示为
\begin{equation}
	H\left(\mbf{q},\frac{\pl S}{\pl \mbf{q}}\right) + \frac{\pl S}{\pl t} = E + \frac{\pl S}{\pl t} = 0
\end{equation}
此情况下Hamilton作用函数可分离变量
\begin{equation*}
	S = S(\mbf{q},\mbf{D},t) = W(\mbf{q},\mbf{D}) + T(t)
\end{equation*}
由此即有
\begin{equation}
	S(\mbf{q},\mbf{D},t) = W(\mbf{q},\mbf{D}) - Et
\end{equation}
Hamilton-Jacobi方程变为
\begin{equation}
	H\left(\mbf{q},\frac{\pl W}{\pl \mbf{q}}\right) = E
	\label{不含时Hamilton-Jacobi方程}
\end{equation}
式\eqref{不含时Hamilton-Jacobi方程}称为{\heiti 不含时Hamilton-Jacobi方程}。其中函数
\begin{equation*}
	W = W(\mbf{q},\mbf{D}) = W(\mbf{q},E,D_2,\cdots,D_s)
\end{equation*}
称为{\heiti Hamilton特征函数}。

考虑到$H=E$,可有
\begin{equation*}
	S = \int L \mathrm{d} t = \int \left(\sum_{\alpha=1}^s p_\alpha \dot{q}_\alpha - H\right) \mathrm{d} t = \int \sum_{\alpha=1}^s p_\alpha \mathrm{d} q_\alpha - Et + D_0
\end{equation*}
上式中$D_0$是相加常数,可略去,由此可得Hamilton特征函数的具体形式
\begin{equation}
	W(\mbf{q},\mbf{D}) = \int \sum_{\alpha=1}^s p_\alpha \mathrm{d} q_\alpha
\end{equation}

%下面来考虑正则方程的完全积分$\mbf{q} = \mbf{q}(t),\mbf{p} = \mbf{p}(t)$以及体系运动轨道的关系。首先有
下面来考虑Hamilton-Jacobi方程的完全积分$S=S(\mbf{q},\mbf{D},t)$与体系运动轨迹$\mbf{q} = \mbf{q}(t),\mbf{p} = \mbf{p}(t)$之间的关系。首先有
\begin{subnumcases}{\label{chp3:Hamilton-Jacobi方程的完全积分与运动轨道-1}}
	p_\alpha = \frac{\pl S}{\pl q_\alpha}(\mbf{q},\mbf{D},t), & $\alpha=1,2,\cdots,s$ \label{chp3:Hamilton-Jacobi方程的完全积分与运动轨道-1.1} \\
	C_\alpha = \frac{\pl S}{\pl D_\alpha}(\mbf{q},\mbf{D},t), & $\alpha=1,2,\cdots,s$ \label{chp3:Hamilton-Jacobi方程的完全积分与运动轨道-1.2}
\end{subnumcases}
由方程\eqref{chp3:Hamilton-Jacobi方程的完全积分与运动轨道-1.2}的$s$个方程中(这是代数方程组)可以解出$s$个广义坐标与时间的关系
\begin{equation*}
	\mbf{q} = \mbf{q}(t)
\end{equation*}
再由方程\eqref{chp3:Hamilton-Jacobi方程的完全积分与运动轨道-1.1}即可得
\begin{equation*}
	\mbf{p} = \mbf{p}(t)
\end{equation*}
由此便得到完整的相空间轨迹。当体系广义能量守恒时,记$E=D_1$,则有
\begin{subnumcases}{}
	p_\alpha = \frac{\pl S}{\pl q_\alpha}(\mbf{q},\mbf{D},t) = \frac{\pl W}{\pl q_\alpha}(\mbf{q},\mbf{D}),& $\alpha=1,2,\cdots,s$ \\
	C_1 = \frac{\pl S}{\pl D_1}(\mbf{q},\mbf{D},t) = \frac{\pl W}{\pl E}(\mbf{q},\mbf{D}) - t \\
	C_\alpha = \frac{\pl S}{\pl D_\alpha} = \frac{\pl W}{\pl D_\alpha}(\mbf{q},\mbf{D}),& $\alpha=2,3,\cdots,s$
\end{subnumcases}
由此可有
\begin{equation}
	C_\alpha = C_\alpha(q_1,q_2,\cdots,q_s;E,D_2,\cdots,D_s),\quad \alpha=2,3,\cdots,s
	\label{chp3:Hamilton-Jacobi方程的完全积分与运动轨道-3}
\end{equation}
由式\eqref{chp3:Hamilton-Jacobi方程的完全积分与运动轨道-3}的$s-1$个方程,可以以$q_1$为参数,解出$q_2,\cdots,q_s$的表达式为
\begin{equation}
	q_\alpha = f_\alpha(q_1;C_2,\cdots,C_s;E,D_2,\cdots,D_s),\quad \alpha = 2,3,\cdots,s
\end{equation}
此即为位形空间的轨道曲线。

\begin{example}[一维谐振子的Hamilton-Jacobi方程求解]
一维谐振子的Hamilton函数为
\begin{equation*}
	H(q,p) = \frac{p^2}{2m} + \frac12 m\omega^2 q^2
\end{equation*}
用Hamilton-Jacobi方程求解此体系的运动。
\end{example}
\begin{solution}
体系的Hamilton函数不显含时间,故根据不含时Hamilton-Jacobi方程\eqref{不含时Hamilton-Jacobi方程}可有
\begin{equation*}
	H\left(q,\frac{\pl W}{\pl q}\right) = \frac12 \left(\frac{\pl W}{\pl q}\right)^2 + \frac12 m\omega^2 q^2 = E
\end{equation*}
即
\begin{equation*}
	\frac{\pl W}{\pl q} = m\omega \sqrt{\frac{2E}{m\omega^2} - q^2} = p
\end{equation*}
因此
\begin{equation*}
	W = m\omega \int \sqrt{\frac{2E}{m\omega^2} - q^2} \mathrm{d} q
\end{equation*}
由此即有
\begin{equation*}
	C = \frac{\pl W}{\pl E} - t = \frac{1}{\omega} \int \frac{\mathrm{d} q}{\sqrt{\dfrac{2E}{m\omega^2}-q^2}} - t = \frac{1}{\omega} \arcsin\left(\sqrt{\frac{m\omega^2}{2E}}q\right) - t
\end{equation*}
由此可得体系的解
\begin{equation*}
\begin{cases}
	\displaystyle q = \sqrt{\frac{2E}{m\omega^2}} \sin \big[\omega(t+C)\big] \\[1.5ex]
	\displaystyle p = m\omega \sqrt{\frac{2E}{m\omega^2} - q^2} = \sqrt{2mE} \cos \big[\omega(t+C)\big]
\end{cases}
\end{equation*}
$C,E$是积分常数,故有
\begin{equation*}
	C = \frac{1}{\omega} \arcsin\left(\sqrt{\frac{m\omega^2}{2E}}q_0\right) - t_0,\quad E = \frac{p_0^2}{2m} + \frac12 m\omega^2 q_0^2
\end{equation*}
\end{solution}

Hamilton-Jacobi理论将常微分方程组求解转化为偏微分方程的全积分问题,在具体应用方面并无方便之处,但是提供了一种全新的力学运动图像。

\subsection{Hamilton-Jacobi方程的分离变量}

在某些特殊情况下,Hamilton-Jacobi方程的完全积分可以通过分离变量的方法求得。假设某一个坐标,将其记作$q_1$,与其相应的导数$\dfrac{\pl S}{\pl q_1}$,在Hamilton-Jacobi方程中仅以某种组合$\phi\left(q_1,\dfrac{\pl S}{\pl q_1}\right)$的方式出现,而这个组合中不包含任何其他坐标,即Hamilton-Jacobi方程\eqref{Hamilton-Jacobi方程}具有如下形式:
\begin{equation}
	\varPhi\left(\mbf{q}',t,\frac{\pl S}{\pl \mbf{q}'},\frac{\pl S}{\pl t},\phi\left(q_1,\dfrac{\pl S}{\pl q_1}\right)\right) = 0
	\label{chp3:Hamilton-Jacobi方程的分离变量-1}
\end{equation}
其中$\mbf{q}'$表示除了$q_1$以外的所有坐标。

假设解具有如下形式
\begin{equation}
	S(\mbf{q},t) = S_0(\mbf{q}',t)+S_1(q_1)
	\label{chp3:Hamilton-Jacobi方程的分离变量-2}
\end{equation}
将\eqref{chp3:Hamilton-Jacobi方程的分离变量-2}代入式\eqref{chp3:Hamilton-Jacobi方程的分离变量-1}中,即有
\begin{equation}
	\varPhi\left(\mbf{q}',t,\frac{\pl S_0}{\pl \mbf{q}'},\frac{\pl S_0}{\pl t},\phi\left(q_1,\dfrac{\mathrm{d} S_1}{\mathrm{d} q_1}\right)\right) = 0
	\label{chp3:Hamilton-Jacobi方程的分离变量-3}
\end{equation}
如果式\eqref{chp3:Hamilton-Jacobi方程的分离变量-2}是方程的解,那么在$q_1$变化时,式\eqref{chp3:Hamilton-Jacobi方程的分离变量-3}应为恒等式。而式\eqref{chp3:Hamilton-Jacobi方程的分离变量-3}中,只有$\phi\left(q_1,\dfrac{\mathrm{d} S_1}{\mathrm{d} q_1}\right)$项含有$q_1$,因此$\phi\left(q_1,\dfrac{\mathrm{d} S_1}{\mathrm{d} q_1}\right)$项应为常数。于是将方程\eqref{chp3:Hamilton-Jacobi方程的分离变量-1}分为两个方程
\begin{align}
	& \phi\left(q_1,\dfrac{\mathrm{d} S_1}{\mathrm{d} q_1}\right) = D_1 \label{chp3:Hamilton-Jacobi方程的分离变量-4.1} \\
	& \varPhi\left(\mbf{q}',t,\frac{\pl S_0}{\pl \mbf{q}'},\frac{\pl S_0}{\pl t},D_1\right) = 0 \label{chp3:Hamilton-Jacobi方程的分离变量-4.2}
\end{align}
方程\eqref{chp3:Hamilton-Jacobi方程的分离变量-4.1}是一个常微分方程,而方程\eqref{chp3:Hamilton-Jacobi方程的分离变量-4.2}虽然仍为偏微分方程,但其独立变量数目较之原来减少了一个。

对于循环坐标的情形,由于循环坐标$q_1$不显含于Hamilton函数中,此时$\phi\left(q_1,\dfrac{\pl S}{\pl q_1}\right)$项即为$\dfrac{\pl S}{\pl q_1}$,因此可有
\begin{equation*}
	S_1(q_1) = D_1q_1
\end{equation*}
而此处常数$D_1$即为与$q_1$共轭的广义动量$p_1=\dfrac{\pl S}{\pl q_1}=D_1$。对于保守系统而言,$-Et$这一项即对应着“循环坐标”时间$t$的分离。

为了能够在Hamilton-Jacobi方程中分离变量,适当选择坐标系非常关键。

\begin{example}[球坐标系中Hamilton-Jacobi方程的分离变量]
在球坐标系中,Hamilton函数为
\begin{equation}
	H(r,\theta,\phi,p_r,p_\theta,p_\phi,t) = \frac{1}{2m}\left(p_r^2+\frac{p_\theta^2}{r^2}+\frac{p_\phi^2}{r^2\sin^2\theta}\right)+V(r,\theta,\phi,t)
	\label{球坐标系中Hamilton-Jacobi方程的分离变量-1.1}
\end{equation}
如果势函数$V(r,\theta,\phi,t)$具有如下形式
\begin{equation}
	V(r,\theta,\phi,t) = a(r)+\frac{b(\theta)}{r^2}+\frac{c(\phi)}{r^2\sin^2\theta}
	\label{球坐标系中Hamilton-Jacobi方程的分离变量-1.2}
\end{equation}
其中$a(r),b(\theta),c(\phi)$是任意函数,则分离变量是可能的。此时关于函数$W(r,\theta,\phi)$的不含时Hamilton-Jacobi方程为
\begin{align}
	\frac{1}{2m}\left(\frac{\pl W}{\pl r}\right)^2+a(r)& {}+\frac{1}{2mr^2}\left[\left(\frac{\pl W}{\pl \theta}\right)^2+2mb(\theta)\right] \nonumber \\
	& {}+ \frac{1}{2mr^2\sin^2\theta}\left[\left(\frac{\pl W}{\pl \phi}\right)^2+2mc(\phi)\right] = E
	\label{球坐标系中Hamilton-Jacobi方程的分离变量-2}
\end{align}

假设解具有形式
\begin{equation}
	S(r,\theta,\phi,t) = -Et+W(r,\theta,\phi) = -Et+S_1(r)+S_2(\theta)+S_3(\phi)
	\label{球坐标系中Hamilton-Jacobi方程的分离变量-3}
\end{equation}
则有
\begin{subnumcases}{}
	\left(\frac{\mathrm{d}S_3}{\mathrm{d}\phi}\right)^2+2mc(\phi) = \alpha \label{球坐标系中Hamilton-Jacobi方程的分离变量-4.1} \\
	\left(\frac{\mathrm{d}S_2}{\mathrm{d}\theta}\right)^2+2mb(\theta)+\frac{\alpha}{\sin^2\theta} = \beta \label{球坐标系中Hamilton-Jacobi方程的分离变量-4.2} \\
	\frac{1}{2m}\left(\frac{\mathrm{d}S_1}{\mathrm{d}r}\right)^2+a(r)+\frac{\beta}{2mr^2} = E \label{球坐标系中Hamilton-Jacobi方程的分离变量-4.3}
\end{subnumcases}
依次积分可得
\begin{subnumcases}{}
	S_3(\phi) = \pm\int \sqrt{\alpha-2mc(\phi)} \mathrm{d}\phi \label{球坐标系中Hamilton-Jacobi方程的分离变量-5.1}\\
	S_2(\theta) = \pm\int \sqrt{\beta-2mb(\theta)-\frac{\alpha}{\sin^2\theta}} \mathrm{d}\theta \label{球坐标系中Hamilton-Jacobi方程的分离变量-5.2} \\
	S_1(r) = \pm\int \sqrt{2m\big[E-a(r)\big]-\frac{\beta}{r^2}} \mathrm{d}r \label{球坐标系中Hamilton-Jacobi方程的分离变量-5.3}
\end{subnumcases}
便得到Hamilton-Jacobi方程的完全积分
\begin{align}
	S(r,\theta,\phi,t) & = -Et \pm \int \sqrt{\alpha-2mc(\phi)} \mathrm{d}\phi \pm \int \sqrt{\beta-2mb(\theta)-\frac{\alpha}{\sin^2\theta}} \mathrm{d}\theta \nonumber \\
	& \quad {} \pm \int \sqrt{2m\big[E-a(r)\big]-\frac{\beta}{r^2}} \mathrm{d}r
	\label{球坐标系中Hamilton-Jacobi方程的分离变量-6}
\end{align}
其中$\alpha,\beta,E$是任意积分常数,将式\eqref{球坐标系中Hamilton-Jacobi方程的分离变量-6}对这三个任意常数求导并使之等于新常数,即可得到运动方程的通解。

在一般的问题中,式\eqref{球坐标系中Hamilton-Jacobi方程的分离变量-1.2}所表示的势场其第三项未必有物理意义,因此在$c(\phi)=0$的情形下,可有Hamilton-Jacobi方程的完全积分为
\begin{align}
	S(r,\theta,\phi,t) & = -Et + p_\phi\phi \pm \int \sqrt{\beta-2mb(\theta)-\frac{p_\phi^2}{\sin^2\theta}} \mathrm{d}\theta \pm \int \sqrt{2m\big[E-a(r)\big]-\frac{\beta}{r^2}} \mathrm{d}r
	\label{球坐标系中Hamilton-Jacobi方程的分离变量-6}
\end{align}
\end{example}

\begin{example}[抛物线坐标系中Hamilton-Jacobi方程的分离变量]
抛物线坐标$(u,v,\phi)$与直角坐标$(x,y,z)$之间的关系由如下变换公式给出
\begin{equation}
\begin{cases}
	x = uv\cos\phi \\
	y = uv\sin\phi \\
	z = \dfrac12(u^2-v^2)
\end{cases}
(0\leqslant u,v<+\infty,0\leqslant \phi < 2\pi)
\end{equation}
$u$和$v$为常数的曲面是两族以$z$轴为对称轴的旋转抛物面,如图\ref{chp3:抛物线坐标系}所示。

\begin{figure}[htb]
\centering
\begin{asy}
	texpreamble("\usepackage{xeCJK}");
	texpreamble("\setCJKmainfont{SimSun}");
	usepackage("amsmath");
	import graph;
	import math;
	size(300);
	//抛物线坐标系的坐标面
	pair O,i,j,k;
	O = (0,0);
	i = (-sqrt(2)/4,-sqrt(14)/12);
	j = (sqrt(14)/4,-sqrt(2)/12);
	k = (0,2*sqrt(2)/3);
	draw(-3*k--2.6*k,dashed);
	draw(Label("$z$",EndPoint),2.6*k--4*k,Arrow);
	real u0,v0,h0,phi0,h,r;
	pair f1(real h){
		real theta,tmp,phi;
		theta = atan(-j.x/i.x);
		tmp = v0/sqrt(v0^2+2*h)*(i.x*j.y-i.y*j.x)/k.y/sqrt(i.x^2+j.x^2);
		phi = asin(tmp)-theta;
		return sqrt(v0^2+2*h)*v0*cos(phi)*i+sqrt(v0^2+2*h)*v0*sin(phi)*j+h*k;
	}
	pair f2(real h){
		real theta,tmp,phi;
		theta = atan(-j.x/i.x);
		tmp = -u0/sqrt(u0^2-2*h)*(i.x*j.y-i.y*j.x)/(k.y*sqrt(i.x^2+j.x^2));
		phi = asin(tmp)-theta;
		return sqrt(u0^2-2*h)*u0*cos(phi)*i+sqrt(u0^2-2*h)*u0*sin(phi)*j+h*k;
	}
	pair para_up(real phi){
		real x,y,z,u;
		u = sqrt(v0^2+2*h);
		x = u*v0*cos(phi);
		y = u*v0*sin(phi);
		z = h;
		return x*i+y*j+z*k;
	}
	pair para_down(real phi){
		real x,y,z,v;
		v = sqrt(u0^2-2*h);
		x = u0*v*cos(phi);
		y = u0*v*sin(phi);
		z = h;
		return x*i+y*j+z*k;
	}
	pair intersection_cur_up(real h){
		real x,y,z,u;
		u = sqrt(v0^2+2*h);
		x = u*v0*cos(phi0);
		y = u*v0*sin(phi0);
		z = h;
		return x*i+y*j+z*k;
	}
	pair intersection_cur_down(real h){
		real x,y,z,v;
		v = sqrt(u0^2-2*h);
		x = u0*v*cos(phi0);
		y = u0*v*sin(phi0);
		z = h;
		return x*i+y*j+z*k;
	}
	path q1,q2,p1,p2;
	picture tmp;
	v0 = 0.6;
	u0 = 0.8;
	h0 = 0.5*(u0^2-v0^2);
	q1 = graph(f1,3,h0);
	p1 = graph(f1,h0,-0.5*v0^2+0.1);
	draw(q1,linewidth(0.8bp)+red);
	draw(p1..xscale(-1)*reverse(p1),dashed+red);
	q2 = graph(f2,-3,h0);
	p2 = graph(f2,h0,0.5*u0^2-0.1);
	draw(q2,linewidth(0.8bp)+blue);
	draw(p2..xscale(-1)*reverse(p2),dashed+blue);
	real tm;
	tm = -1.2;
	h = h0;
	draw(graph(para_up,tm,pi+tm),green+linewidth(0.8bp));
	draw(graph(para_up,pi+tm,2*pi+tm),dashed+green);
	h = 3;
	draw(graph(para_up,0,2*pi),red);
	h = -3;
	draw(graph(para_down,pi+tm,2*pi+tm),dashed+blue);
	phi0 = pi/3;
	r = 3;
	draw(tmp,graph(para_down,tm,pi+tm),blue);
	draw(tmp,xscale(-1)*q1,linewidth(0.8bp)+red);
	draw(tmp,xscale(-1)*q2,linewidth(0.8bp)+blue);
	draw(tmp,Label("$y$",EndPoint),O--3*j,Arrow);
	path clp;
	clp = 3*k--r*cos(phi0)*i+r*sin(phi0)*j+3*k--r*cos(phi0)*i+r*sin(phi0)*j-3*k--(-3*k)--cycle;
	unfill(tmp,clp);
	add(tmp);
	erase(tmp);
	draw(tmp,graph(para_down,tm,pi+tm),blue+dashed);
	draw(tmp,xscale(-1)*q1,dashed+red);
	draw(tmp,xscale(-1)*q2,dashed+blue);
	draw(tmp,Label("$y$",EndPoint),O--3*j,dashed);
	clip(tmp,clp);
	add(tmp);
	draw(3*k--r*cos(phi0)*i+r*sin(phi0)*j+3*k--r*cos(phi0)*i+r*sin(phi0)*j-3*k--intersection_cur_down(-3),green);
	draw(intersection_cur_down(-3)--(-3*k),green+dashed);
	draw(graph(intersection_cur_up,3,h0),red+linewidth(0.8bp));
	draw(graph(intersection_cur_up,h0,-0.5*v0^2),red+dashed);
	draw(graph(intersection_cur_down,-3,h0),blue+linewidth(0.8bp));
	draw(graph(intersection_cur_down,h0,0.5*u0^2),blue+dashed);
	dot(intersection_cur_up(h0));
	draw(Label("$x$",EndPoint),u0^2*i--6*i,Arrow);
	draw(O--u0^2*i,dashed);
	pair P,eu,ev,ephi;
	eu = v0*cos(phi0)/sqrt(u0^2+v0^2)*i+v0*sin(phi0)/sqrt(u0^2+v0^2)*j+u0/sqrt(u0^2+v0^2)*k;
	ev = u0*cos(phi0)/sqrt(u0^2+v0^2)*i+u0*sin(phi0)/sqrt(u0^2+v0^2)*j-v0/sqrt(u0^2+v0^2)*k;
	ephi = -sin(phi0)*i+cos(phi0)*j;
	P = intersection_cur_up(h0);
	draw(Label("$\boldsymbol{e}_u$",EndPoint),P--P+eu,Arrow);
	draw(Label("$\boldsymbol{e}_v$",EndPoint),P--P+ev,Arrow);
	draw(Label("$\boldsymbol{e}_\phi$",EndPoint),P--P+ephi,Arrow);
	label("$O$",O,WNW);
\end{asy}
\caption{抛物线坐标系的坐标面}
\label{chp3:抛物线坐标系}
\end{figure}

在这个坐标系下,系统的Lagrange函数为
\begin{equation}
	L = \frac12 m(u^2+v^2)\dot{u}^2+\frac12 m(u^2+v^2)\dot{v}^2+\frac12 mu^2v^2\dot{\phi}^2 - V(u,v,\phi)
\end{equation}
此时,广义动量为
\begin{equation*}
	p_u = \frac{\pl L}{\pl \dot{u}} = m(u^2+v^2)\dot{u},\quad p_v = \frac{\pl L}{\pl \dot{v}} = m(u^2+v^2)\dot{v},\quad p_\phi = \frac{\pl L}{\pl \dot{\phi}} = mu^2v^2\dot{\phi}
\end{equation*}
由此可得体系的Hamilton函数为
\begin{equation}
	H = \frac{p_u^2+p_v^2}{2m(u^2+v^2)} + \frac{p_\phi^2}{2mu^2v^2} + V(u,v,\phi)
\end{equation}

在这种坐标下,物理上有意义的可分离变量情况对应于以下形式的势能:
\begin{equation}
	V(u,v,\phi) = \frac{a(u)+b(v)}{u^2+v^2}
\end{equation}
所以我们得到关于$W(\mbf{q},\mbf{D})$的方程
\begin{equation}
	\frac{1}{2m(u^2+v^2)}\left[\left(\frac{\pl W}{\pl u}\right)^2+\left(\frac{\pl W}{\pl v}\right)^2\right]+ \frac{1}{2mu^2v^2}\left(\frac{\pl W}{\pl \phi}\right)^2 + \frac{a(u)+b(v)}{u^2+v^2} = E
	\label{chp3:抛物线坐标下的分离变量-1}
\end{equation}
在式\eqref{chp3:抛物线坐标下的分离变量-1}两端乘以$2m(u^2+v^2)$并整理,可得
\begin{align*}
	\left(\frac{\pl W}{\pl u}\right)^2 + a(u) - 2mEu^2 + \left(\frac{\pl W}{\pl v}\right)^2 +b(u) - 2mEv^2 + \left(\frac{1}{u^2}+\frac{1}{v^2}\right)\left(\frac{\pl W}{\pl \phi}\right)^2 = 0
\end{align*}
此处注意到坐标$\phi$是循环坐标,因此可令
\begin{equation}
	S(u,v,\phi,t) = -Et + p_\phi\phi + S_1(u)+S_2(v)
\end{equation}
由此得到两个方程
\begin{subnumcases}{}
	\left(\frac{\mathrm{d}S_2}{\mathrm{d} v}\right)^2 - 2mEv^2+b(v)+\frac{p_\phi^2}{v^2} = \beta \\
	\left(\frac{\mathrm{d}S_1}{\mathrm{d} u}\right)^2 - 2mEu^2+a(u)+\frac{p_\phi^2}{u^2} = -\beta
\end{subnumcases}
由此即有
\begin{align*}
	S & = -Et + p_\phi\phi \pm \int \sqrt{2mEv^2-\frac{p_\phi^2}{v^2}-b(v)+\beta} \mathrm{d}v \pm \int \sqrt{2mEu^2-\frac{p_\phi^2}{u^2}-a(u)-\beta} \mathrm{d}u
\end{align*}
其中$E,p_\phi,\beta$为任意积分常数。
\end{example}

\begin{example}[椭圆坐标系中Hamilton-Jacobi方程的分离变量]
椭圆坐标$(\xi,\eta,\phi)$与直角坐标$(x,y,z)$之间的关系由如下变换公式给出
\begin{equation}
\begin{cases}
	x = \sigma\sqrt{(\xi^2-1)(1-\eta^2)} \cos \phi \\
	y = \sigma\sqrt{(\xi^2-1)(1-\eta^2)} \sin \phi \\
	z = \sigma \xi\eta
\end{cases}(1\leqslant \xi<+\infty,-1\leqslant \eta\leqslant 1,0\leqslant \phi < 2\pi)
\end{equation}
其中$\sigma$为常数。设$z$轴上两个点$A_1$和$A_2$的坐标为$z=\sigma$和$z=-\sigma$,则$\xi$为常数的坐标面为以$A_1$和$A_2$为焦点的一族旋转椭球面
\begin{equation*}
	\frac{x^2+y^2}{\sigma^2(\xi^2-1)}+ \frac{z^2}{\sigma^2\xi^2} = 1
\end{equation*}
而$\eta$为常数的坐标面为以$A_1$和$A_2$为焦点的一族旋转双叶双曲面
\begin{equation*}
	-\frac{x^2+y^2}{\sigma^2(1-\eta^2)}+ \frac{z^2}{\sigma^2\eta^2} = 1
\end{equation*}
此坐标系即为图\ref{chp3:二维椭圆坐标系}所示的二维椭圆坐标系绕其$z$轴旋转而成。

\begin{figure}[htb]
\centering
\begin{asy}
	texpreamble("\usepackage{xeCJK}");
	texpreamble("\setCJKmainfont{SimSun}");
	usepackage("amsmath");
	import graph;
	import math;
	size(250);
	//二维椭圆坐标系
	picture tmp;
	real xi,eta,sigma,xmax,ymax;
	sigma = 1;
	xmax = 2.5;
	ymax = 2.5;
	pair coor_ell(real theta){
		return (sigma*sqrt(xi^2-1)*cos(theta),sigma*xi*sin(theta));
	}
	pair coor_hyp(real theta){
		return (sigma*sqrt(1-eta^2)*sinh(theta),sigma*eta*cosh(theta));
	}
	for(xi=1;xi<=2.2;xi=xi+0.15){
		draw(tmp,graph(coor_ell,0,2*pi),red);
	}
	for(eta=0;eta<1;eta=eta+0.15){
		if (eta==0) continue;
		draw(tmp,graph(coor_hyp,-10,10),blue);
	}
	for(eta=0;eta>-1;eta=eta-0.15){
		if (eta==0) continue;
		draw(tmp,graph(coor_hyp,-10,10),blue);
	}
	clip(tmp,box((-xmax,-ymax),(xmax,ymax)));
	add(tmp);
	draw(Label("$x$",EndPoint),(-xmax,0)--(xmax,0),Arrow);
	draw(Label("$z$",EndPoint),(0,-ymax)--(0,ymax),Arrow);
	dot("$A_1$",(0,sigma),E);
	dot("$A_2$",(0,-sigma),E);
\end{asy}
\caption{二维椭圆坐标系}
\label{chp3:二维椭圆坐标系}
\end{figure}

记$r_1$和$r_2$是点$(\xi,\eta,\phi)$到$A_1$和$A_2$的距离,则有
\begin{subnumcases}{}
	r_1 = \sqrt{x^2+y^2+(z-\sigma)^2} = \sigma(\xi-\eta) \\
	r_2 = \sqrt{x^2+y^2+(z+\sigma)^2} = \sigma(\xi+\eta)
\end{subnumcases}

在这个坐标系下,系统的Lagrange函数为
\begin{equation}
	L = \frac12 m\sigma^2 (\xi^2-\eta^2)\left(\frac{\dot{\xi}^2}{\xi^2-1} + \frac{\dot{\eta}^2}{1-\eta^2}\right) + \frac12 m\sigma^2(\xi^2-1)(1-\eta^2)\dot{\phi}^2 - V(\xi,\eta,\phi)
	\label{chp3:椭圆坐标下的分离变量-1}
\end{equation}
此时,广义动量为
\begin{equation*}
	p_\xi = \frac{\pl L}{\pl \dot{\xi}} = m\sigma^2 \frac{\xi^2-\eta^2}{\xi^2-1} \dot{\xi},\quad p_\eta = \frac{\pl L}{\pl \dot{\eta}} = m\sigma^2 \frac{\xi^2-\eta^2}{1-\eta^2} \dot{\eta},\quad p_\phi = \frac{\pl L}{\pl \dot{\phi}} = m\sigma^2(\xi^2-1)(1-\eta^2)\dot{\phi}
\end{equation*}
由此可得体系的Hamilton函数为
\begin{equation}
	H = \frac{1}{2m\sigma^2(\xi^2-\eta^2)}\left[(\xi^2-1)p_\xi^2+(1-\eta^2)p_\eta^2+\left(\frac{1}{\xi^2-1}+\frac{1}{1-\eta^2}\right)p_\phi^2\right] + V(\xi,\eta,\phi)
	\label{chp3:椭圆坐标下的分离变量-2}
\end{equation}

在这种坐标下,物理上有意义的可分离变量情况对应于势能为
\begin{equation}
	V(\xi,\eta,\phi) = \frac{a(\xi)+b(\eta)}{\xi^2-\eta^2} = \frac{\sigma^2}{r_1r_2}\left[a\left(\frac{r_1+r_2}{2\sigma}\right)+b\left(\frac{r_1-r_2}{2\sigma}\right)\right]
	\label{chp3:椭圆坐标下的分离变量-3}
\end{equation}
所以我们得到关于$W(\mbf{q},\mbf{D})$的方程
\begin{align}
	\frac{1}{2m\sigma^2(\xi^2-\eta^2)} & \left[(\xi^2-1)\left(\frac{\pl W}{\pl \xi}\right)^2+(1-\eta^2)\left(\frac{\pl W}{\pl \eta}\right)^2+\left(\frac{1}{\xi^2-1}+\frac{1}{1-\eta^2}\right)\left(\frac{\pl W}{\pl \phi}\right)^2\right] \nonumber \\
	& \quad \quad \quad {} + \frac{a(\xi)+b(\eta)}{\xi^2-\eta^2} = E
	\label{chp3:椭圆坐标下的分离变量-4}
\end{align}
在式\eqref{chp3:椭圆坐标下的分离变量-4}两端乘以$2m\sigma^2(\xi^2-\eta^2)$并整理,可得
\begin{align*}
	(\xi^2-1)\left(\frac{\pl W}{\pl \xi}\right)^2+2m\sigma^2a(\xi)-2m\sigma^2E\xi^2 & +(1-\eta^2)\left(\frac{\pl W}{\pl \eta}\right)^2+2m\sigma^2b(\eta)+2m\sigma^2E\eta^2 \\
	& {}+\left(\frac{1}{\xi^2-1}+\frac{1}{1-\eta^2}\right)\left(\frac{\pl W}{\pl \phi}\right)^2 = 0
\end{align*}
注意到坐标$\phi$是循环坐标,因此可令
\begin{equation}
	S(\xi,\eta,\phi,t) = -Et+p_\phi\phi + S_1(\xi)+S_2(\eta)
\end{equation}
由此得到两个方程
\begin{subnumcases}{}
	(1-\eta^2)\left(\frac{\mathrm{d}S_2}{\mathrm{d}\eta}\right)^2+2m\sigma^2b(\eta)-2m\sigma^2E(1-\eta^2)+\frac{p_\phi^2}{1-\eta^2} = \beta \\
	(\xi^2-1)\left(\frac{\mathrm{d}S_1}{\mathrm{d}\xi}\right)^2+2m\sigma^2a(\xi)-2m\sigma^2E(\xi^2-1)+\frac{p_\phi^2}{\xi^2-1} = -\beta
\end{subnumcases}
由此即有
\begin{align}
	S & = -Et+p_\phi\phi \pm \int \sqrt{2m\sigma^2E+\frac{\beta-2m\sigma^2b(\eta)}{1-\eta^2}- \frac{p_\phi^2}{(1-\eta^2)^2}} \mathrm{d}\eta \nonumber \\
	& \quad {} \pm \int \sqrt{2m\sigma^2E-\frac{\beta+2m\sigma^2a(\xi)}{\xi^2-1}- \frac{p_\phi^2}{(\xi^2-1)^2}} \mathrm{d}\xi
\end{align}
其中$E,p_\phi,\beta$为任意积分常数。
\end{example}

\subsection{利用Hamilton-Jacobi方程求解Kepler问题}

本节利用Hamilton-Jacobi方程求解Kepler问题,即求解质量为$m$的质点在势场$V = -\dfrac{\alpha}{r}(\alpha>0)$中的运动。

在球坐标系中,此系统的Hamilton函数为
\begin{equation}
	H = \frac{p_r^2}{2m}+\frac{p_\theta^2}{2mr^2}+\frac{p_\phi^2}{2mr^2\sin^2\theta} - \frac{\alpha}{r}
	\label{Hamilton-Jacobi方程解Kepler问题1}
\end{equation}
系统的Hamilton函数不显含时间,故根据不含时Hamilton-Jacobi方程\eqref{不含时Hamilton-Jacobi方程}可有
\begin{equation}
	\frac{1}{2m}\left[\left(\frac{\pl W}{\pl r}\right)^2 + \frac{1}{r^2}\left(\frac{\pl W}{\pl \theta}\right)^2 + \frac{1}{r^2\sin^2\theta}\left(\frac{\pl W}{\pl \phi}\right)^2\right] - \frac{\alpha}{r} = E
	\label{Hamilton-Jacobi方程解Kepler问题2}
\end{equation}
注意到$\phi$是循环坐标,因此可令
\begin{equation}
	S(r,\theta,\phi,t) = -Et + p_\phi\phi + S_1(r) + S_\theta(\theta)
	\label{Hamilton-Jacobi方程解Kepler问题3}
\end{equation}
由此得到两个方程
\begin{subnumcases}{}
	\left(\frac{\mathrm{d}S_2}{\mathrm{d}\theta}\right)^2 + \frac{p_\phi^2}{\sin^2\theta} = \beta^2 \label{Hamilton-Jacobi方程解Kepler问题4.1} \\
	\frac{1}{2m}\left[\left(\frac{\mathrm{d}S_1}{\mathrm{d}r}\right)^2+\frac{\beta^2}{r^2}\right] - \frac{\alpha}{r} = E \label{Hamilton-Jacobi方程解Kepler问题4.2}
\end{subnumcases}
由此即有
\begin{align}
	S = -Et+p_\phi\phi \pm \int \sqrt{2mE+\frac{2m\alpha}{r}-\frac{\beta^2}{r^2}}\mathrm{d}r \pm \int \sqrt{\beta^2-\frac{p_\phi^2}{\sin^2\theta}}\mathrm{d}\theta
	\label{Hamilton-Jacobi方程解Kepler问题5}
\end{align}
其中$E,p_\phi,\beta$为任意积分常数。为了使结果更具有几何意义,现在用坐标$r$和$\theta$的初始值$r_0$和$\theta_0$来取代积分常数$E$和$p_\phi$,令
\begin{equation}
\begin{cases}
	p_\phi^2 = \beta^2 \sin^2\theta_0 \\
	E = \dfrac{\beta^2}{2mr_0^2}-\dfrac{\alpha}{r_0}
\end{cases}\label{Hamilton-Jacobi方程解Kepler问题6}
\end{equation}
此时可有
\begin{align}
	S & = -\left(\frac{\beta^2}{2mr_0^2}-\frac{\alpha}{r_0}\right)t \pm \beta\sin \theta_0 \phi \pm \sqrt{2m}\int \sqrt{\frac{\beta^2}{2m}\left(\frac{1}{r_0^2}-\frac{1}{r^2}\right)-\alpha\left(\frac{1}{r_0}-\frac{1}{r}\right)}\mathrm{d}r \nonumber \\
	& \quad {} \pm \beta \int \sqrt{1-\frac{\sin^2\theta_0}{\sin^2\theta}}\mathrm{d}\theta
	\label{Hamilton-Jacobi方程解Kepler问题7}
\end{align}
于是,质点在空间中的轨迹由如下方程决定
\begin{subnumcases}{}
	\frac{\pl S}{\pl r_0} = -p_{r_0} \label{Hamilton-Jacobi方程解Kepler问题8.1} \\
	\frac{\pl S}{\pl \theta_0} = -p_{\theta_0} \label{Hamilton-Jacobi方程解Kepler问题8.2} \\
	\frac{\pl S}{\pl \beta} = -\gamma \label{Hamilton-Jacobi方程解Kepler问题8.3}
\end{subnumcases}
首先假设在\eqref{Hamilton-Jacobi方程解Kepler问题7}中都取正号,则式\eqref{Hamilton-Jacobi方程解Kepler问题8.1}化为
\begin{equation*}
	-\left(-\frac{\beta^2}{mr_0^3}+\frac{\alpha}{r_0^2}\right)t + \sqrt{\frac{m}{2}} \int \frac{-\dfrac{\beta^2}{mr_0^3}+\dfrac{\alpha}{r_0^2}}{\sqrt{\dfrac{\beta^2}{2m}\left(\dfrac{1}{r_0^2}-\dfrac{1}{r^2}\right) - \alpha\left(\dfrac{1}{r_0}-\dfrac1r\right)}} \mathrm{d}r = -p_{r_0}
\end{equation*}
整理可得\footnote{积分下限的不同取值只会式\eqref{Hamilton-Jacobi方程解Kepler问题9}右端多一个与该固定取值相关的常数,此常数总可以通过选择时间起点来去除。此处将积分下限取为$r_0$,下文会说明,$r=r_0$是$\dot{r}$变号的位置。}
\begin{equation}
	\sqrt{\frac{m}{2}} \int_{r_0}^r \frac{\mathrm{d}r}{\sqrt{\left(\dfrac{1}{r_0}-\dfrac1r\right)\left[\dfrac{\beta^2}{2m}\left(\dfrac{1}{r_0}+\dfrac1r\right)-\alpha\right]}} = t-t_0
	\label{Hamilton-Jacobi方程解Kepler问题9}
\end{equation}
其中$t_0 = \dfrac{p_{r_0}}{-\dfrac{\beta^2}{mr_0^3}+\dfrac{\alpha}{r_0^2}}$。将式\eqref{Hamilton-Jacobi方程解Kepler问题9}改写为
\begin{equation*}
	\sqrt{\frac{mr_0}{2}} \int_{r_0}^r \frac{r\mathrm{d}r}{\sqrt{(r-r_0)\left[\left(\dfrac{\beta^2}{2mr_0}-\alpha\right)r+\dfrac{\beta^2}{2m}\right]}} = t-t_0
\end{equation*}
根据$\dfrac{\beta^2}{2mr_0}-\alpha$的符号,可以分为以下三种情形:
\begin{enumerate}
\item $\dfrac{\beta^2}{2mr_0}<\alpha$,此时$E = \dfrac{\beta^2}{2mr_0^2}-\dfrac{\alpha}{r_0}<0$,令
\begin{equation}
	r_1 = \dfrac{\dfrac{\beta^2}{2m}}{\alpha-\dfrac{\beta^2}{2mr_0}} > 0
	\label{Hamilton-Jacobi方程解Kepler问题10}
\end{equation}
则式\eqref{Hamilton-Jacobi方程解Kepler问题9}可以改写为
\begin{equation}
	\frac{mr_0}{\sqrt{2mr_0\alpha-\beta^2}} \int_{r_0}^r \frac{r\mathrm{d}r}{\sqrt{(r-r_0)(r_1-r)}} = t-t_0
	\label{Hamilton-Jacobi方程解Kepler问题11}
\end{equation}
对式\eqref{Hamilton-Jacobi方程解Kepler问题11}两端同时求对$t$的导数,可得
\begin{equation}
	\dot{r} = \frac{\sqrt{2mr_0\alpha-\beta^2}}{mr_0r} \sqrt{(r-r_0)(r_1-r)}
	\label{Hamilton-Jacobi方程解Kepler问题12}
\end{equation}
根据式\eqref{Hamilton-Jacobi方程解Kepler问题12}可得
\begin{equation}
	\dot{r}\big|_{r=r_0} = 0,\quad \dot{r}\big|_{r=r_1} = 0
\end{equation}
即$r=r_0$和$r=r_1$是轨道的两个拱点。由式\eqref{Hamilton-Jacobi方程解Kepler问题12}中的根式非负也同样可得此运动的径向范围为$r_0\leqslant r \leqslant r_1$,是有界运动。

做变量代换$r = \dfrac{r_0+r_1}{2}-\dfrac{r_1-r_0}{2}\cos \xi$,则有关于矢径长度的参数方程
\begin{equation}
\begin{cases}
	r = \dfrac{r_0+r_1}{2}-\dfrac{r_1-r_0}{2}\cos \xi \\
	t = \dfrac{mr_0(r_0+r_1)}{2\sqrt{2mr_0\alpha-\beta^2}} \left(\xi - \dfrac{r_1-r_0}{r_1+r_0}\sin \xi\right) + t_0
\end{cases}
\label{Hamilton-Jacobi方程解Kepler问题13}
\end{equation}

\item $\dfrac{\beta^2}{2mr_0}=\alpha$,此时$E = \dfrac{\beta^2}{2mr_0^2}-\dfrac{\alpha}{r_0}=0$,则式\eqref{Hamilton-Jacobi方程解Kepler问题9}可以改写为
\begin{equation}
	\frac{m\sqrt{r_0}}{\beta} \int_{r_0}^r \frac{r\mathrm{d}r}{\sqrt{r-r_0}} = t-t_0
	\label{Hamilton-Jacobi方程解Kepler问题14}
\end{equation}
对式\eqref{Hamilton-Jacobi方程解Kepler问题14}两端同时求对$t$的导数,可得
\begin{equation}
	\dot{r} = \frac{\beta}{mr}\sqrt{\frac{r-r_0}{r_0}}
	\label{Hamilton-Jacobi方程解Kepler问题15}
\end{equation}
即仅当$r=r_0$时,才有$\dot{r} = 0$。由式\eqref{Hamilton-Jacobi方程解Kepler问题15}中的根式非负可得此运动的径向范围为$r\geqslant r_0$,是无界运动。

求出式\eqref{Hamilton-Jacobi方程解Kepler问题14}中的积分,可得矢径长度与时间的关系
\begin{equation}
	t = \frac{2m}{3\beta}\sqrt{r_0(r-r_0)}(r+2r_0) + t_0
	\label{Hamilton-Jacobi方程解Kepler问题16}
\end{equation}
由此可见,当$t\to +\infty$时,$r\to +\infty$。

\item $\dfrac{\beta^2}{2mr_0}>\alpha$,此时$E = \dfrac{\beta^2}{2mr_0^2}-\dfrac{\alpha}{r_0}>0$,令
\begin{equation}
	r_1 = \dfrac{\dfrac{\beta^2}{2m}}{\dfrac{\beta^2}{2mr_0}-\alpha} > 0
	\label{Hamilton-Jacobi方程解Kepler问题17}
\end{equation}
则式\eqref{Hamilton-Jacobi方程解Kepler问题9}可以改写为
\begin{equation}
	\frac{mr_0}{\sqrt{\beta^2-2mr_0\alpha}} \int_{r_0}^r \frac{r\mathrm{d}r}{\sqrt{(r-r_0)(r+r_1)}} = t-t_0
	\label{Hamilton-Jacobi方程解Kepler问题18}
\end{equation}
对式\eqref{Hamilton-Jacobi方程解Kepler问题18}两端同时求对$t$的导数,可得
\begin{equation}
	\dot{r} = \frac{\sqrt{\beta^2-2mr_0\alpha}}{mr_0r} \sqrt{(r-r_0)(r+r_1)}
	\label{Hamilton-Jacobi方程解Kepler问题19}
\end{equation}
在$r>0$的范围内,仅当$r=r_0$时,才有$\dot{r}=0$。由式\eqref{Hamilton-Jacobi方程解Kepler问题19}中的根式非负可得此运动的径向范围为$r\geqslant r_0$,是无界运动。

做变量代换$r=-\dfrac{r_1-r_0}{2}+\dfrac{r_0+r_1}{2}\cosh\xi$,则有关于矢径长度的参数方程
\begin{equation}
\begin{cases}
	\displaystyle r = -\dfrac{r_1-r_0}{2}+\dfrac{r_0+r_1}{2}\cosh\xi \\
	\displaystyle t = \frac{mr_0(r_1-r_0)}{2\sqrt{\beta^2-2mr_0\alpha}} \left(\frac{r_1+r_0}{r_1-r_0}\sinh \xi-\xi\right) + t_0
\end{cases}
\label{Hamilton-Jacobi方程解Kepler问题20}
\end{equation}
\end{enumerate}

前述三种情形分别对应了三种完全不同的轨道形状,而当$t=t_0$时,$r=r_0$为行星运动过程中距太阳最近的点,该点即称为{\bf 近日点}。

现在来考虑方程\eqref{Hamilton-Jacobi方程解Kepler问题8.2}。假设在方程\eqref{Hamilton-Jacobi方程解Kepler问题8.2}中都取正号,则式\eqref{Hamilton-Jacobi方程解Kepler问题8.2}化为
\begin{equation*}
	\beta\cos\theta_0\phi - \beta \int \frac{\sin\theta_0\cos\theta_0\csc^2\theta}{\sqrt{1-\sin^2\theta_0\csc^2\theta}}\mathrm{d}\theta = -p_{\theta_0}
\end{equation*}
整理可得\footnote{基于与前面相同的原因,此处将积分下限取为$\theta_0$。}
\begin{equation}
	-\int_{\theta_0}^\theta \frac{\sin\theta_0\csc^2\theta}{\sqrt{1-\sin^2\theta_0\csc^2\theta}} \mathrm{d}\theta = -\phi - \frac{p_{\theta_0}}{\beta\cos\theta_0}
	\label{Hamilton-Jacobi方程解Kepler问题21}
\end{equation}
将式\eqref{Hamilton-Jacobi方程解Kepler问题21}左端积分,即有
\begin{equation}
	\arccos \frac{\cot\theta}{\cot\theta_0} = \phi + \frac{p_{\theta_0}}{\beta\cos \theta_0}
	\label{Hamilton-Jacobi方程解Kepler问题22}
\end{equation}
整理可得
\begin{equation}
	\frac{\cot\theta}{\cot\theta_0} = \sin(\phi-\varOmega)
	\label{Hamilton-Jacobi方程解Kepler问题23}
\end{equation}
其中$\varOmega = -\dfrac{\pi}{2}-\dfrac{p_{\theta_0}}{\beta\cos \theta_0}$。

\begin{figure}[htb]
\centering
\begin{asy}
	texpreamble("\usepackage{xeCJK}");
	texpreamble("\setCJKmainfont{SimSun}");
	usepackage("amsmath");
	usepackage("amssymb");
	usepackage("siunitx");
	import graph;
	import math;
	size(300);
	//行星的轨道示意图
	pair O,i,j,k;
	O = (0,0);
	i = (-sqrt(2)/4,-sqrt(14)/12);
	j = (sqrt(14)/4,-sqrt(2)/12);
	k = (0,2*sqrt(2)/3);
	real r;
	r = 1;
	draw(scale(r)*unitcircle,linewidth(0.8bp));
	real theta0,phi0,phi;
	theta0 = 0.3*pi;
	phi0 = pi/6;
	pair xyplane(real t){
		return r*cos(t)*i+r*sin(t)*j;
	}
	pair yzplane(real t){
		return r*cos(t)*j+r*sin(t)*k;
	}
	pair zxplane(real t){
		return r*cos(t)*k+r*sin(t)*i;
	}
	pair planetplane(real phi){
		real m,st,ct;
		m = sin(phi-phi0)/tan(theta0);
		st = 1/sqrt(m^2+1);
		ct = m/sqrt(m^2+1);
		return st*cos(phi)*i+st*sin(phi)*j+ct*k;
	}
	pair pqplane(real t){
		return r*cos(t)*cos(phi0)*i+r*cos(t)*sin(phi0)*j+r*sin(t)*k;
	}
	real gettheta_forplanet(real phi,real phi0){
		real m;
		m = sin(phi-phi0)/tan(theta0);
		return asin(1/sqrt(m^2+1));
	}
	real tmp;
	tmp = -1.1;
	draw(graph(xyplane,tmp,pi+tmp));
	draw(graph(xyplane,pi+tmp,2*pi+tmp),dashed);
	tmp = -0.8;
	draw(graph(yzplane,tmp,pi+tmp));
	draw(graph(yzplane,pi+tmp,2*pi+tmp),dashed);
	tmp = -0.2;
	draw(graph(zxplane,tmp,pi+tmp));
	draw(graph(zxplane,pi+tmp,2*pi+tmp),dashed);
	tmp = -1;
	draw(graph(planetplane,tmp,pi+tmp),red+linewidth(0.8bp));
	draw(graph(planetplane,pi+tmp,2*pi+tmp),dashed+red);
	pair n,m,q,p,b;
	n = intersectionpoint(graph(planetplane,tmp,pi+tmp),graph(xyplane,tmp,pi+tmp));
	m = intersectionpoint(graph(planetplane,pi+tmp,2*pi+tmp),graph(xyplane,pi+tmp,2*pi+tmp));
	draw(Label("$\boldsymbol{n}$",MidPoint,Relative(E)),O--n,Arrow);
	label("$N$",n,S);
	label("$M$",m,N);
	phi = 1;
	p = planetplane(phi);
	q = r*cos(phi)*i+r*sin(phi)*j;
	draw(Label("$\boldsymbol{q}$",MidPoint,Relative(W)),O--p,Arrow);
	label("$P$",p,E);
	draw(O--q,dashed);
	label("$Q$",q,SE);
	draw(Label("$\varOmega$",MidPoint,Relative(E)),graph(xyplane,0,phi0),blue+linewidth(0.8bp));
	draw(Label("$\phi-\varOmega$",MidPoint,Relative(E)),graph(xyplane,phi0,phi),green+linewidth(0.8bp));
	draw(Label("$\varTheta$",Relative(0.65),Relative(E)),graph(planetplane,phi0,phi),purple+linewidth(0.8bp));
	b = sin(phi0)*cos(theta0)*i-cos(phi0)*cos(theta0)*j+sin(theta0)*k;
	draw(Label("$\boldsymbol{b}$",EndPoint,W),O--b,Arrow);
	tmp = -1.2;
	phi0 = phi;
	draw(graph(pqplane,tmp,pi+tmp),orange);
	draw(graph(pqplane,pi+tmp,2*pi+tmp),dashed+orange);
	draw(Label("$\theta$",MidPoint,Relative(E)),graph(pqplane,pi/2-gettheta_forplanet(phi,pi/6),pi/2),orange+blue+linewidth(0.8bp));
	draw(O--r*i,dashed);
	draw(O--r*j,dashed);
	draw(O--r*k,dashed);
	draw(Label("$x$",EndPoint),r*i--2.5*r*i,Arrow);
	draw(Label("$y$",EndPoint),r*j--1.5*r*j,Arrow);
	draw(Label("$z$",EndPoint),r*k--1.5*r*k,Arrow);
	label("$O$",O,2*W);
\end{asy}
\caption{行星的轨道示意图}
\label{行星的轨道示意图}
\end{figure}

将式\eqref{Hamilton-Jacobi方程解Kepler问题23}与式\eqref{chp3:球面大圆的参数方程}对比,即与式
\begin{equation}
	\frac{\cot\theta}{\cot\left(\theta_0'-\dfrac{\pi}{2}\right)} = \sin\left(\phi-\phi_0'+\frac{\pi}{2}\right)
	\label{Hamilton-Jacobi方程解Kepler问题24}
\end{equation}
对比,可知式\eqref{Hamilton-Jacobi方程解Kepler问题23}所表示的球面曲线为大圆,即行星的轨迹是平面的。如图\ref{行星的轨道示意图}所示,作一个单位球面,利用其上的点表示行星在空间中的角度坐标$(\theta,\phi)$。考虑到当$\phi=\varOmega$时,$\theta=\dfrac{\pi}{2}$,即$\varOmega$表示轨道平面与$xOy$坐标平面的交叉线与$Ox$轴所成的角度。记大圆上任意一点$P$,其球面坐标为$(\theta,\phi)$,则单位向量$\overrightarrow{OP}=:\mbf{q}$可以表示为
\begin{equation}
	\mbf{q} = \sin\theta\cos\phi\mbf{e}_1+\sin\theta\sin\phi\mbf{e}_2+\cos\theta\mbf{e}_3
	\label{Hamilton-Jacobi方程解Kepler问题24-q}
\end{equation}
而轨道平面与$xOy$坐标平面交叉线$\overrightarrow{ON}$的方向向量$\mbf{n}$可以表示为
\begin{equation}
	\mbf{n} = \cos\varOmega\mbf{e}_1+\sin\varOmega\mbf{e}_2
	\label{Hamilton-Jacobi方程解Kepler问题24-n}
\end{equation}
因此,轨道平面的单位法向量可以表示为
\begin{equation}
	\mbf{b} = \frac{\mbf{n}\times \mbf{q}}{|\mbf{n}\times \mbf{q}|}
	\label{Hamilton-Jacobi方程解Kepler问题25}
\end{equation}
其中
\begin{align}
	\mbf{n}\times \mbf{q} & = \begin{vmatrix} \mbf{e}_1 & \mbf{e}_2 & \mbf{e}_3 \\ \cos\varOmega & \sin\varOmega & 0 \\ \sin\theta\cos\phi & \sin\theta\sin\phi & \cos\theta \end{vmatrix} \nonumber \\
	& = \cos\theta\sin\varOmega\mbf{e}_1 - \cos\theta\cos\varOmega\mbf{e}_2+\sin\theta(\sin\phi\cos\varOmega-\cos\phi\sin\varOmega)\mbf{e}_3 \nonumber \\
	& = \cos\theta\sin\varOmega\mbf{e}_1 - \cos\theta\cos\varOmega\mbf{e}_2+\sin\theta\sin(\phi-\varOmega)\mbf{e}_3
	\label{Hamilton-Jacobi方程解Kepler问题26}
\end{align}
将式\eqref{Hamilton-Jacobi方程解Kepler问题23}带入式\eqref{Hamilton-Jacobi方程解Kepler问题26}中,可得
\begin{equation}
	\mbf{n}\times \mbf{q} = \frac{\cos\theta}{\cos\theta_0}\left(\cos\theta_0\sin\varOmega\mbf{e}_1-\cos\theta_0\cos\varOmega\mbf{e}_2+\sin\theta_0\mbf{e}_3\right)
	\label{Hamilton-Jacobi方程解Kepler问题27}
\end{equation}
因此有
\begin{equation}
	\mbf{b} = \cos\theta_0\sin\varOmega\mbf{e}_1-\cos\theta_0\cos\varOmega\mbf{e}_2+\sin\theta_0\mbf{e}_3
	\label{Hamilton-Jacobi方程解Kepler问题28}
\end{equation}
即向量$\mbf{b}$为常向量,亦即说明行星的运转轨道是平面曲线,而该平面与$xOy$坐标平面之间的夹角为$\dfrac{\pi}{2}-\theta_0$。

最后考虑方程\eqref{Hamilton-Jacobi方程解Kepler问题8.3}。假设在方程\eqref{Hamilton-Jacobi方程解Kepler问题8.3}中都取正号,则式\eqref{Hamilton-Jacobi方程解Kepler问题8.3}化为
\begin{align*}
	-\frac{\beta}{mr_0^2}t + \phi\sin\theta_0 & {}+\sqrt{2m}\int_{r_0}^r \frac{\dfrac{\beta}{m}\left(\dfrac{1}{r_0^2}-\dfrac{1}{r^2}\right)}{2\sqrt{\dfrac{\beta^2}{2m}\left(\dfrac{1}{r_0^2}-\dfrac{1}{r^2}\right)-\alpha\left(\dfrac{1}{r_0}-\dfrac{1}{r}\right)}}\mathrm{d}r \\
	& {} + \int_{\theta_0}^\theta \sqrt{1-\frac{\sin^2\theta_0}{\sin^2\theta}}\mathrm{d}\theta = -\gamma
\end{align*}
整理可得\footnote{基于与前面相同的原因,此处将积分下限取为$r_0$和$\theta_0$。}
\begin{align}
	\frac{\beta}{\sqrt{2m}}\int_{r_0}^r \frac{\dfrac{1}{r_0^2}-\dfrac{1}{r^2}}{\sqrt{\dfrac{\beta^2}{2m}\left(\dfrac{1}{r_0^2}-\dfrac{1}{r^2}\right)-\alpha\left(\dfrac{1}{r_0}-\dfrac{1}{r}\right)}} \mathrm{d}r & {}+ \int_{\theta_0}^\theta \sqrt{1-\sin^2\theta_0\csc^2\theta} \mathrm{d}\theta \nonumber \\
	& = -\gamma + \frac{\beta}{mr_0^2}t - \phi\sin\theta_0
	\label{Hamilton-Jacobi方程解Kepler问题29}
\end{align}
在式\eqref{Hamilton-Jacobi方程解Kepler问题9}两端同时取微分,可得
\begin{equation}
	\frac{\mathrm{d}r}{\sqrt{\dfrac{\beta^2}{2m}\left(\dfrac{1}{r_0^2}-\dfrac{1}{r^2}\right)-\alpha\left(\dfrac{1}{r_0}-\dfrac{1}{r}\right)}} = \sqrt{\frac{2}{m}}\mathrm{d}t
	\label{Hamilton-Jacobi方程解Kepler问题30}
\end{equation}
将式\eqref{Hamilton-Jacobi方程解Kepler问题30}代入式\eqref{Hamilton-Jacobi方程解Kepler问题29}中,可得
\begin{equation}
	-\frac{\beta}{m}\int_{t_0}^t \frac{\mathrm{d}t}{r^2(t)} + \int_{\theta_0}^\theta \sqrt{1-\sin^2\theta_0\csc^2\theta} \mathrm{d}\theta = -\gamma + \frac{\beta}{mr_0^2}t_0-\phi\sin\theta_0
	\label{Hamilton-Jacobi方程解Kepler问题31}
\end{equation}
考虑式\eqref{Hamilton-Jacobi方程解Kepler问题23},即
\begin{equation}
	\cot\theta = \cot\theta_0\sin(\phi-\varOmega) = \cot\theta\cos\left(\phi-\varOmega-\dfrac{\pi}{2}\right)
	\label{Hamilton-Jacobi方程解Kepler问题32}
\end{equation}
当$\theta=\theta_0$时,有$\phi=\dfrac{\pi}{2}+\varOmega=:\phi_0$,因此将式\eqref{Hamilton-Jacobi方程解Kepler问题31}代入式\eqref{Hamilton-Jacobi方程解Kepler问题31}中可得
\begin{equation}
	-\frac{\beta}{m}\int_{t_0}^t \frac{\mathrm{d}t}{r^2(t)} + \int_{\phi_0}^\phi \frac{\csc\theta_0}{1+\cot^2\theta_0\cos^2(\phi-\phi_0)}\mathrm{d}\phi = -\gamma+\frac{\beta}{mr_0^2}t_0-\phi_0\sin\theta_0
	\label{Hamilton-Jacobi方程解Kepler问题33}
\end{equation}
计算式\eqref{Hamilton-Jacobi方程解Kepler问题33}中的积分如下
\begin{align*}
	\int_{\phi_0}^\phi \frac{\csc\theta_0}{1+\cot^2\theta_0\cos^2(\phi-\phi_0)}\mathrm{d}\phi & = \int_{\phi_0}^\phi \frac{\csc\theta_0}{\sin^2(\phi-\phi_0)+(1+\cot^2)\theta_0\cos^2(\phi-\phi_0)} \mathrm{d}\phi \\
	& = \int_{\phi_0}^\phi \frac{\csc\theta_0}{\sin^2(\phi-\phi_0)+\csc^2\theta_0\cos^2(\phi-\phi_0)} \mathrm{d}\phi \\
	& = \int_{\phi_0}^\phi \frac{\csc\theta_0\csc^2(\phi-\phi_0)}{1+\csc^2\theta_0\cot^2(\phi-\phi_0)} \mathrm{d}\phi \\
	& = -\arctan\big[\csc\theta_0\cot(\phi-\phi_0)\big]\bigg|_{\phi_0}^\phi \\
	& = -\arctan\big[\csc\theta_0\cot(\phi-\phi_0)\big]+\arctan(+\infty) \\
	& = -\arctan\big[\csc\theta_0\cot(\phi-\phi_0)\big] + \frac{\pi}{2}
\end{align*}
记$\displaystyle \varTheta = -\arctan\big[\csc\theta_0\cot(\phi-\phi_0)\big]$,则式\eqref{Hamilton-Jacobi方程解Kepler问题33}化为
\begin{equation}
	\varTheta = \frac{\beta}{m}\int_{t_0}^t \frac{\mathrm{d}t}{r^2(t)}+\varTheta_0
	\label{Hamilton-Jacobi方程解Kepler问题34}
\end{equation}
其中$\varTheta_0=-\dfrac{\pi}{2}-\gamma+\dfrac{\beta}{mr_0^2}t_0-\phi_0\sin\theta_0$,而$(r_0,\theta_0,\phi_0)$即为近日点的球坐标。此处$\varTheta$的几何意义即为图\ref{行星的轨道示意图}中的$\angle NOP$,这是因为根据式\eqref{Hamilton-Jacobi方程解Kepler问题24-q}、式\eqref{Hamilton-Jacobi方程解Kepler问题24-n}和式\eqref{Hamilton-Jacobi方程解Kepler问题23}可得
\begin{align*}
	\cos \angle NOP & = \mbf{n}\cdot \mbf{q} = \sin\theta\cos\phi\cos\varOmega+ \sin\theta\sin\phi\sin\varOmega \\
	& = \sin\theta\cos(\phi-\varOmega) = \frac{\cos(\phi-\varOmega)}{\sqrt{1+\cot^2\theta_0\sin^2(\phi-\varOmega)}}
\end{align*}
故
\begin{equation*}
	\tan\angle NOP = \csc\theta_0\tan(\phi-\varOmega) = -\csc\theta_0\cot(\phi-\phi_0)
\end{equation*}
时间$t_0$代表行星经过近日点$(r_0,\theta_0,\phi_0)$的日期,因此式\eqref{Hamilton-Jacobi方程解Kepler问题34}中的常数$\varTheta_0$代表近日点与升交点$N$之间在单位球面上的角度差,也常以$\tilde{\omega}$表示,又叫做{\bf 自升交点的近日点黄经}。

下面来考虑行星运行轨道的具体形状。将式\eqref{Hamilton-Jacobi方程解Kepler问题34}两端求对$t$的导数可得
\begin{equation}
	\frac{\mathrm{d}\varTheta}{\mathrm{d}t} = \frac{\beta}{mr^2}
	\label{Hamilton-Jacobi方程解Kepler问题35}
\end{equation}
同理将式\eqref{Hamilton-Jacobi方程解Kepler问题9}两端求对$t$的导数可得
\begin{equation}
	\frac{\mathrm{d}r}{\mathrm{d}t} = \sqrt{\frac{2}{m}\left(\frac{\beta^2}{2mr_0^2}-\frac{\alpha}{r_0}+ \frac{\alpha}{r}- \frac{\beta^2}{2mr^2}\right)} = \sqrt{\frac{2}{m}\left(E+\frac{\alpha}{r}-\frac{\beta^2}{2mr^2}\right)}
	\label{Hamilton-Jacobi方程解Kepler问题36}
\end{equation}
由式\eqref{Hamilton-Jacobi方程解Kepler问题35}和\eqref{Hamilton-Jacobi方程解Kepler问题36}可得
\begin{equation}
	\frac{\mathrm{d}r}{\mathrm{d}\varTheta} = r^2\sqrt{\frac{2mE}{\beta^2}+ \frac{2m\alpha}{\beta^2} \frac1r - \frac{1}{r^2}}
	\label{Hamilton-Jacobi方程解Kepler问题37}
\end{equation}
如果令$u = \dfrac1r$可将式\eqref{Hamilton-Jacobi方程解Kepler问题37}化为
\begin{equation*}
	-\frac{\mathrm{d}u}{\sqrt{\dfrac{m^2\alpha^2}{\beta^4}\left(1+\dfrac{2\beta^2E}{m\alpha^2}\right) - \left(u - \dfrac{m\alpha}{\beta^2}\right)^2}} = \mathrm{d}\varTheta
\end{equation*}
由此可得
\begin{equation*}
	r = \frac{\dfrac{\beta^2}{m\alpha}}{1+\sqrt{1+\dfrac{2\beta^2E}{m\alpha^2}}\cos(\varTheta-\varTheta_0)}
\end{equation*}
记
\begin{equation}
	p = \dfrac{\beta^2}{m\alpha},\quad e = \sqrt{1+\dfrac{2\beta^2E}{m\alpha^2}}
	\label{Hamilton-Jacobi方程解Kepler问题38}
\end{equation}
即有
\begin{equation}
	r = \frac{p}{1+e\cos(\varTheta-\varTheta_0)}
	\label{Hamilton-Jacobi方程解Kepler问题39}
\end{equation}
由此可得当$E<0$时,$e<1$,轨道为椭圆;当$E=0$时,$e=1$,轨道为抛物线;当$E>0$时,$e>1$,轨道为双曲线。他们分别对应式\eqref{Hamilton-Jacobi方程解Kepler问题13}、式\eqref{Hamilton-Jacobi方程解Kepler问题16}和式\eqref{Hamilton-Jacobi方程解Kepler问题20}的三种情况。

至此,行星在空间中的绕日运动已经完全解决了。这个问题的六个积分常数分别为$r_0,~\theta_0,~\beta,~t_0,~\varOmega,~\varTheta_0$,而后面三个是常数$p_{r_0},~p_{\theta_0}$和$\gamma$的函数。

\begin{example}
用Hamilton-Jacobi方程解平面谐振子问题。
\end{example}
\begin{solution}
平面谐振子的Hamilton函数为
\begin{equation}
	H = \frac{1}{2m}(p_x^2+p_y^2) + \frac{k}{2}(x^2+y^2)
	\label{Hamilton-Jacobi方程解平面谐振子问题-1}
\end{equation}
系统的Hamilton函数不显含时间,故根据不含时Hamilton-Jacobi方程\eqref{不含时Hamilton-Jacobi方程}可有
\begin{equation}
	\frac{1}{2m} \left[\left(\frac{\pl W}{\pl x}\right)^2 + \left(\frac{\pl W}{\pl y}\right)^2\right] + \frac{k}{2}(x^2+y^2) = E
	\label{Hamilton-Jacobi方程解平面谐振子问题-2}
\end{equation}
将其分离变量可设
\begin{equation}
	S(x,y,t) = -Et + W(x,y) = -Et + S_1(x) + S_2(y)
	\label{Hamilton-Jacobi方程解平面谐振子问题-3}
\end{equation}
由此得到两个方程
\begin{subnumcases}{}
	\left(\frac{\mathrm{d} S_2}{\mathrm{d} y}\right)^2 + \frac{k}{2}y^2 = \frac{k}{2} \beta^2 \label{Hamilton-Jacobi方程解平面谐振子问题-4.1} \\
	\left(\frac{\mathrm{d} S_1}{\mathrm{d} x}\right)^2 + \frac{k}{2}x^2 = E - \frac{k}{2} \beta^2 \label{Hamilton-Jacobi方程解平面谐振子问题-4.2}
\end{subnumcases}
由此即有
\begin{equation}
	S(x,y,t) = -Et \pm \sqrt{mk} \int \sqrt{\frac{2E}{k}-\beta^2-x^2} \mathrm{d}x \pm \int \sqrt{\beta^2-y^2} \mathrm{d}y
	\label{Hamilton-Jacobi方程解平面谐振子问题-5}
\end{equation}
为了使式子更加具有对称性,令
\begin{equation}
	E = \frac{k}{2}(\alpha^2 + \beta^2)
	\label{Hamilton-Jacobi方程解平面谐振子问题-6}
\end{equation}
则有
\begin{equation}
	S(x,y,t) = -\frac{k}{2}(\alpha^2 + \beta^2) t \pm \sqrt{mk} \int \sqrt{\alpha^2-x^2} \mathrm{d}x \pm \sqrt{mk} \int \sqrt{\beta^2-y^2} \mathrm{d}y
	\label{Hamilton-Jacobi方程解平面谐振子问题-7}
\end{equation}
于是质点在空间中的轨迹由如下方程决定
\begin{subnumcases}{}
	\frac{\pl S}{\pl \alpha} = -p_{x_0} \label{Hamilton-Jacobi方程解平面谐振子问题-8.1} \\
	\frac{\pl S}{\pl \beta} = -p_{y_0} \label{Hamilton-Jacobi方程解平面谐振子问题-8.2}
\end{subnumcases}
由式\eqref{Hamilton-Jacobi方程解平面谐振子问题-8.1}可得
\begin{equation}
	-k\alpha t \pm \sqrt{mk} \int_\alpha^x \frac{\alpha}{\sqrt{\alpha^2-x^2}} \mathrm{d}x = -p_{x_0}
	\label{Hamilton-Jacobi方程解平面谐振子问题-9}
\end{equation}
由此即可得到
\begin{equation}
	x = \alpha \cos \left[\sqrt{\frac{k}{m}}\left(t-\frac{p_{x_0}}{\alpha k}\right)\right]
	\label{Hamilton-Jacobi方程解平面谐振子问题-10}
\end{equation}
由式\eqref{Hamilton-Jacobi方程解平面谐振子问题-8.2}可得
\begin{equation}
	-k\beta t \pm \sqrt{mk} \int_\beta^y \frac{\beta}{\sqrt{\beta^2-y^2}} \mathrm{d}y = -p_{y_0}
	\label{Hamilton-Jacobi方程解平面谐振子问题-11}
\end{equation}
由此即可得到
\begin{equation}
	y = \beta \cos \left[\sqrt{\frac{k}{m}}\left(t-\frac{p_{y_0}}{\beta k}\right)\right]
	\label{Hamilton-Jacobi方程解平面谐振子问题-12}
\end{equation}
至此,平面谐振子的问题就完全解决了,此问题的四个积分常数分别为$\alpha,~\beta,~p_{x_0}$和$p_{y_0}$,其中能量和$\alpha,~\beta$之间的关系为
\begin{equation}
	E = \frac{k}{2}(\alpha^2 + \beta^2)
	\label{Hamilton-Jacobi方程解平面谐振子问题-13}
\end{equation}
\end{solution}

\subsection{从“几何力学”到“波动力学”}

\subsubsection{作用波与作用波面}

考虑单粒子的Hamilton函数
\begin{equation*}
	H(\mbf{r},\mbf{p}) = \frac{\mbf{p}^2}{2m} + V(\mbf{r})
\end{equation*}
则有
\begin{equation*}
	\mbf{p} = \frac{\pl S}{\pl \mbf{r}} = \bnb S
\end{equation*}
考虑不含时Hamilton-Jacobi方程,则有
\begin{equation*}
	S(\mbf{r},t) = W(\mbf{r}) - Et
\end{equation*}
则不含时Hamilton-Jacobi方程可以写作
\begin{equation*}
	\frac{1}{2m} (\bnb W)^2 + V = E
\end{equation*}
此处
\begin{equation*}
	\left|\bnb W\right| = p = \sqrt{2m(E-V(\mbf{r}))}
\end{equation*}
由此可称
\begin{equation*}
	S(\mbf{r},t) = W(\mbf{r}) - Et = S_0
\end{equation*}
为{\heiti 运动曲面}。即在空间传播的等$S$面可视为{\heiti 作用波}波面,其正交曲线族为粒子在不同初始条件(能量相同,初始动量不同)下的轨道。波面法向亦即轨道切向为粒子动量方向。初始条件决定运动表现为粒子在作用波波面引导下前进。

\begin{figure}[htb]
\centering
\begin{asy}
	texpreamble("\usepackage{xeCJK}");
	texpreamble("\setCJKmainfont{SimSun}");
	usepackage("amsmath");
	import graph;
	import math;
	size(300);
	//作用波波面示意图
	picture pic;
	string[] text = {"$t_0$","$t$","$t'$"};
	real[] taus = {2,2.5,3};
	real[] sigmas = {2.5,3.5,4.5};
	for(real tau : taus){
		pair para1(real sigma){
			return (sigma*tau,(tau**2-sigma**2)/2);
		}
		draw(pic,rotate(56)*graph(para1,2,5),Arrow);
	}
	int i = 0;
	for(real sigma:sigmas){
		pair para2(real tau){
			return (sigma*tau,(tau**2-sigma**2)/2);
		}
		draw(pic,Label(text[i],BeginPoint,black),rotate(56)*graph(para2,1.8,3.15),red+dashed);
		i = i+1;
	}
	real tau = taus[1];
	real sigma = sigmas[1];
	pair para1dir(real sigma,real tau){
		return (tau,-sigma);
	}
	pair para1(real sigma,real tau){
		return (sigma*tau,(tau**2-sigma**2)/2);
	}
	dot(pic,rotate(56)*para1(sigma,tau));
	draw(pic,Label("$\boldsymbol{p}$",EndPoint,Relative(W)),rotate(56)*(para1(sigma,tau)--para1(sigma,tau)+0.5*para1dir(sigma,tau)),Arrow);
	label(pic,"$W(\boldsymbol{r})=S_0+Et$",rotate(56)*para1(sigmas[1],taus[2]),6*N);
	add(pic);
\end{asy}
\caption{作用波波面示意图}
\label{作用波波面示意图}
\end{figure}

\subsubsection{波动光学与几何光学的关系}

光的“微粒说”与“波动说”的理论统一由Hamilton完成。考虑介质电磁波方程
\begin{equation}
	\nabla^2 \varPhi - \frac{1}{u^2} \frac{\pl^2 \varPhi}{\pl t^2} = 0
	\label{介质电磁波方程}
\end{equation}
式中$\varPhi$为电磁场的任一分量,$u(\mbf{r}) = \dfrac{c}{n(\mbf{r})}$为非均匀介质中的波速。

对单色波,其波函数可表示为
\begin{equation*}
	\varPhi(\mbf{r},t) = \phi(\mbf{r}) \mathrm{e}^{-\mathrm{i} \omega t}
\end{equation*}
将其代入\eqref{介质电磁波方程}中可得振幅$\phi(\mbf{r})$应满足的方程为
\begin{equation}
	\nabla^2 \phi + k^2 \phi = 0
	\label{单色波振幅方程}
\end{equation}
式中$k = \dfrac{\omega}{u} = n\dfrac{\omega}{c} = nk_0$,其中$k_0$为此电磁波在真空中的波数。

在均匀介质中,单色波的振幅为
\begin{equation*}
	\phi(\mbf{r}) = A\mathrm{e}^{\mathrm{i}\mbf{k}\cdot \mbf{r}} = A \mathrm{e}^{\mathrm{i} k_0 L(\mbf{r})}
\end{equation*}
式中,函数$L(\mbf{r}) = \dfrac{\mbf{k}}{k} \cdot n\mbf{r}$称为{\heiti 光程函数}。即单色波在均匀介质中的波函数为
\begin{equation*}
	\varPhi(\mbf{r},t) = A \mathrm{e}^{\mathrm{i}(k_0 L(\mbf{r})-\omega t)}
\end{equation*}
波函数的等相位面即为{\heiti 波面},单色波在均匀介质中的波面为平面族:
\begin{equation*}
	\theta = k_0 L(\mbf{r}) - \omega t = \theta_0
\end{equation*}
波面的正交曲线族即为{\heiti 光线},单色波在均匀介质中的光线为直线。

\begin{figure}[htb]
\centering
\begin{asy}
	texpreamble("\usepackage{xeCJK}");
	texpreamble("\setCJKmainfont{SimSun}");
	usepackage("amsmath");
	import graph;
	import math;
	size(300);
	//均匀介质中单色波的波面和光线
	pair A,B,h;
	A = (0,0);
	B = (2,0);
	h = 0.25*dir(90);
	draw(A--B,Arrow);
	draw(shift(h)*(A--B),Arrow);
	draw(shift(2*h)*(A--B),Arrow);
	draw(Label("$t_0$",BeginPoint,black),interp(A,B,0.15)-0.5*h--interp(A,B,0.15)+2.5*h,red+dashed);
	draw(Label("$t$",BeginPoint,black),interp(A,B,0.5)-0.5*h--interp(A,B,0.5)+2.5*h,red+dashed);
	draw(Label("$t'$",BeginPoint,black),interp(A,B,0.85)-0.5*h--interp(A,B,0.85)+2.5*h,red+dashed);
	dot(interp(A,B,0.5)+h);
	draw(Label("$\boldsymbol{k}$",EndPoint,Relative(E)),interp(A,B,0.5)+h--interp(A,B,0.5)+h+0.4*dir(0),Arrow);
\end{asy}
\caption{均匀介质中单色波的波面和光线}
\label{均匀介质中单色波的波面和光线}
\end{figure}

在非均匀介质中,单色波的振幅可以表示为
\begin{equation*}
	\phi(\mbf{r}) = A(\mbf{r}) \mathrm{e}^{\mathrm{i}k_0 L(\mbf{r})}
\end{equation*}
将其代入式\eqref{单色波振幅方程}中,可得
\begin{equation*}
	\nabla^2 \phi = \mathrm{e}^{\mathrm{i}k_0 L} \left[\nabla^2 A + \mathrm{i} k_0(A\nabla^2 L + 2\bnb A \cdot \bnb L) - k_0^2 A(\bnb L)^2 \right] = -k^2 A \mathrm{e}^{\mathrm{i} k_0 L}
\end{equation*}
即有
\begin{equation}
	\frac{1}{k_0^2} \nabla^2 A + \frac{\mathrm{i}}{k_0} (A\nabla^2 L + 2\bnb A \cdot \bnb L) - \left[(\bnb L)^2 - n^2\right] A = 0
	\label{非均匀介质中的单色波方程}
\end{equation}
在短波长极限(几何光学极限)下,即$\lambda_0 \to 0$时,此时$k_0 \to +\infty$,在此近似下式\eqref{非均匀介质中的单色波方程}的前两项均为小量,忽略之可得{\heiti 程函方程}
\begin{equation}
	(\bnb L)^2 = n^2
	\label{程函方程}
\end{equation}
在此情形下,单色波的相位为
\begin{equation*}
	\psi(\mbf{r},t) = k_0 L(\mbf{r}) - \omega t
\end{equation*}
波面(即波函数的等相位面)的方程为
\begin{equation*}
	\psi(\mbf{r},t) = k_0 L(\mbf{r}) - \omega t = \psi_0
\end{equation*}
或者
\begin{equation*}
	k_0 L(\mbf{r}) = \psi_0 + \omega t
\end{equation*}
根据程函方程\eqref{程函方程}可得相位$\psi(\mbf{r},t)$满足
\begin{equation*}
	|\bnb \psi| = k_0 n = k
\end{equation*}
由此可有
\begin{equation*}
	\bnb \psi = \mbf{k}
\end{equation*}
即,光线(波面的正交曲线族)可视为“光微粒”的轨道。光的“微粒说”与“波动说”在短波长极限下统一。

\subsubsection{物质波假说与Schr\"{o}dinger方程}

在$\lambda \to 0$时,波动光学的极限即为几何光学。是否在某种极限下,能够有某种波动力学,其极限为经典力学?

de Broglie提出物质波假说,即质点所对应的“物质波”角频率和波数满足
\begin{equation}
\begin{cases}
	E = \hbar \omega \\
	\mbf{p} = \hbar \mbf{k}
\end{cases}
\end{equation}
假设“单色”物质波振幅同样满足方程\eqref{单色波振幅方程}
\begin{equation*}
	\nabla^2 \phi + k^2 \phi = 0
\end{equation*}
再考虑到
\begin{equation*}
	k = \frac{p}{\hbar} = \frac{\sqrt{2m(E-V(\mbf{r}))}}{\hbar}
\end{equation*}
可得物质波振幅应满足的方程为
\begin{equation}
	\left[-\frac{\hbar^2}{2m} \nabla^2 + V(\mbf{r})\right] \phi = E\phi
	\label{定态Schrodinger方程}
\end{equation}
方程\eqref{定态Schrodinger方程}即为{\heiti 定态Schr\"{o}dinger方程}。

对于含时的情况下,类比于光学的情况,可设波函数的形式为
\begin{equation*}
	\varPhi(\mbf{r},t) = \phi(\mbf{r}) \mathrm{e}^{-\mathrm{i} \omega t} = \phi(\mbf{r}) \mathrm{e}^{-\mathrm{i} \frac{E}{\hbar} t}
\end{equation*}
其满足方程
\begin{equation}
	\mathrm{i}\hbar \frac{\pl \varPhi}{\pl t} = E\varPhi
\end{equation}
将定态Schr\"{o}dinger方程代入上式,即有{\heiti 含时Schr\"{o}dinger方程}
\begin{equation}
	\mathrm{i}\hbar \frac{\pl \varPhi}{\pl t} = \left[-\frac{\hbar^2}{2m} \nabla^2 + V(\mbf{r})\right] \phi
	\label{含时Schrodinger方程}
\end{equation}
类比$\bnb \dfrac{S}{\hbar} = \dfrac{\mbf{p}}{\hbar} = \mbf{k}$与波动光学的$\bnb \psi = \mbf{k}$,可知波函数的相位即为$\dfrac{S}{\hbar}$,因此波函数又可表示为
\begin{equation*}
	\varPhi(\mbf{r},t) = |\varPhi(\mbf{r},t)| \mathrm{e}^{\mathrm{i}\frac{S(\mbf{r},t)}{\hbar}}
\end{equation*}
据此考虑含时Schr\"{o}dinger方程\eqref{含时Schrodinger方程}的实部,即
\begin{equation*}
	\frac{\pl S}{\pl t} = -\left[\frac{1}{2m} (\bnb S)^2 + V - \frac{\hbar^2}{2m} \frac{\nabla^2 |\varPhi|}{|\varPhi|}\right]
\end{equation*}
取上式在$\hbar \to 0$的极限,可有
\begin{equation*}
	\frac{1}{2m} (\bnb S)^2 + V + \frac{\pl S}{\pl t} = 0
\end{equation*}
此即为单粒子的Hamilton-Jacobi方程。因此,量子力学在$\hbar \to 0$下的极限即为经典力学。

\section{Hamilton理论在物理学中的应用}

\subsection{连续体系的Lagrange方程}

首先,以一维轻质长弦的横振动为例导出一维弦的Lagrange函数。记弦的质量线密度为$\rho$,各点相对其平衡位置的位移可以表示为$\eta(x)$。首先将弦分为$n$等份,记每段弦的长度为$\Delta x$,作为$n$个质量为$\rho \Delta x$的质点处理。将这些质点的位移分别记作$\eta_i$,则它们的动能可以表示为
\begin{equation}
	T = \sum_{i=1}^n \frac12 \rho \Delta x \dot{\eta}_i^2
	\label{连续体系的Lagrange方程-1}
\end{equation}

\begin{figure}[htb]
\centering
\begin{asy}
	texpreamble("\usepackage{xeCJK}");
	texpreamble("\setCJKmainfont{SimSun}");
	usepackage("amsmath");
	import graph;
	import math;
	size(300);
	//一维弦的横振动
	real x,y;
	x = 2.2;
	y = 1.4;
	real f(real x){
		return atan(x);
	}
	draw(Label("$x$",EndPoint),(0,0)--(x,0),Arrow);
	draw(Label("$y$",EndPoint),(0,0)--(0,y),Arrow);
	path p = graph(f,0.2,2);
	draw(p,linewidth(0.8bp));
	pair A0,A1,A2,A3,A4,A5,A6,T1,T2;
	A0 = relpoint(p,0.2);
	A1 = relpoint(p,0.3);
	A2 = relpoint(p,0.4);
	T1 = -reldir(p,0.4);
	A3 = relpoint(p,0.5);
	A4 = relpoint(p,0.6);
	T2 = reldir(p,0.6);
	A5 = relpoint(p,0.7);
	A6 = relpoint(p,0.8);
	draw((A0.x,0)--A0,dashed);
	draw(Label("$\eta_{i-1}$",MidPoint,Relative(E)),(A1.x,0)--A1);
	draw((A2.x,0)--A2,dashed);
	draw(Label("$\eta_i$",MidPoint,Relative(E)),(A3.x,0)--A3);
	draw((A4.x,0)--A4,dashed);
	draw(Label("$\eta_{i+1}$",MidPoint,Relative(E)),(A5.x,0)--A5);
	draw((A6.x,0)--A6,dashed);
	real l,r;
	l = 0.3;
	dot(A2);
	draw(Label("$\boldsymbol{T}$",EndPoint,Relative(E)),A2--A2+l*T1,Arrow);
	dot(A4);
	draw(Label("$\boldsymbol{T}$",EndPoint,Relative(W)),A4--A4+l*T2,Arrow);
	draw(A2--A2+l*dir(180),dashed);
	draw(A4--A4+l*dir(0),dashed);
	r = 0.15;
	draw(Label("$\theta_{i-\frac12}$",BeginPoint),arc(A2,r,180,degrees(T1)));
	draw(Label("$\theta_{i+\frac12}$",BeginPoint),arc(A4,r,0,degrees(T2)));
\end{asy}
\caption{一维弦的横振动}
\label{chp3:一维弦的横振动}
\end{figure}

如图\ref{chp3:一维弦的横振动}所示,在微振动假设下,角$\theta_{i-\frac12}$和$\theta_{i+\frac12}$都很小,故有第$i$个质点所受的横向力为
\begin{align}
	F_i & = T\sin \theta_{i+\frac12} - T\sin \theta_{i-\frac12} \approx T\tan \theta_{i+\frac12} - T\tan \theta_{i-\frac12} \approx T \frac{\eta_{i+1}-\eta_i}{\Delta x} - T \frac{\eta_i-\eta_{i-1}}{\Delta x} \nonumber \\
	& = T\frac{\eta_{i+1}-\eta_{i-1}}{\Delta x}
	\label{连续体系的Lagrange方程-2}
\end{align}
由此可推知此体系的势能为
\begin{equation}
	V = \sum_{i=1}^n \frac12 T\left(\frac{\eta_{i+1}-\eta_i}{\Delta x}\right)^2 \Delta x
	\label{连续体系的Lagrange方程-3}
\end{equation}
令$n \to \infty$,同时$\Delta x\to 0$,则有
\begin{align}
	T & \to \int \frac12 \rho \left(\frac{\pl \eta}{\pl t}\right)^2 \mathrm{d}x \label{连续体系的Lagrange方程-4} \\
	V & \to \int \frac12 T\left(\frac{\pl \eta}{\pl x}\right)^2 \mathrm{d}x \label{连续体系的Lagrange方程-5}
\end{align}
由此得到一维弦的Lagrange函数为
\begin{equation}
	L = T-V = \int \left[\frac12 \rho \left(\frac{\pl \eta}{\pl t}\right)^2 - \frac12 T\left(\frac{\pl \eta}{\pl x}\right)^2\right] \mathrm{d}x = \int \mathscr{L} \mathrm{d}x
	\label{连续体系的Lagrange方程-5}
\end{equation}
其中
\begin{equation}
	\mathscr{L} = \mathscr{T} - \mathscr{V} = \frac12 \rho \left(\frac{\pl \eta}{\pl t}\right)^2 - \frac12 T \left(\frac{\pl \eta}{\pl x}\right)^2 
	\label{连续体系的Lagrange方程-6}
\end{equation}
称为该连续体系的{\bf Lagrange密度},而$\mathscr{T}$和$\mathscr{V}$则分别称为该连续体系的{\bf 动能密度}和{\bf 势能密度}。

由上面的讨论可以看出,连续体系的位置坐标$x$在Lagrange函数中并不是广义坐标\footnote{实际上,此连续体系的Lagrange函数与位置坐标$x$无关。},而仅仅是取代了分离体系中的求和指标$i$。对于每一个$x$和$t$,$\eta(x,t)$是广义坐标,而$x$和$t$同为Lagrange函数的参数。

对于一般的三维连续体系,其广义坐标可以取为$\eta_1(\mbf{x},t),\eta_2(\mbf{x},t),\cdots,\eta_s(\mbf{x},t)$,而体系的Lagrange函数则表示为
\begin{equation}
	L = \int_V \mathscr{L} \mathrm{d}V
	\label{连续体系的Lagrange方程-7}
\end{equation}
而三维空间中的Lagrange密度的一般形式为
\begin{equation*}
	\mathscr{L} = \mathscr{L} \left(\frac{\pl \eta_1}{\pl t}, \frac{\pl \eta_2}{\pl t}, \cdots, \frac{\pl \eta_s}{\pl t}, \bnb \eta_1, \bnb \eta_2, \cdots, \bnb \eta_s, \eta_1, \eta_2, \cdots, \eta_s, \mbf{x},t\right)
\end{equation*}
或者简单地记作
\begin{equation}
	\mathscr{L} = \mathscr{L} \left(\frac{\pl \mbf{\eta}}{\pl t}, \bnb \mbf{\eta}, \mbf{\eta}, \mbf{x},t\right)
	\label{连续体系的Lagrange方程-8}
\end{equation}
由此,根据Hamilton原理,体系的Lagrange方程由式
\begin{equation}
	\delta S = \delta \int_{t_1}^{t_2} \int_V \mathscr{L} \mathrm{d}V \mathrm{d}t = \int_{t_1}^{t_2} \int_V \delta \mathscr{L} \mathrm{d}V \mathrm{d}t = 0
	\label{连续体系的Lagrange方程-9}
\end{equation}
给出。由于此连续体系的广义坐标为$\mbf{\eta}(\mbf{x},t)$,因此此处的变分运算不仅满足$\delta t = 0$,而且满足$\delta \mbf{x} = \mbf{0}$。因此有
\begin{align}
	\delta \mathscr{L} = \sum_{k=1}^s \Bigg[\frac{\pl \mathscr{L}}{\pl \left(\dfrac{\pl \eta_k}{\pl t}\right)} \delta \dfrac{\pl \eta_k}{\pl t} + \sum_{j=1}^3 \frac{\pl \mathscr{L}}{\pl \left(\dfrac{\pl \eta_k}{\pl x_j}\right)} \delta \dfrac{\pl \eta_k}{\pl x_j} + \frac{\pl \mathscr{L}}{\pl \eta_k} \delta \eta_k\Bigg]
	\label{连续体系的Lagrange方程-10}
\end{align}
由于此处变分是对$\eta_k(\mbf{x},t)$的,因此可有
\begin{equation*}
	\delta \frac{\pl \eta_k}{\pl t} = \frac{\pl (\delta \eta_k)}{\pl t},\quad \delta \frac{\pl \eta_k}{\pl x_j} = \frac{\pl (\delta \eta_k)}{\pl x_j}
\end{equation*}
据此处理式\eqref{连续体系的Lagrange方程-10}中的各项,可有
\begin{align}
	\frac{\pl \mathscr{L}}{\pl \left(\dfrac{\pl \eta_k}{\pl t}\right)} \delta \dfrac{\pl \eta_k}{\pl t} & = \frac{\pl \mathscr{L}}{\pl \left(\dfrac{\pl \eta_k}{\pl t}\right)} \dfrac{\pl (\delta \eta_k)}{\pl t} \nonumber \\
	& = \frac{\mathrm{d}}{\mathrm{d}t} \Bigg(\frac{\pl \mathscr{L}}{\pl \left(\dfrac{\pl \eta_k}{\pl t}\right)} \delta \eta_k\Bigg) - \frac{\mathrm{d}}{\mathrm{d}t} \Bigg(\frac{\pl \mathscr{L}}{\pl \left(\dfrac{\pl \eta_k}{\pl t}\right)}\Bigg) \delta \eta_k \label{连续体系的Lagrange方程-11} \\
	\frac{\pl \mathscr{L}}{\pl \left(\dfrac{\pl \eta_k}{\pl x_j}\right)} \delta \dfrac{\pl \eta_k}{\pl x_j} & = \frac{\pl \mathscr{L}}{\pl \left(\dfrac{\pl \eta_k}{\pl x_j}\right)} \dfrac{\pl (\delta \eta_k)}{\pl x_j} \nonumber \\
	& = \frac{\mathrm{d}}{\mathrm{d}x_j} \Bigg(\frac{\pl \mathscr{L}}{\pl \left(\dfrac{\pl \eta_k}{\pl x_j}\right)} \delta \eta_k\Bigg) - \frac{\mathrm{d}}{\mathrm{d}x_j} \Bigg(\frac{\pl \mathscr{L}}{\pl \left(\dfrac{\pl \eta_k}{\pl x_j}\right)}\Bigg) \delta \eta_k \label{连续体系的Lagrange方程-12}
\end{align}
将式\eqref{连续体系的Lagrange方程-11}和式\eqref{连续体系的Lagrange方程-12}代入式\eqref{连续体系的Lagrange方程-10}中,可得
\begin{align}
	\delta \mathscr{L} & = \sum_{k=1}^s \Bigg[\frac{\mathrm{d}}{\mathrm{d}t} \Bigg(\frac{\pl \mathscr{L}}{\pl \left(\dfrac{\pl \eta_k}{\pl t}\right)} \delta \eta_k\Bigg) + \sum_{j=1}^3 \frac{\mathrm{d}}{\mathrm{d}x_j} \Bigg(\frac{\pl \mathscr{L}}{\pl \left(\dfrac{\pl \eta_k}{\pl x_j}\right)} \delta \eta_k\Bigg)\Bigg] \nonumber \\
	& \quad {}- \sum_{k=1}^s \Bigg[\frac{\mathrm{d}}{\mathrm{d}t} \Bigg(\frac{\pl \mathscr{L}}{\pl \left(\dfrac{\pl \eta_k}{\pl t}\right)}\Bigg) + \sum_{j=1}^3 \frac{\mathrm{d}}{\mathrm{d}x_j} \Bigg(\frac{\pl \mathscr{L}}{\pl \left(\dfrac{\pl \eta_k}{\pl x_j}\right)}\Bigg) - \frac{\pl \mathscr{L}}{\pl \eta_k}\Bigg] \delta \eta_k
	\label{连续体系的Lagrange方程-13}
\end{align}
此处,对函数$f\left(\dfrac{\pl \eta}{\pl t},\bnb \eta,\eta,\mbf{x},t\right)$的求导操作$\dfrac{\mathrm{d} f}{\mathrm{d}t}$和$\dfrac{\mathrm{d} f}{\mathrm{d}x_j}$分别表示对函数
\begin{equation*}
	g(\mbf{x},t) = f\left(\dfrac{\pl \eta}{\pl t}(\mbf{x},t),\bnb \eta(\mbf{x},t),\eta(\mbf{x},t),\mbf{x},t\right)
\end{equation*}
求$x_j$和$t$的偏导数。将式\eqref{连续体系的Lagrange方程-13}代入式\eqref{连续体系的Lagrange方程-9}可得
\begin{align}
	\delta S & = \int_{t_1}^{t_2} \int_V \sum_{k=1}^s \Bigg[\frac{\mathrm{d}}{\mathrm{d}t} \Bigg(\frac{\pl \mathscr{L}}{\pl \left(\dfrac{\pl \eta_k}{\pl t}\right)} \delta \eta_k\Bigg) + \sum_{j=1}^3 \frac{\mathrm{d}}{\mathrm{d}x_j} \Bigg(\frac{\pl \mathscr{L}}{\pl \left(\dfrac{\pl \eta_k}{\pl x_j}\right)} \delta \eta_k\Bigg)\Bigg] \mathrm{d}V\mathrm{d}t \nonumber \\
	& \quad {}- \int_{t_1}^{t_2} \int_V \sum_{k=1}^s \Bigg[\frac{\mathrm{d}}{\mathrm{d}t} \Bigg(\frac{\pl \mathscr{L}}{\pl \left(\dfrac{\pl \eta_k}{\pl t}\right)}\Bigg) + \sum_{j=1}^3 \frac{\mathrm{d}}{\mathrm{d}x_j} \Bigg(\frac{\pl \mathscr{L}}{\pl \left(\dfrac{\pl \eta_k}{\pl x_j}\right)}\Bigg) - \frac{\pl \mathscr{L}}{\pl \eta_k}\Bigg] \delta \eta_k \mathrm{d}V\mathrm{d}t \nonumber \\
	& = \int_V \sum_{k=1}^s \frac{\pl \mathscr{L}}{\pl \left(\dfrac{\pl \eta_k}{\pl t}\right)} \delta \eta_k \mathrm{d}V \bigg|_{t_1}^{t_2} + \int_{t_1}^{t_2} \oint_{\pl V} \sum_{k=1}^s \frac{\pl \mathscr{L}}{\pl \left(\dfrac{\pl \eta_k}{\pl x_j}\right)} \delta \eta_k n_j \mathrm{d}S \mathrm{d}t \nonumber \\
	& \quad {}- \int_{t_1}^{t_2} \int_V \sum_{k=1}^s \Bigg[\frac{\mathrm{d}}{\mathrm{d}t} \Bigg(\frac{\pl \mathscr{L}}{\pl \left(\dfrac{\pl \eta_k}{\pl t}\right)}\Bigg) + \sum_{j=1}^3 \frac{\mathrm{d}}{\mathrm{d}x_j} \Bigg(\frac{\pl \mathscr{L}}{\pl \left(\dfrac{\pl \eta_k}{\pl x_j}\right)}\Bigg) - \frac{\pl \mathscr{L}}{\pl \eta_k}\Bigg] \delta \eta_k \mathrm{d}V\mathrm{d}t
	\label{连续体系的Lagrange方程-14}
\end{align}
由于在边界上有$\delta \eta_k = 0$,即$\delta\eta_k(\mbf{x},t_1) = \delta\eta_k(\mbf{x},t_2) = 0$以及在$\pl V$上,$\delta \eta_k(\mbf{x},t) = 0$,因此式\eqref{连续体系的Lagrange方程-14}可化为
\begin{equation}
	\delta S = - \int_{t_1}^{t_2} \int_V \sum_{k=1}^s \Bigg[\frac{\mathrm{d}}{\mathrm{d}t} \Bigg(\frac{\pl \mathscr{L}}{\pl \left(\dfrac{\pl \eta_k}{\pl t}\right)}\Bigg) + \sum_{j=1}^3 \frac{\mathrm{d}}{\mathrm{d}x_j} \Bigg(\frac{\pl \mathscr{L}}{\pl \left(\dfrac{\pl \eta_k}{\pl x_j}\right)}\Bigg) - \frac{\pl \mathscr{L}}{\pl \eta_k}\Bigg] \delta \eta_k \mathrm{d}V\mathrm{d}t
	\label{连续体系的Lagrange方程-15}
\end{equation}
由于$\delta\eta_k$之间是相互独立的,故可得
\begin{equation}
	\frac{\mathrm{d}}{\mathrm{d}t} \Bigg(\frac{\pl \mathscr{L}}{\pl \left(\dfrac{\pl \eta_k}{\pl t}\right)}\Bigg) + \sum_{j=1}^3 \frac{\mathrm{d}}{\mathrm{d}x_j} \Bigg(\frac{\pl \mathscr{L}}{\pl \left(\dfrac{\pl \eta_k}{\pl x_j}\right)}\Bigg) - \frac{\pl \mathscr{L}}{\pl \eta_k} = 0,\quad k=1,2,\cdots,s
	\label{连续体系的Lagrange方程-16}
\end{equation}
式\eqref{连续体系的Lagrange方程-16}就是{\bf 连续体系的Lagrange方程}。

\begin{example}
获得一维轻质长弦在微振动近似下的横振动方程。
\end{example}
\begin{solution}
前文已得出一维轻质长弦在此情形下的Lagrange密度如式\eqref{连续体系的Lagrange方程-6}所示,即
\begin{equation*}
	\mathscr{L}\left(\frac{\pl \eta}{\pl t},\frac{\pl \eta}{\pl x},\eta,x,t\right) = \frac12 \rho \left(\frac{\pl \eta}{\pl t}\right)^2 - \frac12 T \left(\frac{\pl \eta}{\pl x}\right)^2 
\end{equation*}
将式\eqref{连续体系的Lagrange方程-6}代入\eqref{连续体系的Lagrange方程-16}中,即可得到
\begin{equation}
	\rho \frac{\pl^2 \eta}{\pl t^2} - T\frac{\pl^2 \eta}{\pl x^2} = 0
	\label{连续体系的Lagrange方程-17}
\end{equation}
这就是一维轻质长弦微振动的方程。这是一个波动方程,波速为$v = \sqrt{\dfrac{T}{\rho}}$。
\end{solution}

\subsection{电磁场的Lagrange方程}

真空中的电磁场的运动规律由Maxwell方程组给出:
\begin{subnumcases}{\label{电磁场的Lagrange方程-1}}
	\bnb \cdot \mbf{B} = 0 \label{电磁场的Lagrange方程-1.1} \\
	\bnb \times \mbf{E} + \frac{\pl \mbf{B}}{\pl t} = \mbf{0} \label{电磁场的Lagrange方程-1.2} \\
	\bnb \cdot \mbf{E} = \frac{\rho}{\eps_0} \label{电磁场的Lagrange方程-1.3} \\
	\bnb \times \mbf{B} - \eps_0\mu_0\frac{\pl \mbf{E}}{\pl t} = \mu_0 \mbf{j} \label{电磁场的Lagrange方程-1.4}
\end{subnumcases}
虽然$\mbf{E}$和$\mbf{B}$总共有$6$个分量,但其中只有$4$个分量是相互独立的\footnote{这一点可以用Maxwell方程组只有$4$个方程来印证。},因此它们并不适合取为电磁场的广义坐标。此处我们取电磁场的标量势$\phi$和矢量势$\mbf{A}$作为电磁场的广义坐标。此处标量势$\phi$和矢量势$\mbf{A}$与电磁场场变量$\mbf{E}$和$\mbf{B}$之间的关系可以由式\eqref{电磁场的Lagrange方程-1.1}和式\eqref{电磁场的Lagrange方程-1.2}决定,即
\begin{align}
	\mbf{E} & = -\bnb \phi - \frac{\pl \mbf{A}}{\pl t} \label{电磁场的Lagrange方程-2} \\
	\mbf{B} & = \bnb \times \mbf{A} \label{电磁场的Lagrange方程-3}
\end{align}
因此,当用$\phi$和$\mbf{A}$描述电磁场时,式\eqref{电磁场的Lagrange方程-1.3}和式\eqref{电磁场的Lagrange方程-1.4}是场的运动方程,而式\eqref{电磁场的Lagrange方程-1.1}和式\eqref{电磁场的Lagrange方程-1.2}则只是$\phi$和$\mbf{A}$的定义而已。

下面来根据场的运动方程式\eqref{电磁场的Lagrange方程-1.3}和式\eqref{电磁场的Lagrange方程-1.4}来构造电磁场的Lagrange密度。首先,电磁场的Lagrange密度中需要包含场量$\phi$、$\mbf{A}$和他们的导数。根据电磁场的时间平移对称性和空间平移对称性,电磁场的Lagrange密度中不能显含时间$t$和空间坐标$\mbf{x}$。记$x_0 = t,~A_0 = \phi$,则可设电磁场的Lagrange密度为\footnote{此处利用了Einstein求和约定,用重复的指标表示求和,求和范围为$0$到$3$。}
\begin{equation}
	\mathscr{L} = B_iA_i+ C_{ij}\frac{\pl A_i}{\pl x_j} + D_{ijk}A_i\frac{\pl A_j}{\pl x_k} + E_{ijkl} \frac{\pl A_i}{\pl x_j}\frac{\pl A_k}{\pl x_l}
	\label{电磁场的Lagrange方程-4}
\end{equation}
将式\eqref{电磁场的Lagrange方程-4}代入Lagrange方程\eqref{连续体系的Lagrange方程-16}中,并结合量纲分析可以确定电磁场的Lagrange密度可以取如下的形式\footnote{与Lagrange函数类似,Lagrange密度的形式也是不唯一的,这里取了形式最简单的一种。}
\begin{align}
	\mathscr{L} & = \frac12 \left(\eps_0E^2-\frac{B^2}{\mu_0}\right) - \rho \phi + \mbf{j}\cdot \mbf{A} \nonumber \\
	& = \frac12 \left[\eps_0\left(\bnb\phi+\frac{\pl \mbf{A}}{\pl t}\right)^2-\frac{1}{\mu_0}\left(\bnb \times \mbf{A}\right)^2\right] - \rho \phi + \mbf{j}\cdot \mbf{A}
	\label{电磁场的Lagrange方程-5}
\end{align}

\subsection{Schr\"{o}dinger方程的建立}

非相对论量子力学有三种不同的理论形式,它们正好和经典力学Hamilton理论的三种不同形式的动力学方程相对应。1925年10月Heisenberg建立的通常被称为{\bf 矩阵力学}的量子力学,经Dirac研究可表示为量子Poisson括号表示的正则方程的形式;独立于Heisenberg,1926年3月Schr\"{o}dinger所建立的量子力学的波动方程则是直接从Hamilton-Jacobi方程过渡而来的;1948-1950年,Dirac和Fermi建立的路径积分形式的量子力学,则与Hamilton原理的形式类似。

下面我们遵循Schr\"{o}dinger的本意,从氢原子核外电子的Hamilton-Jacobi方程出发,导出定态Schr\"{o}dinger方程。

对于氢原子核外电子,其Hamilton函数为
\begin{equation}
	H = \frac{1}{2m}(p_x^2+p_y^2+p_z^2) - \frac{\alpha}{r}
	\label{Schrodinger方程的建立-1}
\end{equation}
其中$r = \sqrt{x^2+y^2+z^2},~ \alpha = \dfrac{e^2}{4\pi \eps_0}$。因此其Hamilton-Jacobi方程为
\begin{equation}
	\frac{1}{2m}\left[\left(\frac{\pl W}{\pl x}\right)^2+ \left(\frac{\pl W}{\pl y}\right)^2 + \left(\frac{\pl W}{\pl z}\right)^2\right] - \frac{\alpha}{r} = E
	\label{Schrodinger方程的建立-2}
\end{equation}
其中$W = W(x,y,z)$是经典Hamilton特征函数。作变换
\begin{equation}
	W = \hbar \ln \psi
	\label{Schrodinger方程的建立-3}
\end{equation}
式中$\hbar$是一个常数。由于$\ln \psi$不带量纲,因此$\hbar$与Hamilton特征函数$W(x,y,z)$具有相同的量纲,即作用量的量纲。由此,Hamilton-Jacobi方程\eqref{Schrodinger方程的建立-2}可以变换为
\begin{equation}
	\left(\frac{\pl \psi}{\pl x}\right)^2 + \left(\frac{\pl \psi}{\pl y}\right)^2 + \left(\frac{\pl \psi}{\pl z}\right)^2 - \frac{2m}{\hbar^2}\left(E+\frac{\alpha}{r}\right)\psi^2 = 0
	\label{Schrodinger方程的建立-4}
\end{equation}
式\eqref{Schrodinger方程的建立-4}是完全经典的,没有附加任何量子假设。

现在假设电子不是一个经典粒子,是具有波粒二象性的。把电子看作是像电磁波那样的“物质波”,取一个适当的Lagrange密度,用Hamilton原理得出电子的波动方程。根据Hamilton原理可得
\begin{equation}
	\delta S = \delta \int_{t_1}^{t_2} L \mathrm{d}t = \int_{t_1}^{t_2} \delta L \mathrm{d}t = 0
	\label{Schrodinger方程的建立-5}
\end{equation}
为了使式\eqref{Schrodinger方程的建立-5}成立,可令$\delta L = 0$,即
\begin{equation}
	\delta L = \delta \int_V \mathscr{L}\mathrm{d}V
	\label{Schrodinger方程的建立-6}
\end{equation}
对于波动,有一个普遍适用的结论:体系的平均动能恒等于平均势能,又考虑到$\mathscr{L} = \mathscr{T}-\mathscr{V}$,因此我们可以得到电子波的Lagrange函数应该是一个恒等于零的函数。而上面将电子考虑为一个经典粒子,它的Hamilton-Jacobi方程就是式\eqref{Schrodinger方程的建立-4},这就是一个恒等于零的表达式。因此要将波粒二象性结合起来,最自然的选择就是取式\eqref{Schrodinger方程的建立-4}作为电子波的Lagrange密度\footnote{准确的讲是令其与电子波的Lagrange密度成正比,因为式\eqref{Schrodinger方程的建立-4}与Lagrange密度的量纲是不同的。}。即有
\begin{equation}
	\delta J = \delta \int_V \left[\left(\frac{\pl \psi}{\pl x}\right)^2 + \left(\frac{\pl \psi}{\pl y}\right)^2 + \left(\frac{\pl \psi}{\pl z}\right)^2 - \frac{2m}{\hbar^2}\left(E+\frac{\alpha}{r}\right)\psi^2\right] \mathrm{d}V = 0
	\label{Schrodinger方程的建立-7}
\end{equation}
具体计算
\begin{align*}
	\delta J & = 2\int_V \left[\frac{\pl \psi}{\pl x} \delta \frac{\pl \psi}{\pl x} + \frac{\pl \psi}{\pl y} \delta \frac{\pl \psi}{\pl y} + \frac{\pl \psi}{\pl z} \delta \frac{\pl \psi}{\pl z} - \frac{2m}{\hbar^2}\left(E+\frac{\alpha}{r}\right) \psi\delta \psi\right] \mathrm{d}V \\
	& = 2\int_V \bigg[\frac{\pl}{\pl x}\left(\frac{\pl \psi}{\pl x}\delta \psi\right) - \frac{\pl^2 \psi}{\pl x^2} \delta \psi + \frac{\pl}{\pl y}\left(\frac{\pl \psi}{\pl y}\delta \psi\right) - \frac{\pl^2 \psi}{\pl y^2} \delta \psi + \frac{\pl}{\pl z}\left(\frac{\pl \psi}{\pl z}\delta \psi\right) - \frac{\pl^2 \psi}{\pl z^2} \delta \psi \\
	& \quad {} - \frac{2m}{\hbar^2}\left(E+\frac{\alpha}{r}\right)\psi \delta\psi \bigg] \mathrm{d}V \\
	& = 2\int_V \left[\frac{\pl}{\pl x}\left(\frac{\pl \psi}{\pl x}\delta \psi\right) + \frac{\pl}{\pl y}\left(\frac{\pl \psi}{\pl y}\delta \psi\right) + \frac{\pl}{\pl z}\left(\frac{\pl \psi}{\pl z}\delta \psi\right) \right] \mathrm{d}V \\
	& \quad {} - 2\int_V \left[\frac{\pl^2 \psi}{\pl x^2} + \frac{\pl^2 \psi}{\pl y^2} + \frac{\pl^2 \psi}{\pl z^2} + \frac{2m}{\hbar^2}\left(E+\frac{\alpha}{r}\right)\psi \right] \delta\psi \mathrm{d}V \\
	& = 2\oint_{\pl V} \frac{\pl \psi}{\pl n} \delta \psi \mathrm{d}S - 2\int_V \left[\frac{\pl^2 \psi}{\pl x^2} + \frac{\pl^2 \psi}{\pl y^2} + \frac{\pl^2 \psi}{\pl z^2} + \frac{2m}{\hbar^2}\left(E+\frac{\alpha}{r}\right)\psi \right] \delta\psi \mathrm{d}V = 0
\end{align*}
因此可得
\begin{equation}
	-\frac{\hbar^2}{2m}\left(\frac{\pl^2 \psi}{\pl x^2} + \frac{\pl^2 \psi}{\pl y^2} + \frac{\pl^2 \psi}{\pl z^2}\right) - \frac{\alpha}{r}\psi = E\psi
	\label{Schrodinger方程的建立-8}
\end{equation}
以及
\begin{equation}
	\oint_{\pl V} \frac{\pl \psi}{\pl n} \delta \psi \mathrm{d}S = 0
	\label{Schrodinger方程的建立-9}
\end{equation}
式\eqref{Schrodinger方程的建立-9}是一个面积分为零的表达式,由于$\delta \psi$是任意的,因此可以取$\psi\big|_{\pl V} = 0$以及$\dfrac{\pl \psi}{\pl n}\bigg|_{\pl V} = 0$作为偏微分方程\eqref{Schrodinger方程的建立-8}的边界条件。而偏微分方程\eqref{Schrodinger方程的建立-8}即为氢原子中电子的定态Schr\"{o}dinger方程。
