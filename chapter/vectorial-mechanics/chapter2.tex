\chapter{质点和质点系动力学}

\section{Newton动力学定律}

动力学研究力学系统的运动以及产生或改变运动状态的原因。经典力学中认为质点是具有力学性质的几何点,其运动规律将由本节中介绍的动力学定律(或称公理)来决定。

经典力学的基础是Newton动力学定律(公理),这些公理是一些数学假设,其正确性由人类几个世纪的观察和实验得以验证。在上一章中曾提到,经典力学的时空观是绝对空间和绝对时间的组合,而相对于绝对空间作匀速直线运动的参考系称为{\bf 惯性参考系}。经典力学认为所有的惯性参考系对于所有的力学关系都是等价的,即力学定律和方程不依赖于具体的惯性参考系的选择,这被称为{\bf Galileo相对性原理}。

% \subsection{Newton第一定律}

\begin{law}[Newton第一定律]
	如果质点上没有力的作用,它将保持静止或匀速直线运动。
\end{law}

Newton第一定律也被称为{\bf 惯性公理}。在惯性公理中,这种没有力作用的质点称为孤立质点。惯性公理事实上假定了惯性参考系的存在,并断言存在这样的惯性参考系使得孤立质点相对于其呈静止或匀速直线运动。

% \subsection{Newton第二定律}

\begin{law}[Newton第二定律]
	当质点受力作用时,其加速度与力成正比,用公式可以表达为
	\begin{equation}
		\boldsymbol{F} = m\boldsymbol{a}
		\label{chapter2:牛顿第二定律}
	\end{equation}
\end{law}

Newton第二定律也被称为{\bf 动力学基本公理}。式\eqref{chapter2:牛顿第二定律}中的比例系数即为质点惯性的度量,而这种惯性度量值则正比于质点的质量,即其所含物质的多少。我们总是可以取恰当的单位制,使得这种正比关系的比例系数为$1$。

在经典力学中,质点的质量是具有可加性的恒正标量,并且不依赖于质点的运动状态。

% \subsection{Newton第三定律}

\begin{law}[Newton第三定律]
	如果一个质点受到另外一个质点的作用力,则第二个质点也会受到第一个质点的作用力,并且这两个力大小相等,方向沿两质点之间的连线相反。
\end{law}

Newton第三定律也被称为{\bf 质点相互作用公理}。Newton第三定律描述了一组质点之间作用力的关系,这使得我们可以进一步讨论质点系的力学性质。在Newton第三定律中描述的一对力,其相等关系和方向关系是瞬时成立的,其改变不需要传播时间,因而属于一种{\bf 超距作用}。

% \subsection{力的独立作用定律}

\begin{law}[力的独立作用定律]
	如果质点的位置、速度和其他相关物理状态不改变,两个质点之间的相互作用不可能改变其他质点对它们的相互作用。换言之,当质点$m_i(i=1,2,\cdots,n)$作用在某质点上的力为$\boldsymbol{F}_i(i=1,2,\cdots,n)$时,则它们单独作用时产生的加速度$\boldsymbol{a}_i(i=1,2,\cdots,n)$之间可以相加。
\end{law}

如果质点的质量是$m$,则根据Newton第二定律\eqref{chapter2:牛顿第二定律}可得$\boldsymbol{a}_i = \dfrac{\boldsymbol{F}_i}{m}$,因此根据力的独立作用公理可得该质点的加速度为
\begin{equation}
	\boldsymbol{a} = \boldsymbol{a}_1 + \boldsymbol{a}_2 + \cdots + \boldsymbol{a}_n = \frac{1}{m}\left(\boldsymbol{F}_1 + \boldsymbol{F}_2 + \cdots + \boldsymbol{F}_n\right)
\end{equation}
由此可见,质点获得的加速度$\boldsymbol{a}$就如同作用于其上的是一个等于这$n$个独立力之和的力$\boldsymbol{F}$一样:
\begin{equation}
	\boldsymbol{F} = \boldsymbol{F}_1 + \boldsymbol{F}_2 + \cdots + \boldsymbol{F}_n
\end{equation}
这被称为{\bf 力的合成定律},它与力的独立作用公理是等价的,这个用来代替$n$个独立力作用的力$\boldsymbol{F}$被称为这些独立力的{\bf 合力}。

\section{Galileo变换}

从一个惯性参考系变换至另一个惯性参考系的变换称为{\bf Galileo变换}(如图\ref{chapter2:Galileo变换}所示),它可以表示为
\begin{equation}
	\begin{cases}
		\mbf{r}' = \mbf{r} - \mbf{v}_0t \\
		t' = t
	\end{cases}
\end{equation}
在此变换下,可有
\begin{equation}
	\mbf{v}' = \mbf{v} - \mbf{v}_0,\quad \mbf{a}' = \mbf{a}
\end{equation}
由此可以看出相对速度和加速度是Galileo变换下的不变量。经典力学中假设质点的质量是与其运动状态无关的常标量,因此可以根据Newton第二定律得出{\bf 力也是Gelileo变换下的不变量}。

\begin{figure}[!htb]
\centering
\begin{asy}
	size(200);
	//Galileo变换
	pair O,x,y,z,v,r;
	O = (0,0);
	x = 0.8*dir(-135);
	y = dir(0);
	z = dir(90);
	v = (2/3,1/3);
	r = (0.9,1.2);
	draw(Label("$x$",EndPoint),O--x,Arrow);
	draw(Label("$y$",EndPoint),O--y,Arrow);
	draw(Label("$z$",EndPoint),O--z,Arrow);
	label("$O$",O,W);
	draw(Label("$x'$",EndPoint),v--v+x,Arrow);
	draw(Label("$y'$",EndPoint),v--v+y,Arrow);
	draw(Label("$z'$",EndPoint),v--v+z,Arrow);
	label("$O'$",v,SE);
	draw(Label(rotate(degrees(v))*"$\boldsymbol{v}_0 t$",Relative(0.7),Relative(W)),O--v,red,Arrow);
	draw(Label("$\boldsymbol{r}$",MidPoint,Relative(W)),O--r,Arrow);
	draw(Label("$\boldsymbol{r}'$",MidPoint,Relative(E)),v--r,Arrow);
\end{asy}
\caption{Galileo变换}
\label{chapter2:Galileo变换}
\end{figure}

\section{力系的主向量与主矩}

todo

\section{运动定理与守恒定律}

\subsection{质点系}

质点系即指两个以上相互作用的质点组成的系统。在质点系中,{\heiti 外力}指质点系外物体对质点系内任一质点的作用力,记作$\mbf{F}_i$。{\heiti 内力}指质点系内任意一对质点之间的相互作用力,质点$j$对质点$i$的作用力记作$\mbf{f}_{ij}$。

\subsection{动量及其守恒定律}

质点的{\heiti 动量}定义为
\begin{equation}
	\mbf{p} = m\mbf{v}
\end{equation}
根据Newton第二定律,有
\begin{equation}
	\mbf{F} = m\mbf{a} = m\frac{\mathrm{d} \mbf{v}}{\mathrm{d} t} = \frac{\mathrm{d} \mbf{p}}{\mathrm{d} t}
	\label{动量定理1}
\end{equation}
或者
\begin{equation}
	\mathrm{d} \mbf{p} = \mbf{F} \mathrm{d} t
	\label{动量定理2}
\end{equation}
式\eqref{动量定理1}或\eqref{动量定理2}表示的关系称为{\heiti 动量定理}。

质点系的动量定义为各个质点动量之和,即
\begin{equation}
	\mbf{p} = \sum_{i=1}^n \mbf{p}_i
\end{equation}
则有
\begin{equation*}
	\frac{\mathrm{d} \mbf{p}}{\mathrm{d} t} = \sum_{i=1}^n \frac{\mathrm{d} \mbf{p}_i}{\mathrm{d} t} = \sum_{i=1}^n \left(\mbf{F}_i + \sum_{j=1(j\neq i)}^n \mbf{f}_{ij}\right)
\end{equation*}
式中
\begin{equation*}
	\sum_{j=1(j\neq i)}^n \mbf{f}_{ij} = \sum_{i=1(i \neq j)}^n \mbf{f}_{ji} = \frac12 \sum_{i,j=1(i \neq j)}^n (\mbf{f}_{ij} + \mbf{f}_{ji}) = \mbf{0}
\end{equation*}
因此可有{\heiti 质点系的动量定理}
\begin{equation}
	\frac{\mathrm{d} \mbf{p}}{\mathrm{d} t} = \sum_{i=1}^n \mbf{F}_i = \mbf{F}^{(e)}
\end{equation}
即,质点系总动量的时间变化率等于合外力。

如果质点系所受合外力为零,则质点系的总动量保持不变。此即为{\heiti 动量守恒定律}。即如有$\mbf{F}^{(e)} = \mbf{0}$,根据质点系的动量定理,有
\begin{equation*}
	\mbf{F}^{(e)} = \dfrac{\mathrm{d} \mbf{p}}{\mathrm{d} t} = \mbf{0}
\end{equation*}
即总动量$\mbf{p} = \text{常矢量}$。

如果质点系在某方向上合外力为零,则质点系的总动量在该方向上的分量保持不变。即如果有常单位矢量$\mbf{e}$,满足$\mbf{e} \cdot \mbf{F}^{(e)} = 0$,则有
\begin{equation*}
	\mbf{e} \cdot \mbf{F}^{(e)} = \mbf{e} \cdot \frac{\mathrm{d} \mbf{p}}{\mathrm{d} t} = \frac{\mathrm{d} (\mbf{e} \cdot \mbf{p})}{\mathrm{d} t} = 0
\end{equation*}
即总动量在该方向上的分量$\mbf{e} \cdot \mbf{p} = \text{常数}$。

动量守恒定律在纯质点力学系统中由Newton第三定律保证。

\subsection{角动量及其守恒定律}

质点的{\heiti 角动量}\footnote{此处为对原点的角动量。}定义为
\begin{equation}
	\mbf{L} = \mbf{r} \times \mbf{p}
\end{equation}
{\heiti 力矩}\footnote{此处为对原点的力矩。}定义为
\begin{equation}
	\mbf{M} = \mbf{r} \times \mbf{F}
\end{equation}
则有
\begin{equation}
	\frac{\mathrm{d} \mbf{L}}{\mathrm{d} t} = \frac{\mathrm{d}}{\mathrm{d} t}(\mbf{r} \times \mbf{p}) = \mbf{v} \times \mbf{p} + \mbf{r} \times \frac{\mathrm{d} \mbf{p}}{\mathrm{d} t} = \mbf{r} \times \mbf{F} = \mbf{M}
	\label{角动量定理1}
\end{equation}
或者
\begin{equation}
	\mathrm{d} \mbf{L} = \mbf{M} \mathrm{d} t
	\label{角动量定理2}
\end{equation}
式\eqref{角动量定理1}和式\eqref{角动量定理2}表示的关系称为{\heiti 角动量定理}。

质点系的角动量定义为各个质点角动量之和,即
\begin{equation}
	\mbf{L} = \sum_{i=1}^n \mbf{L}_i
\end{equation}
则有
\begin{equation*}
	\frac{\mathrm{d} \mbf{L}}{\mathrm{d} t} = \sum_{i=1}^n \frac{\mathrm{d} \mbf{L}_i}{\mathrm{d} t} = \sum_{i=1}^n \mbf{r}_i \times \left(\mbf{F}_i + \sum_{j=1(j\neq i)}^n \mbf{f}_{ij}\right)
\end{equation*}
式中
\begin{align*}
	\sum_{i,j=1(i \neq j)}^n \mbf{r}_i \times \mbf{f}_{ij} & = \sum_{i,j=1(i\neq j)}^n \mbf{r}_j \times \mbf{f}_{ji} = \frac12 \sum_{i,j=1(i\neq j)}^n (\mbf{r}_i \times \mbf{f}_{ij} + \mbf{r}_j \times \mbf{f}_{ji}) \\
	& = \frac12 \sum_{i,j=1(i \neq j)}^n (\mbf{r}_i -\mbf{r}_j) \times \mbf{f}_{ij} = \frac12 \sum_{i,j=1(i \neq j)}^n \mbf{r}_{ij} \times \mbf{f}_{ij} = \mbf{0}
\end{align*}
此处利用了$\mbf{f}_{ij} \parallel \mbf{r}_{ij}$。由此即有{\heiti 质点系的角动量定理}
\begin{equation}
	\frac{\mathrm{d} \mbf{L}}{\mathrm{d} t} = \sum_{i=1}^n \mbf{r}_i \times \mbf{F}_i = \mbf{M}^{(e)}
\end{equation}
即,质点系的总角动量时间变化率等于合外力矩。

如果质点系所受合外力矩为零,则质点系的总角动量保持不变。此即为{\heiti 角动量守恒定律}。即如有$\mbf{M}^{(e)} = \mbf{0}$,根据质点系的角动量定理,有
\begin{equation*}
	\mbf{M}^{(e)} = \dfrac{\mathrm{d} \mbf{L}}{\mathrm{d} t} = \mbf{0}
\end{equation*}
即总角动量$\mbf{L} = \text{常矢量}$。

如果质点系在某方向上合外力矩为零,则质点系的总角动量在该方向上的分量保持不变。即如果有常单位矢量$\mbf{e}$,满足$\mbf{e} \cdot \mbf{M}^{(e)} = 0$,则有
\begin{equation*}
	\mbf{e} \cdot \mbf{M}^{(e)} = \mbf{e} \cdot \frac{\mathrm{d} \mbf{L}}{\mathrm{d} t} = \frac{\mathrm{d} (\mbf{e} \cdot \mbf{L})}{\mathrm{d} t} = 0
\end{equation*}
即总角动量在该方向上的分量$\mbf{e} \cdot \mbf{L} = \text{常数}$。

角动量守恒定律在纯质点力学系统中由Newton第三定律保证。

\subsection{能量及其守恒定律}

质点的{\heiti 动能}定义为
\begin{equation}
	T = \frac12 mv^2
\end{equation}
则有
\begin{equation}
	\frac{\mathrm{d} T}{\mathrm{d} t} = m\dot{\mbf{v}} \cdot \mbf{v} = \mbf{F} \cdot \mbf{v}
	\label{动能定理1}
\end{equation}
或者
\begin{equation}
	\mathrm{d}T = \mbf{F} \cdot \mbf{v}\mathrm{d} t = \mbf{F} \cdot \mathrm{d} \mbf{r}
	\label{动能定理2}
\end{equation}
式\eqref{动能定理1}和式\eqref{动能定理2}表示的关系称为{\heiti 动能定理}。

质点系的动能定义为各个质点动能之和,即
\begin{equation}
	T = \sum_{i=1}^n T_i
\end{equation}
则有
\begin{align*}
	\mathrm{d} T & = \sum_{i=1}^n \mathrm{d} T_i = \sum_{i=1}^n \left(\mbf{F}_i + \sum_{j=1(j\neq i)}^n \mbf{f}_{ij}\right) \cdot \mathrm{d} \mbf{r}_i
\end{align*}
式中
\begin{align*}
	\sum_{i,j=1(i\neq j)}^n \mbf{f}_{ij} \cdot \mathrm{d} \mbf{r}_i & = \frac12 \sum_{i,j=1(i\neq j)}^n (\mbf{f}_{ij} \cdot \mathrm{d} \mbf{r}_i +\mbf{f}_{ji} \cdot \mathrm{d} \mbf{r}_j) = \frac12 \sum_{i,j=1(i\neq j)}^n \mbf{f}_{ij} \cdot (\mathrm{d}\mbf{r}_i - \mathrm{d} \mbf{r}_j) \\
	& = \frac12 \sum_{i,j=1(i\neq j)}^n \mbf{f}_{ij} \cdot \mathrm{d} \mbf{r}_{ij}
\end{align*}
由此有{\heiti 质点系的动能定理}
\begin{equation}
	\mathrm{d} T = \sum_{i=1}^n \mbf{F}_i \cdot \mathrm{d} \mbf{r}_i + \frac12 \sum_{i,j=1(i\neq j)}^n \mbf{f}_{ij} \cdot \mathrm{d} \mbf{r}_{ij}
\end{equation}
即,质点系总动能增加量等于外力与内力的总功,内力共决定于质点间的相对运动。

如果力$\mbf{F}$可以写作某标量函数的梯度,即
\begin{equation}
	\mbf{F} = -\bnb V(\mbf{r})
\end{equation}
则该力称为{\heiti 保守力}。根据保守力的定义,可有
\begin{equation}
	\mbf{F} \cdot \mathrm{d} \mbf{r} = -\mathrm{d}V
\end{equation}
所以
\begin{equation*}
	\int_{\mbf{r}_0}^{\mbf{r}} \mbf{F} \cdot \mathrm{d} \mbf{r} = V(\mbf{r}_0) - V(\mbf{r})
\end{equation*}
即,保守力做功与路径无关。由此也可得
\begin{equation}
	\oint_C \mbf{F} \cdot \mathrm{d} \mbf{r} = 0
\end{equation}
即,对任意闭合路径$C$,保守力做功为零。

对于质点系,如果外力是保守力,即满足
\begin{equation}
	\mbf{F}_i = -\bnb_i V_i(\mbf{r}_i)
\end{equation}
其中
\begin{equation}
	\bnb_i = \mbf{e}_1 \frac{\pl}{\pl x_i} + \mbf{e}_2 \frac{\pl}{\pl y_i} + \mbf{e}_3 \frac{\pl}{\pl z_i}
\end{equation}
所以有
\begin{equation}
	\mbf{F}_i \cdot \mathrm{d} \mbf{r}_i = -\mathrm{d} V_i
\end{equation}
定义{\heiti 质点系外势能}为
\begin{equation}
	V^{(e)} = \sum_{i=1}^n V_i
\end{equation}
将有
\begin{equation}
	\sum_{i=1}^n \mbf{F}_i \cdot \mathrm{d} \mbf{r}_i = -\sum_{i=1}^n \mathrm{d} V_i = -\mathrm{d} V^{(e)}
\end{equation}
即,保守外力总功等于外势能减少。

如果质点系的内力是保守力,即有
\begin{align*}
	\mbf{f}_{ij} = -\bnb_i V_{ij}(\mbf{r}_{ij}) \\
	\mbf{f}_{ji} = -\bnb_j V_{ij}(\mbf{r}_{ij})
\end{align*}
为一对保守内力,其中
\begin{align*}
	& \mbf{r}_{ij} = \mbf{r}_i - \mbf{r}_j \\
	& \bnb_i = \bnb_{ij} = -\bnb_j
\end{align*}
由此内力可以表示为
\begin{equation*}
	\mbf{f}_{ij} = -\bnb_{ij} V_{ij}(\mbf{r}_{ij})
\end{equation*}
即有
\begin{equation}
	\mbf{f}_{ij} \cdot \mathrm{d} \mbf{r}_i + \mbf{f}_{ji} \cdot \mathrm{d} \mbf{r}_j = -\mathrm{d} V_{ij}
\end{equation}
定义{\heiti 质点系内势能}为
\begin{equation}
	V^{(i)} = \frac12 \sum_{i,j=1(i \neq j)}^n V_{ij}
\end{equation}
则有
\begin{equation}
	\sum_{i,j=1(i\neq j)}^n \mbf{f}_{ij} \cdot \mathrm{d} \mbf{r}_i = \frac12 \sum_{i,j=1(i \neq j)}^n \mbf{f}_{ij} \cdot \mathrm{d} \mbf{r}_{ij} = -\frac12 \sum_{i,j=1(i\neq j)}^n \mathrm{d} V_{ij} = -\mathrm{d} V^{(i)}
\end{equation}
即,保守内力总功等于内势能减少。

质点系的总势能定义为
\begin{equation}
	V(\mbf{r}_1,\mbf{r}_2,\cdots,\mbf{r}_n) = V^{(e)} + V^{(i)}
\end{equation}
则有
\begin{align*}
	-\bnb_i V^{(e)} & = -\sum_{j=1}^n \bnb_i V_j = \sum_{j=1}^n \delta_{ij} \mbf{F}_i = \mbf{F}_i \\
	-\bnb_i V^{(i)} & = -\frac12 \sum_{j,k=1(j \neq k)}^n \bnb_i V_{jk} = \frac12 \sum_{j,k=1(j\neq k)}^n (\delta_{ij} \mbf{f}_{ik} + \delta_{ik} \mbf{f}_{ij}) \\
	& = \frac12 \sum_{k=1(k\neq i)}^n \mbf{f}_{ik} + \frac12 \sum_{j(j\neq i)} \mbf{f}_{ij} = \sum_{j=1(j\neq i)}^n \mbf{f}_{ij}
\end{align*}
所以有
\begin{equation}
	\mbf{F}_i + \sum_{j=1(j\neq i)}^n \mbf{f}_{ij} = -\bnb_i V
\end{equation}
当所有力都为保守力(或者非保守力不做功)时,将有
\begin{equation}
	\mathrm{d} T = \sum_{i=1}^n \mbf{F}_i \cdot \mathrm{d} \mbf{r}_i + \frac12 \sum_{i,j=1(i\neq j)}^n \mbf{f}_{ij} \cdot \mathrm{d} \mbf{r}_{ij} = -\mathrm{d} V^{(e)} - \mathrm{d} V^{(i)} = -\mathrm{d} V
\end{equation}
定义{\heiti 总能量}\footnote{此处仅指机械能。}为
\begin{equation}
	E = T+V
\end{equation}
则在所有力都为保守力(或者非保守力不做功)时,将有
\begin{equation}
	\mathrm{d} E = 0
	\label{质点系能量守恒定律}
\end{equation}
式\eqref{质点系能量守恒定律}即为能量守恒定律。

\subsection{质心参考系}

质点系的{\heiti 质心}定义为
\begin{equation}
	\mbf{r}_C = \frac{\displaystyle \sum_{i=1}^n m_i \mbf{r}_i}{M} = \frac{\displaystyle \sum_{i=1}^n m_i \mbf{r}_i}{\displaystyle \sum_{i=1}^n m_i}
\end{equation}
质心的速度为
\begin{equation}
	\mbf{v}_C = \frac{\displaystyle \sum_{i=1}^n m_i \mbf{v}_i}{M}
\end{equation}
加速度为
\begin{equation}
	\mbf{a}_C = \frac{\displaystyle \sum_{i=1}^n m_i \mbf{a}_i}{M}
\end{equation}

显然质点系的总动量可以用质心速度表示为
\begin{equation*}
	\mbf{p} = M\mbf{v}_C
\end{equation*}
由此可有
\begin{equation}
	\frac{\mathrm{d} \mbf{p}}{\mathrm{d} t} = M\mbf{a}_C = \mbf{F}^{(e)}
	\label{质心运动定理}
\end{equation}
式\eqref{质心运动定理}表示的关系即为{\heiti 质心运动定理}。即,质点系的整体运动可视为全部质量集中于质心的单质点运动。

质心运动定理是引进质点概念的动力学基础。

\subsection{质心参考系的运动定理}

以质心为原点,并随之平动的参考系称为{\heiti 质心参考系}或{\heiti 质心系}。一般为非惯性系。

在质心系中,各质点的位矢和速度为
\begin{align}
	\mbf{r}'_i & = \mbf{r}_i - \mbf{r}_C \\
	\mbf{v}'_i & = \mbf{v}_i - \mbf{v}_C
\end{align}
即为各质点相对质心的运动(质点系的内部运动)。

\subsubsection{质心系动量定理}

在质心系中,总动量为
\begin{equation}
	\mbf{p}' = \sum_{i=1}^n m_i \mbf{v}'_i = \sum_{i=1}^n m_i(\mbf{v}_i - \mbf{v}_C) = \mbf{p} - M\mbf{v}_C = \mbf{0}
\end{equation}
即,在质心系中质点系的总动量为零。

\subsubsection{质心系角动量定理}
在质心系中,质点系的总角动量
\begin{equation}
	\mbf{L}' = \sum_{i=1}^n \mbf{r}'_i \times \mbf{p}'_i = \sum_{i=1}^n \mbf{r}'_i \times m_i \mbf{v}'_i
\end{equation}
即为质点系的{\heiti 内秉角动量}或{\heiti 自旋角动量}。质心的角动量
\begin{equation}
	\mbf{L}_C = \mbf{r}_C \times M\mbf{v}_C
\end{equation}
称为质点系的{\heiti 轨道角动量}。在实验室参考系中的总角动量为
\begin{align*}
	\mbf{L} & = \sum_{i=1}^n (\mbf{r}_C + \mbf{r}'_i) \times m_i(\mbf{v}_C + \mbf{v}'_i) \\
	& = \mbf{r}_C \times M\mbf{v}_C + \mbf{r}_C\times \mbf{p}' + \sum_{i=1}^n m_i \mbf{r}'_i \times \mbf{v}_C + \sum_{i=1}^n \mbf{r}'_i \times m_i \mbf{v}'_i \\
	& = \mbf{r}_C \times M\mbf{v}_C + \sum_{i=1}^n \mbf{r}'_i \times m_i \mbf{v}'_i
\end{align*}
所以有
\begin{equation}
	\mbf{L} = \mbf{L}_C+\mbf{L}'
\end{equation}
即,质点系的总角动量等于轨道角动量与内秉角动量之和。对于力矩,可有
\begin{equation}
	\mbf{M}^{(e)} = \sum_{i=1}^n \mbf{r}_i \times \mbf{F}_i = \sum_{i=1}^n (\mbf{r}_C+\mbf{r}'_i) \times \mbf{F}_i = \mbf{r}_C \times \mbf{F}^{(e)} + \mbf{M}'^{(e)}
\end{equation}
式中,$\displaystyle \mbf{M}'^{(e)} = \sum_{i=1}^n \mbf{r}'_i \times \mbf{F}_i$表示外力对质心的力矩。考虑到轨道角动量$\mbf{L}_C = \mbf{r}_C \times M\mbf{v}_C$,则有
\begin{equation}
	\frac{\mathrm{d} \mbf{L}_C}{\mathrm{d} t} = \mbf{v}_C \times M\mbf{v}_C + \mbf{r}_C \times M \mbf{a}_C = \mbf{r}_C \times \mbf{F}^{(e)}
	\label{质心角动量定理}
\end{equation}
即,质点系的轨道角动量时间变化率等于外力集中于质心产生的合力矩。根据角动量定理$\dfrac{\mathrm{d} \mbf{L}}{\mathrm{d} t} = \mbf{M}^{(e)}$,以及$\mbf{L} = \mbf{L}_C + \mbf{L}'$,可有
\begin{equation}
	\frac{\mathrm{d} \mbf{L}'}{\mathrm{d} t} = \mbf{M}'^{(e)}
	\label{质心系角动量定理}
\end{equation}
即,质点系的内秉角动量时间变化率等于相对于质心的合外力矩。式\eqref{质心角动量定理}和式\eqref{质心系角动量定理}表示的关系即为{\heiti 质心系角动量定理}。

\subsubsection{质心系动能定理}

定义{\heiti 质点系整体运动动能}为
\begin{equation}
	T_C = \frac12 M v_C^2
\end{equation}
以及{\heiti 质点系内动能}为
\begin{equation}
	T' = \sum_{i=1}^n \frac12 m_i v'_i{}^2
\end{equation}
则有
\begin{align}
	T & = \frac12 \sum_{i=1}^n m_i(\mbf{v}_C + \mbf{v}'_i) \cdot (\mbf{v}_C + \mbf{v}'_i) = \frac12 M v_C^2 + \mbf{v}_C \cdot \mbf{p}' + \frac12\sum_{i=1}^n m_i v'_i{}^2 \nonumber \\
	& = T_C + T' \label{Konig定理}
\end{align}
式\eqref{Konig定理}表示的关系称为{\heiti K\"onig(柯尼希)定理}。即,质点系总动能等于整体运动动能与内动能之和。

根据整体运动动能的定义,有
\begin{equation}
	\frac{\mathrm{d} T_C}{\mathrm{d} t} = M\mbf{v}_C \cdot \mbf{a}_C = \mbf{F}^{(e)} \cdot \mbf{v}_C
\end{equation}
或者
\begin{equation}
	\mathrm{d}T_C = \mbf{F}^{(e)} \cdot \mathrm{d} \mbf{r}_C
	\label{质心动能定理}
\end{equation}
即,质点系整体运动动能增量等于外力集中于质心做的总功。又考虑到相对位矢是Galileo变换的不变量,即有
\begin{equation*}
	\mbf{r}_{ij} = \mbf{r}_i - \mbf{r}_j = \mbf{r}'_i - \mbf{r}'_j = \mbf{r}'_{ij}
\end{equation*}
可得
\begin{align*}
	\mathrm{d} T & = \sum_{i=1}^n \mbf{F}_i \cdot \mathrm{d} \mbf{r}_i + \frac12 \sum_{i,j=1(i\neq j)}^n \mbf{f}_{ij} \cdot \mathrm{d} \mbf{r}_{ij} \\
	& = \sum_{i=1}^n \mbf{F}_i \cdot \mathrm{d} \mbf{r}_C + \sum_i \mbf{F}_i \cdot \mathrm{d} \mbf{r}'_i + \frac12 \sum_{i,j=1(i\neq j)}^n \mbf{f}_{ij} \cdot \mathrm{d} \mbf{r}'_{ij} \\
	& = \mbf{F}^{(e)} \cdot \mathrm{d} \mbf{r}_C + \sum_{i=1}^n \mbf{F}_i \cdot \mathrm{d} \mbf{r}'_i + \frac12 \sum_{i,j=1(i \neq j)}^n \mbf{f}_{ij} \cdot \mathrm{d} \mbf{r}'_{ij}
\end{align*}
所以有
\begin{equation}
	\mathrm{d} T' = \sum_{i=1}^n \mbf{F}_i \cdot \mathrm{d} \mbf{r}'_i + \frac12 \sum_{i,j=1(i \neq j)}^n \mbf{f}_{ij} \cdot \mathrm{d} \mbf{r}'_{ij}
	\label{质心系动能定理}
\end{equation}
式\eqref{质心动能定理}和式\eqref{质心系动能定理}表示的关系称为{\heiti 质心系动能定理}。即,质点系内动能增量等于外力与内力在质心系内做的总功。

如果内力均为保守力,则有
\begin{equation*}
	\frac12 \sum_{i,j=1(i \neq j)}^n \mbf{f}_{ij} \cdot \mathrm{d} \mbf{r}_{ij} = \frac12 \sum_{i,j=1(i \neq j)}^n \mbf{f}_{ij} \cdot \mathrm{d} \mbf{r}'_{ij} = -\mathrm{d} V^{(i)}
	\label{质点系内势能}
\end{equation*}
式中$V^{(i)}$称为{\heiti 质点系内势能}。定义{\heiti 质点系内能}为
\begin{equation}
	E' = T'+V^{(i)}
\end{equation}
由式\eqref{质点系内势能}和式\eqref{质心系动能定理}可得
\begin{equation}
	\mathrm{d} E' = \sum_{i=1}^n \mbf{F}_i \cdot \mathrm{d} \mbf{r}'_i
\end{equation}
即,质点系内能增量等于外力在质心系内做的总功。

\begin{example}
光滑水平面上有一质量为$m$的质点,用一轻绳系着,绳子完全绕在半径为$R$的圆柱上,使质点贴在圆柱体的边缘上。$t=0$时给质点以一径向速度$\mbf{v}_0$,据此求任意时刻绳上的张力大小$T(t)$。
\begin{figure}[htb]
\centering
\begin{asy}
	size(300);
	//第一章例4图
	real R,r,theta,dtheta;
	pair O,C,Cp,P,Pp,v0,vt,vtp;
	O = (0,0);
	R = 2.5;
	theta = 110;
	dtheta = 15;
	dot(O);
	label("$\theta$",O,NE);
	draw(scale(R)*unitcircle,linewidth(0.8bp));
	draw(Label("$R$",MidPoint,W),O--R*dir(90));
	r = 0.3*R;
	draw(arc(O,r,90-theta-dtheta-25,90-theta-dtheta),Arrow);
	draw(Label("$\mathrm{d} \theta$",MidPoint,Relative(W)),arc(O,r,90-theta+25,90-theta),Arrow);
	
	pair jkx(real phi){
		return R*dir(90-phi)+R*phi*pi/180*dir(180-phi);
	}
	pair djkx(real phi){
		real phir;
		phir = phi*pi/180;
		return (sin(phir),cos(phir));
	}
	draw(graph(jkx,0,150),red+linewidth(0.8bp));
	draw(Label("$\boldsymbol{v}_0$",EndPoint),jkx(0)--jkx(0)+djkx(0),Arrow);
	dot(jkx(0));
	draw(Label("$\boldsymbol{v}$",EndPoint,Relative(W)),jkx(theta)--jkx(theta)+djkx(theta),Arrow);
	dot(jkx(theta));
	draw(jkx(theta+dtheta)--jkx(theta+dtheta)+djkx(theta+dtheta),Arrow);
	dot(jkx(theta+dtheta));
	draw(O--R*dir(90-theta));
	draw(Label("$R\theta$",MidPoint,Relative(W)),R*dir(90-theta)--jkx(theta));
	draw(jkx(theta)--jkx(theta)-dir(180-theta),blue,Arrow);
	draw(O--R*dir(90-theta-dtheta)--jkx(theta+dtheta),dashed);
	label("$\mathrm{d} s$",jkx(theta+dtheta/2),S);
	draw(O--(2.95*R,1.6*R),invisible);
\end{asy}
\caption{例\theexample}
\label{第一章例4图}
\end{figure}
\end{example}
\begin{solution}
由题意,小球的轨道为圆的渐开线。即任意时刻绳子与圆相切,绳子的不可伸长性要求质点速度正交于绳子。轨道的参数方程可以表示为
\begin{equation*}
	\mbf{r}(\theta) = R \begin{pmatrix} \cos\left(\dfrac{\pi}{2}-\theta\right) \\[1.5ex] \sin \left(\dfrac{\pi}{2}-\theta\right) \end{pmatrix} + R\theta \begin{pmatrix} \cos(\pi-\theta) \\ \sin(\pi-\theta) \end{pmatrix} = \begin{pmatrix} R\sin \theta-R\theta\cos \theta \\ R\cos \theta+R\theta \sin \theta \end{pmatrix}
\end{equation*}
在切线方向上有
\begin{equation*}
	m\dot{v} = F_\tau = 0
\end{equation*}
即有
\begin{equation*}
	v = v_0
\end{equation*}
在法线方向上有
\begin{equation*}
	m \frac{v^2}{\rho} = T
\end{equation*}
式中$\rho$为该点的曲率半径。根据轨道的方程可得
\begin{equation*}
	\rho = \left|\frac{\left[(x'(\theta))^2+(y'(\theta))^2\right]^{\frac32}}{x'(\theta)y''(\theta)-x''(\theta)y'(\theta)}\right| = R\theta
\end{equation*}
再考虑到
\begin{equation*}
	\frac{\mathrm{d} \theta}{\mathrm{d} t} = \frac{\mathrm{d} \theta}{\mathrm{d} s} \frac{\mathrm{d} s}{\mathrm{d} t} = \frac{v}{\rho} = \frac{v}{R\theta}
\end{equation*}
可得
\begin{equation*}
	\theta(t) = \sqrt{\frac{2v_0t}{R}}
\end{equation*}
因此有
\begin{equation*}
	T = m\frac{v^2}{\rho} = m\frac{v^2}{R\theta} = mv_0 \sqrt{\frac{v_0}{2Rt}}
\end{equation*}
\end{solution}

\begin{example}[计入空气阻力的斜抛运动]
设空气阻力$\mbf{f} = -k\mbf{v}$,重力加速度为$\mbf{g} = -g\mbf{e}_2$,初始速度$\mbf{v}_0 = v_0(\cos \alpha \mbf{e}_1 + \sin \alpha \mbf{e}_2)$,初始位矢$\mbf{r}_0 = \mbf{0}$,求物体的轨迹。
\end{example}

\begin{solution}
根据Newton第二定律,可有矢量形式的微分方程
\begin{equation*}
	m\frac{\mathrm{d} \mbf{v}}{\mathrm{d} t} = m\mbf{g} - k\mbf{v}
\end{equation*}
整理可得
\begin{equation*}
	\frac{\mathrm{d} \mbf{v}}{\mathrm{d} t} + \frac{\mbf{v}}{\tau} = \mbf{g}
\end{equation*}
此处$\tau = \dfrac{m}{k}$。首先解齐次方程
\begin{equation*}
	\frac{\mathrm{d} \mbf{v}}{\mathrm{d} t} + \frac{\mbf{v}}{\tau} = \mbf{0}
\end{equation*}
将其写成分量形式即为
\begin{equation*}
	\begin{cases}
		\dfrac{\mathrm{d} v_x}{\mathrm{d} t} = -\dfrac{v_x}{\tau} \\[1.5ex]
		\dfrac{\mathrm{d} v_y}{\mathrm{d} t} = -\dfrac{v_y}{\tau}
	\end{cases}
\end{equation*}
其解为
\begin{equation*}
	\begin{cases}
		v_x = C_1 \mathrm{e}^{-\frac{t}{\tau}} \\
		v_y = C_2 \mathrm{e}^{-\frac{t}{\tau}}
	\end{cases}
\end{equation*}
或者写作矢量形式为
\begin{equation*}
	\mbf{v} = \mbf{C} \mathrm{e}^{-\frac{t}{\tau}}
\end{equation*}
利用常数变易法,令$\mbf{v}(t) = \mbf{C}(t) \mathrm{e}^{-\frac{t}{\tau}}$,代入原方程,可得
\begin{equation*}
	\frac{\mathrm{d} \mbf{C}}{\mathrm{d} t} = \mbf{g}\mathrm{e}^{\frac{t}{\tau}}
\end{equation*}
即有$\mbf{C}(t) = \mbf{g}\tau \mathrm{e}^{\frac{t}{\tau}} + \mbf{C}_1$,因此
\begin{equation*}
	\mbf{v}(t) = \mbf{C}_1 \mathrm{e}^{-\frac{t}{\tau}} + \mbf{g} \tau
\end{equation*}
利用$\mbf{v}(0) = \mbf{v}_0$,可得
\begin{equation*}
	\mbf{v}(t) = \mbf{v}_0 \mathrm{e}^{-\frac{t}{\tau}} + \mbf{g}\tau \left(1-\mathrm{e}^{-\frac{t}{\tau}}\right)
\end{equation*}
再对时间积分一次,可得轨迹
\begin{equation*}
	\mbf{r}(t) = -\mbf{v}_0 \tau \mathrm{e}^{-\frac{t}{\tau}} + \mbf{g}\tau(t + \tau \mathrm{e}^{-\frac{t}{\tau}}) + \mbf{C}_2
\end{equation*}
利用$\mbf{r}(0) = \mbf{0}$,可得
\begin{equation*}
	\mbf{r}(t) = \mbf{v}_0 \tau \left(1-\mathrm{e}^{-\frac{t}{\tau}}\right) +\mbf{g}\tau \left[t - \tau\left(1 - \mathrm{e}^{-\frac{t}{\tau}}\right)\right]
\end{equation*}
即
\begin{equation*}
	\begin{cases}
		x(t) = v_0 \tau \cos \alpha \left(1-\mathrm{e}^{-\frac{t}{\tau}}\right) \\ 
		y(t) = -g\tau t + \tau(v_0\sin \alpha+g\tau) \left(1-\mathrm{e}^{-\frac{t}{\tau}}\right)
	\end{cases}
\end{equation*}
消去时间可以得到轨道方程
\begin{equation*}
	y = \left(\tau \alpha + \frac{g\tau}{v_0\cos\alpha}\right) x g\tau^2 \ln \left(1-\frac{x}{v_0\tau \cos \alpha}\right)
\end{equation*}

\begin{figure}[htb]
\centering
\begin{asy}
	size(350);
	//不同阻尼系数下的轨道曲线图
	real x,y;
	picture g,tmp;
	real kappa;
	real tra(real x){
		return 2*(1+1/kappa)*x+2/kappa/kappa*log(1-kappa*x);
	}
	real tra0(real x){
		return 2*x-x**2;
	}
	for(y=-2.5;y<1.5;y=y+0.5){
		draw(g,(0,y)--(3,y),0.6white);
	}
	for(x=0.25;x<3;x=x+0.25){
		draw(g,(x,-3)--(x,1.5),0.6white);
	}
	kappa = 0.1;
	draw(tmp,graph(tra,0,0.9/kappa),blue+linewidth(1bp));
	kappa = 0.5;
	draw(tmp,graph(tra,0,0.9/kappa),green+linewidth(1bp));
	kappa = 1.0;
	draw(tmp,graph(tra,0,0.9678),orange+linewidth(1bp));
	kappa = 2.0;
	draw(tmp,graph(tra,0,0.499938),red+linewidth(1bp));
	draw(tmp,graph(tra0,0,3),linewidth(1bp));
	clip(tmp,box((0,-3),(3,1.5)));
	add(g,tmp);
	erase(tmp);
	draw(g,box((0,-3),(3,1.5)),linewidth(1bp));
	label(g,"$\dfrac{x}{x_m}$",(3,-3),2.5*E);
	label(g,"$\dfrac{y}{y_m}$",(0,1.5),W);
	label(g,"$1$",(0,1),W);
	label(g,"$0$",(0,0),W);
	label(g,"$-1$",(0,-1),W);
	label(g,"$-2$",(0,-2),W);
	label(g,"$-3$",(0,-3),W);
	label(g,"$0.0$",(0,-3),S);
	label(g,"$0.5$",(0.5,-3),S);
	label(g,"$1.0$",(1,-3),S);
	label(g,"$1.5$",(1.5,-3),S);
	label(g,"$2.0$",(2,-3),S);
	label(g,"$2.5$",(2.5,-3),S);
	label(g,"$3.0$",(3,-3),S);
	pair A,B;
	A = (2.35,-0.3);
	B = (2.85,1.3);
	real delta;
	delta = (B.y-A.y)/10;
	fill(g,shift((0.025,-0.05))*box(A,B),black);
	unfill(g,box(A,B));
	draw(g,box(A,B));
	label(g,"$\kappa = 0.0$",((A.x+B.x)/2,A.y)+(0,9*delta));
	label(g,"$\kappa = 0.1$",((A.x+B.x)/2,A.y)+(0,7*delta),blue);
	label(g,"$\kappa = 0.5$",((A.x+B.x)/2,A.y)+(0,5*delta),green);
	label(g,"$\kappa = 1.0$",((A.x+B.x)/2,A.y)+(0,3*delta),orange);
	label(g,"$\kappa = 2.0$",((A.x+B.x)/2,A.y)+(0,1*delta),red);
	add(xscale(2)*g);
\end{asy}
\caption{不同阻尼系数下的轨道曲线图}
\label{不同阻尼系数下的轨道曲线图}
\end{figure}

考虑无阻尼极限$k=0$,即$\tau = \dfrac{m}{k} \to +\infty$的情况。根据当$u \to 0$时,有
\begin{equation*}
	\ln (1+u) = u - \frac12 u^2 + o(u^2)
\end{equation*}
当$\tau \to +\infty$时,轨道变为
\begin{equation*}
	y = x\tan \alpha -\frac{gx^2}{2v_0^2\cos^2 \alpha}
\end{equation*}
由此,令
\begin{equation*}
	\kappa = \frac{kv_0 \sin \alpha}{mg} = \frac{v_0 \sin \alpha}{g\tau},\quad x_m = \frac{v_0^2 \sin 2\alpha}{2g},\quad y_m = \frac{v_0^2 \sin^2 \alpha}{2g}
\end{equation*}
则在无阻尼的情形,即$\kappa=0$时,轨道可以表示为
\begin{equation*}
	\frac{y}{y_m} = 2\frac{x}{x_m} - \left(\frac{x}{x_m}\right)^2
\end{equation*}
当$\kappa \neq 0$时,轨道方程为
\begin{equation*}
	\frac{y}{y_m} = 2\left(1+\frac{1}{\kappa}\right) \frac{x}{x_m} + \frac{2}{\kappa^2} \ln \left(1-\kappa \frac{x}{x_m}\right)
\end{equation*}
上式这种形式称为{\heiti 无量纲形式}。不同阻尼系数$\kappa$下的轨道曲线如图\ref{不同阻尼系数下的轨道曲线图}所示。由图可见阻力越大,越早的接近竖直运动。
\end{solution}

\begin{example}
\begin{figure}[htb]
\centering
\begin{asy}
	size(200);
	//第一章例6图
	pair O,P;
	real r,theta;
	O = (0,0);
	r = 2;
	theta = 40;
	draw(scale(r)*unitcircle);
	draw(Label("$R$",MidPoint,N),O--r*dir(0),dashed);
	P = r*dir(-theta);
	draw(O--P,dashed);
	draw(Label("$\theta$",MidPoint,Relative(W)),arc(O,0.3*r,0,-theta),Arrow);
	dot(P);
	draw(Label("$\boldsymbol{G}$",EndPoint),P--P+dir(-90),Arrow);
	draw(Label("$\boldsymbol{N}$",EndPoint,SW),P--P+dir(180-theta),Arrow);
	draw(Label("$\boldsymbol{f}$",EndPoint,N),P--P+dir(90-theta),Arrow);
	dot((0,r+0.01),invisible);
\end{asy}
\caption{例\theexample}
\label{第一章例6图}
\end{figure}

设有竖直圆环轨道上串有钢珠,将钢珠置于$\theta = 0$处释放,已知$v\big|_{\theta=0} = v\big|_{\theta=\frac{\pi}{2}} = 0$,求轨道与钢珠之间的摩擦系数$\mu$。
\end{example}
\begin{solution}
速度和加速度为
\begin{equation*}
	v = R\dot{\theta},\quad a_\tau = R\ddot{\theta},\quad a_n = R\dot{\theta}^2
\end{equation*}
根据Newton第二定律以及摩擦力的关系,可有方程组
\begin{equation*}
	\begin{cases}
		mR\ddot{\theta} = mg\cos \theta - f \\
		mR\dot{\theta}^2 = N-mg\sin \theta \\
		f = \mu N
	\end{cases}
\end{equation*}
从中消去$N$,可得
\begin{equation*}
	R(\ddot{\theta} + \mu\dot{\theta}^2) = g(\cos \theta - \mu\sin \theta)
\end{equation*}
再考虑到
\begin{equation*}
	\ddot{\theta} = \frac{\mathrm{d} \dot{\theta}}{\mathrm{d} t} = \frac{\mathrm{d} \dot{\theta}}{\mathrm{d} \theta} \frac{\mathrm{d} \theta}{\mathrm{d} t} = \dot{\theta} \frac{\mathrm{d} \dot{\theta}}{\mathrm{d} \theta} = \frac12 \frac{\mathrm{d} \dot{\theta}^2}{\mathrm{d} \theta}
\end{equation*}
可有
\begin{equation*}
	\frac{\mathrm{d} \dot{\theta}^2}{\mathrm{d} \theta} + 2\mu\dot{\theta}^2 = \frac{2g}{R}(\cos \theta - \mu\sin \theta)
\end{equation*}
可用常数变易法解此微分方程,得
\begin{equation*}
	\dot{\theta}^2 = \frac{2g}{(1+4\mu^2)R} \left[(1-2\mu^2)\sin \theta + 3\mu \cos \theta\right] + C\mathrm{e}^{-2\mu\theta}
\end{equation*}
再考虑到$\dot{\theta}\big|_{\theta=0} = 0$,可得
\begin{equation*}
	\dot{\theta}^2 = \frac{2g}{(1+4\mu^2)R} \left[(1-2\mu^2)\sin \theta + 3\mu (\cos \theta-\mathrm{e}^{-2\mu\theta})\right]
\end{equation*}
最后利用$\dot{\theta}\big|_{\theta=\frac{\pi}{2}} = 0$,可得$\mu$满足的方程为
\begin{equation*}
	1-2\mu^2-3\mu\mathrm{e}^{-\mu\pi} = 0
\end{equation*}
整理为
\begin{equation*}
	\mu = \sqrt{\frac{1-3\mu\mathrm{e}^{-\mu\pi}}{2}} =: f(\mu)
\end{equation*}
即$\mu$为函数$f(\mu)$的不动点。求此不动点可用压缩映射定理证明中的迭代法,即$\mu^{(n+1)} = f(\mu^{(n)})$来求解。具体如下
\begin{align*}
	& \mu^{(0)} = 0.000 \\
	& \mu^{(1)} = f(\mu^{(0)}) = 0.707\cdots \\
	& \mu^{(2)} = f(\mu^{(1)}) = 0.620\cdots \\
	& \mu^{(3)} = f(\mu^{(2)}) = 0.606\cdots \\
	& \quad \quad \vdots \\
	& \mu \doteq 0.603
\end{align*}
\begin{figure}[htb]
\centering
\begin{asy}
	size(300);
	//第一章例6求解示意
	pair O;
	real x,y,xm,ym;
	O = (0,0);
	xm = 0.8;
	ym = 0.8;
	//画坐标系
	draw(Label("$\mu$",EndPoint),O--(xm,0));
	draw(Label("$f(\mu)$",EndPoint),O--(0,ym));
	label("$0.0$",O,S);
	label("$0.0$",O,W);
	for(x = 0.1;x < xm-0.1;x = x+0.1){
		draw((x,0)--(x,0.01));
	}
	label("$0.5$",(0.5,0),S);
	for(y = 0.1;y < ym-0.1;y = y+0.1){
		draw((0,y)--(0.01,y));
	}
	label("$0.5$",(0,0.5),W);
	
	real lin(real mu){
		return mu;
	}
	real fmu(real mu){
		return sqrt((1-3*mu*exp(-mu*pi))/2);
	}
	draw(graph(lin,0,xm),linewidth(0.8bp));
	draw(Label("$f(\mu)$",EndPoint),graph(fmu,0,xm),red+linewidth(0.8bp));
	draw(O--(0,fmu(0))--(fmu(0),fmu(0))--(fmu(0),fmu(fmu(0)))--(fmu(fmu(0)),fmu(fmu(0)))--(fmu(fmu(0)),fmu(fmu(fmu(0))))--(fmu(fmu(fmu(0))),fmu(fmu(fmu(0)))),blue+linewidth(0.8bp));
	label("$0.707\cdots$",(fmu(0),fmu(0)),E);
	label("$0.620\cdots$",(fmu(fmu(0)),fmu(fmu(0))),NW);
	label("$0.606\cdots$",(fmu(fmu(fmu(0))),fmu(fmu(fmu(0)))),W);
	real m;
	m = 0;
	for(int i=1;i<10;++i){
		m = fmu(m);
	}
	dot((m,m));
	label("$0.603\cdots$",(m,m),SE);
	//draw((0,0)--(xm+0.2,0),invisible);
\end{asy}
\caption{求解示意}
\label{求解示意}
\end{figure}
\end{solution}

\begin{example}
在铅直平面内有一光滑的半圆形管道(如图\ref{第一章例7图}所示),半径为$R$,管道内有一长为$\pi R$,质量线密度为$\rho$的均匀链条。假定链条由于轻微扰动而从管口向外滑出,试用角动量定理求出链条位于任意角度$\theta$时的速度$v$。
\begin{figure}[htb]
\centering
\begin{asy}
	size(200);
	//第一章例7图
	pair O;
	real R,r,rr,theta,alpha,dalpha;
	O = (0,0);
	R = 1;
	r = 1.05;
	rr = 0.3;
	theta = 30;
	alpha = 40;
	dalpha = 15;
	draw(arc(O,R,0,180)--cycle);
	draw(arc(O,r,0,180)--cycle);
	draw(Label("$R$",MidPoint),O--r*dir(180-theta));
	draw(Label("$\theta$",MidPoint,Relative(E)),arc(O,rr*R,180-theta,180),Arrow);
	draw(O--r*dir(alpha));
	draw(Label("$\alpha$",MidPoint,Relative(E)),arc(O,rr*R,0,alpha),Arrow);
	draw(O--r*dir(alpha+dalpha));
	rr = 0.6;
	draw(Label("$\mathrm{d} \alpha$",MidPoint,Relative(E)),arc(O,rr*R,alpha,alpha+dalpha),Arrow);
	draw(arc(O,(r+R)/2,180-theta-1,0)--(r+R)/2*dir(0)+(0,-(R+r)/2*theta*pi/180),red+linewidth(4bp));
	draw(Label("$\boldsymbol{G}_1$",EndPoint),(r+R)/2*dir(0)+(0,-(R+r)/2*theta*pi/180/2)--(r+R)/2*dir(0)+(0,-(R+r)/2*theta*pi/180/2-0.5),Arrow);
	pair P;
	P = (R+r)/2*dir(alpha+dalpha/2);
	draw(Label("$\mathrm{d} \boldsymbol{G}$",EndPoint),P--P+(0,-0.5),Arrow);
	draw(Label("$\boldsymbol{v}$",EndPoint),P--P+0.5*dir(alpha+dalpha/2-90),Arrow);
	draw(Label("$\mathrm{d} \boldsymbol{N}$",EndPoint),P--P+0.5*P,Arrow);
\end{asy}
\caption{例\theexample}
\label{第一章例7图}
\end{figure}
\end{example}
\begin{solution}
由于链条不可伸长,因此其个点速度相等,即
\begin{equation*}
	v = R\dot{\theta}
\end{equation*}
由此可得链条的角动量
\begin{equation*}
	\mbf{L} = \int_C \mbf{r} \times \mbf{v} \mathrm{d} l = \rho \pi R (R\dot{\theta} \mbf{e}) = \rho \pi R^3 \dot{\theta} \mbf{e}
\end{equation*}
式中$\mbf{e}$为垂直纸面向里的单位矢量。

由于$\mbf{r} \times \mathrm{d} \mbf{N} = \mbf{0}$,故只需计算重力力矩。对于直线段,可有
\begin{equation*}
	\mbf{M}_1 = \mbf{r} \times \mbf{G}_1 = (\rho R\theta) Rg\mbf{e} = \rho gR^2 \theta \mbf{e}
\end{equation*}
对于圆弧段,可有
\begin{equation*}
	\mbf{M}_2 = \int_C \mbf{r} \times \rho\mbf{g} \mathrm{d} l = \int_0^{\pi-\theta} Rg\cos \alpha \rho R \mathrm{d} \alpha \mbf{e} = \rho gR^2 \sin \theta \mbf{e}
\end{equation*}
由此可得总力矩为
\begin{equation*}
	\mbf{M} = \mbf{M}_1 + \mbf{M}_2 = \rho gR^2 (\theta+\sin \theta)\mbf{e}
\end{equation*}
根据角动量定理$\dfrac{\mathrm{d} \mbf{L}}{\mathrm{d} t} = \mbf{M}$,可得
\begin{equation*}
	\rho \pi R^3 \ddot{\theta} = \rho gR^2 (\theta+\sin \theta)
\end{equation*}
根据
\begin{equation*}
	\ddot{\theta} = \frac{\mathrm{d} \dot{\theta}}{\mathrm{d} t} = \frac{\mathrm{d} \dot{\theta}}{\mathrm{d} \theta} \frac{\mathrm{d} \theta}{\mathrm{d} t} = \dot{\theta} \frac{\mathrm{d} \dot{\theta}}{\mathrm{d} \theta} = \frac12 \frac{\mathrm{d} \dot{\theta}^2}{\mathrm{d} \theta}
\end{equation*}
可得
\begin{equation*}
	\frac{\mathrm{d} \dot{\theta}^2}{\mathrm{d} \theta} = \frac{2g}{\pi R}(\theta +\sin \theta)
\end{equation*}
积分即可得到
\begin{equation*}
	\dot{\theta} = \sqrt{\frac{2g}{\pi R} \left(\frac12 \theta^2 - \cos \theta + 1\right)} = 2\sqrt{\frac{g}{\pi R}\left[\left(\frac{\theta}{2}\right)^2+ \sin^2 \frac{\theta}{2}\right]}
\end{equation*}
所以速度为
\begin{equation*}
	v = R\dot{\theta} = 2\sqrt{\frac{Rg}{\pi}\left[\left(\frac{\theta}{2}\right)^2+ \sin^2 \frac{\theta}{2}\right]}
\end{equation*}

本体也可使用机械能守恒来求解。此系统中,非保守力(弹力)不做功,因此系统机械能守恒。在初始状态下,链条的动能
\begin{equation*}
	T = 0
\end{equation*}
以半圆柱的底面为零势能面,可求得初始状态下链条的重力势能为
\begin{equation*}
	V = \int_C \rho gh \mathrm{d} l = \int_0^\pi \rho g R\sin \alpha R\mathrm{d} \alpha = 2\rho gR^2
\end{equation*}
在图\ref{第一章例7图}所示的状态下,由于链条上各处速度大小相等,故链条的动能为
\begin{equation*}
	T = \frac12 mv^2 = \frac12 \rho \pi R v^2
\end{equation*}
直线段的重力势能为
\begin{equation*}
	V_1 = \rho R\theta g \cdot \left(-\frac12 R\theta\right) = -\frac12 \rho gR^2 \theta^2 
\end{equation*}
曲线段的重力势能为
\begin{equation*}
	V_2 = \int_C \rho gh \mathrm{d} l = \int_0^{\pi-\theta} \rho g R\sin \alpha R\mathrm{d} \alpha = \rho gR^2 (\cos \theta + 1)
\end{equation*}
根据机械能守恒可得
\begin{equation*}
	2\rho gR^2 = \frac12 \rho \pi R v^2 -\frac12 \rho gR^2 \theta^2 + \rho gR^2 (\cos \theta + 1)
\end{equation*}
由此同样可以解得链条处于任意角度处的速度为
\begin{equation*}
	v = 2\sqrt{\frac{Rg}{\pi}\left[\left(\frac{\theta}{2}\right)^2+ \sin^2 \frac{\theta}{2}\right]}
\end{equation*}
\end{solution}

\begin{example}
质量为$m$的质点沿一半球形光滑碗的内侧以初速度$v_0$沿水平方向运动,碗的内半径为$r$,初位置离碗边缘的高度为$h$。求质点上升到碗口而又不至于飞出碗外时$v_0$和$h$应满足的关系式。
\begin{figure}[htb]
\centering
\begin{asy}
	size(200);
	//第一章例8图
	real hd,hdd;
	pair O,x,y,z;
	hd = 131+25/60;
	hdd = 7+10/60;
	O = (0,0);
	x = (-sqrt(2)/4,-sqrt(14)/12);
	y = (sqrt(14)/4,-sqrt(2)/12);
	z = (0,2*sqrt(2)/3);
	draw(Label("$x$",EndPoint,W),(-1.5*x)--(1.5*x),Arrow);
	draw(Label("$y$",EndPoint),(-1.2*y)--(1.2*y),Arrow);
	draw(Label("$z$",EndPoint),(-1.2*z)--(0.5*z),Arrow);
	real r,vec;
	r = 1;
	pair cirxy(real theta){
		real xx,yy;
		xx = r*cos(theta);
		yy = r*sin(theta);
		return xx*x+yy*y;
	}
	pair ciryz(real theta){
		real yy,zz;
		yy = r*cos(theta);
		zz = r*sin(theta);
		return yy*y+zz*z;
	}
	pair cirxz(real theta){
		real xx,zz;
		xx = r*cos(theta);
		zz = r*sin(theta);
		return xx*x+zz*z;
	}
	real phi,phi2;
	pair Pcir(real theta){
		real xx,yy,zz;
		xx = r*cos(theta)*cos(phi);
		yy = r*cos(theta)*sin(phi);
		zz = r*sin(theta);
		return xx*x+yy*y+zz*z;
	}
	pair Bcir(real theta){
		real xx,yy,zz;
		xx = r*cos(phi2)*cos(theta);
		yy = r*cos(phi2)*sin(theta);
		zz = r*sin(phi2);
		return xx*x+yy*y+zz*z;
	}
	//画曲线
	draw(arc(O,r,180,360));
	draw(graph(cirxy,0,2*pi));
	draw(graph(ciryz,pi,1.72*pi),dashed);
	draw(graph(ciryz,1.72*pi,2*pi));
	draw(graph(cirxz,pi,1.62*pi),dashed);
	draw(graph(cirxz,1.62*pi,2*pi));
	phi = pi/4;
	draw(graph(Pcir,pi,1.62*pi),dashed);
	draw(graph(Pcir,1.62*pi,2*pi));
	draw(Pcir(pi)--Pcir(2*pi));
	phi2 = 2*pi-pi/4;
	pair P;
	P = Pcir(phi2);
	draw(graph(Bcir,-0.3*pi,0.5*pi));
	draw(graph(Bcir,0.5*pi,2*pi-0.3*pi),dashed);
	vec = 0.5;
	draw(Label("$\boldsymbol{G}$",EndPoint),P--(P-vec*z),red,Arrow);
	draw(Label("$\boldsymbol{N}$",EndPoint),P--(P-vec*Pcir(phi2)),red,Arrow);
	draw(Label("$\boldsymbol{v}_\phi$",EndPoint,E,black),P--(P-vec*sin(phi)*x+vec*cos(phi)*y),orange,Arrow);
	draw(Label("$\boldsymbol{v}_\theta$",EndPoint,S,black),P--(P+vec*sin(phi2)*cos(phi)*x+vec*sin(phi2)*cos(phi)*y-vec*cos(phi2)*z),orange,Arrow);
	dot(P);
\end{asy}
\caption{例\theexample}
\label{第一章例8图}
\end{figure}
\end{example}

\begin{solution}
建立球坐标系,考虑到$v_r = \dot{r} = 0$,则
\begin{equation*}
	\mbf{v} = v_\theta \mbf{e}_\theta + v_\phi \mbf{e}_\phi
\end{equation*}
弹力沿径向,故$\mbf{N} \cdot \mbf{v} = 0$,即系统中只有重力做功,机械能守恒。

而$\mbf{r} \times \mbf{N} = \mbf{0}$,$\mbf{e}_3 \cdot \left[\mbf{r} \times (-mg\mbf{e}_3)\right] = 0$,故$z$方向力矩为零,即$z$方向角动量守恒。计算$z$方向的角动量
\begin{equation*}
	L_z = \mbf{e}_3 \cdot \left[r\mbf{e}_r \times m(v_\theta\mbf{e}_\theta + v_\phi \mbf{e}_\phi)\right] = mr\mbf{e}_3 \cdot (v_\theta\mbf{e}_\phi - v_\phi\mbf{e}_\theta) = mr\sin \theta v_\phi
\end{equation*}
考虑到$\mbf{v}\big|_{z=-h} = v_0 \mbf{e}_\phi$,可以列出守恒方程
\begin{equation*}
	\begin{cases}
		\dfrac12 m(v_\theta^2 + v_\phi^2) + mgz = \dfrac12 mv_0^2 - mgh \\[1.5ex]
		mv_\phi \sqrt{r^2-z^2} = mv_0\sqrt{r^2-h^2}
	\end{cases}
\end{equation*}
将末状态取为$z=0$时,此时$v_\theta\big|_{z=0} = 0$,可有
\begin{equation*}
	\begin{cases}
		\dfrac12 mv_\phi^2\big|_{z=0} = \dfrac12 mv_0^2 - mgh \\
		mrv_\phi\big|_{z=0} = mv_0 \sqrt{r^2-h^2}
	\end{cases}
\end{equation*}
由此,可解出
\begin{equation*}
	v_0 = r\sqrt{\frac{2g}{h}}
\end{equation*}
即为$v_0$和$h$应满足的关系。
\end{solution}

\section{变质量系统}

\subsection{变质量系统的概念}

如果系统的质量或者组成系统的质点其中之一,或者两者同时随时间变化,这个系统就称为{\bf 变质量系统}。

在自然现象中,变质量系统很常见。例如浮冰由于融化而减少质量,又由于结冰或者落雪而增加质量。在工程技术中,最常提到的变质量系统就是火箭了。

对于变质量系统的研究,总是做这样的假设:系统分离或并入的质量是小量,而且每次并入或分离质量的时间间隔是小量。在这样的假设下,离开系统和进入系统的质量都是时间的连续可微的函数。

如果将系统的质量记作$M(t)$,将$t=0$时的质量记作$M_0$,则它随时间的变化规律为
\begin{equation}
	M(t) = M_0 - M_1(t) + M_2(t)
	\label{chapter1:变质量系统的质量关系}
\end{equation}
其中$M_1(t)$表示累计离开系统的质量,而$M_2(t)$表示累计进入系统的质量,$M_1,M_2$都是随时间的非递减的非负连续可微函数。

\subsection{变质量系统的动量定理和角动量定理}

设某个质点系相对惯性参考系$Oxyz$运动,研究相对$Oxyz$运动和变形的封闭曲面$S$。系统内的各个质点按照各自的运动可以进入或者离开曲面$S$围成的空间区域,那么曲面$S$围成的质点即组成一个变质量系统,将其动量用$\mbf{p}$表示。

设在时刻$t=t'$,系统的动量为$\mbf{p}$。而在$t=t''=t'+\Delta t$的时刻,系统的动量记作$\mbf{p}+\Delta \mbf{p}$。显然有
\begin{equation}
	\Delta \mbf{p} = \Delta \mbf{p}^*-\Delta \mbf{p}_1+\Delta \mbf{p}_2
	\label{chapter1:变质量系统的动量定理-初步}
\end{equation}
其中,$\Delta\mbf{p}_1$为在$\Delta t$时间内离开变质量系统的所有质点带走的动量,$\Delta\mbf{p}_2$为在这段时间内进入系统的所有质点带来的动量,而$\Delta\mbf{p}^*$则表示在这段时间内一直在系统内的所有质点本身的动量改变。

% \begin{figure}[htb]
% \centering
% \begin{asy}
	% size(300);
	% //变质量系统1
	% picture pic1;
	% pair O,F,v,u,m;
	% real rO,rm,lF,lv,lu;
	% O = (0,0);
	% rO = 2;
	% fill(pic1,yscale(2/3)*scale(rO)*unitcircle,yellow);
	% label(pic1,"$m$",O);
	% v = (1,0.7);
	% lv = 1.2;
	% draw(pic1,Label("$\boldsymbol{v}$",MidPoint,Relative(W)),v--v+lv*dir(30),Arrow);
	% F = (-1.3,-0.8);
	% lF = 1.2;
	% draw(pic1,Label("$\boldsymbol{F}$",MidPoint,Relative(W)),F-lF*dir(30)--F,Arrow);
	% rm = 0.7;
	% m = (3,0.4);
	% fill(pic1,shift(m)*yscale(2/3)*scale(rm)*unitcircle,green);
	% label(pic1,"$\Delta m$",m);
	% u = (-0.1,0.2)+m;
	% lu = 0.9;
	% draw(pic1,Label("$\boldsymbol{u}$",MidPoint,Relative(E)),u--u+lu*dir(135),Arrow);
	% label(pic1,"$t$",(0,-1.5),S);
	% add(pic1);
	% //变质量系统2
	% picture pic2;
	% pair O,F,v,u,m;
	% real rO,rm,lF,lv,lu;
	% O = (0,0);
	% rO = 2;
	% fill(pic2,yscale(2/3)*scale(rO)*unitcircle,yellow);
	% label(pic2,"$m'=m+\Delta m$",O);
	% v = (1,0.7);
	% lv = 1.2;
	% draw(pic2,Label("$\boldsymbol{v}'$",MidPoint,Relative(W)),v--v+lv*dir(30),Arrow);
	% F = (-1.3,-0.8);
	% lF = 1.2;
	% draw(pic2,Label("$\boldsymbol{F}'$",MidPoint,Relative(W)),F-lF*dir(30)--F,Arrow);
	% label(pic2,"$t'=t+\Delta t$",(0,-1.5),S);
	% add(shift((6.5,0))*pic2);
	% //draw(O--(9,0),invisible);
% \end{asy}
% \caption{变质量系统}
% \label{变质量系统}
% \end{figure}

设$\mbf{F}^{(e)}$是$t'$时刻作用在系统上的合外力,那么根据动量定理可得
\begin{equation}
	\dif{\mbf{p}^*}{t} = \mbf{F}^{(e)}
\end{equation}
在式\eqref{chapter1:变质量系统的动量定理-初步}两端除以$\Delta t$并取$\Delta t\to 0$的极限可得
\begin{equation}
	\dif{\mbf{p}}{t} = \mbf{F}^{(e)}+\mbf{F}^{(\text{ex})}
\end{equation}
其中$\mbf{F}^{(\text{ex})} = \mbf{F}_1+\mbf{F}_2$,而
\begin{equation}
	\mbf{F}_1 = -\lim_{\Delta t\to 0} \frac{\Delta \mbf{p}_1}{\Delta t},\quad \mbf{F}_2 = \lim_{\Delta t\to 0} \frac{\Delta \mbf{p}_2}{\Delta t}
\end{equation}
矢量$\mbf{F}_1$和$\mbf{F}_2$具有力的量纲,称为{\bf 反推力}。反推力$\mbf{F}_1$是由质点分离引起的,反推力$\mbf{F}_2$是由质点并入引起的。

设惯性参考系$Oxyz$中的固定参考点为$A$,则类似前面的讨论可得变质量系统的角动量定理
\begin{equation}
	\dif{\mbf{L}_A}{t} = \mbf{M}_A^{(e)} + \mbf{M}_A^{(\text{ex})}
\end{equation}
其中$\mbf{L}_A$为系统对$A$点的角动量,$\mbf{M}_A^{(e)}$为系统受到的对$A$点的合外力矩,而$\mbf{M}_A^{(\text{ex})} = \mbf{M}_{A1} + \mbf{M}_{A2}$是变质量系统的附加力矩,其中
\begin{equation}
	\mbf{M}_{A1} = -\lim_{\Delta t\to 0} \frac{\Delta \mbf{L}_{A1}}{\Delta t},\quad \mbf{M}_{A2} =  \lim_{\Delta t\to 0} \frac{\Delta \mbf{L}_{A2}}{\Delta t}
\end{equation}
这里的$\Delta \mbf{L}_{A1}$为在$\Delta t$时间内离开变质量系统的所有质点的角动量,$\Delta \mbf{L}_{A2}$为在这段时间内进入系统的所有质点的角动量。

% \begin{equation}
	% \dif{\mbf{p}_1}{t} = \lim_{\Delta t\to 0} \frac{\Delta \mbf{p}_1}{\Delta t},\quad \dif{\mbf{p}_2}{t} = \lim_{\Delta t\to 0} \frac{\Delta \mbf{p}_2}{\Delta t}
% \end{equation}

\subsection{变质量质点的运动微分方程}

作为变质量系统的一种特殊情况,现在来考虑变质量质点的运动微分方程。其质量变化同样满足式\eqref{chapter1:变质量系统的质量关系},即
\begin{equation}
	M(t) = M_0 - M_1(t) + M_2(t)
	\label{chapter1:变质量系统的质量关系-重写}
\end{equation}
如果记$\mbf{u}_1$为$t'$时刻离开变质量质点的微粒的速度,$\mbf{u}_2$为$t'$时刻并入质点的微粒的速度,那么精确到$\Delta t$和$\Delta M$的一阶小量,可有
\begin{equation}
	\Delta \mbf{p}_1 = \Delta M_1\mbf{u}_1,\quad \Delta \mbf{p}_2 = \Delta M_2\mbf{u}_2
\end{equation}	
由此可得
\begin{equation}
	\mbf{F}_1 = -\lim_{\Delta t\to 0} \frac{\Delta \mbf{p}_1}{\Delta t} = -\dif{M_1}{t}\mbf{u}_1,\quad \mbf{F}_2 = \lim_{\Delta t\to 0} \frac{\Delta \mbf{p}_2}{\Delta t} = \dif{M_2}{t}\mbf{u}_2
\end{equation}
设$\mbf{v}$为变质量质点的速度,则其动量为
\begin{equation}
	\mbf{p} = M\mbf{v}
\end{equation}
则有
\begin{equation}
	\dif{M}{t}\mbf{v}+M\dif{\mbf{v}}{t} = \mbf{F}^{(e)} - \dif{M_1}{t}\mbf{u}_1 + \dif{M_2}{t}\mbf{u}_2
	\label{chapter1:广义密歇尔斯基方程-初步}
\end{equation}
将式\eqref{chapter1:变质量系统的质量关系-重写}代入方程\eqref{chapter1:广义密歇尔斯基方程-初步}中可得
\begin{equation}
	M\dif{\mbf{v}}{t} = \mbf{F}^{(e)} - \dif{M_1}{t}(\mbf{u}_1-\mbf{v}) + \dif{M_2}{t}(\mbf{u}_2-\mbf{v})
	\label{chapter1:广义密歇尔斯基方程}
\end{equation}
方程\eqref{chapter1:广义密歇尔斯基方程}即为变质量质点的运动微分方程,称为{\bf 广义Meshcherskii方程}\footnote{Meshcherskii,密歇尔斯基,前苏联力学家。}。在方程\eqref{chapter1:广义密歇尔斯基方程}中,$\mbf{u}_1-\mbf{v}=\mbf{u}_{1r}$是分离微粒相对质点的速度,而$\mbf{u}_2-\mbf{v}=\mbf{u}_{2r}$是并入微粒相对质点的速度。

如果只有分离微粒而没有并入微粒,此时$M_2(t)\equiv 0$,此时方程\eqref{chapter1:广义密歇尔斯基方程}变为
\begin{equation}
	M\dif{\mbf{v}}{t} = \mbf{F}^{(e)} + \dif{M}{t}\mbf{u}_{1r}
	\label{chapter1:密歇尔斯基方程}
\end{equation}
方程\eqref{chapter1:密歇尔斯基方程}称为{\bf Meshcherskii方程}。由Meshcherskii方程\eqref{chapter1:密歇尔斯基方程}可以看出,微粒分离等效于在质点上作用附加的反推力$\mbf{F}_1 = \dif{M}{t}\mbf{u}_{1r}$。

\begin{example}[重力场中的火箭]
在假设火箭竖直运动且喷气速度恒定,其所受外力只有恒定重力$\mbf{F}^{(e)} = -Mg\mbf{e}_3$的情况下求解火箭的运动。
\end{example}
\begin{solution}
相对火箭的喷气速度
\begin{equation*}
	\mbf{u}-\mbf{v} = -u_r\mbf{e}_3
\end{equation*}
此时火箭的运动方程即为Meshcherskii方程\eqref{chapter1:密歇尔斯基方程}
\begin{equation*}
	M\frac{\mathrm{d} v}{\mathrm{d} t} = -Mg - \dif{M}{t}u_r
\end{equation*}
化简为
\begin{equation*}
	\mathrm{d} v = -g\mathd t - u_r \frac{\mathd M}{M}
\end{equation*}
直接积分,即可得末速度
\begin{equation*}
	v_f = v_i + v_r \ln \frac{M_i}{M_f} - gt_f = v_i + v_r \ln \left(1+\frac{|\Delta M|}{M_f}\right) - gt_f
\end{equation*}
提高火箭末速度的方法:
\begin{itemize}
	\item 提高喷气速度;
	\item 用级连法提高燃料所占质量比;
	\item 提高燃烧速度,减少加速时间。
\end{itemize}
\end{solution}

\begin{example}
一轻绳跨过定滑轮的两端,一端挂着质量为$m$的重物,另一端和线密度为$\rho$的软链相连。开始时软链全部静止在地上,重物亦静止,后来重物下降,并将软链向上提起。问软链最后可提起多高?
\begin{figure}[htb]
\centering
\begin{asy}
	size(300);
	//第一章例10图
	pair O;
	real r,l1,l2,hand,a,T;
	O = (0,0);
	r = 1;
	l1 = 4;
	l2 = 6;
	T = 3;
	hand = 2;
	a = 0.6;
	draw(scale(r)*unitcircle);
	draw(O--(0,hand),linewidth(3bp));
	draw((-r,0)--(-r,-l1));
	draw((-r,0)--(-r-2*a,0));
	draw((-r,-l1)--(-r-2*a,-l1));
	draw(Label("$X$",MidPoint,W),(-r-1.5*a,0)--(-r-1.5*a,-l1),Arrows);
	fill(box((-r-a,-l1-a),(-r+a,-l1+a)),red);
	dot((-r,-l1));
	draw(Label("$T$",EndPoint,E),(-r,-l1)--(-r,-l1)+dir(90),Arrow);
	draw(Label("$Mg$",EndPoint,E),(-r,-l1)--(-r,-l1)+dir(-90),Arrow);
	draw((r,0)--(r,-l2+0.1));
	draw((r,-T)--(r,-l2+0.1)..(r+0.1,-l2)--(r+3*a,-l2),blue+linewidth(3bp));
	dot((r,-(T+l2)/2));
	draw(Label("$\rho gx$",EndPoint,W),(r,-(T+l2)/2)--(r,-(T+l2)/2)+dir(-90),Arrow);
	draw(Label("$T$",EndPoint,E),(r,-T)--(r,-T)+dir(90),Arrow);
	draw((r,-T)--(r+2*a,-T));
	draw((r,-l2)--(r+2*a,-l2));
	draw(Label("$x$",MidPoint,E),(r+1.5*a,-T)--(r+1.5*a,-l2),Arrows);
	//draw(O--(1.5,0),invisible);
\end{asy}
\caption{例\theexample}
\label{第一章例10图}
\end{figure}
\end{example}
\begin{solution}
地面上的软链在被提起时的速度为$u =0$,根据变质量系统的运动微分方程\eqref{chapter1:密歇尔斯基方程}即有
\begin{equation*}
	\rho x \ddot{x} + \rho \dot{x} \dot{x} = T - \rho gx
\end{equation*}
对左端物体有
\begin{equation*}
	M\ddot{X} = Mg - T
\end{equation*}
约束为
\begin{equation*}
	\dot{X} = \dot{x}
\end{equation*}
消去无关变量即有
\begin{equation*}
	\rho x \ddot{x} + \rho \dot{x} \dot{x} + M\ddot{x} = Mg - \rho g x
\end{equation*}
可化为
\begin{equation*}
	\frac{\mathrm{d}}{\mathrm{d} t} [(M+\rho x) \dot{x}] = (M-\rho x) g
\end{equation*}
即
\begin{equation*}
	\dot{x} \frac{\mathrm{d}}{\mathrm{d} x} [(M+\rho x) \dot{x}] = (M-\rho x) g
\end{equation*}
因此
\begin{equation*}
	\frac12 \frac{\mathrm{d}}{\mathrm{d} x} [(M+\rho x) \dot{x}]^2 = (M-\rho x)(M+\rho x) g
\end{equation*}
积分,并考虑到$\dot{x}\big|_{x=0} = 0$,即得
\begin{equation*}
	\frac12 \left[(M+\rho x)\dot{x}\right]^2 = \left(M^2 - \frac13 \rho^2 x^2\right) gx
\end{equation*}
当软链升到最高时,有$\dot{x}\big|_{t=t_{max}} = 0$,即有
\begin{equation*}
	\left(M^2 - \frac13 \rho^2 x^2_{max}\right) gx_{max} = 0
\end{equation*}
由此得最大高度
\begin{equation*}
	x_{max} = \frac{\sqrt{3}M}{\rho}
\end{equation*}
\end{solution}
