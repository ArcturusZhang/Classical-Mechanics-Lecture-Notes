\section{变质量系统}

\subsection{变质量系统的概念}

如果系统的质量或者组成系统的质点其中之一,或者两者同时随时间变化,这个系统就称为{\bf 变质量系统}。

在自然现象中,变质量系统很常见。例如浮冰由于融化而减少质量,又由于结冰或者落雪而增加质量。在工程技术中,最常提到的变质量系统就是火箭了。

对于变质量系统的研究,总是做这样的假设:系统分离或并入的质量是小量,而且每次并入或分离质量的时间间隔是小量。在这样的假设下,离开系统和进入系统的质量都是时间的连续可微的函数。

如果将系统的质量记作$M(t)$,将$t=0$时的质量记作$M_0$,则它随时间的变化规律为
\begin{equation}
	M(t) = M_0 - M_1(t) + M_2(t)
	\label{chapter1:变质量系统的质量关系}
\end{equation}
其中$M_1(t)$表示累计离开系统的质量,而$M_2(t)$表示累计进入系统的质量,$M_1,M_2$都是随时间的非递减的非负连续可微函数。

\subsection{变质量系统的动量定理和角动量定理}

设某个质点系相对惯性参考系$Oxyz$运动,研究相对$Oxyz$运动和变形的封闭曲面$S$。系统内的各个质点按照各自的运动可以进入或者离开曲面$S$围成的空间区域,那么曲面$S$围成的质点即组成一个变质量系统,将其动量用$\mbf{p}$表示。

设在时刻$t=t'$,系统的动量为$\mbf{p}$。而在$t=t''=t'+\Delta t$的时刻,系统的动量记作$\mbf{p}+\Delta \mbf{p}$。显然有
\begin{equation}
	\Delta \mbf{p} = \Delta \mbf{p}^*-\Delta \mbf{p}_1+\Delta \mbf{p}_2
	\label{chapter1:变质量系统的动量定理-初步}
\end{equation}
其中,$\Delta\mbf{p}_1$为在$\Delta t$时间内离开变质量系统的所有质点带走的动量,$\Delta\mbf{p}_2$为在这段时间内进入系统的所有质点带来的动量,而$\Delta\mbf{p}^*$则表示在这段时间内一直在系统内的所有质点本身的动量改变。

% \begin{figure}[htb]
% \centering
% \begin{asy}
	% size(300);
	% //变质量系统1
	% picture pic1;
	% pair O,F,v,u,m;
	% real rO,rm,lF,lv,lu;
	% O = (0,0);
	% rO = 2;
	% fill(pic1,yscale(2/3)*scale(rO)*unitcircle,yellow);
	% label(pic1,"$m$",O);
	% v = (1,0.7);
	% lv = 1.2;
	% draw(pic1,Label("$\boldsymbol{v}$",MidPoint,Relative(W)),v--v+lv*dir(30),Arrow);
	% F = (-1.3,-0.8);
	% lF = 1.2;
	% draw(pic1,Label("$\boldsymbol{F}$",MidPoint,Relative(W)),F-lF*dir(30)--F,Arrow);
	% rm = 0.7;
	% m = (3,0.4);
	% fill(pic1,shift(m)*yscale(2/3)*scale(rm)*unitcircle,green);
	% label(pic1,"$\Delta m$",m);
	% u = (-0.1,0.2)+m;
	% lu = 0.9;
	% draw(pic1,Label("$\boldsymbol{u}$",MidPoint,Relative(E)),u--u+lu*dir(135),Arrow);
	% label(pic1,"$t$",(0,-1.5),S);
	% add(pic1);
	% //变质量系统2
	% picture pic2;
	% pair O,F,v,u,m;
	% real rO,rm,lF,lv,lu;
	% O = (0,0);
	% rO = 2;
	% fill(pic2,yscale(2/3)*scale(rO)*unitcircle,yellow);
	% label(pic2,"$m'=m+\Delta m$",O);
	% v = (1,0.7);
	% lv = 1.2;
	% draw(pic2,Label("$\boldsymbol{v}'$",MidPoint,Relative(W)),v--v+lv*dir(30),Arrow);
	% F = (-1.3,-0.8);
	% lF = 1.2;
	% draw(pic2,Label("$\boldsymbol{F}'$",MidPoint,Relative(W)),F-lF*dir(30)--F,Arrow);
	% label(pic2,"$t'=t+\Delta t$",(0,-1.5),S);
	% add(shift((6.5,0))*pic2);
	% //draw(O--(9,0),invisible);
% \end{asy}
% \caption{变质量系统}
% \label{变质量系统}
% \end{figure}

设$\mbf{F}^{(e)}$是$t'$时刻作用在系统上的合外力,那么根据动量定理可得
\begin{equation}
	\dif{\mbf{p}^*}{t} = \mbf{F}^{(e)}
\end{equation}
在式\eqref{chapter1:变质量系统的动量定理-初步}两端除以$\Delta t$并取$\Delta t\to 0$的极限可得
\begin{equation}
	\dif{\mbf{p}}{t} = \mbf{F}^{(e)}+\mbf{F}^{(\text{ex})}
\end{equation}
其中$\mbf{F}^{(\text{ex})} = \mbf{F}_1+\mbf{F}_2$,而
\begin{equation}
	\mbf{F}_1 = -\lim_{\Delta t\to 0} \frac{\Delta \mbf{p}_1}{\Delta t},\quad \mbf{F}_2 = \lim_{\Delta t\to 0} \frac{\Delta \mbf{p}_2}{\Delta t}
\end{equation}
矢量$\mbf{F}_1$和$\mbf{F}_2$具有力的量纲,称为{\bf 反推力}。反推力$\mbf{F}_1$是由质点分离引起的,反推力$\mbf{F}_2$是由质点并入引起的。

设惯性参考系$Oxyz$中的固定参考点为$A$,则类似前面的讨论可得变质量系统的角动量定理
\begin{equation}
	\dif{\mbf{L}_A}{t} = \mbf{M}_A^{(e)} + \mbf{M}_A^{(\text{ex})}
\end{equation}
其中$\mbf{L}_A$为系统对$A$点的角动量,$\mbf{M}_A^{(e)}$为系统受到的对$A$点的合外力矩,而$\mbf{M}_A^{(\text{ex})} = \mbf{M}_{A1} + \mbf{M}_{A2}$是变质量系统的附加力矩,其中
\begin{equation}
	\mbf{M}_{A1} = -\lim_{\Delta t\to 0} \frac{\Delta \mbf{L}_{A1}}{\Delta t},\quad \mbf{M}_{A2} =  \lim_{\Delta t\to 0} \frac{\Delta \mbf{L}_{A2}}{\Delta t}
\end{equation}
这里的$\Delta \mbf{L}_{A1}$为在$\Delta t$时间内离开变质量系统的所有质点的角动量,$\Delta \mbf{L}_{A2}$为在这段时间内进入系统的所有质点的角动量。

% \begin{equation}
	% \dif{\mbf{p}_1}{t} = \lim_{\Delta t\to 0} \frac{\Delta \mbf{p}_1}{\Delta t},\quad \dif{\mbf{p}_2}{t} = \lim_{\Delta t\to 0} \frac{\Delta \mbf{p}_2}{\Delta t}
% \end{equation}

\subsection{变质量质点的运动微分方程}

作为变质量系统的一种特殊情况,现在来考虑变质量质点的运动微分方程。其质量变化同样满足式\eqref{chapter1:变质量系统的质量关系},即
\begin{equation}
	M(t) = M_0 - M_1(t) + M_2(t)
	\label{chapter1:变质量系统的质量关系-重写}
\end{equation}
如果记$\mbf{u}_1$为$t'$时刻离开变质量质点的微粒的速度,$\mbf{u}_2$为$t'$时刻并入质点的微粒的速度,那么精确到$\Delta t$和$\Delta M$的一阶小量,可有
\begin{equation}
	\Delta \mbf{p}_1 = \Delta M_1\mbf{u}_1,\quad \Delta \mbf{p}_2 = \Delta M_2\mbf{u}_2
\end{equation}	
由此可得
\begin{equation}
	\mbf{F}_1 = -\lim_{\Delta t\to 0} \frac{\Delta \mbf{p}_1}{\Delta t} = -\dif{M_1}{t}\mbf{u}_1,\quad \mbf{F}_2 = \lim_{\Delta t\to 0} \frac{\Delta \mbf{p}_2}{\Delta t} = \dif{M_2}{t}\mbf{u}_2
\end{equation}
设$\mbf{v}$为变质量质点的速度,则其动量为
\begin{equation}
	\mbf{p} = M\mbf{v}
\end{equation}
则有
\begin{equation}
	\dif{M}{t}\mbf{v}+M\dif{\mbf{v}}{t} = \mbf{F}^{(e)} - \dif{M_1}{t}\mbf{u}_1 + \dif{M_2}{t}\mbf{u}_2
	\label{chapter1:广义密歇尔斯基方程-初步}
\end{equation}
将式\eqref{chapter1:变质量系统的质量关系-重写}代入方程\eqref{chapter1:广义密歇尔斯基方程-初步}中可得
\begin{equation}
	M\dif{\mbf{v}}{t} = \mbf{F}^{(e)} - \dif{M_1}{t}(\mbf{u}_1-\mbf{v}) + \dif{M_2}{t}(\mbf{u}_2-\mbf{v})
	\label{chapter1:广义密歇尔斯基方程}
\end{equation}
方程\eqref{chapter1:广义密歇尔斯基方程}即为变质量质点的运动微分方程,称为{\bf 广义Meshcherskii方程}\footnote{Meshcherskii,密歇尔斯基,前苏联力学家。}。在方程\eqref{chapter1:广义密歇尔斯基方程}中,$\mbf{u}_1-\mbf{v}=\mbf{u}_{1r}$是分离微粒相对质点的速度,而$\mbf{u}_2-\mbf{v}=\mbf{u}_{2r}$是并入微粒相对质点的速度。

如果只有分离微粒而没有并入微粒,此时$M_2(t)\equiv 0$,此时方程\eqref{chapter1:广义密歇尔斯基方程}变为
\begin{equation}
	M\dif{\mbf{v}}{t} = \mbf{F}^{(e)} + \dif{M}{t}\mbf{u}_{1r}
	\label{chapter1:密歇尔斯基方程}
\end{equation}
方程\eqref{chapter1:密歇尔斯基方程}称为{\bf Meshcherskii方程}。由Meshcherskii方程\eqref{chapter1:密歇尔斯基方程}可以看出,微粒分离等效于在质点上作用附加的反推力$\mbf{F}_1 = \dif{M}{t}\mbf{u}_{1r}$。

\begin{example}[重力场中的火箭]
在假设火箭竖直运动且喷气速度恒定,其所受外力只有恒定重力$\mbf{F}^{(e)} = -Mg\mbf{e}_3$的情况下求解火箭的运动。
\end{example}
\begin{solution}
相对火箭的喷气速度
\begin{equation*}
	\mbf{u}-\mbf{v} = -u_r\mbf{e}_3
\end{equation*}
此时火箭的运动方程即为Meshcherskii方程\eqref{chapter1:密歇尔斯基方程}
\begin{equation*}
	M\frac{\mathrm{d} v}{\mathrm{d} t} = -Mg - \dif{M}{t}u_r
\end{equation*}
化简为
\begin{equation*}
	\mathrm{d} v = -g\mathd t - u_r \frac{\mathd M}{M}
\end{equation*}
直接积分,即可得末速度
\begin{equation*}
	v_f = v_i + v_r \ln \frac{M_i}{M_f} - gt_f = v_i + v_r \ln \left(1+\frac{|\Delta M|}{M_f}\right) - gt_f
\end{equation*}
提高火箭末速度的方法:
\begin{itemize}
	\item 提高喷气速度;
	\item 用级连法提高燃料所占质量比;
	\item 提高燃烧速度,减少加速时间。
\end{itemize}
\end{solution}

\begin{example}
一轻绳跨过定滑轮的两端,一端挂着质量为$m$的重物,另一端和线密度为$\rho$的软链相连。开始时软链全部静止在地上,重物亦静止,后来重物下降,并将软链向上提起。问软链最后可提起多高?
\begin{figure}[htb]
\centering
\begin{asy}
	size(300);
	//第一章例10图
	pair O;
	real r,l1,l2,hand,a,T;
	O = (0,0);
	r = 1;
	l1 = 4;
	l2 = 6;
	T = 3;
	hand = 2;
	a = 0.6;
	draw(scale(r)*unitcircle);
	draw(O--(0,hand),linewidth(3bp));
	draw((-r,0)--(-r,-l1));
	draw((-r,0)--(-r-2*a,0));
	draw((-r,-l1)--(-r-2*a,-l1));
	draw(Label("$X$",MidPoint,W),(-r-1.5*a,0)--(-r-1.5*a,-l1),Arrows);
	fill(box((-r-a,-l1-a),(-r+a,-l1+a)),red);
	dot((-r,-l1));
	draw(Label("$T$",EndPoint,E),(-r,-l1)--(-r,-l1)+dir(90),Arrow);
	draw(Label("$Mg$",EndPoint,E),(-r,-l1)--(-r,-l1)+dir(-90),Arrow);
	draw((r,0)--(r,-l2+0.1));
	draw((r,-T)--(r,-l2+0.1)..(r+0.1,-l2)--(r+3*a,-l2),blue+linewidth(3bp));
	dot((r,-(T+l2)/2));
	draw(Label("$\rho gx$",EndPoint,W),(r,-(T+l2)/2)--(r,-(T+l2)/2)+dir(-90),Arrow);
	draw(Label("$T$",EndPoint,E),(r,-T)--(r,-T)+dir(90),Arrow);
	draw((r,-T)--(r+2*a,-T));
	draw((r,-l2)--(r+2*a,-l2));
	draw(Label("$x$",MidPoint,E),(r+1.5*a,-T)--(r+1.5*a,-l2),Arrows);
	//draw(O--(1.5,0),invisible);
\end{asy}
\caption{例\theexample}
\label{第一章例10图}
\end{figure}
\end{example}
\begin{solution}
地面上的软链在被提起时的速度为$u =0$,根据变质量系统的运动微分方程\eqref{chapter1:密歇尔斯基方程}即有
\begin{equation*}
	\rho x \ddot{x} + \rho \dot{x} \dot{x} = T - \rho gx
\end{equation*}
对左端物体有
\begin{equation*}
	M\ddot{X} = Mg - T
\end{equation*}
约束为
\begin{equation*}
	\dot{X} = \dot{x}
\end{equation*}
消去无关变量即有
\begin{equation*}
	\rho x \ddot{x} + \rho \dot{x} \dot{x} + M\ddot{x} = Mg - \rho g x
\end{equation*}
可化为
\begin{equation*}
	\frac{\mathrm{d}}{\mathrm{d} t} [(M+\rho x) \dot{x}] = (M-\rho x) g
\end{equation*}
即
\begin{equation*}
	\dot{x} \frac{\mathrm{d}}{\mathrm{d} x} [(M+\rho x) \dot{x}] = (M-\rho x) g
\end{equation*}
因此
\begin{equation*}
	\frac12 \frac{\mathrm{d}}{\mathrm{d} x} [(M+\rho x) \dot{x}]^2 = (M-\rho x)(M+\rho x) g
\end{equation*}
积分,并考虑到$\dot{x}\big|_{x=0} = 0$,即得
\begin{equation*}
	\frac12 \left[(M+\rho x)\dot{x}\right]^2 = \left(M^2 - \frac13 \rho^2 x^2\right) gx
\end{equation*}
当软链升到最高时,有$\dot{x}\big|_{t=t_{max}} = 0$,即有
\begin{equation*}
	\left(M^2 - \frac13 \rho^2 x^2_{max}\right) gx_{max} = 0
\end{equation*}
由此得最大高度
\begin{equation*}
	x_{max} = \frac{\sqrt{3}M}{\rho}
\end{equation*}
\end{solution}