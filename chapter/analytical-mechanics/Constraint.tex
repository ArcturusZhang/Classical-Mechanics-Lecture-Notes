\chapter{约束}

\section{系统运动的多维空间描述}

\subsection{Descartes位形空间$C$}

与矢量力学类似,我们仍然考虑质点系的运动,将其中的质点记作$P_i(i=1,2,\cdots,n)$,质点$P_i$的质量、位矢、速度和加速度分别记作$m_i, \boldsymbol{r}_i, \boldsymbol{v}_i, \boldsymbol{a}_i$。

在描述质点系中各质点的位置状态时,需要知道每个质点的位矢$\boldsymbol{r}_i(t)$,这在直角坐标系下就是需要知道其位矢的三个分量函数$x_i(t), y_i(t), z_i(t)$。因此,每一时刻决定系统位置形状(简称{\bf 位形})需要$3n$个数,可以抽象地将其表示为下面$3n$维空间中的点:
\begin{equation}
    \boldsymbol{c} = \begin{pmatrix}
        u_1 \\ u_2 \\ u_3 \\ \vdots \\ u_{3n}
    \end{pmatrix} = \begin{pmatrix}
        x_1(t) \\ y_1(t) \\ z_1(t) \\ \vdots \\ x_n(t) \\ y_n(t) \\ z_n(t)
    \end{pmatrix}
\end{equation}
这个$3n$维的空间称为{\bf 位形空间}$C$。此时,质点系每一个时刻的位形对应位形空间$C$中的一个点,而位形空间$C$中的任意一个点,也刚好对应质点系的一个可能位形\footnote{当然并不是位形空间中的每一个点都能够对应质点系实际存在的位形,例如不同质点不能够占据同一个位置,或是质点系受到其他限制条件而使得某些位形空间中的点无法到达。}。当质点系运动时,它在位形空间中的表现点$\boldsymbol{c}$就在位形空间中画出一条轨迹,称为质点系运动的位形轨迹。

在位形空间中,我们仍然按照一般Euclid空间中的方式来定义位形空间中的距离为
\begin{equation}
    D^2 = \sum_{i=1}^{3n} (\Delta u_i)^2
\end{equation}

在后面所考虑的力学系统运动中,我们假定所有的作用一般是有界的,无界的作用仅允许存在于有限时刻上的孤立打击(即撞击),在这种情况下,位形空间中的位形轨迹$\boldsymbol{c}(t)$具有如下性质:
\begin{enumerate}
    \item 轨迹$\boldsymbol{c}(t)$是连续的,即$\boldsymbol{c}(t)$的各个分量是连续的。在经典力学的意义上不允许系统的位形不经过时间间隔就发生有限的变化;
    \item 轨迹$\boldsymbol{c}(t)$可以自我相交,即存在重点;
    \item 除某些孤立的撞击点和静止点之外,轨迹$\boldsymbol{c}(t)$可以定义方向
    \begin{equation}
        \boldsymbol{d}_c = \left(\sum_{i=1}^{3n}\dot{u}_i^2\right)^{-\frac12}\begin{pmatrix}
            \dot{u}_1 \\ \dot{u}_2 \\ \vdots \\ \dot{u}_{3n}
        \end{pmatrix}
    \end{equation}
    即除去某些孤立点之外,函数$u_i(t)(i=1,2,\cdots,3n)$是可微的;
    \item 轨迹$\boldsymbol{c}(t)$应当是分段连续可微的,其拐点仅发生在方向无法确定的点处,即孤立的撞击点和静止点处。
\end{enumerate}

\subsection{事件空间$E$}

将时间$t$与系统的位形联系在一起,就构成了一个事件。经典力学的时空观是绝对空间和绝对时间,因此经典力学中的事件就是系统位形与时间的简单并列\footnote{与此对应的便是相对论的时空观,其中不存在绝对空间和绝对时间,时间和空间之间是相互关联的。}。一个事件可以用一个$3n+1$维空间中的坐标来表示,即
\begin{equation}
    \boldsymbol{e} = \begin{pmatrix}
        u_1 \\ u_2 \\ u_3 \\ \vdots \\ u_{3n} \\ t
    \end{pmatrix}
\end{equation}
这个$3n+1$维的空间称为{\bf 事件空间}$E$。在系统运动时,它的事件点$\boldsymbol{e}$同样在事件空间$E$中形成一条超曲线,称之为质点系运动的事件轨迹。在经典力学中,由于绝对时间的存在,事件轨迹在除时间轴$Ot$以外的子空间上的投影即为位形轨迹,图\ref{chapter2:二维运动的位形轨迹和事件轨迹}显示了二维圆周运动的位形轨迹与事件轨迹。

\begin{figure}[!htb]
\centering
\begin{asy}
    size(300);
    pair i, j, k, O;
    O = (0,0);
    i = 0.5*dir(37);
    j = (0,1);
    k = (1,0);
    draw(Label("$x$", EndPoint, SE), O--2*i, Arrow);
    draw(Label("$y$", EndPoint), O--2*j, Arrow);
    real T = 5;
    draw(Label("$t$", EndPoint), O--T*k, Arrow);
    pair c = (1,1);
    real r = 0.5;
    real omega = 2*pi;
    pair c_tra(real t) {
        real x, y;
        x = c.x + r * cos(omega * t);
        y = c.y + r * sin(omega * t);
        return x * i + y * j;
    }
    pair e_tra(real t) {
        return c_tra(t) + t * k;
    }
    path p = graph(c_tra, 0, 1);
    path q = graph(e_tra, 0, 0.8*T, 1000);
    real[] ts = {3,3.25,3.5,3.75};
    for (real v : ts) {
        pair P = c_tra(v);
        pair Q = e_tra(v);
        dot(P);
        draw(P--Q, dashed);
        while (v > 0) {
            dot(e_tra(v));
            v = v - 1;
        }
    }
    draw(p);
    draw(q);
\end{asy}
\caption{二维圆周运动的位形轨迹和事件轨迹}
\label{chapter2:二维运动的位形轨迹和事件轨迹}
\end{figure}

事件轨迹$e$的一般性质如下:
\begin{enumerate}
    \item $e$轨迹是连续的;
    \item $e$轨迹不可能有重点,这是由函数$u_i(t)$的单值性决定的;
    \item 除有限的几个孤立撞击点以外,$e$轨迹在每一点都可以定义一个方向:
    \begin{equation}
        \boldsymbol{d}_e = \left(\sum_{i=1}^n\dot{u}_i^2 + 1\right)^{-\frac12} \begin{pmatrix}
            \dot{u}_1 \\ \dot{u}_2 \\ \vdots \\ \dot{u}_{3n} \\ 1
        \end{pmatrix}
    \end{equation}
    这些方向是连续变化的;
    \item $e$轨迹在$t=\text{常数}$的超平面上是正向穿越,这说明$e$轨迹的方向与$t=\text{常数}$平面的法向之间的夹角小于$\dfrac{\pi}{2}$。方向矢量$\boldsymbol{d}_e$的最后一个分量永远非零保证了这一点;
    \item $e$轨迹仅在受撞击处存在拐点,其余区域是连续可微的。
\end{enumerate}

\subsection{状态空间$S$}

将$t$时刻系统的位形和该时刻的速度联合在一起,称为该时刻系统的{\bf 状态},它可以用一个$6n$维空间中的坐标来表示,即
\begin{equation}
    \boldsymbol{s} = \begin{pmatrix}
        s_1 \\ \vdots \\ s_{3n} \\ s_{3n+1} \\ \vdots \\ s_{6n}
    \end{pmatrix} = \begin{pmatrix}
        u_1 \\ \vdots \\ u_{3n} \\ \dot{u}_{1} \\ \vdots \\ \dot{u}_{3n}
    \end{pmatrix}
\end{equation}
这个$6n$维的空间称为{\bf 状态空间}$S$。此时,质点系的每一个状态都对应状态空间的一个点,而状态空间中的任意一个点都表示系统的一个可能的状态。另外,根据系统的运动微分方程\eqref{chapter2:系统的运动微分方程——牛顿第二定律}:
\begin{equation}
    m_i\frac{\mathd^2\boldsymbol{r}_i}{\mathd t^2} = \boldsymbol{F}_i, \quad i = 1,2,\cdots,n
\end{equation}
由于这是二阶微分方程,因此系统在某时刻的状态如果给定,那么系统的过去和未来都可以据此唯一确定。

\subsection{状态时间空间$T$}

将系统的状态和时间联合在一起,就构成了由$6n+1$维空间点构成的状态时间空间$T$,即
\begin{equation}
    \boldsymbol{t} = \begin{pmatrix}
        s_1 \\ \vdots \\ s_{6n} \\ t
    \end{pmatrix}
\end{equation}

\section{约束的定义和数学性质}

\subsection{约束及其分类}

在质点系的运动过程中,各质点的位置和速度受到的限制称为{\heiti 约束}。如果系统不受约束,则称系统是{\bf 自由的}或者为{\bf 自由系统},相应地,如果系统受到一个或更多约束的作用,则称系统是{\bf 非自由的}。

如果系统中各质点的位矢为$\mbf{r}_i\,(i=1,2,\cdots,n)$,那么各质点的速度可以表示为$\dot{\mbf{r}}_i\,(i=1,2,\cdots,n)$,约束可以用方程表示为
\begin{equation}
	f(\mbf{r}_1,\mbf{r}_2,\cdots,\mbf{r}_n,\dot{\mbf{r}}_1,\dot{\mbf{r}}_2,\cdots,\dot{\mbf{r}}_n,t) = 0
	\label{约束方程}
\end{equation}
或者
\begin{equation}
	f(\mbf{r}_1,\mbf{r}_2,\cdots,\mbf{r}_n,\dot{\mbf{r}}_1,\dot{\mbf{r}}_2,\cdots,\dot{\mbf{r}}_n,t) \geqslant 0
	\label{单面约束方程}
\end{equation}
类似式\eqref{约束方程}只有等号成立的约束称为{\bf 双面约束},而类似式\eqref{单面约束方程}等号和不等号都可以成立的约束称为{\bf 单面约束}。

约束可以按其是否与质点的速度有关而分为以下两类:
\begin{enumerate}
	\item {\heiti 几何约束}:约束与速度无关,仅对几何位形加以限制。即
	\begin{equation}
		f(\mbf{r}_1,\mbf{r}_2,\cdots,\mbf{r}_n,t) = 0
	\end{equation}
	一个几何约束减少一个独立坐标与一个独立速度分量。例如,单摆(如图\ref{单摆的几何约束}所示)的约束可以表示为
	\begin{equation*}
		|\mbf{r}|^2-l^2 = x^2+y^2-l^2 = 0
	\end{equation*}
\begin{figure}[htb]
\centering
\begin{minipage}[t]{0.45\textwidth}
\begin{asy}
	size(200);
	//单摆的几何约束
	pair O,x,y,r;
	O = (0,0);
	x = (4,0);
	y = (0,-4.5);
	r = (3,-4);
	draw(Label("$x$",EndPoint),O--x,Arrow);
	draw(Label("$y$",EndPoint),O--y,Arrow);
	label("$O$",O,W);
	draw(Label("$l$",MidPoint,Relative(W)),O--r);
	fill(shift(r)*scale(0.15)*unitcircle,black);
\end{asy}
\caption{单摆的几何约束}
\label{单摆的几何约束}
\end{minipage}
\hspace{0.5cm}
\begin{minipage}[t]{0.45\textwidth}
\begin{asy}
	size(200);
	//纯滚动直立圆盘
	real hd,hdd;
	pair O,x,y,z,i,j,k;
	hd = 131+25/60;
	hdd = 7+10/60;
	O = (0,0);
	x = 1.2*dir(90+hd);
	y = 1.5*dir(-hdd);
	z = 1.5*dir(90);
	draw(Label("$x$",EndPoint),O--x,Arrow);
	draw(Label("$y$",EndPoint),O--y,Arrow);
	draw(Label("$z$",EndPoint),O--z,Arrow);
	label("$O$",O,NW);
	i = (-sqrt(2)/4,-sqrt(14)/12);
	j = (sqrt(14)/4,-sqrt(2)/12);
	k = (0,2*sqrt(2)/3);
	
	real x0,y0,r,phi;
	pair P;
	pair planecur(real xx){
		real yy = y0+1*tan(xx-x0);
		return xx*i+yy*j;
	}
	pair dplanecur(real xx){
		real yy = 1/((cos(xx-x0))**2);
		return xx*i+yy*j;
	}
	pair wheel(real theta){
		real xx,yy,zz;
		xx = x0+r*sin(theta)*cos(phi);
		yy = y0+r*sin(theta)*sin(phi);
		zz = r+r*cos(theta);
		return xx*i+yy*j+zz*k;
	}
	real getphi(real xx){
		real yy = 1/((cos(xx-x0))**2);
		return atan(yy/xx);
	}
	x0 = 1;
	y0 = 0.7;
	r = 0.5;
	phi = getphi(x0);
	unfill(graph(wheel,0,2*pi)--cycle);
	draw(graph(wheel,0,2*pi),linewidth(0.8bp));
	draw(graph(planecur,x0-1,x0+1));
	P = planecur(x0);
	dot(P);
	draw((P-0.7*dplanecur(x0))--(P+dplanecur(x0)),dashed);
	draw(P--P+r*k--wheel(-1.2));
	label("$(x,y)$",P+r*k,E);
	label("$\theta$",P+r*k,WSW);
	label("$\phi$",P-0.7*dplanecur(x0),2*S);
\end{asy}
\caption{作曲线运动的纯滚动直立圆盘}
\label{作曲线运动的纯滚动直立圆盘}
\end{minipage}
\end{figure}
	\item {\heiti 运动约束}:约束与速度有关,同时对系统的几何位形和运动情况加以限制。运动约束条件不减少独立坐标个数,但每个运动约束减少一个独立速度分量。例如,作曲线运动的纯滚动直立圆盘(如图\ref{作曲线运动的纯滚动直立圆盘}所示)的约束可以表示为
	\begin{equation*}
		\begin{cases}
			\dot{x} - R\dot{\theta}\cos \phi = 0 \\
			\dot{y} - R\dot{\theta}\sin \phi = 0
		\end{cases}
	\end{equation*}
	圆盘可由任一指定初位形出发,到达任一指定末位形。但如果圆盘作直线运动,此时$\phi = \phi_0\text{(常数)}$,则运动约束可积\footnote{在求解运动前。}(此时即称为{\heiti 可积运动约束}),即
	\begin{equation*}
		\begin{cases}
			x - R\theta\cos \phi_0 + C_1 = 0 \\
			y - R\theta\sin \phi_0 + C_2 = 0
		\end{cases}
	\end{equation*}
	可积运动约束实质上等价于几何约束。
\end{enumerate}

几何约束与可积运动约束称为{\heiti 完整约束},一个完整约束方程同时减少一个独立坐标与一个独立速度分量。不可积运动约束则称为{\heiti 非完整约束},一个非完整约束方程只减少一个独立速度分量。如果一个系统的所有约束皆为完整约束,则该系统称为{\heiti 完整系统},否则称为{\heiti 非完整系统}。

根据前面对约束的分类,今后在研究质点系的运动时,我们只考虑完整系统,或者具有线性非完整约束的非完整系统。即系统的约束为
\begin{subnumcases}{}
	f_j(\mbf{r}_1,\mbf{r}_2,\cdots,\mbf{r}_n,t)=0 \quad (j=1,2,\cdots,k) \label{chapter2:完整约束的一般形式} \\
	\sum_{i=1}^n \mbf{A}_{ji}(\mbf{r}_1,\mbf{r}_2,\cdots,\mbf{r}_n,t) \cdot \mbf{v}_i + A_{j0}(\mbf{r}_1,\mbf{r}_2,\cdots,\mbf{r}_n,t)=0 \quad (j=1,2,\cdots,k') \label{chapter2:非完整约束的一般形式}
\end{subnumcases}
其中$\mbf{v}_i=\dot{\mbf{r}}_i$,在特殊情况下$k$和$k'$可以等于零。

如果在完整约束\eqref{chapter2:完整约束的一般形式}中不显含时间$t$,则该完整约束称为{\bf 定常的}或{\bf 稳定的}。如果在非完整约束\eqref{chapter2:非完整约束的一般形式}中矢量函数$\mbf{A}_{ji}$不显含时间$t$,且标量函数$A_{j0}$恒等于零,则该非完整约束称为{\bf 定常的}或{\bf 稳定的}。如果一个系统是自由的或者所有约束都是定常约束,则该系统称为{\bf 定常系统},如果至少有一个约束是非定常的,则称为{\bf 非定常系统}。

\subsection{约束对质点系的限制·自由度}

非自由系统的质点不能在空间中任意运动,约束允许的位矢、速度和加速度应该满足约束方程\eqref{chapter2:完整约束的一般形式}和\eqref{chapter2:非完整约束的一般形式},或者它们的导出形式。

设给定某个时刻$t=t^*$,如果该时刻内系统内各质点的位矢$\mbf{r}_i=\mbf{r}_i^*$满足完整约束\eqref{chapter2:完整约束的一般形式},则称在该给定时刻系统处于{\bf 可能位形}。

约束同样限制了系统中质点的速度。将式\eqref{chapter2:完整约束的一般形式}两端对时间求导,可得
\begin{equation}
	\sum_{i=1}^n \frac{\pl f_j}{\pl \mbf{r}_i}\cdot \mbf{v}_i + \frac{\pl f_j}{\pl t} = 0 \quad (j=1,2,\cdots,k)
	\label{chapter2:完整约束对质点系中质点速度的限制}
\end{equation}
当然速度还需满足非完整约束\eqref{chapter2:非完整约束的一般形式},即
\begin{equation}
	\sum_{i=1}^n \mbf{A}_{ji} \cdot \mbf{v}_i + A_{j0}=0 \quad (j=1,2,\cdots,k')
	\label{chapter2:非完整约束对质点系中质点速度的限制}
\end{equation}
当系统处于给定时刻的可能位形时,满足线性方程组\eqref{chapter2:完整约束对质点系中质点速度的限制}、\eqref{chapter2:非完整约束对质点系中质点速度的限制}的矢量$\mbf{v}_i=\mbf{v}_i^*$集合称为该时刻的{\bf 可能速度}。

约束也将限制系统中质点的加速度,将式\eqref{chapter2:完整约束对质点系中质点速度的限制}和\eqref{chapter2:非完整约束对质点系中质点速度的限制}分别对时间求导,可得
\begin{equation}
	\sum_{i=1}^n \frac{\pl f_j}{\pl\mbf{r}_i}\cdot \mbf{a}_i + \sum_{i,l=1}^n \left(\frac{\pl^2 f_j}{\pl \mbf{r}_i\pl \mbf{r}_l} \cdot \mbf{v}_l\right) \cdot\mbf{v}_i + 2\sum_{i=1}^n \frac{\pl^2 f_j}{\pl \mbf{r}_i\pl t} \cdot \mbf{v}_i + \frac{\pl^2 f_j}{\pl t^2} = 0 \quad (j=1,2,\cdots,k) \label{chapter2:完整约束对质点系中质点加速度的限制}
\end{equation}
和
\begin{equation}
	\sum_{i=1}^n \mbf{A}_{ji}\cdot \mbf{a}_i + \sum_{i,l=1}^n \left(\frac{\pl \mbf{A}_{ji}}{\pl \mbf{r}_l} \cdot \mbf{v}_l\right) \cdot \mbf{v}_i + \sum_{i=1}^n \frac{\pl A_{j0}}{\pl \mbf{r}_i} \cdot \mbf{v}_i + \frac{\pl A_{j0}}{\pl t} = 0 \quad (j=1,2,\cdots,k') \label{chapter2:非完整约束对质点系中质点加速度的限制}
\end{equation}
当系统处于给定时刻的可能位形并且具有可能速度时,满足线性方程组\eqref{chapter2:完整约束对质点系中质点加速度的限制}、\eqref{chapter2:非完整约束对质点系中质点加速度的限制}的矢量$\mbf{a}_i=\mbf{a}_i^*$集合称为该时刻的{\bf 可能加速度}。

由于可能速度和可能加速度分别需要满足线性方程组\eqref{chapter2:完整约束对质点系中质点速度的限制}、\eqref{chapter2:非完整约束对质点系中质点速度的限制}和方程组\eqref{chapter2:完整约束对质点系中质点加速度的限制}、\eqref{chapter2:非完整约束对质点系中质点加速度的限制},每个方程组中都有$k+k'$个方程,因此整数$3n-k-k'$应该是一个正数,否则约束的限制过强,将使得系统不可能运动或者只能按照约束给定的规律运动。这个正数,记作$f=3n-k-k'$,称为系统的{\bf 自由度}。

由此可以看出,确定系统可能速度和可能加速度的方程数($k+k'$个)小于它们的分量数($3n$个),因此在给定时刻系统将存在无穷多组满足约束的可能速度$\mbf{v}_i^*$和可能加速度$\mbf{a}_i^*$。

设在给定时刻$t=t^*$系统处于可能位形$\mbf{r}_i^*$,并且具有可能速度$\mbf{v}_i^*$和可能加速度$\mbf{a}_i^*$,而在$t=t^*+\Delta t$时刻系统相应的可能位形为$\mbf{r}_i^*+\Delta \mbf{r}_i$,则$\Delta \mbf{r}_i$称为系统从时刻$t=t^*$的给定位置$\mbf{r}_i^*$在$\Delta t$时间内的{\bf 可能位移}。对于充分小的$\Delta t$,系统的可能位移可以写作
\begin{equation}
	\Delta \mbf{r}_i = \mbf{v}_i^*\Delta t + \frac12 \mbf{a}_i^*(\Delta t)^2 + \cdots\quad (i=1,2,\cdots,n)
	\label{chapter2:可能位移}
\end{equation}
忽略式\eqref{chapter2:可能位移}中高于$\Delta t$的项,由式\eqref{chapter2:完整约束对质点系中质点速度的限制}和\eqref{chapter2:非完整约束对质点系中质点速度的限制}可得可能位移满足的相对$\Delta t$的线性方程组:
\begin{subnumcases}{}
	\sum_{i=1}^n \frac{\pl f_j}{\pl \mbf{r}_i}\cdot \Delta \mbf{r}_i + \frac{\pl f_j}{\pl t}\Delta t = 0 \quad (j=1,2,\cdots,k') \label{chapter2:可能位移方程1} \\
	\sum_{i=1}^n \mbf{A}_{ji}\cdot \Delta \mbf{r}_i + A_{j0}\Delta t = 0 \quad (j=1,2,\cdots,k') \label{chapter2:可能位移方程2}
\end{subnumcases}
式中各系数都是在时刻$t=t^*$以及可能位形$\mbf{r}_i=\mbf{r}_i^*$处取值的。

\section{虚位移与广义坐标}

\subsection{实位移与虚位移·等时变分}\label{chapter2:subsection-实位移与虚位移}

当在可能位移\eqref{chapter2:可能位移}中取$\Delta t\to 0$的极限时,可得$\mathd \mbf{r}_i = \mbf{v}_i^*\mathd t$,此处的$\mathd \mbf{r}_i$是无穷小位移,称为{\bf 实位移}或{\bf 真实位移}。实位移满足类似\eqref{chapter2:可能位移方程1}和\eqref{chapter2:可能位移方程2}的方程组:
\begin{subnumcases}{}
	\sum_{i=1}^n \frac{\pl f_j}{\pl \mbf{r}_i}\cdot \mathd \mbf{r}_i + \frac{\pl f_j}{\pl t}\mathd t = 0 \quad (j=1,2,\cdots,k') \label{chapter2:实位移方程1} \\
	\sum_{i=1}^n \mbf{A}_{ji}\cdot \mathd \mbf{r}_i + A_{j0}\mathd t = 0 \quad (j=1,2,\cdots,k') \label{chapter2:实位移方程2}
\end{subnumcases}
式中各系数都是在时刻$t=t^*$以及可能位形$\mbf{r}_i=\mbf{r}_i^*$处取值的。需要注意的是,实位移满足的方程组\eqref{chapter2:实位移方程1}、\eqref{chapter2:实位移方程2}与可能位移的一阶近似方程组\eqref{chapter2:可能位移方程1}、\eqref{chapter2:可能位移方程2}形式上完全一致,但它们的物理意义是完全不同的。实位移是系统真实发生的位移,其不仅与约束有关,还与系统的受力状态等外部条件相关,而可能位移仅是在约束允许下可能产生的位移。所有实位移构成的集合必然包含于可能位移构成的集合中。

由于当自由度$f>0$时,实位移分量数$3n$多于确定实位移方程组\eqref{chapter2:实位移方程1}、\eqref{chapter2:实位移方程2}的方程数,因此无法仅通过约束唯一确定系统的实位移。

与实位移相对应,{\bf 虚位移}在分析力学中有着十分重要的地位。设在$t=t^*$时刻,系统处于可能位形$\mbf{r}_i^*$上,虚位移是指满足下面线性齐次方程组的$\delta \mbf{r}_i$集合:
\begin{subnumcases}{}
	\sum_{i=1}^n \frac{\pl f_j}{\pl \mbf{r}_i}\cdot \delta \mbf{r}_i = 0\quad (j=1,2,\cdots,k) \label{chapter2:虚位移方程1} \\
	\sum_{i=1}^n \mbf{A}_{ji}\cdot \delta \mbf{r}_i = 0 \quad (j=1,2,\cdots,k') \label{chapter2:虚位移方程2}
\end{subnumcases}
式中各系数都是在时刻$t=t^*$以及可能位形$\mbf{r}_i=\mbf{r}_i^*$处取值的。显然所有虚位移分量共有$3n$个,大于方程组\eqref{chapter2:虚位移方程1}、\eqref{chapter2:虚位移方程2}的方程数,因此虚位移有无数多个。对比方程组\eqref{chapter2:虚位移方程1}、\eqref{chapter2:虚位移方程2}和\eqref{chapter2:实位移方程1}、\eqref{chapter2:实位移方程2}可知,对于定常系统,实位移是虚位移的一个。

通常假设虚位移$\delta \mbf{r}_i$是无穷小量,由虚位移满足的方程组\eqref{chapter2:虚位移方程1}、\eqref{chapter2:虚位移方程2}和可能位移满足的方程组\eqref{chapter2:可能位移方程1}、\eqref{chapter2:可能位移方程2}可知,对于定常系统,与$\Delta t$成线性关系的可能位移集合与虚位移集合完全相同,可以说,虚位移是在约束“冻结”在$t=t^*=\text{常数}$情况下的可能位移。

\begin{example}

\begin{figure}[htb]
\centering
\begin{asy}
	size(200);
	//约束在曲面上的质点
	pair A[],O,a,b,ns,g1,g2;
	O = (0,0);
	a = (2,0);
	b = (0.5,1);
	A[1] = O-a-b;
	A[2] = O+a-b;
	A[3] = O+a+b;
	A[4] = O-a+b;
	A[5] = (A[1]+A[2])/2+(0,0.3);
	A[6] = (A[2]+A[3])/2+(0,0.2);
	A[7] = (A[3]+A[4])/2+(0,0.3);
	A[8] = (A[4]+A[1])/2+(0,0.2);
	path surf,row,col;
	surf = A[1]..A[5]..A[2]--A[2]..A[6]..A[3]--A[3]..A[7]..A[4]--A[4]..A[8]..A[1]--cycle;
	ns = (A[5]+A[7])/2+(0,0.2);
	col = A[5]..ns..A[7];
	row = A[6]..ns..A[8];
	draw(surf);
	draw(row,dashed);
	draw(col,dashed);
	g1 = 1*dir(row,1);
	g2 = 0.6*dir(col,1);
	draw(ns-g1-g2--ns+g1-g2--ns+g1+g2--ns-g1+g2--cycle);
	draw(Label("$\boldsymbol{n}$",EndPoint),ns--ns+0.8*dir(90),Arrow);
	draw(Label("$\delta\boldsymbol{r}$",EndPoint),ns--ns+0.5*dir(30),red,Arrow);
	//draw(O--A[3]+(0.4,0),invisible);
\end{asy}
\caption{约束在曲面上的质点}
\label{约束在曲面上的质点}
\end{figure}

考虑被约束在曲面(曲面可移动,可变形)上运动的质点(如图\ref{约束在曲面上的质点}所示),曲面的方程表示为$f(\mbf{r},t)=0$,则$t$时刻该质点的位矢满足方程
\begin{equation*}
	f(\mbf{r},t) = 0
\end{equation*}
设质点具有虚位移$\delta \mbf{r}$,则虚位移满足
\begin{equation*}
	\frac{\pl f}{\pl \mbf{r}} \cdot \delta \mbf{r} = 0
\end{equation*}
由于$\dfrac{\pl f}{\pl \mbf{r}} \parallel \mbf{n}$,所以有$\delta \mbf{r} \perp \mbf{n}$。此时虚位移即为质点所在处曲面切平面内的无限小位移。

对于实位移则有
\begin{equation*}
	\frac{\pl f}{\pl \mbf{r}} \cdot \mathrm{d}\mbf{r} + \frac{\pl f}{\pl t} \mathrm{d} t = 0
\end{equation*}
即有
\begin{equation*}
	\frac{\pl f}{\pl \mbf{r}} \cdot \mathrm{d} \mbf{r} = - \frac{\pl f}{\pl t} \mathrm{d} t
\end{equation*}
当约束为定常约束时,$\dfrac{\pl f}{\pl t} = 0$,即有
\begin{equation*}
	\frac{\pl f}{\pl \mbf{r}} \cdot \mathrm{d} \mbf{r} = 0
\end{equation*}
此时$\mathrm{d} \mbf{r} \in \{\delta \mbf{r}\}$,实位移是全体虚位移中的一个。如果约束为非定常约束,则$\dfrac{\pl f}{\pl t} \neq 0$,所以
\begin{equation*}
	\frac{\pl f}{\pl \mbf{r}} \cdot \mathrm{d} \mbf{r} \neq 0
\end{equation*}
即$\mathrm{d} \mbf{r} \notin \{\delta \mbf{r}\}$,实位移不在虚位移集合中。
\end{example}

无穷小增量$\delta \mbf{r}_i$\footnote{实际上就是坐标$x_i,y_i$和$z_i$的无穷小增量$\delta x_i,\delta y_i$和$\delta z_i$。}是在$t=t^*$固定时,从位矢$\mbf{r}_i^*$确定的位置,变化到无限接近由位矢$\mbf{r}_i^*+\delta \mbf{r}_i$确定的位置过程中的无穷小增量,称为{\bf 等时变分}。在等时变分中我们不考虑系统的运动过程,而是比较系统在给定时刻约束允许的无限接近的构型。

前面也提到,虚位移(或者称为坐标的等时变分)满足的方程组与实位移(它是位矢的普通微分)满足的方程组形式十分相似,只是缺少了所有对应于$\mathd t$的项。因此不妨将等时变分同时也作为一种操作,当其作用在函数上时,与普通微分运算规则完全相同,但是附加一个条件:等时变分作用在时间上结果为零,即$\delta t=0$。例如对函数$f=f(\mbf{r},t)$做等时变分将得到
\begin{equation*}
	\delta f = \frac{\pl f}{\pl \mbf{r}}\cdot \delta \mbf{r}
\end{equation*}

下面考虑一些满足其他条件的等时变分。考虑两个在相同$\Delta t$内的可能位移,根据式\eqref{chapter2:可能位移}可得
\begin{equation*}
\begin{cases}
	\ds \Delta_1\mbf{r}_i = \mbf{v}_{i1}^*\Delta t+\frac12 \mbf{a}_{i1}(\Delta t)^2+\cdots \\[1.5ex]
	\ds \Delta_2\mbf{r}_i = \mbf{v}_{i2}^*\Delta t+\frac12 \mbf{a}_{i2}(\Delta t)^2+\cdots 
\end{cases}
\end{equation*}
其中可能速度$\mbf{v}_{is}^*(s=1,2)$和可能加速度$\mbf{a}_{is}^*(s=1,2)$分别满足方程组\eqref{chapter2:完整约束对质点系中质点速度的限制}、\eqref{chapter2:非完整约束对质点系中质点速度的限制}和方程组\eqref{chapter2:完整约束对质点系中质点加速度的限制}、\eqref{chapter2:非完整约束对质点系中质点加速度的限制}。将两组可能位移相减,可得
\begin{equation}
	\Delta_1\mbf{r}_i-\Delta_2\mbf{r}_i = (\mbf{v}_{i1}^*-\mbf{v}_{i2}^*)\Delta t + \frac12 (\mbf{a}_{i1}^*-\mbf{a}_{i2}^*)(\Delta t)^2+\cdots
	\label{chapter2:其他形式的等时变分}
\end{equation}

如果$\delta \mbf{v}_i = \mbf{v}_{i1}^*-\mbf{v}_{i2}^*\neq \mbf{0}$,则式\eqref{chapter2:其他形式的等时变分}中的主要部分是$\Delta t$的线性项,它等于$\delta\mbf{v}_i\Delta t$,并且根据方程组\eqref{chapter2:完整约束对质点系中质点速度的限制}、\eqref{chapter2:非完整约束对质点系中质点速度的限制}可以证明它满足虚位移方程组\eqref{chapter2:虚位移方程1}、\eqref{chapter2:虚位移方程2},即
\begin{equation}
	\delta \mbf{r}_i = \delta \mbf{v}_i\Delta t \label{chapter2:Jordan变分}
\end{equation}
是虚位移。在$\mbf{v}_{i1}\neq \mbf{v}_{i2}$的假设下,等时变分\eqref{chapter2:Jordan变分}称为{\bf Jordan变分}。

如果$\mbf{v}_{i1}^*=\mbf{v}_{i2}^*$但$\delta \mbf{a}_i = \mbf{a}_{i1}^*-\mbf{a}_{i2}^*\neq \mbf{0}$,则式\eqref{chapter2:其他形式的等时变分}中的主要部分是$\dfrac12 \delta\mbf{a}_i(\Delta t)^2$,并且根据方程组\eqref{chapter2:完整约束对质点系中质点加速度的限制}、\eqref{chapter2:非完整约束对质点系中质点加速度的限制}可以证明它满足虚位移方程组\eqref{chapter2:虚位移方程1}、\eqref{chapter2:虚位移方程2},即
\begin{equation}
	\delta \mbf{r}_i = \frac12 \delta \mbf{a}_i(\Delta t)^2 \label{chapter2:Gauss变分}
\end{equation}
是虚位移。在$\mbf{v}_{i1}=\mbf{v}_{i2}$但$\mbf{a}_{i1}^*\neq \mbf{a}_{i2}^*$的假设下,等时变分\eqref{chapter2:Gauss变分}称为{\bf Gauss变分}。

% 将$t=t^*, \mbf{r}_i=\mbf{r}_i^*$和$\mbf{v}_{is}^*(s=1,2)$分别代入方程组\eqref{chapter2:完整约束对质点系中质点速度的限制}、\eqref{chapter2:非完整约束对质点系中质点速度的限制}中,并用$s=1$的方程减去$s=2$的方程然后两端乘以$\Delta t$,可得
% \begin{subnumcases}{}
	% \sum_{i=1}^n \frac{\pl f_j}{\pl \mbf{r}_i}\cdot (\mbf{v}_{i1}^*-\mbf{v}_{i2}^*)\Delta t = 0 \quad (j=1,2,\cdots,k) \label{chapter2:Jordan变分可能速度方程1}
	% \sum_{i=1}^n \mbf{A}_{ji}\cdot (\mbf{v}_{i1}^*-\mbf{v}_{i2}^*)\Delta t = 0  \quad (j=1,2,\cdots,k')\label{chapter2:Jordan变分可能速度方程2}
% \end{subnumcases}

\subsection{广义坐标与广义坐标空间}\label{chapter2:subsection-广义坐标与位形空间}

假设质点系的约束可以用式\eqref{chapter2:完整约束的一般形式}和\eqref{chapter2:非完整约束的一般形式}来表示,而且其中的$k$个函数$f_j(j=1,2,\cdots,k)$之间是相互独立的。

能够确定系统位形的参数的最小数目称为{\bf 独立广义坐标数},可以唯一确定系统位形的一组参数则称为{\bf 广义坐标}。由于函数$f_j(j=1,2,\cdots,k)$之间是相互独立的,所以此时独立广义坐标数$s=3n-k$。广义坐标可以从各质点的$3n$个笛卡尔坐标$x_i,y_i,z_k$中选取$s$个,使得其他坐标可以利用方程\eqref{chapter2:完整约束的一般形式}解出。但一般来说,广义坐标没有必要取为笛卡尔坐标,可以选取任意$s$个独立的可以确定系统位形的量$q_\alpha(\alpha=1,2,\cdots,s)$,它们可以是距离、角度、面积,也可以是没有直接几何意义的,只需要它们互相之间相互独立,并且可以将各质点的位矢$\mbf{r}_i$用$\mbf{q}$\footnote{此处用$\mbf{q}$来{\it 形式上}表示由所有广义坐标作为分量的一个向量,其各个分量的量纲不必相同,因此任何试图计算$\mbf{q}$内积的运算(例如求其长度)都是不合法的,后文不再解释。}和时间$t$表示出来:
\begin{equation}
	\mbf{r}_i = \mbf{r}_i(\mbf{q},t)\quad (i=1,2,\cdots,n)
	\label{chapter2:广义坐标变换方程}
\end{equation}
式\eqref{chapter2:广义坐标变换方程}称为{\bf 广义坐标变换方程},这些变换将使得式\eqref{chapter2:完整约束的一般形式}所表示的$k$个完整约束自动满足,而且由于$q_\alpha(\alpha=1,2,\cdots,s)$之间互相独立,所以矩阵
\begin{equation}
	\begin{pmatrix}
		\dfrac{\pl \mbf{r}_1}{\pl q_1} & \cdots & \dfrac{\pl \mbf{r}_1}{\pl q_s} \\
		\vdots & & \vdots \\
		\dfrac{\pl \mbf{r}_n}{\pl q_1} & \cdots & \dfrac{\pl \mbf{r}_n}{\pl q_s}
	\end{pmatrix}
	\label{chapter2:坐标变换方程的Jacobi矩阵}
\end{equation}
是满秩的。

我们假设选择广义坐标$\mbf{q}$使得系统的任意位形都能在$\mbf{q}$取某个值的时候利用坐标变换方程\eqref{chapter2:广义坐标变换方程}得到。如果不能得到所有可能的位置,则可以局部地引入广义坐标,对于不同的可能位置集引入不同的广义坐标。并且假设坐标变换方程\eqref{chapter2:广义坐标变换方程}右端的函数对其所有自变量都是二阶连续可微的。另外,如果系统是定常的,则总可以通过选择广义坐标使得坐标变换方程\eqref{chapter2:广义坐标变换方程}中同样不显含时间。

在实际操作时,通常完全不需要首先建立约束方程\eqref{chapter2:完整约束的一般形式}\footnote{尤其是几何约束。},而根据问题的物理意义可以知道需要确定系统可能位形所必须的广义坐标的数量,并利用几何关系直接设出广义坐标。

对于每个时刻$t$,系统的可能位形与$s$维抽象空间$(q_1,q_2,\cdots,q_s)$之间一一对应,这个$s$维抽象空间$(q_1,q_2,\cdots,q_s)$称为{\bf 广义坐标空间}或{\bf 广义位形空间}。系统的每个可能位形都对应于广义坐标空间的某个点,该点称为{\bf 映射点}。系统的运动即对应于映射点在广义坐标空间中的运动。

\begin{example}[单摆]\label{chapter2:example-单摆}
由不可伸长、不计质量的细绳悬挂于固定点的一个质点在竖直平面内的运动即构成一个单摆(数学摆),如图\ref{chapter2:figure-单摆和单摆的广义坐标空间}所示。单摆有一个几何约束,可选择摆角$\theta$作为单摆的广义坐标,由此单摆的位形由摆角$\theta$唯一确定。此时单摆的坐标变换方程为
\begin{equation*}
\begin{cases}
	x = l\cos\theta \\
	y = l\sin\theta
\end{cases}
\end{equation*}

由于数轴上不同的点$\theta$和$\theta+2k\pi(k=\pm 1,\pm 2,\cdots)$对应着单摆的同一个位形,因此单摆的位形和整个数轴不是一一对应。但我们可以在数轴上划分出一个半开区间$0\leqslant\theta<2\pi$\footnote{任意长度为$2\pi$的半开区间都可以。},即可实现与单摆位置的一一对应。

但上面的做法又破坏了连续性,因为单摆的两个临近位置$\theta=0+0$和$\theta=2\pi-0$不是相邻点。为了保持连续性,需要认为$\theta=0$和$\theta=2\pi$是同一点\footnote{这种条件又称为“周期性边界条件”。}。直观上可以这样操作:将区间的$\theta=0$和$\theta=2\pi$“粘接”起来,得到的几何形状——单位圆周$S^1$就是单摆的坐标空间。
\begin{figure}[htb]
\centering
\begin{asy}
	size(300);
	//单摆和单摆的广义坐标空间
	pair O,x,y,r;
	O = (0,0);
	x = (4,0);
	y = (0,-4.5);
	r = (3,-4);
	draw(Label("$x$",EndPoint),O--x,Arrow);
	draw(Label("$y$",EndPoint),O--y,Arrow);
	label("$O$",O,W);
	draw(Label("$l$",MidPoint,Relative(W)),O--r);
	fill(shift(r)*scale(0.1)*unitcircle,black);
	real R=0.4;
	draw(Label("$\theta$",MidPoint,Relative(E)),arc(O,R,-90,degrees(r)));
	picture pic;
	real l = 2;
	real theta = 1.5;
	draw(pic,Label("$\theta$",EndPoint),(-l,0)--(l,0),Arrow);
	draw(pic,(-theta,0)--(theta,0),linewidth(0.8bp));
	real r = 1.5;
	pair C = (0,-2.75);
	draw(pic,Label("$\theta$",Relative(0.095),Relative(E)),circle(C,r),linewidth(0.8bp));
	add(pic,arrow(circle(C,r),invisible,FillDraw(black),Relative(0.1)));
	real b = 0.05;
	draw(pic,Label("$0$",EndPoint),(-theta,b)--(-theta,-b),linewidth(0.8bp));
	draw(pic,Label("$2\pi$",EndPoint),(theta,b)--(theta,-b),linewidth(0.8bp));
	pair P = C+r*dir(0);
	draw(pic,Label("$0$",EndPoint),P+(-b,0)--P+(b,0),linewidth(0.8bp));
	add(shift(7,0)*pic);
\end{asy}
\caption{单摆系统和单摆的广义坐标空间}
\label{chapter2:figure-单摆和单摆的广义坐标空间}
\end{figure}
\end{example}

\begin{example}[双摆]
在一个单摆的质点上再悬挂另外一个单摆,并限制两个单摆都只能在竖直平面内运动,由此构成的系统称为双摆系统。双摆系统有两个几何约束,因此可选择两个摆与竖直方向的夹角$\theta_1,\theta_2$作为广义坐标。

\begin{figure}[htb]
\centering
\begin{asy}
	size(300);
	//双摆系统和双摆的广义坐标空间
	pair O,x,y,P1,P2;
	real theta1,theta2,l1,l2,r;
	O = (0,0);
	x = (4,0);
	y = (0,-4);
	l1 = 2;
	l2 = 3;
	theta1 = 30;
	theta2 = 50;
	draw(Label("$x$",EndPoint),O--x,Arrow);
	draw(Label("$y$",EndPoint),O--y,Arrow);
	P1 = l1*dir(theta1-90);
	P2 = P1 + l2*dir(theta2-90);
	r = 0.1;
	fill(shift(P1)*scale(r)*unitcircle,black);
	fill(shift(P2)*scale(r)*unitcircle,black);
	draw(Label("$l_1$",MidPoint,Relative(W)),O--P1);
	draw(Label("$l_2$",MidPoint,Relative(W)),P1--P2);
	draw(P1--P1+1.25*dir(-90),dashed);
	real R = 0.5;
	draw(Label("$\theta_1$",MidPoint,Relative(E)),arc(O,R,-90,theta1-90));
	R = 0.4;
	draw(Label("$\theta_2$",MidPoint,Relative(E)),arc(P1,R,-90,theta2-90));
	label("$(x_1,y_1)$",P1,NE);
	label("$(x_2,y_2)$",P2+(0.1,0),E);
	picture pic;
	real x,y;
	x = 3.75;
	y = 3.75;
	draw(pic,Label("$\theta_1$",EndPoint),O--(x,0),Arrow);
	draw(pic,Label("$\theta_2$",EndPoint),O--(0,y),Arrow);
	label(pic,"$O$",O,SW);
	real theta = 2.75;
	filldraw(pic,box(O,(theta,theta)),0.75white,linewidth(0.8bp));
	label(pic,"$2\pi$",(theta,0),S);
	label(pic,"$2\pi$",(0,theta),W);
	add(shift(5.5,-4)*pic);
\end{asy}
\caption{双摆系统和双摆的广义坐标空间}
\label{chapter2:figure-双摆系统和双摆的广义坐标空间}
\end{figure}

基于与单摆中类似的原因,广义坐标$\theta_1$和$\theta_2$的取值应该限制在$0\leqslant \theta_1<2\pi, 0\leqslant \theta_2<2\pi$上\footnote{取法不唯一。},并认为$\theta_1=0$的线段和$\theta_1=2\pi$的线段是同一条线段,$\theta_2=0$的线段和$\theta_2=2\pi$的线段也是同一条线段,即得到双摆系统的广义坐标空间。直观上可以这样做:将正方形的对边“粘接”在一起,第一次粘接得到圆柱,第二次粘接则得到圆环面$S^1\times S^1$,如图\ref{chapter2:figure-双摆系统和双摆的广义坐标空间}。
\end{example}

\subsection{广义速度、广义加速度与广义虚位移}

广义坐标的导数$\dot{q}_\alpha(\alpha=1,2,\cdots,s)$称为{\bf 广义速度},广义速度的导数$\ddot{q}_\alpha(\alpha=1,2,\cdots,s)$称为{\bf 广义加速度}。利用广义速度$\dot{q}_\alpha$和广义加速度$\ddot{q}_\alpha$可以表示出系统中各个质点的速度和加速度:
\begin{align}
	 & \mbf{v}_i = \dot{\mbf{r}}_i = \sum_{\alpha=1}^s \frac{\pl \mbf{r}_i}{\pl q_\alpha}\dot{q}_\alpha+\frac{\pl \mbf{r}_i}{\pl t} \label{chapter2:各质点速度与广义速度的关系} \\
	 & \mbf{a}_i = \ddot{\mbf{r}}_i = \sum_{\alpha=1}^s \frac{\pl \mbf{r}_i}{\pl q_\alpha}\ddot{q}_\alpha + \sum_{\alpha,\beta=1}^s \frac{\pl^2 \mbf{r}_i}{\pl q_\alpha\pl q_\beta}\dot{q}_\alpha\dot{q}_\beta + 2\sum_{\alpha=1}^s\frac{\pl^2 \mbf{r}_i}{\pl q_\alpha\pl t} \dot{q}_\alpha + \frac{\pl^2 \mbf{r}_i}{\pl t^2} \label{chapter2:各质点加速度与广义速度和广义加速度的关系}
\end{align}
式中$i=1,2,\cdots,n$。将$\dot{\mbf{r}}_i$看作是广义坐标和广义速度的函数,即$\dot{\mbf{r}}_i=\dot{\mbf{r}}_i(\mbf{q},\dot{\mbf{q}},t)$,则可由式\eqref{chapter2:各质点速度与广义速度的关系}导出两个十分重要的关系式,称为{\bf 经典Lagrange关系}:
\begin{equation}
	\frac{\pl \dot{\mbf{r}}_i}{\pl \dot{q}_\alpha} = \frac{\pl \mbf{r}_i}{\pl q_\alpha}, \quad \frac{\mathd}{\mathd t}\frac{\pl \mbf{r}_i}{\pl q_\alpha} = \frac{\pl \dot{\mbf{r}}_i}{\pl q_\alpha}\quad (\alpha=1,2,\cdots,s)
	\label{chapter2:经典Lagrange关系}
\end{equation}
式\eqref{chapter2:经典Lagrange关系}中的第一个式子可以通过\eqref{chapter2:各质点速度与广义速度的关系}直接求偏导数得到。下面推导式\eqref{chapter2:经典Lagrange关系}中的另一式。注意到$\dfrac{\pl \mbf{r}_i}{\pl q_\alpha}=\dfrac{\pl \mbf{r}_i}{\pl q_\alpha}(\mbf{q},t)$与广义速度无关,所以有
\begin{equation*}
	\frac{\mathd}{\mathd t}\frac{\pl \mbf{r}_i}{\pl q_\alpha} = \sum_{\beta=1}^s \frac{\pl^2 \mbf{r}_i}{\pl q_\alpha\pl q_\beta}\dot{q}_\beta + \frac{\pl^2 \mbf{r}_i}{\pl q_\alpha\pl t} = \frac{\pl}{\pl q_\alpha} \left(\sum_{\beta=1}^s \frac{\pl \mbf{r}_i}{\pl q_\beta}\dot{q}_\beta + \frac{\pl \mbf{r}_i}{\pl t}\right) = \frac{\pl \dot{\mbf{r}}_i}{\pl q_\alpha}
\end{equation*}

在上面的推导中,完全没有用到任何关于函数$\mbf{r}=\mbf{r}(\mbf{q},t)$的具体性质,因此将其中的$\mbf{r}$替换为任何关于广义坐标$\mbf{q}$和时间$t$的函数,经典Lagrange关系都成立,即对于任意函数$f=f(\mbf{q},t)$都有
\begin{equation}
	\frac{\pl \dot{f}}{\pl \dot{q}_\alpha} = \frac{\pl f}{\pl q_\alpha}, \quad \frac{\mathd}{\mathd t}\left(\frac{\pl f}{\pl q_\alpha}\right) = \frac{\pl \dot{f}}{\pl q_\alpha}\quad (\alpha=1,2,\cdots,s)
	\label{chapter2:推广经典Lagrange关系}
\end{equation}
式\eqref{chapter2:推广经典Lagrange关系}表示的关系称为{\bf 推广经典Lagrange关系}。

下面将非完整约束方程\eqref{chapter2:非完整约束的一般形式}写成广义坐标的形式。将坐标变换方程\eqref{chapter2:广义坐标变换方程}和式\eqref{chapter2:各质点速度与广义速度的关系}代入非完整约束方程\eqref{chapter2:非完整约束的一般形式}中,可得
\begin{equation}
	\sum_{\alpha=1}^s B_{j\alpha}(\mbf{q},t) \dot{q}_\alpha + B_{j0}(\mbf{q},t) = 0\quad (j=1,2,\cdots,k')
	\label{chapter2:用广义坐标表示的非完整约束}
\end{equation}
其中的系数$B_{j\alpha}$和$B_{j0}$为
\begin{equation}
	B_{j\alpha}(\mbf{q},t) = \sum_{i=1}^n \mbf{A}_{ji}\cdot \frac{\pl \mbf{r}_i}{\pl q_\alpha},\quad B_{j0}(\mbf{q},t) = \sum_{i=1}^n \mbf{A}_{ji}\cdot \frac{\pl \mbf{r}_i}{\pl t}+A_{j0}
	\label{chapter2:用广义坐标表示的非完整约束-系数}
\end{equation}
对于完整系统,广义速度$\dot{q}_\alpha$之间相互独立且可任意取值。对于非完整系统,广义坐标与完整系统一样可以任意取值,但广义速度不是独立的,它们受$k'$个关系式\eqref{chapter2:用广义坐标表示的非完整约束}限制。

广义坐标的等时变分$\delta q_\alpha(\alpha=1,2,\cdots,s)$称为{\bf 广义虚位移},根据第\ref{chapter2:subsection-实位移与虚位移}节所述的等时变分运算规则可得
\begin{equation}
	\delta \mbf{r}_i = \sum_{\alpha=1}^s\frac{\pl \mbf{r}_i}{\pl q_\alpha}\delta q_\alpha \quad (i=1,2,\cdots,n)
	\label{chapter2:虚位移与广义虚位移之间的关系}
\end{equation}
对于完整系统,广义虚位移$\delta q_\alpha$之间相互独立,可以任意取值。对于非完整系统,将式\eqref{chapter2:用广义坐标表示的非完整约束}两端乘以$\mathd t$,将$\mathd q_\alpha$替换为$\delta q_\alpha$,并去掉含有$\mathd t$的项\footnote{可以如此处理是由于等时变分的运算规则与普通微分完全相同,但对时间结果为零。},可得
\begin{equation}
	\sum_{\alpha=1}^s B_{j\alpha}(\mbf{q},t) \delta q_\alpha=0 \quad (j=1,2,\cdots,k')
	\label{chapter2:非完整约束的虚位移表示}
\end{equation}
由此可知,非完整系统中可以独立取值的广义虚位移数要比完整系统少$k'$个。

\section{约束反力与理想约束}

\subsection{主动力与约束反力}\label{chapter2:subsection-主动力与约束反力}

在一个系统中,设$m_i$是其中第$i$个质点的质量,$\mbf{r}_i$为其位矢,如果系统是自由的,则第$i$个质点的加速度由Newton第二定律确定,即$m_i\ddot{\mbf{r}}_i=\mbf{F}_i$,其中$\mbf{F}_i$为作用在第$i$个质点上的合力。

如果系统不是自由的,则系统中每个质点的加速度都将受到约束的限制,此时,一般来说,加速度$\ddot{\mbf{r}}_i=\dfrac{\mbf{F}_i}{m_i}$将不能满足可能加速度的方程\eqref{chapter2:完整约束对质点系中质点加速度的限制}和\eqref{chapter2:非完整约束对质点系中质点加速度的限制},即非自由系统中第$i$个质点的加速度$\mbf{a}_i$不同于自由系统情况下的加速度$\ddot{\mbf{r}}_i$,因此约束导致出现了附加的加速度$\mbf{a}_i-\ddot{\mbf{r}}_i$。Newton第二定律表明,质点的任何加速度都是由其上作用的某些力而产生的,这个产生附加加速度的力是由于约束的存在才出现的,称为{\bf 约束反力}。为了不混淆约束反力和作用在非自由系统上的其他力,将其他力称为{\bf 主动力}。前面所述的$\mbf{F}_i$实际上是质点$i$所有主动力的合力。

主动力也可以称为{\bf 给定力},如果约束瞬间消失,这些力仍然保持作用在系统上。而约束反力有时也称为{\bf 被动力},它们不是事先已知的,不仅依赖于实现约束的物理机制,而且依赖于主动力和系统的运动。

如果用$\mbf{R}_i$表示作用在质点$m_i$上的约束反力的合力,则可得到系统的运动方程
\begin{equation}
	m_i\mbf{a}_i = \mbf{F}_i+\mbf{R}_i \quad (i=1,2,\cdots,n)
	\label{chapter2:非自由系统中质点的Newton第二定律方程}
\end{equation}
这个方程表明,在动力学中,非自由系统可以看作主动力和约束反力共同作用下的自由系统。

\subsection{理想约束}

如果约束反力在任意虚位移上所做的功都等于零,即
\begin{equation}
	\sum_{i=1}^n\mbf{R}_i\cdot \delta\mbf{r}_i = 0
\end{equation}
则该约束称为{\bf 理想约束}。理想约束的条件不是由约束方程得到的,是附加条件。

下列几种常见的约束都是理想约束。
\begin{enumerate}
\item 质点沿光滑(固定或运动的)曲面运动:如图\ref{质点沿光滑曲面运动约束}所示。此时约束反力$\mbf{R}$沿曲面在该点的法线方向,而虚位移$\delta \mbf{r}$则在切平面内,即有
\begin{equation*}
	\mbf{R} \cdot \delta \mbf{r} = 0
\end{equation*}
\item 刚性联结的两质点(包括刚体\footnote{刚体是一种理想模型,指任何情况下都不会发生形变的物体。}中任意两点):此时约束反力$\mbf{R}_1$和$\mbf{R}_2$沿联结方向,根据Newton第三定律可有$\mbf{R}_2=-\mbf{R}_1$。考虑到
\begin{equation*}
	\mbf{r}_{12} \cdot \mbf{r}_{12} = l^2
\end{equation*}
所以
\begin{equation*}
	2\mbf{r}_{12} \cdot \delta \mbf{r}_{12} = 0
\end{equation*}
即有
\begin{equation*}
	\mbf{R}_1 \cdot \delta \mbf{r}_1 + \mbf{R}_2 \cdot \delta \mbf{r}_2 = \mbf{R}_1 \cdot (\delta \mbf{r}_1 - \delta \mbf{r}_2) = \mbf{R}_1 \cdot \delta \mbf{r}_{12} = 0
\end{equation*}
此处考虑到了$\mbf{R}_1 \parallel \mbf{r}_{12}$。
\begin{figure}[htb]
\centering
\begin{minipage}[t]{0.45\textwidth}
\centering
\begin{asy}
	size(175);
	//质点沿光滑曲面运动约束
	pair O,A[],vec,P,tan;
	path surf;
	O = (0,0);
	A[1] = (-0.7,-1);
	A[2] = O;
	A[3] = (1.2,0.6);
	surf = A[1]..A[2]..A[3];
	draw(surf,linewidth(0.8bp));
	vec = 0.5*(0.1,-0.06);
	int imax=30;
	for(int i=1;i<imax;i=i+1){
		P = relpoint(surf,1./imax*i);
		draw(P--P+vec);
	}
	label("$P$",O+vec,SE);
	tan = 0.9*dir(surf,1);
	draw(Label("$\delta\boldsymbol{r}$",EndPoint),O--O+tan,Arrow);
	draw(Label("$\boldsymbol{R}$",EndPoint),O--O+rotate(90)*tan,Arrow);
	dot(O);
\end{asy}
\caption{质点沿光滑曲面运动}
\label{质点沿光滑曲面运动约束}
\end{minipage}
\hspace{1cm}
\begin{minipage}[t]{0.45\textwidth}
\centering
\begin{asy}
	size(175);
	//定点运动的刚体
	pair A[];
	A[0] = (0,0);
	A[1] = (1,1);
	A[2] = (1,2);
	A[3] = (0,3);
	A[4] = (-1,3);
	A[5] = (-1.3,2);
	A[6] = (-0.9,1);
	guide p;
	for(pair P:A){
		p = p..P;
	}
	p = p..cycle;
	draw(p,linewidth(0.8bp));
	real l = 0.3;
	draw((0,0)--l*dir(-120)--l*dir(-60)--cycle);
	dot("$O$",(0,0),WSW);
	picture dashpic;
	path p = (-10,0)--(10,0);
	draw(dashpic,p,linewidth(0.8bp));
	pair dash = dir(-135);
	real h = 0.1;
	for(real r=0;r<=1;r=r+0.005){
		pair P = relpoint(p,r);
		draw(dashpic,P--P+dash);
	}
	clip(dashpic,box((-l,0.1),(l,-h)));
	add(shift(0,-l*Sin(60))*dashpic);
	draw(Label("$\boldsymbol{R}$",EndPoint),(0,0)--1.2*dir(55),Arrow);
\end{asy}
\caption{定点运动的刚体}
\label{定点运动的刚体约束}
\end{minipage}
\end{figure}

\item 定点运动的刚体:如图\ref{定点运动的刚体约束}所示。因为$\delta\mbf{r}=0$(因为约束反力$\mbf{R}$的作用点不运动),因此有
\begin{equation*}
	\mbf{R} \cdot \delta \mbf{r} = 0
\end{equation*}
\item 定轴转动的刚体:定轴转动相当于刚体上有两个固定点,因此必然有$\mbf{R}_1 \cdot \delta \mbf{r}_1+\mbf{R}_2 \cdot \delta \mbf{r}_2 = 0$。
\item 两个刚体用铰链连接于一点:如图\ref{两个刚体用铰链连接于一点约束}所示。因为$\mbf{R}_1=-\mbf{R}_2$,$\delta\mbf{r}_1=\delta\mbf{r}_2$,所以
\begin{equation*}
	\mbf{R}_1 \cdot \delta \mbf{r}_1+\mbf{R}_2 \cdot \delta \mbf{r}_2 = \mbf{R}_1\cdot (\delta \mbf{r}_1-\delta\mbf{r}_2) = 0
\end{equation*}
\item 两个刚体以光滑表面相切:如图\ref{两个刚体以光滑表面相切约束}所示。两个刚体的切点之间的相对速度位于公共切平面内,因此$\delta\mbf{r}_1-\delta\mbf{r}_2$也在切平面内。而且约束反力$\mbf{R}_1$和$\mbf{R}_2$都垂直于这个切平面,并且$\mbf{R}_1=-\mbf{R}_2$,所以
\begin{equation*}
	\mbf{R}_1 \cdot \delta \mbf{r}_1+\mbf{R}_2 \cdot \delta \mbf{r}_2 = \mbf{R}_1\cdot (\delta \mbf{r}_1-\delta\mbf{r}_2) = 0
\end{equation*}

\begin{figure}[htb]
\centering
\begin{minipage}[t]{0.45\textwidth}
\centering
\begin{asy}
	size(175);
	//两个刚体用铰链连接于一点
	pair O,A[],B[],P,Q,B[];
	O = (0,0);
	A[0] = (0,0);
	A[1] = (1,1);
	A[2] = (1.3,2);
	A[3] = (0,3);
	A[4] = (-1.5,2);
	A[5] = (-1,0.5);
	path bodyA,bodyB;
	bodyA = rotate(40)*yscale(0.8)*(A[0]..A[1]..A[2]..A[3]..A[4]..A[5]..A[0]);
	P = A[0];
	B[0] = (0,0);
	B[1] = (0.56,-0.2);
	B[2] = (1,-1);
	B[3] = (0.5,-1.5);
	B[4] = (-0.7,-0.8);
	B[5] = (-0.2,-0.25);
	bodyB = rotate(45)*(B[0]..B[1]..B[2]..B[3]..B[4]..B[5]..B[0]);
	draw(bodyA,linewidth(0.8bp));
	draw(bodyB,linewidth(0.8bp));
	pair R = dir(-30);
	draw(Label("$\boldsymbol{R}_1$",EndPoint),P--P-R,Arrow);
	draw(Label("$\boldsymbol{R}_2$",EndPoint),P--P+R,Arrow);
	dot("$O$",P,1.5*dir(-110),UnFill);
\end{asy}
\caption{两个刚体用铰链连接于一点}
\label{两个刚体用铰链连接于一点约束}
\end{minipage}
\hspace{1cm}
\begin{minipage}[t]{0.45\textwidth}
\centering
\begin{asy}
	size(175);
	//两个刚体以光滑表面相切
	pair O,A[],B[],vec,P,Q,tan;
	O = (0,0);
	A[1] = (0.2,0);
	A[2] = (0.3,0.1);
	A[3] = (0.5,0.6);
	A[4] = (1,0.8);
	A[5] = (1.7,0.3);
	A[6] = (1.65,0);
	path bodyA,bodyB;
	bodyA = A[1]..A[2]..A[3]..A[4]..A[5]..A[6];
	draw(bodyA,linewidth(0.8bp));
	real l = 2;
	real h = 0.07;
	picture pic;
	pair dash = dir(-135);
	path p = (-1,0)--(l+1,0);
	draw(pic,p,linewidth(0.8bp));
	for(real r=0;r<=1;r=r+0.0175){
		pair P = relpoint(p,r);
		draw(pic,P--P+dash);
	}
	clip(pic,box((0,0.1),(l,-h)));
	add(pic);
	P = relpoint(bodyA,0.65);
	tan = 0.5*reldir(bodyA,0.65);
	Q = P+0.7*(0.2,0.5);
	label("$P$",P,SW);
	draw(circle(Q,length(Q-P)),linewidth(0.8bp));
	draw(Label("$\delta \boldsymbol{r}_1-\delta \boldsymbol{r}_2$",EndPoint),P--P-tan,Arrow);
	draw(Label("$\boldsymbol{R}_1$",EndPoint),P--P-rotate(90)*tan,Arrow);
	draw(Label("$\boldsymbol{R}_2$",EndPoint),P--P+rotate(90)*tan,Arrow);
\end{asy}
\caption{两个刚体以光滑表面相切}
\label{两个刚体以光滑表面相切约束}
\end{minipage}\\

\begin{minipage}[t]{0.45\textwidth}
\centering
\begin{asy}
	size(175);
	//两个刚体以粗糙表面相切
	pair O,O1,A[],B[],vec,P,Q,tan;
	O = (0,0);
	O1 = (2,0);
	A[1] = (0.2,0);
	A[2] = (0.3,0.1);
	A[3] = (0.5,0.6);
	A[4] = (1,0.8);
	A[5] = (1.7,0.3);
	A[6] = (1.65,0);
	path bodyA,bodyB;
	bodyA = A[1]..A[2]..A[3]..A[4]..A[5]..A[6];
	draw(bodyA,linewidth(0.8bp));
	draw(O--O1);
	int imax = 25;
	vec = 0.35*(-0.1,-0.2);
	for(int i=1;i<imax;i=i+1){
		P = interp(O,O1,1./imax*i);
		draw(P--P+vec);
	}
	P = relpoint(bodyA,0.65);
	tan = 0.5*reldir(bodyA,0.65);
	Q = P+0.7*(0.2,0.5);
	label("$P$",P,SW);
	draw(shift(Q)*scale(length(Q-P))*unitcircle,linewidth(0.8bp));
	draw(Label("$\delta \boldsymbol{r}_1$",EndPoint,NNE),P--P+tan,Arrow);
	draw(Label("$\delta \boldsymbol{r}_2$",EndPoint,S),P--P+tan,Arrow);
	draw(Label("$\boldsymbol{R}_1$",EndPoint),P--P-0.5*dir(80),Arrow);
	draw(Label("$\boldsymbol{R}_2$",EndPoint),P--P+0.5*dir(80),Arrow);
\end{asy}
\caption{两个刚体以粗糙表面相切}
\label{两个刚体以粗糙表面相切约束}
\end{minipage}
\hspace{1cm}
\begin{minipage}[t]{0.45\textwidth}
\centering
\begin{asy}
	size(175);
	//两个质点以柔软而不可伸长的绳子相联结约束
	pair O,P1,P2,tan1,tan2;
	real r,l1,l2;
	path cyl;
	O = (0,0);
	r = 0.08;
	cyl = shift(O)*scale(r)*unitcircle;
	draw(cyl);
	P1 = relpoint(cyl,0.45);
	tan1 = reldir(cyl,0.45);
	P2 = relpoint(cyl,0.22);
	tan2 = -reldir(cyl,0.22);
	l1 = 1.5;
	l2 = 1.8;
	draw(P1--P1+l1*tan1);
	draw(P2--P2+l2*tan2);
	draw(Label("$\boldsymbol{R}_1$",EndPoint,E),P1+l1*tan1--P1+(l1-0.5)*tan1,Arrow);
	draw(Label("$\delta \boldsymbol{r}_1^\parallel$",EndPoint),P1+l1*tan1--P1+(l1+0.5)*tan1,Arrow);
	draw(Label("$\delta \boldsymbol{r}_1^\perp$",EndPoint),P1+l1*tan1--P1+l1*tan1+rotate(90)*0.5*tan1,Arrow);
	draw(Label("$\boldsymbol{R}_2$",EndPoint,N),P2+l2*tan2--P2+(l2-0.5)*tan2,Arrow);
	draw(Label("$\delta \boldsymbol{r}_2^\parallel$",EndPoint),P2+l2*tan2--P2+(l2+0.5)*tan2,Arrow);
	draw(Label("$\delta \boldsymbol{r}_2^\perp$",EndPoint),P2+l2*tan2--P2+l2*tan2+rotate(90)*0.5*tan2,Arrow);
	dot(P1+l1*tan1);
	dot(P2+l2*tan2);
\end{asy}
\caption{两个质点以柔软而不可伸长的绳子相联结}
\label{两个质点以柔软而不可伸长的绳子相联结约束}
\end{minipage}
\end{figure}

\item 两个刚体以粗糙表面相切(不滑动,只能作纯滚动):如图\ref{两个刚体以粗糙表面相切约束}所示。此时接触点上的约束反力满足$\mbf{R}_2 =- \mbf{R}_1$,相对速度$\mbf{v}_2-\mbf{v}_1= \mbf{0}$,即有$\delta \mbf{r}_2-\delta \mbf{r}_1 = \mbf{0}$,因此
\begin{equation*}
	\mbf{R}_1 \cdot \delta \mbf{r}_1 + \mbf{R}_2 \cdot \delta \mbf{r}_2 = \mbf{R}_1 \cdot (\delta \mbf{r}_1 - \delta \mbf{r}_2) = 0
\end{equation*}
\item 两个质点以柔软而不可伸长的绳子相联结:如图\ref{两个质点以柔软而不可伸长的绳子相联结约束}所示。此时约束反力$\mbf{R}_1$和$\mbf{R}_2$都沿相应质点处的绳子方向,且大小相等记作$T$。将两个质点的虚位移$\delta \mbf{r}_1$和$\delta \mbf{r}_2$按沿绳和正交于绳的方向进行分解,将沿绳方向的虚位移记作$\delta \mbf{r}_1^\parallel = \delta l_1\mbf{e}_1$和$\delta \mbf{r}_2^\parallel = \delta l_2 \mbf{e}_2$。正交于绳方向的虚位移对应的虚功自然为零,考虑沿绳方向虚位移对应的虚功
\begin{equation*}
	\mbf{R}_1 \cdot \delta \mbf{r}_1 + \mbf{R}_2 \cdot \delta \mbf{r}_2 = T\delta l_1 + T\delta l_2
\end{equation*}
由于绳不可伸长,所以有$\delta l_1 + \delta l_2 = 0$,故有$\mbf{R}_1 \cdot \delta \mbf{r}_1 + \mbf{R}_2 \cdot \delta \mbf{r}_2 = \mbf{0}$。
\end{enumerate}

很多机构可以看作是上面提到的几种简单“零件”的组合。但是实际上不存在绝对光滑和绝对粗糙的曲面,也不存在绝对刚体和不可伸长的绳,因此实际问题中约束反力的功不等于零,通常这些功很小,可以在允许的近似意义下认为等于零。

所有约束皆为理想约束的系统称为{\heiti 理想系统},即理想系统需满足
\begin{equation*}
	\sum_{i=1}^n \mbf{R}_i \cdot \delta \mbf{r}_i = 0
\end{equation*}

很多情况下约束不能看作理想的,例如,刚体在粗糙的表面上进行带有滑动的滚动,这时仍然可以将约束看作理想的,而将摩擦力看作未知的主动力。新的未知数的出现要求附加新的由实验给出的经验定律。

以后将主要讨论理想系统。提出的问题一般可以表述为:给定质点系统和约束,已知每个质点的质量$m_i$和作用在它们上的主动力$\mbf{F}_i$,并且给定每个质点的初位置$\mbf{r}_{i0}$和初速度$\mbf{v}_{i0}$,求作为时间函数的位矢$\mbf{r}_i(t)$和约束反力$\mbf{R}_i(t)$。这要求$6n$个未知数。