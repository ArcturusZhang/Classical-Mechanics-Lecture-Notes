\chapter{正则变换与Hamilton-Jacobi方程}

\section{正则变换}

%对称性越高,方程的求解越简便。对称性可以通过变换发现。

\subsection{点变换}

正则方程的对称性越高,方程的求解也就越方便。有很多时候,同一个问题,广义坐标选取的方式不同,系统的对称性也不同,求解的难易程度也不同。这一点可以用循环坐标来举例说明。例如在研究两体相对运动时,如果取直角坐标作为广义坐标,则Lagrange函数或Hamilton函数中没有循环坐标,但如果变换到平面极坐标,则极角$\theta$是循环变量。

广义坐标之间的变换叫做{\bf 点变换}。设在一组广义坐标$\mbf{q}$下,Lagrange函数形式为
\begin{equation*}
	L = L_q(\mbf{q},\dot{\mbf{q}},t)
\end{equation*}
系统方程为
\begin{equation}
	\frac{\mathrm{d}}{\mathrm{d} t} \frac{\pl L}{\pl \dot{q}_\alpha} - \frac{\pl L}{\pl q_\alpha} = 0 \quad (\alpha = 1,2,\cdots,s)
\end{equation}
做广义坐标变换
\begin{equation}
	\mbf{q} = \mbf{q}(\mbf{Q},t)
	\label{chapter10:点变换}
\end{equation}
则有
\begin{equation}
	L = L_Q(\mbf{Q},\dot{\mbf{Q}},t) = L_q(\mbf{q}(\mbf{Q},t),\dot{\mbf{q}}(\mbf{Q},\dot{\mbf{Q}},t),t)
\end{equation}
根据链式法则,可以得到
\begin{align}
	\frac{\pl L}{\pl \dot{Q}_\alpha} & = \sum_{\beta=1}^s \frac{\pl L}{\pl \dot{q}_\beta} \frac{\pl \dot{q}_\beta}{\pl \dot{Q}_\alpha} \\
	\frac{\pl L}{\pl Q_\alpha} & = \sum_{\beta=1}^s \frac{\pl L}{\pl q_\beta} \frac{\pl q_\beta}{\pl Q_\alpha} + \sum_{\alpha=1}^s \frac{\pl L}{\pl \dot{q}_\beta} \frac{\pl \dot{q}_\beta}{\pl Q_\alpha} 
\end{align}
由此可得
\begin{align*}
	\frac{\mathrm{d}}{\mathrm{d} t} \frac{\pl L}{\pl \dot{Q}_\alpha} - \frac{\pl L}{\pl Q_\alpha} & = \frac{\mathrm{d}}{\mathrm{d} t} \left(\sum_{\beta=1}^s \frac{\pl L}{\pl \dot{q}_\beta} \frac{\pl \dot{q}_\beta}{\pl \dot{Q}_\alpha}\right) - \sum_{\beta=1}^s \frac{\pl L}{\pl q_\beta} \frac{\pl q_\beta}{\pl Q_\alpha} - \sum_{\alpha=1}^s \frac{\pl L}{\pl \dot{q}_\beta} \frac{\pl \dot{q}_\beta}{\pl Q_\alpha} \\
	& = \sum_{\alpha=1}^s \left[\frac{\mathrm{d}}{\mathrm{d} t} \left(\frac{\pl L}{\pl \dot{q}_\beta}\right) \frac{\pl \dot{q}_\beta}{\pl \dot{Q}_\alpha} - \frac{\pl L}{\pl q_\beta} \frac{\pl q_\beta}{\pl Q_\alpha}\right] + \sum_{\beta=1}^s \frac{\pl L}{\pl \dot{q}_\beta} \left(\frac{\mathrm{d}}{\mathrm{d} t} \frac{\pl q_\beta}{\pl Q_\alpha} - \frac{\pl \dot{q}_\beta}{\pl Q_\alpha}\right) \\
	& = \sum_{\alpha=1}^s \left(\frac{\mathrm{d}}{\mathrm{d} t} \frac{\pl L}{\pl \dot{q}_\beta} - \frac{\pl L}{\pl q_\beta}\right) \frac{\pl q_\beta}{\pl Q_\alpha} = 0
\end{align*}
此处利用了推广经典Lagrange关系\eqref{chapter2:推广经典Lagrange关系}:
\begin{equation*}
	\frac{\pl \dot{q}_\beta}{\pl \dot{Q}_\alpha} = \frac{\pl q_\beta}{\pl Q_\alpha},\quad \frac{\pl \dot{q}_\beta}{\pl Q_\alpha} = \frac{\mathrm{d}}{\mathrm{d} t} \frac{\pl q_\beta}{\pl Q_\alpha}
\end{equation*}
可见,在点变换下Lagrange方程不变。

\subsection{正则变换}

在Hamilton力学中,广义坐标和广义动量处于平等地位,变换的自由度更大。因此不必拘泥于点变换,可以考虑更为广义的变换:
\begin{equation}
	\begin{cases}
		Q_\alpha = Q_\alpha(\mbf{q},\mbf{p},t), \\
		P_\alpha = P_\alpha(\mbf{q},\mbf{p},t),
	\end{cases}\quad \alpha = 1,2,\cdots,s
	\label{正则坐标变换}
\end{equation}
在变换\eqref{正则坐标变换}下,Hamilton函数将变换为
\begin{equation}
	K = K(\mbf{Q},\mbf{P},t) = H(\mbf{q}(\mbf{Q},\mbf{P},t),\mbf{p}(\mbf{Q},\mbf{P},t),t)
\end{equation}
一般情况下,在变换之后正则方程将不能保持原来的形式。如果在变换\eqref{正则坐标变换}下,新的Hamilton函数$K(\mbf{Q},\mbf{P},t)$仍然满足正则方程
\begin{equation}
	\begin{cases}
		\displaystyle \dot{Q}_\alpha = \frac{\pl K}{\pl P_\alpha}, \\[1.5ex]
		\displaystyle \dot{P}_\alpha = -\frac{\pl K}{\pl Q_\alpha},
	\end{cases}\qquad (\alpha =1,2,\cdots,s)
\end{equation}
这种变换就称为{\heiti 正则变换}。

\subsection{正则变换的条件}

由于正则方程与Hamilton原理是等价的,因此只要变换\eqref{正则坐标变换}可以满足Hamilton原理的要求,就可以保证新的正则方程形式保持不变。

在变换前后,正则变量都需要满足边界条件
\begin{align}
	\begin{cases}
		\delta\mbf{q}(t_1) = \delta \mbf{q}(t_2) = 0, \\
		\delta\mbf{p}(t_1) = \delta \mbf{p}(t_2) = 0,
	\end{cases}\quad 
	\begin{cases}
		\delta\mbf{Q}(t_1) = \delta \mbf{Q}(t_2) = 0, \\
		\delta\mbf{P}(t_1) = \delta \mbf{P}(t_2) = 0,
	\end{cases}
\end{align}
根据相空间Hamilton原理,有
\begin{align}
	& \delta \int_{t_1}^{t_2} \left[\sum_{\alpha=1}^s p_\alpha \dot{q}_\alpha - H(\mbf{q},\mbf{p},t)\right] \mathrm{d} t = 0 \label{正则变换前的Hamilton原理} %\\
	%& \delta \int_{t_1}^{t_2} \left[\sum_{\alpha=1}^s P_\alpha \dot{Q}_\alpha - H(\mbf{Q},\mbf{P},t)\right] \mathrm{d} t = 0 \label{正则变换后的Hamilton原理}
\end{align}
由于$L$具有规范不定性,即$L$和$L+\dfrac{\mathrm{d} f}{\mathrm{d} t}(\mbf{q},t)$是等价的,而且考虑到式\eqref{正则变换前的Hamilton原理}中的被积函数加上一项全微分之后,定积分$\displaystyle \int_{t_1}^{t_2} \mathrm{d} f$只和端点的值有关,而端点的变分恒为零,因此$f$不仅可以是$\mbf{q}$和$t$的函数,还可以将$\mbf{p}$作为参数包含在内。即正则变换前的Hamilton原理应该表达为
\begin{equation}
	\delta \int_{t_1}^{t_2} \left[\sum_{\alpha=1}^s p_\alpha \dot{q}_\alpha - H(\mbf{q},\mbf{p},t) + \frac{\mathrm{d} f_1}{\mathrm{d} t}(\mbf{q},\mbf{p},t)\right] \mathrm{d} t = 0 \label{正则变换前的Hamilton原理一般形式}
\end{equation}
同理正则变换之后的Hamilton原理应该表达为
\begin{equation}
	\delta \int_{t_1}^{t_2} \left[\sum_{\alpha=1}^s P_\alpha \dot{Q}_\alpha - K(\mbf{Q},\mbf{P},t) + \frac{\mathrm{d} f_2}{\mathrm{d} t}(\mbf{Q},\mbf{P},t)\right] \mathrm{d} t = 0 \label{正则变换前的Hamilton原理一般形式}
\end{equation}
因此,可有
\begin{equation}
	\delta \int_{t_1}^{t_2} \left[\lambda \left(\sum_{\alpha=1}^s p_\alpha \dot{q}_\alpha - H\right) - \left(\sum_{\alpha=1}^s P_\alpha \dot{Q}_\alpha - K\right) + \lambda \frac{\mathrm{d} f_1}{\mathrm{d} t} - \frac{\mathrm{d} f_2}{\mathrm{d} t}\right] \mathrm{d} t = 0
\end{equation}
记
\begin{equation}
	F_1 = f_2(\mbf{Q},\mbf{P},t) - \lambda f_1(\mbf{q},\mbf{p},t)
\end{equation}
则有
\begin{equation*}
	\delta \int_{t_1}^{t_2} \left[\lambda \left(\sum_{\alpha=1}^s p_\alpha \dot{q}_\alpha - H(\mbf{q},\mbf{p},t)\right) - \left(\sum_{\alpha=1}^s P_\alpha \dot{Q}_\alpha - K(\mbf{Q},\mbf{P},t)\right) - \frac{\mathrm{d} F_1}{\mathrm{d} t}(\mbf{Q},\mbf{P},t)\right] \mathrm{d} t = 0
\end{equation*}
要使上式成立,只要取
\begin{equation}
	\lambda \left(\sum_{\alpha=1}^s p_\alpha \dot{q}_\alpha - H(\mbf{q},\mbf{p},t)\right) = \sum_{\alpha=1}^s P_\alpha \dot{Q}_\alpha - K(\mbf{Q},\mbf{P},t) + \dot{F_1}(\mbf{Q},\mbf{P},t) = 0
	\label{chapter10:正则变换需满足的一般条件1}
\end{equation}
式\eqref{chapter10:正则变换需满足的一般条件1}中$\lambda$称为{\heiti 标度因子},表示不同的单位度量引起的数值放大,不影响物理实质,因此可取$\lambda=1$,则有
\begin{equation}
	\mathrm{d} F_1 = \sum_{\alpha=1}^s p_\alpha \mathrm{d} q_\alpha - \sum_{\alpha=1}^s P_\alpha \mathrm{d} Q_\alpha + (K-H) \mathrm{d} t
	\label{正则变换需满足的一般条件}
\end{equation}
满足式\eqref{正则变换需满足的一般条件}的变换即为{\heiti 正则变换}。如果$\lambda \neq 1$,则为{\heiti 拓展正则变换}。

\subsection{母函数}

%\subsubsection{第一类母函数}

考虑式\eqref{正则变换需满足的一般条件}的右端,同时出现新旧两组正则变量$\mbf{q},\mbf{p}$和$\mbf{Q},\mbf{P}$,但这$4s$个变量之间由于存在$2s$个变换方程\eqref{正则坐标变换},因此独立的变量只有$2s$个。在式\eqref{正则变换需满足的一般条件}中,独立的变量是正则坐标$\mbf{q}$和$\mbf{Q}$,因此函数$F_1$也应取$\mbf{q}$和$\mbf{Q}$加上时间$t$作为自己的变量,即
\begin{equation}
	\mathrm{d} F_1(\mbf{q},\mbf{Q},t) = \sum_{\alpha=1}^s p_\alpha \mathrm{d} q_\alpha - \sum_{\alpha=1}^s P_\alpha \mathrm{d} Q_\alpha + (K-H) \mathrm{d} t
	\label{第一类母函数下的正则变换}
\end{equation}
函数$F_1 = F_1(\mbf{q},\mbf{Q},t)$称为{\heiti 第一类母函数}。考虑
\begin{equation}
	\mathrm{d} F_1 = \sum_{\alpha=1}^s \frac{\pl F_1}{\pl q_\alpha} \mathrm{d} q_\alpha + \sum_{\alpha=1}^s \frac{\pl F_1}{\pl Q_\alpha} \mathrm{d} Q_\alpha + \frac{\pl F_1}{\pl t} \mathrm{d} t
\end{equation}
与式\eqref{正则变换需满足的一般条件}对比,可得变换的具体形式
\begin{equation}
	\begin{cases}
		\displaystyle p_\alpha = \frac{\pl F_1}{\pl q_\alpha} = \bar{p}_\alpha(\mbf{q},\mbf{Q},t) \\[1.5ex]
		\displaystyle P_\alpha = -\frac{\pl F_1}{\pl Q_\alpha} = \bar{P}_\alpha(\mbf{q},\mbf{Q},t) \\[1.5ex]
		\displaystyle K = H+\frac{\pl F_1}{\pl t}
	\end{cases}
	\label{chapter10:第一类母函数确定的正则变换形式}
\end{equation}
式\eqref{chapter10:第一类母函数确定的正则变换形式}所表示的变换公式与式\eqref{正则坐标变换}的形式仍然有所差别,但是只要通过代数运算就可以将式\eqref{chapter10:第一类母函数确定的正则变换形式}化为式\eqref{正则坐标变换}的形式。具体来讲,可以先从\eqref{chapter10:第一类母函数确定的正则变换形式}的第一式中解出$\mbf{Q}$作为$\mbf{q},\mbf{p},t$的函数,再将其代入式\eqref{chapter10:第一类母函数确定的正则变换形式}的第二式,就获得了式\eqref{正则坐标变换}的形式。最后再将获得的变换代入式\eqref{chapter10:第一类母函数确定的正则变换形式}的第三式中,就得到了变换之后的Hamilton函数$K$。

%\subsubsection{第二类母函数}

第一类母函数$F_1$是用$\mbf{q},\mbf{Q},t$表示的,利用Legendre变换,可以改变母函数的自变量,得到其它类型的母函数(共四类)。

首先考虑用$\mbf{P}$代替$\mbf{Q}$,即
\begin{equation*}
	\sum_{\alpha=1}^s P_\alpha \mathrm{d} Q_\alpha = \mathrm{d} \left(\sum_{\alpha=1}^s P_\alpha Q_\alpha\right) - \sum_{\alpha=1}^s Q_\alpha \mathrm{d} P_\alpha
\end{equation*}
因此可以将式\eqref{正则变换需满足的一般条件}改写为
\begin{equation*}
	\mathrm{d} \left(F_1 + \sum_{\alpha=1}^s P_\alpha Q_\alpha\right) = \sum_{\alpha=1}^s p_\alpha \mathrm{d} q_\alpha + \sum_{\alpha=1}^s Q_\alpha \mathrm{d} P_\alpha + (K-H) \mathrm{d} t
\end{equation*}
定义{\bf 第二类母函数}
\begin{equation}
	F_2 = F_2(\mbf{q},\mbf{P},t) = F_1 + \sum_{\alpha=1}^s P_\alpha Q_\alpha
	\label{第二类母函数}
\end{equation}
于是得到
\begin{equation}
	\mathrm{d} F_2(\mbf{q},\mbf{P},t) = \sum_{\alpha=1}^s p_\alpha \mathrm{d} q_\alpha + \sum_{\alpha=1}^s Q_\alpha \mathrm{d} P_\alpha + (K-H) \mathrm{d} t
	\label{第二类母函数下的正则变换}
\end{equation}
由此可得变换的具体形式
\begin{equation}
	\begin{cases}
		\displaystyle p_\alpha = \frac{\pl F_2}{\pl q_\alpha} \\[1.5ex]
		\displaystyle Q_\alpha = \frac{\pl F_2}{\pl P_\alpha} \\[1.5ex]
		\displaystyle K = H + \frac{\pl F_2}{\pl t}
	\end{cases}
	\label{chapter10:第二类母函数下的正则变换-变换式}
\end{equation}

%\subsubsection{第三类母函数}

借助另一个Legendre变换,可用$\mbf{p}$代替$F_1$中的$\mbf{q}$,即考虑
\begin{equation*}
	\sum_{\alpha=1}^s p_\alpha \mathrm{d} q_\alpha = \mathrm{d} \left(\sum_{\alpha=1}^s p_\alpha q_\alpha\right) - \sum_{\alpha=1}^s q_\alpha \mathrm{d} p_\alpha
\end{equation*}
因此可以将式\eqref{正则变换需满足的一般条件}改写为
\begin{equation*}
	\mathrm{d} \left(F_1 - \sum_{\alpha=1}^s p_\alpha q_\alpha\right) = -\sum_{\alpha=1}^s q_\alpha \mathrm{d} p_\alpha - \sum_{\alpha=1}^s P_\alpha \mathrm{d} Q_\alpha + (K-H) \mathrm{d} t
\end{equation*}
定义{\bf 第三类母函数}
\begin{equation}
	F_3 = F_3(\mbf{p},\mbf{Q},t) = F_1 - \sum_{\alpha=1}^s p_\alpha q_\alpha
	\label{第三类母函数}
\end{equation}
于是得到
\begin{equation}
	\mathrm{d} F_3(\mbf{p},\mbf{Q},t) = -\sum_{\alpha=1}^s q_\alpha \mathrm{d} p_\alpha - \sum_{\alpha=1}^s P_\alpha \mathrm{d} Q_\alpha + (K-H) \mathrm{d} t
	\label{第三类母函数下的正则变换}
\end{equation}
由此可得变换的具体形式
\begin{equation}
	\begin{cases}
		\displaystyle q_\alpha = -\frac{\pl F_3}{\pl p_\alpha} \\[1.5ex]
		\displaystyle P_\alpha = -\frac{\pl F_3}{\pl Q_\alpha} \\[1.5ex]
		\displaystyle K = H + \frac{\pl F_3}{\pl t}
	\end{cases}
\end{equation}

%\subsubsection{第四类母函数}

同理,借助一个不同的Legendre变换,可以用$\mbf{p}$替代$F_2$中的$\mbf{q}$,即考虑
\begin{equation*}
	\sum_{\alpha=1}^s p_\alpha \mathrm{d} q_\alpha = \mathrm{d} \left(\sum_{\alpha=1}^s p_\alpha q_\alpha\right) - \sum_{\alpha=1}^s q_\alpha \mathrm{d} p_\alpha
\end{equation*}
因此可以将式\eqref{正则变换需满足的一般条件}改写为
\begin{equation*}
	\mathrm{d} \left(F_2 - \sum_{\alpha=1}^s p_\alpha q_\alpha\right) = -\sum_{\alpha=1}^s q_\alpha \mathrm{d} p_\alpha + \sum_{\alpha=1}^s Q_\alpha \mathrm{d} P_\alpha + (K-H) \mathrm{d} t
\end{equation*}
定义{\bf 第四类母函数}
\begin{equation}
	F_4 = F_4(\mbf{p},\mbf{P},t) = F_2 - \sum_{\alpha=1}^s p_\alpha q_\alpha
	\label{第四类母函数}
\end{equation}
于是得到
\begin{equation}
	\mathrm{d} F_4(\mbf{p},\mbf{P},t) = -\sum_{\alpha=1}^s q_\alpha \mathrm{d} p_\alpha + \sum_{\alpha=1}^s Q_\alpha \mathrm{d} P_\alpha + (K-H) \mathrm{d} t
	\label{第四类母函数下的正则变换}
\end{equation}
由此可得变换的具体形式
\begin{equation}
	\begin{cases}
		\displaystyle q_\alpha = -\frac{\pl F_4}{\pl p_\alpha} \\[1.5ex]
		\displaystyle Q_\alpha = \frac{\pl F_4}{\pl P_\alpha} \\[1.5ex]
		\displaystyle K = H + \frac{\pl F_4}{\pl t}
	\end{cases}
\end{equation}

特别地,如果母函数$F_i$不显含时间$t$,即$\dfrac{\pl F_i}{\pl t} = 0$,则变换前后的Hamilton函数相等,即有
\begin{equation*}
	K=H
\end{equation*}

\begin{example}[恒等变换]
恒等变换可以用第二类母函数表示为
\begin{equation*}
	F_2(\mbf{q},\mbf{P},t) = \sum_{\beta=1}^s q_\beta P_\beta
\end{equation*}
则有
\begin{equation*}
	\begin{cases}
		p_\alpha = \dfrac{\pl F_2}{\pl q_\alpha} = P_\alpha \\[1.5ex]
		Q_\alpha = \dfrac{\pl F_2}{\pl P_\alpha} = q_\alpha \\[1.5ex]
		K = H + \dfrac{\pl F_2}{\pl t} = H
	\end{cases}
\end{equation*}
用第三类母函数可以表示为
\begin{equation*}
	F_3(\mbf{q},\mbf{P},t) = -\sum_{\beta=1}^s p_\beta Q_\beta
\end{equation*}
则有
\begin{equation*}
	\begin{cases}
		q_\alpha = -\dfrac{\pl F_3}{\pl p_\alpha} = Q_\alpha \\[1.5ex]
		P_\alpha = -\dfrac{\pl F_3}{\pl Q_\alpha} = p_\alpha \\[1.5ex]
		K = H + \dfrac{\pl F_3}{\pl t} = H
	\end{cases}
\end{equation*}
\end{example}

\begin{example}[平移变换]
平移变换可以用第二类母函数表示为
\begin{equation*}
	F_2(\mbf{q},\mbf{P},t) = \sum_{\beta=1}^s \left(q_\beta P_\beta + a_\beta P_\beta - b_\beta q_\beta\right)
\end{equation*}
则有
\begin{equation*}
	\begin{cases}
		p_\alpha = \dfrac{\pl F_2}{\pl q_\alpha} = P_\alpha-b_\alpha \\[1.5ex]
		Q_\alpha = \dfrac{\pl F_2}{\pl P_\alpha} = q_\alpha+a_\alpha \\[1.5ex]
		K = H + \dfrac{\pl F_2}{\pl t} = H
	\end{cases}
\end{equation*}
\end{example}

\begin{example}[对偶变换]
对偶变换可以用第一类母函数表示为
\begin{equation*}
	F_1(\mbf{q},\mbf{Q},t) = \sum_{\beta=1}^s q_\beta Q_\beta
\end{equation*}
则有
\begin{equation*}
	\begin{cases}
		p_\alpha = \dfrac{\pl F_1}{\pl q_\alpha} = Q_\alpha \\[1.5ex]
		P_\alpha = -\dfrac{\pl F_1}{\pl Q_\alpha} = -q_\alpha \\[1.5ex]
		K = H + \dfrac{\pl F_1}{\pl t} = H
	\end{cases}
\end{equation*}
即此变换可以将广义坐标变换为广义动量,将广义动量变换为广义坐标,相当于在相空间做了一个旋转。
\end{example}

\begin{example}[点变换]
点变换可以用第二类母函数表示为
\begin{equation*}
	F_2(\mbf{q},\mbf{P},t) = \sum_{\beta=1}^s Q_\beta(\mbf{q},t) P_\beta
\end{equation*}
则有
\begin{equation*}
	\begin{cases}
		\displaystyle p_\alpha = \frac{\pl F_2}{\pl q_\alpha} = \sum_{\beta=1}^s P_\beta \frac{\pl Q_\beta}{\pl q_\alpha} \\[1.5ex]
		\displaystyle Q_\alpha = \frac{\pl F_2}{\pl P_\alpha} = Q_\alpha(\mbf{q},t) \\[1.5ex]
		\displaystyle K = H + \frac{\pl F_2}{\pl t} = H + \sum_{\beta=1}^s P_\beta \frac{\pl Q_\beta}{\pl t}
	\end{cases}
\end{equation*}
\end{example}

\begin{example}[一维谐振子的正则变换求解]
一维谐振子的Hamilton函数为
\begin{equation*}
	H = \frac{p^2}{2m} + \frac12 m\omega^2 q^2
\end{equation*}
构造正则变换,使得
\begin{equation*}
	\frac{\pl K}{\pl Q} = 0,\quad \frac{\mathrm{d} P}{\mathrm{d} t} = 0
\end{equation*}
进而使得新正则方程的求解简化。构造母函数$F_1 = F_1(q,Q,t)$,则有
\begin{equation*}
	\begin{cases}
		\displaystyle p = \frac{\pl F_1}{\pl q} \\[1.5ex]
		\displaystyle P = -\frac{\pl F_1}{\pl Q} \\[1.5ex]
		\displaystyle K = H + \frac{\pl F_1}{\pl t}
	\end{cases}
\end{equation*}
考虑到$\dfrac{\mathrm{d} P}{\mathrm{d} t} = 0$,则$P$为运动积分。由于$H$不显含时间,则$H$也为运动积分(为常数),即能量守恒,于是可以令
\begin{equation}
	p = f(P) \cos Q,\quad q = \frac{f(P)}{m\omega} \sin Q
	\label{p,q}
\end{equation}
并假设母函数$F_1$不显含时间,则
\begin{equation*}
	K = H = \frac{p^2}{2m} + \frac12 m\omega^2 q^2 = \frac{[f(P)]^2}{2m}
\end{equation*}
根据变换关系式可得
\begin{equation*}
	p = \frac{\pl F_1}{\pl q}(q,Q)
\end{equation*}
根据式\eqref{p,q}可得
\begin{equation*}
	p = m\omega q \cot Q
\end{equation*}
则可得到
\begin{equation*}
	F_1 = \frac12 m\omega q^2 \cot Q 
\end{equation*}
根据变换关系式可得
\begin{equation*}
	P = -\frac{\pl F_1}{\pl Q} = \frac{m\omega q^2}{2\sin^2 Q}
\end{equation*}
所以有
\begin{align*}
	& q = \sqrt{\frac{2P}{m\omega}} \sin Q \\
	& f(P) = \sqrt{2m\omega P}
\end{align*}
此时
\begin{equation*}
	K = H = \omega P
\end{equation*}
将系统总能量记作$E$,可有
\begin{equation*}
	P = \frac{E}{\omega}
\end{equation*}
再由
\begin{equation*}
	\dot{Q} = \frac{\pl K}{\pl P} = \omega
\end{equation*}
可得
\begin{equation*}
	Q = \omega t + \phi
\end{equation*}
因此,系统的解为
\begin{equation*}
	\begin{cases}
		q = \sqrt{\dfrac{2E}{m\omega^2}} \sin (\omega t + \phi) \\[1.5ex]
		p = \sqrt{2mE} \cos(\omega t+\phi)
	\end{cases}
\end{equation*}
\end{example}

\subsection{无穷小正则变换}\label{chapter10:subsection-无穷小正则变换}

考虑第二类母函数
\begin{equation}
	F_2(\mbf{q},\mbf{P},t) = \sum_{\alpha=1}^s q_\alpha P_\alpha + \eps G(\mbf{q},\mbf{P},t),\quad \eps \to 0
\end{equation}
则有
\begin{equation*}
	\begin{cases}
		\displaystyle p_\alpha = \frac{\pl F_2}{\pl q_\alpha} = P_\alpha + \eps \frac{\pl G}{\pl q_\alpha} \\[1.5ex]
		\displaystyle Q_\alpha = \frac{\pl F_2}{\pl P_\alpha} = q_\alpha + \eps \frac{\pl G}{\pl P_\alpha} \\[1.5ex]
		\displaystyle K = H+\frac{\pl F_2}{\pl t} = H + \eps \frac{\pl G}{\pl t}
	\end{cases}
\end{equation*}
其中
\begin{equation}
	\eps \frac{\pl G}{\pl P_\alpha} = \eps \frac{\pl G}{\pl p_\alpha} \frac{\pl p_\alpha}{\pl P_\alpha} = \eps \frac{\pl G}{\pl p_\alpha} + o(\eps)
\end{equation}
略去高阶小量,可得{\heiti 无穷小正则变换}
\begin{subnumcases}{\label{chapter10:无穷小正则变换}}
	Q_\alpha - q_\alpha = \eps \frac{\pl G}{\pl p_\alpha} \\[1.5ex]
	P_\alpha - p_\alpha = -\eps \frac{\pl G}{\pl q_\alpha} \\[1.5ex]
	K - H = \eps \frac{\pl G}{\pl t}
\end{subnumcases}
相应地,在精确至一阶小量时,可以将函数$G(\mbf{q},\mbf{P},t)$记作$G(\mbf{q},\mbf{p},t)$,函数$G(\mbf{q},\mbf{p},t)$相应称为{\heiti 无穷小正则变换生成函数}。

%无穷小正则变换为对称变换,变换后正则方程结构相同,因而运动规律也是相同的。亦即,对系统操作前后,运动是完全相同的。

% 考虑变换
% \begin{equation*}
	% (\mbf{Q},\mbf{P},t) = (\mbf{q}+\delta \mbf{q},\mbf{p}+\delta \mbf{p},t)
% \end{equation*}
% 略去高阶小量,将有
% \begin{equation*}
	% \begin{cases}
		% \delta q_\alpha = Q_\alpha-q_\alpha = \eps \dfrac{\pl G}{\pl p_\alpha} \\[1.5ex]
		% \delta p_\alpha = P_\alpha-p_\alpha = -\eps \dfrac{\pl G}{\pl q_\alpha} \\[1.5ex]
		% K(\mbf{Q},\mbf{P},t) - H(\mbf{q},\mbf{p},t) = \eps \dfrac{\pl G}{\pl t}
	% \end{cases}
% \end{equation*}
% 由第三式即有
% \begin{equation*}
	% K(\mbf{Q},\mbf{P},t) - H(\mbf{q},\mbf{p},t) - \eps \frac{\pl G}{\pl t} = \delta H - \eps \frac{\pl G}{\pl t} = 0
% \end{equation*}
% 而
% \begin{align*}
	% \delta H & = \sum_{\alpha=1}^s \left(\frac{\pl H}{\pl q_\alpha} \delta q_\alpha + \frac{\pl H}{\pl p_\alpha} \delta p_\alpha\right) = \sum_{\alpha=1}^s \left(\frac{\pl H}{\pl q_\alpha} \eps \frac{\pl G}{\pl p_\alpha} - \frac{\pl H}{\pl p_\alpha} \eps \frac{\pl G}{\pl q_\alpha}\alpha\right) \\
	% & = \eps [H,G]
% \end{align*}
% 所以有
% \begin{equation*}
	% K(\mbf{Q},\mbf{P},t) - H(\mbf{q},\mbf{p},t) - \eps \frac{\pl G}{\pl t} = -\eps \left([G,H] + \frac{\pl G}{\pl t}\right) = -\eps \frac{\mathrm{d} G}{\mathrm{d} t} = 0
% \end{equation*}
% 因此,系统对称则无穷小正则变换生成元守恒;以运动积分为生成元的无穷小正则变换为系统对称变换。

如果将变换前后的正则变量看作是相空间中在不同时刻的新旧坐标,那么无穷小正则变换就是相空间中的无穷小移动,将其位移记作
\begin{equation}
	\mathd \mbf{q} = \mbf{Q}-\mbf{q},\quad \mathd \mbf{p} = \mbf{P}-\mbf{p}
\end{equation}
那么变换公式\eqref{chapter10:无穷小正则变换}可以写成
\begin{equation}
\begin{cases}
	\mathd q_\alpha = \eps\dfrac{\pl G}{\pl p_\alpha} \\[1.5ex]
	\mathd p_\alpha = -\eps\dfrac{\pl G}{\pl q_\alpha}
\end{cases}\,\, (\alpha=1,2,\cdots,s)
\label{chapter10:系统运动观点下的无穷小正则变换}
\end{equation}

\begin{example}%[广义动量守恒]
取无穷小正则变换生成函数为
\begin{equation}
	G(\mbf{q},\mbf{P},t) = p_\beta
	\label{chapter10:无穷小正则变换举例1}
\end{equation}
则此时的无穷小正则变换根据变换式\eqref{chapter10:系统运动观点下的无穷小正则变换}可以写为
\begin{equation}
\begin{cases}
	\mathd q_\alpha = 0\,(\alpha\neq\beta),\quad \mathd q_\beta = \eps \\
	\mathd p_\alpha = 0\quad (\alpha=1,2,\cdots,s)
\end{cases}
\end{equation}
即以函数\eqref{chapter10:无穷小正则变换举例1}为生成函数的无穷小正则变换使得力学系统在坐标$q_\beta$的方向上作微小移动,它的广义坐标$q_\beta$增加$\eps$,其它广义坐标不变。

% 如果Hamilton函数在这个变换下不变,即
% \begin{equation}
	% \mathd H = K-H = \frac{\pl H}{\pl p_\beta} \mathd p_\beta = \eps\frac{\pl H}{\pl p_\beta} = 0
% \end{equation}
% 即$\dfrac{\pl H}{\pl p_\beta}=0$,根据广义动量定理可知$p_\beta=\text{常数}$为运动积分。
\end{example}

%\subsection{运动方程的无穷小正则变换表示}

如果将无穷小正则变换生成函数取为$G = H(\mbf{q},\mbf{p},t)$,并将无穷小参数取为$\eps = \mathrm{d} t$,则根据变换式\eqref{chapter10:系统运动观点下的无穷小正则变换}可得
\begin{equation}
\begin{cases}
	\ds \mathd q_\alpha(t) = \frac{\pl H}{\pl p_\alpha} \mathrm{d} t \\[1.5ex]
	\ds \mathd p_\alpha(t) = -\frac{\pl H}{\pl q_\alpha} \mathrm{d} t \\[1.5ex]
	\ds \mathd H = H(\mbf{q}+\mathd \mbf{q},\mbf{p}+\mathd \mbf{p},t) - H(\mbf{q},\mbf{p},t) = \frac{\pl H}{\pl t} \mathrm{d} t
\end{cases}
\end{equation}
这就是系统的正则方程,即
\begin{equation*}
\begin{cases}
	\ds \dot{q}_\alpha = \frac{\pl H}{\pl p_\alpha} \\[1.5ex]
	\ds \dot{p}_\alpha = -\frac{\pl H}{\pl q_\alpha}
\end{cases}\,\, (\alpha=1,2,\cdots,s)
\end{equation*}
所以,以系统Hamilton函数为生成函数、以$\mathd t$为参数的无穷小正则变换,正好描述系统在$\mathd t$时间内的演化。由此可得。系统在有限时间内的演化,则可以由一系列相继的无穷小正则变换描述,其总效果也是一个正则变换\label{chapter10:footnote-正则变换的传递性}\footnote{得出这个结论首先需要证明,如果变换$\mbf{q},\mbf{p} \mapsto \mbf{Q}',\mbf{P}'$和$\mbf{Q}',\mbf{P}' \mapsto \mbf{Q},\mbf{P}$都是正则的,那么变换$\mbf{q},\mbf{p} \mapsto \mbf{Q},\mbf{P}$也是正则的。利用正则变换的矩阵表示很容易证明这一点,这将在第\ref{chapter10:subsection-正则变换条件的辛表示}节说明。}。换句话说,系统的正则变量在时刻$t$的值,可以由其初始值通过正则变换得出,这个正则变换是时间$t$的函数。反过来说,系统正则变量在时刻$t$的值,必然可以通过某一正则变换变换为其初始值。那么如果能够找到这样一个正则变换,在这个变换下所有的正则变量都变为常数即其初始值,那么系统的动力学问题也就解决了。这是Hamilton-Jacobi方程的研究内容。

既然无穷小正则变换可以表示系统在相空间的移动,那么对任意一个力学量$u$,如果其不显含时间,则有
\begin{equation}
	\mathd u = \sum_{\alpha=1}^s \left(\frac{\pl u}{\pl q_\alpha} \mathd q_\alpha + \frac{\pl u}{\pl p_\alpha}\mathd p_\alpha\right)
	\label{chapter10:无穷小正则变换与运动积分1}
\end{equation}
将变换式\eqref{chapter10:系统运动观点下的无穷小正则变换}代入,可得
\begin{equation}
	\mathd u = \eps\sum_{\alpha=1}^s \left(\frac{\pl u}{\pl q_\alpha} \frac{\pl G}{\pl p_\alpha} - \frac{\pl u}{\pl p_\alpha} \frac{\pl G}{\pl q_\alpha}\right) = \eps [u,G]
	\label{chapter10:无穷小正则变换与运动积分2}
\end{equation}

将式\eqref{chapter10:无穷小正则变换与运动积分2}中的$u$取为Hamilton函数$H$,则有
\begin{equation}
	\mathd H = \eps [H,G]
\end{equation}
如果$G$是系统的某个运动积分,且不显含时间,则根据式\eqref{chapter3:不显含时间的力学量对时间导数的Poisson括号表示}有$[H,G]=0$,此时可得
\begin{equation*}
	\mathd H = 0
\end{equation*}
即在以$G$为生成函数的无穷小正则变换下,Hamilton函数$H$保持不变。反过来说,如果以$G$为生成函数的无穷小正则变换能够使得$H$保持不变,则$G$是系统的一个运动积分。这也再次得到了在第\ref{chapter2:section-时空对称性与守恒量}节所描述的结论,检查Hamilton函数在哪些变换下保持不变,就能确定系统的运动积分。

\subsection{正则变换条件的辛表示}\label{chapter10:subsection-正则变换条件的辛表示}

设变换前后的正则变量分别为
\begin{equation*}
	\mbf{\xi} = \begin{pmatrix} \mbf{q} \\ \mbf{p} \end{pmatrix} = \begin{pmatrix} q_1 \\ \vdots \\ q_s \\ p_1 \\ \vdots \\ p_s \end{pmatrix},\quad \mrm{\Xi} = \begin{pmatrix} \mbf{Q} \\ \mbf{P} \end{pmatrix} = \begin{pmatrix} Q_1 \\ \vdots \\ Q_s \\ P_1 \\ \vdots \\ P_s \end{pmatrix}
\end{equation*}
设正则变换可以表示为
\begin{equation}
	\mrm{\Xi} = \mrm{\Xi}(\mbf{\xi},t),\quad \mbf{\xi} = \mbf{\xi}(\mrm{\Xi},t)
\end{equation}
则有
\begin{equation*}
	\delta \mrm{\Xi} = \mbf{M} \delta \mbf{\xi},\quad \delta \mbf{\xi} = \mbf{W} \delta \mrm{\Xi}
\end{equation*}
其中Jacobi矩阵以及逆变换的Jacobi矩阵可以表示为
\begin{equation}
	\mbf{M} = \frac{\pl \mrm{\Xi}}{\pl \mbf{\xi}} = \begin{pmatrix} \dfrac{\pl \mbf{Q}}{\pl \mbf{q}} & \dfrac{\pl \mbf{Q}}{\pl \mbf{p}} \\[1.5ex] \dfrac{\pl \mbf{P}}{\pl \mbf{q}} & \dfrac{\pl \mbf{P}}{\pl \mbf{p}} \end{pmatrix},\quad \mbf{W} = \frac{\pl \mbf{\xi}}{\pl \mrm{\Xi}} = \begin{pmatrix} \dfrac{\pl \mbf{q}}{\pl \mbf{Q}} & \dfrac{\pl \mbf{q}}{\pl \mbf{P}} \\[1.5ex] \dfrac{\pl \mbf{p}}{\pl \mbf{Q}} & \dfrac{\pl \mbf{p}}{\pl \mbf{P}} \end{pmatrix} = \mbf{M}^{-1}
\end{equation}
用第一类母函数$F_1=F_1(\mbf{q},\mbf{Q},t)$可以将正则变换的条件表示为
\begin{equation*}
	\mathrm{d} F_1 = \sum_{\gamma=1}^s p_\gamma \mathrm{d} q_\gamma - \sum_{\zeta=1}^s P_\zeta \mathrm{d} Q_\zeta + (K-H)\mathrm{d} t
\end{equation*}
考虑到$\mbf{Q} = \mbf{Q}(\mbf{q},\mbf{p},t)$,可有
\begin{equation*}
	\mathrm{d} Q_\zeta = \sum_{\gamma=1}^s \left(\frac{\pl Q_\zeta}{\pl q_\gamma} \mathrm{d} q_\gamma + \frac{\pl Q_\zeta}{\pl p_\gamma} \mathrm{d} p_\gamma\right) + \frac{\pl Q_\zeta}{\pl t} \mathrm{d} t
\end{equation*}
所以
\begin{align*}
	\mathrm{d} \bar{F}_1(\mbf{q},\mbf{p},t) & = \sum_{\gamma=1}^s \left[\left(p_\gamma - \sum_{\zeta=1}^s P_\zeta \frac{\pl Q_\zeta}{\pl q_\gamma}\right) \mathrm{d} q_\gamma - \left(\sum_{\zeta=1}^s P_\zeta \frac{\pl Q_\zeta}{\pl p_\gamma}\right) \mathrm{d} p_\gamma\right] \\
	& \quad {}+ \left(K-H - \sum_{\zeta=1}^s P_\zeta \frac{\pl Q_\zeta}{\pl t}\right) \mathrm{d} t
\end{align*}
此处,函数$\bar{F}_1(\mbf{q},\mbf{p},t) = F_1(\mbf{q},\mbf{Q}(\mbf{q},\mbf{p},t),t)$,则有
\begin{equation*}
\begin{array}{ll}
	\displaystyle \frac{\pl \bar{F}_1}{\pl q_\alpha} = p_\alpha - \sum_{\zeta=1}^s P_\zeta \frac{\pl Q_\zeta}{\pl q_\alpha}, & \displaystyle \frac{\pl \bar{F}_1}{\pl q_\beta} = p_\beta - \sum_{\zeta=1}^s P_\zeta \frac{\pl Q_\zeta}{\pl q_\beta} \\[1.5ex]
	\displaystyle \frac{\pl \bar{F}_1}{\pl p_\alpha} = -\sum_{\zeta=1}^s P_\zeta \frac{\pl Q_\zeta}{\pl p_\alpha}, & \displaystyle \frac{\pl \bar{F}_1}{\pl p_\beta} = -\sum_{\zeta=1}^s P_\zeta \frac{\pl Q_\zeta}{\pl p_\beta}
\end{array}
\end{equation*}
根据$\dfrac{\pl^2 \bar{F}_1}{\pl q_\alpha \pl q_\beta} = \dfrac{\pl^2 \bar{F}_1}{\pl q_\beta \pl q_\alpha}$可得
\begin{equation*}
	-\sum_{\zeta=1}^s \frac{\pl P_\zeta}{\pl q_\beta} \frac{\pl Q_\zeta}{\pl q_\alpha} - \sum_{\zeta=1}^s P_\zeta \frac{\pl^2 Q_\zeta}{\pl q_\alpha \pl q_\beta} = -\sum_{\zeta=1}^s \frac{\pl P_\zeta}{\pl q_\alpha} \frac{\pl Q_\zeta}{\pl q_\beta} - \sum_{\zeta=1}^s P_\zeta \frac{\pl^2 Q_\zeta}{\pl q_\beta \pl q_\alpha}
\end{equation*}
所以有
\begin{equation*}
	\sum_{\zeta=1}^s \frac{\pl P_\zeta}{\pl q_\beta} \frac{\pl Q_\zeta}{\pl q_\alpha} = \sum_{\zeta=1}^s \frac{\pl P_\zeta}{\pl q_\alpha} \frac{\pl Q_\zeta}{\pl q_\beta}
\end{equation*}
用矩阵表示即
\begin{equation}
	\left(\frac{\pl \mbf{Q}}{\pl \mbf{q}}\right)^{\mathrm{T}} \frac{\pl \mbf{P}}{\pl \mbf{q}} - \left(\frac{\pl \mbf{P}}{\pl \mbf{q}}\right)^{\mathrm{T}} \frac{\pl \mbf{Q}}{\pl \mbf{q}} = \mbf{0}
	\label{正则变换的矩阵表示1}
\end{equation}
再根据$\dfrac{\pl^2 \bar{F}_1}{\pl q_\alpha \pl p_\beta} = \dfrac{\pl^2 \bar{F}_1}{\pl p_\beta \pl q_\alpha}$可得
\begin{equation*}
	-\sum_{\zeta=1}^s \frac{\pl P_\zeta}{\pl q_\alpha} \frac{\pl Q_\zeta}{\pl p_\beta} - \sum_{\zeta=1}^s P_\zeta \frac{\pl^2 Q_\zeta}{\pl q_\alpha \pl p_\beta} = \delta_{\alpha\beta}-\sum_{\zeta=1}^s \frac{\pl P_\zeta}{\pl p_\beta} \frac{\pl Q_\zeta}{\pl q_\alpha} - \sum_{\zeta=1}^s P_\zeta \frac{\pl^2 Q_\zeta}{\pl p_\beta \pl q_\alpha}
\end{equation*}
所以有
\begin{equation*}
	\sum_{\zeta=1}^s \frac{\pl P_\zeta}{\pl q_\alpha} \frac{\pl Q_\zeta}{\pl p_\beta} = \sum_{\zeta=1}^s \frac{\pl P_\zeta}{\pl p_\beta} \frac{\pl Q_\zeta}{\pl q_\alpha} - \delta_{\alpha\beta}
\end{equation*}
用矩阵表示即
\begin{equation}
	\left(\frac{\pl \mbf{Q}}{\pl \mbf{q}}\right)^{\mathrm{T}} \frac{\pl \mbf{P}}{\pl \mbf{p}} - \left(\frac{\pl \mbf{P}}{\pl \mbf{q}}\right)^{\mathrm{T}} \frac{\pl \mbf{Q}}{\pl \mbf{p}} = \mbf{I}
	\label{正则变换的矩阵表示2}
\end{equation}
最后根据$\dfrac{\pl^2 \bar{F}_1}{\pl p_\alpha \pl p_\beta} = \dfrac{\pl^2 \bar{F}_1}{\pl p_\beta \pl p_\alpha}$可得
\begin{equation*}
	-\sum_{\zeta=1}^s \frac{\pl P_\zeta}{\pl p_\alpha} \frac{\pl Q_\zeta}{\pl p_\beta} - \sum_{\zeta=1}^s P_\zeta \frac{\pl^2 Q_\zeta}{\pl p_\alpha \pl p_\beta} = -\sum_{\zeta=1}^s \frac{\pl P_\zeta}{\pl p_\beta} \frac{\pl Q_\zeta}{\pl p_\alpha} - \sum_{\zeta=1}^s P_\zeta \frac{\pl^2 Q_\zeta}{\pl p_\beta \pl p_\alpha}
\end{equation*}
所以有
\begin{equation*}
	\sum_{\zeta=1}^s \frac{\pl P_\zeta}{\pl p_\alpha} \frac{\pl Q_\zeta}{\pl p_\beta} = \sum_{\zeta=1}^s \frac{\pl P_\zeta}{\pl p_\beta} \frac{\pl Q_\zeta}{\pl p_\alpha}
\end{equation*}
用矩阵表示即
\begin{equation}
	\left(\frac{\pl \mbf{Q}}{\pl \mbf{p}}\right)^{\mathrm{T}} \frac{\pl \mbf{P}}{\pl \mbf{p}} - \left(\frac{\pl \mbf{P}}{\pl \mbf{p}}\right)^{\mathrm{T}} \frac{\pl \mbf{Q}}{\pl \mbf{p}} = \mbf{0}
	\label{正则变换的矩阵表示3}
\end{equation}
综合式\eqref{正则变换的矩阵表示1}、式\eqref{正则变换的矩阵表示2}和式\eqref{正则变换的矩阵表示3}可得
\begin{equation*}
	\begin{pmatrix} \left(\dfrac{\pl \mbf{Q}}{\pl \mbf{q}}\right)^{\mathrm{T}} & \left(\dfrac{\pl \mbf{P}}{\pl \mbf{q}}\right)^{\mathrm{T}} \\[1.5ex] \left(\dfrac{\pl \mbf{Q}}{\pl \mbf{p}}\right)^{\mathrm{T}} & \left(\dfrac{\pl \mbf{P}}{\pl \mbf{p}}\right)^{\mathrm{T}} \end{pmatrix} \begin{pmatrix} \mbf{0} & \mbf{I}_s \\ -\mbf{I}_s & \mbf{0} \end{pmatrix} \begin{pmatrix} \dfrac{\pl \mbf{Q}}{\pl \mbf{q}} & \dfrac{\pl \mbf{Q}}{\pl \mbf{p}} \\[1.5ex] \dfrac{\pl \mbf{P}}{\pl \mbf{q}} & \dfrac{\pl \mbf{P}}{\pl \mbf{p}} \end{pmatrix} = \begin{pmatrix} \mbf{0} & \mbf{I}_s \\ -\mbf{I}_s & \mbf{0} \end{pmatrix}
\end{equation*}
或者
\begin{equation}
	\mbf{M}^{\mathrm{T}} \mbf{J} \mbf{M} = \mbf{J}
	\label{正则变换的辛条件}
\end{equation}
式\eqref{正则变换的辛条件}称为{\heiti 正则变换的辛条件},反映了正则方程的对称性。所有正则变换的Jacobi矩阵组成的集合在矩阵乘法下构成{\heiti 辛群}。

如果有两个正则变换
\begin{equation*}
	\mbf{\xi} = \begin{pmatrix} \mbf{q} \\ \mbf{q} \end{pmatrix} \mapsto \mbf{\varXi}' = \begin{pmatrix} \mbf{Q}' \\ \mbf{P}' \end{pmatrix},\quad \mbf{\varXi}' = \begin{pmatrix} \mbf{Q}' \\ \mbf{P}' \end{pmatrix} \mapsto \mbf{\varXi} = \begin{pmatrix} \mbf{Q} \\ \mbf{P} \end{pmatrix}
\end{equation*}
它们的变换矩阵分别表示为$\mbf{M}_1$和$\mbf{M}_2$,则有
\begin{equation}
	\mbf{M}_1^\trans \mbf{J} \mbf{M}_1 = \mbf{J},\quad \mbf{M}_2^\trans \mbf{J} \mbf{M}_2 = \mbf{J}
\end{equation}
则变换
\begin{equation*}
	\mbf{\xi} = \begin{pmatrix} \mbf{q} \\ \mbf{q} \end{pmatrix} \mapsto \mbf{\varXi} = \begin{pmatrix} \mbf{Q} \\ \mbf{P} \end{pmatrix}
\end{equation*}
的变换矩阵为$\mbf{M}=\mbf{M}_2\mbf{M}_1$,由于
\begin{equation}
	\mbf{M}^\trans\mbf{J}\mbf{M} = \mbf{M}_1^\trans\mbf{M}_2^\trans\mbf{J} \mbf{M}_2\mbf{M}_1 = \mbf{M}_1^\trans(\mbf{M}_2^\trans\mbf{J} \mbf{M}_2)\mbf{M}_1 = \mbf{M}_1^\trans \mbf{J} \mbf{M}_1 = \mbf{J}
\end{equation}
所以变换$\mbf{\xi}\mapsto \mbf{\varXi}$也是正则变换,这就证明了第\pageref{chapter10:footnote-正则变换的传递性}页脚注中的结论,有限正则变换可以看作是一系列无限小正则变换的累积,即
\begin{equation}
	\mbf{M} = \prod_{i=1}^\infty \mbf{M}_i(\eps_i,G_i),\quad \eps_i \to 0
\end{equation}

由式\eqref{正则变换的辛条件}可有
\begin{equation*}
	\det(\mbf{M}^{\mathrm{T}} \mbf{J} \mbf{M}) = \det \mbf{J} = 1
\end{equation*}
所以可有
\begin{equation}
	\left|\det \mbf{M}\right| = 1
\end{equation}
即,正则变换Jacobi行列式的绝对值为$1$。


\subsection{正则变换的不变量}

\subsubsection{相体积元不变}

在正则变换前,相体积元为
\begin{equation*}
	\prod_{i=1}^{2s} \mathrm{d} \xi_i = \prod_{\alpha=1}^s \mathrm{d} q_\alpha \mathrm{d} p_\alpha
\end{equation*}
因此,正则变换之后的相体积元为
\begin{equation*}
	\prod_{j=1}^{2s} \mathrm{d} \mathnormal{\Xi}_j = \prod_{\beta=1}^s \mathrm{d} Q_\beta \mathrm{d} P_\beta = \left|\det\frac{\pl \mrm{\Xi}}{\pl \mbf{\xi}}\right| \prod_{i=1}^{2s} \mathrm{d} \xi_i = \left|\det \mbf{M}\right| \prod_{i=1}^{2s} \mathrm{d} \xi_i = \prod_{i=1}^{2s} \mathrm{d} \xi_i
\end{equation*}
因此,相体积为不依赖于正则变量选取的绝对量。这是经典统计物理的理论基础。

\subsubsection{Poisson括号不变}

设有
\begin{align*}
	& f = f_\xi(\mbf{\xi},t) = f_{\mathnormal{\Xi}}(\mrm{\Xi},t) \\
	& g = g_\xi(\mbf{\xi},t) = g_{\mathnormal{\Xi}}(\mrm{\Xi},t)
\end{align*}
变换之前的Poisson括号为
\begin{equation*}
	[f,g]_\xi = \sum_{\alpha=1}^s \frac{\pl (f,g)}{\pl (q_\alpha,p_\alpha)} = \left(\frac{\pl f}{\pl \mbf{\xi}}\right)^{\mathrm{T}} \mbf{J} \frac{\pl g}{\pl \mbf{\xi}}
\end{equation*}
正则变换之后的Poisson括号为
\begin{align*}
	[f,g]_{\mathnormal{\Xi}} & = \left(\frac{\pl f}{\pl \mrm{\Xi}}\right)^{\mathrm{T}} \mbf{J} \frac{\pl g}{\pl \mrm{\Xi}} = \left(\frac{\pl \mbf{\xi}}{\pl \mrm{\Xi}} \frac{\pl f}{\pl \mbf{\xi}} \right)^{\mathrm{T}} \mbf{J} \frac{\pl \mbf{\xi}}{\pl \mrm{\Xi}}  \frac{\pl g}{\pl \mbf{\xi}} \\
	& = \left(\frac{\pl f}{\pl \mbf{\xi}}\right)^{\mathrm{T}} \mbf{M}^{-\mathrm{T}} \mbf{J} \mbf{M}^{-1} \frac{\pl g}{\pl \mbf{\xi}}
\end{align*}
由式\eqref{正则变换的辛条件}可有
\begin{equation*}
	\mbf{M}^{-\mathrm{T}} \mbf{J} \mbf{M}^{-1} = \mbf{J}
\end{equation*}
所以有
\begin{align*}
	[f,g]_{\mathnormal{\Xi}} & = \left(\frac{\pl f}{\pl \mbf{\xi}}\right)^{\mathrm{T}} \mbf{J} \frac{\pl g}{\pl \mbf{\xi}} = [f,g]_\xi
\end{align*}
因此,Poisson括号为不依赖于正则变量选取的绝对量。力学Poisson括号形式为不依赖于相坐标系的绝对表示。因此相应地,基本Poisson括号在正则变换下不变,此可作为正则变换的判据。

角动量分量不符合基本泊松括号,因此不可能同时选为正则变量。

\begin{example}
若新旧正则变量的变换关系为
\begin{equation*}
	Q = q^\alpha \cos \beta p,\quad P = q^\alpha \sin \beta p
\end{equation*}
当$\alpha,\beta$为何值时,这个变换是正则变换?并找出其母函数。
\end{example}
\begin{solution}
考虑基本泊松括号
\begin{equation*}
	[Q,P]_\xi = \begin{vmatrix} \dfrac{\pl Q}{\pl q} & \dfrac{\pl Q}{\pl p} \\[1.5ex] \dfrac{\pl P}{\pl q} & \dfrac{\pl P}{\pl p} \end{vmatrix} = \alpha \beta q^{2\alpha-1} = 1 = [Q,P]_{\mathnormal{\Xi}}
\end{equation*}
由此可得$\alpha=\dfrac12,\beta = 2$。所以
\begin{equation*}
	\begin{cases}
		\displaystyle Q = \sqrt{q} \cos 2p \\
		\displaystyle P = \sqrt{q} \sin 2p
	\end{cases}
\end{equation*}
因此有
\begin{equation*}
	\begin{cases}
		\displaystyle q = \frac{Q^2}{\cos^2 2p} \\
		\displaystyle P = Q\tan 2p
	\end{cases}
\end{equation*}
考虑第三类母函数$F_3 = F_3(p,Q)$,则有
\begin{equation*}
	\mathrm{d} F_3 = -q\mathrm{d} p - P\mathrm{d} Q = -\frac{Q^2}{\cos^2 2p} \mathrm{d} p - Q\tan 2p \mathrm{d}Q = \mathrm{d} \left(-\frac12 Q^2 \tan 2p\right)
\end{equation*}
即有第三类母函数
\begin{equation*}
	F_3(p,Q) = -\frac12 Q^2 \tan 2p
\end{equation*}
\end{solution}

\section{Hamilton-Jacobi方程}\label{chapter10:section-Hamilton-Jacobi方程}

第\ref{chapter10:subsection-无穷小正则变换}节已经提到,可以找到这样的正则变换,它将所有正则变量都变成常数,在这种情况下的正则方程就变为
\begin{equation}
	\begin{cases}
		\dot{Q}_\alpha = 0, \\
		\dot{P}_\alpha = 0,
	\end{cases}\,\, (\alpha = 1,2,\cdots,s)
\end{equation}
最简单的情况就是变换后的Hamilton函数$K(\mbf{Q},\mbf{P},t)$恒等于零,于是所有正则变量都是循环变量,即有
\begin{equation}
	\begin{cases}
		Q_\alpha = C_\alpha = \text{常数}, \\
		P_\alpha = D_\alpha = \text{常数},
	\end{cases}\,\, (\alpha=1,2,\cdots,s)
\end{equation}
然后再利用变换关系
\begin{equation}
	\begin{cases}
		Q_\alpha = Q_\alpha(\mbf{q},\mbf{p},t), \\
		P_\alpha = P_\alpha(\mbf{q},\mbf{p},t),
	\end{cases}\,\, (\alpha = 1,2,\cdots,s)
\end{equation}
即可得到原来正则方程的解
\begin{equation}
	\begin{cases}
		q_\alpha = q_\alpha(\mbf{C},\mbf{D},t) \\
		p_\alpha = p_\alpha(\mbf{C},\mbf{D},t)
	\end{cases}
\end{equation}
对于四类正则变换母函数,新旧Hamilton函数间的关系都是
\begin{equation}
	K(\mbf{Q},\mbf{P},t) = H(\mbf{q},\mbf{p},t) + \frac{\pl S}{\pl t}
\end{equation}
此处将母函数记作$S$,欲使$K = 0$,则要求母函数$S$满足关系式
\begin{equation}
	H(\mbf{q},\mbf{p},t) + \frac{\pl S}{\pl t} = 0
\end{equation}
由此,求解这个正则变换的问题便归结为求$S$。

\subsection{Hamilton-Jacobi方程}

考虑第二类母函数$F_2 = S(\mbf{q},\mbf{P},t)$,则由变换关系式\eqref{chapter10:第二类母函数下的正则变换-变换式}可有
\begin{equation}
	p_\alpha = \frac{\pl S}{\pl q_\alpha}
	\label{chapter10:Hamilton-Jacobi方程-动量与Hamilton主函数之间的关系}
\end{equation}
并根据变换需满足的条件,令
\begin{equation}
	P_\alpha = D_\alpha\quad \text{(常数)}
\end{equation}
即有
\begin{equation}
	H\left(\mbf{q},\frac{\pl S}{\pl \mbf{q}},t\right) + \frac{\pl S}{\pl t} = 0
	\label{Hamilton-Jacobi方程}
\end{equation}
式\eqref{Hamilton-Jacobi方程}称为{\heiti Hamilton-Jacobi方程}。它是函数$S(\mbf{q},\mbf{D},t)$的一阶偏微分方程,有$s+1$个自变量($\mbf{q}$和$t$),所以应有$s+1$个独立的积分常数,其中一个为相加常数,不影响变换,可略去;其余的$s$个即为$D_\alpha$。含有$s$个任意常数(不包括相加常数)的函数$S(\mbf{q},\mbf{D},t)$称为Hamilton-Jacobi方程的{\bf 完全积分}。
%利用Hamilton-Jacobi方程\eqref{Hamilton-Jacobi方程}得到函数$S=S(\mbf{q},\mbf{D},t)$之后,便可利用关系
% \begin{equation*}
	% \frac{\pl S}{\pl D_\alpha}(\mbf{q},\mbf{D},t) = C_\alpha,\quad \alpha=1,2,\cdots,s
% \end{equation*}
% 来直接得到系统的轨道方程$\mbf{q}=\mbf{q}(t)$,其中$C_\alpha$为任意常数。

\subsection{Hamilton主函数}

在Hamilton-Jacobi方程\eqref{Hamilton-Jacobi方程}中,函数$S(\mbf{q},\mbf{D},t)$称为{\heiti Hamilton主函数}。下面来讨论Hamilton主函数的物理意义。考虑
\begin{equation}
	\frac{\mathrm{d} S}{\mathrm{d} t} = \sum_{\alpha=1}^s \frac{\pl S}{\pl q_\alpha} \dot{q}_\alpha + \frac{\pl S}{\pl t} = \sum_{\alpha=1}^s p_\alpha \dot{q}_\alpha - H = L
\end{equation}
由此可得
\begin{equation}
	S = \int L \mathrm{d} t
	\label{Hamilton作用函数}
\end{equation}
函数$S(\mbf{q},\mbf{D},t)$即为积分限不定的Hamilton作用量,因此也称为{\heiti Hamilton作用函数}。

当Hamilton函数不显含时间$t$,即系统广义能量守恒时\footnote{纯力学问题都属于这一类问题。},这种情况下方程\eqref{Hamilton-Jacobi方程}可以表示为
\begin{equation}
	H\left(\mbf{q},\frac{\pl S}{\pl \mbf{q}}\right) + \frac{\pl S}{\pl t} = E + \frac{\pl S}{\pl t} = 0
\end{equation}
此情况下Hamilton作用函数可分离变量
\begin{equation*}
	S = S(\mbf{q},\mbf{D},t) = W(\mbf{q},\mbf{D}) + T(t)
\end{equation*}
由此即有
\begin{equation}
	S(\mbf{q},\mbf{D},t) = W(\mbf{q},\mbf{D}) - Et
\end{equation}
Hamilton-Jacobi方程变为
\begin{equation}
	H\left(\mbf{q},\frac{\pl W}{\pl \mbf{q}}\right) = E
	\label{不含时Hamilton-Jacobi方程}
\end{equation}
式\eqref{不含时Hamilton-Jacobi方程}称为{\heiti 不含时Hamilton-Jacobi方程}。其中函数
\begin{equation*}
	W = W(\mbf{q},\mbf{D}) = W(\mbf{q},E,D_2,\cdots,D_s)
\end{equation*}
称为{\heiti Hamilton特征函数}。

考虑到$H=E$,可有
\begin{equation*}
	S = \int L \mathrm{d} t = \int \left(\sum_{\alpha=1}^s p_\alpha \dot{q}_\alpha - H\right) \mathrm{d} t = \int \sum_{\alpha=1}^s p_\alpha \mathrm{d} q_\alpha - Et + D_0
\end{equation*}
上式中$D_0$是相加常数,可略去,由此可得Hamilton特征函数的具体形式
\begin{equation}
	W(\mbf{q},\mbf{D}) = \int \sum_{\alpha=1}^s p_\alpha \mathrm{d} q_\alpha
\end{equation}
函数$W(\mbf{q},\mbf{D})$即为积分限不定的Lagrange作用量(见式\eqref{chapter10:Lagrange作用量})。

下面来考虑Hamilton-Jacobi方程的完全积分$S=S(\mbf{q},\mbf{D},t)$与系统运动轨迹$\mbf{q} = \mbf{q}(t),\mbf{p} = \mbf{p}(t)$之间的关系。首先有
\begin{equation}
\begin{cases}
	p_\alpha = \dfrac{\pl S}{\pl q_\alpha}(\mbf{q},\mbf{D},t) \\[1.5ex]
	C_\alpha = \dfrac{\pl S}{\pl D_\alpha}(\mbf{q},\mbf{D},t)
\end{cases}\,\, (\alpha=1,2,\cdots,s)
\label{chp3:Hamilton-Jacobi方程的完全积分与运动轨道-1}
\end{equation}
由方程\eqref{chp3:Hamilton-Jacobi方程的完全积分与运动轨道-1}第二式的$s$个方程中(这是代数方程组)可以解出$s$个广义坐标与时间的关系
\begin{equation*}
	\mbf{q} = \mbf{q}(t)
\end{equation*}
再由方程\eqref{chp3:Hamilton-Jacobi方程的完全积分与运动轨道-1}第一式即可得
\begin{equation*}
	\mbf{p} = \mbf{p}(t)
\end{equation*}
由此便得到完整的相空间轨迹。当系统广义能量守恒时,记$E=D_1$,则有
\begin{subnumcases}{}
	p_\alpha = \frac{\pl S}{\pl q_\alpha}(\mbf{q},\mbf{D},t) = \frac{\pl W}{\pl q_\alpha}(\mbf{q},\mbf{D}) & $(\alpha=1,2,\cdots,s)$ \\
	C_1 = \frac{\pl S}{\pl D_1}(\mbf{q},\mbf{D},t) = \frac{\pl W}{\pl E}(\mbf{q},\mbf{D}) - t \\
	C_\alpha = \frac{\pl S}{\pl D_\alpha} = \frac{\pl W}{\pl D_\alpha}(\mbf{q},\mbf{D})& $(\alpha=2,3,\cdots,s)$
\end{subnumcases}
由此可有
\begin{equation}
	C_\alpha = C_\alpha(q_1,q_2,\cdots,q_s,E,D_2,\cdots,D_s) \quad (\alpha=2,3,\cdots,s)
	\label{chp3:Hamilton-Jacobi方程的完全积分与运动轨道-3}
\end{equation}
由式\eqref{chp3:Hamilton-Jacobi方程的完全积分与运动轨道-3}的$s-1$个方程,可以以$q_1$为参数,解出$q_2,\cdots,q_s$的表达式为
\begin{equation}
	q_\alpha = f_\alpha(q_1,C_2,\cdots,C_s,E,D_2,\cdots,D_s) \quad (\alpha = 2,3,\cdots,s)
\end{equation}
此即为位形空间的轨道曲线。

\begin{example}[一维谐振子的Hamilton-Jacobi方程求解]
一维谐振子的Hamilton函数为
\begin{equation*}
	H(q,p) = \frac{p^2}{2m} + \frac12 m\omega^2 q^2
\end{equation*}
用Hamilton-Jacobi方程求解此系统的运动。
\end{example}
\begin{solution}
系统的Hamilton函数不显含时间,故根据不含时Hamilton-Jacobi方程\eqref{不含时Hamilton-Jacobi方程}可有
\begin{equation*}
	H\left(q,\frac{\pl W}{\pl q}\right) = \frac12 \left(\frac{\pl W}{\pl q}\right)^2 + \frac12 m\omega^2 q^2 = E
\end{equation*}
即
\begin{equation*}
	\frac{\pl W}{\pl q} = m\omega \sqrt{\frac{2E}{m\omega^2} - q^2} = p
\end{equation*}
因此
\begin{equation*}
	W = m\omega \int \sqrt{\frac{2E}{m\omega^2} - q^2} \mathrm{d} q
\end{equation*}
由此即有
\begin{equation*}
	C = \frac{\pl W}{\pl E} - t = \frac{1}{\omega} \int \frac{\mathrm{d} q}{\sqrt{\dfrac{2E}{m\omega^2}-q^2}} - t = \frac{1}{\omega} \arcsin\left(\sqrt{\frac{m\omega^2}{2E}}q\right) - t
\end{equation*}
由此可得系统的解
\begin{equation*}
\begin{cases}
	\displaystyle q = \sqrt{\frac{2E}{m\omega^2}} \sin \big[\omega(t+C)\big] \\[1.5ex]
	\displaystyle p = m\omega \sqrt{\frac{2E}{m\omega^2} - q^2} = \sqrt{2mE} \cos \big[\omega(t+C)\big]
\end{cases}
\end{equation*}
$C,E$是积分常数,故有
\begin{equation*}
	C = \frac{1}{\omega} \arcsin\left(\sqrt{\frac{m\omega^2}{2E}}q_0\right) - t_0,\quad E = \frac{p_0^2}{2m} + \frac12 m\omega^2 q_0^2
\end{equation*}
\end{solution}

Hamilton-Jacobi理论将常微分方程组求解转化为偏微分方程的全积分问题,在具体应用方面并无方便之处,但是提供了一种全新的力学运动图像。

\subsection{Hamilton-Jacobi方程的分离变量}

在某些特殊情况下,Hamilton-Jacobi方程的完全积分可以通过分离变量的方法求得。假设某一个坐标,将其记作$q_1$,与其相应的导数$\dfrac{\pl S}{\pl q_1}$,在Hamilton-Jacobi方程中仅以某种组合$\phi\left(q_1,\dfrac{\pl S}{\pl q_1}\right)$的方式出现,而这个组合中不包含任何其他坐标,即Hamilton-Jacobi方程\eqref{Hamilton-Jacobi方程}具有如下形式:
\begin{equation}
	\varPhi\left(\mbf{q}',t,\frac{\pl S}{\pl \mbf{q}'},\frac{\pl S}{\pl t},\phi\left(q_1,\dfrac{\pl S}{\pl q_1}\right)\right) = 0
	\label{chp3:Hamilton-Jacobi方程的分离变量-1}
\end{equation}
其中$\mbf{q}'$表示除了$q_1$以外的所有坐标。

假设解具有如下形式
\begin{equation}
	S(\mbf{q},t) = S_0(\mbf{q}',t)+S_1(q_1)
	\label{chp3:Hamilton-Jacobi方程的分离变量-2}
\end{equation}
将\eqref{chp3:Hamilton-Jacobi方程的分离变量-2}代入式\eqref{chp3:Hamilton-Jacobi方程的分离变量-1}中,即有
\begin{equation}
	\varPhi\left(\mbf{q}',t,\frac{\pl S_0}{\pl \mbf{q}'},\frac{\pl S_0}{\pl t},\phi\left(q_1,\dfrac{\mathrm{d} S_1}{\mathrm{d} q_1}\right)\right) = 0
	\label{chp3:Hamilton-Jacobi方程的分离变量-3}
\end{equation}
假设函数$\varPhi$满足$\dfrac{\pl \varPhi}{\pl \phi} \neq 0$\footnote{对于通常的系统方程来说,这个假设是十分合理的。},则可以在方程\eqref{chp3:Hamilton-Jacobi方程的分离变量-3}中解出$\phi$,并将其表示为:
\begin{equation}
	\varPsi\left(\mbf{q}',t,\frac{\pl S_0}{\pl \mbf{q}'},\frac{\pl S_0}{\pl t}\right) = \phi\left(q_1,\dfrac{\mathrm{d} S_1}{\mathrm{d} q_1}\right)
	\label{chapter10:Hamilton-Jacobi方程的分离变量-补充式1}
\end{equation}
在方程\eqref{chapter10:Hamilton-Jacobi方程的分离变量-补充式1}中,其左端是$\mbf{q}'$和$t$的函数,右端则只是$q_1$的函数,因此两端都只能是常数,于是将方程\eqref{chp3:Hamilton-Jacobi方程的分离变量-1}分为两个方程
\begin{align}
	& \phi\left(q_1,\dfrac{\mathrm{d} S_1}{\mathrm{d} q_1}\right) = D_1 \label{chp3:Hamilton-Jacobi方程的分离变量-4.1} \\
	& \varPhi\left(\mbf{q}',t,\frac{\pl S_0}{\pl \mbf{q}'},\frac{\pl S_0}{\pl t},D_1\right) = 0 \label{chp3:Hamilton-Jacobi方程的分离变量-4.2}
\end{align}
方程\eqref{chp3:Hamilton-Jacobi方程的分离变量-4.1}是一个常微分方程,而方程\eqref{chp3:Hamilton-Jacobi方程的分离变量-4.2}虽然仍为偏微分方程,但其独立变量数目较之原来减少了一个。如果类似的操作还可以进行,那么可以继续对方程\eqref{chp3:Hamilton-Jacobi方程的分离变量-4.2}重复上面的操作,继续分离变量,逐渐减少偏微分方程\eqref{chp3:Hamilton-Jacobi方程的分离变量-4.2}中的独立变量数目。

对于循环坐标的情形,由于循环坐标$q_1$不显含于Hamilton函数中,此时$\phi\left(q_1,\dfrac{\pl S}{\pl q_1}\right)$项即为$\dfrac{\pl S}{\pl q_1}$,因此可有
\begin{equation*}
	S_1(q_1) = D_1q_1
\end{equation*}
而此处常数$D_1$即为与$q_1$共轭的广义动量$p_1=\dfrac{\pl S}{\pl q_1}=D_1$。对于保守系统而言,$-Et$这一项即对应着“循环坐标”时间$t$的分离。

为了能够在Hamilton-Jacobi方程中分离变量,适当选择坐标系非常关键。

\begin{example}[球坐标系中Hamilton-Jacobi方程的分离变量]
在球坐标系中,Hamilton函数为
\begin{equation}
	H(r,\theta,\phi,p_r,p_\theta,p_\phi,t) = \frac{1}{2m}\left(p_r^2+\frac{p_\theta^2}{r^2}+\frac{p_\phi^2}{r^2\sin^2\theta}\right)+V(r,\theta,\phi,t)
	\label{球坐标系中Hamilton-Jacobi方程的分离变量-1.1}
\end{equation}
如果势函数$V(r,\theta,\phi,t)$具有如下形式
\begin{equation}
	V(r,\theta,\phi,t) = a(r)+\frac{b(\theta)}{r^2}+\frac{c(\phi)}{r^2\sin^2\theta}
	\label{球坐标系中Hamilton-Jacobi方程的分离变量-1.2}
\end{equation}
其中$a(r),b(\theta),c(\phi)$是任意函数,则分离变量是可能的。此时关于函数$W(r,\theta,\phi)$的不含时Hamilton-Jacobi方程为
\begin{align}
	\frac{1}{2m}\left(\frac{\pl W}{\pl r}\right)^2+a(r)& {}+\frac{1}{2mr^2}\left[\left(\frac{\pl W}{\pl \theta}\right)^2+2mb(\theta)\right] \nonumber \\
	& {}+ \frac{1}{2mr^2\sin^2\theta}\left[\left(\frac{\pl W}{\pl \phi}\right)^2+2mc(\phi)\right] = E
	\label{球坐标系中Hamilton-Jacobi方程的分离变量-2}
\end{align}

假设解具有形式
\begin{equation}
	S(r,\theta,\phi,t) = -Et+W(r,\theta,\phi) = -Et+S_1(r)+S_2(\theta)+S_3(\phi)
	\label{球坐标系中Hamilton-Jacobi方程的分离变量-3}
\end{equation}
则有
\begin{subnumcases}{}
	\left(\frac{\mathrm{d}S_3}{\mathrm{d}\phi}\right)^2+2mc(\phi) = \alpha \label{球坐标系中Hamilton-Jacobi方程的分离变量-4.1} \\
	\left(\frac{\mathrm{d}S_2}{\mathrm{d}\theta}\right)^2+2mb(\theta)+\frac{\alpha}{\sin^2\theta} = \beta \label{球坐标系中Hamilton-Jacobi方程的分离变量-4.2} \\
	\frac{1}{2m}\left(\frac{\mathrm{d}S_1}{\mathrm{d}r}\right)^2+a(r)+\frac{\beta}{2mr^2} = E \label{球坐标系中Hamilton-Jacobi方程的分离变量-4.3}
\end{subnumcases}
依次积分可得
\begin{subnumcases}{}
	S_3(\phi) = \pm\int \sqrt{\alpha-2mc(\phi)} \mathrm{d}\phi \label{球坐标系中Hamilton-Jacobi方程的分离变量-5.1}\\
	S_2(\theta) = \pm\int \sqrt{\beta-2mb(\theta)-\frac{\alpha}{\sin^2\theta}} \mathrm{d}\theta \label{球坐标系中Hamilton-Jacobi方程的分离变量-5.2} \\
	S_1(r) = \pm\int \sqrt{2m\big[E-a(r)\big]-\frac{\beta}{r^2}} \mathrm{d}r \label{球坐标系中Hamilton-Jacobi方程的分离变量-5.3}
\end{subnumcases}
便得到Hamilton-Jacobi方程的完全积分
\begin{align}
	S(r,\theta,\phi,t) & = -Et \pm \int \sqrt{\alpha-2mc(\phi)} \mathrm{d}\phi \pm \int \sqrt{\beta-2mb(\theta)-\frac{\alpha}{\sin^2\theta}} \mathrm{d}\theta \nonumber \\
	& \quad {} \pm \int \sqrt{2m\big[E-a(r)\big]-\frac{\beta}{r^2}} \mathrm{d}r
	\label{球坐标系中Hamilton-Jacobi方程的分离变量-6}
\end{align}
其中$\alpha,\beta,E$是任意积分常数,将式\eqref{球坐标系中Hamilton-Jacobi方程的分离变量-6}对这三个任意常数求导并使之等于新常数,即可得到运动方程的通解。

在一般的问题中,式\eqref{球坐标系中Hamilton-Jacobi方程的分离变量-1.2}所表示的势场其第三项未必有物理意义,因此在$c(\phi)=0$的情形下,可有Hamilton-Jacobi方程的完全积分为
\begin{align}
	S(r,\theta,\phi,t) & = -Et + p_\phi\phi \pm \int \sqrt{\beta-2mb(\theta)-\frac{p_\phi^2}{\sin^2\theta}} \mathrm{d}\theta \pm \int \sqrt{2m\big[E-a(r)\big]-\frac{\beta}{r^2}} \mathrm{d}r
	\label{球坐标系中Hamilton-Jacobi方程的分离变量-6}
\end{align}
\end{example}

\begin{example}[抛物线坐标系中Hamilton-Jacobi方程的分离变量]
抛物线坐标$(u,v,\phi)$与直角坐标$(x,y,z)$之间的关系由如下变换公式给出
\begin{equation}
\begin{cases}
	x = uv\cos\phi \\
	y = uv\sin\phi \\
	z = \dfrac12(u^2-v^2)
\end{cases}
(0\leqslant u,v<+\infty,0\leqslant \phi < 2\pi)
\end{equation}
$u$和$v$为常数的曲面是两族以$z$轴为对称轴的旋转抛物面,如图\ref{chp3:抛物线坐标系}所示。

\begin{figure}[htb]
\centering
\begin{asy}
	size(300);
	//抛物线坐标系的坐标面
	pair O,i,j,k;
	O = (0,0);
	i = (-sqrt(2)/4,-sqrt(14)/12);
	j = (sqrt(14)/4,-sqrt(2)/12);
	k = (0,2*sqrt(2)/3);
	draw(-3*k--2.6*k,dashed);
	draw(Label("$z$",EndPoint),2.6*k--4*k,Arrow);
	real u0,v0,h0,phi0,h,r;
	pair f1(real h){
		real theta,tmp,phi;
		theta = atan(-j.x/i.x);
		tmp = v0/sqrt(v0^2+2*h)*(i.x*j.y-i.y*j.x)/k.y/sqrt(i.x^2+j.x^2);
		phi = asin(tmp)-theta;
		return sqrt(v0^2+2*h)*v0*cos(phi)*i+sqrt(v0^2+2*h)*v0*sin(phi)*j+h*k;
	}
	pair f2(real h){
		real theta,tmp,phi;
		theta = atan(-j.x/i.x);
		tmp = -u0/sqrt(u0^2-2*h)*(i.x*j.y-i.y*j.x)/(k.y*sqrt(i.x^2+j.x^2));
		phi = asin(tmp)-theta;
		return sqrt(u0^2-2*h)*u0*cos(phi)*i+sqrt(u0^2-2*h)*u0*sin(phi)*j+h*k;
	}
	pair para_up(real phi){
		real x,y,z,u;
		u = sqrt(v0^2+2*h);
		x = u*v0*cos(phi);
		y = u*v0*sin(phi);
		z = h;
		return x*i+y*j+z*k;
	}
	pair para_down(real phi){
		real x,y,z,v;
		v = sqrt(u0^2-2*h);
		x = u0*v*cos(phi);
		y = u0*v*sin(phi);
		z = h;
		return x*i+y*j+z*k;
	}
	pair intersection_cur_up(real h){
		real x,y,z,u;
		u = sqrt(v0^2+2*h);
		x = u*v0*cos(phi0);
		y = u*v0*sin(phi0);
		z = h;
		return x*i+y*j+z*k;
	}
	pair intersection_cur_down(real h){
		real x,y,z,v;
		v = sqrt(u0^2-2*h);
		x = u0*v*cos(phi0);
		y = u0*v*sin(phi0);
		z = h;
		return x*i+y*j+z*k;
	}
	path q1,q2,p1,p2;
	picture tmp;
	v0 = 0.6;
	u0 = 0.8;
	h0 = 0.5*(u0^2-v0^2);
	q1 = graph(f1,3,h0);
	p1 = graph(f1,h0,-0.5*v0^2+0.1);
	draw(q1,linewidth(0.8bp)+red);
	draw(p1..xscale(-1)*reverse(p1),dashed+red);
	q2 = graph(f2,-3,h0);
	p2 = graph(f2,h0,0.5*u0^2-0.1);
	draw(q2,linewidth(0.8bp)+blue);
	draw(p2..xscale(-1)*reverse(p2),dashed+blue);
	real tm;
	tm = -1.2;
	h = h0;
	draw(graph(para_up,tm,pi+tm),green+linewidth(0.8bp));
	draw(graph(para_up,pi+tm,2*pi+tm),dashed+green);
	h = 3;
	draw(graph(para_up,0,2*pi),red);
	h = -3;
	draw(graph(para_down,pi+tm,2*pi+tm),dashed+blue);
	phi0 = pi/3;
	r = 3;
	draw(tmp,graph(para_down,tm,pi+tm),blue);
	draw(tmp,xscale(-1)*q1,linewidth(0.8bp)+red);
	draw(tmp,xscale(-1)*q2,linewidth(0.8bp)+blue);
	draw(tmp,Label("$y$",EndPoint),O--3*j,Arrow);
	path clp;
	clp = 3*k--r*cos(phi0)*i+r*sin(phi0)*j+3*k--r*cos(phi0)*i+r*sin(phi0)*j-3*k--(-3*k)--cycle;
	unfill(tmp,clp);
	add(tmp);
	erase(tmp);
	draw(tmp,graph(para_down,tm,pi+tm),blue+dashed);
	draw(tmp,xscale(-1)*q1,dashed+red);
	draw(tmp,xscale(-1)*q2,dashed+blue);
	draw(tmp,Label("$y$",EndPoint),O--3*j,dashed);
	clip(tmp,clp);
	add(tmp);
	draw(3*k--r*cos(phi0)*i+r*sin(phi0)*j+3*k--r*cos(phi0)*i+r*sin(phi0)*j-3*k--intersection_cur_down(-3),green);
	draw(intersection_cur_down(-3)--(-3*k),green+dashed);
	draw(graph(intersection_cur_up,3,h0),red+linewidth(0.8bp));
	draw(graph(intersection_cur_up,h0,-0.5*v0^2),red+dashed);
	draw(graph(intersection_cur_down,-3,h0),blue+linewidth(0.8bp));
	draw(graph(intersection_cur_down,h0,0.5*u0^2),blue+dashed);
	dot(intersection_cur_up(h0));
	draw(Label("$x$",EndPoint),u0^2*i--6*i,Arrow);
	draw(O--u0^2*i,dashed);
	pair P,eu,ev,ephi;
	eu = v0*cos(phi0)/sqrt(u0^2+v0^2)*i+v0*sin(phi0)/sqrt(u0^2+v0^2)*j+u0/sqrt(u0^2+v0^2)*k;
	ev = u0*cos(phi0)/sqrt(u0^2+v0^2)*i+u0*sin(phi0)/sqrt(u0^2+v0^2)*j-v0/sqrt(u0^2+v0^2)*k;
	ephi = -sin(phi0)*i+cos(phi0)*j;
	P = intersection_cur_up(h0);
	draw(Label("$\boldsymbol{e}_u$",EndPoint),P--P+eu,Arrow);
	draw(Label("$\boldsymbol{e}_v$",EndPoint),P--P+ev,Arrow);
	draw(Label("$\boldsymbol{e}_\phi$",EndPoint),P--P+ephi,Arrow);
	label("$O$",O,WNW);
\end{asy}
\caption{抛物线坐标系的坐标面}
\label{chp3:抛物线坐标系}
\end{figure}

在这个坐标系下,系统的Lagrange函数为
\begin{equation}
	L = \frac12 m(u^2+v^2)\dot{u}^2+\frac12 m(u^2+v^2)\dot{v}^2+\frac12 mu^2v^2\dot{\phi}^2 - V(u,v,\phi)
\end{equation}
此时,广义动量为
\begin{equation*}
	p_u = \frac{\pl L}{\pl \dot{u}} = m(u^2+v^2)\dot{u},\quad p_v = \frac{\pl L}{\pl \dot{v}} = m(u^2+v^2)\dot{v},\quad p_\phi = \frac{\pl L}{\pl \dot{\phi}} = mu^2v^2\dot{\phi}
\end{equation*}
由此可得系统的Hamilton函数为
\begin{equation}
	H = \frac{p_u^2+p_v^2}{2m(u^2+v^2)} + \frac{p_\phi^2}{2mu^2v^2} + V(u,v,\phi)
\end{equation}

在这种坐标下,物理上有意义的可分离变量情况对应于以下形式的势能:
\begin{equation}
	V(u,v,\phi) = \frac{a(u)+b(v)}{u^2+v^2}
\end{equation}
所以我们得到关于$W(\mbf{q},\mbf{D})$的方程
\begin{equation}
	\frac{1}{2m(u^2+v^2)}\left[\left(\frac{\pl W}{\pl u}\right)^2+\left(\frac{\pl W}{\pl v}\right)^2\right]+ \frac{1}{2mu^2v^2}\left(\frac{\pl W}{\pl \phi}\right)^2 + \frac{a(u)+b(v)}{u^2+v^2} = E
	\label{chp3:抛物线坐标下的分离变量-1}
\end{equation}
在式\eqref{chp3:抛物线坐标下的分离变量-1}两端乘以$2m(u^2+v^2)$并整理,可得
\begin{align*}
	\left(\frac{\pl W}{\pl u}\right)^2 + a(u) - 2mEu^2 + \left(\frac{\pl W}{\pl v}\right)^2 +b(u) - 2mEv^2 + \left(\frac{1}{u^2}+\frac{1}{v^2}\right)\left(\frac{\pl W}{\pl \phi}\right)^2 = 0
\end{align*}
此处注意到坐标$\phi$是循环坐标,因此可令
\begin{equation}
	S(u,v,\phi,t) = -Et + p_\phi\phi + S_1(u)+S_2(v)
\end{equation}
由此得到两个方程
\begin{subnumcases}{}
	\left(\frac{\mathrm{d}S_2}{\mathrm{d} v}\right)^2 - 2mEv^2+b(v)+\frac{p_\phi^2}{v^2} = \beta \\
	\left(\frac{\mathrm{d}S_1}{\mathrm{d} u}\right)^2 - 2mEu^2+a(u)+\frac{p_\phi^2}{u^2} = -\beta
\end{subnumcases}
由此即有
\begin{align*}
	S & = -Et + p_\phi\phi \pm \int \sqrt{2mEv^2-\frac{p_\phi^2}{v^2}-b(v)+\beta} \mathrm{d}v \pm \int \sqrt{2mEu^2-\frac{p_\phi^2}{u^2}-a(u)-\beta} \mathrm{d}u
\end{align*}
其中$E,p_\phi,\beta$为任意积分常数。
\end{example}

\begin{example}[椭圆坐标系中Hamilton-Jacobi方程的分离变量]
椭圆坐标$(\xi,\eta,\phi)$与直角坐标$(x,y,z)$之间的关系由如下变换公式给出
\begin{equation}
\begin{cases}
	x = \sigma\sqrt{(\xi^2-1)(1-\eta^2)} \cos \phi \\
	y = \sigma\sqrt{(\xi^2-1)(1-\eta^2)} \sin \phi \\
	z = \sigma \xi\eta
\end{cases}(1\leqslant \xi<+\infty,-1\leqslant \eta\leqslant 1,0\leqslant \phi < 2\pi)
\end{equation}
其中$\sigma$为常数。设$z$轴上两个点$A_1$和$A_2$的坐标为$z=\sigma$和$z=-\sigma$,则$\xi$为常数的坐标面为以$A_1$和$A_2$为焦点的一族旋转椭球面
\begin{equation*}
	\frac{x^2+y^2}{\sigma^2(\xi^2-1)}+ \frac{z^2}{\sigma^2\xi^2} = 1
\end{equation*}
而$\eta$为常数的坐标面为以$A_1$和$A_2$为焦点的一族旋转双叶双曲面
\begin{equation*}
	-\frac{x^2+y^2}{\sigma^2(1-\eta^2)}+ \frac{z^2}{\sigma^2\eta^2} = 1
\end{equation*}
此坐标系即为图\ref{chp3:二维椭圆坐标系}所示的二维椭圆坐标系绕其$z$轴旋转而成。

\begin{figure}[htb]
\centering
\begin{asy}
	size(250);
	//二维椭圆坐标系
	picture tmp;
	real xi,eta,sigma,xmax,ymax;
	sigma = 1;
	xmax = 2.5;
	ymax = 2.5;
	pair coor_ell(real theta){
		return (sigma*sqrt(xi^2-1)*cos(theta),sigma*xi*sin(theta));
	}
	pair coor_hyp(real theta){
		return (sigma*sqrt(1-eta^2)*sinh(theta),sigma*eta*cosh(theta));
	}
	for(xi=1;xi<=2.2;xi=xi+0.15){
		draw(tmp,graph(coor_ell,0,2*pi),red);
	}
	for(eta=0;eta<1;eta=eta+0.15){
		if (eta==0) continue;
		draw(tmp,graph(coor_hyp,-10,10),blue);
	}
	for(eta=0;eta>-1;eta=eta-0.15){
		if (eta==0) continue;
		draw(tmp,graph(coor_hyp,-10,10),blue);
	}
	clip(tmp,box((-xmax,-ymax),(xmax,ymax)));
	add(tmp);
	draw(Label("$x$",EndPoint),(-xmax,0)--(xmax,0),Arrow);
	draw(Label("$z$",EndPoint),(0,-ymax)--(0,ymax),Arrow);
	dot("$A_1$",(0,sigma),E);
	dot("$A_2$",(0,-sigma),E);
\end{asy}
\caption{二维椭圆坐标系}
\label{chp3:二维椭圆坐标系}
\end{figure}

记$r_1$和$r_2$是点$(\xi,\eta,\phi)$到$A_1$和$A_2$的距离,则有
\begin{subnumcases}{}
	r_1 = \sqrt{x^2+y^2+(z-\sigma)^2} = \sigma(\xi-\eta) \\
	r_2 = \sqrt{x^2+y^2+(z+\sigma)^2} = \sigma(\xi+\eta)
\end{subnumcases}

在这个坐标系下,系统的Lagrange函数为
\begin{equation}
	L = \frac12 m\sigma^2 (\xi^2-\eta^2)\left(\frac{\dot{\xi}^2}{\xi^2-1} + \frac{\dot{\eta}^2}{1-\eta^2}\right) + \frac12 m\sigma^2(\xi^2-1)(1-\eta^2)\dot{\phi}^2 - V(\xi,\eta,\phi)
	\label{chp3:椭圆坐标下的分离变量-1}
\end{equation}
此时,广义动量为
\begin{equation*}
	p_\xi = \frac{\pl L}{\pl \dot{\xi}} = m\sigma^2 \frac{\xi^2-\eta^2}{\xi^2-1} \dot{\xi},\quad p_\eta = \frac{\pl L}{\pl \dot{\eta}} = m\sigma^2 \frac{\xi^2-\eta^2}{1-\eta^2} \dot{\eta},\quad p_\phi = \frac{\pl L}{\pl \dot{\phi}} = m\sigma^2(\xi^2-1)(1-\eta^2)\dot{\phi}
\end{equation*}
由此可得系统的Hamilton函数为
\begin{equation}
	H = \frac{1}{2m\sigma^2(\xi^2-\eta^2)}\left[(\xi^2-1)p_\xi^2+(1-\eta^2)p_\eta^2+\left(\frac{1}{\xi^2-1}+\frac{1}{1-\eta^2}\right)p_\phi^2\right] + V(\xi,\eta,\phi)
	\label{chp3:椭圆坐标下的分离变量-2}
\end{equation}

在这种坐标下,物理上有意义的可分离变量情况对应于势能为
\begin{equation}
	V(\xi,\eta,\phi) = \frac{a(\xi)+b(\eta)}{\xi^2-\eta^2} = \frac{\sigma^2}{r_1r_2}\left[a\left(\frac{r_1+r_2}{2\sigma}\right)+b\left(\frac{r_1-r_2}{2\sigma}\right)\right]
	\label{chp3:椭圆坐标下的分离变量-3}
\end{equation}
所以我们得到关于$W(\mbf{q},\mbf{D})$的方程
\begin{align}
	\frac{1}{2m\sigma^2(\xi^2-\eta^2)} & \left[(\xi^2-1)\left(\frac{\pl W}{\pl \xi}\right)^2+(1-\eta^2)\left(\frac{\pl W}{\pl \eta}\right)^2+\left(\frac{1}{\xi^2-1}+\frac{1}{1-\eta^2}\right)\left(\frac{\pl W}{\pl \phi}\right)^2\right] \nonumber \\
	& \quad \quad \quad {} + \frac{a(\xi)+b(\eta)}{\xi^2-\eta^2} = E
	\label{chp3:椭圆坐标下的分离变量-4}
\end{align}
在式\eqref{chp3:椭圆坐标下的分离变量-4}两端乘以$2m\sigma^2(\xi^2-\eta^2)$并整理,可得
\begin{align*}
	(\xi^2-1)\left(\frac{\pl W}{\pl \xi}\right)^2+2m\sigma^2a(\xi)-2m\sigma^2E\xi^2 & +(1-\eta^2)\left(\frac{\pl W}{\pl \eta}\right)^2+2m\sigma^2b(\eta)+2m\sigma^2E\eta^2 \\
	& {}+\left(\frac{1}{\xi^2-1}+\frac{1}{1-\eta^2}\right)\left(\frac{\pl W}{\pl \phi}\right)^2 = 0
\end{align*}
注意到坐标$\phi$是循环坐标,因此可令
\begin{equation}
	S(\xi,\eta,\phi,t) = -Et+p_\phi\phi + S_1(\xi)+S_2(\eta)
\end{equation}
由此得到两个方程
\begin{subnumcases}{}
	(1-\eta^2)\left(\frac{\mathrm{d}S_2}{\mathrm{d}\eta}\right)^2+2m\sigma^2b(\eta)-2m\sigma^2E(1-\eta^2)+\frac{p_\phi^2}{1-\eta^2} = \beta \\
	(\xi^2-1)\left(\frac{\mathrm{d}S_1}{\mathrm{d}\xi}\right)^2+2m\sigma^2a(\xi)-2m\sigma^2E(\xi^2-1)+\frac{p_\phi^2}{\xi^2-1} = -\beta
\end{subnumcases}
由此即有
\begin{align}
	S & = -Et+p_\phi\phi \pm \int \sqrt{2m\sigma^2E+\frac{\beta-2m\sigma^2b(\eta)}{1-\eta^2}- \frac{p_\phi^2}{(1-\eta^2)^2}} \mathrm{d}\eta \nonumber \\
	& \quad {} \pm \int \sqrt{2m\sigma^2E-\frac{\beta+2m\sigma^2a(\xi)}{\xi^2-1}- \frac{p_\phi^2}{(\xi^2-1)^2}} \mathrm{d}\xi
\end{align}
其中$E,p_\phi,\beta$为任意积分常数。
\end{example}

\subsection{利用Hamilton-Jacobi方程求解Kepler问题}\label{chapter10:subsection-利用Hamilton-Jacobi方程求解Kepler问题}

本节利用Hamilton-Jacobi方程求解Kepler问题,即求解质量为$\mu$的质点在Kepler势$V = -\dfrac{\alpha}{r}(\alpha>0)$中的运动。

在天体力学中,习惯上以太阳为原点\footnote{严格来讲应该以日地系统的质心为原点。},以地球的赤道平面为$xOy$平面,以北极方向为$z$轴建立球坐标系来研究地球或其它行星的运动,这样选取的球坐标系称为{\bf 天球坐标系},角度$\theta$称为{\bf 赤经},角度$\dfrac\pi2-\phi$称为{\bf 赤纬}。本节的讨论虽然研究的不限于地球的轨道,但仍然沿用这些名词。在球坐标系中,此系统的Hamilton函数为
\begin{equation}
	H = \frac{p_r^2}{2\mu}+\frac{p_\theta^2}{2\mu r^2}+\frac{p_\phi^2}{2\mu r^2\sin^2\theta} - \frac{\alpha}{r}
	\label{Hamilton-Jacobi方程解Kepler问题1}
\end{equation}
系统的Hamilton函数不显含时间,故根据不含时Hamilton-Jacobi方程\eqref{不含时Hamilton-Jacobi方程}可有
\begin{equation}
	\frac{1}{2\mu}\left[\left(\frac{\pl W}{\pl r}\right)^2 + \frac{1}{r^2}\left(\frac{\pl W}{\pl \theta}\right)^2 + \frac{1}{r^2\sin^2\theta}\left(\frac{\pl W}{\pl \phi}\right)^2\right] - \frac{\alpha}{r} = E
	\label{Hamilton-Jacobi方程解Kepler问题2}
\end{equation}
注意到$\phi$是循环坐标,因此可令
\begin{equation}
	S(r,\theta,\phi,t) = -Et + p_\phi\phi + S_1(r) + S_2(\theta)
	\label{Hamilton-Jacobi方程解Kepler问题3}
\end{equation}
由此得到两个方程
\begin{subnumcases}{}
	\left(\frac{\mathrm{d}S_2}{\mathrm{d}\theta}\right)^2 + \frac{p_\phi^2}{\sin^2\theta} = \beta^2 \label{Hamilton-Jacobi方程解Kepler问题4.1} \\
	\frac{1}{2\mu}\left[\left(\frac{\mathrm{d}S_1}{\mathrm{d}r}\right)^2+\frac{\beta^2}{r^2}\right] - \frac{\alpha}{r} = E \label{Hamilton-Jacobi方程解Kepler问题4.2}
\end{subnumcases}
由此即有
\begin{align}
	S = -Et+p_\phi\phi \pm \int \sqrt{2\mu E+\frac{2\mu\alpha}{r}-\frac{\beta^2}{r^2}}\mathrm{d}r \pm \int \sqrt{\beta^2-\frac{p_\phi^2}{\sin^2\theta}}\mathrm{d}\theta
	\label{Hamilton-Jacobi方程解Kepler问题5}
\end{align}
其中$E,p_\phi,\beta$为任意积分常数。为了使结果更具有几何意义,现在用坐标$r$和$\theta$的初始值$r_0$和$\theta_0$来取代积分常数$E$和$p_\phi$,令
\begin{equation}
\begin{cases}
	p_\phi^2 = \beta^2 \sin^2\theta_0 \\
	E = \dfrac{\beta^2}{2\mu r_0^2}-\dfrac{\alpha}{r_0}
\end{cases}\label{Hamilton-Jacobi方程解Kepler问题6}
\end{equation}
此时可有
\begin{align}
	S & = -\left(\frac{\beta^2}{2\mu r_0^2}-\frac{\alpha}{r_0}\right)t \pm \beta\sin \theta_0 \phi \pm \sqrt{2\mu}\int \sqrt{\frac{\beta^2}{2\mu}\left(\frac{1}{r_0^2}-\frac{1}{r^2}\right)-\alpha\left(\frac{1}{r_0}-\frac{1}{r}\right)}\mathrm{d}r \nonumber \\
	& \quad {} \pm \beta \int \sqrt{1-\frac{\sin^2\theta_0}{\sin^2\theta}}\mathrm{d}\theta
	\label{Hamilton-Jacobi方程解Kepler问题7}
\end{align}
于是,质点在空间中的轨迹由如下方程决定
\begin{subnumcases}{}
	\frac{\pl S}{\pl r_0} = -p_{r_0} \label{Hamilton-Jacobi方程解Kepler问题8.1} \\
	\frac{\pl S}{\pl \theta_0} = -p_{\theta_0} \label{Hamilton-Jacobi方程解Kepler问题8.2} \\
	\frac{\pl S}{\pl \beta} = -\gamma \label{Hamilton-Jacobi方程解Kepler问题8.3}
\end{subnumcases}
首先假设在\eqref{Hamilton-Jacobi方程解Kepler问题7}中都取正号,则式\eqref{Hamilton-Jacobi方程解Kepler问题8.1}化为
\begin{equation*}
	-\left(-\frac{\beta^2}{\mu r_0^3}+\frac{\alpha}{r_0^2}\right)t + \sqrt{\frac{\mu}{2}} \int \frac{-\dfrac{\beta^2}{\mu r_0^3}+\dfrac{\alpha}{r_0^2}}{\sqrt{\dfrac{\beta^2}{2\mu}\left(\dfrac{1}{r_0^2}-\dfrac{1}{r^2}\right) - \alpha\left(\dfrac{1}{r_0}-\dfrac1r\right)}} \mathrm{d}r = -p_{r_0}
\end{equation*}
整理可得\footnote{积分下限的不同取值只会式\eqref{Hamilton-Jacobi方程解Kepler问题9}右端多一个与该固定取值相关的常数,此常数总可以通过选择时间起点来去除。此处将积分下限取为$r_0$,下文会说明,$r=r_0$是$\dot{r}$变号的位置。}
\begin{equation}
	\sqrt{\frac{\mu}{2}} \int_{r_0}^r \frac{\mathrm{d}r}{\sqrt{\left(\dfrac{1}{r_0}-\dfrac1r\right)\left[\dfrac{\beta^2}{2\mu}\left(\dfrac{1}{r_0}+\dfrac1r\right)-\alpha\right]}} = t-t_0
	\label{Hamilton-Jacobi方程解Kepler问题9}
\end{equation}
其中$t_0 = \dfrac{p_{r_0}}{-\dfrac{\beta^2}{\mu r_0^3}+\dfrac{\alpha}{r_0^2}}$。将式\eqref{Hamilton-Jacobi方程解Kepler问题9}改写为
\begin{equation*}
	\sqrt{\frac{\mu r_0}{2}} \int_{r_0}^r \frac{r\mathrm{d}r}{\sqrt{(r-r_0)\left[\left(\dfrac{\beta^2}{2\mu r_0}-\alpha\right)r+\dfrac{\beta^2}{2\mu}\right]}} = t-t_0
\end{equation*}
根据$\dfrac{\beta^2}{2\mu r_0}-\alpha$的符号,可以分为以下三种情形:
\begin{enumerate}
\item $\dfrac{\beta^2}{2\mu r_0}<\alpha$,此时$E = \dfrac{\beta^2}{2\mu r_0^2}-\dfrac{\alpha}{r_0}<0$,令
\begin{equation}
	r_1 = \dfrac{\dfrac{\beta^2}{2\mu}}{\alpha-\dfrac{\beta^2}{2\mu r_0}} > 0
	\label{Hamilton-Jacobi方程解Kepler问题10}
\end{equation}
则式\eqref{Hamilton-Jacobi方程解Kepler问题9}可以改写为
\begin{equation}
	\frac{\mu r_0}{\sqrt{2\mu r_0\alpha-\beta^2}} \int_{r_0}^r \frac{r\mathrm{d}r}{\sqrt{(r-r_0)(r_1-r)}} = t-t_0
	\label{Hamilton-Jacobi方程解Kepler问题11}
\end{equation}
对式\eqref{Hamilton-Jacobi方程解Kepler问题11}两端同时求对$t$的导数,可得
\begin{equation}
	\dot{r} = \frac{\sqrt{2\mu r_0\alpha-\beta^2}}{\mu r_0r} \sqrt{(r-r_0)(r_1-r)}
	\label{Hamilton-Jacobi方程解Kepler问题12}
\end{equation}
根据式\eqref{Hamilton-Jacobi方程解Kepler问题12}可得
\begin{equation}
	\dot{r}\big|_{r=r_0} = 0,\quad \dot{r}\big|_{r=r_1} = 0
\end{equation}
即$r=r_0$和$r=r_1$是轨道的两个拱点。由式\eqref{Hamilton-Jacobi方程解Kepler问题12}中的根式非负也同样可得此运动的径向范围为$r_0\leqslant r \leqslant r_1$,是有界运动。

做变量代换$r = \dfrac{r_0+r_1}{2}-\dfrac{r_1-r_0}{2}\cos \xi$,则有关于矢径长度的参数方程
\begin{equation}
\begin{cases}
	r = \dfrac{r_0+r_1}{2}-\dfrac{r_1-r_0}{2}\cos \xi \\
	t = \dfrac{\mu r_0(r_0+r_1)}{2\sqrt{2\mu r_0\alpha-\beta^2}} \left(\xi - \dfrac{r_1-r_0}{r_1+r_0}\sin \xi\right) + t_0
\end{cases}
\label{Hamilton-Jacobi方程解Kepler问题13}
\end{equation}

\item $\dfrac{\beta^2}{2\mu r_0}=\alpha$,此时$E = \dfrac{\beta^2}{2\mu r_0^2}-\dfrac{\alpha}{r_0}=0$,则式\eqref{Hamilton-Jacobi方程解Kepler问题9}可以改写为
\begin{equation}
	\frac{\mu\sqrt{r_0}}{\beta} \int_{r_0}^r \frac{r\mathrm{d}r}{\sqrt{r-r_0}} = t-t_0
	\label{Hamilton-Jacobi方程解Kepler问题14}
\end{equation}
对式\eqref{Hamilton-Jacobi方程解Kepler问题14}两端同时求对$t$的导数,可得
\begin{equation}
	\dot{r} = \frac{\beta}{\mu r}\sqrt{\frac{r-r_0}{r_0}}
	\label{Hamilton-Jacobi方程解Kepler问题15}
\end{equation}
即仅当$r=r_0$时,才有$\dot{r} = 0$。由式\eqref{Hamilton-Jacobi方程解Kepler问题15}中的根式非负可得此运动的径向范围为$r\geqslant r_0$,是无界运动。

求出式\eqref{Hamilton-Jacobi方程解Kepler问题14}中的积分,可得矢径长度与时间的关系
\begin{equation}
	t = \frac{2\mu}{3\beta}\sqrt{r_0(r-r_0)}(r+2r_0) + t_0
	\label{Hamilton-Jacobi方程解Kepler问题16}
\end{equation}
由此可见,当$t\to +\infty$时,$r\to +\infty$。

\item $\dfrac{\beta^2}{2\mu r_0}>\alpha$,此时$E = \dfrac{\beta^2}{2\mu r_0^2}-\dfrac{\alpha}{r_0}>0$,令
\begin{equation}
	r_1 = \dfrac{\dfrac{\beta^2}{2\mu}}{\dfrac{\beta^2}{2\mu r_0}-\alpha} > 0
	\label{Hamilton-Jacobi方程解Kepler问题17}
\end{equation}
则式\eqref{Hamilton-Jacobi方程解Kepler问题9}可以改写为
\begin{equation}
	\frac{\mu r_0}{\sqrt{\beta^2-2\mu r_0\alpha}} \int_{r_0}^r \frac{r\mathrm{d}r}{\sqrt{(r-r_0)(r+r_1)}} = t-t_0
	\label{Hamilton-Jacobi方程解Kepler问题18}
\end{equation}
对式\eqref{Hamilton-Jacobi方程解Kepler问题18}两端同时求对$t$的导数,可得
\begin{equation}
	\dot{r} = \frac{\sqrt{\beta^2-2\mu r_0\alpha}}{\mu r_0r} \sqrt{(r-r_0)(r+r_1)}
	\label{Hamilton-Jacobi方程解Kepler问题19}
\end{equation}
在$r>0$的范围内,仅当$r=r_0$时,才有$\dot{r}=0$。由式\eqref{Hamilton-Jacobi方程解Kepler问题19}中的根式非负可得此运动的径向范围为$r\geqslant r_0$,是无界运动。

做变量代换$r=-\dfrac{r_1-r_0}{2}+\dfrac{r_0+r_1}{2}\cosh\xi$,则有关于矢径长度的参数方程
\begin{equation}
\begin{cases}
	\displaystyle r = -\dfrac{r_1-r_0}{2}+\dfrac{r_0+r_1}{2}\cosh\xi \\
	\displaystyle t = \frac{\mu r_0(r_1-r_0)}{2\sqrt{\beta^2-2\mu r_0\alpha}} \left(\frac{r_1+r_0}{r_1-r_0}\sinh \xi-\xi\right) + t_0
\end{cases}
\label{Hamilton-Jacobi方程解Kepler问题20}
\end{equation}
\end{enumerate}

前述三种情形分别对应了三种完全不同的轨道形状,而当$t=t_0$时,$r=r_0$为行星运动过程中距力心最近的点,该点即称为{\bf 近心点}\footnote{对于行星绕太阳的运动,该点即称为{\bf 近日点}。}。

现在来考虑方程\eqref{Hamilton-Jacobi方程解Kepler问题8.2}。假设在方程\eqref{Hamilton-Jacobi方程解Kepler问题8.2}中都取正号,则式\eqref{Hamilton-Jacobi方程解Kepler问题8.2}化为
\begin{equation*}
	\beta\cos\theta_0\phi - \beta \int \frac{\sin\theta_0\cos\theta_0\csc^2\theta}{\sqrt{1-\sin^2\theta_0\csc^2\theta}}\mathrm{d}\theta = -p_{\theta_0}
\end{equation*}
整理可得\footnote{基于与前面相同的原因,此处将积分下限取为$\theta_0$。}
\begin{equation}
	-\int_{\theta_0}^\theta \frac{\sin\theta_0\csc^2\theta}{\sqrt{1-\sin^2\theta_0\csc^2\theta}} \mathrm{d}\theta = -\phi - \frac{p_{\theta_0}}{\beta\cos\theta_0}
	\label{Hamilton-Jacobi方程解Kepler问题21}
\end{equation}
将式\eqref{Hamilton-Jacobi方程解Kepler问题21}左端积分,即有
\begin{equation}
	\arccos \frac{\cot\theta}{\cot\theta_0} = \phi + \frac{p_{\theta_0}}{\beta\cos \theta_0}
	\label{Hamilton-Jacobi方程解Kepler问题22}
\end{equation}
整理可得
\begin{equation}
	\frac{\cot\theta}{\cot\theta_0} = \sin(\phi-\varOmega)
	\label{Hamilton-Jacobi方程解Kepler问题23}
\end{equation}
其中$\varOmega = -\dfrac{\pi}{2}-\dfrac{p_{\theta_0}}{\beta\cos \theta_0}$。

\begin{figure}[htb]
\centering
\begin{asy}
	size(300);
	//行星的轨道示意图
	pair O,i,j,k;
	O = (0,0);
	i = (-sqrt(2)/4,-sqrt(14)/12);
	j = (sqrt(14)/4,-sqrt(2)/12);
	k = (0,2*sqrt(2)/3);
	real r;
	r = 1;
	draw(scale(r)*unitcircle,linewidth(0.8bp));
	real theta0,phi0,phi;
	theta0 = 0.3*pi;
	phi0 = pi/6;
	pair xyplane(real t){
		return r*cos(t)*i+r*sin(t)*j;
	}
	pair yzplane(real t){
		return r*cos(t)*j+r*sin(t)*k;
	}
	pair zxplane(real t){
		return r*cos(t)*k+r*sin(t)*i;
	}
	pair planetplane(real phi){
		real m,st,ct;
		m = sin(phi-phi0)/tan(theta0);
		st = 1/sqrt(m^2+1);
		ct = m/sqrt(m^2+1);
		return st*cos(phi)*i+st*sin(phi)*j+ct*k;
	}
	pair pqplane(real t){
		return r*cos(t)*cos(phi0)*i+r*cos(t)*sin(phi0)*j+r*sin(t)*k;
	}
	real gettheta_forplanet(real phi,real phi0){
		real m;
		m = sin(phi-phi0)/tan(theta0);
		return asin(1/sqrt(m^2+1));
	}
	real tmp;
	tmp = -1.1;
	draw(graph(xyplane,tmp,pi+tmp));
	draw(graph(xyplane,pi+tmp,2*pi+tmp),dashed);
	tmp = -0.8;
	draw(graph(yzplane,tmp,pi+tmp));
	draw(graph(yzplane,pi+tmp,2*pi+tmp),dashed);
	tmp = -0.2;
	draw(graph(zxplane,tmp,pi+tmp));
	draw(graph(zxplane,pi+tmp,2*pi+tmp),dashed);
	tmp = -1;
	draw(graph(planetplane,tmp,pi+tmp),red+linewidth(0.8bp));
	draw(graph(planetplane,pi+tmp,2*pi+tmp),dashed+red);
	pair n,m,q,p,b;
	n = intersectionpoint(graph(planetplane,tmp,pi+tmp),graph(xyplane,tmp,pi+tmp));
	m = intersectionpoint(graph(planetplane,pi+tmp,2*pi+tmp),graph(xyplane,pi+tmp,2*pi+tmp));
	draw(Label("$\boldsymbol{n}$",MidPoint,Relative(E)),O--n,Arrow);
	draw(-n--O,dashed);
	label("$N$",n,S);
	label("$M$",m,N);
	phi = 1;
	p = planetplane(phi);
	q = r*cos(phi)*i+r*sin(phi)*j;
	draw(Label("$\boldsymbol{q}$",MidPoint,Relative(W)),O--p,Arrow);
	label("$P$",p,E);
	draw(O--q,dashed);
	label("$Q$",q,SE);
	draw(Label("$\varOmega$",MidPoint,Relative(E)),graph(xyplane,0,phi0),blue+linewidth(0.8bp));
	draw(Label("$\phi-\varOmega$",MidPoint,Relative(E)),graph(xyplane,phi0,phi),0.5green+0.5darkgreen+linewidth(0.8bp));
	draw(Label("$\varTheta$",Relative(0.65),Relative(E)),graph(planetplane,phi0,phi),purple+linewidth(0.8bp));
	b = sin(phi0)*cos(theta0)*i-cos(phi0)*cos(theta0)*j+sin(theta0)*k;
	draw(Label("$\boldsymbol{b}$",EndPoint,dir(-90-30)),O--b,Arrow);
	tmp = -0.6;
	phi0 = phi0-pi/2;
	draw(graph(pqplane,tmp,pi+tmp),yellow);
	draw(graph(pqplane,pi+tmp,2*pi+tmp),dashed+yellow);
	draw(Label("$i$",Relative(0.5),Relative(W)),graph(pqplane,pi/2,theta0),0.4yellow+0.6red+linewidth(0.8bp));
	tmp = -1.2;
	phi0 = phi;
	draw(graph(pqplane,tmp,pi+tmp),orange);
	draw(graph(pqplane,pi+tmp,2*pi+tmp),dashed+orange);
	draw(Label("$\theta$",MidPoint,Relative(E)),graph(pqplane,pi/2-gettheta_forplanet(phi,pi/6),pi/2),orange+blue+linewidth(0.8bp));
	draw(O--r*i,dashed);
	draw(O--r*j,dashed);
	draw(O--r*k,dashed);
	draw(Label("$x$",EndPoint),r*i--2.5*r*i,Arrow);
	draw(Label("$y$",EndPoint),r*j--1.5*r*j,Arrow);
	draw(Label("$z$",EndPoint),r*k--1.5*r*k,Arrow);
	label("$O$",O,2*W);
	draw(scale(r)*unitcircle,linewidth(0.8bp));
\end{asy}
\caption{行星的轨道在天球上的投影}
\label{行星的轨道示意图}
\end{figure}

将式\eqref{Hamilton-Jacobi方程解Kepler问题23}与式\eqref{chp3:球面大圆的参数方程}对比,即与式
\begin{equation}
	\frac{\cot\theta}{\cot\left(\theta_0'-\dfrac{\pi}{2}\right)} = \sin\left(\phi-\phi_0'+\frac{\pi}{2}\right)
	\label{Hamilton-Jacobi方程解Kepler问题24}
\end{equation}
对比,可知式\eqref{Hamilton-Jacobi方程解Kepler问题23}所表示的球面曲线为大圆,即行星的轨迹是平面的。如图\ref{行星的轨道示意图}所示,作一个单位球面(称为{\bf 天球}),利用其上的点表示行星在空间中的角度坐标$(\theta,\phi)$。考虑到当$\phi=\varOmega$时,$\theta=\dfrac{\pi}{2}$,即$\varOmega$表示轨道平面与$xOy$坐标平面的交叉线与$Ox$轴所成的角度,称为{\bf 升交点赤经}。记大圆上任意一点$P$,其球面坐标为$(\theta,\phi)$,则单位向量$\overrightarrow{OP}=:\mbf{q}$可以表示为
\begin{equation}
	\mbf{q} = \sin\theta\cos\phi\mbf{e}_1+\sin\theta\sin\phi\mbf{e}_2+\cos\theta\mbf{e}_3
	\label{Hamilton-Jacobi方程解Kepler问题24-q}
\end{equation}
而轨道平面与$xOy$坐标平面交叉线$\overrightarrow{ON}$的方向向量$\mbf{n}$可以表示为
\begin{equation}
	\mbf{n} = \cos\varOmega\mbf{e}_1+\sin\varOmega\mbf{e}_2
	\label{Hamilton-Jacobi方程解Kepler问题24-n}
\end{equation}
因此,轨道平面的单位法向量可以表示为
\begin{equation}
	\mbf{b} = \frac{\mbf{n}\times \mbf{q}}{|\mbf{n}\times \mbf{q}|}
	\label{Hamilton-Jacobi方程解Kepler问题25}
\end{equation}
其中
\begin{align}
	\mbf{n}\times \mbf{q} & = \begin{vmatrix} \mbf{e}_1 & \mbf{e}_2 & \mbf{e}_3 \\ \cos\varOmega & \sin\varOmega & 0 \\ \sin\theta\cos\phi & \sin\theta\sin\phi & \cos\theta \end{vmatrix} \nonumber \\
	& = \cos\theta\sin\varOmega\mbf{e}_1 - \cos\theta\cos\varOmega\mbf{e}_2+\sin\theta(\sin\phi\cos\varOmega-\cos\phi\sin\varOmega)\mbf{e}_3 \nonumber \\
	& = \cos\theta\sin\varOmega\mbf{e}_1 - \cos\theta\cos\varOmega\mbf{e}_2+\sin\theta\sin(\phi-\varOmega)\mbf{e}_3
	\label{Hamilton-Jacobi方程解Kepler问题26}
\end{align}
将式\eqref{Hamilton-Jacobi方程解Kepler问题23}带入式\eqref{Hamilton-Jacobi方程解Kepler问题26}中,可得
\begin{equation}
	\mbf{n}\times \mbf{q} = \frac{\cos\theta}{\cos\theta_0}\left(\cos\theta_0\sin\varOmega\mbf{e}_1-\cos\theta_0\cos\varOmega\mbf{e}_2+\sin\theta_0\mbf{e}_3\right)
	\label{Hamilton-Jacobi方程解Kepler问题27}
\end{equation}
因此有
\begin{equation}
	\mbf{b} = \cos\theta_0\sin\varOmega\mbf{e}_1-\cos\theta_0\cos\varOmega\mbf{e}_2+\sin\theta_0\mbf{e}_3
	\label{Hamilton-Jacobi方程解Kepler问题28}
\end{equation}
即向量$\mbf{b}$为常向量,亦即说明行星的运转轨道是平面曲线,而该平面与$xOy$坐标平面之间的夹角为$\dfrac{\pi}{2}-\theta_0$,这个角度称为{\bf 轨道倾角},也常以$i$表示。

最后考虑方程\eqref{Hamilton-Jacobi方程解Kepler问题8.3}。假设在方程\eqref{Hamilton-Jacobi方程解Kepler问题8.3}中都取正号,则式\eqref{Hamilton-Jacobi方程解Kepler问题8.3}化为
\begin{align*}
	-\frac{\beta}{\mu r_0^2}t + \phi\sin\theta_0 & {}+\sqrt{2\mu}\int_{r_0}^r \frac{\dfrac{\beta}{\mu}\left(\dfrac{1}{r_0^2}-\dfrac{1}{r^2}\right)}{2\sqrt{\dfrac{\beta^2}{2\mu}\left(\dfrac{1}{r_0^2}-\dfrac{1}{r^2}\right)-\alpha\left(\dfrac{1}{r_0}-\dfrac{1}{r}\right)}}\mathrm{d}r \\
	& {} + \int_{\theta_0}^\theta \sqrt{1-\frac{\sin^2\theta_0}{\sin^2\theta}}\mathrm{d}\theta = -\gamma
\end{align*}
整理可得\footnote{基于与前面相同的原因,此处将积分下限取为$r_0$和$\theta_0$。}
\begin{align}
	\frac{\beta}{\sqrt{2\mu}}\int_{r_0}^r \frac{\dfrac{1}{r_0^2}-\dfrac{1}{r^2}}{\sqrt{\dfrac{\beta^2}{2\mu}\left(\dfrac{1}{r_0^2}-\dfrac{1}{r^2}\right)-\alpha\left(\dfrac{1}{r_0}-\dfrac{1}{r}\right)}} \mathrm{d}r & {}+ \int_{\theta_0}^\theta \sqrt{1-\sin^2\theta_0\csc^2\theta} \mathrm{d}\theta \nonumber \\
	& = -\gamma + \frac{\beta}{\mu r_0^2}t - \phi\sin\theta_0
	\label{Hamilton-Jacobi方程解Kepler问题29}
\end{align}
在式\eqref{Hamilton-Jacobi方程解Kepler问题9}两端同时取微分,可得
\begin{equation}
	\frac{\mathrm{d}r}{\sqrt{\dfrac{\beta^2}{2\mu}\left(\dfrac{1}{r_0^2}-\dfrac{1}{r^2}\right)-\alpha\left(\dfrac{1}{r_0}-\dfrac{1}{r}\right)}} = \sqrt{\frac{2}{\mu}}\mathrm{d}t
	\label{Hamilton-Jacobi方程解Kepler问题30}
\end{equation}
将式\eqref{Hamilton-Jacobi方程解Kepler问题30}代入式\eqref{Hamilton-Jacobi方程解Kepler问题29}中,可得
\begin{equation}
	-\frac{\beta}{\mu}\int_{t_0}^t \frac{\mathrm{d}t}{r^2(t)} + \int_{\theta_0}^\theta \sqrt{1-\sin^2\theta_0\csc^2\theta} \mathrm{d}\theta = -\gamma + \frac{\beta}{\mu r_0^2}t_0-\phi\sin\theta_0
	\label{Hamilton-Jacobi方程解Kepler问题31}
\end{equation}
考虑式\eqref{Hamilton-Jacobi方程解Kepler问题23},即
\begin{equation}
	\cot\theta = \cot\theta_0\sin(\phi-\varOmega) = \cot\theta\cos\left(\phi-\varOmega-\dfrac{\pi}{2}\right)
	\label{Hamilton-Jacobi方程解Kepler问题32}
\end{equation}
当$\theta=\theta_0$时,有$\phi=\dfrac{\pi}{2}+\varOmega=:\phi_0$,因此将式\eqref{Hamilton-Jacobi方程解Kepler问题31}代入式\eqref{Hamilton-Jacobi方程解Kepler问题31}中可得
\begin{equation}
	-\frac{\beta}{\mu}\int_{t_0}^t \frac{\mathrm{d}t}{r^2(t)} + \int_{\phi_0}^\phi \frac{\csc\theta_0}{1+\cot^2\theta_0\cos^2(\phi-\phi_0)}\mathrm{d}\phi = -\gamma+\frac{\beta}{\mu r_0^2}t_0-\phi_0\sin\theta_0
	\label{Hamilton-Jacobi方程解Kepler问题33}
\end{equation}
计算式\eqref{Hamilton-Jacobi方程解Kepler问题33}中的积分如下
\begin{align*}
	\int_{\phi_0}^\phi \frac{\csc\theta_0}{1+\cot^2\theta_0\cos^2(\phi-\phi_0)}\mathrm{d}\phi & = \int_{\phi_0}^\phi \frac{\csc\theta_0}{\sin^2(\phi-\phi_0)+(1+\cot^2)\theta_0\cos^2(\phi-\phi_0)} \mathrm{d}\phi \\
	& = \int_{\phi_0}^\phi \frac{\csc\theta_0}{\sin^2(\phi-\phi_0)+\csc^2\theta_0\cos^2(\phi-\phi_0)} \mathrm{d}\phi \\
	& = \int_{\phi_0}^\phi \frac{\csc\theta_0\csc^2(\phi-\phi_0)}{1+\csc^2\theta_0\cot^2(\phi-\phi_0)} \mathrm{d}\phi \\
	& = -\arctan\big[\csc\theta_0\cot(\phi-\phi_0)\big]\bigg|_{\phi_0}^\phi \\
	& = -\arctan\big[\csc\theta_0\cot(\phi-\phi_0)\big]+\arctan(+\infty) \\
	& = -\arctan\big[\csc\theta_0\cot(\phi-\phi_0)\big] + \frac{\pi}{2}
\end{align*}
记
\begin{equation}
	\varTheta = -\arctan\big[\csc\theta_0\cot(\phi-\phi_0)\big]
	\label{Hamilton-Jacobi方程解Kepler问题34-1}
\end{equation}
称为{\bf 真近点角},则式\eqref{Hamilton-Jacobi方程解Kepler问题33}化为
\begin{equation}
	\varTheta = \frac{\beta}{\mu}\int_{t_0}^t \frac{\mathrm{d}t}{r^2(t)}+\varTheta_0
	\label{Hamilton-Jacobi方程解Kepler问题34}
\end{equation}
其中$\varTheta_0=-\dfrac{\pi}{2}-\gamma+\dfrac{\beta}{\mu r_0^2}t_0-\phi_0\sin\theta_0$,而$(r_0,\theta_0,\phi_0)$即为近心点的球坐标。此处$\varTheta$的几何意义即为图\ref{行星的轨道示意图}中的$\angle NOP$,这是因为根据式\eqref{Hamilton-Jacobi方程解Kepler问题24-q}、式\eqref{Hamilton-Jacobi方程解Kepler问题24-n}和式\eqref{Hamilton-Jacobi方程解Kepler问题23}可得
\begin{align*}
	\cos \angle NOP & = \mbf{n}\cdot \mbf{q} = \sin\theta\cos\phi\cos\varOmega+ \sin\theta\sin\phi\sin\varOmega \\
	& = \sin\theta\cos(\phi-\varOmega) = \frac{\cos(\phi-\varOmega)}{\sqrt{1+\cot^2\theta_0\sin^2(\phi-\varOmega)}}
\end{align*}
故
\begin{equation*}
	\tan\angle NOP = \csc\theta_0\tan(\phi-\varOmega) = -\csc\theta_0\cot(\phi-\phi_0)
\end{equation*}
时间$t_0$代表行星经过近心点$(r_0,\theta_0,\phi_0)$的时刻,因此式\eqref{Hamilton-Jacobi方程解Kepler问题34}中的常数$\varTheta_0$代表近心点与升交点$N$之间在轨道平面上的角度差,称为{\bf 近心点幅角},也常以$\tilde{\omega}$表示。%,又叫做{\bf 自升交点的近日点黄经}。

下面来考虑行星运行轨道的具体形状。将式\eqref{Hamilton-Jacobi方程解Kepler问题34}两端求对$t$的导数可得
\begin{equation}
	\frac{\mathrm{d}\varTheta}{\mathrm{d}t} = \frac{\beta}{\mu r^2}
	\label{Hamilton-Jacobi方程解Kepler问题35}
\end{equation}
同理将式\eqref{Hamilton-Jacobi方程解Kepler问题9}两端求对$t$的导数可得
\begin{equation}
	\frac{\mathrm{d}r}{\mathrm{d}t} = \sqrt{\frac{2}{\mu}\left(\frac{\beta^2}{2\mu r_0^2}-\frac{\alpha}{r_0}+ \frac{\alpha}{r}- \frac{\beta^2}{2\mu r^2}\right)} = \sqrt{\frac{2}{\mu}\left(E+\frac{\alpha}{r}-\frac{\beta^2}{2\mu r^2}\right)}
	\label{Hamilton-Jacobi方程解Kepler问题36}
\end{equation}
由式\eqref{Hamilton-Jacobi方程解Kepler问题35}和\eqref{Hamilton-Jacobi方程解Kepler问题36}可得
\begin{equation}
	\frac{\mathrm{d}r}{\mathrm{d}\varTheta} = r^2\sqrt{\frac{2\mu E}{\beta^2}+ \frac{2\mu\alpha}{\beta^2} \frac1r - \frac{1}{r^2}}
	\label{Hamilton-Jacobi方程解Kepler问题37}
\end{equation}
如果令$u = \dfrac1r$可将式\eqref{Hamilton-Jacobi方程解Kepler问题37}化为
\begin{equation*}
	-\frac{\mathrm{d}u}{\sqrt{\dfrac{\mu^2\alpha^2}{\beta^4}\left(1+\dfrac{2\beta^2E}{\mu\alpha^2}\right) - \left(u - \dfrac{\mu\alpha}{\beta^2}\right)^2}} = \mathrm{d}\varTheta
\end{equation*}
由此可得
\begin{equation*}
	r = \frac{\dfrac{\beta^2}{\mu\alpha}}{1+\sqrt{1+\dfrac{2\beta^2E}{\mu\alpha^2}}\cos(\varTheta-\varTheta_0)}
\end{equation*}
记
\begin{equation}
	p = \dfrac{\beta^2}{\mu\alpha},\quad e = \sqrt{1+\dfrac{2\beta^2E}{\mu\alpha^2}}
	\label{Hamilton-Jacobi方程解Kepler问题38}
\end{equation}
其中$p$称为{\bf 轨道参数},$e$为{\bf 离心率}。由此即有
\begin{equation}
	r = \frac{p}{1+e\cos(\varTheta-\varTheta_0)}
	\label{Hamilton-Jacobi方程解Kepler问题39}
\end{equation}
由此可得当$E<0$时,$e<1$,轨道为椭圆;当$E=0$时,$e=1$,轨道为抛物线;当$E>0$时,$e>1$,轨道为双曲线。他们分别对应式\eqref{Hamilton-Jacobi方程解Kepler问题13}、式\eqref{Hamilton-Jacobi方程解Kepler问题16}和式\eqref{Hamilton-Jacobi方程解Kepler问题20}的三种情况。

至此,行星在空间中的绕日运动已经完全解决了。这个问题的六个积分常数分别为$r_0,~\theta_0,~\beta,~t_0,~\varOmega,~\varTheta_0$,而后面三个是常数$p_{r_0},~p_{\theta_0}$和$\gamma$的函数。而在天体力学中,常将这六个积分常数用其它一些更具有几何意义的常数来代替,它们分别是轨道参数$p$,离心率$e$,过近心点时刻$t_0$,升交点赤经$\varOmega$,轨道倾角$i$,近心点幅角$\tilde{\omega}$。

\begin{example}
用Hamilton-Jacobi方程解平面谐振子问题。
\end{example}
\begin{solution}
平面谐振子的Hamilton函数为
\begin{equation}
	H = \frac{1}{2m}(p_x^2+p_y^2) + \frac{k}{2}(x^2+y^2)
	\label{Hamilton-Jacobi方程解平面谐振子问题-1}
\end{equation}
系统的Hamilton函数不显含时间,故根据不含时Hamilton-Jacobi方程\eqref{不含时Hamilton-Jacobi方程}可有
\begin{equation}
	\frac{1}{2m} \left[\left(\frac{\pl W}{\pl x}\right)^2 + \left(\frac{\pl W}{\pl y}\right)^2\right] + \frac{k}{2}(x^2+y^2) = E
	\label{Hamilton-Jacobi方程解平面谐振子问题-2}
\end{equation}
将其分离变量可设
\begin{equation}
	S(x,y,t) = -Et + W(x,y) = -Et + S_1(x) + S_2(y)
	\label{Hamilton-Jacobi方程解平面谐振子问题-3}
\end{equation}
由此得到两个方程
\begin{subnumcases}{}
	\left(\frac{\mathrm{d} S_2}{\mathrm{d} y}\right)^2 + \frac{k}{2}y^2 = \frac{k}{2} \beta^2 \label{Hamilton-Jacobi方程解平面谐振子问题-4.1} \\
	\left(\frac{\mathrm{d} S_1}{\mathrm{d} x}\right)^2 + \frac{k}{2}x^2 = E - \frac{k}{2} \beta^2 \label{Hamilton-Jacobi方程解平面谐振子问题-4.2}
\end{subnumcases}
由此即有
\begin{equation}
	S(x,y,t) = -Et \pm \sqrt{mk} \int \sqrt{\frac{2E}{k}-\beta^2-x^2} \mathrm{d}x \pm \int \sqrt{\beta^2-y^2} \mathrm{d}y
	\label{Hamilton-Jacobi方程解平面谐振子问题-5}
\end{equation}
为了使式子更加具有对称性,令
\begin{equation}
	E = \frac{k}{2}(\alpha^2 + \beta^2)
	\label{Hamilton-Jacobi方程解平面谐振子问题-6}
\end{equation}
则有
\begin{equation}
	S(x,y,t) = -\frac{k}{2}(\alpha^2 + \beta^2) t \pm \sqrt{mk} \int \sqrt{\alpha^2-x^2} \mathrm{d}x \pm \sqrt{mk} \int \sqrt{\beta^2-y^2} \mathrm{d}y
	\label{Hamilton-Jacobi方程解平面谐振子问题-7}
\end{equation}
于是质点在空间中的轨迹由如下方程决定
\begin{subnumcases}{}
	\frac{\pl S}{\pl \alpha} = -p_{x_0} \label{Hamilton-Jacobi方程解平面谐振子问题-8.1} \\
	\frac{\pl S}{\pl \beta} = -p_{y_0} \label{Hamilton-Jacobi方程解平面谐振子问题-8.2}
\end{subnumcases}
由式\eqref{Hamilton-Jacobi方程解平面谐振子问题-8.1}可得
\begin{equation}
	-k\alpha t \pm \sqrt{mk} \int_\alpha^x \frac{\alpha}{\sqrt{\alpha^2-x^2}} \mathrm{d}x = -p_{x_0}
	\label{Hamilton-Jacobi方程解平面谐振子问题-9}
\end{equation}
由此即可得到
\begin{equation}
	x = \alpha \cos \left[\sqrt{\frac{k}{m}}\left(t-\frac{p_{x_0}}{\alpha k}\right)\right]
	\label{Hamilton-Jacobi方程解平面谐振子问题-10}
\end{equation}
由式\eqref{Hamilton-Jacobi方程解平面谐振子问题-8.2}可得
\begin{equation}
	-k\beta t \pm \sqrt{mk} \int_\beta^y \frac{\beta}{\sqrt{\beta^2-y^2}} \mathrm{d}y = -p_{y_0}
	\label{Hamilton-Jacobi方程解平面谐振子问题-11}
\end{equation}
由此即可得到
\begin{equation}
	y = \beta \cos \left[\sqrt{\frac{k}{m}}\left(t-\frac{p_{y_0}}{\beta k}\right)\right]
	\label{Hamilton-Jacobi方程解平面谐振子问题-12}
\end{equation}
至此,平面谐振子的问题就完全解决了,此问题的四个积分常数分别为$\alpha,~\beta,~p_{x_0}$和$p_{y_0}$,其中能量和$\alpha,~\beta$之间的关系为
\begin{equation}
	E = \frac{k}{2}(\alpha^2 + \beta^2)
	\label{Hamilton-Jacobi方程解平面谐振子问题-13}
\end{equation}
\end{solution}

\section{作用-角变量}

\subsection{单自由度系统的作用-角变量}

在物理学中,具有周期性的运动往往有特殊的重要性。对于单自由度系统,相空间是二维平面,周期运动只能有两种类型:
\begin{enumerate}
\item 正则坐标$q(t)$和$p(t)$都是时间的周期函数,且周期相同。在这种情况下,相轨道是闭合曲线,这是单自由度振动\footnote{在多自由度的一般情形下,这类周期运动称为{\bf 天平动}。}。
\item 函数$q(t)$不是周期函数,但是$q$的值每增加$q_0$时,系统的状态便重现一次。在这种情况下,相轨道不是闭合曲线,但$p$是$q$的周期函数。这类周期运动称为{\bf 转动}。
\end{enumerate}

设$H=H(q,p)$是单自由度系统的Hamilton函数,其Hamilton特征函数为$W=W(q,E)$,其中$E$为广义能量积分$H(q,p)=E$的积分常数。此时,应有
\begin{equation}
	p = \frac{\pl W}{\pl q}
	\label{chapter10:正则变量和特征函数的关系}
\end{equation}
根据第\ref{chapter10:section-Hamilton-Jacobi方程}节的相关讨论,将Hamilton主函数$S$作为正则变换的母函数,得到的新正则变量都是循环变量,都是积分常数。但这里不用$E$作为变换之后的“动量”,而是引入如下定义的新变量作为正则变换后的新动量:
\begin{equation}
	J = \frac{1}{2\pi}\oint p\mathd q
	\label{chapter10:作用量变量的定义}
\end{equation}
其中积分沿着$q$的一个变化周期内所对应的相轨道进行。由\eqref{chapter10:作用量变量的定义}可知,$J$在振动情形下等于封闭相轨道内的面积除以$2\pi$,而在转动情形下等于相曲线与$q$轴上长度为$q_0$的线段之间的面积除以$2\pi$。

将式\eqref{chapter10:正则变量和特征函数的关系}代入式\eqref{chapter10:作用量变量的定义}中可得
\begin{equation}
	J = \frac{1}{2\pi}\oint \frac{\pl W}{\pl q}(q,E)\mathd q
	\label{chapter10:作用量变量的定义-变形}
\end{equation}
由此可知$J$仅是积分常数$E$的函数。当$\dif{J}{E} \neq 0$时可以由$J=J(E)$反解得到$E=E(J)$,由此可得新的Hamilton函数$H = E(J)$,而这个正则变换$q,p \mapsto \omega, J$的正则变换母函数就是Hamilton特征函数$W=W(q,E(J))$。其中与变量$J$共轭的正则坐标可根据下式
\begin{equation}
	\omega = \frac{\pl W}{\pl J}
	\label{chapter10:角变量的定义}
\end{equation}
确定。

由于变量$J$具有与作用量相同的量纲(式\eqref{chapter10:作用量变量的定义}),因此变量$J$称为{\bf 作用量变量},而变量$\omega$是无量纲的(式\eqref{chapter10:角变量的定义}),但其在各种系统中的物理意义往往与某种角度相关,因此变量$\omega$称为{\bf 角变量}。

在作用-角变量下,Hamilton函数为$K=E(J)$,正则方程为
\begin{equation}
\begin{cases}
	\dif{J}{t} = -\dfrac{\pl K}{\pl \omega} = 0 \\[1.5ex]
	\dif{\omega}{t} = \dfrac{\pl K}{\pl J} = \dif{E}{J}
\end{cases}
\end{equation}
据此可以解得
\begin{equation}
	J = J_0,\quad \omega = \left(\dif{E}{J}\right) t + \omega_0
\end{equation}
其中$J_0$和$\omega_0$是积分常数。如果用$\Delta\omega$表示$q$改变一个周期内的增量,考虑到式\eqref{chapter10:角变量的定义}可得
\begin{equation}
	\Delta \omega = \oint \frac{\pl \omega}{\pl q}\mathd q = \oint \frac{\pl^2 W}{\pl q\pl J}\mathd q = \frac{\pl}{\pl J} \oint \frac{\pl W}{\pl q} \mathd q
\end{equation}
再利用式\eqref{chapter10:作用量变量的定义-变形}可得
\begin{equation}
	\Delta \omega = \frac{\pl}{\pl J} (2\pi J) = 2\pi
\end{equation}
在一个周期内,角变量$\omega$改变$2\pi$。相应地,这个正则变换的母函数$W=W(q,E(J))=W(q,J)$是坐标$q$的多值函数,每经过一个$q$的周期,这个函数不回到原来的值,而是有一个增量
\begin{equation}
	\Delta W = \oint \frac{\pl W}{\pl q}\mathd q = \oint p\mathd q = 2\pi J
\end{equation}

\begin{example}
求谐振子的频率,Hamilton函数为
\begin{equation*}
	H = \frac{1}{2m}p^2 + \frac12 kx^2
\end{equation*}
\end{example}
\begin{solution}
在相空间中,相轨道
\begin{equation*}
	\frac{1}{2m}p^2 + \frac12 kx^2 = E
\end{equation*}
为椭圆,因此可有
\begin{equation*}
	J = \frac{1}{2\pi} \oint p\mathd x = 2\times\frac{\sqrt{mk}}{2\pi} \int_{-\sqrt{\frac{2E}{k}}}^{\sqrt{\frac{2E}{k}}} \sqrt{\frac{2E}{k}-x^2}\mathd x = E\sqrt{\frac{m}{k}}
\end{equation*}
所以$E = \sqrt{\dfrac{k}{m}}J$,即有
\begin{equation*}
	\omega = \sqrt{\dfrac{k}{m}} t + \omega_0
\end{equation*}
由此即可得到谐振子的角频率应为$\sqrt{\dfrac{k}{m}}$。

这个正则变换可根据式\eqref{chapter10:正则变量和特征函数的关系}和式\eqref{chapter10:角变量的定义}确定,所以有
\begin{equation*}
	W = \int p\mathd x = \sqrt{km} \int \sqrt{\frac{2E}{k}-x^2}\mathd x = \sqrt{km} \int \sqrt{\frac{2J}{\sqrt{km}}-x^2}\mathd x
\end{equation*}
由此可得
\begin{equation*}
	\omega = \frac{\pl W}{\pl J} = \int \frac{\mathd x}{\sqrt{\dfrac{2J}{\sqrt{km}}-x^2}} = \arcsin \frac{x}{\sqrt{\dfrac{2J}{\sqrt{km}}}}
\end{equation*}
即
\begin{equation*}
	x = \sqrt{\dfrac{2J}{\sqrt{km}}}\sin \omega = \sqrt{\frac{2E}{k}} \sin \omega
\end{equation*}
由此可知,此时角变量$\omega$的物理意义即为谐振子的角度。
\end{solution}

\subsection{单摆运动问题的作用-角变量}

摆的运动问题在第\ref{chapter3:subsection-数学摆和物理摆·单自由度保守系统的运动}节中进行过详细讨论,如果用$q$表示摆角,作为其广义坐标,则摆的运动微分方程为
\begin{equation}
	\ddot{q} + \omega_0^2 \sin q = 0
\end{equation}
其Hamilton函数为
\begin{equation}
	H = \frac12 p^2 - \omega_0^2 \cos q
\end{equation}

为了引入作用-角变量,需要分别考虑振动和转动两种情况。

在振动情况下,系统的相轨道如图\ref{chapter3:figure-势函数与对应的相轨道}中的封闭线所示。将能量积分常数记作$E_0$,即
\begin{equation}
	\frac12 p^2 - \omega_0^2 \cos q = E_0
	\label{chapter10:摆运动的能量积分}
\end{equation}
做变量代换$E_0 = -\omega_0^2\cos \beta$,则作用量变量为
\begin{align*}
	J & = \frac{1}{2\pi} \oint p\mathd q = \frac{4}{2\pi} \int_0^\beta \sqrt{\omega_0^2(\cos q - \cos \beta)} \mathd q = \frac{8\omega_0}{\pi} \int_0^\beta \sqrt{k_1^2-\sin^2\frac q2}\mathd q
\end{align*}
记$k_1=\sin\dfrac{\beta}{2}$并做变量代换
\begin{equation}
	\sin\dfrac q2 = k_1\sin u
	\label{chapter10:单摆作用-角变量求解的变量代换}
\end{equation}
可得
\begin{align}
	J & = \frac{8\omega_0}{\pi} \int_0^{\frac{\pi}{2}} \frac{k_1^2\cos^2u}{\sqrt{1-k_1^2\sin^2u}} \mathd u \nonumber \\
	& = \frac{8\omega_0}{\pi} \left[\int_0^{\frac{\pi}{2}} \sqrt{1-k^2\sin^2u}\mathd u - (1-k_1^2) \int_0^{\frac{\pi}{2}} \frac{\mathd u}{\sqrt{1-k_1^2\sin^2u}}\right] \nonumber \\
	& = \frac{8\omega_0}{\pi} \left[E(k_1) - (1-k_1^2)K(k_1)\right] \label{chapter10:单摆振动情形下的作用量变量}
\end{align}
其中$K$和$E$分别是第一类和第二类完全椭圆积分。式\eqref{chapter10:单摆振动情形下的作用量变量}确定了作为$k_1$函数的$J$,将其两边对$k_1$求导,并利用附录中的恒等式\eqref{Appendix-A:第一类完全椭圆积分的导数}和\eqref{Appendix-A:第二类完全椭圆积分的导数}可得
\begin{equation}
	\frac{\pl J}{\pl k_1} = \frac{8\omega_0}{\pi}k_1K(k_1) \neq 0
\end{equation}
所以$k_1$可以表示为$J$的函数,而且有
\begin{equation}
	\frac{\pl k_1}{\pl J} = \frac{\pi}{8\omega_0k_1K(k_1)}
\end{equation}
新的Hamilton函数$\mathcal{H}$只依赖于$J$,可以表示为
\begin{equation}
	\mathcal{H} = -\omega_0^2\cos \beta = 2\omega_0^2k_1^2 - \omega_0^2
\end{equation}
其中$k_1=k_1(J)$是式\eqref{chapter10:单摆振动情形下的作用量变量}定义的$J=J(k_1)$的反函数。据此可得角变量$\omega$满足的方程为
\begin{equation}
	\dif{\omega}{t} = \frac{\pl \mathcal{H}}{\pl J} = \frac{\pl \mathcal{H}}{\pl k_1} \frac{\pl k_1}{\pl J} = \frac{\pi\omega_0}{2K(k_1)}
\end{equation}
由此可得此运动的振动周期为
\begin{equation}
	T = \frac{2\pi}{\dot{\omega}} = \frac{4K(k_1)}{\omega_0}
\end{equation}
这与第\ref{chapter3:subsubsection-摆运动方程的积分}节中所得的结果相同。

为了得到角变量的具体形式,首先求得Hamilton特征函数
\begin{align}
	W = \int p\mathd q = 2\omega_0 \int \sqrt{k_1^2-\sin^2\frac q2}\mathd q
\end{align}
利用变量代换\eqref{chapter10:单摆作用-角变量求解的变量代换},可得
\begin{equation}
	W = 4\omega_0\left[E(k_1,\sin u)-(1-k_1^2)F(k_1,\sin u)\right]
\end{equation}
其中$F$和$E$分别为第一类和第二类不完全椭圆积分,$u$是由式\eqref{chapter10:单摆作用-角变量求解的变量代换}确定的函数$u=u(k_1,q)$,$k_1=k_1(J)$是由式\eqref{chapter10:单摆振动情形下的作用量变量}确定的函数$J=J(k_1)$的反函数。根据角变量的定义式\eqref{chapter10:角变量的定义}可得
\begin{equation}
	\omega = \frac{\pl W}{\pl J} = \frac{\pl W}{\pl k_1} \frac{\pl k_1}{\pl J}
\end{equation}
对函数$W=W(k_1,u)$求$k_1$的偏导数,可得
\begin{equation}
	\frac{\pl W}{\pl k_1} = 4\omega_0\left[\frac{\pl E}{\pl k_1} + \frac{\pl E}{\pl u}\frac{\pl u}{\pl k_1} + 2k_1F - (1-k_1^2)\left(\frac{\pl F}{\pl k_1} + \frac{\pl F}{\pl u}\frac{\pl u}{\pl k_1}\right)\right]
\end{equation}
根据式\eqref{chapter10:单摆作用-角变量求解的变量代换}可得
\begin{equation}
	\frac{\pl u}{\pl k_1} = -\frac{\sin u}{k_1\cos u}
\end{equation}
再根据第一类和第二类不完全椭圆积分的定义式\eqref{第一类椭圆积分-定义}和\eqref{第二类椭圆积分-定义}可得
\begin{align}
	\frac{\pl F}{\pl u} = \frac{1}{\sqrt{1-k_1^2\sin^2u}},\quad \frac{\pl E}{\pl u} = \sqrt{1-k_1^2\sin^2u}
\end{align}
最后利用附录中的恒等式\eqref{Appendix-A:第一类不完全椭圆积分的导数}和\eqref{Appendix-A:第二类不完全椭圆积分的导数}可得
\begin{equation}
	\frac{\pl W}{\pl k_1} = 4\omega_0k_1F(k_1,\sin u)
\end{equation}
所以角变量为
\begin{equation}
	\omega = \frac{\pi}{2K(k_1)}F(k_1,\sin u) = \frac{\pi}{2K(k_1)}F\left(k_1,\frac{1}{k_1}\sin \frac q2\right)
	\label{chapter10:单摆振动情况的角变量}
\end{equation}
最后,根据式\eqref{chapter10:单摆振动情况的角变量}和式\eqref{chapter10:摆运动的能量积分}可得
\begin{equation}
\begin{cases}
	\ds q = 2\arcsin\left[k_1\sn \left(\frac{2K(k_1)\omega}{\pi}\right)\right] \\[1.5ex]
	\ds p = 2\omega_0k_1\cn \left(\frac{2K(k_1)\omega}{\pi}\right)
\end{cases}
\end{equation}
其中$k_1=k_1(J)$是由式\eqref{chapter10:单摆振动情形下的作用量变量}确定的函数$J=J(k_1)$的反函数。

在转动情况下,系统的相轨道如图\ref{chapter3:figure-势函数与对应的相轨道}中最上面一条线或最下面一条线。此时能量积分常数满足$E_0>\omega_0^2$,而且将Hamilton函数中的$p$换成$-p$,Hamilton函数的形式不变,因此此处只考虑$p>0$的转动情况。设$t=0$时有$q=0, p=p_0>0$,即有
\begin{equation}
	\frac12 p^2-\omega_0^2\cos q = E_0 = \frac12p_0^2 - \omega_0^2 = \frac{2\omega_0^2}{k_2^2} - \omega_0^2
	\label{chapter10:单摆转动情形下能量积分的具体形式}
\end{equation}
其中
\begin{equation}
	k_2^2 = \frac{4\omega_0^2}{p_0^2} = \frac{2\omega_0^2}{\omega_0^2+E_0}
\end{equation}
此时作用量变量为
\begin{align}
	J = \frac{1}{2\pi} \oint p\mathd q = \frac{1}{2\pi} \oint \sqrt{2(E_0+\omega_0^2\cos q)}\mathd q = \frac{1}{2\pi} \int_{-\pi}^\pi \frac{2\omega_0}{k_2}\sqrt{1-k_2^2\sin^2\frac{q}{2}} \mathd q
\end{align}
做变量代换$q=2u$,并注意到被积函数的周期性可得
\begin{align}
	J &= \frac{4\omega_0}{\pi k_2} \int_0^{\frac{\pi}{2}}\sqrt{1-k_2^2\sin^2u}\mathd u = \frac{4\omega_0}{\pi k_2} E(k_2)
	\label{chapter10:单摆转动情形下的作用量变量}
\end{align}
式\eqref{chapter10:单摆转动情形下的作用量变量}确定了作为$k_2$函数的$J$,将其两边对$k_2$求导,并利用附录中的恒等式\eqref{Appendix-A:第二类完全椭圆积分的导数}可得
\begin{equation}
	\frac{\pl J}{\pl k_2} = -\frac{4\omega_0}{\pi k_2^2} K(k_2) \neq 0
\end{equation}
所以$k_2$可以表示为$J$的函数,而且有
\begin{equation}
	\frac{\pl k_2}{\pl J} = -\frac{\pi k_2^2}{4\omega_0K(k_2)}
\end{equation}
新的Hamilton函数$\mathcal{H}$只依赖于$J$,可以表示为
\begin{equation}
	\mathcal{H} = E_0 = \frac{2\omega_0^2}{k_2^2}-\omega_0^2
\end{equation}
其中$k_2=k_2(J)$是式\eqref{chapter10:单摆转动情形下的作用量变量}定义的$J=J(k_2)$的反函数。据此可得角变量$\omega$满足的方程为
\begin{equation}
	\dif{\omega}{t} = \frac{\pl \mathcal{H}}{\pl J} = \frac{\pl \mathcal{H}}{\pl k_2} \frac{\pl k_2}{\pl J} = \frac{\pi \omega_0^2}{k_2K(k_2)}
\end{equation}
由于这个运动并不是周期运动,这说明在时间$T=\dfrac{2\pi}{\dot{\omega}}=\dfrac{2k_2K(k_2)}{\omega_0^2}$内,$q$的增量为$2\pi$。

为了得到角变量的具体形式,首先求得Hamilton特征函数
\begin{equation}
	W = \int p\mathd q = \frac{2\omega_0}{k_2} \int \sqrt{1-k_2^2\sin^2\frac q2}\mathd q = \frac{4\omega_0}{k_2} E\left(k_2, \sin\frac q2\right)
\end{equation}
其中$k_2=k_2(J)$是由式\eqref{chapter10:单摆转动情形下的作用量变量}定义的$J=J(k_2)$的反函数。根据角变量的定义式\eqref{chapter10:角变量的定义}可得
\begin{equation}
	\omega = \frac{\pl W}{\pl J} = \frac{\pl W}{\pl k_2} \frac{\pl k_2}{\pl J}
\end{equation}
对函数$W=W(k_2,q)$求$k_2$的偏导数,可得
\begin{equation}
	\frac{\pl W}{\pl k_2} = -\frac{4\omega_0}{k_2^2}E\left(k_2, \sin\frac q2\right) + \frac{4\omega_0}{k_2}\frac{\pl E}{\pl k_2}\left(k_2, \sin\frac q2\right)
\end{equation}
利用附录中的恒等式\eqref{Appendix-A:第二类不完全椭圆积分的导数}可得
\begin{equation}
	\frac{\pl W}{\pl k_2} = -\frac{4\omega_0}{k_2^2}F\left(k_2, \sin\frac q2\right)
\end{equation}
所以角变量为
\begin{equation}
	\omega = \frac{\pi}{K(k_2)}F\left(k_2,\sin\frac q2\right)
	\label{chapter10:单摆转动情形下的角变量}
\end{equation}
最后,根据式\eqref{chapter10:单摆转动情形下的角变量}和式\eqref{chapter10:单摆转动情形下能量积分的具体形式}可得
\begin{equation}
\begin{cases}
	\ds q = 2\am\left(\frac{K(k_2)\omega}{\pi}\right) \\[1.5ex]
	\ds p = \frac{2\omega_0}{k_2}\dn\left(\frac{K(k_2)\omega}{\pi}\right)
\end{cases}
\end{equation}
其中$k_2=k_2(J)$是由式\eqref{chapter10:单摆转动情形下的作用量变量}定义的$J=J(k_2)$的反函数。

\subsection{多自由度系统的作用-角变量}

如果一个系统的Hamilton-Jacobi方程可以完全分离变量,则其Hamilton特征函数具有下面的形式:
\begin{equation}
	W = \sum_{i=1}^s W_i(q_i,E,D_2,\cdots,D_s)
	\label{chapter10:多自由度系统的Hamilton特征函数}
\end{equation}
与单自由度系统的情形类似,此时有
\begin{equation}
	p_i = \frac{\pl W}{\pl q_i} = \frac{\pl W_i}{\pl q_i} = p_i(q_i,E,D_2,\cdots,D_s)\quad (i=1,2,\cdots,s)
	\label{chapter10:多自由度系统相轨迹在某一相平面内的投影}
\end{equation}
这些方程给出了$2n$维相空间中的相轨迹在平面$q_i,p_i$上的投影。类似于单自由度系统的情形,如果每个平面内的运动都是周期的,即在平面$q_i,p_i$内的相轨迹投影\eqref{chapter10:多自由度系统相轨迹在某一相平面内的投影}是闭合曲线,或者变量$q_i$是以$q_{i0}$为周期的,那么可以对此系统引入作用-角变量来考察其运动。

由于系统是完全分离变量的,所以对于固定的$E,D_2,\cdots,D_s$,在平面$q_i,p_i$内的运动都是独立的,按照单自由度系统的做法,考虑
\begin{equation}
	J_i = \frac{1}{2\pi} \oint p_i\mathd q_i = \frac{1}{2\pi} \oint \frac{\pl W_i}{\pl q_i}\mathd q_i\quad (i=1,2,\cdots,s)
	\label{chapter10:多自由度系统的作用量变量}
\end{equation}
其中积分沿着运动的一个整周期。式\eqref{chapter10:多自由度系统的作用量变量}确定了$s$个函数$J_i=J_i(E,D_2,\cdots,D_s)$,由于$q_i,p_i$的独立性,这些函数$J_i$之间是相互独立的。用$J_1,J_2,\cdots,J_s$代替原来的$E,D_2,\cdots,D_s$作为新的广义动量,那么有
\begin{equation*}
	E = f_1(J_1,J_2,\cdots,J_s),\quad D_2 = f_2(J_1,J_2,\cdots,J_s),\quad \cdots,D_s = f_s(J_1,J_2,\cdots,J_s)
\end{equation*}
将其代入式\eqref{chapter10:多自由度系统的Hamilton特征函数}中可得
\begin{equation}
	W = W(q_1,q_2,\cdots,q_s,J_1,J_2,\cdots,J_s)
\end{equation}
由关系式
\begin{equation}
	p_i = \frac{\pl W}{\pl q_i},\quad \omega_i = \frac{\pl W}{\pl J_i}\quad (i=1,2,\cdots,s)
\end{equation}
给出的正则变换将原变量$q_i,p_i$变为作用-角变量$J_i,\omega_i$,新的Hamilton函数为
\begin{equation}
	\mathcal{H} = f_1(J_1,J_2,\cdots,J_s)
\end{equation}
角坐标$\omega_i$是循环坐标。在新坐标下的正则方程为
\begin{equation}
\begin{cases}
	\ds \dif{J_i}{t} = -\dfrac{\pl \mathcal{H}}{\pl \omega_i} = 0 \\[1.5ex]
	\ds \dif{\omega_i}{t} = \dfrac{\pl \mathcal{H}}{\pl J_i}(J_1,J_2,\cdots,J_s)
\end{cases}\quad (i=1,2,\cdots,s)
\end{equation}
其中$\dfrac{\pl \mathcal{H}}{\pl J_i}$即为在相平面$q_i,p_i$中周期运动的角频率,这也说明利用这种方法可以通过研究函数$H$和$W$的性质直接得到所有周期运动的频率,完全不需要研究系统的运动。记第$i$个角变量$\omega_i$在第$j$个广义坐标$q_j$变化一个完整周期后得到的改变量为$\Delta\omega_{ij}$,则有
\begin{align*}
	\Delta\omega_{ij} = \oint \frac{\pl \omega_i}{\pl q_j}\mathd q_j = \oint \frac{\pl^2 W}{\pl J_i\pl q_j} \mathd q_j = \frac{\pl}{\pl J_i} \oint \frac{\pl W}{\pl q_j}\mathd q_j = \frac{\pl}{\pl J_i} (2\pi J_j) = 2\pi\delta_{ij}
\end{align*}

特别地,如果$q_i$是循环坐标,则相应的$p_i$是常数,在$q_i,p_i$内的相轨道是直线,那么在$q_i,p_i$平面内的运动可以认为是任意$q_{i0}$为周期的,不妨取$q_{i0}=2\pi$,那么
\begin{equation*}
	J_i = \frac{1}{2\pi}\oint p_i\mathd q_j = \frac{1}{2\pi} \int_0^{2\pi}p_i\mathd q_i = p_i
\end{equation*}

\subsection{Kepler问题中的作用-角变量}

Kepler问题中,轨道可以是椭圆、双曲线或抛物线。当轨道是双曲线或抛物线时,运动没有任何形式的周期性,故此处只考虑轨道是椭圆的情形。在椭圆轨道的情形下,运动范围为$r_1\leqslant r \leqslant r_2$,其中$r_1=\dfrac{p}{1+e}, r_2=\dfrac{p}{1-e}$,此处$p$为轨道的半通径\footnote{半通径是指,圆锥曲线过某一焦点作其长轴(或实轴、对称轴)的垂线,该垂线与圆锥曲线交点到该焦点的距离。在椭圆的情形下,它与半长轴的关系为$p=a(1-e^2)$。},$e$为轨道离心率。由此可得
\begin{equation}
	r_1+r_2 = \frac{2p}{1-e^2},\quad r_1r_2 = \frac{p^2}{1-e^2}
	\label{chapter10:Kepler问题中的作用-角变量:几何参数之间的关系}
\end{equation}

这个系统的Hamilton函数如式\eqref{Hamilton-Jacobi方程解Kepler问题1}所示,即
\begin{equation}
	H = \frac{p_r^2}{2\mu}+\frac{p_\theta^2}{2\mu r^2}+\frac{p_\phi^2}{2\mu r^2\sin^2\theta} - \frac{\alpha}{r}
	\label{chapter10:Kepler问题中的作用-角变量:Hamilton函数}
\end{equation}
其不含时Hamilton-Jacobi方程为
\begin{equation}
	\frac{1}{2\mu}\left[\left(\frac{\pl W}{\pl r}\right)^2 + \frac{1}{r^2}\left(\frac{\pl W}{\pl \theta}\right)^2 + \frac{1}{r^2\sin^2\theta}\left(\frac{\pl W}{\pl \phi}\right)^2\right] - \frac{\alpha}{r} = E
\end{equation}
这个方程的全积分已经在第\ref{chapter10:subsection-利用Hamilton-Jacobi方程求解Kepler问题}节求出,根据式\eqref{Hamilton-Jacobi方程解Kepler问题5}可得其Hamilton特征函数为\footnote{此处为简明起见,省去了积分号前的负号。}
\begin{equation}
	W = p_\phi\phi + \int \sqrt{2\mu E+\frac{2\mu\alpha}{r}-\frac{\beta^2}{r^2}} \mathd r + \int \sqrt{\beta^2-\frac{p_\phi^2}{\sin^2\theta}}\mathd \theta
\end{equation}
其中的常数$\alpha,\beta$与轨道参数的关系由式\eqref{Hamilton-Jacobi方程解Kepler问题38}给出,即
\begin{equation}
	p = \frac{\beta^2}{\mu\alpha},\quad e = \sqrt{1+\frac{2\beta^2E}{\mu\alpha^2}}
	\label{chapter10:Kepler问题中的作用-角变量:几何参数与积分常数之间的关系}
\end{equation}

下面引入作用量变量$J_r, J_\theta, J_\phi$。因为$\phi$是循环坐标,所以根据上一节末尾的叙述,可以将作用量变量$J_\phi$取为
\begin{equation}
	J_\phi = p_\phi
\end{equation}
而$J_r$和$J_\theta$则由
\begin{align}
	J_r & = \frac{1}{2\pi} \oint p_r\mathd r = \frac{1}{2\pi} \oint \frac{\pl W}{\pl r} \mathd r = \frac{1}{2\pi} \oint \sqrt{2\mu E+\frac{2\mu\alpha}{r}-\frac{\beta^2}{r^2}} \mathd r \label{chapter10:Kepler问题中的作用-角变量:作用量变量定义1} \\
	J_\theta & = \frac{1}{2\pi} \oint p_\theta\mathd \theta = \frac{1}{2\pi} \oint \sqrt{\beta^2-\frac{p_\phi^2}{\sin^2\theta}} \mathd \theta \label{chapter10:Kepler问题中的作用-角变量:作用量变量定义2}
\end{align}
来确定。

为了计算作用量变量$J_\theta$,在球坐标系下,动能可以表示为
\begin{equation}
	T = \frac12\mu\left(\dot{r}^2 + r^2\dot{\theta}^2 + r^2\sin^2\theta\dot{\phi}^2\right) = \frac12 \left(p_r\dot{r} + p_\theta\dot{\theta} + p_\phi\dot{\phi}\right)
	\label{chapter10:Kepler问题中的作用-角变量:球坐标中的动能}
\end{equation}
如果在轨道平面内引入极坐标$r, \varTheta$($\varTheta$的具体数值由式\eqref{Hamilton-Jacobi方程解Kepler问题34-1}确定),则有
\begin{equation}
	T = \frac12\mu\left(\dot{r}^2 + r^2\dot{\varTheta}^2\right) = \frac12\left(p_r\dot{r} + \mu r^2\dot{\varTheta}^2\right)
	\label{chapter10:Kepler问题中的作用-角变量:极坐标中的动能}
\end{equation}
对比式\eqref{chapter10:Kepler问题中的作用-角变量:球坐标中的动能}和式\eqref{chapter10:Kepler问题中的作用-角变量:极坐标中的动能}可得
\begin{equation}
	p_\theta\mathd \theta = \mu r^2\dot{\varTheta}\mathd \varTheta - p_\phi\mathd \phi
\end{equation}
将式\eqref{Hamilton-Jacobi方程解Kepler问题35}代入,可得
\begin{equation}
	p_\theta\mathd \theta = \beta\mathd \varTheta - p_\phi\mathd \phi
\end{equation}
质点运动一个周期的过程,角$\theta$完成一次振动,而角$\varTheta$和$\phi$将改变$2\pi$,所以根据式\eqref{chapter10:Kepler问题中的作用-角变量:作用量变量定义2}可得
\begin{equation}
	J_\theta = \beta - J_\phi
	\label{chapter10:Kepler问题中的作用-角变量:作用量变量J_theta}
\end{equation}

下面计算作用量变量$J_r$,在运动的一个周期内,$r$将从$r_1$增加至$r_2$,随后减小至$r_1$,所以可将式\eqref{chapter10:Kepler问题中的作用-角变量:作用量变量定义1}化为
\begin{equation}
	J_r = 2\times\frac{1}{2\pi} \int_{r_1}^{r_2} \sqrt{2\mu E+\frac{2\mu\alpha}{r}-\frac{\beta^2}{r^2}} \mathd r = \frac{1}{\pi} \int_{r_1}^{r_2} \sqrt{2\mu E+\frac{2\mu\alpha}{r}-\frac{\beta^2}{r^2}} \mathd r
	\label{chapter10:Kepler问题中的作用-角变量:作用量变量I_r的推导1}
\end{equation}
考虑到椭圆轨道的情形下,$E<0$,以及式\eqref{chapter10:Kepler问题中的作用-角变量:几何参数之间的关系},可以将式\eqref{chapter10:Kepler问题中的作用-角变量:作用量变量I_r的推导1}改写为
\begin{equation}
	J_r = \frac{\sqrt{-2\mu E}}{\pi} \int_{r_1}^{r_2} \frac{\sqrt{(r-r_1)(r_2-r)}}{r}\mathd r
\end{equation}
做变量代换$r=\dfrac{r_1+r_2x^2}{1+x^2}$,则有
\begin{equation}
	J_r = \frac{2\sqrt{-2\mu E}(r_2-r_1)^2}{\pi} \int_0^{+\infty} \frac{x^2}{(r_1+r_2x^2)(1+x^2)^2}\mathd x
	\label{chapter10:Kepler问题中的作用-角变量:中间过程1}
\end{equation}
将式\eqref{chapter10:Kepler问题中的作用-角变量:中间过程1}中的被积函数表示为
\begin{align*}
	\frac{x^2}{(r_1+r_2x^2)(1+x^2)^2} = \frac{1}{r_2-r_1}\left[-\frac{r_1r_2}{r_2-r_1} \frac{1}{r_1+r_2x^2} + \frac{r_1}{r_2-r_1} \frac{1}{1+x^2} + \frac{1}{(1+x^2)^2}\right]
\end{align*}
由于
\begin{equation*}
	\int_0^{+\infty} \frac{\mathd x}{r_1+r_2x^2} = \frac{\pi}{2\sqrt{r_1r_2}},\quad \int_0^{+\infty} \frac{\mathd x}{1+x^2} = \frac\pi2,\quad \int_0^{+\infty} \frac{\mathd x}{(1+x^2)^2} = \frac\pi4
\end{equation*}
将这些关系式代入式\eqref{chapter10:Kepler问题中的作用-角变量:中间过程1}中可得
\begin{equation}
	J_r = \frac{\sqrt{-2\mu E}}{2}(r_1+r_2-2\sqrt{r_1r_2})
	\label{chapter10:Kepler问题中的作用-角变量:中间过程2}
\end{equation}
将式\eqref{chapter10:Kepler问题中的作用-角变量:几何参数之间的关系}和式\eqref{chapter10:Kepler问题中的作用-角变量:几何参数与积分常数之间的关系}代入,并注意到式\eqref{chapter10:Kepler问题中的作用-角变量:作用量变量J_theta},可以将式\eqref{chapter10:Kepler问题中的作用-角变量:中间过程2}写成
\begin{equation}
	J_r = \alpha\sqrt{-\frac{\mu}{2E}} - (J_\theta+J_\phi)
	\label{chapter10:Kepler问题中的作用-角变量:作用量变量J_r}
\end{equation}
从式\eqref{chapter10:Kepler问题中的作用-角变量:作用量变量J_r}中可得
\begin{equation}
	E = -\frac{\mu\alpha^2}{2(J_r+J_\theta+J_\phi)^2}
\end{equation}
进而可得变换后的Hamilton函数为
\begin{equation}
	\mathcal{H} = -\frac{\mu\alpha^2}{2(J_r+J_\theta+J_\phi)^2}
\end{equation}
与作用量变量$J_r, J_\theta, J_\phi$的共轭的角变量记为$\omega_r, \omega_\theta, \omega_\phi$,因为有
\begin{equation*}
	\frac{\pl \mathcal{H}}{\pl J_r} = \frac{\pl \mathcal{H}}{\pl J_\theta} = \frac{\pl \mathcal{H}}{\pl J_\phi}
\end{equation*}
所以三个角变量的频率相等,这也与椭圆轨道的周期性和封闭性相对应。

\subsection{Delaunay元素}

下面引入比上节给出的作用-角变量$J_r, J_\theta, J_\phi, \omega_r, \omega_\theta, \omega_\phi$具有更明确几何意义和物理意义的新变量$J_i, \omega_i(i=1,2,3)$。为此,先做变换
\begin{align}
	& \omega_1 = \omega_\phi-\omega_\theta,\quad \omega_2 = \omega_\theta-\omega_r,\quad \omega_3 = \omega_r \label{chapter10:德洛内元素的定义-角变量} \\
	& J_1 = J_\phi,\quad J_2 = J_\theta+J_\phi,\quad J_3 = J_r+J_\theta+J_\phi \label{chapter10:德洛内元素的定义-作用量变量}
\end{align}
容易验证式\eqref{chapter10:德洛内元素的定义-角变量}和式\eqref{chapter10:德洛内元素的定义-作用量变量}表示的变换是正则变换。

利用这组新变量,新的Hamilton函数可以表示为
\begin{equation}
	\mathcal{H} = -\frac{\mu\alpha^2}{2J_3^2}
\end{equation}

下面解释新变量$J_i,\omega_i$的物理意义。由式\eqref{chapter10:Kepler问题中的作用-角变量:作用量变量J_r}可得
\begin{equation}
	J_3 = \alpha\sqrt{-\frac{\mu}{2E}}
\end{equation}
由于半长轴满足
\begin{equation}
	a = \frac12(r_1+r_2) = \frac{p}{1-e^2} = -\frac{\alpha}{2E}
	\label{chapter10:德洛内元素-半长轴}
\end{equation}
所以
\begin{equation}
	J_3 = \sqrt{\mu\alpha a}
\end{equation}
进而有
\begin{equation}
	\dif{\omega_3}{t} = \frac{\pl \mathcal{H}}{\pl J_3} = \frac{\mu\alpha^2}{J_3^3} = \sqrt{\frac{\alpha}{\mu a^3}}
	\label{chapter10:德洛内元素的物理意义1}
\end{equation}
根据角变量的物理意义可知,式\eqref{chapter10:德洛内元素的物理意义1}中的$\dif{\omega_3}{t}$实际上应该等于$\dfrac{2\pi}{T}$,其中$T$为椭圆轨道的周期,这个结果与第\ref{chapter4:subsection-Kepler行星运动定律}节的Kepler第三定律式\eqref{chapter4:Kepler第三定律1}的结果是相同的。矢径转动的平均角速度$n=\dfrac{2\pi}{T}$在天体力学中称为{\bf 平运动},由此可知
\begin{equation*}
	\dif{\omega_3}{t} = n
\end{equation*}
于是可令
\begin{equation}
	\omega_3 = n(t-t_0)
\end{equation}
其中$t_0$为过近心点时刻,这个角度在天体力学中称为{\bf 平近点角}。

下面来看共轭正则变量$J_2, \omega_2$。根据式\eqref{chapter10:Kepler问题中的作用-角变量:作用量变量J_theta}可得$J_2=\beta$。而对比式\eqref{chapter10:Kepler问题中的作用-角变量:几何参数与积分常数之间的关系}和第\ref{chapter4:subsection-轨道方程}节的式\eqref{chapter4:Kepler问题的轨道参数}可知,作用量变量$J_2$实际上是质点$\mu$对力心的角动量大小。利用式\eqref{chapter10:德洛内元素-半长轴}和式\eqref{chapter10:Kepler问题中的作用-角变量:几何参数与积分常数之间的关系}可得
\begin{equation}
	J_2 = \sqrt{\mu\alpha a(1-e^2)}
\end{equation}
因为$\dif{\omega_2}{t} = \dfrac{\pl \mathcal{H}}{\pl J_2} = 0$,所以$\omega_2$是在轨道平面内的某个常值角度,可令
\begin{equation}
	\omega_2 = \tilde{\omega}
\end{equation}
其中$\tilde{\omega}$是近心点幅角(见第\ref{chapter10:subsection-利用Hamilton-Jacobi方程求解Kepler问题}节)。

最后来看变量$J_1, \omega_1$,根据式\eqref{chapter10:德洛内元素的定义-作用量变量}可知$J_1=J_\phi=p_\phi$,即$J_1$是质点$\mu$对力心角动量在$z$轴上的投影,所以
\begin{equation}
	J_1 = \beta\cos i = J_2\cos i = \sqrt{\mu\alpha a(1-e^2)}\cos i
\end{equation}
其中$i$是轨道倾角。再考虑到$\dif{\omega_1}{t} = \dfrac{\pl \mathcal{H}}{\pl J_1} = 0$可得,$\omega_1$是平面$Oxy$内的某个常值角度,可令
\begin{equation}
	\omega_1 = \varOmega
\end{equation}
其中$\varOmega$为升交点赤经。

这样引入的共轭正则变量$J_1,J_2,J_3,\omega_1,\omega_2,\omega_3$称为Delaunay正则变量或简称Delaunay元素\footnote{Charles-Eugène Delaunay,德洛内,法国天文学家、数学家。},常用记号$H,G,L,h,g,l$来表示(不要将这些符号与其它符号相同的物理量混淆!),现列举如下:
\begin{equation}
\begin{cases}
	H = \sqrt{\mu\alpha a(1-e^2)}\cos i,\quad h=\varOmega \\
	G = \sqrt{\mu\alpha a(1-e^2)},\quad g=\tilde{\omega} \\
	L = \sqrt{\mu\alpha a},\quad l = n(t-t_0)
\end{cases}
\end{equation}

\section{从“几何力学”到“波动力学”}

\subsection{作用波与作用波面}

简明起见,首先考虑一个质点的Hamilton函数
\begin{equation*}
	H(\mbf{r},\mbf{p}) = \frac{\mbf{p}^2}{2m} + V(\mbf{r})
\end{equation*}
此时,质点的Hamilton主函数为
\begin{equation*}
	S(\mbf{r},\mbf{D},t) = W(\mbf{r},\mbf{D}) - Et
\end{equation*}
其中$W$为该质点的Hamilton特征函数。

在这里,方程“$W=\text{常数}$”给出了空间中的一族静止曲面,而方程“$S=\text{常数}$”给出的则是在空间中一族随时间变化的曲面。在某一时刻$t$,曲面族“$S=\text{常数}$”的某一成员“$S=S_0$”将与曲面族“$W=\text{常数}$”中的某一成员“$W=S_0+Et$”重合,而在下一时刻,“$S=S_0$”将重合于“$W=\text{常数}$”中的另一成员。这样,曲面“$S=S_0$”可以看作是在空间中传播的某种“波面”。

根据变换公式\eqref{chapter10:Hamilton-Jacobi方程-动量与Hamilton主函数之间的关系}可知
\begin{equation}
	\mbf{p} = \frac{\pl S}{\pl \mbf{r}} = \bnb S
	\label{chapter10:作用波与作用波面-式1}
\end{equation}
即该质点的动量总是沿着波面的法线方向,与波面相正交的曲线族就是波线族。既然波线的方向与质点的动量方向相同,由此可知,每一根波线实际上都是质点在不同初始条件下的可能轨道。据此,可以将等$S$面形成的这种波称为{\bf 作用波},其波面即称为{\bf 作用波面}。

% 对$S=W(\mbf{r},\mbf{D})-Et=S_0$两端求微分可得
% \begin{equation}
	% \mathd S = \bnb W \cdot \mathd \mbf{r}-E\mathd t = 0
% \end{equation}
% 考虑到$\bnb W \parallel \mathd \mbf{r}$,据此可以得到作用波的相速度
% \begin{equation}
	% u = \dif{r}{t} = \frac{E}{|\bnb W|}
% \end{equation}
% 由式\eqref{chapter10:作用波与作用波面-式1}可得
% \begin{equation*}
	% \mbf{p} = \bnb S = \bnb W
% \end{equation*}
% 所以有
% \begin{equation}
	% u = \frac{E}{p}
% \end{equation}

\begin{figure}[htb]
\centering
\begin{asy}
	size(300);
	//作用波波面示意图
	picture pic;
	string[] text = {"$t_0$","$t$","$t'$"};
	real[] taus = {2,2.5,3};
	real[] sigmas = {2.5,3.5,4.5};
	for(real tau : taus){
		pair para1(real sigma){
			return (sigma*tau,(tau**2-sigma**2)/2);
		}
		draw(pic,rotate(56)*graph(para1,2,5),Arrow);
	}
	int i = 0;
	for(real sigma:sigmas){
		pair para2(real tau){
			return (sigma*tau,(tau**2-sigma**2)/2);
		}
		draw(pic,Label(text[i],BeginPoint,black),rotate(56)*graph(para2,1.8,3.15),red+dashed);
		i = i+1;
	}
	real tau = taus[1];
	real sigma = sigmas[1];
	pair para1dir(real sigma,real tau){
		return (tau,-sigma);
	}
	pair para1(real sigma,real tau){
		return (sigma*tau,(tau**2-sigma**2)/2);
	}
	dot(pic,rotate(56)*para1(sigma,tau));
	draw(pic,Label("$\boldsymbol{p}$",EndPoint,Relative(W)),rotate(56)*(para1(sigma,tau)--para1(sigma,tau)+0.5*para1dir(sigma,tau)),Arrow);
	//label(pic,"$W(\boldsymbol{r})=S_0+Et$",rotate(56)*para1(sigmas[1],taus[2]),6*N);
	add(pic);
\end{asy}
\caption{作用波面示意图}
\label{作用波波面示意图}
\end{figure}

\subsection{波动光学与几何光学的关系}

光是一种电磁波,介质中电磁波满足的波动方程为
\begin{equation}
	\nabla^2 \varPhi - \frac{1}{u^2} \frac{\pl^2 \varPhi}{\pl t^2} = 0
	\label{介质电磁波方程}
\end{equation}
式中$\varPhi$为电磁场的任一分量,$u(\mbf{r}) = \dfrac{c}{n(\mbf{r})}$为非均匀介质中的波速。

对单色波,其波函数可表示为
\begin{equation*}
	\varPhi(\mbf{r},t) = \phi(\mbf{r}) \mathrm{e}^{-\mathrm{i} \omega t}
\end{equation*}
将其代入式\eqref{介质电磁波方程}中可得振幅$\phi(\mbf{r})$应满足的方程为
\begin{equation}
	\nabla^2 \phi + k^2 \phi = 0
	\label{单色波振幅方程}
\end{equation}
式中$k = \dfrac{\omega}{u} = n\dfrac{\omega}{c} = nk_0$,其中$k_0$为此电磁波在真空中的波数。

在均匀介质中,单色波的振幅为
\begin{equation*}
	\phi(\mbf{r}) = A\mathrm{e}^{\mathrm{i}\mbf{k}\cdot \mbf{r}} = A \mathrm{e}^{\mathrm{i} k_0 L(\mbf{r})}
\end{equation*}
式中,函数$L(\mbf{r}) = \dfrac{\mbf{k}}{k} \cdot n\mbf{r}$称为{\heiti 光程函数},简称{\bf 程函}。即单色波在均匀介质中的波函数为
\begin{equation*}
	\varPhi(\mbf{r},t) = A \mathrm{e}^{\mathrm{i}(k_0 L(\mbf{r})-\omega t)}
\end{equation*}
波函数的等相位面即为{\heiti 波面},单色波在均匀介质中的波面为平面族:
\begin{equation*}
	\theta = k_0 L(\mbf{r}) - \omega t = \theta_0
\end{equation*}
波面的正交曲线族即为{\heiti 光线},单色波在均匀介质中的光线为直线。

\begin{figure}[htb]
\centering
\begin{asy}
	size(300);
	//均匀介质中单色波的波面和光线
	pair A,B,h;
	A = (0,0);
	B = (2,0);
	h = 0.25*dir(90);
	draw(A--B,Arrow);
	draw(shift(h)*(A--B),Arrow);
	draw(shift(2*h)*(A--B),Arrow);
	draw(Label("$t_0$",BeginPoint,black),interp(A,B,0.15)-0.5*h--interp(A,B,0.15)+2.5*h,red+dashed);
	draw(Label("$t$",BeginPoint,black),interp(A,B,0.5)-0.5*h--interp(A,B,0.5)+2.5*h,red+dashed);
	draw(Label("$t'$",BeginPoint,black),interp(A,B,0.85)-0.5*h--interp(A,B,0.85)+2.5*h,red+dashed);
	dot(interp(A,B,0.5)+h);
	draw(Label("$\boldsymbol{k}$",EndPoint,Relative(E)),interp(A,B,0.5)+h--interp(A,B,0.5)+h+0.4*dir(0),Arrow);
\end{asy}
\caption{均匀介质中单色波的波面和光线}
\label{均匀介质中单色波的波面和光线}
\end{figure}

在非均匀介质中,单色波的振幅可以表示为
\begin{equation*}
	\phi(\mbf{r}) = A(\mbf{r}) \mathrm{e}^{\mathrm{i}k_0 L(\mbf{r})}
\end{equation*}
将其代入式\eqref{单色波振幅方程}中,可得
\begin{equation*}
	\nabla^2 \phi = \mathrm{e}^{\mathrm{i}k_0 L} \left[\nabla^2 A + \mathrm{i} k_0(A\nabla^2 L + 2\bnb A \cdot \bnb L) - k_0^2 A(\bnb L)^2 \right] = -k^2 A \mathrm{e}^{\mathrm{i} k_0 L}
\end{equation*}
即有
\begin{equation}
	\frac{1}{k_0^2} \nabla^2 A + \frac{\mathrm{i}}{k_0} (A\nabla^2 L + 2\bnb A \cdot \bnb L) - \left[(\bnb L)^2 - n^2\right] A = 0
	\label{非均匀介质中的单色波方程}
\end{equation}
在短波长极限(几何光学极限)下,即$\lambda_0 \to 0$时,此时$k_0 \to +\infty$,在此近似下式\eqref{非均匀介质中的单色波方程}的前两项均为小量,忽略之可得
\begin{equation}
	(\bnb L)^2 = n^2
	\label{程函方程}
\end{equation}
方程\eqref{程函方程}是几何光学的基本方程,称为{\heiti 程函方程}。在此情形下,单色波的相位为
\begin{equation*}
	\psi(\mbf{r},t) = k_0 L(\mbf{r}) - \omega t
\end{equation*}
波面(即波函数的等相位面)的方程为
\begin{equation*}
	\psi(\mbf{r},t) = k_0 L(\mbf{r}) - \omega t = \psi_0
\end{equation*}
或者
\begin{equation*}
	k_0 L(\mbf{r}) = \psi_0 + \omega t
\end{equation*}
根据程函方程\eqref{程函方程}可得相位$\psi(\mbf{r},t)$满足
\begin{equation*}
	|\bnb \psi| = k_0 n = k
\end{equation*}
由此可有
\begin{equation*}
	\bnb \psi = \mbf{k}
\end{equation*}
即,光线(波面的正交曲线族)可视为“光微粒”的轨道。光的“微粒说”与“波动说”在短波长极限下统一。

\subsection{物质波假说与Schr\"{o}dinger方程}

在$\lambda \to 0$时,波动光学的极限即为几何光学。是否在某种极限下,能够有某种波动力学,其极限为经典力学?

de Broglie提出物质波假说,即质点所对应的“物质波”角频率和波数满足
\begin{equation}
\begin{cases}
	E = \hbar \omega \\
	\mbf{p} = \hbar \mbf{k}
\end{cases}
\end{equation}
假设“单色”物质波振幅同样满足方程\eqref{单色波振幅方程},即
\begin{equation*}
	\nabla^2 \phi + k^2 \phi = 0
\end{equation*}
再考虑到
\begin{equation*}
	k = \frac{p}{\hbar} = \frac{\sqrt{2m(E-V(\mbf{r}))}}{\hbar}
\end{equation*}
可得物质波振幅应满足的方程为
\begin{equation}
	-\frac{\hbar^2}{2m} \nabla^2\phi + V(\mbf{r})\phi = E\phi
	\label{定态Schrodinger方程}
\end{equation}
方程\eqref{定态Schrodinger方程}即为{\heiti 定态Schr\"{o}dinger方程}。

对于含时的情况下,类比于光学的情况,可设波函数的形式为
\begin{equation*}
	\varPhi(\mbf{r},t) = \phi(\mbf{r}) \mathrm{e}^{-\mathrm{i} \omega t} = \phi(\mbf{r}) \mathrm{e}^{-\mathrm{i} \frac{E}{\hbar} t}
\end{equation*}
其满足方程
\begin{equation}
	\mathrm{i}\hbar \frac{\pl \varPhi}{\pl t} = E\varPhi
\end{equation}
将定态Schr\"{o}dinger方程代入上式,即有{\heiti 含时Schr\"{o}dinger方程}
\begin{equation}
	\mathrm{i}\hbar \frac{\pl \varPhi}{\pl t} = -\frac{\hbar^2}{2m} \nabla^2\varPhi + V(\mbf{r})\varPhi
	\label{含时Schrodinger方程}
\end{equation}
类比$\bnb \dfrac{S}{\hbar} = \dfrac{\mbf{p}}{\hbar} = \mbf{k}$与波动光学的$\bnb \psi = \mbf{k}$,可知波函数的相位即为$\dfrac{S}{\hbar}$,因此波函数又可表示为
\begin{equation*}
	\varPhi(\mbf{r},t) = |\varPhi(\mbf{r},t)| \mathrm{e}^{\mathrm{i}\frac{S(\mbf{r},t)}{\hbar}}
\end{equation*}
据此考虑含时Schr\"{o}dinger方程\eqref{含时Schrodinger方程}的实部,即
\begin{equation*}
	\frac{\pl S}{\pl t} = -\left[\frac{1}{2m} (\bnb S)^2 + V - \frac{\hbar^2}{2m} \frac{\nabla^2 |\varPhi|}{|\varPhi|}\right]
\end{equation*}
取上式在$\hbar \to 0$的极限,可有
\begin{equation*}
	\frac{1}{2m} (\bnb S)^2 + V + \frac{\pl S}{\pl t} = 0
\end{equation*}
此即为单粒子的Hamilton-Jacobi方程。因此,量子力学在$\hbar \to 0$下的极限即为经典力学。

\section{Hamilton理论在物理学中的应用}

\subsection{连续系统的Lagrange方程}

首先,以一维轻质长弦的横振动为例导出一维弦的Lagrange函数。记弦的质量线密度为$\rho$,各点相对其平衡位置的位移可以表示为$\eta(x)$。首先将弦分为$n$等份,记每段弦的长度为$\Delta x$,作为$n$个质量为$\rho \Delta x$的质点处理。将这些质点的位移分别记作$\eta_i$,则它们的动能可以表示为
\begin{equation}
	T = \sum_{i=1}^n \frac12 \rho \Delta x \dot{\eta}_i^2
	\label{连续系统的Lagrange方程-1}
\end{equation}

\begin{figure}[htb]
\centering
\begin{asy}
	size(300);
	//一维弦的横振动
	real x,y;
	x = 2.2;
	y = 1.4;
	real f(real x){
		return atan(x);
	}
	draw(Label("$x$",EndPoint),(0,0)--(x,0),Arrow);
	draw(Label("$y$",EndPoint),(0,0)--(0,y),Arrow);
	path p = graph(f,0.2,2);
	draw(p,linewidth(0.8bp));
	pair A0,A1,A2,A3,A4,A5,A6,T1,T2;
	A0 = relpoint(p,0.2);
	A1 = relpoint(p,0.3);
	A2 = relpoint(p,0.4);
	T1 = -reldir(p,0.4);
	A3 = relpoint(p,0.5);
	A4 = relpoint(p,0.6);
	T2 = reldir(p,0.6);
	A5 = relpoint(p,0.7);
	A6 = relpoint(p,0.8);
	draw((A0.x,0)--A0,dashed);
	draw(Label("$\eta_{i-1}$",MidPoint,Relative(E)),(A1.x,0)--A1);
	draw((A2.x,0)--A2,dashed);
	draw(Label("$\eta_i$",MidPoint,Relative(E)),(A3.x,0)--A3);
	draw((A4.x,0)--A4,dashed);
	draw(Label("$\eta_{i+1}$",MidPoint,Relative(E)),(A5.x,0)--A5);
	draw((A6.x,0)--A6,dashed);
	real l,r;
	l = 0.3;
	dot(A2);
	draw(Label("$\boldsymbol{T}$",EndPoint,Relative(E)),A2--A2+l*T1,Arrow);
	dot(A4);
	draw(Label("$\boldsymbol{T}$",EndPoint,Relative(W)),A4--A4+l*T2,Arrow);
	draw(A2--A2+l*dir(180),dashed);
	draw(A4--A4+l*dir(0),dashed);
	r = 0.15;
	draw(Label("$\theta_{i-\frac12}$",BeginPoint),arc(A2,r,180,degrees(T1)));
	draw(Label("$\theta_{i+\frac12}$",BeginPoint),arc(A4,r,0,degrees(T2)));
\end{asy}
\caption{一维弦的横振动}
\label{chp3:一维弦的横振动}
\end{figure}

如图\ref{chp3:一维弦的横振动}所示,在微振动假设下,角$\theta_{i-\frac12}$和$\theta_{i+\frac12}$都很小,故有第$i$个质点所受的横向力为
\begin{align}
	F_i & = T\sin \theta_{i+\frac12} - T\sin \theta_{i-\frac12} \approx T\tan \theta_{i+\frac12} - T\tan \theta_{i-\frac12} \approx T \frac{\eta_{i+1}-\eta_i}{\Delta x} - T \frac{\eta_i-\eta_{i-1}}{\Delta x} \nonumber \\
	& = T\frac{\eta_{i+1}-\eta_{i-1}}{\Delta x}
	\label{连续系统的Lagrange方程-2}
\end{align}
由此可推知此系统的势能为
\begin{equation}
	V = \sum_{i=1}^n \frac12 T\left(\frac{\eta_{i+1}-\eta_i}{\Delta x}\right)^2 \Delta x
	\label{连续系统的Lagrange方程-3}
\end{equation}
令$n \to \infty$,同时$\Delta x\to 0$,则有
\begin{align}
	T & \to \int \frac12 \rho \left(\frac{\pl \eta}{\pl t}\right)^2 \mathrm{d}x \label{连续系统的Lagrange方程-4} \\
	V & \to \int \frac12 T\left(\frac{\pl \eta}{\pl x}\right)^2 \mathrm{d}x \label{连续系统的Lagrange方程-5}
\end{align}
由此得到一维弦的Lagrange函数为
\begin{equation}
	L = T-V = \int \left[\frac12 \rho \left(\frac{\pl \eta}{\pl t}\right)^2 - \frac12 T\left(\frac{\pl \eta}{\pl x}\right)^2\right] \mathrm{d}x = \int \mathscr{L} \mathrm{d}x
	\label{连续系统的Lagrange方程-5}
\end{equation}
其中
\begin{equation}
	\mathscr{L} = \mathscr{T} - \mathscr{V} = \frac12 \rho \left(\frac{\pl \eta}{\pl t}\right)^2 - \frac12 T \left(\frac{\pl \eta}{\pl x}\right)^2 
	\label{连续系统的Lagrange方程-6}
\end{equation}
称为该连续系统的{\bf Lagrange密度},而$\mathscr{T}$和$\mathscr{V}$则分别称为该连续系统的{\bf 动能密度}和{\bf 势能密度}。

由上面的讨论可以看出,连续系统的位置坐标$x$在Lagrange函数中并不是广义坐标\footnote{实际上,此连续系统的Lagrange函数与位置坐标$x$无关。},而仅仅是取代了分离系统中的求和指标$i$。对于每一个$x$和$t$,$\eta(x,t)$是广义坐标,而$x$和$t$同为Lagrange函数的参数。

对于一般的三维连续系统,其广义坐标可以取为$\eta_1(\mbf{x},t),\eta_2(\mbf{x},t),\cdots,\eta_s(\mbf{x},t)$,而系统的Lagrange函数则表示为
\begin{equation}
	L = \int_V \mathscr{L} \mathrm{d}V
	\label{连续系统的Lagrange方程-7}
\end{equation}
而三维空间中的Lagrange密度的一般形式为
\begin{equation*}
	\mathscr{L} = \mathscr{L} \left(\frac{\pl \eta_1}{\pl t}, \frac{\pl \eta_2}{\pl t}, \cdots, \frac{\pl \eta_s}{\pl t}, \bnb \eta_1, \bnb \eta_2, \cdots, \bnb \eta_s, \eta_1, \eta_2, \cdots, \eta_s, \mbf{x},t\right)
\end{equation*}
或者简单地记作
\begin{equation}
	\mathscr{L} = \mathscr{L} \left(\frac{\pl \mbf{\eta}}{\pl t}, \bnb \mbf{\eta}, \mbf{\eta}, \mbf{x},t\right)
	\label{连续系统的Lagrange方程-8}
\end{equation}
由此,根据Hamilton原理,系统的Lagrange方程由式
\begin{equation}
	\delta S = \delta \int_{t_1}^{t_2} \int_V \mathscr{L} \mathrm{d}V \mathrm{d}t = \int_{t_1}^{t_2} \int_V \delta \mathscr{L} \mathrm{d}V \mathrm{d}t = 0
	\label{连续系统的Lagrange方程-9}
\end{equation}
给出。由于此连续系统的广义坐标为$\mbf{\eta}(\mbf{x},t)$,因此此处的变分运算不仅满足$\delta t = 0$,而且满足$\delta \mbf{x} = \mbf{0}$。因此有
\begin{align}
	\delta \mathscr{L} = \sum_{k=1}^s \Bigg[\frac{\pl \mathscr{L}}{\pl \left(\dfrac{\pl \eta_k}{\pl t}\right)} \delta \dfrac{\pl \eta_k}{\pl t} + \sum_{j=1}^3 \frac{\pl \mathscr{L}}{\pl \left(\dfrac{\pl \eta_k}{\pl x_j}\right)} \delta \dfrac{\pl \eta_k}{\pl x_j} + \frac{\pl \mathscr{L}}{\pl \eta_k} \delta \eta_k\Bigg]
	\label{连续系统的Lagrange方程-10}
\end{align}
由于此处变分是对$\eta_k(\mbf{x},t)$的,因此可有
\begin{equation*}
	\delta \frac{\pl \eta_k}{\pl t} = \frac{\pl (\delta \eta_k)}{\pl t},\quad \delta \frac{\pl \eta_k}{\pl x_j} = \frac{\pl (\delta \eta_k)}{\pl x_j}
\end{equation*}
据此处理式\eqref{连续系统的Lagrange方程-10}中的各项,可有
\begin{align}
	\frac{\pl \mathscr{L}}{\pl \left(\dfrac{\pl \eta_k}{\pl t}\right)} \delta \dfrac{\pl \eta_k}{\pl t} & = \frac{\pl \mathscr{L}}{\pl \left(\dfrac{\pl \eta_k}{\pl t}\right)} \dfrac{\pl (\delta \eta_k)}{\pl t} \nonumber \\
	& = \frac{\mathrm{d}}{\mathrm{d}t} \Bigg(\frac{\pl \mathscr{L}}{\pl \left(\dfrac{\pl \eta_k}{\pl t}\right)} \delta \eta_k\Bigg) - \frac{\mathrm{d}}{\mathrm{d}t} \Bigg(\frac{\pl \mathscr{L}}{\pl \left(\dfrac{\pl \eta_k}{\pl t}\right)}\Bigg) \delta \eta_k \label{连续系统的Lagrange方程-11} \\
	\frac{\pl \mathscr{L}}{\pl \left(\dfrac{\pl \eta_k}{\pl x_j}\right)} \delta \dfrac{\pl \eta_k}{\pl x_j} & = \frac{\pl \mathscr{L}}{\pl \left(\dfrac{\pl \eta_k}{\pl x_j}\right)} \dfrac{\pl (\delta \eta_k)}{\pl x_j} \nonumber \\
	& = \frac{\mathrm{d}}{\mathrm{d}x_j} \Bigg(\frac{\pl \mathscr{L}}{\pl \left(\dfrac{\pl \eta_k}{\pl x_j}\right)} \delta \eta_k\Bigg) - \frac{\mathrm{d}}{\mathrm{d}x_j} \Bigg(\frac{\pl \mathscr{L}}{\pl \left(\dfrac{\pl \eta_k}{\pl x_j}\right)}\Bigg) \delta \eta_k \label{连续系统的Lagrange方程-12}
\end{align}
将式\eqref{连续系统的Lagrange方程-11}和式\eqref{连续系统的Lagrange方程-12}代入式\eqref{连续系统的Lagrange方程-10}中,可得
\begin{align}
	\delta \mathscr{L} & = \sum_{k=1}^s \Bigg[\frac{\mathrm{d}}{\mathrm{d}t} \Bigg(\frac{\pl \mathscr{L}}{\pl \left(\dfrac{\pl \eta_k}{\pl t}\right)} \delta \eta_k\Bigg) + \sum_{j=1}^3 \frac{\mathrm{d}}{\mathrm{d}x_j} \Bigg(\frac{\pl \mathscr{L}}{\pl \left(\dfrac{\pl \eta_k}{\pl x_j}\right)} \delta \eta_k\Bigg)\Bigg] \nonumber \\
	& \quad {}- \sum_{k=1}^s \Bigg[\frac{\mathrm{d}}{\mathrm{d}t} \Bigg(\frac{\pl \mathscr{L}}{\pl \left(\dfrac{\pl \eta_k}{\pl t}\right)}\Bigg) + \sum_{j=1}^3 \frac{\mathrm{d}}{\mathrm{d}x_j} \Bigg(\frac{\pl \mathscr{L}}{\pl \left(\dfrac{\pl \eta_k}{\pl x_j}\right)}\Bigg) - \frac{\pl \mathscr{L}}{\pl \eta_k}\Bigg] \delta \eta_k
	\label{连续系统的Lagrange方程-13}
\end{align}
此处,对函数$f\left(\dfrac{\pl \eta}{\pl t},\bnb \eta,\eta,\mbf{x},t\right)$的求导操作$\dfrac{\mathrm{d} f}{\mathrm{d}t}$和$\dfrac{\mathrm{d} f}{\mathrm{d}x_j}$分别表示对函数
\begin{equation*}
	g(\mbf{x},t) = f\left(\dfrac{\pl \eta}{\pl t}(\mbf{x},t),\bnb \eta(\mbf{x},t),\eta(\mbf{x},t),\mbf{x},t\right)
\end{equation*}
求$x_j$和$t$的偏导数。将式\eqref{连续系统的Lagrange方程-13}代入式\eqref{连续系统的Lagrange方程-9}可得
\begin{align}
	\delta S & = \int_{t_1}^{t_2} \int_V \sum_{k=1}^s \Bigg[\frac{\mathrm{d}}{\mathrm{d}t} \Bigg(\frac{\pl \mathscr{L}}{\pl \left(\dfrac{\pl \eta_k}{\pl t}\right)} \delta \eta_k\Bigg) + \sum_{j=1}^3 \frac{\mathrm{d}}{\mathrm{d}x_j} \Bigg(\frac{\pl \mathscr{L}}{\pl \left(\dfrac{\pl \eta_k}{\pl x_j}\right)} \delta \eta_k\Bigg)\Bigg] \mathrm{d}V\mathrm{d}t \nonumber \\
	& \quad {}- \int_{t_1}^{t_2} \int_V \sum_{k=1}^s \Bigg[\frac{\mathrm{d}}{\mathrm{d}t} \Bigg(\frac{\pl \mathscr{L}}{\pl \left(\dfrac{\pl \eta_k}{\pl t}\right)}\Bigg) + \sum_{j=1}^3 \frac{\mathrm{d}}{\mathrm{d}x_j} \Bigg(\frac{\pl \mathscr{L}}{\pl \left(\dfrac{\pl \eta_k}{\pl x_j}\right)}\Bigg) - \frac{\pl \mathscr{L}}{\pl \eta_k}\Bigg] \delta \eta_k \mathrm{d}V\mathrm{d}t \nonumber \\
	& = \int_V \sum_{k=1}^s \frac{\pl \mathscr{L}}{\pl \left(\dfrac{\pl \eta_k}{\pl t}\right)} \delta \eta_k \mathrm{d}V \bigg|_{t_1}^{t_2} + \int_{t_1}^{t_2} \oint_{\pl V} \sum_{k=1}^s \frac{\pl \mathscr{L}}{\pl \left(\dfrac{\pl \eta_k}{\pl x_j}\right)} \delta \eta_k n_j \mathrm{d}S \mathrm{d}t \nonumber \\
	& \quad {}- \int_{t_1}^{t_2} \int_V \sum_{k=1}^s \Bigg[\frac{\mathrm{d}}{\mathrm{d}t} \Bigg(\frac{\pl \mathscr{L}}{\pl \left(\dfrac{\pl \eta_k}{\pl t}\right)}\Bigg) + \sum_{j=1}^3 \frac{\mathrm{d}}{\mathrm{d}x_j} \Bigg(\frac{\pl \mathscr{L}}{\pl \left(\dfrac{\pl \eta_k}{\pl x_j}\right)}\Bigg) - \frac{\pl \mathscr{L}}{\pl \eta_k}\Bigg] \delta \eta_k \mathrm{d}V\mathrm{d}t
	\label{连续系统的Lagrange方程-14}
\end{align}
由于在边界上有$\delta \eta_k = 0$,即$\delta\eta_k(\mbf{x},t_1) = \delta\eta_k(\mbf{x},t_2) = 0$以及在$\pl V$上,$\delta \eta_k(\mbf{x},t) = 0$,因此式\eqref{连续系统的Lagrange方程-14}可化为
\begin{equation}
	\delta S = - \int_{t_1}^{t_2} \int_V \sum_{k=1}^s \Bigg[\frac{\mathrm{d}}{\mathrm{d}t} \Bigg(\frac{\pl \mathscr{L}}{\pl \left(\dfrac{\pl \eta_k}{\pl t}\right)}\Bigg) + \sum_{j=1}^3 \frac{\mathrm{d}}{\mathrm{d}x_j} \Bigg(\frac{\pl \mathscr{L}}{\pl \left(\dfrac{\pl \eta_k}{\pl x_j}\right)}\Bigg) - \frac{\pl \mathscr{L}}{\pl \eta_k}\Bigg] \delta \eta_k \mathrm{d}V\mathrm{d}t
	\label{连续系统的Lagrange方程-15}
\end{equation}
由于$\delta\eta_k$之间是相互独立的,故可得
\begin{equation}
	\frac{\mathrm{d}}{\mathrm{d}t} \Bigg(\frac{\pl \mathscr{L}}{\pl \left(\dfrac{\pl \eta_k}{\pl t}\right)}\Bigg) + \sum_{j=1}^3 \frac{\mathrm{d}}{\mathrm{d}x_j} \Bigg(\frac{\pl \mathscr{L}}{\pl \left(\dfrac{\pl \eta_k}{\pl x_j}\right)}\Bigg) - \frac{\pl \mathscr{L}}{\pl \eta_k} = 0,\quad k=1,2,\cdots,s
	\label{连续系统的Lagrange方程-16}
\end{equation}
式\eqref{连续系统的Lagrange方程-16}就是{\bf 连续系统的Lagrange方程}。

\begin{example}
获得一维轻质长弦在微振动近似下的横振动方程。
\end{example}
\begin{solution}
前文已得出一维轻质长弦在此情形下的Lagrange密度如式\eqref{连续系统的Lagrange方程-6}所示,即
\begin{equation*}
	\mathscr{L}\left(\frac{\pl \eta}{\pl t},\frac{\pl \eta}{\pl x},\eta,x,t\right) = \frac12 \rho \left(\frac{\pl \eta}{\pl t}\right)^2 - \frac12 T \left(\frac{\pl \eta}{\pl x}\right)^2 
\end{equation*}
将式\eqref{连续系统的Lagrange方程-6}代入\eqref{连续系统的Lagrange方程-16}中,即可得到
\begin{equation}
	\rho \frac{\pl^2 \eta}{\pl t^2} - T\frac{\pl^2 \eta}{\pl x^2} = 0
	\label{连续系统的Lagrange方程-17}
\end{equation}
这就是一维轻质长弦微振动的方程。这是一个波动方程,波速为$v = \sqrt{\dfrac{T}{\rho}}$。
\end{solution}

\subsection{电磁场的Lagrange方程}

真空中的电磁场的运动规律由Maxwell方程组给出:
\begin{subnumcases}{\label{电磁场的Lagrange方程-1}}
	\bnb \cdot \mbf{B} = 0 \label{电磁场的Lagrange方程-1.1} \\
	\bnb \times \mbf{E} + \frac{\pl \mbf{B}}{\pl t} = \mbf{0} \label{电磁场的Lagrange方程-1.2} \\
	\bnb \cdot \mbf{E} = \frac{\rho}{\eps_0} \label{电磁场的Lagrange方程-1.3} \\
	\bnb \times \mbf{B} - \eps_0\mu_0\frac{\pl \mbf{E}}{\pl t} = \mu_0 \mbf{j} \label{电磁场的Lagrange方程-1.4}
\end{subnumcases}
虽然$\mbf{E}$和$\mbf{B}$总共有$6$个分量,但其中只有$4$个分量是相互独立的\footnote{这一点可以用Maxwell方程组只有$4$个方程来印证。},因此它们并不适合取为电磁场的广义坐标。此处我们取电磁场的标量势$\phi$和矢量势$\mbf{A}$作为电磁场的广义坐标。此处标量势$\phi$和矢量势$\mbf{A}$与电磁场场变量$\mbf{E}$和$\mbf{B}$之间的关系可以由式\eqref{电磁场的Lagrange方程-1.1}和式\eqref{电磁场的Lagrange方程-1.2}决定,即
\begin{align}
	\mbf{E} & = -\bnb \phi - \frac{\pl \mbf{A}}{\pl t} \label{电磁场的Lagrange方程-2} \\
	\mbf{B} & = \bnb \times \mbf{A} \label{电磁场的Lagrange方程-3}
\end{align}
因此,当用$\phi$和$\mbf{A}$描述电磁场时,式\eqref{电磁场的Lagrange方程-1.3}和式\eqref{电磁场的Lagrange方程-1.4}是场的运动方程,而式\eqref{电磁场的Lagrange方程-1.1}和式\eqref{电磁场的Lagrange方程-1.2}则只是$\phi$和$\mbf{A}$的定义而已。

下面来根据场的运动方程式\eqref{电磁场的Lagrange方程-1.3}和式\eqref{电磁场的Lagrange方程-1.4}来构造电磁场的Lagrange密度。首先,电磁场的Lagrange密度中需要包含场量$\phi$、$\mbf{A}$和他们的导数。根据电磁场的时间平移对称性和空间平移对称性,电磁场的Lagrange密度中不能显含时间$t$和空间坐标$\mbf{x}$。记$x_0 = t,~A_0 = \phi$,则可设电磁场的Lagrange密度为\footnote{此处利用了Einstein求和约定,用重复的指标表示求和,求和范围为$0$到$3$。}
\begin{equation}
	\mathscr{L} = B_iA_i+ C_{ij}\frac{\pl A_i}{\pl x_j} + D_{ijk}A_i\frac{\pl A_j}{\pl x_k} + E_{ijkl} \frac{\pl A_i}{\pl x_j}\frac{\pl A_k}{\pl x_l}
	\label{电磁场的Lagrange方程-4}
\end{equation}
将式\eqref{电磁场的Lagrange方程-4}代入Lagrange方程\eqref{连续系统的Lagrange方程-16}中,并结合量纲分析可以确定电磁场的Lagrange密度可以取如下的形式\footnote{与Lagrange函数类似,Lagrange密度的形式也是不唯一的,这里取了形式最简单的一种。}
\begin{align}
	\mathscr{L} & = \frac12 \left(\eps_0E^2-\frac{B^2}{\mu_0}\right) - \rho \phi + \mbf{j}\cdot \mbf{A} \nonumber \\
	& = \frac12 \left[\eps_0\left(\bnb\phi+\frac{\pl \mbf{A}}{\pl t}\right)^2-\frac{1}{\mu_0}\left(\bnb \times \mbf{A}\right)^2\right] - \rho \phi + \mbf{j}\cdot \mbf{A}
	\label{电磁场的Lagrange方程-5}
\end{align}

\subsection{Schr\"{o}dinger方程的建立}

非相对论量子力学有三种不同的理论形式,它们正好和经典力学Hamilton理论的三种不同形式的动力学方程相对应。1925年10月Heisenberg建立的通常被称为{\bf 矩阵力学}的量子力学,经Dirac研究可表示为量子Poisson括号表示的正则方程的形式;独立于Heisenberg,1926年3月Schr\"{o}dinger所建立的量子力学的波动方程则是直接从Hamilton-Jacobi方程过渡而来的;1948-1950年,Dirac和Fermi建立的路径积分形式的量子力学,则与Hamilton原理的形式类似。

下面我们遵循Schr\"{o}dinger的本意,从氢原子核外电子的Hamilton-Jacobi方程出发,导出定态Schr\"{o}dinger方程。

对于氢原子核外电子,其Hamilton函数为
\begin{equation}
	H = \frac{1}{2m}(p_x^2+p_y^2+p_z^2) - \frac{\alpha}{r}
	\label{Schrodinger方程的建立-1}
\end{equation}
其中$r = \sqrt{x^2+y^2+z^2},~ \alpha = \dfrac{e^2}{4\pi \eps_0}$。因此其Hamilton-Jacobi方程为
\begin{equation}
	\frac{1}{2m}\left[\left(\frac{\pl W}{\pl x}\right)^2+ \left(\frac{\pl W}{\pl y}\right)^2 + \left(\frac{\pl W}{\pl z}\right)^2\right] - \frac{\alpha}{r} = E
	\label{Schrodinger方程的建立-2}
\end{equation}
其中$W = W(x,y,z)$是经典Hamilton特征函数。作变换
\begin{equation}
	W = \hbar \ln \psi
	\label{Schrodinger方程的建立-3}
\end{equation}
式中$\hbar$是一个常数。由于$\ln \psi$不带量纲,因此$\hbar$与Hamilton特征函数$W(x,y,z)$具有相同的量纲,即作用量的量纲。由此,Hamilton-Jacobi方程\eqref{Schrodinger方程的建立-2}可以变换为
\begin{equation}
	\left(\frac{\pl \psi}{\pl x}\right)^2 + \left(\frac{\pl \psi}{\pl y}\right)^2 + \left(\frac{\pl \psi}{\pl z}\right)^2 - \frac{2m}{\hbar^2}\left(E+\frac{\alpha}{r}\right)\psi^2 = 0
	\label{Schrodinger方程的建立-4}
\end{equation}
式\eqref{Schrodinger方程的建立-4}是完全经典的,没有附加任何量子假设。

现在假设电子不是一个经典粒子,是具有波粒二象性的。把电子看作是像电磁波那样的“物质波”,取一个适当的Lagrange密度,用Hamilton原理得出电子的波动方程。根据Hamilton原理可得
\begin{equation}
	\delta S = \delta \int_{t_1}^{t_2} L \mathrm{d}t = \int_{t_1}^{t_2} \delta L \mathrm{d}t = 0
	\label{Schrodinger方程的建立-5}
\end{equation}
为了使式\eqref{Schrodinger方程的建立-5}成立,可令$\delta L = 0$,即
\begin{equation}
	\delta L = \delta \int_V \mathscr{L}\mathrm{d}V
	\label{Schrodinger方程的建立-6}
\end{equation}
对于波动,有一个普遍适用的结论:系统的平均动能恒等于平均势能,又考虑到$\mathscr{L} = \mathscr{T}-\mathscr{V}$,因此我们可以得到电子波的Lagrange函数应该是一个恒等于零的函数。而上面将电子考虑为一个经典粒子,它的Hamilton-Jacobi方程就是式\eqref{Schrodinger方程的建立-4},这就是一个恒等于零的表达式。因此要将波粒二象性结合起来,最自然的选择就是取式\eqref{Schrodinger方程的建立-4}作为电子波的Lagrange密度\footnote{准确的讲是令其与电子波的Lagrange密度成正比,因为式\eqref{Schrodinger方程的建立-4}与Lagrange密度的量纲是不同的。}。即有
\begin{equation}
	\delta J = \delta \int_V \left[\left(\frac{\pl \psi}{\pl x}\right)^2 + \left(\frac{\pl \psi}{\pl y}\right)^2 + \left(\frac{\pl \psi}{\pl z}\right)^2 - \frac{2m}{\hbar^2}\left(E+\frac{\alpha}{r}\right)\psi^2\right] \mathrm{d}V = 0
	\label{Schrodinger方程的建立-7}
\end{equation}
具体计算
\begin{align*}
	\delta J & = 2\int_V \left[\frac{\pl \psi}{\pl x} \delta \frac{\pl \psi}{\pl x} + \frac{\pl \psi}{\pl y} \delta \frac{\pl \psi}{\pl y} + \frac{\pl \psi}{\pl z} \delta \frac{\pl \psi}{\pl z} - \frac{2m}{\hbar^2}\left(E+\frac{\alpha}{r}\right) \psi\delta \psi\right] \mathrm{d}V \\
	& = 2\int_V \bigg[\frac{\pl}{\pl x}\left(\frac{\pl \psi}{\pl x}\delta \psi\right) - \frac{\pl^2 \psi}{\pl x^2} \delta \psi + \frac{\pl}{\pl y}\left(\frac{\pl \psi}{\pl y}\delta \psi\right) - \frac{\pl^2 \psi}{\pl y^2} \delta \psi + \frac{\pl}{\pl z}\left(\frac{\pl \psi}{\pl z}\delta \psi\right) - \frac{\pl^2 \psi}{\pl z^2} \delta \psi \\
	& \quad {} - \frac{2m}{\hbar^2}\left(E+\frac{\alpha}{r}\right)\psi \delta\psi \bigg] \mathrm{d}V \\
	& = 2\int_V \left[\frac{\pl}{\pl x}\left(\frac{\pl \psi}{\pl x}\delta \psi\right) + \frac{\pl}{\pl y}\left(\frac{\pl \psi}{\pl y}\delta \psi\right) + \frac{\pl}{\pl z}\left(\frac{\pl \psi}{\pl z}\delta \psi\right) \right] \mathrm{d}V \\
	& \quad {} - 2\int_V \left[\frac{\pl^2 \psi}{\pl x^2} + \frac{\pl^2 \psi}{\pl y^2} + \frac{\pl^2 \psi}{\pl z^2} + \frac{2m}{\hbar^2}\left(E+\frac{\alpha}{r}\right)\psi \right] \delta\psi \mathrm{d}V \\
	& = 2\oint_{\pl V} \frac{\pl \psi}{\pl n} \delta \psi \mathrm{d}S - 2\int_V \left[\frac{\pl^2 \psi}{\pl x^2} + \frac{\pl^2 \psi}{\pl y^2} + \frac{\pl^2 \psi}{\pl z^2} + \frac{2m}{\hbar^2}\left(E+\frac{\alpha}{r}\right)\psi \right] \delta\psi \mathrm{d}V = 0
\end{align*}
因此可得
\begin{equation}
	-\frac{\hbar^2}{2m}\left(\frac{\pl^2 \psi}{\pl x^2} + \frac{\pl^2 \psi}{\pl y^2} + \frac{\pl^2 \psi}{\pl z^2}\right) - \frac{\alpha}{r}\psi = E\psi
	\label{Schrodinger方程的建立-8}
\end{equation}
以及
\begin{equation}
	\oint_{\pl V} \frac{\pl \psi}{\pl n} \delta \psi \mathrm{d}S = 0
	\label{Schrodinger方程的建立-9}
\end{equation}
式\eqref{Schrodinger方程的建立-9}是一个面积分为零的表达式,由于$\delta \psi$是任意的,因此可以取
\begin{equation}
	\psi\big|_{\pl V} = 0,\quad \dfrac{\pl \psi}{\pl n}\bigg|_{\pl V} = 0
\end{equation}
作为偏微分方程\eqref{Schrodinger方程的建立-8}的边界条件。而偏微分方程\eqref{Schrodinger方程的建立-8}即为氢原子中电子的定态Schr\"{o}dinger方程。