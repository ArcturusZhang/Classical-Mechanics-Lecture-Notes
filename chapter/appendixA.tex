%\setcounter{chapter}{0}
%\renewcommand{\thechapter}{\Alph{chapter}}

% \appendix

% \titleformat{\chapter}[hang]{\hwyk\LARGE}
% {}{0mm}{\hspace{-0.4cm}\myappendix}
 
% \titleformat{\section}[hang]{\hwyk\LARGE}
% {}{0mm}{\hspace{-0.5cm}\appendixsection}

% \titleformat{\subsection}[hang]{\hwyk\large}
% {}{0mm}{\hspace{-0.5cm}\appendixsubsection}

% \titleformat{\subsubsection}[hang]{\hwyk\large}
% {}{0mm}{\hspace{-0.5cm}\appendixsubsubsection}

\chapter{Legendre椭圆积分与Jacobi椭圆函数}\label{椭圆积分与Jacobi椭圆函数}

\section{Legendre椭圆积分}

定义函数\footnote{本节提到的三类椭圆积分在不同的文献上可能会有不同的形式。}
\begin{equation}
	F(k,x) = \int_0^x \frac{\mathrm{d}t}{\sqrt{(1-t^2)(1-k^2t^2)}} = \int_0^{\arcsin x} \frac{\mathrm{d}\phi}{\sqrt{1-k^2\sin^2\phi}}
	\label{第一类椭圆积分-定义}
\end{equation}
为{\bf 第一类椭圆积分},其中参数$k$称为椭圆积分的{\bf 模}。当$x=1$时,记
\begin{equation}
	K(k) = \int_0^1 \frac{\mathrm{d}t}{\sqrt{(1-t^2)(1-k^2t^2)}} = \int_0^{\frac{\pi}{2}} \frac{\mathrm{d}\phi}{\sqrt{1-k^2\sin^2\phi}}
	\label{chp6:第一类完全椭圆积分-定义}
\end{equation}
称为{\bf 第一类完全椭圆积分}。

第一类椭圆积分可以用级数表示为
\begin{align}
	F(k,x) & = \int_0^{\arcsin x} \left[1+\sum_{n=1}^\infty \frac{(2n-1)!!}{(2n)!!}k^{2n}\sin^{2n}\phi\right]\mathrm{d}\phi \nonumber \\
	& = \arcsin x + \sum_{n=1}^\infty \frac{(2n-1)!!}{(2n)!!}k^{2n}\int_0^{\arcsin x} \sin^{2n}\phi \mathrm{d}\phi
	\label{chp6:第一类椭圆积分-级数表示}
\end{align}
由此可得第一类完全椭圆积分的级数表示为
\begin{align}
	K(k) & = \arcsin 1 + \sum_{n=1}^\infty \frac{(2n-1)!!}{(2n)!!}k^{2n}\int_0^{\frac{\pi}{2}} \sin^{2n}\phi \mathrm{d}\phi \nonumber \\
	& = \frac{\pi}{2}\left[1+\sum_{n=1}^\infty \left(\frac{(2n-1)!!}{(2n)!!}\right)^2k^{2n}\right]
	\label{chp6:第一类完全椭圆积分-级数表示}
\end{align}

定义函数
\begin{equation}
	E(k,x) = \int_0^x \sqrt{\frac{1-k^2t^2}{1-t^2}}\mathrm{d}t = \int_0^{\arcsin x} \sqrt{1-k^2\sin^2\phi}\mathrm{d}\phi
	\label{第二类椭圆积分-定义}
\end{equation}
为{\bf 第二类椭圆积分}。当$x=1$时,记
\begin{equation}
	E(k) = \int_0^1 \sqrt{\frac{1-k^2t^2}{1-t^2}}\mathrm{d}t = \int_0^{\frac{\pi}{2}} \sqrt{1-k^2\sin^2\phi}\mathrm{d}\phi
	\label{第二类完全椭圆积分-定义}
\end{equation}
称为{\bf 第二类完全椭圆积分}。

第二类椭圆积分可以用级数表示为
\begin{align}
	E(k,x) & = \int_0^{\arcsin x} \left[1-\sum_{n=1}^\infty \frac{1}{2n-1}\frac{(2n-1)!!}{(2n)!!}k^{2n}\sin^{2n}\phi\right]\mathrm{d}\phi \nonumber \\
	& = \arcsin x - \sum_{n=1}^\infty \frac{1}{2n-1}\frac{(2n-1)!!}{(2n)!!}k^{2n}\int_0^{\arcsin x} \sin^{2n}\phi \mathrm{d}\phi
	\label{第二类椭圆积分-级数表示}
\end{align}
由此可得第二类完全椭圆积分的级数表示为
\begin{align}
	E(k) & = \arcsin 1 - \sum_{n=1}^\infty \frac{1}{2n-1}\frac{(2n-1)!!}{(2n)!!}k^{2n}\int_0^{\frac{\pi}{2}} \sin^{2n}\phi \mathrm{d}\phi \nonumber \\
	& = \frac{\pi}{2}\left[1-\sum_{n=1}^\infty \frac{1}{2n-1}\left(\frac{(2n-1)!!}{(2n)!!}\right)^2k^{2n}\right]
	\label{chp6:第二类完全椭圆积分-级数表示}
\end{align}

根据第一类和第二类椭圆积分的定义式\eqref{第一类椭圆积分-定义}和\eqref{第二类椭圆积分-定义}可得
\begin{align}
	\dfrac{\pl F}{\pl k}(k,x) & = \frac{1}{k'^2}\left(\frac{E(k,x)-k'^2F(k,x)}{k}-\frac{kx\sqrt{1-x^2}}{1-k^2x^2}\right) \\
	\dfrac{\pl E}{\pl k}(k,x) & = \frac{E(k,x)-F(k,x)}{k}
\end{align}
其中$k'^2=1-k^2$,$k'$称为{\bf 补模}。

定义函数
\begin{align}
	\varPi(h,k,x) & = \int_0^x \frac{\mathrm{d}t}{(1+ht^2)\sqrt{(1-t^2)(1-k^2t^2)}} \nonumber \\
	& = \int_0^{\arcsin x} \frac{\mathrm{d}\phi}{(1+h\sin^2\phi)\sqrt{1-k^2\sin^2\phi}}
\end{align}
为{\bf 第三类椭圆积分}。当$x=1$时,记
\begin{equation}
	\varPi(h,k) = \int_0^1 \frac{\mathrm{d}t}{(1+ht^2)\sqrt{(1-t^2)(1-k^2t^2)}} = \int_0^{\frac{\pi}{2}} \frac{\mathrm{d}\phi}{(1+h\sin^2\phi)\sqrt{1-k^2\sin^2\phi}}
\end{equation}
称为{\bf 第三类完全椭圆积分}。

\section{Jacobi椭圆函数}

第一类椭圆积分\eqref{第一类椭圆积分-定义}
\begin{equation*}
	x = \int_0^s \frac{\mathrm{d}t}{\sqrt{(1-t^2)(1-k^2t^2)}}
\end{equation*}
的反函数记作
\begin{equation}
	s = \sn x
\end{equation}
称为{\bf 椭圆正弦函数}。

令
\begin{equation}
	\sin \phi = \sn x
\end{equation}
则称$\phi$为{\bf 椭圆振幅函数},记作$\phi = \am x$。

定义
\begin{equation}
	\cn x = \cos \phi = \sqrt{1-\sn^2 x}
\end{equation}
称为{\bf 椭圆余弦函数}。定义
\begin{equation}
	\tn x = \tan \phi = \frac{\sn x}{\cn x} = \frac{\sn x}{\sqrt{1-\sn^2 x}}
\end{equation}
称为{\bf 椭圆正切函数}。定义
\begin{equation}
	\dn x = \sqrt{1-k^2\sn^2x}
\end{equation}
称为{\bf 椭圆幅度函数}。

其中函数$\sn x, \cn x$和$\tn x$是周期为$4K(k)$的周期函数,而$\dn x$是周期为$2K(k)$的周期函数。它们的图像如图\ref{appendixA:椭圆函数的图像}所示。

\begin{figure}[htb]
\centering
\begin{asy}
	import gsl;
	size(350);
	picture pic;
	real k;
	real sn(real x){
		return sncndn(x,k^2)[0];
	}
	real cn(real x){
		return sncndn(x,k^2)[1];
	}
	real dn(real x){
		return sncndn(x,k^2)[2];
	}
	real x,y,K;
	k = 0.5;
	K = F(pi/2,k);
	x = 7.5;
	y = 1.25;
	draw(pic,Label("$x$",EndPoint,N),(0,0)--(x,0),Arrow);
	draw(pic,Label("$y$",EndPoint,W),(0,-y)--(0,y),Arrow);
	draw(pic,Label("$\mathrm{sn}\,x$",Relative(0.45),NE),graph(sn,0,4*K));
	draw(pic,Label("$\mathrm{cn}\,x$",Relative(0.375),SW),graph(cn,0,4*K));
	draw(pic,Label("$\mathrm{dn}\,x$",Relative(0.75),S),graph(dn,0,4*K),linewidth(0.8bp));
	label(pic,"$O$",(0,0),W);
	draw(pic,(0,-1)--(x-0.5,-1));
	draw(pic,(0,1)--(x-0.5,1));
	label(pic,"$-1$",(0,-1),W);
	label(pic,"$1$",(0,1),W);
	label(pic,"$K$",(K,0),SW);
	label(pic,"$2K$",(2*K,0),SW);
	label(pic,"$3K$",(3*K,0),SE);
	label(pic,"$4K$",(4*K,0),SE);
	add(xscale(0.5)*pic);
\end{asy}
\caption{椭圆函数的图像($k=\dfrac12$)}
\label{appendixA:椭圆函数的图像}
\end{figure}

当$k\to 0$时,根据式\eqref{第一类椭圆积分-定义}可得
\begin{equation*}
	F(0,s) = \int_0^s \frac{\mathrm{d}t}{\sqrt{1-t^2}} = \arcsin s
\end{equation*}
由此,当$k\to 0$时,有
\begin{equation}
	\sn x \to \sin x
\end{equation}
根据其余椭圆函数的定义,当$k\to 0$时,有
\begin{equation}
	\cn x \to \cos x,\quad \dn x \to 1
\end{equation}

Jacobi椭圆函数满足如下很容易验证的恒等式:
\begin{align}
	& \sn^2x+\cn^2x=1 \\
	& \dn^2x+k^2\sn^2x=1
\end{align}
以及如下的导数关系
\begin{align}
	\difOp{x}\sn x & = \cn x\cdot \dn x \\
	\difOp{x}\cn x & = -\sn x\cdot \dn x \\
	\difOp{x}\dn x & = -k^2\sn x\cdot \cn x
\end{align}
