\documentclass[hyperref,UTF8,a4paper,openany]{ctexbook}
%\usepackage{ctex}
\usepackage{zhnumber}
\usepackage{graphicx}
\usepackage{subfigure}

\usepackage{latexsym}
\usepackage{amsmath}
\usepackage{amssymb}
\usepackage{amsfonts}
\usepackage{mathrsfs}
\usepackage{amsthm}
\usepackage{siunitx}

\usepackage{cases}

\usepackage{multirow}
\usepackage{textcomp}
\usepackage[top=2.54cm, bottom=2.54cm, left=3.18cm, right=3.18cm]{geometry}
\usepackage[toc,lof]{multitoc}

\usepackage{asymptote}

%\usepackage{breqn}

\usepackage{hyperref}
\hypersetup{colorlinks,
        linkcolor=black,
        filecolor=black,
        urlcolor=blue,
        citecolor=black,
        pdftitle={《经典力学》讲稿},
        pdfauthor={张大鹏},
        pdfsubject={经典力学, 分析力学},
        pdfkeywords={经典力学, 分析力学},
        pdfproducer={XeLaTeX},
		pdfborder=0 0 1}

\setcounter{secnumdepth}{3} %使subsubsection也有编号

\usepackage{fancyhdr}	%页眉页脚

\usepackage{enumitem}
\setlist[enumerate,1]{leftmargin=0pt,itemindent=2em,itemsep=0ex,listparindent=2em,parsep = 0ex,topsep = 1ex}
\setlist[enumerate,2]{leftmargin=0pt,itemsep=0ex,listparindent=2em,parsep = 0ex,topsep = 1ex}
\setlist[enumerate,3]{leftmargin=0pt,itemsep=0ex,listparindent=2em,parsep = 0ex,topsep = 1ex}
\setlist[itemize,1]{itemsep = 0ex, parsep = 0ex, topsep = 1ex}
\setlist[description]{listparindent=2em, itemsep=0ex, parsep = 0ex,topsep = 1ex}

\ctexset{
	figurename={\kaishu 图},
	tablename={表},
	contentsname={目\quad 录},
	listfigurename={插图目录}
}

\newcommand{\ds}{\displaystyle}
\newcommand{\eps}{\varepsilon}
\newcommand{\mbf}{\boldsymbol}
\newcommand{\mrm}[1]{\boldsymbol{\mathnormal{#1}}}
\newcommand{\bnb}{\mbf{\nabla}}
\newcommand{\pl}{\partial}
\newcommand{\arcosh}{\mathrm{arcosh}\,}
\newcommand{\sn}{\mathrm{sn}\,}
\newcommand{\cn}{\mathrm{cn}\,}
\newcommand{\tn}{\mathrm{tn}\,}
\newcommand{\dn}{\mathrm{dn}\,}
\newcommand{\am}{\mathrm{am}\,}
\newcommand{\mathd}{\mathrm{d}}
\renewcommand{\bf}{\heiti}
\renewcommand{\bfseries}{\heiti}

\newtheoremstyle{theoremwithoutdot}% 类型名
  {}%                   Space above, empty = `usual value'
  {}%                   Space below
  {\kaishu}%                   Body font
  {}%         Indent amount (empty = no indent, \parindent = para indent)
  {\heiti}%          Thm head font
  {}%                   Punctuation after thm head
  {1em}%                Space after thm head
  {\thmname{#1}\thmnumber{~#2}\thmnote{~(#3)}}%                   Thm head spec
  
 \newtheoremstyle{solutionstyle}% 类型名
  {}%                   Space above, empty = `usual value'
  {}%                   Space below
  {}%                   Body font
  {}%         Indent amount (empty = no indent, \parindent = para indent)
  {\heiti}%          Thm head font
  {}%                   Punctuation after thm head
  {1em}%                Space after thm head
  {\thmname{#1}\thmnumber{~#2}\thmnote{~(#3)}}%                   Thm head spec

\theoremstyle{theoremwithoutdot}
\newtheorem{defi}{定义}[section]
\newtheorem{theorem}{定理}[section]
\newtheorem{lemma}[theorem]{引理}
\newtheorem{prop}[theorem]{命题}
\newtheorem{coro}{推论}[theorem]
\newtheorem{remark}{注}
\newtheorem{example}{例}[chapter]
\newtheorem{question}{题}[section]
\newtheorem{property}[theorem]{性质}
%\renewcommand{\thequestion}{\arabic{chapter}.\arabic{section}.\arabic{question}}
\theoremstyle{solutionstyle}
\newtheorem*{solution}{解}
%去掉证明后面的点
\makeatletter
\renewenvironment{proof}[1][\proofname]{\par%
\pushQED{\qed}%
\normalfont \topsep6\p@\@plus6\p@\relax%
\trivlist%
\item[\hskip\labelsep%
#1]\ignorespaces%
}{%
\popQED\endtrivlist\@endpefalse%
}
\makeatother
\renewcommand{\proofname}{{\heiti 证明}}

%\renewcommand{\thefootnote}{\fnsymbol{footnote}}
%带圈脚注
\usepackage{pifont}
\usepackage[perpage,stable]{footmisc}  %每页脚注重新编号
\renewcommand{\thefootnote}{\ding{\numexpr191+\value{footnote}}}
% 脚注中的脚注序号不用上标,正文中的脚注号保持不变
\makeatletter
\def\my@makefnmark{\hbox{\normalfont\@thefnmark\space}}
\let\my@save@makefntext\@makefntext
\long\def\@makefntext#1{{%
  \let\@makefnmark\my@makefnmark
  \my@save@makefntext{#1}}}

\allowdisplaybreaks[3]

\title{《经典力学》讲义}
\author{徐晓华\\ 整理:张大鹏\thanks{有修改和删节。所有插图均为重新绘制。}\\ Email:\url{13110290004@fudan.edu.cn}}
%\date{}

\makeatletter
\def\cleardoublepage{\clearpage\if@twoside \ifodd\c@page\else
\hbox{}
\vspace*{\fill}

\vspace{\fill}
\thispagestyle{empty}
\newpage
\if@twocolumn\hbox{}\newpage\fi\fi\fi}
\makeatother

%在插图目录图编号前加上“图”
%\usepackage{titletoc}
%\titlecontents{figure}[0.5cm]{\songti}{\figurename~\thecontentslabel\quad}{\hspace*{-1.5cm}}{\titlerule*[0.12cm]{.}\contentspage}[\addvspace{6pt}]

%旧的样式
%\usepackage{titlesec}
%\titleformat{\section}{\bf \xiaosan}{\thesection}{1em}{}

%新的样式
\usepackage{titlesec}
\usepackage{xcolor,colortbl}
\usepackage{tgpagella}
\usepackage[T1]{fontenc}
 
\definecolor{titlecolor}{RGB}{129,129,188}
\definecolor{contentcolor}{RGB}{129,188,129}
\definecolor{backcolor}{RGB}{129,188,129}
\newcommand{\hwyk}{\heiti}
 
\newcommand{\mytitle}[1]{
\begin{tabular}{p{0.01\textwidth}p{0.99\textwidth}}
\cellcolor{black} &\cellcolor{titlecolor} \textcolor{white}
{\newline\hwyk\LARGE 第\zhnumber{\thechapter}章 \  \ #1}
\end{tabular}
\arrayrulewidth=0.4pt
}
 
\newcommand{\mycont}[1]{
\vspace{-0.5cm}
\begin{tabular}{p{0.01\textwidth}p{0.99\textwidth}}
\cellcolor{black} &\cellcolor{contentcolor} \textcolor{white}
{\newline\hwyk\LARGE #1}
\end{tabular}
\arrayrulewidth=0.4pt
\vspace{-1.5cm}
}
 
\newcommand{\mysection}[1]{
\setlength\arrayrulewidth{1pt}\arrayrulecolor{titlecolor}
\begin{tabular}{p{0.01\textwidth}p{0.99\textwidth}}
\hline
\cellcolor{titlecolor} &  \textcolor{black}{\hwyk\LARGE \thesection ~ #1}
\end{tabular}
\arrayrulewidth=0.4pt
\vspace{-0.7cm}
}

\newcommand{\mysubsection}[1]{
\setlength\arrayrulewidth{1pt}\arrayrulecolor{titlecolor}
\begin{tabular}{p{0.01\textwidth}p{0.99\textwidth}}
\hline
\cellcolor{titlecolor} &  \textcolor{black}{\hwyk\Large \thesubsection ~ #1} \\
\hline
\end{tabular}
\arrayrulewidth=0.4pt
\vspace{-0.5cm}
}

\newcommand{\mysubsubsection}[1]{
\setlength\arrayrulewidth{1pt}\arrayrulecolor{titlecolor}
\begin{tabular}{p{0.01\textwidth}p{0.95\textwidth}p{0.01\textwidth}}
\hline
\cellcolor{titlecolor} &  \textcolor{black}{\hwyk\Large \thesubsubsection ~ #1} & \cellcolor{titlecolor} \\
\hline
\end{tabular}
\arrayrulewidth=0.4pt
\vspace{-0.5cm}
}


\newcommand{\myappendix}[1]{
\begin{tabular}{p{0.01\textwidth}p{0.99\textwidth}}
\cellcolor{black} &\cellcolor{backcolor} \textcolor{white}
{\newline\hwyk\LARGE 附录\thechapter \  \ #1}%\Alph{\thechapter}
\end{tabular}
\arrayrulewidth=0.4pt
}

 
\newcommand{\appendixsection}[1]{
\setlength\arrayrulewidth{1pt}\arrayrulecolor{backcolor}
\begin{tabular}{p{0.01\textwidth}p{0.99\textwidth}}
\hline
\cellcolor{backcolor} &  \textcolor{black}{\hwyk\LARGE \thesection ~ #1}
\end{tabular}
\arrayrulewidth=0.4pt
\vspace{-0.7cm}
}

\newcommand{\appendixsubsection}[1]{
\setlength\arrayrulewidth{1pt}\arrayrulecolor{backcolor}
\begin{tabular}{p{0.01\textwidth}p{0.99\textwidth}}
\hline
\cellcolor{backcolor} &  \textcolor{black}{\hwyk\Large \thesubsection ~ #1} \\
\hline
\end{tabular}
\arrayrulewidth=0.4pt
\vspace{-0.5cm}
}

\newcommand{\appendixsubsubsection}[1]{
\setlength\arrayrulewidth{1pt}\arrayrulecolor{backcolor}
\begin{tabular}{p{0.01\textwidth}p{0.95\textwidth}p{0.01\textwidth}}
\hline
\cellcolor{backcolor} &  \textcolor{black}{\hwyk\Large \thesubsubsection ~ #1} & \cellcolor{backcolor} \\
\hline
\end{tabular}
\arrayrulewidth=0.4pt
\vspace{-0.5cm}
}

\newcommand{\mybackmatter}[1]{
\begin{tabular}{p{0.01\textwidth}p{0.99\textwidth}}
\cellcolor{black} &\cellcolor{backcolor} \textcolor{white}
{\newline\hwyk\LARGE #1}
\end{tabular}
\arrayrulewidth=0.4pt
}

\begin{document}

\begin{asydef}
texpreamble("\usepackage[nosetpagesize]{color,graphicx}"); 
\end{asydef}

\maketitle

\frontmatter

\pagestyle{fancy}
\renewcommand{\chaptermark}[1]{\markboth{第\zhnumber{\thechapter}章\ \ #1}{}}
\renewcommand{\sectionmark}[1]{\markright{\thesection\ \ \rm #1}{}}	%这两个命令第一次必须出现在\pagestyle{fancy}之后,否则\pagestyle{fancy}会覆盖掉其效果
\markboth{\leftmark}{\rightmark}
\fancyhf{}
\fancyhead[CO]{\rightmark}
\fancyhead[LE,RO]{$\cdot$\, \thepage\, $\cdot$}
\fancyhead[CE]{\leftmark}
\renewcommand{\headrulewidth}{0.4pt}

\titleformat{\chapter}[hang]{\hwyk\LARGE}
{}{0mm}{\hspace{-0.4cm}\mycont}

\tableofcontents

\clearpage

\mainmatter
 
\titleformat{\chapter}[hang]{\hwyk\LARGE}
{}{0mm}{\hspace{-0.4cm}\mytitle}
 
\titleformat{\section}[hang]{\hwyk\LARGE}
{}{0mm}{\hspace{-0.5cm}\mysection}

\titleformat{\subsection}[hang]{\hwyk\large}
{}{0mm}{\hspace{-0.5cm}\mysubsection}

\titleformat{\subsubsection}[hang]{\hwyk\large}
{}{0mm}{\hspace{-0.5cm}\mysubsubsection}
 
\titlespacing{\chapter}
{0pc}{1.5ex plus .1ex minus .2ex}{.2pc}

%第零章——引言

\setcounter{chapter}{-1}
\chapter{引言}

\section{经典力学的发展}

17世纪由Galileo和Newton奠基。1687年发表的《自然哲学之数学原理》标志着经典力学体系的建立,即Newton力学。
\begin{itemize}
    \item {\heiti 核心概念}:质点、力、加速度;
    \item {\heiti 基本原理}:Newton三大定律;
    \item {\heiti 数学方法}:三维实空间几何、矢量代数。
\end{itemize}

18至19世纪,d'Alembert、Euler、Lagrange、Hamilton、Jacobi、Gauss、Poisson等建立力学的后Newton形式——分析力学。
\begin{itemize}
    \item {\heiti 核心概念}:能量、作用量;
    \item {\heiti 基本原理}:Hamilton原理(最小作用量原理);
    \item {\heiti 数学方法}:抽象空间、数学分析;
    \item {\heiti Lagrange形式}:广义坐标的引入使力学具备描述其它非力学体系的潜力。
    \item {\heiti Hamilton形式或正则形式}:力学体系内在对称性的揭示开始支配物理学思维,从而完成从力学到相对论和量子力学的升华。
\end{itemize}

分析力学给出了动量、角动量和能量的普适定义,使力学成为整个物理学的原型(量子力学、统计物理和量子场论的理论基石),为物理理论的统一奠定基础。

分析力学当代的发展结果:非线性物理与混沌、随机性、宿命论因果律破产。

\section{经典力学的适用范围}

{\heiti 宏观物体}在{\heiti 弱引力场}中的{\heiti 低速}运动。

\begin{itemize}
    \item {\heiti 高速运动}:狭义相对论
    \item {\heiti 强引力场}:广义相对论
    \item {\heiti 微观体系}:量子力学
\end{itemize}

% \section{课程的价值取向}

% 经典力学:通向现代物理宏伟圣殿的阶梯,非线性前沿科学成长的摇篮。

% 总结先贤理论遗产精华,领悟物理学的基本原理、通用语言和理论方法。%——醉翁之意不在酒:着眼于整个物理学。

% \begin{quote}
    % \kai ……科学知识使人们能制造许多产品,做许多事业。……科学的另一个价值是提供智能与思辩的享受。……如果我们社会进步的最终目标正是为了让各种人能享受它想做的事,那么科学家们思辩求知的享受就和其它事具有同等的重要性了。
%   \begin{flushright}
        % —— R.P. Feynman,《科学的价值》
    % \end{flushright}
% \end{quote}

%第一章——牛顿力学
\chapter{Newton力学}

{\heiti 运动学},描述运动现象;{\heiti 动力学},揭示运动规律。

\section{正交曲线坐标系}

定量描述质点的运动必须选择作为基准的参考系,并在其上建立坐标系。

\subsection{直角坐标系}

\begin{figure}[htb]
\centering
\begin{asy}
	size(200);
	//直角坐标系
	pair O,x,y,z,r;
	O = (0,0);
	x = 0.8*dir(-135);
	y = (1,0);
	z = (0,1);
	draw(Label("$x$",EndPoint),O--x,Arrow);
	draw(Label("$y$",EndPoint),O--y,Arrow);
	draw(Label("$z$",EndPoint),O--z,Arrow);
	label("$O$",O,SE);
	draw(Label("$\boldsymbol{e}_1$",EndPoint,NW),O--0.5*x,red,Arrow);
	draw(Label("$\boldsymbol{e}_2$",EndPoint,NE),O--0.5*y,red,Arrow);
	draw(Label("$\boldsymbol{e}_3$",EndPoint,W),O--0.5*z,red,Arrow);
	real xx,yy,zz;
	xx = 0.3;
	yy = 0.6;
	zz = 0.8;
	r = xx*x+yy*y+zz*z;
	draw(Label("$\boldsymbol{r}$",EndPoint),O--r,Arrow);
	draw(xx*x--xx*x+yy*y--yy*y,dashed);
	draw(O--xx*x+yy*y--xx*x+yy*y+zz*z--zz*z,dashed);
	//draw(O--z+(1.5,0),invisible);
\end{asy}
\caption{直角坐标系}
\label{直角坐标系}
\end{figure}
直角坐标系的基矢量是常矢量,即
\begin{equation}
	\mathrm{d} \mbf{e}_\alpha = \mbf{0},\quad \alpha = 1,2,3
\end{equation}
直角坐标系是正交坐标系,它的基矢量满足
\begin{equation}
	\mbf{e}_\alpha \cdot \mbf{e}_\beta = \delta_{\alpha\beta},\quad \alpha,\beta = 1,2,3
\end{equation}
直角坐标系是右手系,它的基矢量之间满足
\begin{equation}
	\mbf{e}_\alpha \times \mbf{e}_\beta = \sum_{\gamma=1}^3 \eps_{\alpha \beta \gamma} \mbf{e}_\gamma
\end{equation}
其中$\eps_{\alpha \beta \gamma}$为Levi-Civita反对称符号,它满足
\begin{equation}
	\eps_{123} = 1,\quad \eps_{\alpha\beta\gamma} = -\eps_{\beta\alpha\gamma} = -\eps_{\alpha\gamma\beta}
\end{equation}
直角坐标系中的位矢可以表示为
\begin{align}	
	\mbf{r} & = x \mbf{e}_1 + y \mbf{e}_2 + z \mbf{e}_3
\end{align}
位移元素为
\begin{align}
	\mathrm{d} \mbf{r} & = \mathrm{d} x \mbf{e}_1 + \mathrm{d} y \mbf{e}_2 + \mathrm{d} z \mbf{e}_3 \\
\end{align}
距离元素为
\begin{align}
	(\mathrm{d} s)^2 & = \mathrm{d} \mbf{r} \cdot \mathrm{d} \mbf{r} = (\mathrm{d} x)^2 + (\mathrm{d} y)^2 + (\mathrm{d} z)^2
\end{align}

\subsection{球坐标系}

\begin{figure}[htb]
\centering
\begin{asy}
	size(250);
	//球坐标系
	pair O,i,j,k;
	real x,y,z,r;
	path clp;
	picture tmp;
	O = (0,0);
	x = 2.5;
	y = 1.5;
	z = 1.5;
	r = 1;
	i = (-sqrt(2)/4,-sqrt(14)/12);
	j = (sqrt(14)/4,-sqrt(2)/12);
	k = (0,2*sqrt(2)/3);
	draw(Label("$x$",EndPoint),O--x*i,Arrow);
	draw(Label("$y$",EndPoint),O--y*j,Arrow);
	draw(Label("$z$",EndPoint),O--z*k,Arrow);
	clp = r*i--r*j--r*k--cycle;
	unfill(clp);
	draw(tmp,Label("$x$",EndPoint),O--x*i,dashed,Arrow);
	draw(tmp,Label("$y$",EndPoint),O--y*j,dashed,Arrow);
	draw(tmp,Label("$z$",EndPoint),O--z*k,dashed,Arrow);
	label(tmp,"$O$",O,NW);
	clip(tmp,clp);
	add(tmp);
	erase(tmp);
	real phi,theta;
	pair vcir(real t){
		return r*sin(t)*cos(phi)*i+r*sin(t)*sin(phi)*j+r*cos(t)*k;
	}
	pair hcir(real t){
		return r*sin(theta)*cos(t)*i+r*sin(theta)*sin(t)*j+r*cos(theta)*k;
	}
	pair sphcoor(real R,real theta,real phi){
		return R*sin(theta)*cos(phi)*i+R*sin(theta)*sin(phi)*j+R*cos(theta)*k;
	}
	phi = 0;
	draw(graph(vcir,-0.4,pi-0.4));
	draw(graph(vcir,pi-0.4,2*pi-0.4),dashed);
	phi = pi/2;
	draw(graph(vcir,-0.9,pi-1));
	draw(graph(vcir,pi-1,2*pi-0.9),dashed);
	theta = pi/2;
	draw(graph(hcir,-1,pi-1.2));
	draw(graph(hcir,pi-1.2,2*pi-1),dashed);
	theta = pi/4;
	draw(graph(hcir,-1.3,pi-1.1),orange);
	draw(graph(hcir,pi-1.1,2*pi-1.3),dashed+orange);
	phi = pi/3;
	draw(graph(vcir,-0.5,pi-0.6),orange);
	draw(graph(vcir,pi-0.6,2*pi-0.5),dashed+orange);
	draw(scale(r)*unitcircle,linewidth(1bp));
	pair P,Q,er,et,ep;
	P = sphcoor(r,theta,phi);
	Q = sphcoor(r,pi/2,phi);
	er = 0.6*sphcoor(r,theta,phi);
	et = 0.6*sphcoor(r,theta+pi/2,phi);
	ep = 0.6*sphcoor(r,pi/2,phi+pi/2);
	draw(Label("$r$",MidPoint,Relative(E)),O--P,dashed);
	draw(O--Q,dashed);
	draw(Label("$\boldsymbol{e}_r$",EndPoint,black),P--P+er,red,Arrow);
	draw(Label("$\boldsymbol{e}_\theta$",EndPoint,Relative(W),black),P--P+et,red,Arrow);
	draw(Label("$\boldsymbol{e}_\phi$",EndPoint,black),P--P+ep,red,Arrow);
	dot(P);
	r = 0.3;
	draw(Label("$\theta$",MidPoint,Relative(W)),graph(vcir,0,theta),Arrow);
	theta = pi/2;
	draw(Label("$\phi$",MidPoint,Relative(E)),graph(hcir,0,phi),Arrow);
\end{asy}
\caption{球坐标系}
\label{球坐标系}
\end{figure}

球坐标$(r,\theta,\phi)$与直角坐标$(x,y,z)$之间的转换关系为
\begin{equation}
	\begin{cases}
		x = r\sin \theta \cos \phi \\
		y = r\sin \theta \sin \phi \\
		z = r\cos \theta
	\end{cases}
\end{equation}
在此转换关系下,位矢可以表示为
\begin{equation}
	\mbf{r} = x\mbf{e}_1 + y\mbf{e}_2 + z\mbf{e}_3 = \mbf{r}(r,\theta,\phi)
\end{equation}
形如$r = C_1$、$\theta = C_2$或$\phi = C_3$的曲面称为{\heiti 坐标面},而坐标面两两相交形成{\heiti 坐标线},在坐标在线只有一个坐标可变。取坐标线切线方向为坐标轴,正方向为相应坐标增大的方向。因此球坐标系的基矢量可以表示为
\begin{equation}
	\mbf{e}_r = \frac{1}{h_r} \frac{\pl \mbf{r}}{\pl r},\quad \mbf{e}_\theta = \frac{1}{h_\theta} \frac{\pl \mbf{r}}{\pl \theta},\quad \mbf{e}_\phi = \frac{1}{h_\phi} \frac{\pl \mbf{r}}{\pl \phi}
\end{equation}
其中
\begin{equation}
	\begin{cases}
		h_r = \left|\dfrac{\pl \mbf{r}}{\pl r}\right| = 1 \\
		h_\theta = \left|\dfrac{\pl \mbf{r}}{\pl \theta}\right| = r \\
		h_\phi = \left|\dfrac{\pl \mbf{r}}{\pl \phi}\right| = r\sin \theta
	\end{cases}
\end{equation}
称为{\heiti 度规系数}。球坐标系的基矢量可以具体计算如下:
\begin{equation}
	\begin{cases}
		\mbf{e}_r = \sin \theta(\cos \phi \mbf{e}_1 + \sin \phi \mbf{e}_2) + \cos \theta \mbf{e}_3 \\
		\mbf{e}_\theta = \cos \theta(\cos \phi \mbf{e}_1 + \sin \phi \mbf{e}_2) - \sin \theta \mbf{e}_3 \\
		\mbf{e}_\phi = -\sin \phi \mbf{e}_1 + \cos \phi \mbf{e}_2
	\end{cases}
\end{equation}
球坐标系为正交系,也是右手系。其位移元素和距离元素分别为
\begin{align*}
	& \mathrm{d} \mbf{r} = \frac{\pl \mbf{r}}{\pl r} \mathrm{d} r + \frac{\pl \mbf{r}}{\pl \theta} \mathrm{d} \theta + \frac{\pl \mbf{r}}{\pl \phi} \mathrm{d} \phi = \mathrm{d} r \mbf{e}_r + r\mathrm{d} \theta \mbf{e}_\theta + r\sin \theta \mathrm{d} \phi \mbf{e}_\phi \\
	& (\mathrm{d} s)^2 = \mathrm{d} \mbf{r} \cdot \mathrm{d} \mbf{r} = (\mathrm{d} r)^2 + r^2 (\mathrm{d} \theta)^2 + r^2 \sin^2 \theta (\mathrm{d} \phi)^2
\end{align*}

\subsection{柱坐标系}

\begin{figure}[htb]
\centering
\begin{asy}
	size(250);
	//柱坐标系
	pair O,i,j,k;
	real x,y,z,r,a,h,H,pz;
	path cir,clp;
	picture tmp;
	O = (0,0);
	x = 2;
	y = 1.5;
	z = 3.3;
	i = (-sqrt(2)/4,-sqrt(14)/12);
	j = (sqrt(14)/4,-sqrt(2)/12);
	k = (0,2*sqrt(2)/3);
	draw(Label("$x$",EndPoint),O--x*i,Arrow);
	draw(Label("$y$",EndPoint),O--y*j,Arrow);
	draw(Label("$z$",EndPoint),O--z*k,Arrow);
	r = 1;
	h = 2.5;
	pair hcir(real t){
		return r*cos(t)*i+r*sin(t)*j+H*k;
	}
	clp = r*i--r*j--h*k--cycle;
	unfill(clp);
	draw(tmp,Label("$x$",EndPoint),O--x*i,dashed,Arrow);
	draw(tmp,Label("$y$",EndPoint),O--y*j,dashed,Arrow);
	draw(tmp,Label("$z$",EndPoint),O--z*k,dashed,Arrow);
	label(tmp,"$O$",O,W);
	clip(tmp,clp);
	add(tmp);
	erase(tmp);
	H = 0;
	draw(graph(hcir,-1.2,pi-1.2),linewidth(1bp));
	draw(graph(hcir,pi-1.2,2*pi-1.2),dashed);
	H = h;
	draw(graph(hcir,0,2*pi),linewidth(1bp));
	pz = 0.65*h;
	H = pz;
	draw(graph(hcir,-1.2,pi-1.2),orange);
	draw(graph(hcir,pi-1.2,2*pi-1.2),dashed+orange);
	draw(r*dir(180)--r*dir(180)+h*k,linewidth(1bp));
	draw(r*dir(0)--r*dir(0)+h*k,linewidth(1bp));
	pair P,er,et,ez;
	real theta;
	pair clycoor(real r,real theta,real z){
		return r*cos(theta)*i+r*sin(theta)*j+z*k;
	}
	theta = pi/3;
	P = clycoor(r,theta,pz);
	draw(O--P-pz*k,dashed+orange);
	draw(Label("$z$",MidPoint,Relative(W),black),P-pz*k--P,orange);
	draw(Label("$\rho$",MidPoint,Relative(W)),pz*k--P,dashed);
	er = clycoor(r,theta,0);
	et = clycoor(r,theta+pi/2,0);
	ez = clycoor(0,theta,1);
	draw(Label("$\boldsymbol{e}_\rho$",EndPoint),P--P+er,red,Arrow);
	draw(Label("$\boldsymbol{e}_\phi$",EndPoint),P--P+et,red,Arrow);
	draw(Label("$\boldsymbol{e}_z$",EndPoint),P--P+ez,red,Arrow);
	dot(P);
	r = 0.3;
	H = 0;
	draw(Label("$\phi$",MidPoint,Relative(E)),graph(hcir,0,theta),Arrow);
\end{asy}
\caption{柱坐标系}
\label{柱坐标系}
\end{figure}

柱坐标$(\rho,\phi,z)$与直角坐标$(x,y,z)$之间的转换关系为
\begin{equation}
	\begin{cases}
		x = \rho \cos \phi \\
		y = \rho \sin \phi \\
		z = z
	\end{cases}
\end{equation}
在此转换关系下,位矢可以表示为
\begin{equation}
	\mbf{r} = x\mbf{e}_1 + y\mbf{e}_2 + z\mbf{e}_3 = \mbf{r}(\rho,\phi,z)
\end{equation}
柱坐标系的坐标面为$\rho = C_1$、$\phi = C_2$和$z = C_3$。柱坐标系的基矢量可以表示为
\begin{equation}
	\mbf{e}_\rho = \frac{1}{h_\rho} \frac{\pl \mbf{r}}{\pl \rho},\quad \mbf{e}_\phi = \frac{1}{h_\phi} \frac{\pl \mbf{r}}{\pl \phi},\quad \mbf{e}_z = \frac{1}{h_z} \frac{\pl \mbf{r}}{\pl z}
\end{equation}
其中度规系数为
\begin{equation}
	\begin{cases}
		h_\rho = \left|\dfrac{\pl \mbf{r}}{\pl \rho}\right| = 1 \\
		h_\phi = \left|\dfrac{\pl \mbf{r}}{\pl \phi}\right| = \rho \\
		h_z = \left|\dfrac{\pl \mbf{r}}{\pl z}\right| = 1
	\end{cases}
\end{equation}
柱坐标系的基矢量可以具体计算如下:
\begin{equation}
	\begin{cases}
		\mbf{e}_\rho = \cos \phi \mbf{e}_1 + \sin \phi \mbf{e}_2 \\
		\mbf{e}_\phi = -\sin \phi \mbf{e}_1 + \cos \phi \mbf{e}_2 \\
		\mbf{e}_z = \mbf{e}_3
	\end{cases}
\end{equation}
柱坐标系为正交系,也是右手系。其位移元素和距离元素分别为
\begin{align}
	\mathrm{d} \mbf{r} & = \frac{\pl \mbf{r}}{\pl \rho} \mathrm{d} \rho + \frac{\pl \mbf{r}}{\pl \phi} \mathrm{d} \phi + \frac{\pl \mbf{r}}{\pl z} \mathrm{d} z = \mathrm{d} \rho \mbf{e}_\rho + \rho\mathrm{d} \phi \mbf{e}_\phi + \mathrm{d} z \mbf{e}_z \\
	(\mathrm{d} s)^2 & = \mathrm{d} \mbf{r} \cdot \mathrm{d} \mbf{r} = (\mathrm{d} \rho)^2 + \rho^2 (\mathrm{d} \phi)^2 + (\mathrm{d} z)^2
\end{align}

\subsection{一般正交曲线坐标系}

设曲线坐标系的坐标为$(q_1,q_2,q_3)$,其与直角坐标$(x,y,z)$之间的转换关系为
\begin{equation}
\begin{cases}
	x = x(q_1,q_2,q_3) \\
	y = y(q_1,q_2,q_3) \\
	z = z(q_1,q_2,q_3)
\end{cases}
\end{equation}
在此转换关系下,位矢可以表示为
\begin{equation}
	\mbf{r} = x\mbf{e}_1 + y \mbf{e}_2 + z \mbf{e}_3 = \mbf{r}(q_1,q_2,q_3)
\end{equation}
此曲线坐标系的单位基矢量可以表示为
\begin{equation}
	\mbf{e}_\alpha = \dfrac{1}{h_\alpha} \dfrac{\pl \mbf{r}}{\pl q_\alpha}
\end{equation}
其中度规系数表示为
\begin{equation}
	h_\alpha = \left|\dfrac{\pl \mbf{r}}{\pl q_\alpha}\right|
\end{equation}
而曲线坐标系的正交性要求
\begin{equation}
	\mbf{e}_\alpha \cdot \mbf{e}_\beta = \delta_{\alpha\beta}
\end{equation}
由此,位移元素和距离元素分别为
\begin{align}
	\mathrm{d}\mbf{r} & = h_1\mbf{e}_1\mathrm{d}q_1 + h_2\mbf{e}_2\mathrm{d}q_2 + h_3\mbf{e}_3\mathrm{d}q_3 \\
	\mathrm{d}s^2 & = h_1^2(\mathrm{d}q_1)^2 + h_2^2(\mathrm{d}q_2)^2 + h_3^2(\mathrm{d}q_3)^2
\end{align}

\subsection{正交曲线坐标系中速度与加速度的表示}

速度定义为$\mbf{v} = \dot{\mbf{r}}(t) = \dfrac{\mathrm{d} \mbf{r}}{\mathrm{d} t}(t)$,因此在曲线坐标系中有
\begin{equation}
	\mbf{v} = \frac{\mathrm{d} \mbf{r}}{\mathrm{d} t}(t) = \sum_{\alpha=1}^3 \frac{\pl \mbf{r}}{\pl q_\alpha} \dot{q}_\alpha = \sum_{\alpha=1}^3 h_\alpha \dot{q}_\alpha \mbf{e}_\alpha
\end{equation}
记$\displaystyle \mbf{v} = \sum_{\alpha=1}^3 h_\alpha \dot{q}_\alpha \mbf{e}_\alpha =: \mbf{v}(\mbf{q},\dot{\mbf{q}})$,以及
\begin{equation}
	T = \frac12 v^2 = \frac12 \sum_{\alpha=1}^3 (h_\alpha \dot{q}_\alpha)^2 =: T (\mbf{q},\dot{\mbf{q}})
\end{equation}
加速度$\mbf{a} = \dot{\mbf{v}}(t) = \ddot{\mbf{r}}(t)$,在此曲线坐标系下可以分解为
\begin{equation*}
	\mbf{a} = \sum_{\alpha=1}^3 a_\alpha \mbf{e}_\alpha
\end{equation*}
加速度的分量为
\begin{equation}
	a_\alpha = \frac{1}{h_\alpha} \left[\frac{\mathrm{d}}{\mathrm{d} t} \frac{\pl}{\pl \dot{q}_\alpha} \left(\frac{v^2}{2}\right) - \frac{\pl}{\pl q_\alpha} \left(\frac{v^2}{2}\right)\right]
	\label{加速度的Lagrange公式}
\end{equation}
式\eqref{加速度的Lagrange公式}称为{\heiti Lagrange公式}。为了证明式\eqref{加速度的Lagrange公式},首先考虑
\begin{equation}
	\mbf{v}(\mbf{q},\dot{\mbf{q}}) = \sum_{\alpha=1}^3 h_\alpha \dot{q}_\alpha \mbf{e}_\alpha = \sum_{\alpha=1}^3 \frac{\pl \mbf{r}}{\pl q_\alpha} \dot{q}_\alpha
\end{equation}
因此有
\begin{align}
	\frac{\pl \mbf{v}}{\pl \dot{q}_\alpha} & = \sum_{\beta=1}^3 \frac{\pl \mbf{r}}{\pl q_\beta} \frac{\pl \dot{q}_\beta}{\pl \dot{q}_\alpha} = \frac{\pl \mbf{r}}{\pl q_\alpha} \label{Lagrange公式中间步骤1} \\
	\frac{\mathrm{d}}{\mathrm{d} t} \frac{\pl \mbf{r}}{\pl q_\alpha} & = \sum_{\beta=1}^3 \frac{\pl}{\pl q_\beta} \left(\frac{\pl \mbf{r}}{\pl q_\alpha}\right) \dot{q}_\beta = \sum_{\beta=1}^3 \frac{\pl}{\pl q_\alpha} \left(\frac{\pl \mbf{r}}{\pl q_\beta}\right) \dot{q}_\beta = \frac{\pl}{\pl q_\alpha} \sum_{\beta=1}^3 \frac{\pl \mbf{r}}{\pl q_\beta} \dot{q}_\beta = \frac{\pl \mbf{v}}{\pl q_\alpha}
	\label{Lagrange公式中间步骤2}
\end{align}
根据式\eqref{Lagrange公式中间步骤1}和\eqref{Lagrange公式中间步骤2},即有
\begin{align*}
	h_\alpha a_\alpha & = \ddot{\mbf{r}} \cdot \frac{\pl \mbf{r}}{\pl q_\alpha} = \frac{\mathrm{d}}{\mathrm{d} t} \left(\dot{\mbf{r}} \cdot \frac{\pl \mbf{r}}{\pl q_\alpha}\right) - \dot{\mbf{r}} \cdot \frac{\mathrm{d}}{\mathrm{d} t} \frac{\pl \mbf{r}}{\pl q_\alpha} = \frac{\mathrm{d}}{\mathrm{d} t} \left(\dot{\mbf{r}} \cdot \frac{\pl \mbf{v}}{\pl \dot{q}_\alpha}\right) - \dot{\mbf{r}} \cdot \frac{\pl \mbf{v}}{\pl q_\alpha} \\
	& = \frac{\mathrm{d}}{\mathrm{d} t} \frac{\pl}{\pl \dot{q}_\alpha} \left(\frac12 \mbf{v} \cdot \mbf{v}\right) - \frac{\pl}{\pl q_\alpha} \left(\frac12 \mbf{v} \cdot \mbf{v}\right) = \frac{\mathrm{d}}{\mathrm{d} t} \frac{\pl}{\pl \dot{q}_\alpha} \left(\frac{v^2}{2}\right) - \frac{\pl}{\pl q_\alpha} \left(\frac{v^2}{2}\right)
\end{align*}

\begin{example}[球坐标系中的速度、加速度表示]
在球坐标中,$h_r = 1,h_\theta = r,h_\phi =r\sin \theta$,因此
\begin{equation}
	\mbf{v} = \sum_{\alpha=1}^3 h_\alpha \dot{q}_\alpha \mbf{e}_\alpha = \dot{r} \mbf{e}_r + r\dot{\theta} \mbf{e}_\theta + r\dot{\phi}\sin \theta \mbf{e}_\phi
	\label{第一章:球坐标系中的速度}
\end{equation}
所以
\begin{equation*}
	v^2 = \mbf{v} \cdot \mbf{v} = \dot{r}^2 + r^2 \dot{\theta}^2 + r^2 \dot{\phi}^2 \sin^2 \theta
\end{equation*}
故加速度的各个分量为
\begin{subnumcases}{\label{第一章:球坐标系中的加速度}}
	a_r = \frac{1}{h_r} \left[\frac{\mathrm{d}}{\mathrm{d} t} \frac{\pl}{\pl \dot{r}}\left(\frac{v^2}{2}\right) - \frac{\pl}{\pl r}\left(\frac{v^2}{2}\right)\right] = \ddot{r}-r\dot{\theta}^2-r\dot{\phi}^2 \sin^2 \theta \\
	a_\theta = \frac{1}{h_\theta} \left[\frac{\mathrm{d}}{\mathrm{d} t} \frac{\pl}{\pl \dot{\theta}}\left(\frac{v^2}{2}\right) - \frac{\pl}{\pl \theta}\left(\frac{v^2}{2}\right)\right] = r\ddot{\theta}+2\dot{r}\dot{\theta}-r\dot{\phi}^2 \sin \theta \cos \theta \\
	a_\phi = \frac{1}{h_\phi} \left[\frac{\mathrm{d}}{\mathrm{d} t} \frac{\pl}{\pl \dot{\phi}}\left(\frac{v^2}{2}\right) - \frac{\pl}{\pl \phi}\left(\frac{v^2}{2}\right)\right] = r\ddot{\phi} \sin \theta + 2\dot{r}\dot{\phi}\sin \theta + 2r\dot{\theta}\dot{\phi}\cos \theta
\end{subnumcases}
\end{example}

\begin{example}[柱坐标系中的速度、加速度表示]
柱坐标中,$h_\rho = 1,h_\phi = r,h_z = 1$,因此
\begin{equation}
	\mbf{v} = \sum_{\alpha=1}^3 h_\alpha \dot{q}_\alpha \mbf{e}_\alpha = \dot{\rho} \mbf{e}_\rho + \rho\dot{\phi} \mbf{e}_\phi + \dot{z} \mbf{e}_z
	\label{第一章:柱坐标系中的速度}
\end{equation}
所以
\begin{equation*}
	v^2 = \mbf{v} \cdot \mbf{v} = \dot{\rho}^2 + \rho^2 \dot{\phi}^2 + \dot{z}^2
\end{equation*}
故加速度的各个分量为
\begin{subnumcases}{\label{第一章:柱坐标系中的加速度}}
	a_\rho = \frac{1}{h_\rho} \left[\frac{\mathrm{d}}{\mathrm{d} t} \frac{\pl}{\pl \dot{\rho}}\left(\frac{v^2}{2}\right) - \frac{\pl}{\pl \rho}\left(\frac{v^2}{2}\right)\right] = \ddot{\rho}-\rho\dot{\phi}^2 \\
	a_\phi = \frac{1}{h_\phi} \left[\frac{\mathrm{d}}{\mathrm{d} t} \frac{\pl}{\pl \dot{\phi}}\left(\frac{v^2}{2}\right) - \frac{\pl}{\pl \phi}\left(\frac{v^2}{2}\right)\right] = \rho\ddot{\phi}+2\dot{\rho}\dot{\phi} \\
	a_z = \frac{1}{h_z} \left[\frac{\mathrm{d}}{\mathrm{d} t} \frac{\pl}{\pl \dot{z}}\left(\frac{v^2}{2}\right) - \frac{\pl}{\pl z}\left(\frac{v^2}{2}\right)\right] = \ddot{z}
\end{subnumcases}
\end{example}

\subsection{自然坐标系}

动点的轨道记作$\mbf{r} = \mbf{r} (s)$,此处$s$为起点到当前位置的弧长。速度为
\begin{equation}
	\mbf{v} = \frac{\mathrm{d} \mbf{r}}{\mathrm{d} t} = \frac{\mathrm{d} s}{\mathrm{d} t} \frac{\mathrm{d} \mbf{r}}{\mathrm{d} s} = \dot{s} \mbf{\tau}
\end{equation}
式中$\mbf{\tau} = \dfrac{\mathrm{d} \mbf{r}}{\mathrm{d} s}$为切矢量。弧长可以计算为
\begin{equation}
	s(t) = \int_0^t \left|\frac{\mathrm{d} \mbf{r}}{\mathrm{d} t}(\tau) \right| \mathrm{d}\tau
\end{equation}
因此$\dfrac{\mathrm{d} s}{\mathrm{d} t} = \left|\dfrac{\mathrm{d} \mbf{r}}{\mathrm{d} t}\right|$。计算切矢量的长度
\begin{equation*}
	|\mbf{\tau}| = \left|\frac{\mathrm{d} \mbf{r}}{\mathrm{d} s}\right| = \left|\dfrac{\mathrm{d} \mbf{r}}{\mathrm{d} t} \dfrac{\mathrm{d} t}{\mathrm{d} s}\right| = \frac{1}{\left|\dfrac{\mathrm{d} \mbf{r}}{\mathrm{d} t}\right|} \left|\dfrac{\mathrm{d} \mbf{r}}{\mathrm{d} t}\right| = 1
\end{equation*}
因此,$\mbf{\tau}(s)$为单位切矢量。考虑到
\begin{equation*}
	\frac{\mathrm{d} (\mbf{\tau} \cdot \mbf{\tau})}{\mathrm{d} s} = 2 \frac{\mathrm{d} \mbf{\tau}}{\mathrm{d} s} \cdot \mbf{\tau} = 0
\end{equation*}
即有$\dfrac{\mathrm{d} \mbf{\tau}}{\mathrm{d} s} \perp \mbf{\tau}$。定义$\mbf{n} = \dfrac{\mbf{\tau}'(s)}{|\mbf{\tau}'(s)|} = \dfrac{\mbf{r}''(s)}{|\mbf{r}''(s)|}$为主法向单位矢量,其中$\kappa(s) = |\mbf{\tau}'(s)| = |\mbf{r}''(s)|$为曲率,其倒数$R = \dfrac{1}{\kappa(s)}$称为曲率半径,主法向单位矢量也可记作
\begin{equation}
	\mbf{n} = R\frac{\mathrm{d} \mbf{\tau}}{\mathrm{d} s}
\end{equation}
定义$\mbf{b} = \mbf{\tau} \times \mbf{n}$为副法向单位矢量。

\begin{figure}[htb]
\centering
\begin{minipage}[t]{0.45\textwidth}
\centering
\begin{asy}
	size(180);
	//动点的轨迹
	pair O,A[],Os,P,P1,tP,tP1;
	O = (0,0);
	A[1] = (-2.5,1.5);
	A[2] = (-2,2);
	A[3] = (-1,2.5);
	A[4] = (1,2.5);
	A[5] = (2,1);
	A[6] = (1,1);
	A[7] = (1,2);
	A[8] = (3,3);
	path tra;
	tra = A[1]..A[2]..A[3]..A[4]..A[5]..A[6]..A[7]..A[8];
	draw(tra,Arrow(Relative(0.8)));
	Os = relpoint(tra,0.05);
	P = relpoint(tra,0.15);
	tP = 0.8*reldir(tra,0.15);
	P1 = relpoint(tra,0.30);
	tP1 = 0.8*reldir(tra,0.30);
	dot(O);
	label("$O$",O,S);
	dot(Os);
	label("$O_s$",Os,NW);
	dot(P);
	label("$P$",P,N);
	dot(P1);
	label("$P_1$",P1,N);
	draw(Label("$\boldsymbol{r}$",MidPoint,Relative(W)),O--P,Arrow);
	draw(Label("$\boldsymbol{r}_1$",MidPoint,Relative(E)),O--P1,Arrow);
	draw(Label("$\Delta \boldsymbol{r}$",MidPoint,Relative(E)),P--P1,Arrow);
	label("$s$",relpoint(tra,0.1),S);
	draw(Label("$\boldsymbol{\tau}$",EndPoint),P--P+tP,red,Arrow);
	draw(Label("$\boldsymbol{\tau}_1$",EndPoint,Relative(W)),P1--P1+tP1,red,Arrow);
\end{asy}
\caption{动点的轨迹}
\label{动点的轨迹}
\end{minipage}
\hspace{1cm}
\begin{minipage}[t]{0.45\textwidth}
\centering
\begin{asy}
	size(180);
	//切线法线副法线
	real hd,hdd;
	pair O,x,y,z;
	hd = 131+25/60;
	hdd = 7+10/60;
	O = (0,0);
	x = 0.8*dir(90+hd);
	y = dir(-hdd);
	z = dir(90);
	draw(Label("切线",EndPoint),O--x,linewidth(0.8bp));
	draw(Label("主法线",EndPoint),O--y,linewidth(0.8bp));
	draw(Label("副法线",EndPoint),O--z,linewidth(0.8bp));
	draw(x--x+y--y--y+z--z--z+x--cycle);
	draw(Label("$\boldsymbol{\tau}$",EndPoint,NW),O--0.4*x,red,Arrow);
	draw(Label("$\boldsymbol{n}$",EndPoint,N),O--0.4*y,red,Arrow);
	draw(Label("$\boldsymbol{b}$",EndPoint,E),O--0.4*z,red,Arrow);
	
	pair para(real xx){
		real yy;
		yy = xx**2;
		return xx*x+yy*y;
	}
	draw(graph(para,-1,1));
	label("$\begin{array}{c} \mbox{从} \\[-0.5ex] \mbox{切} \\[-0.5ex] \mbox{面} \end{array}$",0.5*x+0.5*z);
	label("法平面",0.5*y+0.5*z);
	label("密切面",0.4*x+0.6*y);
	//draw((0,0)--(1.5,0),invisible);
\end{asy}
\caption{切线、主法线和副法线}
\label{切线、主法线和副法线}
\end{minipage}
\end{figure}

有了速度,就可以计算加速度
\begin{align}
	\mbf{a} & = \frac{\mathrm{d} \mbf{v}}{\mathrm{d} t}(t) = \dot{v} \mbf{\tau} + v \dot{\mbf{\tau}} = \dot{v} \mbf{\tau} + v\dot{s} \frac{\mathrm{d} \mbf{\tau}}{\mathrm{d} s} = \dot{v} \mbf{\tau} + \frac{v^2}{R} \mbf{n}
\end{align}
其中切向加速度
\begin{equation}
	a_\tau = \dot{v}
\end{equation}
法向加速度
\begin{equation}
	a_n = \frac{v^2}{R}
\end{equation}

\begin{example}
已知$y = \dfrac{x^2}{2p}$,切向加速度与法向加速度之比为$\dfrac{a_\tau}{a_n} = \alpha$,经过$A$点的速度为$v_A = u$,求动点经过$B$点的速度$v_B$。
\begin{figure}[htb]
\centering
\begin{asy}
	size(200);
	//第一章例3图
	pair O,x,y,A,B;
	O = (0,0);
	x = (1.5,0);
	y = (0,2.5);
	draw(Label("$x$",EndPoint),(-x)--x,Arrow);
	draw(Label("$y$",EndPoint),(-0.2*y)--y,Arrow);
	
	real para(real xx){
		return 2*xx**2;
	}
	draw(graph(para,-1,1));
	//draw((0,0)--(2,0),invisible);
	A = (-0.5,para(-0.5));
	B = (0.5,para(0.5));
	dot(A);
	label("$A\left(-p,\dfrac{p}{2}\right)$",A,W);
	dot(B);
	label("$B\left(p,\dfrac{p}{2}\right)$",B,E);
	draw(A--B,dashed);
\end{asy}
\caption{例\theexample}
\label{第一章例3图}
\end{figure}
\end{example}

\begin{solution}
切向加速度
\begin{equation*}
	a_\tau = \frac{\mathrm{d} v}{\mathrm{d} t} = \frac{\mathrm{d} v}{\mathrm{d} s} \frac{\mathrm{d} s}{\mathrm{d} t} = v\frac{\mathrm{d} v}{\mathrm{d} s}
\end{equation*}
法向加速度
\begin{equation*}
	a_n = \frac{v^2}{R}
\end{equation*}
故有
\begin{equation*}
	\frac{a_\tau}{a_n} = \frac{R}{v} \frac{\mathrm{d} v}{\mathrm{d} s} = \frac{R}{v} \frac{\mathrm{d} v}{\mathrm{d} x} \frac{\mathrm{d} x}{\mathrm{d} s} = \alpha
\end{equation*}
其中
\begin{align*}
	& R = \frac{(1+y'^2)^{\frac32}}{y''} \\
	& \frac{\mathrm{d} s}{\mathrm{d} x} = \sqrt{1+y'^2}
\end{align*}
故有微分方程
\begin{equation*}
	\frac{\mathrm{d} v}{v} = \frac{\alpha y'' \mathrm{d}x}{1+y'^2} = \frac{\alpha p}{p^2+x^2} \mathrm{d}x
\end{equation*}
对其进行积分,即有
\begin{equation*}
	\int_u^{v_B} \frac{\mathrm{d} v}{v} = \int_{-p}^p \frac{\alpha p}{p^2+x^2} \mathrm{d} x
\end{equation*}
可得
\begin{equation*}
	\ln v \bigg|_u^{v_B} = \alpha \arctan \frac{x}{p} \bigg|_{-p}^p
\end{equation*}
所以
\begin{equation*}
	v_B = u \mathrm{e}^{\frac{\alpha\pi}{2}}
\end{equation*}
\end{solution}

\section{Newton运动定律}

质点动力学的基本原理,实验事实通过理性思维后的抽象概括。

\subsection{质点}

{\heiti 质点}是一个理想模型,忽略形状和大小,保留力学性质的物体。

适用条件:平动物体或物体体积远小于运动范围。

\subsection{Newton运动定律}

\begin{itemize}
	\item {\heiti Newton第一定律}:不受外力作用的质点始终保持静止或匀速直线运动状态。
	
	建立了{\heiti 力}和{\heiti 惯性}的概念,指出静止与匀速直线运动等价。物理经验的极限外推:物体越孤立,越接近惯性运动。
	
	第一定律的实质:定义惯性系(第一定律成立的参考系),断言其存在。等价于孤立系的动量守恒定律。
	\item {\heiti Newton第二定律}:$\mbf{F} = m\mbf{a}$。
	通过加速度规定惯性和力的度量。Newton第二定律是运动定律,而非单纯定义。Newton第二定律只适用于惯性系,非惯性系必须引入虚拟的惯性力。
	
	\item {\heiti Newton第三定律}:作用力与反作用力共线,大小相等,方向相反,即$\mbf{F}_{ij} = -\mbf{F}_{ji}$。
	
	适用条件:物体运动速度远低于相互作用传播速度。
\end{itemize}

\subsection{基本相互作用}

\begin{itemize}
	\item {\heiti 强力、弱力}:只在围观世界起作用。
	\item {\heiti 引力、电磁力}:经典力学研究的作用力。
\end{itemize}

摩擦力、弹性力等是电磁力的宏观表现。

\subsection{伽利略变换}

\begin{figure}[htb]
\centering
\begin{asy}
	size(300);
	//伽利略变换
	pair O,x,y,z,v,r;
	O = (0,0);
	x = 0.8*dir(-135);
	y = dir(0);
	z = dir(90);
	v = (2/3,1/3);
	r = (0.9,1.2);
	draw(Label("$x$",EndPoint),O--x,Arrow);
	draw(Label("$y$",EndPoint),O--y,Arrow);
	draw(Label("$z$",EndPoint),O--z,Arrow);
	label("$O$",O,W);
	draw(Label("$x'$",EndPoint),v--v+x,Arrow);
	draw(Label("$y'$",EndPoint),v--v+y,Arrow);
	draw(Label("$z'$",EndPoint),v--v+z,Arrow);
	label("$O'$",v,SE);
	draw(Label(rotate(degrees(v))*"$\boldsymbol{v}_0 t$",Relative(0.7),Relative(W)),O--v,red,Arrow);
	draw(Label("$\boldsymbol{r}$",MidPoint,Relative(W)),O--r,Arrow);
	draw(Label("$\boldsymbol{r}'$",MidPoint,Relative(E)),v--r,Arrow);
	
	//draw((0,0)--(2,0),invisible);
\end{asy}
\caption{伽利略变换}
\label{伽利略变换}
\end{figure}

伽利略变换可以表示为
\begin{equation}
	\begin{cases}
		\mbf{r}' = \mbf{r} - \mbf{v}_0t \\
		t' = t
	\end{cases}
\end{equation}
在此变换下,可有
\begin{equation}
	\mbf{v}' = \mbf{v} - \mbf{v}_0,\quad \mbf{a}' = \mbf{a}
\end{equation}
力决定于质点间的相对位矢,故有$\mbf{F} = \mbf{F}'$。相对速度是伽利略变换下的不变量。

如果$Oxyz$参考系为惯性参考系,即满足$\mbf{F} = m\mbf{a}$,则$O'x'y'z'$也满足$\mbf{F}' = m\mbf{a}'$,因此也是惯性参考系。

上述说明中的隐含假定:长度、时间和质量是伽利略变换下的不变量。

{\heiti 伽利略相对性原理} \quad 力学定理在所有惯性系中形式相同。一切惯性系等价,无法通过力学实验区分哪个惯性系更基本。

相对性原理推广到一切物理现象将导致狭义相对论的诞生。

\section{运动定理与守恒定律}

\subsection{质点系}

质点系即指两个以上相互作用的质点组成的系统。在质点系中,{\heiti 外力}指质点系外物体对质点系内任一质点的作用力,记作$\mbf{F}_i$。{\heiti 内力}指质点系内任意一对质点之间的相互作用力,质点$j$对质点$i$的作用力记作$\mbf{f}_{ij}$。

\subsection{动量及其守恒定律}

质点的{\heiti 动量}定义为
\begin{equation}
	\mbf{p} = m\mbf{v}
\end{equation}
根据Newton第二定律,有
\begin{equation}
	\mbf{F} = m\mbf{a} = m\frac{\mathrm{d} \mbf{v}}{\mathrm{d} t} = \frac{\mathrm{d} \mbf{p}}{\mathrm{d} t}
	\label{动量定理1}
\end{equation}
或者
\begin{equation}
	\mathrm{d} \mbf{p} = \mbf{F} \mathrm{d} t
	\label{动量定理2}
\end{equation}
式\eqref{动量定理1}或\eqref{动量定理2}表示的关系称为{\heiti 动量定理}。

质点系的动量定义为各个质点动量之和,即
\begin{equation}
	\mbf{p} = \sum_{i=1}^n \mbf{p}_i
\end{equation}
则有
\begin{equation*}
	\frac{\mathrm{d} \mbf{p}}{\mathrm{d} t} = \sum_{i=1}^n \frac{\mathrm{d} \mbf{p}_i}{\mathrm{d} t} = \sum_{i=1}^n \left(\mbf{F}_i + \sum_{j=1(j\neq i)}^n \mbf{f}_{ij}\right)
\end{equation*}
式中
\begin{equation*}
	\sum_{j=1(j\neq i)}^n \mbf{f}_{ij} = \sum_{i=1(i \neq j)}^n \mbf{f}_{ji} = \frac12 \sum_{i,j=1(i \neq j)}^n (\mbf{f}_{ij} + \mbf{f}_{ji}) = \mbf{0}
\end{equation*}
因此可有{\heiti 质点系的动量定理}
\begin{equation}
	\frac{\mathrm{d} \mbf{p}}{\mathrm{d} t} = \sum_{i=1}^n \mbf{F}_i = \mbf{F}^{(e)}
\end{equation}
即,质点系总动量的时间变化率等于合外力。

如果质点系所受合外力为零,则质点系的总动量保持不变。此即为{\heiti 动量守恒定律}。即如有$\mbf{F}^{(e)} = \mbf{0}$,根据质点系的动量定理,有
\begin{equation*}
	\mbf{F}^{(e)} = \dfrac{\mathrm{d} \mbf{p}}{\mathrm{d} t} = \mbf{0}
\end{equation*}
即总动量$\mbf{p} = \text{常矢量}$。

如果质点系在某方向上合外力为零,则质点系的总动量在该方向上的分量保持不变。即如果有常单位矢量$\mbf{e}$,满足$\mbf{e} \cdot \mbf{F}^{(e)} = 0$,则有
\begin{equation*}
	\mbf{e} \cdot \mbf{F}^{(e)} = \mbf{e} \cdot \frac{\mathrm{d} \mbf{p}}{\mathrm{d} t} = \frac{\mathrm{d} (\mbf{e} \cdot \mbf{p})}{\mathrm{d} t} = 0
\end{equation*}
即总动量在该方向上的分量$\mbf{e} \cdot \mbf{p} = \text{常数}$。

动量守恒定律在纯质点力学系统中由Newton第三定律保证。

\subsection{角动量及其守恒定律}

质点的{\heiti 角动量}\footnote{此处为对原点的角动量。}定义为
\begin{equation}
	\mbf{L} = \mbf{r} \times \mbf{p}
\end{equation}
{\heiti 力矩}\footnote{此处为对原点的力矩。}定义为
\begin{equation}
	\mbf{M} = \mbf{r} \times \mbf{F}
\end{equation}
则有
\begin{equation}
	\frac{\mathrm{d} \mbf{L}}{\mathrm{d} t} = \frac{\mathrm{d}}{\mathrm{d} t}(\mbf{r} \times \mbf{p}) = \mbf{v} \times \mbf{p} + \mbf{r} \times \frac{\mathrm{d} \mbf{p}}{\mathrm{d} t} = \mbf{r} \times \mbf{F} = \mbf{M}
	\label{角动量定理1}
\end{equation}
或者
\begin{equation}
	\mathrm{d} \mbf{L} = \mbf{M} \mathrm{d} t
	\label{角动量定理2}
\end{equation}
式\eqref{角动量定理1}和式\eqref{角动量定理2}表示的关系称为{\heiti 角动量定理}。

质点系的角动量定义为各个质点角动量之和,即
\begin{equation}
	\mbf{L} = \sum_{i=1}^n \mbf{L}_i
\end{equation}
则有
\begin{equation*}
	\frac{\mathrm{d} \mbf{L}}{\mathrm{d} t} = \sum_{i=1}^n \frac{\mathrm{d} \mbf{L}_i}{\mathrm{d} t} = \sum_{i=1}^n \mbf{r}_i \times \left(\mbf{F}_i + \sum_{j=1(j\neq i)}^n \mbf{f}_{ij}\right)
\end{equation*}
式中
\begin{align*}
	\sum_{i,j=1(i \neq j)}^n \mbf{r}_i \times \mbf{f}_{ij} & = \sum_{i,j=1(i\neq j)}^n \mbf{r}_j \times \mbf{f}_{ji} = \frac12 \sum_{i,j=1(i\neq j)}^n (\mbf{r}_i \times \mbf{f}_{ij} + \mbf{r}_j \times \mbf{f}_{ji}) \\
	& = \frac12 \sum_{i,j=1(i \neq j)}^n (\mbf{r}_i -\mbf{r}_j) \times \mbf{f}_{ij} = \frac12 \sum_{i,j=1(i \neq j)}^n \mbf{r}_{ij} \times \mbf{f}_{ij} = \mbf{0}
\end{align*}
此处利用了$\mbf{f}_{ij} \parallel \mbf{r}_{ij}$。由此即有{\heiti 质点系的角动量定理}
\begin{equation}
	\frac{\mathrm{d} \mbf{L}}{\mathrm{d} t} = \sum_{i=1}^n \mbf{r}_i \times \mbf{F}_i = \mbf{M}^{(e)}
\end{equation}
即,质点系的总角动量时间变化率等于合外力矩。

如果质点系所受合外力矩为零,则质点系的总角动量保持不变。此即为{\heiti 角动量守恒定律}。即如有$\mbf{M}^{(e)} = \mbf{0}$,根据质点系的角动量定理,有
\begin{equation*}
	\mbf{M}^{(e)} = \dfrac{\mathrm{d} \mbf{L}}{\mathrm{d} t} = \mbf{0}
\end{equation*}
即总角动量$\mbf{L} = \text{常矢量}$。

如果质点系在某方向上合外力矩为零,则质点系的总角动量在该方向上的分量保持不变。即如果有常单位矢量$\mbf{e}$,满足$\mbf{e} \cdot \mbf{M}^{(e)} = 0$,则有
\begin{equation*}
	\mbf{e} \cdot \mbf{M}^{(e)} = \mbf{e} \cdot \frac{\mathrm{d} \mbf{L}}{\mathrm{d} t} = \frac{\mathrm{d} (\mbf{e} \cdot \mbf{L})}{\mathrm{d} t} = 0
\end{equation*}
即总角动量在该方向上的分量$\mbf{e} \cdot \mbf{L} = \text{常数}$。

角动量守恒定律在纯质点力学系统中由Newton第三定律保证。

\subsection{能量及其守恒定律}

质点的{\heiti 动能}定义为
\begin{equation}
	T = \frac12 mv^2
\end{equation}
则有
\begin{equation}
	\frac{\mathrm{d} T}{\mathrm{d} t} = m\dot{\mbf{v}} \cdot \mbf{v} = \mbf{F} \cdot \mbf{v}
	\label{动能定理1}
\end{equation}
或者
\begin{equation}
	\mathrm{d}T = \mbf{F} \cdot \mbf{v}\mathrm{d} t = \mbf{F} \cdot \mathrm{d} \mbf{r}
	\label{动能定理2}
\end{equation}
式\eqref{动能定理1}和式\eqref{动能定理2}表示的关系称为{\heiti 动能定理}。

质点系的动能定义为各个质点动能之和,即
\begin{equation}
	T = \sum_{i=1}^n T_i
\end{equation}
则有
\begin{align*}
	\mathrm{d} T & = \sum_{i=1}^n \mathrm{d} T_i = \sum_{i=1}^n \left(\mbf{F}_i + \sum_{j=1(j\neq i)}^n \mbf{f}_{ij}\right) \cdot \mathrm{d} \mbf{r}_i
\end{align*}
式中
\begin{align*}
	\sum_{i,j=1(i\neq j)}^n \mbf{f}_{ij} \cdot \mathrm{d} \mbf{r}_i & = \frac12 \sum_{i,j=1(i\neq j)}^n (\mbf{f}_{ij} \cdot \mathrm{d} \mbf{r}_i +\mbf{f}_{ji} \cdot \mathrm{d} \mbf{r}_j) = \frac12 \sum_{i,j=1(i\neq j)}^n \mbf{f}_{ij} \cdot (\mathrm{d}\mbf{r}_i - \mathrm{d} \mbf{r}_j) \\
	& = \frac12 \sum_{i,j=1(i\neq j)}^n \mbf{f}_{ij} \cdot \mathrm{d} \mbf{r}_{ij}
\end{align*}
由此有{\heiti 质点系的动能定理}
\begin{equation}
	\mathrm{d} T = \sum_{i=1}^n \mbf{F}_i \cdot \mathrm{d} \mbf{r}_i + \frac12 \sum_{i,j=1(i\neq j)}^n \mbf{f}_{ij} \cdot \mathrm{d} \mbf{r}_{ij}
\end{equation}
即,质点系总动能增加量等于外力与内力的总功,内力共决定于质点间的相对运动。

如果力$\mbf{F}$可以写作某标量函数的梯度,即
\begin{equation}
	\mbf{F} = -\bnb V(\mbf{r})
\end{equation}
则该力称为{\heiti 保守力}。根据保守力的定义,可有
\begin{equation}
	\mbf{F} \cdot \mathrm{d} \mbf{r} = -\mathrm{d}V
\end{equation}
所以
\begin{equation*}
	\int_{\mbf{r}_0}^{\mbf{r}} \mbf{F} \cdot \mathrm{d} \mbf{r} = V(\mbf{r}_0) - V(\mbf{r})
\end{equation*}
即,保守力做功与路径无关。由此也可得
\begin{equation}
	\oint_C \mbf{F} \cdot \mathrm{d} \mbf{r} = 0
\end{equation}
即,对任意闭合路径$C$,保守力做功为零。

对于质点系,如果外力是保守力,即满足
\begin{equation}
	\mbf{F}_i = -\bnb_i V_i(\mbf{r}_i)
\end{equation}
其中
\begin{equation}
	\bnb_i = \mbf{e}_1 \frac{\pl}{\pl x_i} + \mbf{e}_2 \frac{\pl}{\pl y_i} + \mbf{e}_3 \frac{\pl}{\pl z_i}
\end{equation}
所以有
\begin{equation}
	\mbf{F}_i \cdot \mathrm{d} \mbf{r}_i = -\mathrm{d} V_i
\end{equation}
定义{\heiti 质点系外势能}为
\begin{equation}
	V^{(e)} = \sum_{i=1}^n V_i
\end{equation}
将有
\begin{equation}
	\sum_{i=1}^n \mbf{F}_i \cdot \mathrm{d} \mbf{r}_i = -\sum_{i=1}^n \mathrm{d} V_i = -\mathrm{d} V^{(e)}
\end{equation}
即,保守外力总功等于外势能减少。

如果质点系的内力是保守力,即有
\begin{align*}
	\mbf{f}_{ij} = -\bnb_i V_{ij}(\mbf{r}_{ij}) \\
	\mbf{f}_{ji} = -\bnb_j V_{ij}(\mbf{r}_{ij})
\end{align*}
为一对保守内力,其中
\begin{align*}
	& \mbf{r}_{ij} = \mbf{r}_i - \mbf{r}_j \\
	& \bnb_i = \bnb_{ij} = -\bnb_j
\end{align*}
由此内力可以表示为
\begin{equation*}
	\mbf{f}_{ij} = -\bnb_{ij} V_{ij}(\mbf{r}_{ij})
\end{equation*}
即有
\begin{equation}
	\mbf{f}_{ij} \cdot \mathrm{d} \mbf{r}_i + \mbf{f}_{ji} \cdot \mathrm{d} \mbf{r}_j = -\mathrm{d} V_{ij}
\end{equation}
定义{\heiti 质点系内势能}为
\begin{equation}
	V^{(i)} = \frac12 \sum_{i,j=1(i \neq j)}^n V_{ij}
\end{equation}
则有
\begin{equation}
	\sum_{i,j=1(i\neq j)}^n \mbf{f}_{ij} \cdot \mathrm{d} \mbf{r}_i = \frac12 \sum_{i,j=1(i \neq j)}^n \mbf{f}_{ij} \cdot \mathrm{d} \mbf{r}_{ij} = -\frac12 \sum_{i,j=1(i\neq j)}^n \mathrm{d} V_{ij} = -\mathrm{d} V^{(i)}
\end{equation}
即,保守内力总功等于内势能减少。

质点系的总势能定义为
\begin{equation}
	V(\mbf{r}_1,\mbf{r}_2,\cdots,\mbf{r}_n) = V^{(e)} + V^{(i)}
\end{equation}
则有
\begin{align*}
	-\bnb_i V^{(e)} & = -\sum_{j=1}^n \bnb_i V_j = \sum_{j=1}^n \delta_{ij} \mbf{F}_i = \mbf{F}_i \\
	-\bnb_i V^{(i)} & = -\frac12 \sum_{j,k=1(j \neq k)}^n \bnb_i V_{jk} = \frac12 \sum_{j,k=1(j\neq k)}^n (\delta_{ij} \mbf{f}_{ik} + \delta_{ik} \mbf{f}_{ij}) \\
	& = \frac12 \sum_{k=1(k\neq i)}^n \mbf{f}_{ik} + \frac12 \sum_{j(j\neq i)} \mbf{f}_{ij} = \sum_{j=1(j\neq i)}^n \mbf{f}_{ij}
\end{align*}
所以有
\begin{equation}
	\mbf{F}_i + \sum_{j=1(j\neq i)}^n \mbf{f}_{ij} = -\bnb_i V
\end{equation}
当所有力都为保守力(或者非保守力不做功)时,将有
\begin{equation}
	\mathrm{d} T = \sum_{i=1}^n \mbf{F}_i \cdot \mathrm{d} \mbf{r}_i + \frac12 \sum_{i,j=1(i\neq j)}^n \mbf{f}_{ij} \cdot \mathrm{d} \mbf{r}_{ij} = -\mathrm{d} V^{(e)} - \mathrm{d} V^{(i)} = -\mathrm{d} V
\end{equation}
定义{\heiti 总能量}\footnote{此处仅指机械能。}为
\begin{equation}
	E = T+V
\end{equation}
则在所有力都为保守力(或者非保守力不做功)时,将有
\begin{equation}
	\mathrm{d} E = 0
	\label{质点系能量守恒定律}
\end{equation}
式\eqref{质点系能量守恒定律}即为能量守恒定律。

\subsection{质心参考系}

质点系的{\heiti 质心}定义为
\begin{equation}
	\mbf{r}_C = \frac{\displaystyle \sum_{i=1}^n m_i \mbf{r}_i}{M} = \frac{\displaystyle \sum_{i=1}^n m_i \mbf{r}_i}{\displaystyle \sum_{i=1}^n m_i}
\end{equation}
质心的速度为
\begin{equation}
	\mbf{v}_C = \frac{\displaystyle \sum_{i=1}^n m_i \mbf{v}_i}{M}
\end{equation}
加速度为
\begin{equation}
	\mbf{a}_C = \frac{\displaystyle \sum_{i=1}^n m_i \mbf{a}_i}{M}
\end{equation}

显然质点系的总动量可以用质心速度表示为
\begin{equation*}
	\mbf{p} = M\mbf{v}_C
\end{equation*}
由此可有
\begin{equation}
	\frac{\mathrm{d} \mbf{p}}{\mathrm{d} t} = M\mbf{a}_C = \mbf{F}^{(e)}
	\label{质心运动定理}
\end{equation}
式\eqref{质心运动定理}表示的关系即为{\heiti 质心运动定理}。即,质点系的整体运动可视为全部质量集中于质心的单质点运动。

质心运动定理是引进质点概念的动力学基础。

\subsection{质心参考系的运动定理}

以质心为原点,并随之平动的参考系称为{\heiti 质心参考系}或{\heiti 质心系}。一般为非惯性系。

在质心系中,各质点的位矢和速度为
\begin{align}
	\mbf{r}'_i & = \mbf{r}_i - \mbf{r}_C \\
	\mbf{v}'_i & = \mbf{v}_i - \mbf{v}_C
\end{align}
即为各质点相对质心的运动(质点系的内部运动)。

\subsubsection{质心系动量定理}

在质心系中,总动量为
\begin{equation}
	\mbf{p}' = \sum_{i=1}^n m_i \mbf{v}'_i = \sum_{i=1}^n m_i(\mbf{v}_i - \mbf{v}_C) = \mbf{p} - M\mbf{v}_C = \mbf{0}
\end{equation}
即,在质心系中质点系的总动量为零。

\subsubsection{质心系角动量定理}
在质心系中,质点系的总角动量
\begin{equation}
	\mbf{L}' = \sum_{i=1}^n \mbf{r}'_i \times \mbf{p}'_i = \sum_{i=1}^n \mbf{r}'_i \times m_i \mbf{v}'_i
\end{equation}
即为质点系的{\heiti 内秉角动量}或{\heiti 自旋角动量}。质心的角动量
\begin{equation}
	\mbf{L}_C = \mbf{r}_C \times M\mbf{v}_C
\end{equation}
称为质点系的{\heiti 轨道角动量}。在实验室参考系中的总角动量为
\begin{align*}
	\mbf{L} & = \sum_{i=1}^n (\mbf{r}_C + \mbf{r}'_i) \times m_i(\mbf{v}_C + \mbf{v}'_i) \\
	& = \mbf{r}_C \times M\mbf{v}_C + \mbf{r}_C\times \mbf{p}' + \sum_{i=1}^n m_i \mbf{r}'_i \times \mbf{v}_C + \sum_{i=1}^n \mbf{r}'_i \times m_i \mbf{v}'_i \\
	& = \mbf{r}_C \times M\mbf{v}_C + \sum_{i=1}^n \mbf{r}'_i \times m_i \mbf{v}'_i
\end{align*}
所以有
\begin{equation}
	\mbf{L} = \mbf{L}_C+\mbf{L}'
\end{equation}
即,质点系的总角动量等于轨道角动量与内秉角动量之和。对于力矩,可有
\begin{equation}
	\mbf{M}^{(e)} = \sum_{i=1}^n \mbf{r}_i \times \mbf{F}_i = \sum_{i=1}^n (\mbf{r}_C+\mbf{r}'_i) \times \mbf{F}_i = \mbf{r}_C \times \mbf{F}^{(e)} + \mbf{M}'^{(e)}
\end{equation}
式中,$\displaystyle \mbf{M}'^{(e)} = \sum_{i=1}^n \mbf{r}'_i \times \mbf{F}_i$表示外力对质心的力矩。考虑到轨道角动量$\mbf{L}_C = \mbf{r}_C \times M\mbf{v}_C$,则有
\begin{equation}
	\frac{\mathrm{d} \mbf{L}_C}{\mathrm{d} t} = \mbf{v}_C \times M\mbf{v}_C + \mbf{r}_C \times M \mbf{a}_C = \mbf{r}_C \times \mbf{F}^{(e)}
	\label{质心角动量定理}
\end{equation}
即,质点系的轨道角动量时间变化率等于外力集中于质心产生的合力矩。根据角动量定理$\dfrac{\mathrm{d} \mbf{L}}{\mathrm{d} t} = \mbf{M}^{(e)}$,以及$\mbf{L} = \mbf{L}_C + \mbf{L}'$,可有
\begin{equation}
	\frac{\mathrm{d} \mbf{L}'}{\mathrm{d} t} = \mbf{M}'^{(e)}
	\label{质心系角动量定理}
\end{equation}
即,质点系的内秉角动量时间变化率等于相对于质心的合外力矩。式\eqref{质心角动量定理}和式\eqref{质心系角动量定理}表示的关系即为{\heiti 质心系角动量定理}。

\subsubsection{质心系动能定理}

定义{\heiti 质点系整体运动动能}为
\begin{equation}
	T_C = \frac12 M v_C^2
\end{equation}
以及{\heiti 质点系内动能}为
\begin{equation}
	T' = \sum_{i=1}^n \frac12 m_i v'_i{}^2
\end{equation}
则有
\begin{align}
	T & = \frac12 \sum_{i=1}^n m_i(\mbf{v}_C + \mbf{v}'_i) \cdot (\mbf{v}_C + \mbf{v}'_i) = \frac12 M v_C^2 + \mbf{v}_C \cdot \mbf{p}' + \frac12\sum_{i=1}^n m_i v'_i{}^2 \nonumber \\
	& = T_C + T' \label{Konig定理}
\end{align}
式\eqref{Konig定理}表示的关系称为{\heiti K\"onig(柯尼希)定理}。即,质点系总动能等于整体运动动能与内动能之和。

根据整体运动动能的定义,有
\begin{equation}
	\frac{\mathrm{d} T_C}{\mathrm{d} t} = M\mbf{v}_C \cdot \mbf{a}_C = \mbf{F}^{(e)} \cdot \mbf{v}_C
\end{equation}
或者
\begin{equation}
	\mathrm{d}T_C = \mbf{F}^{(e)} \cdot \mathrm{d} \mbf{r}_C
	\label{质心动能定理}
\end{equation}
即,质点系整体运动动能增量等于外力集中于质心做的总功。又考虑到相对位矢是伽利略变换的不变量,即有
\begin{equation*}
	\mbf{r}_{ij} = \mbf{r}_i - \mbf{r}_j = \mbf{r}'_i - \mbf{r}'_j = \mbf{r}'_{ij}
\end{equation*}
可得
\begin{align*}
	\mathrm{d} T & = \sum_{i=1}^n \mbf{F}_i \cdot \mathrm{d} \mbf{r}_i + \frac12 \sum_{i,j=1(i\neq j)}^n \mbf{f}_{ij} \cdot \mathrm{d} \mbf{r}_{ij} \\
	& = \sum_{i=1}^n \mbf{F}_i \cdot \mathrm{d} \mbf{r}_C + \sum_i \mbf{F}_i \cdot \mathrm{d} \mbf{r}'_i + \frac12 \sum_{i,j=1(i\neq j)}^n \mbf{f}_{ij} \cdot \mathrm{d} \mbf{r}'_{ij} \\
	& = \mbf{F}^{(e)} \cdot \mathrm{d} \mbf{r}_C + \sum_{i=1}^n \mbf{F}_i \cdot \mathrm{d} \mbf{r}'_i + \frac12 \sum_{i,j=1(i \neq j)}^n \mbf{f}_{ij} \cdot \mathrm{d} \mbf{r}'_{ij}
\end{align*}
所以有
\begin{equation}
	\mathrm{d} T' = \sum_{i=1}^n \mbf{F}_i \cdot \mathrm{d} \mbf{r}'_i + \frac12 \sum_{i,j=1(i \neq j)}^n \mbf{f}_{ij} \cdot \mathrm{d} \mbf{r}'_{ij}
	\label{质心系动能定理}
\end{equation}
式\eqref{质心动能定理}和式\eqref{质心系动能定理}表示的关系称为{\heiti 质心系动能定理}。即,质点系内动能增量等于外力与内力在质心系内做的总功。

如果内力均为保守力,则有
\begin{equation*}
	\frac12 \sum_{i,j=1(i \neq j)}^n \mbf{f}_{ij} \cdot \mathrm{d} \mbf{r}_{ij} = \frac12 \sum_{i,j=1(i \neq j)}^n \mbf{f}_{ij} \cdot \mathrm{d} \mbf{r}'_{ij} = -\mathrm{d} V^{(i)}
	\label{质点系内势能}
\end{equation*}
式中$V^{(i)}$称为{\heiti 质点系内势能}。定义{\heiti 质点系内能}为
\begin{equation}
	E' = T'+V^{(i)}
\end{equation}
由式\eqref{质点系内势能}和式\eqref{质心系动能定理}可得
\begin{equation}
	\mathrm{d} E' = \sum_{i=1}^n \mbf{F}_i \cdot \mathrm{d} \mbf{r}'_i
\end{equation}
即,质点系内能增量等于外力在质心系内做的总功。

\begin{example}
光滑水平面上有一质量为$m$的质点,用一轻绳系着,绳子完全绕在半径为$R$的圆柱上,使质点贴在圆柱体的边缘上。$t=0$时给质点以一径向速度$\mbf{v}_0$,据此求任意时刻绳上的张力大小$T(t)$。
\begin{figure}[htb]
\centering
\begin{asy}
	size(300);
	//第一章例4图
	real R,r,theta,dtheta;
	pair O,C,Cp,P,Pp,v0,vt,vtp;
	O = (0,0);
	R = 2.5;
	theta = 110;
	dtheta = 15;
	dot(O);
	label("$\theta$",O,NE);
	draw(scale(R)*unitcircle,linewidth(0.8bp));
	draw(Label("$R$",MidPoint,W),O--R*dir(90));
	r = 0.3*R;
	draw(arc(O,r,90-theta-dtheta-25,90-theta-dtheta),Arrow);
	draw(Label("$\mathrm{d} \theta$",MidPoint,Relative(W)),arc(O,r,90-theta+25,90-theta),Arrow);
	
	pair jkx(real phi){
		return R*dir(90-phi)+R*phi*pi/180*dir(180-phi);
	}
	pair djkx(real phi){
		real phir;
		phir = phi*pi/180;
		return (sin(phir),cos(phir));
	}
	draw(graph(jkx,0,150),red+linewidth(0.8bp));
	draw(Label("$\boldsymbol{v}_0$",EndPoint),jkx(0)--jkx(0)+djkx(0),Arrow);
	dot(jkx(0));
	draw(Label("$\boldsymbol{v}$",EndPoint,Relative(W)),jkx(theta)--jkx(theta)+djkx(theta),Arrow);
	dot(jkx(theta));
	draw(jkx(theta+dtheta)--jkx(theta+dtheta)+djkx(theta+dtheta),Arrow);
	dot(jkx(theta+dtheta));
	draw(O--R*dir(90-theta));
	draw(Label("$R\theta$",MidPoint,Relative(W)),R*dir(90-theta)--jkx(theta));
	draw(jkx(theta)--jkx(theta)-dir(180-theta),blue,Arrow);
	draw(O--R*dir(90-theta-dtheta)--jkx(theta+dtheta),dashed);
	label("$\mathrm{d} s$",jkx(theta+dtheta/2),S);
	draw(O--(2.95*R,1.6*R),invisible);
\end{asy}
\caption{例\theexample}
\label{第一章例4图}
\end{figure}
\end{example}
\begin{solution}
由题意,小球的轨道为圆的渐开线。即任意时刻绳子与圆相切,绳子的不可伸长性要求质点速度正交于绳子。轨道的参数方程可以表示为
\begin{equation*}
	\mbf{r}(\theta) = R \begin{pmatrix} \cos\left(\dfrac{\pi}{2}-\theta\right) \\[1.5ex] \sin \left(\dfrac{\pi}{2}-\theta\right) \end{pmatrix} + R\theta \begin{pmatrix} \cos(\pi-\theta) \\ \sin(\pi-\theta) \end{pmatrix} = \begin{pmatrix} R\sin \theta-R\theta\cos \theta \\ R\cos \theta+R\theta \sin \theta \end{pmatrix}
\end{equation*}
在切线方向上有
\begin{equation*}
	m\dot{v} = F_\tau = 0
\end{equation*}
即有
\begin{equation*}
	v = v_0
\end{equation*}
在法线方向上有
\begin{equation*}
	m \frac{v^2}{\rho} = T
\end{equation*}
式中$\rho$为该点的曲率半径。根据轨道的方程可得
\begin{equation*}
	\rho = \left|\frac{\left[(x'(\theta))^2+(y'(\theta))^2\right]^{\frac32}}{x'(\theta)y''(\theta)-x''(\theta)y'(\theta)}\right| = R\theta
\end{equation*}
再考虑到
\begin{equation*}
	\frac{\mathrm{d} \theta}{\mathrm{d} t} = \frac{\mathrm{d} \theta}{\mathrm{d} s} \frac{\mathrm{d} s}{\mathrm{d} t} = \frac{v}{\rho} = \frac{v}{R\theta}
\end{equation*}
可得
\begin{equation*}
	\theta(t) = \sqrt{\frac{2v_0t}{R}}
\end{equation*}
因此有
\begin{equation*}
	T = m\frac{v^2}{\rho} = m\frac{v^2}{R\theta} = mv_0 \sqrt{\frac{v_0}{2Rt}}
\end{equation*}
\end{solution}

\begin{example}[计入空气阻力的斜抛运动]
设空气阻力$\mbf{f} = -k\mbf{v}$,重力加速度为$\mbf{g} = -g\mbf{e}_2$,初始速度$\mbf{v}_0 = v_0(\cos \alpha \mbf{e}_1 + \sin \alpha \mbf{e}_2)$,初始位矢$\mbf{r}_0 = \mbf{0}$,求物体的轨迹。
\end{example}

\begin{solution}
根据Newton第二定律,可有矢量形式的微分方程
\begin{equation*}
	m\frac{\mathrm{d} \mbf{v}}{\mathrm{d} t} = m\mbf{g} - k\mbf{v}
\end{equation*}
整理可得
\begin{equation*}
	\frac{\mathrm{d} \mbf{v}}{\mathrm{d} t} + \frac{\mbf{v}}{\tau} = \mbf{g}
\end{equation*}
此处$\tau = \dfrac{m}{k}$。首先解齐次方程
\begin{equation*}
	\frac{\mathrm{d} \mbf{v}}{\mathrm{d} t} + \frac{\mbf{v}}{\tau} = \mbf{0}
\end{equation*}
将其写成分量形式即为
\begin{equation*}
	\begin{cases}
		\dfrac{\mathrm{d} v_x}{\mathrm{d} t} = -\dfrac{v_x}{\tau} \\[1.5ex]
		\dfrac{\mathrm{d} v_y}{\mathrm{d} t} = -\dfrac{v_y}{\tau}
	\end{cases}
\end{equation*}
其解为
\begin{equation*}
	\begin{cases}
		v_x = C_1 \mathrm{e}^{-\frac{t}{\tau}} \\
		v_y = C_2 \mathrm{e}^{-\frac{t}{\tau}}
	\end{cases}
\end{equation*}
或者写作矢量形式为
\begin{equation*}
	\mbf{v} = \mbf{C} \mathrm{e}^{-\frac{t}{\tau}}
\end{equation*}
利用常数变易法,令$\mbf{v}(t) = \mbf{C}(t) \mathrm{e}^{-\frac{t}{\tau}}$,代入原方程,可得
\begin{equation*}
	\frac{\mathrm{d} \mbf{C}}{\mathrm{d} t} = \mbf{g}\mathrm{e}^{\frac{t}{\tau}}
\end{equation*}
即有$\mbf{C}(t) = \mbf{g}\tau \mathrm{e}^{\frac{t}{\tau}} + \mbf{C}_1$,因此
\begin{equation*}
	\mbf{v}(t) = \mbf{C}_1 \mathrm{e}^{-\frac{t}{\tau}} + \mbf{g} \tau
\end{equation*}
利用$\mbf{v}(0) = \mbf{v}_0$,可得
\begin{equation*}
	\mbf{v}(t) = \mbf{v}_0 \mathrm{e}^{-\frac{t}{\tau}} + \mbf{g}\tau \left(1-\mathrm{e}^{-\frac{t}{\tau}}\right)
\end{equation*}
再对时间积分一次,可得轨迹
\begin{equation*}
	\mbf{r}(t) = -\mbf{v}_0 \tau \mathrm{e}^{-\frac{t}{\tau}} + \mbf{g}\tau(t + \tau \mathrm{e}^{-\frac{t}{\tau}}) + \mbf{C}_2
\end{equation*}
利用$\mbf{r}(0) = \mbf{0}$,可得
\begin{equation*}
	\mbf{r}(t) = \mbf{v}_0 \tau \left(1-\mathrm{e}^{-\frac{t}{\tau}}\right) +\mbf{g}\tau \left[t - \tau\left(1 - \mathrm{e}^{-\frac{t}{\tau}}\right)\right]
\end{equation*}
即
\begin{equation*}
	\begin{cases}
		x(t) = v_0 \tau \cos \alpha \left(1-\mathrm{e}^{-\frac{t}{\tau}}\right) \\ 
		y(t) = -g\tau t + \tau(v_0\sin \alpha+g\tau) \left(1-\mathrm{e}^{-\frac{t}{\tau}}\right)
	\end{cases}
\end{equation*}
消去时间可以得到轨道方程
\begin{equation*}
	y = \left(\tau \alpha + \frac{g\tau}{v_0\cos\alpha}\right) x g\tau^2 \ln \left(1-\frac{x}{v_0\tau \cos \alpha}\right)
\end{equation*}

\begin{figure}[htb]
\centering
\begin{asy}
	size(350);
	//不同阻尼系数下的轨道曲线图
	real x,y;
	picture g,tmp;
	real kappa;
	real tra(real x){
		return 2*(1+1/kappa)*x+2/kappa/kappa*log(1-kappa*x);
	}
	real tra0(real x){
		return 2*x-x**2;
	}
	for(y=-2.5;y<1.5;y=y+0.5){
		draw(g,(0,y)--(3,y),0.6white);
	}
	for(x=0.25;x<3;x=x+0.25){
		draw(g,(x,-3)--(x,1.5),0.6white);
	}
	kappa = 0.1;
	draw(tmp,graph(tra,0,0.9/kappa),blue+linewidth(1bp));
	kappa = 0.5;
	draw(tmp,graph(tra,0,0.9/kappa),green+linewidth(1bp));
	kappa = 1.0;
	draw(tmp,graph(tra,0,0.9678),orange+linewidth(1bp));
	kappa = 2.0;
	draw(tmp,graph(tra,0,0.499938),red+linewidth(1bp));
	draw(tmp,graph(tra0,0,3),linewidth(1bp));
	clip(tmp,box((0,-3),(3,1.5)));
	add(g,tmp);
	erase(tmp);
	draw(g,box((0,-3),(3,1.5)),linewidth(1bp));
	label(g,"$\dfrac{x}{x_m}$",(3,-3),2.5*E);
	label(g,"$\dfrac{y}{y_m}$",(0,1.5),W);
	label(g,"$1$",(0,1),W);
	label(g,"$0$",(0,0),W);
	label(g,"$-1$",(0,-1),W);
	label(g,"$-2$",(0,-2),W);
	label(g,"$-3$",(0,-3),W);
	label(g,"$0.0$",(0,-3),S);
	label(g,"$0.5$",(0.5,-3),S);
	label(g,"$1.0$",(1,-3),S);
	label(g,"$1.5$",(1.5,-3),S);
	label(g,"$2.0$",(2,-3),S);
	label(g,"$2.5$",(2.5,-3),S);
	label(g,"$3.0$",(3,-3),S);
	pair A,B;
	A = (2.35,-0.3);
	B = (2.85,1.3);
	real delta;
	delta = (B.y-A.y)/10;
	fill(g,shift((0.025,-0.05))*box(A,B),black);
	unfill(g,box(A,B));
	draw(g,box(A,B));
	label(g,"$\kappa = 0.0$",((A.x+B.x)/2,A.y)+(0,9*delta));
	label(g,"$\kappa = 0.1$",((A.x+B.x)/2,A.y)+(0,7*delta),blue);
	label(g,"$\kappa = 0.5$",((A.x+B.x)/2,A.y)+(0,5*delta),green);
	label(g,"$\kappa = 1.0$",((A.x+B.x)/2,A.y)+(0,3*delta),orange);
	label(g,"$\kappa = 2.0$",((A.x+B.x)/2,A.y)+(0,1*delta),red);
	add(xscale(2)*g);
\end{asy}
\caption{不同阻尼系数下的轨道曲线图}
\label{不同阻尼系数下的轨道曲线图}
\end{figure}

考虑无阻尼极限$k=0$,即$\tau = \dfrac{m}{k} \to +\infty$的情况。根据当$u \to 0$时,有
\begin{equation*}
	\ln (1+u) = u - \frac12 u^2 + o(u^2)
\end{equation*}
当$\tau \to +\infty$时,轨道变为
\begin{equation*}
	y = x\tan \alpha -\frac{gx^2}{2v_0^2\cos^2 \alpha}
\end{equation*}
由此,令
\begin{equation*}
	\kappa = \frac{kv_0 \sin \alpha}{mg} = \frac{v_0 \sin \alpha}{g\tau},\quad x_m = \frac{v_0^2 \sin 2\alpha}{2g},\quad y_m = \frac{v_0^2 \sin^2 \alpha}{2g}
\end{equation*}
则在无阻尼的情形,即$\kappa=0$时,轨道可以表示为
\begin{equation*}
	\frac{y}{y_m} = 2\frac{x}{x_m} - \left(\frac{x}{x_m}\right)^2
\end{equation*}
当$\kappa \neq 0$时,轨道方程为
\begin{equation*}
	\frac{y}{y_m} = 2\left(1+\frac{1}{\kappa}\right) \frac{x}{x_m} + \frac{2}{\kappa^2} \ln \left(1-\kappa \frac{x}{x_m}\right)
\end{equation*}
上式这种形式称为{\heiti 无量纲形式}。不同阻尼系数$\kappa$下的轨道曲线如图\ref{不同阻尼系数下的轨道曲线图}所示。由图可见阻力越大,越早的接近竖直运动。
\end{solution}

\begin{example}
\begin{figure}[htb]
\centering
\begin{asy}
	size(200);
	//第一章例6图
	pair O,P;
	real r,theta;
	O = (0,0);
	r = 2;
	theta = 40;
	draw(scale(r)*unitcircle);
	draw(Label("$R$",MidPoint,N),O--r*dir(0),dashed);
	P = r*dir(-theta);
	draw(O--P,dashed);
	draw(Label("$\theta$",MidPoint,Relative(W)),arc(O,0.3*r,0,-theta),Arrow);
	dot(P);
	draw(Label("$\boldsymbol{G}$",EndPoint),P--P+dir(-90),Arrow);
	draw(Label("$\boldsymbol{N}$",EndPoint,SW),P--P+dir(180-theta),Arrow);
	draw(Label("$\boldsymbol{f}$",EndPoint,N),P--P+dir(90-theta),Arrow);
	dot((0,r+0.01),invisible);
\end{asy}
\caption{例\theexample}
\label{第一章例6图}
\end{figure}

设有竖直圆环轨道上串有钢珠,将钢珠置于$\theta = 0$处释放,已知$v\big|_{\theta=0} = v\big|_{\theta=\frac{\pi}{2}} = 0$,求轨道与钢珠之间的摩擦系数$\mu$。
\end{example}
\begin{solution}
速度和加速度为
\begin{equation*}
	v = R\dot{\theta},\quad a_\tau = R\ddot{\theta},\quad a_n = R\dot{\theta}^2
\end{equation*}
根据Newton第二定律以及摩擦力的关系,可有方程组
\begin{equation*}
	\begin{cases}
		mR\ddot{\theta} = mg\cos \theta - f \\
		mR\dot{\theta}^2 = N-mg\sin \theta \\
		f = \mu N
	\end{cases}
\end{equation*}
从中消去$N$,可得
\begin{equation*}
	R(\ddot{\theta} + \mu\dot{\theta}^2) = g(\cos \theta - \mu\sin \theta)
\end{equation*}
再考虑到
\begin{equation*}
	\ddot{\theta} = \frac{\mathrm{d} \dot{\theta}}{\mathrm{d} t} = \frac{\mathrm{d} \dot{\theta}}{\mathrm{d} \theta} \frac{\mathrm{d} \theta}{\mathrm{d} t} = \dot{\theta} \frac{\mathrm{d} \dot{\theta}}{\mathrm{d} \theta} = \frac12 \frac{\mathrm{d} \dot{\theta}^2}{\mathrm{d} \theta}
\end{equation*}
可有
\begin{equation*}
	\frac{\mathrm{d} \dot{\theta}^2}{\mathrm{d} \theta} + 2\mu\dot{\theta}^2 = \frac{2g}{R}(\cos \theta - \mu\sin \theta)
\end{equation*}
可用常数变易法解此微分方程,得
\begin{equation*}
	\dot{\theta}^2 = \frac{2g}{(1+4\mu^2)R} \left[(1-2\mu^2)\sin \theta + 3\mu \cos \theta\right] + C\mathrm{e}^{-2\mu\theta}
\end{equation*}
再考虑到$\dot{\theta}\big|_{\theta=0} = 0$,可得
\begin{equation*}
	\dot{\theta}^2 = \frac{2g}{(1+4\mu^2)R} \left[(1-2\mu^2)\sin \theta + 3\mu (\cos \theta-\mathrm{e}^{-2\mu\theta})\right]
\end{equation*}
最后利用$\dot{\theta}\big|_{\theta=\frac{\pi}{2}} = 0$,可得$\mu$满足的方程为
\begin{equation*}
	1-2\mu^2-3\mu\mathrm{e}^{-\mu\pi} = 0
\end{equation*}
整理为
\begin{equation*}
	\mu = \sqrt{\frac{1-3\mu\mathrm{e}^{-\mu\pi}}{2}} =: f(\mu)
\end{equation*}
即$\mu$为函数$f(\mu)$的不动点。求此不动点可用压缩映射定理证明中的迭代法,即$\mu^{(n+1)} = f(\mu^{(n)})$来求解。具体如下
\begin{align*}
	& \mu^{(0)} = 0.000 \\
	& \mu^{(1)} = f(\mu^{(0)}) = 0.707\cdots \\
	& \mu^{(2)} = f(\mu^{(1)}) = 0.620\cdots \\
	& \mu^{(3)} = f(\mu^{(2)}) = 0.606\cdots \\
	& \quad \quad \vdots \\
	& \mu \doteq 0.603
\end{align*}
\begin{figure}[htb]
\centering
\begin{asy}
	size(300);
	//第一章例6求解示意
	pair O;
	real x,y,xm,ym;
	O = (0,0);
	xm = 0.8;
	ym = 0.8;
	//画坐标系
	draw(Label("$\mu$",EndPoint),O--(xm,0));
	draw(Label("$f(\mu)$",EndPoint),O--(0,ym));
	label("$0.0$",O,S);
	label("$0.0$",O,W);
	for(x = 0.1;x < xm-0.1;x = x+0.1){
		draw((x,0)--(x,0.01));
	}
	label("$0.5$",(0.5,0),S);
	for(y = 0.1;y < ym-0.1;y = y+0.1){
		draw((0,y)--(0.01,y));
	}
	label("$0.5$",(0,0.5),W);
	
	real lin(real mu){
		return mu;
	}
	real fmu(real mu){
		return sqrt((1-3*mu*exp(-mu*pi))/2);
	}
	draw(graph(lin,0,xm),linewidth(0.8bp));
	draw(Label("$f(\mu)$",EndPoint),graph(fmu,0,xm),red+linewidth(0.8bp));
	draw(O--(0,fmu(0))--(fmu(0),fmu(0))--(fmu(0),fmu(fmu(0)))--(fmu(fmu(0)),fmu(fmu(0)))--(fmu(fmu(0)),fmu(fmu(fmu(0))))--(fmu(fmu(fmu(0))),fmu(fmu(fmu(0)))),blue+linewidth(0.8bp));
	label("$0.707\cdots$",(fmu(0),fmu(0)),E);
	label("$0.620\cdots$",(fmu(fmu(0)),fmu(fmu(0))),NW);
	label("$0.606\cdots$",(fmu(fmu(fmu(0))),fmu(fmu(fmu(0)))),W);
	real m;
	m = 0;
	for(int i=1;i<10;++i){
		m = fmu(m);
	}
	dot((m,m));
	label("$0.603\cdots$",(m,m),SE);
	//draw((0,0)--(xm+0.2,0),invisible);
\end{asy}
\caption{求解示意}
\label{求解示意}
\end{figure}
\end{solution}

\begin{example}
在铅直平面内有一光滑的半圆形管道(如图\ref{第一章例7图}所示),半径为$R$,管道内有一长为$\pi R$,质量线密度为$\rho$的均匀链条。假定链条由于轻微扰动而从管口向外滑出,试用角动量定理求出链条位于任意角度$\theta$时的速度$v$。
\begin{figure}[htb]
\centering
\begin{asy}
	size(200);
	//第一章例7图
	pair O;
	real R,r,rr,theta,alpha,dalpha;
	O = (0,0);
	R = 1;
	r = 1.05;
	rr = 0.3;
	theta = 30;
	alpha = 40;
	dalpha = 15;
	draw(arc(O,R,0,180)--cycle);
	draw(arc(O,r,0,180)--cycle);
	draw(Label("$R$",MidPoint),O--r*dir(180-theta));
	draw(Label("$\theta$",MidPoint,Relative(E)),arc(O,rr*R,180-theta,180),Arrow);
	draw(O--r*dir(alpha));
	draw(Label("$\alpha$",MidPoint,Relative(E)),arc(O,rr*R,0,alpha),Arrow);
	draw(O--r*dir(alpha+dalpha));
	rr = 0.6;
	draw(Label("$\mathrm{d} \alpha$",MidPoint,Relative(E)),arc(O,rr*R,alpha,alpha+dalpha),Arrow);
	draw(arc(O,(r+R)/2,180-theta-1,0)--(r+R)/2*dir(0)+(0,-(R+r)/2*theta*pi/180),red+linewidth(4bp));
	draw(Label("$\boldsymbol{G}_1$",EndPoint),(r+R)/2*dir(0)+(0,-(R+r)/2*theta*pi/180/2)--(r+R)/2*dir(0)+(0,-(R+r)/2*theta*pi/180/2-0.5),Arrow);
	pair P;
	P = (R+r)/2*dir(alpha+dalpha/2);
	draw(Label("$\mathrm{d} \boldsymbol{G}$",EndPoint),P--P+(0,-0.5),Arrow);
	draw(Label("$\boldsymbol{v}$",EndPoint),P--P+0.5*dir(alpha+dalpha/2-90),Arrow);
	draw(Label("$\mathrm{d} \boldsymbol{N}$",EndPoint),P--P+0.5*P,Arrow);
\end{asy}
\caption{例\theexample}
\label{第一章例7图}
\end{figure}
\end{example}
\begin{solution}
由于链条不可伸长,因此其个点速度相等,即
\begin{equation*}
	v = R\dot{\theta}
\end{equation*}
由此可得链条的角动量
\begin{equation*}
	\mbf{L} = \int_C \mbf{r} \times \mbf{v} \mathrm{d} l = \rho \pi R (R\dot{\theta} \mbf{e}) = \rho \pi R^3 \dot{\theta} \mbf{e}
\end{equation*}
式中$\mbf{e}$为垂直纸面向里的单位矢量。

由于$\mbf{r} \times \mathrm{d} \mbf{N} = \mbf{0}$,故只需计算重力力矩。对于直线段,可有
\begin{equation*}
	\mbf{M}_1 = \mbf{r} \times \mbf{G}_1 = (\rho R\theta) Rg\mbf{e} = \rho gR^2 \theta \mbf{e}
\end{equation*}
对于圆弧段,可有
\begin{equation*}
	\mbf{M}_2 = \int_C \mbf{r} \times \rho\mbf{g} \mathrm{d} l = \int_0^{\pi-\theta} Rg\cos \alpha \rho R \mathrm{d} \alpha \mbf{e} = \rho gR^2 \sin \theta \mbf{e}
\end{equation*}
由此可得总力矩为
\begin{equation*}
	\mbf{M} = \mbf{M}_1 + \mbf{M}_2 = \rho gR^2 (\theta+\sin \theta)\mbf{e}
\end{equation*}
根据角动量定理$\dfrac{\mathrm{d} \mbf{L}}{\mathrm{d} t} = \mbf{M}$,可得
\begin{equation*}
	\rho \pi R^3 \ddot{\theta} = \rho gR^2 (\theta+\sin \theta)
\end{equation*}
根据
\begin{equation*}
	\ddot{\theta} = \frac{\mathrm{d} \dot{\theta}}{\mathrm{d} t} = \frac{\mathrm{d} \dot{\theta}}{\mathrm{d} \theta} \frac{\mathrm{d} \theta}{\mathrm{d} t} = \dot{\theta} \frac{\mathrm{d} \dot{\theta}}{\mathrm{d} \theta} = \frac12 \frac{\mathrm{d} \dot{\theta}^2}{\mathrm{d} \theta}
\end{equation*}
可得
\begin{equation*}
	\frac{\mathrm{d} \dot{\theta}^2}{\mathrm{d} \theta} = \frac{2g}{\pi R}(\theta +\sin \theta)
\end{equation*}
积分即可得到
\begin{equation*}
	\dot{\theta} = \sqrt{\frac{2g}{\pi R} \left(\frac12 \theta^2 - \cos \theta + 1\right)} = 2\sqrt{\frac{g}{\pi R}\left[\left(\frac{\theta}{2}\right)^2+ \sin^2 \frac{\theta}{2}\right]}
\end{equation*}
所以速度为
\begin{equation*}
	v = R\dot{\theta} = 2\sqrt{\frac{Rg}{\pi}\left[\left(\frac{\theta}{2}\right)^2+ \sin^2 \frac{\theta}{2}\right]}
\end{equation*}

本体也可使用机械能守恒来求解。此系统中,非保守力(弹力)不做功,因此系统机械能守恒。在初始状态下,链条的动能
\begin{equation*}
	T = 0
\end{equation*}
以半圆柱的底面为零势能面,可求得初始状态下链条的重力势能为
\begin{equation*}
	V = \int_C \rho gh \mathrm{d} l = \int_0^\pi \rho g R\sin \alpha R\mathrm{d} \alpha = 2\rho gR^2
\end{equation*}
在图\ref{第一章例7图}所示的状态下,由于链条上各处速度大小相等,故链条的动能为
\begin{equation*}
	T = \frac12 mv^2 = \frac12 \rho \pi R v^2
\end{equation*}
直线段的重力势能为
\begin{equation*}
	V_1 = \rho R\theta g \cdot \left(-\frac12 R\theta\right) = -\frac12 \rho gR^2 \theta^2 
\end{equation*}
曲线段的重力势能为
\begin{equation*}
	V_2 = \int_C \rho gh \mathrm{d} l = \int_0^{\pi-\theta} \rho g R\sin \alpha R\mathrm{d} \alpha = \rho gR^2 (\cos \theta + 1)
\end{equation*}
根据机械能守恒可得
\begin{equation*}
	2\rho gR^2 = \frac12 \rho \pi R v^2 -\frac12 \rho gR^2 \theta^2 + \rho gR^2 (\cos \theta + 1)
\end{equation*}
由此同样可以解得链条处于任意角度处的速度为
\begin{equation*}
	v = 2\sqrt{\frac{Rg}{\pi}\left[\left(\frac{\theta}{2}\right)^2+ \sin^2 \frac{\theta}{2}\right]}
\end{equation*}
\end{solution}

\begin{example}
质量为$m$的质点沿一半球形光滑碗的内侧以初速度$v_0$沿水平方向运动,碗的内半径为$r$,初位置离碗边缘的高度为$h$。求质点上升到碗口而又不至于飞出碗外时$v_0$和$h$应满足的关系式。
\begin{figure}[htb]
\centering
\begin{asy}
	size(200);
	//第一章例8图
	real hd,hdd;
	pair O,x,y,z;
	hd = 131+25/60;
	hdd = 7+10/60;
	O = (0,0);
	x = (-sqrt(2)/4,-sqrt(14)/12);
	y = (sqrt(14)/4,-sqrt(2)/12);
	z = (0,2*sqrt(2)/3);
	draw(Label("$x$",EndPoint,W),(-1.5*x)--(1.5*x),Arrow);
	draw(Label("$y$",EndPoint),(-1.2*y)--(1.2*y),Arrow);
	draw(Label("$z$",EndPoint),(-1.2*z)--(0.5*z),Arrow);
	real r,vec;
	r = 1;
	pair cirxy(real theta){
		real xx,yy;
		xx = r*cos(theta);
		yy = r*sin(theta);
		return xx*x+yy*y;
	}
	pair ciryz(real theta){
		real yy,zz;
		yy = r*cos(theta);
		zz = r*sin(theta);
		return yy*y+zz*z;
	}
	pair cirxz(real theta){
		real xx,zz;
		xx = r*cos(theta);
		zz = r*sin(theta);
		return xx*x+zz*z;
	}
	real phi,phi2;
	pair Pcir(real theta){
		real xx,yy,zz;
		xx = r*cos(theta)*cos(phi);
		yy = r*cos(theta)*sin(phi);
		zz = r*sin(theta);
		return xx*x+yy*y+zz*z;
	}
	pair Bcir(real theta){
		real xx,yy,zz;
		xx = r*cos(phi2)*cos(theta);
		yy = r*cos(phi2)*sin(theta);
		zz = r*sin(phi2);
		return xx*x+yy*y+zz*z;
	}
	//画曲线
	draw(arc(O,r,180,360));
	draw(graph(cirxy,0,2*pi));
	draw(graph(ciryz,pi,1.72*pi),dashed);
	draw(graph(ciryz,1.72*pi,2*pi));
	draw(graph(cirxz,pi,1.62*pi),dashed);
	draw(graph(cirxz,1.62*pi,2*pi));
	phi = pi/4;
	draw(graph(Pcir,pi,1.62*pi),dashed);
	draw(graph(Pcir,1.62*pi,2*pi));
	draw(Pcir(pi)--Pcir(2*pi));
	phi2 = 2*pi-pi/4;
	pair P;
	P = Pcir(phi2);
	draw(graph(Bcir,-0.3*pi,0.5*pi));
	draw(graph(Bcir,0.5*pi,2*pi-0.3*pi),dashed);
	vec = 0.5;
	draw(Label("$\boldsymbol{G}$",EndPoint),P--(P-vec*z),red,Arrow);
	draw(Label("$\boldsymbol{N}$",EndPoint),P--(P-vec*Pcir(phi2)),red,Arrow);
	draw(Label("$\boldsymbol{v}_\phi$",EndPoint,E,black),P--(P-vec*sin(phi)*x+vec*cos(phi)*y),orange,Arrow);
	draw(Label("$\boldsymbol{v}_\theta$",EndPoint,S,black),P--(P+vec*sin(phi2)*cos(phi)*x+vec*sin(phi2)*cos(phi)*y-vec*cos(phi2)*z),orange,Arrow);
	dot(P);
\end{asy}
\caption{例\theexample}
\label{第一章例8图}
\end{figure}
\end{example}

\begin{solution}
建立球坐标系,考虑到$v_r = \dot{r} = 0$,则
\begin{equation*}
	\mbf{v} = v_\theta \mbf{e}_\theta + v_\phi \mbf{e}_\phi
\end{equation*}
弹力沿径向,故$\mbf{N} \cdot \mbf{v} = 0$,即系统中只有重力做功,机械能守恒。

而$\mbf{r} \times \mbf{N} = \mbf{0}$,$\mbf{e}_3 \cdot \left[\mbf{r} \times (-mg\mbf{e}_3)\right] = 0$,故$z$方向力矩为零,即$z$方向角动量守恒。计算$z$方向的角动量
\begin{equation*}
	L_z = \mbf{e}_3 \cdot \left[r\mbf{e}_r \times m(v_\theta\mbf{e}_\theta + v_\phi \mbf{e}_\phi)\right] = mr\mbf{e}_3 \cdot (v_\theta\mbf{e}_\phi - v_\phi\mbf{e}_\theta) = mr\sin \theta v_\phi
\end{equation*}
考虑到$\mbf{v}\big|_{z=-h} = v_0 \mbf{e}_\phi$,可以列出守恒方程
\begin{equation*}
	\begin{cases}
		\dfrac12 m(v_\theta^2 + v_\phi^2) + mgz = \dfrac12 mv_0^2 - mgh \\[1.5ex]
		mv_\phi \sqrt{r^2-z^2} = mv_0\sqrt{r^2-h^2}
	\end{cases}
\end{equation*}
将末状态取为$z=0$时,此时$v_\theta\big|_{z=0} = 0$,可有
\begin{equation*}
	\begin{cases}
		\dfrac12 mv_\phi^2\big|_{z=0} = \dfrac12 mv_0^2 - mgh \\
		mrv_\phi\big|_{z=0} = mv_0 \sqrt{r^2-h^2}
	\end{cases}
\end{equation*}
由此,可解出
\begin{equation*}
	v_0 = r\sqrt{\frac{2g}{h}}
\end{equation*}
即为$v_0$和$h$应满足的关系。
\end{solution}

\section{变质量系统}

\subsection{变质量系统的概念}

如果系统的质量或者组成系统的质点其中之一,或者两者同时随时间变化,这个系统就称为{\bf 变质量系统}。

在自然现象中,变质量系统很常见。例如浮冰由于融化而减少质量,又由于结冰或者落雪而增加质量。在工程技术中,最常提到的变质量系统就是火箭了。

对于变质量系统的研究,总是做这样的假设:系统分离或并入的质量是小量,而且每次并入或分离质量的时间间隔是小量。在这样的假设下,离开系统和进入系统的质量都是时间的连续可微的函数。

如果将系统的质量记作$M(t)$,将$t=0$时的质量记作$M_0$,则它随时间的变化规律为
\begin{equation}
	M(t) = M_0 - M_1(t) + M_2(t)
	\label{chapter1:变质量系统的质量关系}
\end{equation}
其中$M_1(t)$表示累计离开系统的质量,而$M_2(t)$表示累计进入系统的质量,$M_1,M_2$都是随时间的非递减的非负连续可微函数。

\subsection{变质量系统的动量定理和角动量定理}

设某个质点系相对惯性参考系$Oxyz$运动,研究相对$Oxyz$运动和变形的封闭曲面$S$。系统内的各个质点按照各自的运动可以进入或者离开曲面$S$围成的空间区域,那么曲面$S$围成的质点即组成一个变质量系统,将其动量用$\mbf{p}$表示。

设在时刻$t=t'$,系统的动量为$\mbf{p}$。而在$t=t''=t'+\Delta t$的时刻,系统的动量记作$\mbf{p}+\Delta \mbf{p}$。显然有
\begin{equation}
	\Delta \mbf{p} = \Delta \mbf{p}^*-\Delta \mbf{p}_1+\Delta \mbf{p}_2
	\label{chapter1:变质量系统的动量定理-初步}
\end{equation}
其中,$\Delta\mbf{p}_1$为在$\Delta t$时间内离开变质量系统的所有质点带走的动量,$\Delta\mbf{p}_2$为在这段时间内进入系统的所有质点带来的动量,而$\Delta\mbf{p}^*$则表示在这段时间内一直在系统内的所有质点本身的动量改变。

% \begin{figure}[htb]
% \centering
% \begin{asy}
	% size(300);
	% //变质量系统1
	% picture pic1;
	% pair O,F,v,u,m;
	% real rO,rm,lF,lv,lu;
	% O = (0,0);
	% rO = 2;
	% fill(pic1,yscale(2/3)*scale(rO)*unitcircle,yellow);
	% label(pic1,"$m$",O);
	% v = (1,0.7);
	% lv = 1.2;
	% draw(pic1,Label("$\boldsymbol{v}$",MidPoint,Relative(W)),v--v+lv*dir(30),Arrow);
	% F = (-1.3,-0.8);
	% lF = 1.2;
	% draw(pic1,Label("$\boldsymbol{F}$",MidPoint,Relative(W)),F-lF*dir(30)--F,Arrow);
	% rm = 0.7;
	% m = (3,0.4);
	% fill(pic1,shift(m)*yscale(2/3)*scale(rm)*unitcircle,green);
	% label(pic1,"$\Delta m$",m);
	% u = (-0.1,0.2)+m;
	% lu = 0.9;
	% draw(pic1,Label("$\boldsymbol{u}$",MidPoint,Relative(E)),u--u+lu*dir(135),Arrow);
	% label(pic1,"$t$",(0,-1.5),S);
	% add(pic1);
	% //变质量系统2
	% picture pic2;
	% pair O,F,v,u,m;
	% real rO,rm,lF,lv,lu;
	% O = (0,0);
	% rO = 2;
	% fill(pic2,yscale(2/3)*scale(rO)*unitcircle,yellow);
	% label(pic2,"$m'=m+\Delta m$",O);
	% v = (1,0.7);
	% lv = 1.2;
	% draw(pic2,Label("$\boldsymbol{v}'$",MidPoint,Relative(W)),v--v+lv*dir(30),Arrow);
	% F = (-1.3,-0.8);
	% lF = 1.2;
	% draw(pic2,Label("$\boldsymbol{F}'$",MidPoint,Relative(W)),F-lF*dir(30)--F,Arrow);
	% label(pic2,"$t'=t+\Delta t$",(0,-1.5),S);
	% add(shift((6.5,0))*pic2);
	% //draw(O--(9,0),invisible);
% \end{asy}
% \caption{变质量系统}
% \label{变质量系统}
% \end{figure}

设$\mbf{F}^{(e)}$是$t'$时刻作用在系统上的合外力,那么根据动量定理可得
\begin{equation}
	\dif{\mbf{p}^*}{t} = \mbf{F}^{(e)}
\end{equation}
在式\eqref{chapter1:变质量系统的动量定理-初步}两端除以$\Delta t$并取$\Delta t\to 0$的极限可得
\begin{equation}
	\dif{\mbf{p}}{t} = \mbf{F}^{(e)}+\mbf{F}^{(\text{ex})}
\end{equation}
其中$\mbf{F}^{(\text{ex})} = \mbf{F}_1+\mbf{F}_2$,而
\begin{equation}
	\mbf{F}_1 = -\lim_{\Delta t\to 0} \frac{\Delta \mbf{p}_1}{\Delta t},\quad \mbf{F}_2 = \lim_{\Delta t\to 0} \frac{\Delta \mbf{p}_2}{\Delta t}
\end{equation}
矢量$\mbf{F}_1$和$\mbf{F}_2$具有力的量纲,称为{\bf 反推力}。反推力$\mbf{F}_1$是由质点分离引起的,反推力$\mbf{F}_2$是由质点并入引起的。

设惯性参考系$Oxyz$中的固定参考点为$A$,则类似前面的讨论可得变质量系统的角动量定理
\begin{equation}
	\dif{\mbf{L}_A}{t} = \mbf{M}_A^{(e)} + \mbf{M}_A^{(\text{ex})}
\end{equation}
其中$\mbf{L}_A$为系统对$A$点的角动量,$\mbf{M}_A^{(e)}$为系统受到的对$A$点的合外力矩,而$\mbf{M}_A^{(\text{ex})} = \mbf{M}_{A1} + \mbf{M}_{A2}$是变质量系统的附加力矩,其中
\begin{equation}
	\mbf{M}_{A1} = -\lim_{\Delta t\to 0} \frac{\Delta \mbf{L}_{A1}}{\Delta t},\quad \mbf{M}_{A2} =  \lim_{\Delta t\to 0} \frac{\Delta \mbf{L}_{A2}}{\Delta t}
\end{equation}
这里的$\Delta \mbf{L}_{A1}$为在$\Delta t$时间内离开变质量系统的所有质点的角动量,$\Delta \mbf{L}_{A2}$为在这段时间内进入系统的所有质点的角动量。

% \begin{equation}
	% \dif{\mbf{p}_1}{t} = \lim_{\Delta t\to 0} \frac{\Delta \mbf{p}_1}{\Delta t},\quad \dif{\mbf{p}_2}{t} = \lim_{\Delta t\to 0} \frac{\Delta \mbf{p}_2}{\Delta t}
% \end{equation}

\subsection{变质量质点的运动微分方程}

作为变质量系统的一种特殊情况,现在来考虑变质量质点的运动微分方程。其质量变化同样满足式\eqref{chapter1:变质量系统的质量关系},即
\begin{equation}
	M(t) = M_0 - M_1(t) + M_2(t)
	\label{chapter1:变质量系统的质量关系-重写}
\end{equation}
如果记$\mbf{u}_1$为$t'$时刻离开变质量质点的微粒的速度,$\mbf{u}_2$为$t'$时刻并入质点的微粒的速度,那么精确到$\Delta t$和$\Delta M$的一阶小量,可有
\begin{equation}
	\Delta \mbf{p}_1 = \Delta M_1\mbf{u}_1,\quad \Delta \mbf{p}_2 = \Delta M_2\mbf{u}_2
\end{equation}	
由此可得
\begin{equation}
	\mbf{F}_1 = -\lim_{\Delta t\to 0} \frac{\Delta \mbf{p}_1}{\Delta t} = -\dif{M_1}{t}\mbf{u}_1,\quad \mbf{F}_2 = \lim_{\Delta t\to 0} \frac{\Delta \mbf{p}_2}{\Delta t} = \dif{M_2}{t}\mbf{u}_2
\end{equation}
设$\mbf{v}$为变质量质点的速度,则其动量为
\begin{equation}
	\mbf{p} = M\mbf{v}
\end{equation}
则有
\begin{equation}
	\dif{M}{t}\mbf{v}+M\dif{\mbf{v}}{t} = \mbf{F}^{(e)} - \dif{M_1}{t}\mbf{u}_1 + \dif{M_2}{t}\mbf{u}_2
	\label{chapter1:广义密歇尔斯基方程-初步}
\end{equation}
将式\eqref{chapter1:变质量系统的质量关系-重写}代入方程\eqref{chapter1:广义密歇尔斯基方程-初步}中可得
\begin{equation}
	M\dif{\mbf{v}}{t} = \mbf{F}^{(e)} - \dif{M_1}{t}(\mbf{u}_1-\mbf{v}) + \dif{M_2}{t}(\mbf{u}_2-\mbf{v})
	\label{chapter1:广义密歇尔斯基方程}
\end{equation}
方程\eqref{chapter1:广义密歇尔斯基方程}即为变质量质点的运动微分方程,称为{\bf 广义Meshcherskii方程}\footnote{Meshcherskii,密歇尔斯基,前苏联力学家。}。在方程\eqref{chapter1:广义密歇尔斯基方程}中,$\mbf{u}_1-\mbf{v}=\mbf{u}_{1r}$是分离微粒相对质点的速度,而$\mbf{u}_2-\mbf{v}=\mbf{u}_{2r}$是并入微粒相对质点的速度。

如果只有分离微粒而没有并入微粒,此时$M_2(t)\equiv 0$,此时方程\eqref{chapter1:广义密歇尔斯基方程}变为
\begin{equation}
	M\dif{\mbf{v}}{t} = \mbf{F}^{(e)} + \dif{M}{t}\mbf{u}_{1r}
	\label{chapter1:密歇尔斯基方程}
\end{equation}
方程\eqref{chapter1:密歇尔斯基方程}称为{\bf Meshcherskii方程}。由Meshcherskii方程\eqref{chapter1:Meshcherskii方程}可以看出,微粒分离等效于在质点上作用附加的反推力$\mbf{F}_1 = \dif{M}{t}\mbf{u}_{1r}$。

\begin{example}[重力场中的火箭]
在假设火箭竖直运动且喷气速度恒定,其所受外力只有恒定重力$\mbf{F}^{(e)} = -Mg\mbf{e}_3$的情况下求解火箭的运动。
\end{example}
\begin{solution}
相对火箭的喷气速度
\begin{equation*}
	\mbf{u}-\mbf{v} = -u_r\mbf{e}_3
\end{equation*}
此时火箭的运动方程即为Meshcherskii方程\eqref{chapter1:密歇尔斯基方程}
\begin{equation*}
	M\frac{\mathrm{d} v}{\mathrm{d} t} = -Mg - \dif{M}{t}u_r
\end{equation*}
化简为
\begin{equation*}
	\mathrm{d} v = -g\mathd t - u_r \frac{\mathd M}{M}
\end{equation*}
直接积分,即可得末速度
\begin{equation*}
	v_f = v_i + v_r \ln \frac{M_i}{M_f} - gt_f = v_i + v_r \ln \left(1+\frac{|\Delta M|}{M_f}\right) - gt_f
\end{equation*}
提高火箭末速度的方法:
\begin{itemize}
	\item 提高喷气速度;
	\item 用级连法提高燃料所占质量比;
	\item 提高燃烧速度,减少加速时间。
\end{itemize}
\end{solution}

\begin{example}
一轻绳跨过定滑轮的两端,一端挂着质量为$m$的重物,另一端和线密度为$\rho$的软链相连。开始时软链全部静止在地上,重物亦静止,后来重物下降,并将软链向上提起。问软链最后可提起多高?
\begin{figure}[htb]
\centering
\begin{asy}
	size(300);
	//第一章例10图
	pair O;
	real r,l1,l2,hand,a,T;
	O = (0,0);
	r = 1;
	l1 = 4;
	l2 = 6;
	T = 3;
	hand = 2;
	a = 0.6;
	draw(scale(r)*unitcircle);
	draw(O--(0,hand),linewidth(3bp));
	draw((-r,0)--(-r,-l1));
	draw((-r,0)--(-r-2*a,0));
	draw((-r,-l1)--(-r-2*a,-l1));
	draw(Label("$X$",MidPoint,W),(-r-1.5*a,0)--(-r-1.5*a,-l1),Arrows);
	fill(box((-r-a,-l1-a),(-r+a,-l1+a)),red);
	dot((-r,-l1));
	draw(Label("$T$",EndPoint,E),(-r,-l1)--(-r,-l1)+dir(90),Arrow);
	draw(Label("$Mg$",EndPoint,E),(-r,-l1)--(-r,-l1)+dir(-90),Arrow);
	draw((r,0)--(r,-l2+0.1));
	draw((r,-T)--(r,-l2+0.1)..(r+0.1,-l2)--(r+3*a,-l2),blue+linewidth(3bp));
	dot((r,-(T+l2)/2));
	draw(Label("$\rho gx$",EndPoint,W),(r,-(T+l2)/2)--(r,-(T+l2)/2)+dir(-90),Arrow);
	draw(Label("$T$",EndPoint,E),(r,-T)--(r,-T)+dir(90),Arrow);
	draw((r,-T)--(r+2*a,-T));
	draw((r,-l2)--(r+2*a,-l2));
	draw(Label("$x$",MidPoint,E),(r+1.5*a,-T)--(r+1.5*a,-l2),Arrows);
	//draw(O--(1.5,0),invisible);
\end{asy}
\caption{例\theexample}
\label{第一章例10图}
\end{figure}
\end{example}
\begin{solution}
地面上的软链在被提起时的速度为$u =0$,根据变质量系统的运动微分方程\eqref{chapter1:密歇尔斯基方程}即有
\begin{equation*}
	\rho x \ddot{x} + \rho \dot{x} \dot{x} = T - \rho gx
\end{equation*}
对左端物体有
\begin{equation*}
	M\ddot{X} = Mg - T
\end{equation*}
约束为
\begin{equation*}
	\dot{X} = \dot{x}
\end{equation*}
消去无关变量即有
\begin{equation*}
	\rho x \ddot{x} + \rho \dot{x} \dot{x} + M\ddot{x} = Mg - \rho g x
\end{equation*}
可化为
\begin{equation*}
	\frac{\mathrm{d}}{\mathrm{d} t} [(M+\rho x) \dot{x}] = (M-\rho x) g
\end{equation*}
即
\begin{equation*}
	\dot{x} \frac{\mathrm{d}}{\mathrm{d} x} [(M+\rho x) \dot{x}] = (M-\rho x) g
\end{equation*}
因此
\begin{equation*}
	\frac12 \frac{\mathrm{d}}{\mathrm{d} x} [(M+\rho x) \dot{x}]^2 = (M-\rho x)(M+\rho x) g
\end{equation*}
积分,并考虑到$\dot{x}\big|_{x=0} = 0$,即得
\begin{equation*}
	\frac12 \left[(M+\rho x)\dot{x}\right]^2 = \left(M^2 - \frac13 \rho^2 x^2\right) gx
\end{equation*}
当软链升到最高时,有$\dot{x}\big|_{t=t_{max}} = 0$,即有
\begin{equation*}
	\left(M^2 - \frac13 \rho^2 x^2_{max}\right) gx_{max} = 0
\end{equation*}
由此得最大高度
\begin{equation*}
	x_{max} = \frac{\sqrt{3}M}{\rho}
\end{equation*}
\end{solution}


%第二章——拉格朗日力学
\chapter{Lagrange动力学}

1788年,Lagrange发表名著《分析力学》。全书无图,藉助数学分析建立运动方程。

\section{约束与广义坐标}

\subsection{约束及其分类}

在运动过程中,质点的位置和速度受到的限制称为{\heiti 约束}。对于质点系,其中各质点的位矢为$\mbf{r}_i,\,i=1,2,\cdots,n$,其约束方程可以表示为
\begin{equation}
	f(\mbf{r}_1,\mbf{r}_2,\cdots,\mbf{r}_n;\dot{\mbf{r}}_1,\dot{\mbf{r}}_2,\cdots,\dot{\mbf{r}}_n;t) = 0
	\label{约束方程}
\end{equation}

约束可以分为以下两类:
\begin{itemize}
	\item {\heiti 几何约束}:与速度无关,仅对几何位形加以限制。即
	\begin{equation}
		f(\mbf{r}_1,\mbf{r}_2,\cdots,\mbf{r}_n;t) = 0
	\end{equation}
	一个几何约束减少一个独立坐标与一个独立速度分量。例如,单摆(如图\ref{单摆的几何约束}所示)的约束可以表示为
	\begin{equation*}
		|\mbf{r}|^2-l^2 = x^2+y^2-l^2 = 0
	\end{equation*}
\begin{figure}[htb]
\centering
\begin{minipage}[t]{0.45\textwidth}
\begin{asy}
	texpreamble("\usepackage{xeCJK}");
	texpreamble("\setCJKmainfont{SimSun}");
	usepackage("amsmath");
	import graph;
	import math;
	size(200);
	//单摆的几何约束
	pair O,x,y,r;
	O = (0,0);
	x = (4,0);
	y = (0,-4.5);
	r = (3,-4);
	draw(Label("$x$",EndPoint),O--x,Arrow);
	draw(Label("$y$",EndPoint),O--y,Arrow);
	label("$O$",O,W);
	filldraw(shift(r)*scale(0.2)*unitcircle,yellow,black);
	draw(Label("$\boldsymbol{r}$",MidPoint,Relative(E)),O--r,Arrow);
	draw(Label("$l$",MidPoint,Relative(W)),O--r,Arrow);
	//draw(O--(4.5,0),invisible);
\end{asy}
\caption{单摆的几何约束}
\label{单摆的几何约束}
\end{minipage}
\hspace{0.5cm}
\begin{minipage}[t]{0.45\textwidth}
\begin{asy}
	texpreamble("\usepackage{xeCJK}");
	texpreamble("\setCJKmainfont{SimSun}");
	usepackage("amsmath");
	import graph;
	import math;
	size(200);
	//纯滚动直立圆盘
	real hd,hdd;
	pair O,x,y,z,i,j,k;
	hd = 131+25/60;
	hdd = 7+10/60;
	O = (0,0);
	x = 1.2*dir(90+hd);
	y = 1.5*dir(-hdd);
	z = 1.5*dir(90);
	draw(Label("$x$",EndPoint),O--x,Arrow);
	draw(Label("$y$",EndPoint),O--y,Arrow);
	draw(Label("$z$",EndPoint),O--z,Arrow);
	label("$O$",O,NW);
	i = (-sqrt(2)/4,-sqrt(14)/12);
	j = (sqrt(14)/4,-sqrt(2)/12);
	k = (0,2*sqrt(2)/3);
	
	real x0,y0,r,phi;
	pair P;
	pair planecur(real xx){
		real yy = y0+1*tan(xx-x0);
		return xx*i+yy*j;
	}
	pair dplanecur(real xx){
		real yy = 1/((cos(xx-x0))**2);
		return xx*i+yy*j;
	}
	pair wheel(real theta){
		real xx,yy,zz;
		xx = x0+r*sin(theta)*cos(phi);
		yy = y0+r*sin(theta)*sin(phi);
		zz = r+r*cos(theta);
		return xx*i+yy*j+zz*k;
	}
	real getphi(real xx){
		real yy = 1/((cos(xx-x0))**2);
		return atan(yy/xx);
	}
	x0 = 1;
	y0 = 0.7;
	r = 0.5;
	phi = getphi(x0);
	unfill(graph(wheel,0,2*pi)--cycle);
	draw(graph(wheel,0,2*pi),linewidth(0.8bp));
	draw(graph(planecur,x0-1,x0+1));
	P = planecur(x0);
	dot(P);
	draw((P-0.7*dplanecur(x0))--(P+dplanecur(x0)),dashed);
	draw(P--P+r*k--wheel(-1.2));
	label("$(x,y)$",P+r*k,E);
	label("$\theta$",P+r*k,WSW);
	label("$\phi$",P-0.7*dplanecur(x0),2*S);
	
	//draw(O--(1.7,0),invisible);
\end{asy}
\caption{作曲线运动的纯滚动直立圆盘}
\label{作曲线运动的纯滚动直立圆盘}
\end{minipage}
\end{figure}
	\item {\heiti 运动约束}:涉及体系运动情况(即涉及质点的速度)。运动约束条件对坐标取值无限制,只对坐标变动有限制,不减少独立坐标个数。但一个运动约束减少一个独立速度分量。例如,作曲线运动的纯滚动直立圆盘(如图\ref{作曲线运动的纯滚动直立圆盘}所示)的约束可以表示为
	\begin{equation*}
		\begin{cases}
			\dot{x} - R\dot{\theta}\cos \phi = 0 \\
			\dot{y} - R\dot{\theta}\sin \phi = 0
		\end{cases}
	\end{equation*}
	圆盘可由任一指定初位形出发,到达任一指定末位形。但如果圆盘作直线运动,此时$\phi = \phi_0(\text{常数})$,则运动约束可积\footnote{在求解运动前。}(此时即称为{\heiti 可积运动约束}),即
	\begin{equation*}
		\begin{cases}
			x - R\theta\cos \phi_0 + C_1 = 0 \\
			y - R\theta\sin \phi_0 + C_2 = 0
		\end{cases}
	\end{equation*}
	可积运动约束实质上等价于几何约束。
\end{itemize}

几何约束与可积运动约束称为{\heiti 完整约束},一个完整约束方程同时减少一个独立坐标与一个独立速度分量。不可积运动约束则称为{\heiti 非完整约束},一个非完整约束方程只减少一个独立速度分量。如果一个体系的所有约束皆为完整约束,则该体系称为{\heiti 完整体系},否则称为{\heiti 非完整体系}。

\subsection{自由度}

体系可独立变动(并非取值)坐目标个数,即独立速度分量个数,称为体系的{\heiti 自由度}。

设某$n$质点体系具有完整约束
\begin{equation}
	f_j(\mbf{r}_1,\mbf{r}_2,\cdots,\mbf{r}_n;t) = 0,\quad j = 1,2,\cdots,k
\end{equation}
和非完整约束
\begin{equation}
	f_{j'}(\mbf{r}_1,\mbf{r}_2,\cdots,\mbf{r}_n;\dot{\mbf{r}}_1,\dot{\mbf{r}}_2,\cdots,\dot{\mbf{r}}_n;t) = 0,\quad j' = 1,2,\cdots,k'
\end{equation}
则体系的独立坐标数$s = 3n-k$,自由度$f = 3n-k-k'$。对于完整体系,$k'=0$,自由度与独立坐标数相等$f = s$。对于非完整体系,$k'>0$,则有$f < s$。

\subsection{广义坐标}

以双摆为例(如图\ref{双摆系统}所示),其约束为
\begin{equation*}
	\begin{cases}
		x_1^2 + y_1^2 - l_1^2 = 0 \\
		(x_2-x_1)^2 + (y_2-y_1)^2 - l_2^2 = 0
	\end{cases}
\end{equation*}
因此,体系的自由度为$2$,可以在四个坐标中任选独立的两个确定位形。但如此做并不方便,因此任选两个独立参量确定位形,如两个角度$(\theta_1,\theta_2)$,其转换关系为
\begin{equation*}
	\begin{cases}
		x_1 = l_1 \cos \theta_1 \\ 
		y_1 = l_1 \sin \theta_1 \\
		x_2 = l_1 \cos \theta_1 + l_2 \cos \theta_2 \\
		y_2 = l_1 \sin \theta_1 + l_2 \sin \theta_2 
	\end{cases}
\end{equation*}
\begin{figure}[htb]
\centering
\begin{asy}
	texpreamble("\usepackage{xeCJK}");
	texpreamble("\setCJKmainfont{SimSun}");
	usepackage("amsmath");
	import graph;
	import math;
	size(200);
	//双摆系统
	pair O,x,y,P1,P2;
	real theta1,theta2,l1,l2,r;
	O = (0,0);
	x = (4,0);
	y = (0,-4);
	l1 = 2;
	l2 = 3;
	theta1 = 30;
	theta2 = 50;
	draw(Label("$x$",EndPoint),O--x,Arrow);
	draw(Label("$y$",EndPoint),O--y,Arrow);
	P1 = l1*dir(theta1-90);
	P2 = P1 + l2*dir(theta2-90);
	r = 0.1;
	fill(shift(P1)*scale(r)*unitcircle,black);
	fill(shift(P2)*scale(r)*unitcircle,black);
	draw(Label("$l_1$",MidPoint,Relative(W)),O--P1);
	draw(Label("$l_2$",MidPoint,Relative(W)),P1--P2);
	draw(P1--P1+1.5*dir(-90),dashed);
	label("$\theta_1$",0.6*dir(theta1/2-90));
	label("$\theta_2$",P1+0.5*dir(theta2/2-90));
	label("$(x_1,y_1)$",P1,NE);
	label("$(x_2,y_2)$",P2+(0.1,0),E);
	
	//draw(O--(4.5,0),invisible);
\end{asy}
\caption{双摆系统}
\label{双摆系统}
\end{figure}

此处,确定力学体系位形的任意一组独立变量,称为{\heiti 广义坐标}。各质点坐标与广义坐标间的函数关系称为{\heiti 坐标变换方程},其使得约束方程自动满足。广义坐目标引入使完整约束方程与不独立坐标一起消去。

广义坐目标特点:
\begin{enumerate}
	\item 可任选,一般要求相互独立,并能确定体系的位形;
	\item 不必具有长度量纲或明显的几何意义;
	\item 属于整个体系,而非个别质点,因而任一广义坐目标变化一般会引起全部质点的位移,反之亦然。
\end{enumerate}

由广义坐标张成的$s$维抽象空间,称为{\heiti 位形空间}。体系的位形及其随时间的演化分别对应该空间的一点与一条曲线。

\section{虚位移与约束力}

\subsection{虚位移}

实时的、约束允许的任意虚拟无限小位置变动,称为{\heiti 虚位移},记作$\delta \mbf{r}$以区别于实位移$\mathrm{d} \mbf{r}$。力与虚位移之内积相应地称为{\heiti 虚功}。
\begin{figure}[htb]
\centering
\begin{asy}
	texpreamble("\usepackage{xeCJK}");
	texpreamble("\setCJKmainfont{SimSun}");
	usepackage("amsmath");
	import graph;
	import math;
	size(200);
	//约束在曲面上的质点
	pair A[],O,a,b,ns,g1,g2;
	O = (0,0);
	a = (2,0);
	b = (0.5,1);
	A[1] = O-a-b;
	A[2] = O+a-b;
	A[3] = O+a+b;
	A[4] = O-a+b;
	A[5] = (A[1]+A[2])/2+(0,0.3);
	A[6] = (A[2]+A[3])/2+(0,0.2);
	A[7] = (A[3]+A[4])/2+(0,0.3);
	A[8] = (A[4]+A[1])/2+(0,0.2);
	path surf,row,col;
	surf = A[1]..A[5]..A[2]--A[2]..A[6]..A[3]--A[3]..A[7]..A[4]--A[4]..A[8]..A[1]--cycle;
	ns = (A[5]+A[7])/2+(0,0.2);
	col = A[5]..ns..A[7];
	row = A[6]..ns..A[8];
	draw(surf);
	draw(row,dashed);
	draw(col,dashed);
	g1 = 1*dir(row,1);
	g2 = 0.6*dir(col,1);
	draw(ns-g1-g2--ns+g1-g2--ns+g1+g2--ns-g1+g2--cycle);
	draw(Label("$\boldsymbol{n}$",EndPoint),ns--ns+0.8*dir(90),Arrow);
	draw(Label("$\delta\boldsymbol{r}$",EndPoint),ns--ns+0.5*dir(30),red,Arrow);
	//draw(O--A[3]+(0.4,0),invisible);
\end{asy}
\caption{约束在曲面上的质点}
\label{约束在曲面上的质点}
\end{figure}

考虑于约束在曲面(曲面可移动,可变形)上运动的质点,曲面的方程表示为$f(\mbf{r},t)=0$,则$t$时刻该质点的位矢满足方程
\begin{equation*}
	f(\mbf{r},t) = 0
\end{equation*}
设质点具有虚位移$\delta \mbf{r}$则有\footnote{此处需考虑到虚位移是“实时”的无限小位置变动。}
\begin{equation*}
	f(\mbf{r}+\delta \mbf{r},t) = 0
\end{equation*}
所以
\begin{equation*}
	0 = f(\mbf{r}+\delta \mbf{r},t)-f(\mbf{r},t) = \frac{\pl f}{\pl \mbf{r}} \cdot \delta \mbf{r}
\end{equation*}
由于$\dfrac{\pl f}{\pl \mbf{r}} \parallel \mbf{n}$\footnote{此处,算符$\dfrac{\pl}{\pl \mbf{r}}$定义为\begin{equation*} \frac{\pl}{\pl \mbf{r}} = \mbf{e}_1 \frac{\pl}{\pl x}+\mbf{e}_2 \frac{\pl}{\pl y}+\mbf{e}_3 \frac{\pl}{\pl z} = \bnb \end{equation*}后文不再特别说明。},所以有$\delta \mbf{r} \perp \mbf{n}$。此时虚位移即为质点所在处曲面切平面内的无限小位移。

对于实位移则有
\begin{equation*}
	\begin{cases}
		f(\mbf{r},t) = 0 \\
		f(\mbf{r}+\mathrm{d} \mbf{r},t+\mathrm{d} t) = 0
	\end{cases}
\end{equation*}
所以
\begin{equation*}
	0 = f(\mbf{r}+\mathrm{d} \mbf{r},t+\mathrm{d} t) - f(\mbf{r},t) = \frac{\pl f}{\pl \mbf{r}} \cdot \mathrm{d}\mbf{r} + \frac{\pl f}{\pl t} \mathrm{d} t
\end{equation*}
即有
\begin{equation*}
	\frac{\pl f}{\pl \mbf{r}} \cdot \mathrm{d} \mbf{r} = - \frac{\pl f}{\pl t} \mathrm{d} t
\end{equation*}
如果约束为{\heiti 稳定约束}(约束方程不显含时间),则有$\dfrac{\pl f}{\pl t} = 0$,所以
\begin{equation*}
	\frac{\pl f}{\pl \mbf{r}} \cdot \mathrm{d} \mbf{r} = 0
\end{equation*}
即$\mathrm{d} \mbf{r} \in \{\delta \mbf{r}\}$,实位移是全体虚位移中的一个。如果约束为{\heiti 不稳定约束}(约束方程显含时间),则有$\dfrac{\pl f}{\pl t} \neq 0$,所以
\begin{equation*}
	\frac{\pl f}{\pl \mbf{r}} \cdot \mathrm{d} \mbf{r} \neq 0
\end{equation*}
即$\mathrm{d} \mbf{r} \notin \{\delta \mbf{r}\}$,实位移不在虚位移集合中。

引入等时变分$\delta$运算,其运算规则等同于微分运算$\mathrm{d}$,但作用于时间时结果为零,即$\delta t=0$。考虑
\begin{equation*}
	\delta f = \frac{\pl f}{\pl \mbf{r}} \cdot \delta \mbf{r} + \frac{\pl f}{\pl t} \delta t = \frac{\pl f}{\pl \mbf{r}} \cdot \delta \mbf{r} = 0
\end{equation*}
可知,位矢的等时变分即为虚位移。

对于完整约束
\begin{equation*}
	f_j(\mbf{r}_1,\mbf{r}_2,\cdots,\mbf{r}_n;t) = 0,\quad j = 1,2,\cdots,k
\end{equation*}
则有
\begin{equation*}
	\delta f_j = \sum_{i=1}^n \frac{\pl f_j}{\pl \mbf{r}_i} \cdot \delta \mbf{r}_i = 0,\quad j = 1,2,\cdots,k
\end{equation*}
即各虚位移分量线性相关,$k$个方程共减少$k$独立分量。对于线性非完整约束(只含有速度一次项)
\begin{equation*}
	\sum_{i=1}^n \mbf{A}_{ji}(\mbf{r},t) \cdot \dot{\mbf{r}}_i + A_{j0}(\mbf{r},t) = 0,\quad j=1,2,\cdots,k'
\end{equation*}
或者
\begin{equation*}
	\sum_{i=1}^n \mbf{A}_{ji}(\mbf{r},t) \cdot \mathrm{d} \mbf{r}_i + A_{j0}(\mbf{r},t) \mathrm{d}t = 0,\quad j=1,2,\cdots,k'
\end{equation*}
则有
\begin{equation*}
	\sum_{i=1}^n \mbf{A}_{ji}(\mbf{r},t) \cdot \delta \mbf{r}_i = 0,\quad j=1,2,\cdots,k'
\end{equation*}
即$k'$个方程共减少$k'$个独立虚位移分量。因此,系统的独立虚位移分量(即坐标变分)数为$3n-k-k'=f$。体系自由度即等于独立坐标变分个数。

\subsection{约束力}

起约束作用的物体对被约束质点施加的作用力$\mbf{R}_i$称为{\heiti 约束力},而体系质点除约束力之外的受力$\mbf{F}_i$称为{\heiti 主动力}。一般情况下,主动力是已知力,约束力是未知力,由主动力和质点运动情况决定。

约束力总虚功为零的约束称为{\heiti 理想约束},下列几种常见的约束都是理想约束。
\begin{enumerate}
	\item 质点沿光滑曲面运动,如图\ref{质点沿光滑曲面运动约束}所示。此时约束力$\mbf{R}$沿曲面在该点的法线方向,而虚位移$\delta \mbf{r}$根据上节讨论则在切平面内,即有
	\begin{equation*}
		\mbf{R} \cdot \delta \mbf{r} = 0
	\end{equation*}
	\item 刚性联结的两质点(包括刚体中任意两点),如图\ref{刚性联结的两质点约束}所示。此时约束力$\mbf{R}_1$和$\mbf{R}_2$沿联机方向,根据Newton第三定律可有$\mbf{R}_2=-\mbf{R}_1$。考虑到
	\begin{equation*}
		\mbf{r}_{12} \cdot \mbf{r}_{12} = l^2
	\end{equation*}
	所以
	\begin{equation*}
		2\mbf{r}_{12} \cdot \delta \mbf{r}_{12} = 0
	\end{equation*}
	即有
	\begin{equation*}
		\mbf{R}_1 \cdot \delta \mbf{r}_1 + \mbf{R}_2 \cdot \delta \mbf{r}_2 = \mbf{R}_1 \cdot (\delta \mbf{r}_1 - \delta \mbf{r}_2) = \mbf{R}_1 \cdot \delta \mbf{r}_{12} = 0
	\end{equation*}
	此处考虑到了$\mbf{R}_1 \parallel \mbf{r}_{12}$。
	\item 两个物体以完全粗糙的表面相接触(不滑动,只能作纯滚动),如图\ref{两个物体以完全粗糙的表面相接触约束}所示。此时接触点上的约束力满足$\mbf{R}_2 =- \mbf{R}_1$,相对速度$\mbf{v}_2-\mbf{v}_1= \mbf{0}$,即有$\delta \mbf{r}_2-\delta \mbf{r}_1 = \mbf{0}$,因此
	\begin{equation*}
		\mbf{R}_1 \cdot \delta \mbf{r}_1 + \mbf{R}_2 \cdot \delta \mbf{r}_2 = \mbf{R}_1 \cdot (\delta \mbf{r}_1 - \delta \mbf{r}_2) = 0
	\end{equation*}
	\item 两个质点以柔软而不可伸长的绳子相联结,如图\ref{两个质点以柔软而不可伸长的绳子相联结约束}所示。此时约束力$\mbf{R}_1$和$\mbf{R}_2$都沿相应质点处的绳子方向,且大小相等记作$T$。将两个质点的虚位移$\delta \mbf{r}_1$和$\delta \mbf{r}_2$按沿绳和正交于绳的方向进行分解,将沿绳方向的虚位移记作$\delta \mbf{r}_1^\parallel = \delta l_1\mbf{e}_1$和$\delta \mbf{r}_2^\parallel = \delta l_2 \mbf{e}_2$。正交于绳方向的虚位移对应的虚功自然为零,考虑沿绳方向虚位移对应的虚功
	\begin{equation*}
		\mbf{R}_1 \cdot \delta \mbf{r}_1 + \mbf{R}_2 \cdot \delta \mbf{r}_2 = T\delta l_1 + T\delta l_2
	\end{equation*}
	由于绳不可伸长,所以有$\delta l_1 + \delta l_2 = 0$,故有
	\begin{equation*}
		\mbf{R}_1 \cdot \delta \mbf{r}_1 + \mbf{R}_2 \cdot \delta \mbf{r}_2 = \mbf{0}
	\end{equation*}
\end{enumerate}
\begin{figure}[htb]
\centering
\begin{minipage}[t]{0.45\textwidth}
\centering
\begin{asy}
	texpreamble("\usepackage{xeCJK}");
	texpreamble("\setCJKmainfont{SimSun}");
	usepackage("amsmath");
	import graph;
	import math;
	size(200);
	//质点沿光滑曲面运动约束
	pair O,A[],vec,P,tan;
	path surf;
	O = (0,0);
	A[1] = (-0.7,-1);
	A[2] = O;
	A[3] = (1.2,0.6);
	surf = A[1]..A[2]..A[3];
	draw(surf,linewidth(0.8bp));
	vec = 0.5*(0.1,-0.06);
	int imax=30;
	for(int i=1;i<imax;i=i+1){
		P = relpoint(surf,1./imax*i);
		draw(P--P+vec);
	}
	label("$P$",O+vec,SE);
	tan = 0.9*dir(surf,1);
	draw(Label("$\delta\boldsymbol{r}$",EndPoint),O--O+tan,Arrow);
	draw(Label("$\boldsymbol{R}$",EndPoint),O--O+rotate(90)*tan,Arrow);
	dot(O);
\end{asy}
\caption{质点沿光滑曲面运动}
\label{质点沿光滑曲面运动约束}
\end{minipage}
\hspace{1cm}
\begin{minipage}[t]{0.45\textwidth}
\centering
\begin{asy}
	texpreamble("\usepackage{xeCJK}");
	texpreamble("\setCJKmainfont{SimSun}");
	usepackage("amsmath");
	import graph;
	import math;
	size(200);
	//刚性联结的两质点约束
	pair O,P1,P2;
	O = (0,0);
	P1 = (0,1.5);
	P2 = (0.8,1);
	label("$O$",O,W);
	dot(P1);
	dot(P2);
	label("$P_1$",P1,N);
	label("$P_2$",P2,N);
	draw(Label("$\boldsymbol{r}_1$",MidPoint,Relative(W)),O--P1,Arrow);
	draw(Label("$\boldsymbol{r}_2$",MidPoint,Relative(E)),O--P2,Arrow);
	draw(Label("$\boldsymbol{r}$",MidPoint,Relative(W)),P2--P1,Arrow);
	draw(Label("$\boldsymbol{R}_1$",EndPoint),P1--interp(P1,P2,-0.6),Arrow);
	draw(Label("$\boldsymbol{R}_2$",EndPoint),P2--interp(P2,P1,-0.6),Arrow);
\end{asy}
\caption{刚性联结的两质点}
\label{刚性联结的两质点约束}
\end{minipage}
\hspace{0.4cm}

\begin{minipage}[t]{0.45\textwidth}
\centering
\begin{asy}
	texpreamble("\usepackage{xeCJK}");
	texpreamble("\setCJKmainfont{SimSun}");
	usepackage("amsmath");
	import graph;
	import math;
	size(200);
	//两个物体以完全粗糙的表面相接触约束
	pair O,O1,A[],B[],vec,P,Q,tan;
	O = (0,0);
	O1 = (2,0);
	A[1] = (0.2,0);
	A[2] = (0.3,0.1);
	A[3] = (0.5,0.6);
	A[4] = (1,0.8);
	A[5] = (1.7,0.3);
	A[6] = (1.65,0);
	path bodyA,bodyB;
	bodyA = A[1]..A[2]..A[3]..A[4]..A[5]..A[6];
	draw(bodyA,linewidth(0.8bp));
	draw(O--O1);
	int imax = 25;
	vec = 0.35*(-0.1,-0.2);
	for(int i=1;i<imax;i=i+1){
		P = interp(O,O1,1./imax*i);
		draw(P--P+vec);
	}
	P = relpoint(bodyA,0.65);
	tan = 0.5*reldir(bodyA,0.65);
	Q = P+0.7*(0.2,0.5);
	label("$P$",P,SW);
	draw(shift(Q)*scale(length(Q-P))*unitcircle,linewidth(0.8bp));
	draw(Label("$\delta \boldsymbol{r}_1$",EndPoint,NNE),P--P+tan,Arrow);
	draw(Label("$\delta \boldsymbol{r}_2$",EndPoint,S),P--P+tan,Arrow);
	draw(Label("$\boldsymbol{R}_1$",EndPoint),P--P-0.5*dir(80),Arrow);
	draw(Label("$\boldsymbol{R}_2$",EndPoint),P--P+0.5*dir(80),Arrow);
\end{asy}
\caption{两个物体以完全粗糙的表面相接触}
\label{两个物体以完全粗糙的表面相接触约束}
\end{minipage}
\hspace{1cm}
\begin{minipage}[t]{0.45\textwidth}
\centering
\begin{asy}
	texpreamble("\usepackage{xeCJK}");
	texpreamble("\setCJKmainfont{SimSun}");
	usepackage("amsmath");
	import graph;
	import math;
	size(200);
	//两个质点以柔软而不可伸长的绳子相联结约束
	pair O,P1,P2,tan1,tan2;
	real r,l1,l2;
	path cyl;
	O = (0,0);
	r = 0.2;
	cyl = shift(O)*scale(r)*unitcircle;
	filldraw(cyl,blue,black);
	P1 = relpoint(cyl,0.45);
	tan1 = reldir(cyl,0.45);
	P2 = relpoint(cyl,0.22);
	tan2 = -reldir(cyl,0.22);
	l1 = 1.5;
	l2 = 1.8;
	draw(P1--P1+l1*tan1);
	draw(P2--P2+l2*tan2);
	draw(Label("$\boldsymbol{R}_1$",EndPoint,E),P1+l1*tan1--P1+(l1-0.5)*tan1,Arrow);
	draw(Label("$\delta \boldsymbol{r}_1^\parallel$",EndPoint),P1+l1*tan1--P1+(l1+0.5)*tan1,Arrow);
	draw(Label("$\delta \boldsymbol{r}_1^\perp$",EndPoint),P1+l1*tan1--P1+l1*tan1+rotate(90)*0.5*tan1,Arrow);
	draw(Label("$\boldsymbol{R}_2$",EndPoint,N),P2+l2*tan2--P2+(l2-0.5)*tan2,Arrow);
	draw(Label("$\delta \boldsymbol{r}_2^\parallel$",EndPoint),P2+l2*tan2--P2+(l2+0.5)*tan2,Arrow);
	draw(Label("$\delta \boldsymbol{r}_2^\perp$",EndPoint),P2+l2*tan2--P2+l2*tan2+rotate(90)*0.5*tan2,Arrow);
	dot(P1+l1*tan1);
	dot(P2+l2*tan2);
\end{asy}
\caption{两个质点以柔软而不可伸长的绳子相联结}
\label{两个质点以柔软而不可伸长的绳子相联结约束}
\end{minipage}
\end{figure}

所有约束皆为理想约束的系统称为{\heiti 理想系统},即理想系统需满足
\begin{equation*}
	\sum_{i=1}^n \mbf{R}_i \cdot \delta \mbf{r}_i = 0
\end{equation*}
对于非理想体系,可将摩擦力等虚功非零的约束力看作未知的主动力,而认为约束仍然是理想的,缺少的方程则由实验定律来补充。

下面可以看到,虚位移和虚功的引入将使未知的约束力自动消去。

\section{分析静力学的虚功原理}

\subsection{虚功原理}

如果理想体系处于静平衡状态,其中每个质点所受合力为零,即
\begin{equation*}
	\mbf{F}_i + \mbf{R}_i = \mbf{0},\quad i = 1,2,\cdots,n
\end{equation*}
由此可有
\begin{equation*}
	\sum_{i=1}^n (\mbf{F}_i+\mbf{R}_i) \cdot \delta \mbf{r}_i = 0
\end{equation*}
由于体系是理想的,故约束力的总虚功为零,由此可有
\begin{equation}
	\sum_{i=1}^n \mbf{F}_i \cdot \delta \mbf{r}_i = 0
	\label{虚功原理}
\end{equation}
式\eqref{虚功原理}表示的关系称为{\heiti 虚功原理}。即,平衡位形处,对任意虚位移,主动力的总虚功为零。

\subsection{广义虚位移和广义力}

约束\footnote{在静平衡条件下,所有的约束都是完整约束。}使得各质点的虚位移之间彼此不相互独立,设体系自由度为$s$,则可以引入$s$个广义坐标表征体系的位形,坐标转换关系表示为
\begin{equation}
	\mbf{r}_i = \mbf{r}_i(q_1,q_2,\cdots,q_s),\quad i = 1,2,\cdots,n
\end{equation}
由此可有
\begin{equation}
	\delta \mbf{r}_i = \sum_{\alpha=1}^s \frac{\pl \mbf{r}_i}{\pl q_\alpha} \delta q_\alpha
\end{equation}
此处$\delta q_\alpha$称为{\heiti 广义虚位移}。由此可有
\begin{equation*}
	\sum_{i=1}^n \mbf{F}_i \cdot \delta \mbf{r}_i = \sum_{\alpha=1}^s \left(\sum_{i=1}^n \mbf{F}_i \cdot \frac{\pl \mbf{r}_i}{\pl q_\alpha}\right)\delta q_\alpha
\end{equation*}
定义
\begin{equation}
	Q_\alpha = \sum_{i=1}^n \mbf{F}_i \cdot \frac{\pl \mbf{r}_i}{\pl q_\alpha}
\end{equation}
称为与广义坐标$q_\alpha$对应的广义主动力,则虚功原理可以用广义坐标表示为
\begin{equation}
	\sum_{i=1}^s Q_\alpha \delta q_\alpha = 0
	\label{虚功原理的广义表示}
\end{equation}

\subsection{完整系的平衡条件}

广义坐标使全部完整约束方程自动满足,所有广义坐标可以独立取值与变动,因此完整系的所有广义虚位移独立,因此由式\eqref{虚功原理的广义表示}可得完整系的平衡条件
\begin{equation}
	Q_\alpha\big|_{\mbf{q}_0} = 0,\quad \alpha =1,2,\cdots,s
\end{equation}
即,完整系平衡位形处,各广义主动力分量为零。

如果系统为保守体系,所有主动力均对应有势能
\begin{equation}
	\mbf{F}_i = -\frac{\pl V}{\pl \mbf{r}_i}(\mbf{r}_i),\quad i=1,2,\cdots,n
\end{equation}
则有
\begin{equation*}
	V = V(\mbf{r}(\mbf{q},t))
\end{equation*}
此时,广义主动力可以表示为
\begin{equation}
	Q_\alpha = -\sum_{i=1}^n \frac{\pl V}{\pl \mbf{r}_i} \cdot \frac{\pl \mbf{r}_i}{\pl q_\alpha} = -\frac{\pl V}{\pl q_\alpha}
\end{equation}
此时平衡条件可以表示为
\begin{equation}
	\frac{\pl V}{\pl q_\alpha}\bigg|_{\mbf{q}_0} = 0,\quad \alpha = 1,2,\cdots,s
\end{equation}
即完整保守体系在平衡位形处势能取驻值。可以证明,势能取极小值,平衡为{\heiti 稳定平衡};势能取极大值,平衡为{\heiti 不稳定平衡};势能取常数值,平衡为{\heiti 随遇平衡}。

\begin{example}
质量为$m$,固有半径为$a$,弹性系数为$k$的弹性圈置于半顶角$\alpha$的光滑直立圆锥上,求弹性圈的平衡半径与张力。\label{第二章例1}
\begin{figure}[htb]
\centering
\begin{asy}
	texpreamble("\usepackage{xeCJK}");
	texpreamble("\setCJKmainfont{SimSun}");
	usepackage("amsmath");
	import graph;
	import math;
	size(200);
	//第二章例1图
	pair O,x,y;
	real h,r,x0;
	O = (0,0);
	x = (1,0);
	y = (0,0.5);
	h = 1.7;
	r = 0.8;
	x0 = 0.6;
	draw(Label("$x$",EndPoint),-x--x,Arrow);
	draw(Label("$y$",EndPoint),(0,-h)--y,Arrow);
	draw(O--(r,-h)--(-r,-h)--cycle);
	label("$\alpha$",(0,-0.25*h),E);
	draw(interp(O,(-r,-h),x0)--interp(O,(r,-h),x0),red+linewidth(1bp));
	label("$x_0$",(0,-x0*h),SE);
	//draw(O--x+(0.3,0),invisible);
\end{asy}
\caption{例\theexample}
\label{第二章例1图}
\end{figure}
\end{example}
\begin{solution}
光滑锥面的完整约束为
\begin{equation*}
	y = -x\cot \alpha
\end{equation*}
主动力包括重力和弹性力,均为保守力。广义坐标取为$x$,则体系的势能表示为
\begin{equation*}
	V(x) = \frac12 k\left[2\pi(x-a)\right]^2 - mgx\cot \alpha
\end{equation*}
平衡条件为
\begin{equation*}
	\frac{\pl V}{\pl x}\bigg|_{x_0} = 4\pi^2k(x_0-a)-mg\cot\alpha = 0
\end{equation*}
解得平衡位置为
\begin{equation*}
	x_0 = a+\frac{mg\cot\alpha}{4\pi^2k}
\end{equation*}
平衡时的弹性力为
\begin{equation*}
	T = 2\pi k(x_0-a) = \frac{mg\cot \alpha}{2\pi}
\end{equation*}
\end{solution}

\begin{example}
重新设计例\ref{第二章例1}中旋转面的形状,使弹性圈可平衡于任意高度。
\end{example}
\begin{solution}
设旋转面由曲线$y=f(x)$绕$y$轴旋转形成,此时系统的势能
\begin{equation*}
	V(x) = \frac12 k\left[2\pi(x-a)\right]^2 + mgf(x)
\end{equation*}
随遇平衡要求$V(x) = \text{常数}$,即
\begin{equation*}
	\frac12 k\left[2\pi(x-a)\right]^2 + mgf(x) = C
\end{equation*}
由此有
\begin{equation*}
	f(x) = C-\frac{2\pi^2 k}{mg}(x-a)^2
\end{equation*}
\end{solution}

\begin{example}
将例\ref{第二章例1}中的弹性圈改为刚性圈,求其在圆锥面上平衡时的张力。
\end{example}
\begin{solution}
此问题有两个约束:约束$y=-x\cot\alpha$,约束力为正压力;约束$x=a$,约束力为张力。虚功原理只能求平衡位形,而不涉及约束力。

因此,此处解除第二个约束,将张力视为主动力。问题归结为已知平衡位形,求部分未知主动力。根据虚功原理,有
\begin{equation*}
	-T(2\pi \delta x) - mg\delta y = 0
\end{equation*}
根据约束,可有
\begin{equation*}
	\delta y = -\delta x \cot \alpha
\end{equation*}
所以有
\begin{equation*}
	(mg\cot \alpha - 2\pi T) \delta x = 0
\end{equation*}
$\delta x$是独立的广义虚位移,因此有
\begin{equation*}
	mg\cot \alpha - 2\pi T = 0
\end{equation*}
所以,张力为
\begin{equation*}
	T = \frac{mg\cot \alpha}{2\pi}
\end{equation*}
\end{solution}

\section{d'Alambert原理与Lagrange方程}

\subsection{d'Alambert原理}

对质点系中的每个质点应用Newton第二定律,可有
\begin{equation*}
	\mbf{F}_i + \mbf{R}_i = m_i \ddot{\mbf{r}}_i,\quad i=1,2,\cdots,n
\end{equation*}
将上式整理为
\begin{equation*}
	\mbf{F}_i + \mbf{R}_i - m_i \ddot{\mbf{r}}_i = \mbf{0},\quad i=1,2,\cdots,n
\end{equation*}
将上式中的$-m_i \ddot{\mbf{r}}_i$看作一个主动力,称为{\heiti d'Alambert惯性力},即主动力、约束力与惯性力构成平衡力系\footnote{这种将动力学问题转化为静力学问题考虑的方法,称为{\heiti 动静法}。},由此根据虚功原理可以得到
\begin{equation}
	\sum_{i=1}^n(\mbf{F}_i - m_i \ddot{\mbf{r}}_i) \cdot \delta \mbf{r}_i = 0
	\label{d'Alambert原理}
\end{equation}
式\eqref{d'Alambert原理}表示的关系称为{\heiti d'Alambert原理}。即理想力学体系的实际运动使任意虚位移下的主动力与惯性力的总虚功为零——此为与Newton定律等价的力学基本原理。

\subsection{理想完整系的Lagrange方程}

在理想完整体系中,质点位矢与广义坐标之间的坐标转换关系为
\begin{equation*}
	\mbf{r}_i = \mbf{r}_i(\mbf{q},t),\quad i=1,2,\cdots,n
\end{equation*}
所以
\begin{equation*}
	\delta \mbf{r}_i = \sum_{\alpha=1}^s \frac{\pl \mbf{r}_i}{\pl q_\alpha} \delta q_\alpha
\end{equation*}
质点的速度为
\begin{equation}
	\dot{\mbf{r}}_i = \sum_{\alpha=1}^s \frac{\pl \mbf{r}_i}{\pl q_\alpha} \dot{q}_\alpha + \frac{\pl \mbf{r}_i}{\pl t} = \dot{\mbf{r}}_i(\mbf{q},\dot{\mbf{q}},t)
	\label{质点的速度}
\end{equation}
式中,$\dot{q}_\alpha$称为{\heiti 广义速度}。在式\eqref{质点的速度}两端的微分,可得
\begin{align}
	\mathrm{d} \dot{\mbf{r}}_i & = \sum_{\alpha=1}^s \mathrm{d} \left(\frac{\pl \mbf{r}_i}{\pl q_\alpha}\right) \dot{q}_\alpha + \sum_{\alpha=1}^s \frac{\pl \mbf{r}_i}{\pl q_\alpha} \mathrm{d} \dot{q}_\alpha + \mathrm{d} \left(\frac{\pl \mbf{r}_i}{\pl t}\right) \nonumber \\
	& = \sum_{\alpha=1}^s \frac{\mathrm{d}}{\mathrm{d} t} \frac{\pl \mbf{r}_i}{\pl q_\alpha} \mathrm{d} q_\alpha + \sum_{\alpha=1}^s \frac{\pl \mbf{r}_i}{\pl q_\alpha} \mathrm{d} \dot{q}_\alpha + \frac{\mathrm{d}}{\mathrm{d} t} \frac{\pl \mbf{r}_i}{\pl t} \mathrm{d} t
	\label{经典Lagrange关系中间步骤1}
\end{align}
此处将$\mbf{q}$和$\dot{\mbf{q}}$作为相互独立的变量来处理。又考虑到$\dot{\mbf{r}}_i = \dot{\mbf{r}}_i(\mbf{q},\dot{\mbf{q}},t)$,其全微分为
\begin{equation}
	\mathrm{d} \dot{\mbf{r}}_i = \sum_{\alpha=1}^s \frac{\pl \dot{\mbf{r}}_i}{\pl q_\alpha} \mathrm{d} q_\alpha + \sum_{\alpha=1}^s \frac{\pl \dot{\mbf{r}}_i}{\pl \dot{q}_\alpha} \mathrm{d} \dot{q}_\alpha + \frac{\pl \dot{\mbf{r}}_i}{\pl t} \mathrm{d} t
	\label{经典Lagrange关系中间步骤2}
\end{equation}
对比式\eqref{经典Lagrange关系中间步骤1}和式\eqref{经典Lagrange关系中间步骤2}的各项可得如下的{\heiti 经典Lagrange关系}:
\begin{equation}
	\frac{\pl \dot{\mbf{r}}_i}{\pl \dot{q}_\alpha} = \frac{\pl \mbf{r}_i}{\pl q_\alpha},\quad \frac{\pl \dot{\mbf{r}}_i}{\pl q_\alpha} = \frac{\mathrm{d}}{\mathrm{d} t} \frac{\pl \mbf{r}_i}{\pl q_\alpha}
	\label{经典Lagrange关系}
\end{equation}
根据d'Alambert原理,可有
\begin{equation}
	\sum_{i=1}^n (\mbf{F}_i - m_i \ddot{\mbf{r}}_i) \cdot \delta \mbf{r}_i = \sum_{\alpha=1}^s \left[\sum_{i=1}^n (\mbf{F}_i - m_i \ddot{\mbf{r}}_i) \cdot \frac{\pl \mbf{r}_i}{\pl q_\alpha}\right] \delta q_\alpha
	\label{Lagrange方程中间结果}
\end{equation}
其中主动力满足关系$\mbf{F}_i = \mbf{F}_i(\mbf{r},\dot{\mbf{r}},t) = \mbf{F}_i(\mbf{r}(\mbf{q},t),\dot{\mbf{r}}(\mbf{q},\dot{\mbf{q}},t),t)$,因此式\eqref{Lagrange方程中间结果}的第一项即为广义力
\begin{equation*}
	Q_\alpha = \sum_{i=1}^n \mbf{F}_i \cdot \frac{\pl \mbf{r}_i}{\pl q_\alpha} = Q_\alpha(\mbf{q},\dot{\mbf{q}},t)
\end{equation*}
现在考虑式\eqref{Lagrange方程中间结果}的第二项
\begin{align*}
	\sum_{i=1}^n m_i \ddot{\mbf{r}}_i \cdot \frac{\pl \mbf{r}_i}{\pl q_\alpha} & = \frac{\mathrm{d}}{\mathrm{d} t} \left(\sum_{i=1}^n m_i\dot{\mbf{r}}_i \cdot \frac{\pl \mbf{r}_i}{\pl q_\alpha}\right) - \sum_{i=1}^n m_i \dot{\mbf{r}}_i \cdot \frac{\mathrm{d}}{\mathrm{d} t} \frac{\pl \mbf{r}_i}{\pl q_\alpha} \\
	& = \frac{\mathrm{d}}{\mathrm{d} t} \left(\sum_{i=1}^n m_i\dot{\mbf{r}}_i \cdot \frac{\pl \dot{\mbf{r}}_i}{\pl \dot{q}_\alpha}\right) - \sum_{i=1}^n m_i \dot{\mbf{r}}_i \cdot \frac{\pl \dot{\mbf{r}}_i}{\pl q_\alpha} \\
	& = \frac{\mathrm{d}}{\mathrm{d} t} \left(\sum_{i=1}^n \frac12 m_i\frac{\pl \dot{\mbf{r}}_i^2}{\pl \dot{q}_\alpha}\right) - \sum_{i=1}^n \frac12 m_i \frac{\pl \dot{\mbf{r}}_i^2}{\pl q_\alpha} \\
	& = \frac{\mathrm{d}}{\mathrm{d} t} \frac{\pl T}{\pl \dot{q}_\alpha} - \frac{\pl T}{\pl q_\alpha}
\end{align*}
式中应用了经典Lagrange关系\eqref{经典Lagrange关系},而
\begin{equation*}
	T = \sum_{i=1}^n \frac12 m_i\frac{\pl \dot{\mbf{r}}_i^2}{\pl \dot{q}_\alpha} = T(\mbf{q},\dot{\mbf{q}},t)
\end{equation*}
为系统的总动能。由此d'Alambert原理式变为
\begin{equation*}
	\sum_{i=1}^n (\mbf{F}_i - m_i \ddot{\mbf{r}}_i) \cdot \delta \mbf{r}_i = \sum_{\alpha=1}^s \left(Q_\alpha+\frac{\pl T}{\pl q_\alpha} - \frac{\mathrm{d}}{\mathrm{d} t} \frac{\pl T}{\pl \dot{q}_\alpha}\right) \delta q_\alpha = 0
\end{equation*}
由于$\delta q_\alpha\,(\alpha = 1,2,\cdots,s)$相互独立,则有
\begin{equation}
	\frac{\mathrm{d}}{\mathrm{d} t} \frac{\pl T}{\pl \dot{q}_\alpha} - \frac{\pl T}{\pl q_\alpha} = Q_\alpha,\quad \alpha = 1,2,\cdots,s
	\label{理想完整系的Lagrange方程}
\end{equation}
式\eqref{理想完整系的Lagrange方程}即为理想完整系的Lagrange方程,其为$s$个广义坐目标二阶常微分方程组(约束力已经被消去)。

\subsection{理想完整系Lagrange方程的数学结构}

速度可以表示为
\begin{equation*}
	\dot{\mbf{r}}_i = \sum_{\alpha=1}^s \frac{\pl \mbf{r}_i}{\pl q_\alpha} \dot{q}_\alpha + \frac{\pl \mbf{r}_i}{\pl t}
\end{equation*}
由此,动能可以表示为
\begin{align*}
	T & = \sum_{i=1}^n \frac12 m_i \dot{\mbf{r}}_i \cdot \dot{\mbf{r}}_i = \sum_{i=1}^n \frac12 m_i \left(\sum_{\alpha=1}^s \frac{\pl \mbf{r}_i}{\pl q_\alpha} \dot{q}_\alpha + \frac{\pl \mbf{r}_i}{\pl t}\right) \cdot \left(\sum_{\beta=1}^s \frac{\pl \mbf{r}_i}{\pl q_\beta} \dot{q}_\alpha + \frac{\pl \mbf{r}_i}{\pl t}\right) \\
	& = \frac12 \sum_{\alpha,\beta=1}^s \sum_{i=1}^n m_i \frac{\pl \mbf{r}_i}{\pl q_\alpha} \cdot \frac{\pl \mbf{r}_i}{\pl q_\beta} \dot{q}_\alpha \dot{q}_\beta + \sum_{\alpha=1}^s \sum_{i=1}^n m_i \frac{\pl \mbf{r}_i}{\pl q_\alpha} \cdot \frac{\pl \mbf{r}_i}{\pl t} \dot{q}_\alpha + \frac12 \sum_{i=1}^n m_i \left(\frac{\pl \mbf{r}_i}{\pl t}\right)^2 \\
	& = \frac12 \sum_{\alpha,\beta=1}^s M_{\alpha\beta}(\mbf{q},t) \dot{q}_\alpha \dot{q}_\beta + \sum_{\alpha=1}^s P_\alpha(\mbf{q},t) \dot{q}_\alpha + T_0(\mbf{q},t)
\end{align*}
其中
\begin{equation*}
	M_{\alpha\beta}(\mbf{q},t) = \sum_{i=1}^n m_i \frac{\pl \mbf{r}_i}{\pl q_\alpha} \cdot \frac{\pl \mbf{r}_i}{\pl q_\beta},\quad P_\alpha(\mbf{q},t) = \sum_{i=1}^n m_i \frac{\pl \mbf{r}_i}{\pl q_\alpha} \cdot \frac{\pl \mbf{r}_i}{\pl t},\quad T_0(\mbf{q},t) = \frac12 \sum_{i=1}^n m_i \left(\frac{\pl \mbf{r}_i}{\pl t}\right)^2
\end{equation*}
则可将动能$T$按广义速度的次数分为三项,即
\begin{equation}
	T = \frac12 \sum_{\alpha,\beta=1}^s M_{\alpha\beta}(\mbf{q},t) \dot{q}_\alpha \dot{q}_\beta + \sum_{\alpha=1}^s P_\alpha(\mbf{q},t) \dot{q}_\alpha + T_0(\mbf{q},t) = T_2 + T_1 + T_0
	\label{动能的数学结构}
\end{equation}
其中
\begin{align*}
	T_2 & = \frac12 \sum_{\alpha,\beta=1}^s M_{\alpha\beta}(\mbf{q},t) \dot{q}_\alpha \dot{q}_\beta \\
	T_1 & = \sum_{\alpha=1}^s P_\alpha(\mbf{q},t) \dot{q}_\alpha \\
	T_0 & = \frac12 \sum_{i=1}^n m_i \left(\frac{\pl \mbf{r}_i}{\pl t}\right)^2
\end{align*}
由此可有
\begin{equation*}
	\frac{\pl T}{\pl \dot{q}_\alpha} = \sum_{\beta=1}^s M_{\alpha\beta}(\mbf{q},t) \dot{q}_\beta + P_\alpha(\mbf{q},t)
\end{equation*}
即有
\begin{align*}
	\frac{\mathrm{d}}{\mathrm{d} t} \frac{\pl T}{\pl \dot{q}_\alpha} & = \sum_{\beta=1}^s M_{\alpha\beta} \ddot{q}_\beta + \sum_{\beta,\gamma=1}^s \frac{\pl M_{\alpha\beta}}{\pl q_\gamma} \dot{q}_\beta \dot{q}_\gamma + \sum_{\beta=1}^s \left(\frac{\pl M_{\alpha\beta}}{\pl t} + \frac{\pl P_\alpha}{\pl q_\beta}\right) \dot{q}_\beta + \frac{\pl P_\alpha}{\pl t}
\end{align*}
以及
\begin{align*}
	\frac{\pl T}{\pl q_\alpha} = \frac12 \sum_{\beta,\gamma=1}^s \frac{\pl M_{\beta\gamma}}{\pl q_\alpha} \dot{q}_\beta \dot{q}_\gamma + \sum_{\beta=1}^s \frac{\pl P_\beta}{\pl q_\alpha} \dot{q}_\beta + \frac{\pl T_0}{\pl q_\alpha}
\end{align*}

\subsection{有势系的Lagrange函数}

如果外力是有势力\footnote{保守力与有势力是不同的,考虑
\begin{equation*}
	\mathrm{d} V = \sum_{i=1}^n \frac{\pl V}{\pl \mbf{r}_i} \cdot \mathrm{d} \mbf{r}_i + \frac{\pl V}{\pl t}\mathrm{d} t
\end{equation*}
因此有
\begin{equation*}
	\sum_{i=1}^n \mbf{F}_i \cdot \mathrm{d} \mbf{r}_i = -\mathrm{d} V + \frac{\pl V}{\pl t}\mathrm{d} t
\end{equation*}
所以仅当势函数不显含时间$\dfrac{\pl V}{\pl t}=0$时,有势力才是保守力。},即
\begin{equation*}
	\mbf{F}_i = -\frac{\pl V}{\pl \mbf{r}_i},\quad i = 1,2,\cdots,n
\end{equation*}
其中$V(\mbf{r},t)$为势函数。考虑$V = V(\mbf{r}(\mbf{q},t),t)$,可有
\begin{equation*}
	Q_\alpha = -\sum_{i=1}^n \frac{\pl V}{\pl \mbf{r}_i} \cdot \frac{\pl \mbf{r}_i}{\pl q_\alpha} = -\frac{\pl V}{\pl q_\alpha}
\end{equation*}
由此,定义
\begin{equation}
	L(\mbf{q},\dot{\mbf{q}},t) = T(\mbf{q},\dot{\mbf{q}},t) - V(\mbf{q},t)
	\label{Lagrange函数}
\end{equation}
则有{\heiti 理想完整有势系的Lagrange方程}
\begin{equation}
	\frac{\mathrm{d}}{\mathrm{d} t} \frac{\pl L}{\pl \dot{q}_\alpha} - \frac{\pl L}{\pl q_\alpha} = 0,\quad \alpha =1,2,\cdots,s
	\label{理想完整有势系的Lagrange方程}
\end{equation}
如果体系中部分主动力为有势力,则可将其势归入Lagrange函数$L$,非势力表示为广义力,则同样可有Lagrange方程
\begin{equation}
	\frac{\mathrm{d}}{\mathrm{d} t} \frac{\pl L}{\pl \dot{q}_\alpha} - \frac{\pl L}{\pl q_\alpha} = Q'_\alpha,\quad \alpha =1,2,\cdots,s
\end{equation}

\begin{example}
质量为$m$的质点被约束在一光滑的水平平台上运动,质点上系着一根长为$l$的轻绳,绳子穿过平台上的小孔$O$,另一端挂着另一个质量为$M$的质点,写出系统的运动方程。
\begin{figure}[htb]
\centering
\begin{asy}
	texpreamble("\usepackage{xeCJK}");
	texpreamble("\setCJKmainfont{SimSun}");
	usepackage("amsmath");
	import graph;
	import math;
	size(200);
	//第二章例3图
	pair O,a,b,m,M;
	real r;
	O = (0,0);
	a = (1.5,0);
	b = 0.8*dir(60);
	m = dir(30);
	M = (0,-1.5);
	r = 0.05;
	draw((O+a+b)--(O-a+b)--(O-a-b)--(O+a-b)--cycle);
	draw(O--0.6*a,dashed);
	draw(Label("$r$",MidPoint,Relative(W)),O--m);
	draw(O--(0,-b.y),dashed);
	draw((0,-b.y)--M);
	draw(Label("$z$",EndPoint),O--(0,1),Arrow);
	fill(shift(m)*scale(r)*unitcircle,black);
	label("$m$",m,E);
	fill(shift(M)*scale(r)*unitcircle,black);
	label("$M$",M,W);
	label("$\theta$",0.5*dir(15));
	unfill(scale(r)*unitcircle);
	draw(scale(r)*unitcircle);
	label("$O$",O,W);
\end{asy}
\caption{例\theexample}
\label{第二章例3图}
\end{figure}
\end{example}

\begin{solution}
广义坐标取为$r$和$\theta$,则有
\begin{equation*}
	T = \frac12 m(\dot{r}^2 + r^2 \dot{\theta}^2) + \frac12 M\dot{r}^2,\quad V = Mg(r-l)
\end{equation*}
Lagrange函数为
\begin{equation*}
	L(r,\theta;\dot{r},\dot{\theta};t) = \frac12 m(\dot{r}^2 + r^2 \dot{\theta}^2) + \frac12 M\dot{r}^2 - Mg(r-l)
\end{equation*}
计算
\begin{align*}
\begin{array}{ll}
	\dfrac{\pl L}{\pl \dot{r}} = (m+M) \dot{r}, & \dfrac{\pl L}{\pl \dot{\theta}} = mr^2 \dot{\theta} \\[1.5ex]
	\dfrac{\pl L}{\pl r} = mr\dot{\theta}^2 - Mg, & \dfrac{\pl L}{\pl \theta} = 0
\end{array}
\end{align*}
所以系统的运动方程为
\begin{equation*}
	\begin{cases}
		(m+M) \ddot{r} - mr\dot{\theta}^2 + Mg = 0 \\
		\dfrac{\mathrm{d}}{\mathrm{d} t}(mr^2 \dot{\theta}) = 0
	\end{cases}
\end{equation*}
\end{solution}

\section{广义势}

理想完整有势系的外力与速度无关,即
\begin{equation*}
	\mbf{F}_i(\mbf{r},t) = -\frac{\pl V}{\pl \mbf{r}_i}(\mbf{r}_i,t),\quad i=1,2,\cdots,n
\end{equation*}
此时系统存在相应的Lagrange函数。那么对于速度相关的外力,是否也存在类似的Lagrange函数?

\subsection{广义势}

如果质点系中的质点所受外力可以用标量函数$U(\mbf{r},\dot{\mbf{r}},t)$表示为\footnote{与$\dfrac{\pl}{\pl \mbf{r}}$的含义类似,此处$\dfrac{\pl}{\pl \dot{\mbf{r}}_i}$定义为\begin{equation*} \frac{\pl}{\pl \dot{\mbf{r}}_i} = \mbf{e}_1 \frac{\pl}{\pl \dot{x}_i} + \mbf{e}_2 \frac{\pl}{\pl \dot{y}_i} + \mbf{e}_3 \frac{\pl}{\pl \dot{z}_i} \end{equation*}}
\begin{equation}
	\mbf{F}_i = \frac{\mathrm{d}}{\mathrm{d} t} \frac{\pl U}{\pl \dot{\mbf{r}}_i} - \frac{\pl U}{\pl \mbf{r}_i},\quad i = 1,2,\cdots,n
	\label{广义势关系}
\end{equation}
则标量函数$U(\mbf{r},\dot{\mbf{r}},t)$称为体系的{\heiti 广义势}。

\subsection{广义势体系的Lagrange函数}

在广义坐标变换下,广义势函数满足
\begin{equation*}
	U = U(\mbf{r}(\mbf{q},t),\dot{\mbf{r}}(\mbf{q},\dot{\mbf{q}},t),t)
\end{equation*}
则广义力可以表示为
\begin{align}
	Q_\alpha & = \sum_{i=1}^n \left(\frac{\mathrm{d}}{\mathrm{d} t} \frac{\pl U}{\pl \dot{\mbf{r}}_i} - \frac{\pl U}{\pl \mbf{r}_i}\right) \cdot \frac{\pl \mbf{r}_i}{\pl q_\alpha} \nonumber \\
	& = \frac{\mathrm{d}}{\mathrm{d} t} \left(\sum_{i=1}^n \frac{\pl U}{\pl \dot{\mbf{r}}_i} \cdot \frac{\pl \mbf{r}_i}{\pl q_\alpha} \right) - \sum_{i=1}^n \left(\frac{\pl U}{\pl \dot{\mbf{r}}_i} \cdot \frac{\mathrm{d}}{\mathrm{d} t} \frac{\pl \mbf{r}_i}{\pl q_\alpha} + \frac{\pl U}{\pl \mbf{r}_i} \cdot \frac{\pl \mbf{r}_i}{\pl q_\alpha}\right) \nonumber \\
	& = \frac{\mathrm{d}}{\mathrm{d} t} \left(\sum_{i=1}^n \frac{\pl U}{\pl \dot{\mbf{r}}_i} \cdot \frac{\pl \dot{\mbf{r}}_i}{\pl \dot{q}_\alpha} \right) - \sum_{i=1}^n \left(\frac{\pl U}{\pl \dot{\mbf{r}}_i} \cdot \frac{\pl \dot{\mbf{r}}_i}{\pl q_\alpha} + \frac{\pl U}{\pl \mbf{r}_i} \cdot \frac{\pl \mbf{r}_i}{\pl q_\alpha}\right) \nonumber \\
	& = \frac{\mathrm{d}}{\mathrm{d} t} \frac{\pl U}{\pl \dot{q}_\alpha} - \frac{\pl U}{\pl q_\alpha}
	\label{广义势系统中的广义力}
\end{align}
式中利用了经典Lagrange关系\eqref{经典Lagrange关系}。由此可定义广义势系统的Lagrange函数
\begin{equation}
	L(\mbf{q},\dot{\mbf{q}},t) = T(\mbf{q},\dot{\mbf{q}},t) - U(\mbf{q},\dot{\mbf{q}},t)
\end{equation}
则系统的Lagrange方程与式\eqref{理想完整有势系的Lagrange方程}相同,即
\begin{equation}
	\frac{\mathrm{d}}{\mathrm{d}} \frac{\pl L}{\pl \dot{q}_\alpha} - \frac{\pl L}{\pl q_\alpha} = 0,\quad \alpha = 1,2,\cdots,s
\end{equation}
考虑动能$\displaystyle T = \sum_{i=1}^n \frac12 m_i \dot{\mbf{r}}_i \cdot \dot{\mbf{r}}_i$满足
\begin{equation*}
	-m_i \ddot{\mbf{r}}_i = \frac{\mathrm{d}}{\mathrm{d} t} \frac{\pl (-T)}{\pl \dot{\mbf{r}}_i} - \frac{\pl (-T)}{\pl \mbf{r}_i}
\end{equation*}
即,惯性力的广义势是负动能。由此,Lagrange函数即(除负号)惯性力与主动力广义势能之和,又称为体系的{\heiti 动势}。

\subsection{电磁场中带电粒子的广义势能}

电磁场中的带电粒子受到Lorentz力,即
\begin{equation}
	\mbf{F} = e\mbf{E}(\mbf{r},t) + e\dot{\mbf{r}} \times \mbf{B}(\mbf{r},t)
\end{equation}
根据Maxwell方程组中的$\bnb \cdot \mbf{B} = 0$,可令
\begin{equation}
	\mbf{B} = \bnb \times \mbf{A}
	\label{磁矢势}
\end{equation}
此处,式\eqref{磁矢势}中的矢量函数$\mbf{A}(\mbf{r},t)$称为{\heiti 电磁场的矢势}。再根据Maxwell方程组中的$\displaystyle \bnb \times \mbf{E} = -\frac{\pl \mbf{B}}{\pl t}$,可得
\begin{equation*}
	\bnb \times \left(\mbf{E} + \frac{\pl \mbf{A}}{\pl t}\right) = 0
\end{equation*}
由此可令
\begin{equation}
	\mbf{E} = -\bnb \phi - \frac{\pl \mbf{A}}{\pl t}
	\label{电标势}
\end{equation}
此处,式\eqref{电标势}中的标量函数$\phi(\mbf{r},t)$称为{\heiti 电磁场的标势}。

由此,Lorentz力可以表示为
\begin{align}
	\mbf{F} & = e\left[-\bnb \phi - \frac{\pl \mbf{A}}{\pl t} + \dot{\mbf{r}} \times (\bnb \times \mbf{A})\right] = e\left[-\frac{\pl \phi}{\pl \mbf{r}} - \frac{\pl \mbf{A}}{\pl t} - \dot{\mbf{r}} \cdot \frac{\pl \mbf{A}}{\pl \mbf{r}} + \frac{\pl (\dot{\mbf{r}} \cdot \mbf{A})}{\pl \mbf{r}}\right] \nonumber \\
	& = e\left[-\frac{\pl \phi}{\pl \mbf{r}} - \frac{\mathrm{d} \mbf{A}}{\mathrm{d} t} + \frac{\pl (\dot{\mbf{r}} \cdot \mbf{A})}{\pl \mbf{r}}\right] \nonumber \\
	& = \frac{\mathrm{d}}{\mathrm{d} t} \frac{\pl (e\phi - e\dot{\mbf{r}}\cdot \mbf{A})}{\pl \dot{\mbf{r}}} - \frac{\pl (e\phi - e\dot{\mbf{r}}\cdot \mbf{A})}{\pl \mbf{r}}
\end{align}
由此,广义势为
\begin{equation}
	U = e(\phi - \mbf{A} \cdot \dot{\mbf{r}})
	\label{电磁场中带电粒子的广义势}
\end{equation}
在电磁场中运动的自由带电粒子的Lagrange函数为
\begin{equation}
	L(\mbf{r},\dot{\mbf{r}},t) = \frac12 m \dot{\mbf{r}}^2 + e\mbf{A}(\mbf{r},t) \cdot \dot{\mbf{r}} - e\phi(\mbf{r},t)
	\label{电磁场中点电粒子的Lagrange函数}
\end{equation}

\section{守恒定律}

\subsection{广义动量及其守恒定律}

定义与广义坐标$q_\alpha$共轭的{\heiti 广义动量}
\begin{equation}
	p_\alpha = \frac{\pl L}{\pl \dot{q}_\alpha}
	\label{广义动量定义}
\end{equation}
则理想完整有势系的Lagrange方程\eqref{理想完整有势系的Lagrange方程}变为
\begin{equation}
	\dot{p}_\alpha = \frac{\pl L}{\pl q_\alpha}
	\label{广义动量定理}
\end{equation}
式\eqref{广义动量定理}表示的关系称为{\heiti 广义动量定理}。

如果某体系的Lagrange函数不显含某个广义坐标,即
\begin{equation*}
	\frac{\pl L}{\pl q_\alpha} = 0
\end{equation*}
则称坐标$q_\alpha$为{\heiti 循环坐标}或{\heiti 可遗坐标},根据广义动量定理\eqref{广义动量定理},可有
\begin{equation}
	\dot{p}_\alpha = 0
\end{equation}
即,与循环坐标共轭的广义动量守恒。

\subsection{广义能量及其守恒定律}

考虑Lagrange函数$L = L(\mbf{q},\dot{\mbf{q}},t)$的时间导数
\begin{align}
	\frac{\mathrm{d} L}{\mathrm{d} t} & = \sum_{\alpha=1}^s \frac{\pl L}{\pl q_\alpha}\dot{q}_\alpha + \sum_{\alpha=1}^s \frac{\pl L}{\pl \dot{q}_\alpha} \ddot{q}_\alpha + \frac{\pl L}{\pl t} \nonumber \\ 
	& = \sum_{\alpha=1}^s \frac{\pl L}{\pl q_\alpha} \dot{q}_\alpha + \frac{\mathrm{d}}{\mathrm{d} t} \sum_{\alpha=1}^s \frac{\pl L}{\pl \dot{q}_\alpha} \dot{q}_\alpha - \sum_{i=1}^s \frac{\mathrm{d}}{\mathrm{d} t} \left(\frac{\pl L}{\pl \dot{q}_\alpha}\right) \dot{q}_\alpha + \frac{\pl L}{\pl t} \nonumber \\
	& = -\sum_{\alpha=1}^s \left[\frac{\mathrm{d}}{\mathrm{d} t} \left(\frac{\pl L}{\pl \dot{q}_\alpha}\right) - \frac{\pl L}{\pl q_\alpha}\right] \dot{q}_\alpha + \frac{\mathrm{d}}{\mathrm{d} t} \sum_{\alpha=1}^s \frac{\pl L}{\pl \dot{q}_\alpha} \dot{q}_\alpha + \frac{\pl L}{\pl t}
	\label{广义能量定理中间步骤}
\end{align}
利用理想完整有势系的Lagrange方程,式\eqref{广义能量定理中间步骤}中的第一项为零。由此可令
\begin{equation}
	H = \sum_{\alpha=1}^s \frac{\pl L}{\pl \dot{q}_\alpha} \dot{q}_\alpha - L
	\label{广义能量}
\end{equation}
称为{\heiti 广义能量},由此式\eqref{广义能量定理中间步骤}化为
\begin{equation}
	\dot{H} = -\frac{\pl L}{\pl t}
	\label{广义能量定理}
\end{equation}
式\eqref{广义能量定理}表示的关系称为{\heiti 广义能量定理}。如果系统的Lagrange函数不显含时间,即
\begin{equation*}
	\frac{\pl L}{\pl t} = 0
\end{equation*}
则有
\begin{equation}
	\dot{H} = 0
\end{equation}
即,系统的广义能量守恒。

\subsection{广义能量的数学结构}\label{第二章:广义能量的数学结构}

\begin{theorem}[齐次函数的Euler定理]
\label{齐次函数的Euler定理}
如果$n$元函数$f(x_1,x_2,\cdots,x_n)$对$\forall \, \lambda \neq 0$满足
\begin{equation*}
	f(\lambda x_1,\lambda x_2,\cdots,\lambda x_n) = \lambda^m f(x_1,x_2,\cdots,x_n)
\end{equation*}
则称为$m$次{\heiti 齐次函数}。其满足
\begin{equation}
	\sum_{i=1}^n x_i \frac{\pl f}{\pl x_i} = mf(x_1,x_2,\cdots,x_n)
\end{equation}
\end{theorem}
\begin{proof}
将式
\begin{equation*}
	f(\lambda x_1,\lambda x_2,\cdots,\lambda x_n) = \lambda^m f(x_1,x_2,\cdots,x_n)
\end{equation*}
两端对$\lambda$求偏导数,可得
\begin{equation*}
	\sum_{i=1}^n x_i f'_i = m\lambda^{m-1} f(x_1,x_2,\cdots,x_n)
\end{equation*}
由此,令$\lambda = 1$,即得证齐次函数的Euler定理
\begin{equation*}
	\sum_{i=1}^n x_i \frac{\pl f}{\pl x_i} = mf(x_1,x_2,\cdots,x_n) \qedhere
\end{equation*}
\end{proof}

根据动能的表达式\eqref{动能的数学结构},可有
\begin{equation*}
	L = T-V = T_2 + T_1 + (T_0 - V) = L_2 + L_1 + L_0
\end{equation*}
由此,根据广义速度$\dot{q}_\alpha$的次数,可将Lagrange函数$L$分为三项,即
\begin{align*}
	L_2 & = \frac12 \sum_{\alpha,\beta=1}^s M_{\alpha\beta}(\mbf{q},t) \dot{q}_\alpha \dot{q}_\beta \\
	L_1 & = \sum_{\alpha=1}^s P_\alpha(\mbf{q},t) \dot{q}_\alpha \\
	L_0 & = \frac12 \sum_{i=1}^n m_i \left(\frac{\pl \mbf{r}_i}{\pl t}\right)^2 - V
\end{align*}
显然,$L_2,L_1$分别为广义速度$\dot{q}_\alpha$的$2$次、$1$次齐次函数,根据Euler定理,它们满足
\begin{equation*}
	\sum_{\alpha=1}^s \dot{q}_\alpha \frac{\pl L_2}{\pl \dot{q}_\alpha} = 2L_2,\quad \sum_{\alpha=1}^s \dot{q}_\alpha \frac{\pl L_1}{\pl \dot{q}_\alpha} = L_1
\end{equation*}
因此,广义能量为
\begin{equation*}
	H = \sum_{\alpha=1}^s \frac{\pl L}{\pl \dot{q}_\alpha} \dot{q}_\alpha - L = 2L_2 + L_1 - (L_2+L_1+L_0) = L_2 - L_0 = T_2 - T_0 + V
\end{equation*}

对于完整稳定约束系统,即
\begin{equation*}
	\frac{\pl f_j}{\pl t} = 0,\quad f_j(\mbf{r}) = 0,\quad j = 1,2,\cdots,k
\end{equation*}
可根据约束取广义坐标$\mbf{r}_i = \mbf{r}_i(\mbf{q})$,此时即有
\begin{equation*}
	\frac{\pl \mbf{r}_i}{\pl t} = \mbf{0},\quad i = 1,2,\cdots,n
\end{equation*}
此时$T_1 = T_0 = 0,\, T = T_2$,所以
\begin{equation*}
	H = T+V
\end{equation*}
稳定约束力实功为零,因为实位移属于虚位移集合。即,对稳定约束体系,广义能量即惯性系内的体系总能量。

对于完整非稳定约束系统,此时广义坐标取为$\mbf{r}_i = \mbf{r}_i(\mbf{q},t)$,即有
\begin{equation*}
	\frac{\pl \mbf{r}_i}{\pl t} \neq \mbf{0},\quad i = 1,2,\cdots,n
\end{equation*}
此时有
\begin{equation*}
	H = T_2-T_0+V
\end{equation*}
记$T' = T_2$,$V'=V-T_0$分别为广义曲线坐标系的等效动能与势能,则有
\begin{equation*}
	H = T'+V'
\end{equation*}
即,对非稳定约束体系,广义能量即广义坐标系(如非惯性系)内的体系总能量。

\begin{example}
半径为$R$的光滑圆环以匀角速度$\mathnormal{\Omega}$绕铅直直径旋转。质量为$m$的质点穿在圆环上,求其平衡位置。
\begin{figure}[htb]
\centering
\begin{asy}
	texpreamble("\usepackage{xeCJK}");
	texpreamble("\setCJKmainfont{SimSun}");
	usepackage("amsmath");
	import graph;
	import math;
	size(200);
	//第二章例5图
	pair O;
	real R,r,theta;
	O = (0,0);
	R = 1;
	r = 0.05;
	theta = -30;
	draw(shift(O)*scale(R)*unitcircle,linewidth(1bp));
	draw(O--R*dir(0),dashed);
	draw(O--R*dir(theta));
	fill(shift(R*dir(theta))*scale(r)*unitcircle,black);
	draw(1.2*R*dir(-90)--1.4*R*dir(90),dashed);
	draw(shift(1.2*R*dir(90))*xscale(2)*arc(O,2*r,110,360+70),Arrow);
	label("$O$",O,W);
	draw(Label("$\theta$",MidPoint,Relative(E)),arc(O,0.2,-90,theta),Arrow);
	label("$m$",R*dir(theta),3*W);
	label("$\mathnormal{\Omega}$",1.2*R*dir(90)+4*r*dir(0),E);
\end{asy}
\caption{例\theexample}
\label{第二章例5图}
\end{figure}
\end{example}
\begin{solution}
选择$\theta$作为广义坐标,则体系的Lagrange函数为
\begin{equation*}
	L = \frac12 m(R^2 \mathnormal{\Omega}^2 \sin^2 \theta + R^2 \dot{\theta}^2) + mgR(1-\cos \theta)
\end{equation*}
Lagrange函数不显含时间,故广义能量守恒,即
\begin{equation*}
	H = \frac{\pl L}{\pl \dot{\theta}} \dot{\theta} - L = \frac12 mR^2 \dot{\theta}^2 - \frac12 mR^2 \mathnormal{\Omega}^2 \sin^2 \theta + mgR(1-\cos \theta) = H_0
\end{equation*}
根据此广义能量的数学结构,可得旋转系中的动能为
\begin{equation*}
	T' = \frac12 mR^2 \dot{\theta}^2
\end{equation*}
惯性离心力与重力总势能(等效势能)为
\begin{equation*}
	V' = -\frac12 mR^2 \mathnormal{\Omega}^2 \sin^2 \theta + mgR(1-\cos \theta)
\end{equation*}
体系的势能曲线如图\ref{第二章例5势能曲线}所示。
\begin{figure}[htb]
\centering
\begin{asy}
	texpreamble("\usepackage{xeCJK}");
	texpreamble("\setCJKmainfont{SimSun}");
	usepackage("amsmath");
	import graph;
	import math;
	size(300);
	//第二章例5势能曲线
	real Rg,scale;
	real f(real theta){
		return scale*(-0.5*Rg*(sin(theta))**2+1-cos(theta));
	}
	scale = 2.5;
	Rg = 0;
	draw(Label("$\theta$",EndPoint),(-3.5,0)--(3.5,0),Arrow);
	draw(Label("$V'$",EndPoint),(0,-0.5)--(0,f(pi)+0.4),Arrow);
	pair P;
	for(int i=1;i<=4;i=i+1){
		P = (0,f(pi)/4*i);
		draw((P-(0.1,0))--(P+(0.1,0)));
	}
	label("$mgR$",(0.1,f(pi)/2),E);
	draw(Label("$-\pi$",BeginPoint),(-pi,-0.1)--(-pi,0.1));
	draw(Label("$\pi$",BeginPoint),(pi,-0.1)--(pi,0.1));
	draw(graph(f,-pi,pi),linewidth(0.8bp));
	Rg = 0.5;
	draw(graph(f,-pi,pi),red+linewidth(0.8bp));
	Rg = 1;
	draw(graph(f,-pi,pi),yellow+linewidth(0.8bp));
	Rg = 2;
	draw(graph(f,-pi,pi),green+linewidth(0.8bp));
	
	pair A,B;
	A = (-3,0.4*f(pi));
	B = (-1,0.85*f(pi));
	real delta;
	delta = (B.y-A.y)/8;
	fill(shift((0.05,-0.05))*box(A,B),black);
	unfill(box(A,B));
	draw(box(A,B));
	label("$R\mathnormal{\Omega}^2/g = 0.0$",((A.x+B.x)/2,A.y)+(0,7*delta));
	label("$R\mathnormal{\Omega}^2/g = 0.5$",((A.x+B.x)/2,A.y)+(0,5*delta),red);
	label("$R\mathnormal{\Omega}^2/g = 1.0$",((A.x+B.x)/2,A.y)+(0,3*delta),yellow);
	label("$R\mathnormal{\Omega}^2/g = 2.0$",((A.x+B.x)/2,A.y)+(0,1*delta),green);
\end{asy}
\caption{势能曲线}
\label{第二章例5势能曲线}
\end{figure}

平衡位置需满足
\begin{equation*}
	\frac{\pl V'}{\pl \theta}\bigg|_{\theta_0} = -mR^2 \mathnormal{\Omega}^2 \sin \theta_0 \cos \theta_0 + mgR \sin \theta_0 = 0
\end{equation*}
平衡是否稳定需考察
\begin{equation*}
	\frac{\pl^2 V'}{\pl \theta^2}\bigg|_{\theta_0} = mgR\cos \theta_0 - mR^2 \mathnormal{\Omega}^2 (2\cos^2 \theta_0-1)
\end{equation*}

当$\displaystyle \mathnormal{\Omega}^2 < \frac{g}{R}$时,平衡位置为$\theta_0 = 0$和$\theta_0 = \pi$。当$\theta_0 = 0$时,
\begin{equation*}
	\frac{\pl^2 V'}{\pl \theta^2}\bigg|_{0} = mR^2 \left(\frac{g}{R} - \mathnormal{\Omega}^2\right) > 0
\end{equation*}
平衡是稳定的。当$\theta_0 = \pi$时,
\begin{equation*}
	\frac{\pl^2 V'}{\pl \theta^2}\bigg|_{\pi} = -mR^2 \left(\frac{g}{R} + \mathnormal{\Omega}^2\right) < 0
\end{equation*}
平衡是不稳定的。

当$\displaystyle \mathnormal{\Omega}^2 > \frac{g}{R}$时,平衡位置为$\theta_0 = 0$、$\displaystyle \theta_0 = \pm \arccos \frac{g}{R\mathnormal{\Omega}^2}$和$\theta_0 = \pi$。当$\theta_0 = 0$时,
\begin{equation*}
	\frac{\pl^2 V'}{\pl \theta^2}\bigg|_{0} = mR^2 \left(\frac{g}{R} - \mathnormal{\Omega}^2\right) < 0
\end{equation*}
平衡是不稳定的。当$\displaystyle \theta_0 = \pm \arccos \frac{g}{R\mathnormal{\Omega}^2}$时,
\begin{equation*}
	\frac{\pl^2 V'}{\pl \theta^2}\bigg|_{\pm \arccos \frac{g}{R\mathnormal{\Omega}^2}} = \frac{mR^2}{\mathnormal{\Omega}^2} \left(\mathnormal{\Omega}^4-\frac{g^2}{R^2}\right) > 0
\end{equation*}
平衡是稳定的。当$\theta_0 = \pi$时,
\begin{equation*}
	\frac{\pl^2 V'}{\pl \theta^2}\bigg|_{\pi} = -mR^2 \left(\frac{g}{R} + \mathnormal{\Omega}^2\right) < 0
\end{equation*}
平衡是不稳定的。
\end{solution}

\begin{example}[电磁场中自由带电粒子的动量、能量及其守恒定律]
设电磁场的标势和矢势分别为
\begin{equation*}
	\phi = \phi(\mbf{r},t),\quad \mbf{A} = \mbf{A}(\mbf{r},t)
\end{equation*}
则该电磁场中的自由粒子Lagrange函数为
\begin{equation}
	L = \frac12 m\dot{\mbf{r}}^2 + e \mbf{A} \cdot \dot{\mbf{r}} - e\phi
\end{equation}
因此其动量为
\begin{equation}
	\mbf{p} = \frac{\pl L}{\pl \dot{\mbf{r}}} = m\mbf{\dot{r}} + e\mbf{A}
\end{equation}
其中$m\dot{\mbf{r}}$可称为{\heiti 机械动量},$e\mbf{A}$可称为{\heiti 电磁动量}。其能量为
\begin{equation}
	H = \frac{\pl L}{\pl \dot{\mbf{r}}} \cdot \dot{\mbf{r}} - L = L_2 - L_0 = \frac12 m\dot{\mbf{r}}^2 + e\phi
\end{equation}
其中$\dfrac12 m\dot{\mbf{r}}^2$可称为{\heiti 机械动能},$e\phi$可称为{\heiti 电磁势能}。

\iffalse
对于均匀电磁场,此时标势和矢势与空间坐标无关\footnote{这里均匀电磁场并非指场变量$\mbf{E}$和$\mbf{B}$而是指标势和矢势在空间上是均匀的。如果场变量在空间上均匀,标势和矢势并不一定与空间坐标无关。此时有\begin{equation*} \mbf{E} = -\bnb \phi - \frac{\pl \mbf{A}}{\pl t} = - \frac{\pl \mbf{A}}{\pl t},\quad \mbf{B} = \bnb \times \mbf{A} = \mbf{0} \end{equation*}即空间中只存在均匀的时变电场,这可由场源的特定分布来实现。},即
\begin{equation*}
	\phi = \phi(t),\quad \mbf{A} = \mbf{A}(t)
\end{equation*}
此时$\mbf{r}$为循环坐标,因此有
\begin{equation*}
	\dot{\mbf{p}} = \frac{\pl L}{\pl \mbf{r}} = \mbf{0}
\end{equation*}
粒子的动量守恒。\fi

对于稳恒电磁场,此时标势和矢势与时间无关,即
\begin{equation*}
	\phi = \phi(\mbf{r}),\quad \mbf{A} = \mbf{A}(\mbf{r})
\end{equation*}
此时Lagrange函数不显含时间,即
\begin{equation*}
	\dot{H} = \frac{\pl L}{\pl t} = 0
\end{equation*}
粒子的能量守恒。
\end{example}

\begin{example}[稳恒均匀电磁场中的自由带电粒子]
设空间中存在稳恒、均匀的电磁场$\mbf{E}$和$\mbf{B}$,取磁场方向为$z$方向,则可取
\begin{equation*}
	\mbf{A} = \frac12 B(-y\mbf{e}_x + x\mbf{e}_y) = \frac12 \mbf{B} \times \mbf{r},\quad \phi = -\mbf{E} \cdot \mbf{r}
\end{equation*}
此时,该粒子的Lagrange函数可表示为
\begin{equation*}
	L = \frac12 m\dot{\mbf{r}}^2 + \frac12 e\dot{\mbf{r}} \cdot (\mbf{B} \times \mbf{r}) + e\mbf{E} \cdot \mbf{r} = \frac12 m\dot{\mbf{r}}^2 + \frac12 e \mbf{r} \cdot (\dot{\mbf{r}} \times \mbf{B}) + e\mbf{E} \cdot \mbf{r} 
\end{equation*}
根据Lagrange方程,可得稳恒均匀电磁场中的自由带电粒子的运动微分方程为
\begin{equation*}
	m \ddot{\mbf{r}} = e\dot{\mbf{r}} \times \mbf{B} + e\mbf{E}
\end{equation*}
上式右端即为该粒子受到的Lorentz力。
\end{example}

\section{时空对称性与守恒量}

\subsection{时空对称性}

某物理定律或物理量在某变换下形式或量值保持不变,则称该变换为{\heiti 对称变换},此物理定律或物理量具有该变换下的{\heiti 对称性}({\heiti 不变性})。

Lagrange函数是决定体系力学性质的特征函数,其变换不变性又称为体系对称性。力学体系在时空变换下的对称性称为{\heiti 时空对称性},包括
\begin{itemize}
	\item {\heiti 空间平移对称性}:体系作任意整体平移,Lagrange函数不变;
	\item {\heiti 空间转动对称性}:体系作任意整体转动,Lagrange函数不变;
	\item {\heiti 时间平移对称性}:体系时间作任意平移,Lagrange函数不变。
\end{itemize}

\subsection{动量、角动量与能量守恒}

\subsubsection{空间平移对称性与动量守恒}

在平移方向上取广义坐标$q_1$,满足
\begin{equation*}
	\frac{\pl \mbf{r}_i}{\pl q_1} = \mbf{e},i =1,2,\cdots,n
\end{equation*}
固定其它广义坐标,只改变广义坐标$q_1$,即令
\begin{equation*}
	\delta q_\alpha = 0 ,\quad \alpha = 2,3,\cdots,n
\end{equation*}
则有
\begin{equation*}
	\delta \mbf{r}_i = \sum_{\alpha=1}^s \frac{\pl \mbf{r}_i}{\pl q_\alpha} \delta q_\alpha = \delta q_1 \mbf{e},\quad i = 1,2,\cdots,n
\end{equation*}
因此,广义坐标$q_1$的变化将导致体系的整体平移。体系的Lagrange函数在平移变换下不变,因此Lagrange函数不显含$q_1$,即
\begin{equation*}
	\frac{\pl L}{\pl q_1} = 0
\end{equation*}
由广义动量定理,可有
\begin{equation*}
	\dot{p}_1 = \frac{\pl L}{\pl q_1} = 0
\end{equation*}
又考虑到
\begin{align*}
	p_1 & = \frac{\pl L}{\pl \dot{q}_1} = \frac{\pl T}{\pl \dot{q}_1} = \frac{\pl}{\pl \dot{q}_1} \sum_{i=1}^n \frac12 m_i \dot{\mbf{r}}_i \cdot \dot{\mbf{r}}_i = \sum_{i=1}^n m_i \dot{\mbf{r}}_i \cdot \frac{\pl \dot{\mbf{r}}_i}{\pl \dot{q}_1} \\
	& = \sum_{i=1}^n m_i \dot{\mbf{r}}_i \cdot \frac{\pl \mbf{r}_i}{\pl q_1} = \mbf{e} \cdot \mbf{p}
\end{align*}
由此,体系沿$\mbf{e}$方向平移对称,则体系总动量的$\mbf{e}$分量守恒。

空间平移对称性又称{\heiti 空间均匀性},即绝对位置不可测。

\subsubsection{空间转动对称性与角动量守恒}

\begin{figure}[htb]
\centering
\begin{asy}
	texpreamble("\usepackage{xeCJK}");
	texpreamble("\setCJKmainfont{SimSun}");
	usepackage("amsmath");
	import graph;
	import math;
	size(200);
	//空间转动
	pair O,OO,R1,R2;
	real h,r,theta;
	path cir;
	O = (0,0);
	h = 2;
	r = 1;
	theta = -30;
	OO = rotate(theta)*(0,2);
	cir = shift(OO)*rotate(theta)*yscale(1/2)*scale(r)*unitcircle;
	draw(Label("$\boldsymbol{e}$",EndPoint),O--3/2*OO,Arrow);
	draw(cir);
	R1 = relpoint(cir,0.85);
	R2 = relpoint(cir,0.95);
	draw(R1--OO--R2);
	draw(Label("$\delta \boldsymbol{r}_i$",MidPoint,N),R1--R2,red,Arrow);
	draw(Label("$\boldsymbol{r}_i$",MidPoint,Relative(W)),O--R1,Arrow);
	draw(Label("$\boldsymbol{r}_i+\delta \boldsymbol{r}_i$",MidPoint,Relative(E)),O--R2,Arrow);
	label("$\delta q_1$",OO,5*dir((R1+R2)/2-OO));
	//draw(O--(2.1,0),invisible);
\end{asy}
\caption{空间转动}
\label{空间转动}
\end{figure}
在转动方向上取广义坐标$q_1$,满足
\begin{equation*}
	\frac{\pl \mbf{r}_i}{\pl q_1} = \mbf{e} \times \mbf{r}_i,i =1,2,\cdots,n
\end{equation*}
固定其它广义坐标,只改变广义坐标$q_1$,即令
\begin{equation*}
	\delta q_\alpha = 0 ,\quad \alpha = 2,3,\cdots,n
\end{equation*}
则有
\begin{equation*}
	\delta \mbf{r}_i = \sum_{\alpha=1}^s \frac{\pl \mbf{r}_i}{\pl q_\alpha} \delta q_\alpha = \delta q_1 \mbf{e} \times \mbf{r}_i,\quad i = 1,2,\cdots,n
\end{equation*}
因此,广义坐标$q_1$的变化将导致体系的整体转动。体系的Lagrange函数在转动变换下不变,因此Lagrange函数不显含$q_1$,即
\begin{equation*}
	\frac{\pl L}{\pl q_1} = 0
\end{equation*}
由广义动量定理,可有
\begin{equation*}
	\dot{p}_1 = \frac{\pl L}{\pl q_1} = 0
\end{equation*}
又考虑到
\begin{align*}
	p_1 & = \frac{\pl L}{\pl \dot{q}_1} = \frac{\pl T}{\pl \dot{q}_1} = \sum_{i=1}^n m_i \dot{\mbf{r}}_i \cdot \frac{\pl \mbf{r}_i}{\pl q_1} = \sum_{i=1}^n m_i\dot{\mbf{r}}_i \cdot (\mbf{e} \times \mbf{r}_i) \\
	& = \sum_{i=1}^n \mbf{e} \cdot (\mbf{r}_i \times m_i \dot{\mbf{r}}_i) = \mbf{e} \cdot \mbf{L}
\end{align*}
由此,体系绕$\mbf{e}$整体转动对称,则体系总角动量的$\mbf{e}$分量守恒。

空间转动对称性又称{\heiti 空间各项同性},即绝对方向不可测。

\subsubsection{时间平移对称性与能量守恒}

体系的Lagrange函数在时间平移变换下不变,因此Lagrange函数不显含$t$,即
\begin{equation*}
	\frac{\pl L}{\pl t} = 0
\end{equation*}
由广义能量定理有
\begin{equation*}
	\dot{H} = -\frac{\pl L}{\pl t} = 0
\end{equation*}
由此,体系时间平移对称,则体系总能量守恒。

时间平移对称性又称{\heiti 时间均匀性},即绝对时间不可测。

守恒定律与时空对称性有关,全部结论适用于整个物理学。循环坐标反映体系的对称性(变换不变性),可用其共轭动量守恒定律(只含广义速度和广义坐标)取代运动方程(含广义加速度)。对称性使体系的求解简化。

对称性是体系的客观属性,不依赖于主观描述。广义坐标选取适当,可使对称性通过循环坐标直观呈现。根据体系的对称性选择广义坐标,使循环坐标越多越好。

\begin{example}
自由质点受势场作用,场源均匀分布在以$z$轴为轴线、螺距为$h$的无限长圆柱螺旋线上。找出体系的守恒量。
\end{example}
\begin{solution}
选取广义坐标$(\rho,\phi,\zeta)$满足
\begin{equation*}
\begin{cases}
	x = \rho \cos \phi \\
	y = \rho \sin \phi \\
	z = \zeta + \dfrac{h\phi}{2\pi}
\end{cases}
\end{equation*}
此处,$\begin{cases} \rho = C_1 \\ \zeta = C_3 \end{cases}$表示的即为圆柱螺旋线族。此体系具有对称性:将质点沿任意螺距为$h$的螺旋线移动,Lagrange函数不变,即$\phi$为循环坐标,因此有
\begin{equation*}
	\dot{p}_\phi = \frac{\pl L}{\pl \phi} = 0
\end{equation*}
为了了解此处$p_\phi$的物理含义,考虑柱坐标系$(\rho,\psi,z)$,则广义坐标与柱坐标之间的关系为
\begin{equation*}
\begin{cases}
	\rho = \rho \\
	\psi = \phi \\
	z = \zeta + \dfrac{h\phi}{2\pi}
\end{cases}
\end{equation*}
如果取柱坐标$(\rho,\psi,z)$作为广义坐标,其与广义坐标共轭的广义动量物理意义比较清晰,即有
\begin{equation*}
	p_\psi = \frac{\pl L'}{\pl \dot{\psi}} = L_z,\quad p_z = \frac{\pl L'}{\pl \dot{z}}
\end{equation*}
此处$L_z$即为质点对$z$轴的角动量,$p_z$为沿$z$方向的动量。由此可有
\begin{equation*}
	p_\phi = \frac{\pl L}{\pl \dot{\phi}} = \frac{\pl L'}{\pl \dot{\psi}} \frac{\pl \dot{\psi}}{\pl \dot{\phi}} + \frac{\pl L'}{\pl \dot{z}} \frac{\pl \dot{z}}{\pl \dot{\phi}} = p_\psi + \frac{h}{2\pi} p_z = L_z + \frac{h}{2\pi} p_z
\end{equation*}
由此可得,此体系的守恒量为
\begin{equation*}
	p_\phi = L_z + \frac{h}{2\pi} p_z
\end{equation*}
\end{solution}

\section{位形时空的Lagrange方程}

\subsection{位形时空}

理想完整有势系的Lagrange方程
\begin{equation}
	\frac{\mathrm{d}}{\mathrm{d} t} \frac{\pl L}{\pl \dot{q}_\alpha} - \frac{\pl L}{\pl q_\alpha} = 0,\quad \alpha = 1,2,\cdots,s
\end{equation}
是$s$个二阶常微分方程构成的常微分方程组,给定$2s$个初始条件,即
\begin{equation*}
	\mbf{q}(t_0) = \mbf{q}_0,\quad \dot{\mbf{q}}(t_0) = \dot{\mbf{q}}_0
\end{equation*}
可以唯一确定位形空间轨道的参数方程
\begin{equation*}
	\mbf{q} = \mbf{q}(t)
\end{equation*}
此处,$t$为参数。另选参数$\tau$,将时间看作第$s+1$个广义坐标$q_{s+1}$,则$-H$即为与$q_{s+1}$共轭的广义动量$p_{s+1} = p_t = -H$。由此广义动量定理和广义能量定理将统一为
\begin{equation*}
	\dot{p}_\alpha = \frac{\pl L}{\pl q_\alpha},\quad \alpha = 1,2,\cdots,s,s+1
\end{equation*}

由广义坐标与时间张成的$s+1$维抽象空间称为{\heiti 位形时空}。位形时空中的点称为{\heiti 世界点},代表体系的物理事件。世界点在位形时空运动而描出的轨道称为{\heiti 世界线},反映体系物理过程的演化史。

\subsection{推广的Lagrange方程}

引入新的参数$\tau$,则有
\begin{equation*}
	\begin{cases}
		\mbf{q} = \mbf{q}(t(\tau)) \\
		t = t(\tau)
	\end{cases}
\end{equation*}
得到的对应Lagrange函数的状态函数可以表示为
\begin{equation*}
	\mathnormal{\Lambda} = \mathnormal{\Lambda}(\mbf{q},t;\mbf{q}',t';\tau)
\end{equation*}
由此可有关系\footnote{此处,用$t'$来表示$t$对$\tau$的导数,今后不再特别说明。}
\begin{align*}
	t' &= \frac{\mathrm{d} t}{\mathrm{d} \tau} \\
	\frac{\mathrm{d}}{\mathrm{d} \tau} & = \frac{\mathrm{d} t}{\mathrm{d} \tau} \frac{\mathrm{d}}{\mathrm{d} t} = t' \frac{\mathrm{d}}{\mathrm{d} t} \\
	\mbf{q}' &= \frac{\mathrm{d} \mbf{q}}{\mathrm{d} \tau} = \frac{\mathrm{d} t}{\mathrm{d} \tau} \frac{\mathrm{d} \mbf{q}}{\mathrm{d} t} = t' \dot{\mbf{q}}
\end{align*}
考虑
\begin{equation*}
	\mathrm{d} L = \sum_{\alpha=1}^s \frac{\pl L}{\pl q_\alpha} \mathrm{d} q_\alpha + \sum_{\alpha=1}^s \frac{\pl L}{\pl \dot{q}_\alpha} \mathrm{d} \dot{q}_\alpha + \frac{\pl L}{\pl t} \mathrm{d} t = \sum_{\alpha=1}^s \dot{p}_\alpha \mathrm{d} q_\alpha + \sum_{\alpha=1}^s p_\alpha \mathrm{d} \dot{q}_\alpha - \dot{H} \mathrm{d} t
\end{equation*}
可有
\begin{equation*}
	t'\mathrm{d} L = \sum_{\alpha=1}^s p'_\alpha \mathrm{d} q_\alpha + \sum_{\alpha=1}^s p_\alpha t' \mathrm{d} \dot{q}_\alpha - H' \mathrm{d} t
\end{equation*}
所以
\begin{align}
	\mathrm{d} (Lt') & = L\mathrm{d} t' + t'\mathrm{d} L = \sum_{\alpha=1}^s p'_\alpha \mathrm{d} q_\alpha - H' \mathrm{d} t + \sum_{\alpha=1}^s p_\alpha t' (\mathrm{d} q'_\alpha - \dot{q}_\alpha \mathrm{d} t') + L\mathrm{d} t' \nonumber \\
	& = \sum_{\alpha=1}^s p'_\alpha \mathrm{d} q_\alpha + p'_t \mathrm{d} t + \sum_{\alpha=1}^s p_\alpha t' \mathrm{d} q'_\alpha + p_t\mathrm{d} t'
	\label{推广的Lagrange函数中间步骤}
\end{align}
由此,定义
\begin{equation*}
	\mathnormal{\Lambda}(\mbf{q},t;\mbf{q}',t';\tau) = L \left(\mbf{q},\frac{\mbf{q}'}{t'},t\right) t'
\end{equation*}
求其全微分,可有
\begin{equation*}
	\mathrm{d} \mathnormal{\Lambda} = \sum_{\alpha=1}^s \frac{\pl \mathnormal{\Lambda}}{\pl q_\alpha} \mathrm{d} q_\alpha + \frac{\pl \mathnormal{\Lambda}}{\pl t} \mathrm{d} t + \sum_{\alpha=1}^s \frac{\pl \mathnormal{\Lambda}}{\pl q'_\alpha} \mathrm{d} q'_\alpha + \frac{\pl \mathnormal{\Lambda}}{\pl t'} \mathrm{d} t' + \frac{\pl \mathnormal{\Lambda}}{\pl \tau} \mathrm{d} \tau
\end{equation*}
将上式与\eqref{推广的Lagrange函数中间步骤}对比,可得
\begin{equation}
	\begin{cases}
		\displaystyle p_\alpha = \frac{\pl \mathnormal{\Lambda}}{\pl q'_\alpha},\quad p_t = \frac{\pl \mathnormal{\Lambda}}{\pl t'} \\[1.5ex]
		\displaystyle p'_\alpha = \frac{\pl \mathnormal{\Lambda}}{\pl q_\alpha},\quad p'_t = \frac{\pl \mathnormal{\Lambda}}{\pl t} \\[1.5ex]
		\displaystyle \frac{\pl \mathnormal{\Lambda}}{\pl \tau} = 0
	\end{cases}
	\label{位形时空的Lagrange方程中间步骤}
\end{equation}
消去式\eqref{位形时空的Lagrange方程中间步骤}中的广义动量,即可得到{\heiti 位形时空的Lagrange方程}
\begin{equation}
	\frac{\mathrm{d}}{\mathrm{d} \tau} \frac{\pl \mathnormal{\Lambda}}{\pl q'_\alpha} - \frac{\pl \mathnormal{\Lambda}}{\pl q_\alpha} = 0,\quad \alpha = 1,2,\cdots,s+1
	\label{位形时空的Lagrange方程}
\end{equation}
由此,$\mbf{q},t$统一为地位平等的坐标,$\mbf{p},H$统一为地位平等的动量,时间均匀性与空间均匀性得到统一。

这个过程对经典力学过渡为相对论力学具有启发性,有助于建立满足相对性原理(Lorentz变换不变性或对称性)的理论体系。时空作为统一体同时变换,时间不再具有绝对参数地位,需另选Lorentz变换下的不变量作为参数。此时$\mathnormal{\Lambda}$在Lorentz变换下不变即能保证体系满足相对性原理(惯性系变换对称性)。

\iffalse
这部分并不是完全基于上节内容,用有限的篇幅很难说清楚,故去掉。
\subsection{狭义相对论的分析力学形式}

狭义相对论下的变换应满足
\begin{equation*}
	\sum_{i=1}^3 \tilde{x}_i^2 - (c\tilde{t})^2 = \sum_{i=1}^3 x_i^2 - (ct)^2
\end{equation*}
取坐标系使得两参考系之间的相对速度方向为$x_1$轴方向,则有{\heiti Lorentz变换}
\begin{equation}
\begin{cases}
	\displaystyle \tilde{x}_1 = \frac{x_1 - Vt}{\sqrt{1-\dfrac{V^2}{c^2}}} \\[1.5ex]
	\tilde{x}_2 = x_2 \\[1.5ex]
	\tilde{x}_3 = x_3 \\[1.5ex]
	\displaystyle \tilde{t} = \frac{t - \dfrac{V}{c^2}x_1}{\sqrt{1-\dfrac{V^2}{c^2}}}
\end{cases}
\end{equation}\fi


%第三章——哈密顿力学
\chapter{Hamilton动力学}

% 理想完整有势系统的Lagrange函数可以表示为
% \begin{equation*}
	% L = T(\mbf{q},\dot{\mbf{q}},t) - U(\mbf{q},\dot{\mbf{q}},t)
% \end{equation*}
% 则系统的Lagrange方程为
% \begin{equation*}
	% \frac{\mathrm{d}}{\mathrm{d} t} \frac{\pl L}{\pl \dot{q}_\alpha} - \frac{\pl L}{\pl q_\alpha} = 0\quad (\alpha = 1,2,\cdots,s)
% \end{equation*}
% 利用系统的对称性可以简化求解上述常微分方程组。

% Lagrange力学有如下不足:
% \begin{enumerate}
% \item 运动学:速度$\dot{\mbf{q}} = \dfrac{\mathrm{d} \mbf{q}}{\mathrm{d} t}$作为坐目标衍生变量出现;
% \item 动力学:坐标、速度的初值$\mbf{q}(t_0),\dot{\mbf{q}}(t_0)$不独立,不能体现坐标、速度在因果律上的独立性;
% \item 在位形空间描述运动,方程为二阶常微分方程组,同一点可能代表系统的不同状态(速度不同的状态),导致运动轨道的不同。
% \end{enumerate}

% 1835年,Hamilton发表了两篇论文,将Lagrange方程化为更对称的一阶方程组,创立了Hamilton动力学,为量子力学、统计物理、量子场论奠定理论基础。

\section{正则方程}

% 广义动量和广义动量定理可以表示为
% \begin{equation*}
	% \begin{cases}
		% \displaystyle p_\alpha = \frac{\pl L}{\pl \dot{q}_\alpha} \\[1.5ex]
		% \displaystyle \dot{p}_\alpha = \frac{\pl L}{\pl q_\alpha}
	% \end{cases}
% \end{equation*}
% 由此定义的广义动量具有
% \begin{equation*}
	% p_\alpha = p_\alpha(\mbf{q},\dot{\mbf{p}},t)
% \end{equation*}
% 的形式,则有
% \begin{equation*}
	% \dot{\mbf{q}} = \dot{\mbf{q}}(\mbf{q},\mbf{p},t)
% \end{equation*}
% 由此,可作力学状态参量变换$\mbf{q},\dot{\mbf{q}} \to \mbf{q},\mbf{p}$,找到新的特征函数,通过对$\mbf{p},\mbf{q}$的偏导数生成力学方程。

\subsection{Legendre变换·Hamilton函数}

描述理想完整系统在有势力作用下的Lagrange方程为
\begin{equation}
	\frac{\mathrm{d}}{\mathrm{d} t} \frac{\pl L}{\pl \dot{q}_\alpha} - \frac{\pl L}{\pl q_\alpha} = 0 \quad (\alpha = 1,2,\cdots,s)
	\label{chapter3:Lagrange方程-重写}
\end{equation}
其中Lagrange函数$L$依赖于变量$\mbf{q},\dot{\mbf{q}}$和$t$,即$L = L(\mbf{q},\dot{\mbf{q}},t)$。这些变量确定了时刻和相应时刻系统的状态,即各质点的位置和速度,因此变量$\mbf{q},\dot{\mbf{q}},t$称为{\bf Lagrange变量}。

系统的状态也可以利用其它参数确定,可以取为$\mbf{q},\mbf{p},t$,其中$\mbf{p}$为广义动量,定义为
\begin{equation}
	p_\alpha = \frac{\pl L}{\pl \dot{q}_\alpha}(\mbf{q},\dot{\mbf{q}},t)\quad (\alpha=1,2,\cdots,s)
	\label{chapter3:广义动量的定义式}
\end{equation}
变量$\mbf{q},\mbf{p},t$称为{\bf Hamilton变量}。

在广义势系统中,Lagrange函数可以按广义速度的次数写为
\begin{equation*}
	L = L_2+L_1+L_0
\end{equation*}
其中$L_2=T_2$\footnote{参见第\ref{chapter2:subsection-系统动能的数学结构}节。},因此必然有
\begin{equation}
	\frac{\pl^2 L}{\pl \dot{q}_\alpha\pl \dot{q}_\beta} = \frac{\pl^2 T_2}{\pl \dot{q}_\alpha\pl \dot{q}_\beta} = M_{\alpha\beta}
\end{equation}
其中$M_{\alpha\beta}$的定义见式\eqref{chapter2:以广义坐标表示的系统动能表达式中的系数},再考虑到式\eqref{chapter2:定常系统动能T2的正定性推论}可得其Hesse行列式非零:
\begin{equation}
	\det\begin{pmatrix} \dfrac{\pl^2 L}{\pl \dot{q}_\alpha\pl \dot{q}_\beta} \end{pmatrix} \neq 0
	\label{chapter3:Lagrange函数的Hesse行列式}
\end{equation}
式\eqref{chapter3:Lagrange函数的Hesse行列式}说明式\eqref{chapter3:广义动量的定义式}右端的Jacobi行列式非零,根据隐函数定理可知,其中的广义速度相对广义动量可解:
\begin{equation}
	\dot{q}_\alpha = \phi_\alpha(\mbf{q},\mbf{p},t)\quad (\alpha=1,2,\cdots,s)
\end{equation}
因此Lagrange变量可以用Hamilton变量表示,反之亦然。

Hamilton提出了用变量$\mbf{q},\mbf{p},t$描述的运动方程,使得Lagrange方程\eqref{chapter3:Lagrange方程-重写}变为$2n$个具有对称形式的对于导数可解的一阶方程,这些方程称为{\bf Hamilton方程}或{\bf 正则方程},变量$\mbf{q}$和$\mbf{q}$称为{\bf 正则共轭变量}或简称为{\bf 正则变量}。

首先给出上文所述变量变换的数学表述。设给定函数$X(x_1,x_2,\cdots,x_n)$,其Hesse行列式非零:
\begin{equation}
	\det\begin{pmatrix} \dfrac{\pl^2 X}{\pl x_i\pl x_j}\end{pmatrix} \neq 0
	\label{chapter3:Legendre变换-前提条件}
\end{equation}
则从变量$x_1,x_2,\cdots,x_n$到$y_1,y_2,\cdots,y_n$的变换由下面的公式定义:
\begin{equation}
	y_i = \frac{\pl X}{\pl x_i} \quad (i=1,2,\cdots,n)
	\label{chapter3:Legendre变换-新变量定义}
\end{equation}
而函数$X(x_1,x_2,\cdots,x_n)$的{\bf Legendre变换}是指由此确定的新函数
\begin{equation}
	Y = Y(y_1,y_2,\cdots,y_n) = \sum_{i=1}^n y_ix_i - X(x_1,x_2,\cdots,x_n)
	\label{chapter3:Legendre变换}
\end{equation}
此处等式\eqref{chapter3:Legendre变换}右端的变量$x_i$需要借助方程\eqref{chapter3:Legendre变换-新变量定义}用新变量$y_1,y_2,\cdots,y_n$表示出来\footnote{由式\eqref{chapter3:Legendre变换-前提条件}可知式\eqref{chapter3:Legendre变换-新变量定义}右端的Jacobi行列式非零,故式\eqref{chapter3:Legendre变换-新变量定义}对$x_i(i=1,2,\cdots,n)$是可解的。}。

Legendre变换有逆变换,其逆变换也是Legendre变换。即再对函数$Y(y_1,y_2,\cdots,y_n)$做一次Legendre变换,令
\begin{equation}
	z_i = \frac{\pl Y}{\pl y_i}\quad (i=1,2,\cdots,n)
	\label{chapter3:Legendre变换-逆变换的新变量定义}
\end{equation}
其中的函数$Y$由Legendre变换式\eqref{chapter3:Legendre变换}得到,此时有
\begin{align*}
	z_i & = \frac{\pl}{\pl y_i}\left(\sum_{j=1}^n y_jx_j - X\right) = x_i + \sum_{j=1}^n y_j\frac{\pl x_j}{\pl y_i} - \sum_{j=1}^n \frac{\pl X}{\pl x_j} \frac{\pl x_j}{\pl y_i} \\
	& = x_i + \sum_{j=1}^n \left(y_j-\frac{\pl X}{\pl x_j}\right) \frac{\pl x_j}{\pl y_i} = x_i \quad (i=1,2,\cdots,n)
\end{align*}
这说明变换式\eqref{chapter3:Legendre变换-逆变换的新变量定义}再次将$y_1,y_2,\cdots,y_n$变回$x_1,x_2,\cdots,x_n$,同时
\begin{equation}
	Z = \sum_{i=1}^n z_iy_i - Y = \sum_{i=1}^n x_iy_i - \left(\sum_{i=1}^n y_ix_i - X\right) = X
\end{equation}
也将$Y$重新变为$X$。

系统的Lagrange函数$L(\mbf{q},\dot{\mbf{q}},t)$对变量$\dot{\mbf{q}}$的Legendre变换为函数
\begin{equation}
	H(\mbf{q},\mbf{p},t) = \sum_{\alpha=1}^s p_\alpha\dot{q}_\alpha - L(\mbf{q},\dot{\mbf{q}},t)
	\label{chapter3:Hamilton函数定义}
\end{equation}
其中变量$\dot{q}_\alpha(\alpha=1,2,\cdots,s)$需要借方程\eqref{chapter3:广义动量的定义式}用$\mbf{q},\mbf{p},t$表示出来。这个函数$H$称为{\bf Hamilton函数}。

\begin{example}[Legendre变换的几何意义]
以单个变量的情形来考虑Legendre变换的几何意义。设变量为$x$,相应地有函数$X(x)$,要求这个函数是严格的凸(或凹)函数,这样即满足了式\eqref{chapter3:Legendre变换-前提条件}\footnote{这是由于函数具有连续的二阶导数,所以在一个点的邻域内满足式\eqref{chapter3:Legendre变换-前提条件}则必然有在该邻域内二阶导数都为正或负。},由此定义新变量$y$为
\begin{equation*}
	y = \dif{X}{x}
\end{equation*}
便得到函数$X(x)$的Legendre变换:
\begin{equation*}
	Y = Y(y) = xy - X(x)
\end{equation*}
其中$x=x(y)$。从图形上来看,$y$对应于曲线$X(x)$在$x$点处切线的斜率,而$Y(y)$则为切线在纵轴上截距的负值(如图\ref{chapter3:figure-Legendre变换的几何意义}所示)。

\begin{figure}[htb]
\centering
\begin{minipage}[t]{0.45\textwidth}
\centering
\begin{asy}
	size(200);
	real a,b,h;
	real f(real x){
		return a*(x-b)^2+h;
	}
	real df(real x){
		return 2*a*(x-b);
	}
	a = 0.3;
	b = 1;
	h = 0.5;
	real x1,x2,y1,y2;
	x1 = -1;
	x2 = 3.5;
	y1 = -1;
	y2 = 2;
	draw(Label("$x$",EndPoint),(x1,0)--(x2,0),Arrow);
	draw((0,y1)--(0,y2),Arrow);
	draw(graph(f,-0.5,3),linewidth(0.8bp));
	real x0,y0,k;
	real l(real x){
		return y0+k*(x-x0);
	}
	x0 = 2.3;
	y0 = f(x0);
	k = df(x0);
	draw(graph(l,0,3),linewidth(0.8bp));
	real l1,l2;
	l1 = 0.3;
	l2 = 0.2;
	draw(Label("$Y(y)$",MidPoint,W),(-l2,0)--(-l2,l(0)),Arrows);
	draw((-l1,l(0))--(0,l(0)));
	draw((0,l(0))--(x0,l(0))--(x0,y0),dashed);
	label("$x$",(x0,0),SW);
	l2 = 0.1;
	draw(Label(scale(.85)*rotate(-90)*"$X(x)$",MidPoint,E),(x0+l2,y0)--(x0+l2,0),Arrows);
	l2 = 0.5;
	draw(Label(scale(.85)*rotate(-90)*"$xy$",Relative(0.3),E),(x0+l2,y0)--(x0+l2,l(0)),Arrows);
	l1 = 0.6;
	draw((x0,y0)--(x0+l1,y0));
	draw((x0,l(0))--(x0+l1,l(0)));
\end{asy}
\caption{Legendre变换的几何意义}
\label{chapter3:figure-Legendre变换的几何意义}\end{minipage}
\hspace{0.5cm}
\begin{minipage}[t]{0.45\textwidth}
\centering
\begin{asy}
	size(200);
	real a,b,h,x0;
	real f(real x){
		return a*(x-b)^2+h;
	}
	real df(real x){
		return 2*a*(x-b);
	}
	real tanl(real x){
		return f(x0)+df(x0)*(x-x0);
	}
	a = 0.3;
	b = 1;
	h = 0.5;
	real x1,x2,y1,y2;
	x1 = -1.2;
	x2 = 3.2;
	y1 = -1;
	y2 = 2;
	draw(Label("$x$",EndPoint),(x1,0)--(x2,0),Arrow);
	draw((0,y1)--(0,y2),Arrow);
	picture pic;
	for(x0=-1;x0<=3;x0=x0+0.1){
		draw(pic,graph(tanl,-1,3),linewidth(0.8bp));
	}
	path clp;
	clp = box((x1,y1),(x2,y2));
	clip(pic,clp);
	add(pic);
	draw(graph(f,-1,3),linewidth(0.8bp)+red);
\end{asy}
\caption{曲线作为切线的包络线}
\label{chapter3:figure-曲线作为切线的包络线}
\end{minipage}
\end{figure}

在变换前,用坐标$(x,X(x))$来描述曲线,曲线是一系列这样的坐标点的集合。变换后用切线的斜率$y$和截距$Y(y)$来描述曲线,曲线是各种斜率的切线的包络线(如图\ref{chapter3:figure-曲线作为切线的包络线}所示)。这两种表述是等价的,包含了相同的信息。
\end{example}

\subsection{Hamilton正则方程}

Hamilton函数$H=H(\mbf{q},\mbf{p},t)$的全微分为
\begin{equation}
	\mathd H = \sum_{\alpha=1}^s \frac{\pl H}{\pl p_\alpha} \mathrm{d} p_\alpha + \sum_{\alpha=1}^s \frac{\pl H}{\pl q_\alpha} \mathrm{d} q_\alpha + \frac{\pl H}{\pl t} \mathrm{d} t
	\label{chapter3:Hamilton函数的全微分1}
\end{equation}
另一方面,直接对式\eqref{chapter3:Hamilton函数定义}右端求全微分可得
\begin{align}
	\mathd H & = \sum_{\alpha=1}^s \dot{q}_\alpha\mathd p_\alpha + \sum_{\alpha=1}^s p_\alpha\mathd \dot{q}_\alpha - \left(\sum_{\alpha=1}^s \frac{\pl L}{\pl q_\alpha}\mathd q_\alpha + \sum_{\alpha=1}^s \frac{\pl L}{\pl \dot{q}_\alpha}\mathd \dot{q}_\alpha + \frac{\pl L}{\pl t} \mathd t\right) \nonumber \\
	& = \sum_{\alpha=1}^s \left(p_\alpha - \frac{\pl L}{\pl \dot{q}_\alpha}\right) \mathd \dot{q}_\alpha + \sum_{\alpha=1}^s \dot{q}_\alpha\mathd p_\alpha - \sum_{\alpha=1}^s \frac{\pl L}{\pl \dot{q}_\alpha}\mathd \dot{q}_\alpha - \frac{\pl L}{\pl t} \mathd t \nonumber \\
	& = \sum_{\alpha=1}^s \dot{q}_\alpha\mathd p_\alpha - \sum_{\alpha=1}^s \frac{\pl L}{\pl \dot{q}_\alpha}\mathd \dot{q}_\alpha - \frac{\pl L}{\pl t} \mathd t
	\label{chapter3:Hamilton函数的全微分2}
\end{align}
对比式\eqref{chapter3:Hamilton函数的全微分1}和式\eqref{chapter3:Hamilton函数的全微分2}可得
\begin{equation}
	\frac{\pl H}{\pl q_\alpha} = -\frac{\pl L}{\pl q_\alpha},\quad \frac{\pl H}{\pl p_\alpha} = \dot{q}_\alpha\quad (\alpha=1,2,\cdots,s)
	\label{chapter3:正则方程-初步}
\end{equation}
以及
\begin{equation}
	\frac{\pl H}{\pl t} = -\frac{\pl L}{\pl t}
	\label{chapter3:正则方程-初步推论}
\end{equation}
根据Lagrange方程\eqref{chapter3:Lagrange方程-重写}和式\eqref{chapter3:广义动量的定义式}可得
\begin{equation*}
	\dot{p}_\alpha = \frac{\pl L}{\pl q_\alpha}\quad (\alpha=1,2,\cdots,s)
\end{equation*}
因此由式\eqref{chapter3:正则方程-初步}可得运动方程
\begin{equation}
\begin{cases}
	\dot{q}_\alpha = \dfrac{\pl H}{\pl p_\alpha} \\[1.5ex]
	\dot{p}_\alpha = -\dfrac{\pl H}{\pl q_\alpha} 
\end{cases}
(\alpha = 1,2,\cdots,s)
\label{chapter3:Hamilton正则方程}
\end{equation}
这些方程称为{\bf Hamilton方程}或{\bf 正则方程}。%根据式\eqref{广义能量},Hamilton函数$H = H(\mbf{q},\mbf{p},t)$即为以正则变量表示的广义能量。

顺便我们根据式\eqref{chapter3:正则方程-初步推论}可知,如果系统的Lagrange函数不显含时间,则Hamilton函数也不显含时间,反之亦然。类似地,根据式\eqref{chapter3:正则方程-初步}可知,如果系统的Lagrange函数不显含某个广义坐标,则Hamilton函数也不显含这个广义坐标,反之亦然。

由正则变量张成了一个$2s$维空间,称为{\heiti 相空间}。相空间中的点称为{\heiti 相点},代表系统的力学状态。相点在相空间运动而描述的轨道称为{\heiti 相轨道},反映了系统状态的演化。

正则方程是由$2s$个一阶常微分方程构成的常微分方程组,决定了相点的速度。给定正则变量的$2s$个初始值,可以唯一确定正则变量的时间演化关系。

% \subsection{守恒定律}

% 根据正则方程,如果Hamilton函数中不显含坐标$q_\alpha$,则有
% \begin{equation*}
	% \dot{p}_\alpha = -\frac{\pl H}{\pl q_\alpha} = 0
% \end{equation*}
% 即有与该坐标共轭的动量$p_\alpha$守恒。如果Hamilton函数中不显含时间$t$,则有
% \begin{equation*}
	% \dot{H} = \frac{\pl H}{\pl t} = 0
% \end{equation*}
% 即有广义能量守恒。

% 在Hamilton力学中,系统的对称性表现为$H$的变换不变性。

\subsection{Hamilton函数的物理意义·广义能量积分}

将Hamilton函数的形式\eqref{chapter3:Hamilton函数定义}与式\eqref{广义能量}对比,可知Hamilton函数$H = H(\mbf{q},\mbf{p},t)$即为以正则变量表示的广义能量。

下面考虑Hamilton函数对时间的全导数,即
\begin{align*}
	\frac{\mathd H}{\mathd t} & = \sum_{\alpha=1}^s \dot{p}_\alpha\frac{\pl H}{\pl p_\alpha} + \sum_{\alpha=1}^s \dot{q}_\alpha\frac{\pl H}{\pl q_\alpha} + \frac{\pl H}{\pl t} = \sum_{\alpha=1}^s \left(\frac{\pl H}{\pl q_\alpha}\frac{\pl H}{\pl p_\alpha} - \frac{\pl H}{\pl p_\alpha}\frac{\pl H}{\pl q_\alpha}\right) + \frac{\pl H}{\pl t} = \frac{\pl H}{\pl t}
\end{align*}
即Hamilton函数对时间的全导数恒等于它对时间的偏导数
\begin{equation}
	\frac{\mathd H}{\mathd t} = \frac{\pl H}{\pl t}
\end{equation}

如果Hamilton函数不显含时间,则称系统是{\bf 广义保守的}。在这种情况下有$\ds \frac{\mathd H}{\mathd t} = \frac{\pl H}{\pl t} = 0$,因此在系统的运动过程中恒有
\begin{equation}
	H(\mbf{q},\mbf{p}) = E
	\label{chapter3:广义能量积分}
\end{equation}
其中$E$为任意常数,式\eqref{chapter3:广义能量积分}称为{\bf 广义能量积分}。

\begin{example}[一维谐振子]
写出一维谐振子的正则方程和相空间轨迹。
\end{example}
\begin{solution}
一维谐振子的Lagrange函数为
\begin{equation*}
	L = \frac12 m\dot{q}^2 - \frac12 kq^2
\end{equation*}
考虑$p = \dfrac{\pl L}{\pl \dot{q}} = m\dot{q}$,则有
\begin{equation*}
	\dot{q} = \frac{p}{m}
\end{equation*}
所以Hamilton函数为
\begin{equation*}
	H = p\dot{q} - L = \frac{p^2}{2m} + \frac12 kq^2
\end{equation*}
正则方程为
\begin{equation*}
	\begin{cases}
		\dot{q} = \dfrac{\pl H}{\pl p} = \dfrac{p}{m} \\[1.5ex]
		\dot{p} = -\dfrac{\pl H}{\pl q} = -kq
	\end{cases}
\end{equation*}
消去广义动量$p$即可得到系统的动力学方程
\begin{equation*}
	m\ddot{q} + kq = 0
\end{equation*}
由于
\begin{equation*}
	\dot{H} = \frac{\pl H}{\pl t} = 0
\end{equation*}
所以$H = E(\text{常数})$,相空间轨迹即为
\begin{equation*}
	\frac{p^2}{2mE} + \frac{q^2}{2E/k} = 1
\end{equation*}
是一个椭圆。
\end{solution}

\begin{example}[电磁场中带电粒子的Hamilton函数]
求电磁场中带电粒子的Hamilton函数。
\end{example}
\begin{solution}
电磁场中带电粒子的Lagrange函数为
\begin{equation*}
	L(\mbf{r},\dot{\mbf{r}},t) = \frac12 m \dot{\mbf{r}}^2 + e\mbf{A}(\mbf{r},t) \cdot \dot{\mbf{r}} - e\phi(\mbf{r},t)
\end{equation*}
则有
\begin{equation*}
	\mbf{p} = \frac{\pl L}{\pl \dot{\mbf{r}}} = m\dot{\mbf{r}} + e\mbf{A}
\end{equation*}
所以
\begin{equation*}
	\dot{\mbf{r}} = \frac{\mbf{p} - e\mbf{A}}{m}
\end{equation*}
由此可得Hamilton函数
\begin{equation*}
	H = \mbf{p} \cdot \dot{\mbf{r}} - L = \frac{1}{2m} (\mbf{p}-e\mbf{A})^2 + e\phi
\end{equation*}
\end{solution}

\subsection{正则方程的矩阵形式}

记
\begin{equation*}
	\begin{array}{l}
	\displaystyle \mbf{\xi} = \begin{pmatrix} \mbf{q} \\ \mbf{p} \end{pmatrix} = \begin{pmatrix} q_1 \\ \vdots \\ q_s \\ p_1 \\ \vdots \\ p_s \end{pmatrix},\quad \mbf{J} = \begin{pmatrix} \mbf{0} & \mbf{I} \\ -\mbf{I} & \mbf{0} \end{pmatrix} = \begin{pmatrix} 0 & \cdots & 0 & 1 & \cdots & 0 \\ \vdots & & \vdots & \vdots & & \vdots \\ 0 & \cdots & 0 & 0 & \cdots & 1 \\ -1 & \cdots & 0 & 0 & \cdots & 0 \\ \vdots & & \vdots & \vdots & & \vdots \\ 0 & \cdots & -1 & 0 & \cdots & 0 \end{pmatrix} \\
	\displaystyle \frac{\pl H}{\pl \mbf{\xi}} = \begin{pmatrix} \dfrac{\pl H}{\pl \mbf{q}} \\[1.5ex] \dfrac{\pl H}{\pl \mbf{p}} \end{pmatrix} = \begin{pmatrix} \dfrac{\pl H}{\pl q_1} \\[1.5ex] \vdots \\[1.5ex] \dfrac{\pl H}{\pl q_s} \\[1.5ex] \dfrac{\pl H}{\pl p_1} \\[1.5ex] \vdots \\[1.5ex] \dfrac{\pl H}{\pl p_s} \end{pmatrix}
	\end{array}
\end{equation*}
则正则方程可以表示为
\begin{equation}
	\dot{\mbf{\xi}} = \mbf{J} \frac{\pl H}{\pl \mbf{\xi}}
	\label{正则方程的辛形式}
\end{equation}
式\eqref{正则方程的辛形式}称为{\heiti 正则方程的矩阵形式}或{\heiti 正则方程的辛(sympletic)形式}。

可通过直接计算的方式验证矩阵$\mbf{J}$满足如下性质:
\begin{equation}
	\mbf{J}^2 = -\mbf{I}_{2s},\quad \mbf{J}^{\mathrm{T}} = \mbf{J}^{-1} = -\mbf{J},\quad \det \mbf{J} = 1
\end{equation}

\subsection{Whittaker方程与Jacobi方程}

设系统的运动由正则方程
\begin{equation}
\begin{cases}
	\dot{p}_\alpha= \dfrac{\pl H}{\pl q_\alpha} \\[1.5ex]
	\dot{q}_\alpha= -\dfrac{\pl H}{\pl p_\alpha}
\end{cases}
\label{chapter3:正则方程-重写}
\end{equation}
描述,如果系统的Hamilton函数不显含时间,则系统存在广义能量积分
\begin{equation}
	H(q_1,q_2,\cdots,q_s,p_1,p_2,\cdots,p_s) = E
	\label{chapter3:广义能量积分-重写}
\end{equation}
其中$E$是由初始条件确定的常数$E=H(q_1^0,q_2^0,\cdots,q_s^0,p_1^0,p_2^0,\cdots,p_s^0)$。在相空间中,方程\eqref{chapter3:广义能量积分-重写}确定了一个曲面,系统的相点必然落在该曲面上。

假设在相空间的某个区域内有$\dfrac{\pl H}{\pl p_1}\neq 0$,那么在该区域内式\eqref{chapter3:广义能量积分-重写}对于$p_1$可解,记作
\begin{equation}
	p_1 = -K(q_1,q_2,\cdots,q_s,p_2,p_3,\cdots,p_s,E)
	\label{chapter3:惠特克方程的分离变量}
\end{equation}
由此可将正则方程\eqref{chapter3:正则方程-重写}分为两组
\begin{equation}
\begin{cases}
	\ds \frac{\mathd q_1}{\mathd t} = \frac{\pl H}{\pl p_1},\quad \frac{\mathd p_1}{\mathd t} = -\frac{\pl H}{\pl q_1} \\[1.5ex]
	\ds \frac{\mathd q_\alpha}{\mathd t} = \frac{\pl H}{\pl p_\alpha},\quad \frac{\mathd p_\alpha}{\mathd t} = -\frac{\pl H}{\pl q_\alpha}\quad (\alpha=2,3,\cdots,s)
\end{cases}
\label{chapter3:分组的正则方程}
\end{equation}
根据式\eqref{chapter3:分组的正则方程}可得
\begin{equation}
\begin{cases}
	\ds \frac{\mathd q_\alpha}{\mathd q_1} = \frac{\dfrac{\pl H}{\pl p_\alpha}}{\dfrac{\pl H}{\pl p_1}} \\[3.0ex]
	\ds \frac{\mathd p_\alpha}{\mathd q_1} = -\frac{\dfrac{\pl H}{\pl q_\alpha}}{\dfrac{\pl H}{\pl p_1}} 
\end{cases}
(\alpha = 2,3,\cdots,s)
\label{chapter3:惠特克方程-中间结果1}
\end{equation}
将式\eqref{chapter3:广义能量积分-重写}两端对$p_\alpha(\alpha=2,3,\cdots,s)$求偏导数,并注意到式\eqref{chapter3:惠特克方程的分离变量}可得
\begin{equation}
	\frac{\pl H}{\pl p_\alpha}-\frac{\pl H}{\pl p_1}\frac{\pl K}{\pl p_\alpha} = 0
	\label{chapter3:惠特克方程-中间结果2}
\end{equation}
由式\eqref{chapter3:惠特克方程-中间结果2}可将方程组\eqref{chapter3:惠特克方程-中间结果1}中的第一式改写为
\begin{equation*}
	\frac{\mathd q_\alpha}{\mathd q_1} = \frac{\pl K}{\pl p_\alpha}
\end{equation*}
同理,将式\eqref{chapter3:广义能量积分-重写}两端对$q_\alpha(\alpha=2,3,\cdots,s)$求偏导数,并注意到式\eqref{chapter3:惠特克方程的分离变量}可得
\begin{equation}
	\frac{\pl H}{\pl q_\alpha}-\frac{\pl H}{\pl p_1}\frac{\pl K}{\pl q_\alpha} = 0
	\label{chapter3:惠特克方程-中间结果3}
\end{equation}
由式\eqref{chapter3:惠特克方程-中间结果3}可将方程组\eqref{chapter3:惠特克方程-中间结果1}中的第二式改写为
\begin{equation*}
	\frac{\mathd p_\alpha}{\mathd q_1} = -\frac{\pl K}{\pl q_\alpha}
\end{equation*}
综上有
\begin{equation}
\begin{cases}
	\ds \frac{\mathd q_\alpha}{\mathd q_1} = \frac{\pl K}{\pl p_\alpha} \\[1.5ex]
	\ds \frac{\mathd p_\alpha}{\mathd q_1} = -\frac{\pl K}{\pl q_\alpha}
\end{cases}
(\alpha=2,3,\cdots,s)
\label{chapter3:惠特克方程}
\end{equation}
方程\eqref{chapter3:惠特克方程}描述系统在$H=E\,\text{(常数)}$时的运动,称为{\bf Whittaker方程}。如果将方程\eqref{chapter3:惠特克方程}中的函数$K$看作Hamilton函数,将坐标$q_1$看作时间,则Whittaker方程具有正则方程的形式。

积分Whittaker方程\eqref{chapter3:惠特克方程}将得到
\begin{equation}
\begin{cases}
	q_\alpha = q_\alpha(q_1,E,C_1,C_2,\cdots,C_{2n-2}) \\
	p_\alpha = p_\alpha(q_1,E,C_1,C_2,\cdots,C_{2n-2})
\end{cases}
\label{chapter3:惠特克方程的解1}
\end{equation}
其中$C_1,C_2,\cdots,C_{2n-2}$为$2n-2$个积分常数。将Whittaker方程的解\eqref{chapter3:惠特克方程的解1}代入式\eqref{chapter3:惠特克方程的分离变量}中,可得
\begin{equation}
	p_1 = p_1(q_1,E,C_1,C_2,\cdots,C_{2n-2})
	\label{chapter3:惠特克方程的解2}
\end{equation}
式\eqref{chapter3:惠特克方程的解1}和\eqref{chapter3:惠特克方程的解2}确定了系统在相空间中相轨迹的方程。再利用正则方程\eqref{chapter3:分组的正则方程}中的第一式即可得到$q_1$与时间的关系,进而得到系统运动与时间的关系。

Whittaker方程\eqref{chapter3:惠特克方程}具有与Hamilton方程相同的结构,也可以写成类似Lagrange方程的形式。设函数$K$对$p_\alpha(\alpha=2,3,\cdots,s)$的Hesse行列式非零,对其做一次Legendre变换,记作
\begin{equation}
	P = P(q_2,q_3,\cdots,q_s,q_2',q_3',\cdots,q_s',q_1,E) = \sum_{\alpha=2}^s q_\alpha'p_\alpha-K
\end{equation}
其中$q_\alpha' = \dfrac{\mathd q_\alpha}{\mathd q_1}$,式中的$p_\alpha$需要从Whittaker方程\eqref{chapter3:惠特克方程}的前$n-1$个方程
\begin{equation*}
	q_\alpha' = \frac{\pl K}{\pl p_\alpha}\quad (\alpha=2,3,\cdots,s)
\end{equation*}
中解出,用$q_1,q_2,\cdots,q_s$和$q_2',q_3',\cdots,q_s'$表示出来。

利用函数$P$可将Whittaker方程\eqref{chapter3:惠特克方程}表示为下面的等价形式
\begin{equation}
	\frac{\mathd}{\mathd t}\frac{\pl P}{\pl q_\alpha'} - \frac{\pl P}{\pl q_\alpha} = 0\quad (\alpha=2,3,\cdots,s)
	\label{chapter3:Jacobi方程}
\end{equation}
方程\eqref{chapter3:Jacobi方程}称为Jacobi方程。如果将Jacobi方程中的函数$P$看作Lagrange函数,将坐标$q_1$看作时间,则Jacobi方程具有Lagrange方程的形式。

下面讨论一下函数$P$的具体形式。考虑到$q_1'\equiv 1$以及式\eqref{chapter3:惠特克方程的分离变量}可得
\begin{equation}
	P = \sum_{\alpha=2}^sq_\alpha'p_\alpha+p_1 = \sum_{\alpha=1}^s q_\alpha'p_\alpha = \frac{1}{\dot{q}_1} \sum_{\alpha=1}^n p_\alpha\dot{q}_\alpha = \frac{1}{\dot{q}_1}(L+H)
\end{equation}
如果系统是保守的,那么$L=T-V, H=T+V$,由此可得
\begin{equation}
	P = \frac{2T}{\dot{q}_1}
	\label{chapter3:Jacobi方程中Jacobi函数的结构1}
\end{equation}
而在保守系统中有
\begin{equation}
	T = T_2 = \frac12 \sum_{\alpha,\beta=1}^s M_{\alpha\beta}\dot{q}_\alpha\dot{q}_\beta = \dot{q}_1^2 \left(\frac12 \sum_{\alpha,\beta=1}^s M_{\alpha\beta}q'_\alpha q'_\beta\right)
	\label{chapter3:Jacobi方程中Jacobi函数的结构2}
\end{equation}
记
\begin{equation}
	G = \sum_{\alpha,\beta=1}^s M_{\alpha\beta}q'_\alpha q'_\beta
\end{equation}
保守系统有能量积分$T+V=E$,将式\eqref{chapter3:Jacobi方程中Jacobi函数的结构2}代入,可得
\begin{equation*}
	\dot{q}_1 = \sqrt{\frac{E-V}{G}}
\end{equation*}
最后,根据式\eqref{chapter3:Jacobi方程中Jacobi函数的结构1}和\eqref{chapter3:Jacobi方程中Jacobi函数的结构2}可得在保守系统中,函数$P$的形式:
\begin{equation}
	P = 2\sqrt{(E-V)G}
\end{equation}

\subsection{数学摆和物理摆·单自由度保守系统的运动}\label{chapter3:subsection-数学摆和物理摆·单自由度保守系统的运动}

\subsubsection{相平面与摆运动的定性图像}

数学摆即单摆,其广义坐标选取和位形空间已在例\ref{chapter2:example-单摆}中讨论过,而单摆的微振动近似解则在例\ref{chapter5:example-单摆}中进行过详细讨论。

在重力作用下绕固定水平轴转动的刚体称为{\bf 物理摆}或{\bf 复摆}。此处取空间系$OXYZ$使得$OZ$轴与刚体转动轴重合,$OY$轴竖直向下。再取本系统$Oxyz$,使刚体质心位于$Oy$轴上,而$Oz$轴与$OZ$轴重合(见第\ref{chapter6:subsection-运动方程与约束反力}节图\ref{chapter6:刚体的定轴转动与约束反力})。如果记刚体质心到转轴的距离为$L$,刚体的自转角设为$\psi$,则$M_z=-mgL\sin\psi$,由刚体定轴转动的运动微分方程\eqref{chapter6:定轴转动刚体动力学方程的分量形式-6}可得物理摆的运动微分方程为
\begin{equation}
	\ddot{\psi}+\frac{mgL}{I_{33}}\sin\psi = 0
	\label{chapter3:物理摆的运动微分方程}
\end{equation}
将方程\eqref{chapter3:物理摆的运动微分方程}与单摆的运动方程
\begin{equation*}
	\ddot{\theta} + \frac{g}{l} \sin\theta = 0
\end{equation*}
比较,可知物理摆的运动规律与摆长为$l=\dfrac{I_{33}}{mL}$的单摆相同,这个摆长称为物理摆的{\bf 等价摆长}。

方程\eqref{chapter3:物理摆的运动微分方程}代表了一种特殊的单自由度系统,即单自由度保守系统,其运动方程可以统一表示为
\begin{equation}
	\ddot{x} = f(x)
	\label{chapter3:单自由度保守系统的运动微分方程}
\end{equation}
其中$x$为广义坐标,则该系统的动能和势能可以如下确定
\begin{equation}
	T = \frac12 \dot{x}^2 ,\quad V = -\int f(x)\mathd x
\end{equation}
由此可得,系统的Hamilton函数为
\begin{equation}
	H(x,p_x) = \frac12 p_x^2 + V(x)
\end{equation}
由于Hamilton函数不显含时间,故系统有广义能量积分\footnote{对于保守系统即为机械能守恒。}
\begin{equation}
	H(x,p_x) = \frac12 p_x^2 + V(x) = E\quad \text{(常数)}
	\label{chapter3:单自由度保守系统的广义能量积分}
\end{equation}
系统的正则方程为
\begin{equation}
\begin{cases}
	\dot{x} = \dfrac{\pl H}{\pl p_x} = p_x \\[1.5ex]
	\dot{p}_x = -\dfrac{\pl H}{\pl x} = f(x)
\end{cases}
\label{chapter3:单自由度保守系统的正则方程}
\end{equation}
如果将$p_x$简记为$y$,则称平面$Oxy$为方程\eqref{chapter3:单自由度保守系统的运动微分方程}的{\bf 相平面},在函数$f(x)$确定的相平面的每个点上,正则方程\eqref{chapter3:单自由度保守系统的正则方程}给出一个以$\dot{x},\dot{y}$为分量的向量,该向量称为{\bf 相速度}\footnote{这里的相速度指的是相点的速度,与机械波的相速度是完全不相干的量。}。正则方程\eqref{chapter3:单自由度保守系统的正则方程}的解给出了相点在相平面上的运动,并且相点的运动速度等于该点所在位置的相速度。相点在相平面上的轨道称为相轨道,在特殊情况下,相轨道可以仅由一个点构成,这样的点称为{\bf 平衡位置},在平衡位置处相速度等于零。

利用运动方程的首次积分\eqref{chapter3:单自由度保守系统的广义能量积分}很容易得到相轨道,在每一条相轨道上机械能都是常数,所以每条相轨道对应一个能量条件$H(x,y)=E$。将首次积分\eqref{chapter3:单自由度保守系统的广义能量积分}改写为
\begin{equation}
	y^2 = 2(E-V(x))
	\label{chapter3:单自由度保守系统的广义能量积分-改写}
\end{equation}
则可看出相轨道具有下列便于分析方程\eqref{chapter3:单自由度保守系统的运动微分方程}的性质:
\begin{enumerate}
\item 当给定$E$时,由于方程\eqref{chapter3:单自由度保守系统的广义能量积分-改写}的左端是非负的,因此相轨迹只能分布在相平面上满足不等式$V(x)\leqslant E$的区域内,这个区域称为{\bf 可能运动区域}。

\item 由正则方程\eqref{chapter3:单自由度保守系统的正则方程}可知,平衡位置位于相平面的$x$轴上,并且在平衡位置$(x_*,0)$处,势能取极值,即$\dif{V}{x}(x_*)=0$。

\item 如果$x=x_*$是函数$V(x)$的极小值点(即该点$\dfrac{\mathd^2 V}{\mathd x^2}(x_*)>0$),则相平面上的点$(x_*,0)$是中心型奇点;如果$x=x_*$是函数$V(x)$的极大值点(即该点$\dfrac{\mathd^2 V}{\mathd x^2}(x_*)<0$),则相平面上的点$(x_*,0)$是鞍型奇点。

\item 相轨道相对于$x$轴对称。

\item 在$x$轴上非平衡位置的点,经过该点的相轨道垂直于$x$轴。这是因为在这些点$\dot{x}=0$但$\dot{y}=f(x)\neq 0$。
\end{enumerate}

根据这些性质,只要画出函数$V(x)$的曲线就可以得到方程\eqref{chapter3:单自由度保守系统的运动微分方程}所描述运动的特性。

\begin{figure}[htb]
\centering
\begin{asy}
	import contour;
	size(350);
	//
	real alpha,beta,Vm,V0,xm;
	real V(real x){
		return Vm*exp(alpha*x)*cos(beta*x)+V0;
	}
	Vm = 1.5;
	V0 = 1.25;
	alpha = -0.35;
	beta = 2;
	xm = 4.5;
	path p = graph(V,0.1,xm);
	draw(p,linewidth(0.8bp));
	real x,y;
	x = 5;
	y = 3;
	draw(Label("$x$",EndPoint),(0,0)--(x,0),Arrow);
	draw(Label("$V(x)$",EndPoint,W),(0,0)--(0,y),Arrow);
	real E1,E2,E3,E4;
	real xmin,xmax;
	xmin = (atan(alpha/beta)+pi)/beta;
	xmax = (atan(alpha/beta)+2*pi)/beta;
	E1 = V(xmin);
	E2 = 1.1;
	E3 = V(xmax);
	E4 = 2.5;
	draw(Label("$E_1$",BeginPoint),(0,E1)--(x,E1));
	draw(Label("$E_2$",BeginPoint),(0,E2)--(x,E2));
	draw(Label("$E_3$",BeginPoint),(0,E3)--(x,E3));
	draw(Label("$E_4$",BeginPoint),(0,E4)--(x,E4));
	real H(real x,real y){
		return 0.5*y^2+V(x);
	}
	picture pic;
	real xm,ym,h;
	xm = 4.5;
	ym = 2.5;
	h = -2;
	guide[][] g = contour(H,(0,-ym),(xm,ym),new real[]{E1,E2,E4},100);
	draw(pic,g);
	guide[][] g = contour(H,(0,-ym),(xm,ym),new real[]{E3},200);
	draw(pic,g);
	draw(pic,Label("$x$",EndPoint),(0,0)--(x,0),Arrow);
	draw(pic,Label("$y$",EndPoint,W),(0,-ym)--(0,ym),Arrow);
	dot(pic,"$E_1$",(xmin,0),S);
	dot(pic,(xmax,0));
	label(pic,"$E_2$",(2.4,0.25));
	label(pic,"$E_2$",(xm,0.6),E);
	label(pic,"$E_3$",(2.8,.65));
	label(pic,"$E_3$",(xm,1.2),E);
	label(pic,"$E_4$",(1.9,2.2));
	label(pic,"$E_4$",(xm,1.65),E);
	add(shift(0,h)*yscale(0.6)*pic);
	pair P4;
	P4 = intersectionpoint(p,(0,E4)--(x,E4));
	draw(P4--(P4.x,h),dashed);
	pair P3;
	P3 = intersectionpoint(p,(0,E3)--(x,E3));
	draw(P3--(P3.x,h),dashed);
	pair[] P2;
	P2 = intersectionpoints(p,(0,E2)--(x,E2));
	draw(P2[0]--(P2[0].x,h),dashed);
	draw(P2[1]--(P2[1].x,h),dashed);
	draw(P2[2]--(P2[2].x,h),dashed);
	draw((xmin,V(xmin))--(xmin,h),dashed);
	draw((xmax,V(xmax))--(xmax,h),dashed);
\end{asy}
\caption{势能曲线与对应的相轨道}
\label{chapter3:figure-势函数与对应的相轨道}
\end{figure}

图\ref{chapter3:figure-势函数与对应的相轨道}中给出了势能曲线与相对应的相轨道的例子。$E=E_1$是中心型平衡位置,这个平衡位置被封闭的相轨道包围。在$E>E_3$时,相轨道不封闭,$E=E_3$是鞍型平衡位置。当$E=E_3$时,相轨道在初始时刻起于鞍点附近,当$t\to\infty$时回到鞍点,这条曲线是围绕中心型平衡位置的封闭曲线族和相应于$E>E_3$的不封闭曲线族的分界线。这种分离不同性质相轨道区域的轨道称为{\bf 分离线}。

通过摆\footnote{由于单摆和物理摆的运动方程是相同的,这里将它们统称为“摆”,并统一用等效摆长表示它们的运动方程。}的运动微分方程\eqref{chapter3:物理摆的运动微分方程}可知,摆的动能和势能分别为
\begin{equation}
	T = \frac12\dot{\psi}^2 ,\quad V = -\frac gl\cos\psi
\end{equation}
如果令$\omega_0^2=\dfrac gl$,则$V=-\omega_0^2\cos\psi$,则能量积分可以写作
\begin{equation}
	\frac12 \dot{\psi}^2 + V = E \quad\text{(常数)}
	\label{chapter3:摆的能量积分}
\end{equation}

\begin{figure}[htb]
\centering
\begin{asy}
	import contour;
	size(400);
	//
	real omega,V_sy,sy;
	real V(real x){
		return -V_sy*omega^2*cos(x);
	}
	real x,y,xm,ym;
	x = 12.5;
	y = 3.5;
	V_sy = 3.75;
	xm = 3.75*pi;
	omega = 0.8;
	draw(Label("$\psi$",EndPoint),(-x,0)--(x,0),Arrow);
	draw(Label("$V(\psi)$",EndPoint,W),(0,-y)--(0,y),Arrow);
	path p;
	p = graph(V,-xm,xm,1000);
	draw(p,linewidth(0.8bp));
	draw(Label("$E=-\omega_0^2$",BeginPoint,S),(-xm-0.25,-V_sy*omega^2)--(xm+0.25,-V_sy*omega^2));
	draw(Label("$E=\omega_0^2$",BeginPoint,N),(-xm-0.25,V_sy*omega^2)--(xm+0.25,V_sy*omega^2));
	picture pic;
	sy = 1.25;
	real H(real x,real y){
		return 0.5*y^2-omega^2*cos(x);
	}
	real[] h = {0,omega^2+1.5};
	ym = 3.25;
	draw(pic,Label("$\psi$",EndPoint),(-x,0)--(x,0),Arrow);
	draw(pic,Label("$\dot{\psi}$",EndPoint,W),(0,-ym)--(0,ym),Arrow);
	guide[][] g = contour(H,(-xm,-ym),(xm,ym),new real[]{omega^2},200);
	draw(pic,g,linewidth(0.8bp));
	guide[][] g = contour(H,(-xm,-ym),(xm,ym),h,100);
	draw(pic,g,linewidth(0.8bp));
	real h = -9;
	add(shift(0,h)*yscale(sy)*pic);
	for(int i=-3;i<=3;i=i+2){
	draw((-i*pi,V(-i*pi))--(-i*pi,h),dashed);
	}
	for(int i=-2;i<=2;i=i+4){
	draw((-i*pi,0)--(-i*pi,h),dashed);
	}
	real a,b;
	a = 1.5;
	b = 0.85;
	for(int i=-3;i<=3;i=i+1){
		real pos = i*pi;
		dot((pos,h));
		if (i==0) continue;
		unfill(shift(pos,0)*box((-a/2,0.01),(a/2,b)));
		if (i==-1){
			label("$-\pi$",(pos,b/2));
			continue;
		}
		if (i==1){
			label("$\pi$",(pos,b/2));
			continue;
		}
		label("$"+string(i)+"\pi$",(pos,b/2));
	}
\end{asy}
\caption{摆的势函数和相轨道}
\label{chapter3:figure-摆的势函数和相轨道}
\end{figure}

势函数$V(\psi)$的曲线和相轨道如图\ref{chapter3:figure-摆的势函数和相轨道}所示。
\begin{enumerate}
\item 当$E<-\omega_0^2$时,运动是不可能的;

\item 当$E=-\omega_0^2$时,摆处于平衡位置刚体的质心处于可能位置中的最低点。在相平面上,这些平衡位置在相平面上相应于点$\psi=2k\pi(k=0,\pm 1,\pm 2,\cdots), \dot{\psi}=0$。这些点是中心型的,它们被表示摆振动的封闭相轨道包围,摆振动相应于满足$-\omega_0^2<E<\omega_0^2$的$E$值;

\item 当$E=\omega_0^2$时,有两种可能的运动:一种是对应于摆的平衡位置,刚体质心位于可能位置的最高点,这个平衡位置在相平面上相应于点$\psi=(2k+1)\pi(k=0,\pm 1,\pm 2,\cdots), \dot{\psi}=0$,这些点是鞍型的;另一种是刚体质心再$t\to\infty$时渐进地趋于最高位置,这种渐进运动在相平面上对应于连接鞍点的曲线,这些曲线是分离线;

\item 当$E>\omega_0^2$时,刚体的运动是转动,对于这种运动,角度$\phi$的绝对值单调增加,这种运动对应于相平面上下两条非封闭曲线,分离线将振动和非振动区域分开。
\end{enumerate}

\subsubsection{摆运动方程的积分}\label{chapter3:subsubsection-摆运动方程的积分}

根据式\eqref{chapter3:摆的能量积分}中$E$的不同取值进行分类讨论。
\begin{enumerate}
\item $-\omega_0^2<E<\omega_0^2$,这种情形对应摆的振动。设此时$E=-\omega_0^2\cos\beta$,其中$\beta$是任意无量纲常数,显然$\beta$的物理意义即为摆偏离竖直方向$\psi=0$的最大角度。此时能量积分式\eqref{chapter3:摆的能量积分}可以写作
\begin{equation}
	\dot{\psi}^2 = 2\omega_0^2(\cos\psi-\cos\beta)
	\label{chapter3:摆的能量积分-摆动情形}
\end{equation}
记$k_1=\sin\dfrac{\beta}{2}$并做变量代换
\begin{equation*}
	\sin\dfrac{\psi}{2}=k_1\sin u
\end{equation*}
则能量积分\eqref{chapter3:摆的能量积分-摆动情形}具有如下形式
\begin{equation}
	\dot{u}^2 = \omega_0^2(1-k_1^2\sin^2u)
\end{equation}
如果$t=0$时取$\psi=0$,则有
\begin{equation}
	\omega_0t = \int_0^u \frac{\mathd \xi}{\sqrt{1-k_1^2\sin^2\xi}} = F(\sin u,k_1)
\end{equation}
所以有$\sin u = \sn \omega_0t$,即
\begin{equation}
	\psi = 2\arcsin(k_1\sn \omega_0t)
\end{equation}
由于椭圆正弦函数$\sn u$的周期为$4K(k_1)$,因此摆运动的周期为
\begin{equation}
	T = \frac{4K(k_1)}{\omega_0}
\end{equation}
其级数形式为
\begin{equation}
	T = \frac{2\pi}{\omega_0}\left(1+\frac14k_1^2+\frac{9}{64}k^4+\cdots\right)
\end{equation}
因此可知摆角不大时,周期的近似值为$T=\dfrac{2\pi}{\omega_0}=2\pi\sqrt{\dfrac lg}$,与微振动结果相同。

\item $E>\omega_0^2$,这种情形对应摆的转动。记$t=0$时$\psi=0,\dot{\psi}=\dot{\psi}_0$,于是$E=\dfrac12\dot{\psi}_0^2-\omega_0^2$,此时能量积分式\eqref{chapter3:摆的能量积分}可以写作
\begin{equation}
	\dot{\psi} = \dot{\psi}_0\left(1-k_2^2\sin^2\dfrac{\psi}{2}\right)
\end{equation}
其中$k_2^2=\dfrac{4\omega_0^2}{\dot{\psi}_0^2}$。因为$E>\omega_0^2$,所以$\dot{\psi}_0^2>4\omega_0^2$,进而有$k_2^2<1$,由此可得
\begin{equation}
	\frac12 \dot{\psi}_0t=F\left(\sin\dfrac{\psi}{2},k_2\right) = \int_0^{\frac{\psi}{2}}\frac{\mathd\phi}{\sqrt{1-k_2^2\sin^2\phi}}
\end{equation}
由此可得
\begin{equation}
	\psi = 2\arcsin\left(\sn\dfrac{\dot{\psi}_0}{2}t\right) = 2\am\dfrac{\dot{\psi}_0}{2}t
\end{equation}
当初始角速度很大时,即$\dot{\psi}_0^2\gg\omega_0^2$,此时$k_2\to 0$,此时$\sn x\to \sin x$,因此$\am x=\arcsin (\sn x)\to x$,则近似地有$\psi=\dot{\psi}_0t$,摆动接近等速。

\item $E=\omega_0^2$,这种情形对应摆的渐进运动。此时能量积分式\eqref{chapter3:摆的能量积分}可以写作
\begin{equation}
	\dot{\psi}^2=4\omega_0^2\cos^2\frac{\psi}{2}
\end{equation}
如果$t=0$时,$\psi=0,\dot{\psi}>0$,则有\footnote{最后一个等号可以通过导数相等和$0$点函数值相等来验证。}
\begin{equation}
	\psi = 2\arcsin(\tanh \omega_0t) = -\pi+4\arctan(\mathe^{\omega_0t})
\end{equation}
\end{enumerate}

\section{Routh方程}

\subsection{Routh函数}

在对理想完整系统的Lagrange函数做Legendre变换的过程中,也可以只将Lagrange变量的一部分转化为Hamilton变量,这种Lagrange和Hamilton的组合变量称为{\bf Routh变量},记作
\begin{equation*}
	q_i,\dot{q}_i;\quad q_\alpha,p_\alpha; \quad t\quad (i=1,2,\cdots,l;\,\,\alpha=l+1,\cdots,s)
\end{equation*}
其中$l$是小于$s$的任意固定整数。

由于Lagrange函数对全体广义速度$\dot{q}_\alpha(\alpha=1,2,\cdots,s)$的Hesse行列式非零,因此Lagrange函数对$\dot{q}_\alpha(\alpha=l+1,\cdots,s)$的Hesse行列式也非零,即
\begin{equation*}
	\det\begin{pmatrix} \dfrac{\pl^2L}{\pl \dot{q}_\alpha\pl \dot{q}_\beta}\end{pmatrix} \neq 0 \quad (l+1\leqslant \alpha,\beta\leqslant s)
\end{equation*}
广义动量仍按照原方式定义:
\begin{equation}
	p_\alpha=\frac{\pl L}{\pl \dot{q}_\alpha} \quad (\alpha=l+1,\cdots,s)
	\label{chapter3:Routh方程中广义动量的定义}
\end{equation}
由此,{\bf Routh函数}$R=R(q_1,\cdots,q_l,q_{l+1},\cdots,q_s,\dot{q}_1,\cdots,\dot{q}_l,p_{l+1},\cdots,p_s,t)$为Lagrange函数$L$对变量$\dot{q}_{l+1},\cdots,\dot{q}_s$的Legendre变换,即
\begin{equation}
	R = \sum_{\alpha=l+1}^sp_\alpha\dot{q}_\alpha-L
	\label{chapter3:Routh函数的定义}
\end{equation}
其中右端的$\dot{q}_\alpha(\alpha=l+1,\cdots,s)$\footnote{包括Lagrange函数$L$中的$\dot{q}_\alpha$。}需要从方程\eqref{chapter3:Routh方程中广义动量的定义}中解出,用Routh坐标$q_i,q_\alpha,\dot{q}_i,p_\alpha,t$表示。

\subsection{Routh方程}

Routh函数$R=R(q_1,\cdots,q_l,q_{l+1},\cdots,q_s,\dot{q}_1,\cdots,\dot{q}_l,p_{l+1},\cdots,p_s,t)$的全微分为
\begin{equation}
	\mathd R = \sum_{i=1}^l \left(\frac{\pl R}{\pl q_i}\mathd q_i + \frac{\pl R}{\pl \dot{q}_i}\mathd \dot{q}_i\right) + \sum_{\alpha=l+1}^s \left(\frac{\pl R}{\pl q_\alpha} \mathd q_\alpha + \frac{\pl R}{\pl p_\alpha} \mathd p_\alpha\right) + \frac{\pl R}{\pl t}
	\label{chapter3:Routh函数的全微分1}
\end{equation}
另一方面,式\eqref{chapter3:Routh函数的定义}右端的全微分为
\begin{align}
	\mathd R & = -\sum_{i=1}^l \left(\frac{\pl L}{\pl q_i}\mathd q_i + \frac{\pl L}{\pl \dot{q}_i} \mathd \dot{q}_i\right) + \sum_{\alpha=l+1}^s \left[-\frac{\pl L}{\pl q_\alpha}\mathd q_\alpha + \dot{q}_\alpha\mathd p_\alpha + \left(p_\alpha-\frac{\pl L}{\pl \dot{q}_\alpha}\right) \mathd \dot{q}_\alpha\right] - \frac{\pl L}{\pl t}\mathd t \nonumber \\
	& = \sum_{i=1}^l \left(-\frac{\pl L}{\pl q_i}\mathd q_i - \frac{\pl L}{\pl \dot{q}_i} \mathd \dot{q}_i\right) + \sum_{\alpha=l+1}^s \left(-\frac{\pl L}{\pl q_\alpha}\mathd q_\alpha + \dot{q}_\alpha\mathd p_\alpha\right) - \frac{\pl L}{\pl t}\mathd t 
	\label{chapter3:Routh函数的全微分2}
\end{align}
比较式\eqref{chapter3:Routh函数的全微分1}和式\eqref{chapter3:Routh函数的全微分2}即可得到
\begin{align}
	\frac{\pl R}{\pl q_i} = -\frac{\pl L}{\pl q_i},\quad \frac{\pl R}{\pl \dot{q}_i} = -\frac{\pl L}{\pl \dot{q}_i}\quad (i=1,2,\cdots,l) \label{chapter3:Routh函数的全微分导出式1} \\
	\frac{\pl R}{\pl q_\alpha} = -\frac{\pl L}{\pl q_\alpha},\quad \frac{\pl R}{\pl p_\alpha} = \dot{q}_\alpha\quad (\alpha=l+1,\cdots,s) \label{chapter3:Routh函数的全微分导出式2}
\end{align}
以及
\begin{equation}
	\frac{\pl R}{\pl t} = -\frac{\pl L}{\pl t}
	\label{chapter3:Routh函数的全微分导出式3}
\end{equation}
将理想完整系统的Lagrange方程分为两组:
\begin{align}
	& \ddt \frac{\pl L}{\pl q_i} - \frac{\pl L}{\pl q_i} = 0\quad (i=1,2,\cdots,l) \label{chapter3:Routh方程节Lagrange方程分组1} \\
	& \ddt \frac{\pl L}{\pl q_\alpha} - \frac{\pl L}{\pl q_\alpha} = 0\quad (\alpha=l+1,\cdots,l) \label{chapter3:Routh方程节Lagrange方程分组2}
\end{align}
由式\eqref{chapter3:Routh函数的全微分导出式1}和式\eqref{chapter3:Routh方程节Lagrange方程分组1}可得
\begin{equation}
	\ddt \frac{\pl R}{\pl q_i} - \frac{\pl R}{\pl q_i} = 0\quad (i=1,2,\cdots,l) 
	\label{chapter3:Routh方程第一部分}
\end{equation}
而由式\eqref{chapter3:Routh方程中广义动量的定义}、\eqref{chapter3:Routh函数的全微分导出式2}和方程\eqref{chapter3:Routh方程节Lagrange方程分组2}可得
\begin{equation}
\begin{cases}
	\dot{q}_\alpha = \dfrac{\pl R}{\pl p_\alpha} \\[1.5ex]
	\dot{p}_\alpha = -\dfrac{\pl R}{\pl q_\alpha}
\end{cases}
(\alpha=l+1,\cdots,s)
\label{chapter3:Routh方程第二部分}
\end{equation}
方程\eqref{chapter3:Routh方程第一部分}和\eqref{chapter3:Routh方程第二部分}构成了{\bf Routh方程},它由$l$个具有Lagrange方程结构的二阶方程\eqref{chapter3:Routh方程第一部分}和$2(s-l)$个具有Hamilton方程结构的一阶方程构成。

由式\eqref{chapter3:Routh函数的全微分导出式3}可以看出,如果系统的Lagrange函数不显含时间,则其Routh函数也不显含时间。将系统的Hamilton函数表示为
\begin{equation}
	H = \sum_{i=1}^l p_i\dot{q}_i + \sum_{\alpha=l+1}^s p_\alpha\dot{q}_\alpha-L = \sum_{i=1}^l p_i\dot{q}_i + R
\end{equation}
注意到$p_i = \dfrac{\pl L}{\pl \dot{q}_i} = -\dfrac{\pl R}{\pl \dot{q}_i}$,因此此时系统的广义能量积分可以表达为
\begin{equation}
	R - \sum_{i=1}^l \frac{\pl R}{\pl \dot{q}_i}\dot{q}_i = E
	\label{chapter3:用Routh函数表示的广义能量积分}
\end{equation}

\subsection{利用Routh方程对带有循环坐标的系统降阶}

设$q_\alpha(\alpha=l+1,\cdots,s)$是理想完整系统的循环坐标,那么这个系统有$s-l$个首次积分
\begin{equation}
	p_\alpha=\frac{\pl L}{\pl \dot{q}_\alpha} = C_\alpha \quad \text{(常数)}\quad (\alpha=l+1,\cdots,s)
	\label{chapter3:利用Routh方程降阶-广义动量积分}
\end{equation}
构造Routh函数
\begin{equation}
	R = \sum_{\alpha=l+1}^s C_\alpha \dot{q}_\alpha - L
\end{equation}
其中$\dot{q}_\alpha$需要利用式\eqref{chapter3:利用Routh方程降阶-广义动量积分}用$q_i,\dot{q}_i,C_\alpha,t$表示。在这种情况下,Routh函数不包含循环坐标相应的广义速度$\dot{q}_\alpha$,即
\begin{equation}
	R = R(q_i,\dot{q}_i,C_\alpha,t)\quad (i=1,2,\cdots,l;\,\,\alpha=l+1,\cdots,s)
\end{equation}
所以Routh方程的第一部分
\begin{equation}
	\ddt\frac{\pl R}{\pl \dot{q}_i} - \frac{\pl R}{\pl q_i} = 0\quad (i=1,2,\cdots,l)
\end{equation}
描述非循环坐标随时间的变化,可以独立于Routh方程的第二部分。Routh方程的第二部分
\begin{equation}
	\dif{q_\alpha}{t} = \frac{\pl R}{\pl C_\alpha},\quad \dif{p_\alpha}{t} = 0\quad (\alpha=l+1,\cdots,s)
\end{equation}
相应于循环坐标,由此便达到了将原系统降阶的目的。

在实际应用的过程中,先根据Lagrange方程或Hamilton方程列出系统的运动方程,再利用相应的循环坐标所导出的首次积分式消元得到的结果,与利用Routh方程理论所得结果是相同的。但利用Routh方程理论可以省去很多不必要的复杂运算。

\begin{example}[球面摆的运动]
设球面摆上质点的质量为$m$,摆长为$l$,广义坐标取为以悬点为原点的球坐标系的极角$\theta$和方位角$\phi$,则其Lagrange函数为
\begin{equation}
	L = T-V = \frac12 ml^2(\dot{\theta}^2+\sin^2\theta\dot{\phi}^2) - mgl\cos\theta
	\label{chapter3:例-球面摆的Lagrange函数}
\end{equation}
由于球面摆的Lagrange函数中不显含$\phi$,故$\phi$为循环坐标,其相应的首次积分为
\begin{equation}
	p_\phi = \frac{\pl L}{\pl \dot{\phi}} = ml^2 \sin^2\theta\dot{\phi} = ml^2\omega_0\alpha
\end{equation}
式中$\omega_0 = \sqrt{\dfrac gl}$,$\alpha$为任意无量纲积分常数。由此可得
\begin{equation}
	\dot{\phi} = \frac{\omega_0}{\sin^2\theta}\alpha
	\label{chapter3:例-球面摆的方位角导数}
\end{equation}
所以此系统的Routh函数表示为
\begin{equation}
	R = p_\phi\dot{\phi} - L = -\frac12 ml^2 \dot{\theta}^2 + \frac{ml^2\omega_0^2\alpha^2}{2\sin^2 \theta}+mgl\cos \theta
	\label{chapter3:例-球面摆的Routh函数}
\end{equation}
系统的Routh函数不显含时间,因此系统存在广义能量积分,根据式\eqref{chapter3:用Routh函数表示的广义能量积分}可得其广义能量积分为
\begin{equation}
	\frac12 ml^2 \dot{\theta}^2 + \frac{ml^2\omega_0^2\alpha^2}{2\sin^2 \theta}+mgl\cos \theta = \frac12 ml^2\omega_0^2\beta
	\label{chapter3:例-球面摆的广义能量积分}
\end{equation}
其中$\beta$为任意无量纲积分常数。

做变量代换$u=\cos\theta$,则由式\eqref{chapter3:例-球面摆的广义能量积分}可得
\begin{equation}
	\frac{1}{\omega_0^2}\dot{u}^2 = (1-u^2)(\beta-2u)-\alpha^2 =: G(u)
	\label{chapter3:球面摆的G函数}
\end{equation}
记$G(u)=(1-u^2)(\beta-2u)-\alpha^2=0$的三个根为$u_1,u_2,u_3$,则$G(u)$也可以写作
\begin{equation}
	G(u) = 2(u-u_1)(u-u_2)(u-u_3)
	\label{chapter3:球面摆的G函数-以根来表示}
\end{equation}
其图像如图\ref{chapter3:figure-函数G的图像}所示。

\begin{figure}[htb]
\centering
\begin{asy}
	size(250);
	//
	real alpha,beta;
	real f(real u){
		return (1-u*u)*(beta-2*u)-alpha*alpha;
	}
	real x1,x2,y1,y2;
	x1 = -2;
	x2 = 2;
	y1 = -2;
	y2 = 2;
	draw(Label("$u$",EndPoint),(x1,0)--(x2,0),Arrow);
	draw(Label("$G(u)$",EndPoint,W),(0,y1)--(0,y2),Arrow);
	alpha = 0.8;
	beta = 0.5;
	picture pic;
	path p = graph(f,x1,x2);
	draw(pic,p,linewidth(0.8bp));
	clip(pic,box((x1,y1),(x2,y2)));
	add(pic);
	real l = 0.05;
	draw(Label("$-1$",EndPoint),(-1,0)--(-1,l));
	draw(Label("$1$",EndPoint),(1,0)--(1,l));
	pair P[];
	P = intersectionpoints(p,(x1,0)--(x2,0));
	label("$u_1$",P[0],dir(-60));
	label("$u_2$",P[1],SSW);
	label("$u_3$",P[2],SE);
	real um = (beta-sqrt(beta^2+12))/6;
	draw(Label("$u_*$",BeginPoint),(um,0)--(um,f(um)),dashed);
\end{asy}
\caption{函数$G(u)$的图像}
\label{chapter3:figure-函数G的图像}
\end{figure}

显然有$G(+\infty)=+\infty, G(-\infty)=-\infty$,而且$G(\pm 1) = -\alpha^2 \leqslant 0$。因为$G(u)$是连续函数,故至少其至少有一个根不小于$1$,不妨记作$u_3$。由于$u=\cos\theta$,所以球面摆的解要求在$-1\leqslant u\leqslant 1$的范围内存在$G(u)\geqslant 0$的点。由此可知,函数$G(u)$在$-1\leqslant u\leqslant 1$有两个实根$u_1,u_2$和实根$u_3\geqslant 1$。

因为在实际的运动中有$G(u)\geqslant 0$,所以只需考虑$u$在区间$u_1\leqslant u \leqslant u_2$内的解,这对应摆在运动中$\theta$的变化范围为$\theta_2\leqslant \theta\leqslant \theta_1$。

下面来考虑系统在不同初始条件下的运动情况,这相应于考虑系统在不同常数$\alpha,\beta$下的运动情况。由于实际的解存在需要$u_1\geqslant -1$,因此必须有$G(-1)\leqslant 0$,即
\begin{equation*}
	2\beta+4-\alpha^2 \leqslant 0
\end{equation*}
由此可知,$\beta$的取值应该满足$\beta\geqslant -2$。如果$\beta=-2$,此时$\alpha=0$,方程\eqref{chapter3:例-球面摆的广义能量积分}为
\begin{equation*}
	\frac12 ml^2 \dot{\theta}^2 = -mgl(1+\cos\theta)
\end{equation*}
这个方程只有解$\theta=\pi$,即对应于摆的竖直平衡位置。

函数$G(u)$有极大值点
\begin{equation}
	u = u_* = \frac16(\beta-\sqrt{\beta^2+12})
\end{equation}
此时
\begin{equation}
	G(u_*) = \frac{1}{54}\left[(\beta^2+12)^{\frac32}+36\beta-\beta^3\right]-\alpha^2 =: f(\beta)-\alpha^2
\end{equation}
其中
\begin{equation*}
	f(\beta) = \frac{1}{54}\left[(\beta^2+12)^{\frac32}+36\beta-\beta^3\right]
\end{equation*}
实际运动必须满足$G(u_*)\geqslant 0$,即
\begin{equation}
	0 \leqslant \alpha^2 \leqslant f(\beta)
\end{equation}
参数$\alpha^2,\beta$的允许区域为图\ref{chapter3:figure-球面摆参数的允许区域}中曲线$\alpha^2=f(\beta)$与横轴所夹的阴影区域(包括边界)。

\begin{figure}[htb]
\centering
\begin{asy}
	size(250);
	//
	real x1,x2,y1,y2;
	x1 = -4;
	x2 = 6;
	y1 = -2;
	y2 = 6;
	real f(real x){
		return ((x^2+12)^(3/2)+36*x-x^3)/54;
	}
	path p;
	p = graph(f,-2,0.95*x2)--(0.95*x2,0)--cycle;
	fill(p,0.75white);
	draw(Label("$\beta$",EndPoint),(x1,0)--(x2,0),Arrow);
	draw(Label("$\alpha^2$",EndPoint,W),(0,y1)--(0,y2),Arrow);
	label("$O$",(0,0),SE);
	label("$-2$",(-2,0),S);
	draw(Label("$\alpha^2=f(\beta)$",Relative(0.7),Relative(W)),graph(f,-2,x2),linewidth(0.8bp));
\end{asy}
\caption{参数$\alpha^2,\beta$的允许区域}
\label{chapter3:figure-球面摆参数的允许区域}
\end{figure}

为了对摆的运动分类,考虑下面三种可能的情况:
\begin{enumerate}
\item $\alpha=0$。由式\eqref{chapter3:例-球面摆的方位角导数}可知,此时$\phi=\phi_0\,\text{(常数)}$,此时球面摆即为在平面$\phi=\phi_0$上的单摆,关于单摆的运动,第\ref{chapter3:subsubsection-摆运动方程的积分}节中已进行过详细讨论。

\item $0<\alpha^2<f(\beta)$,这种情况下角$\theta$在区间$\theta_2\leqslant \theta\leqslant \theta_2$内变化。在以悬挂点为中心半径为$l$的球面上,$\theta=\theta_1$和$\theta=\theta_2$画出两个圆,它们位于平面$z=z_1=l\cos\theta_1$和$z=z_2=l\cos\theta_2$内。质点在两个平面所夹的球面区域上运动,交替地与两个平面相切(如图\ref{chapter3:figure-球面摆的摆动}所示)。

\begin{figure}[htb]
\centering
\begin{minipage}[t]{0.48\textwidth}
\centering
\begin{asy}
	size(200);
	//
	pair i,j,k;
	real r,theta0,thetam,w,theta;
	r = 1;
	theta0 = pi/2+pi/8;
	thetam = pi/12;
	w = 9;
	i = (-sqrt(2)/4,-sqrt(14)/12);
	j = (sqrt(14)/4,-sqrt(2)/12);
	k = (0,2*sqrt(2)/3);
	pair spconvert(real r,real theta,real phi){
		return r*sin(theta)*cos(phi)*i+r*sin(theta)*sin(phi)*j+r*cos(theta)*k;
	}
	pair thetaphi(real phi){
		return spconvert(r,theta0-thetam*sin(w*r*sin(theta0)*phi),phi);
	}
	pair phicir(real phi){
		return spconvert(r,theta,phi);
	}
	draw(Label("$z$",EndPoint),r*k--1.4*r*k,Arrow);
	theta = theta0-thetam;
	draw(graph(phicir,-1.25,1.95),dashed);
	label("$z=z_2$",phicir(1.9),SE);
	theta = theta0+thetam;
	draw(graph(phicir,-1.15,1.75),dashed);
	label("$z=z_1$",phicir(1.75),E);
	draw(graph(thetaphi,-1.175,1.1,200),linewidth(0.8bp)+red);
	draw(graph(thetaphi,-1.175,1.11,200),red,Arrow);
	draw(scale(r)*unitcircle,linewidth(0.8bp));
\end{asy}
\caption{球面摆的摆动}
\label{chapter3:figure-球面摆的摆动}
\end{minipage}
\hspace{0.1cm}
\begin{minipage}[t]{0.48\textwidth}
\centering
\begin{asy}
	size(180);
	//球面摆摆动轨迹的投影
	real p,e,w;
	pair tra(real theta){
		real r = p/(1-e*cos(w*theta));
		return (r*cos(theta),r*sin(theta));
	}
	p = 1;
	e = 0.6;
	w = 1.6;
	real thetai,thetam,rmax,rmin;
	thetai = 0.35;
	thetam = 3.9*pi;
	path q = graph(tra,thetai,thetam,400);
	path qprime = graph(tra,thetai,thetam+0.01,200);
	draw(q,linewidth(0.8bp));
	add(arrow(qprime,invisible,FillDraw(black),Relative(0.2)));
	add(arrow(qprime,invisible,FillDraw(black),Relative(0.55)));
	add(arrow(qprime,invisible,FillDraw(black),Relative(0.8)));
	add(arrow(qprime,invisible,FillDraw(black),EndPoint));
	rmax = p/(1-e);
	rmin = p/(1+e);
	draw(circle((0,0),rmax));
	draw(circle((0,0),rmin));
	draw(Label(scale(0.75)*"$\rho_1$",MidPoint,Relative(W)),(0,0)--rmin*dir(110),Arrow);
	draw(Label("$\rho_2$",Relative(0.6),Relative(W)),(0,0)--rmax*dir(0),Arrow);
	draw((0,0)--(0,rmax+0.025),invisible);
\end{asy}
\caption{球面摆摆动轨迹在$Oxy$平面的投影}
\label{chapter3:figure-球面摆摆动轨迹的投影}
\end{minipage}
\end{figure}

这种情形下,点的中间位置总是位于过悬挂点$O$的水平面以下,即$z_1+z_2<0$或者$u_1+u_2<0$。在式\eqref{chapter3:球面摆的G函数}和式\eqref{chapter3:球面摆的G函数-以根来表示}中对比$u$一次项的系数可得
\begin{equation*}
	2(u_1u_2+u_2u_3+u_3u_1)=-2
\end{equation*}
即有$u_3=-\dfrac{1+u_1u_2}{u_1+u_2}$,由于$-1<u_1<u_2<1$,所以$u_1u_2>-1$,再考虑到$u_3>0$,即有$u_1+u_2<0$。

式\eqref{chapter3:例-球面摆的方位角导数}说明角$\phi$或者单调增加(当$\alpha>0$时),或者单调减少(当$\alpha<0$时)。图\ref{chapter3:figure-球面摆摆动轨迹的投影}给出了图\ref{chapter3:figure-球面摆的摆动}描述的运动轨迹在$Oxy$平面上的投影,这时平面$z=z_1$和$z=z_2$都在悬挂点下方(此时$\alpha^2>\beta$并取$\alpha>0$)。这个投影交替地与半径为$\rho_1=l\sin\theta_1$和$\rho_2=l\sin\theta_2$的两个圆相切。

作变量代换
\begin{equation*}
	u=u_1+(u_2-u_1)\sin^2v
\end{equation*}
方程\eqref{chapter3:球面摆的G函数}变为
\begin{equation}
	\dot{v}^2=\frac{\omega_0^2}{2}(u_3-u_1)(1-k^2\sin^2v)
	\label{chapter3:摆动情形的球面摆运动微分方程}
\end{equation}
其中
\begin{equation}
	k^2=\frac{u_2-u_1}{u_3-u_1}\quad (0\leqslant k^2\leqslant 1)
\end{equation}
取$u=u_1$的时刻为初始时刻,则积分方程\eqref{chapter3:摆动情形的球面摆运动微分方程}可得
\begin{equation}
	\omega_0\sqrt{\frac{u_3-u_1}{2}}t = \int_0^v\frac{\mathd w}{\sqrt{1-k^2\sin^2w}} = F(\sin v,k)
\end{equation}
由此可得
\begin{equation*}
	\sin v = \sn \left(\omega_0\sqrt{\frac{u_3-u_1}{2}}t\right)
\end{equation*}
即
\begin{equation}
	u = u_1+(u_2-u_1)\sn^2\left(\omega_0\sqrt{\frac{u_3-u_1}{2}}t\right)=u_1+(u_2-u_1)\sn^2\tau
\end{equation}
其中$\tau=\omega_0\sqrt{\dfrac{u_3-u_1}{2}}t$。由于椭圆函数$\sn \tau$的周期为$4K(k)$,故$\sn^2\tau$的周期为$2K(k)$。由此可知,当$\tau=2nK(k)$时,有$u=\cos\theta=u_1$,当$\tau=(2n+1)K(k)$时,有$u=\cos\theta=u_2$,即角$\theta$在$\theta_1$和$\theta_2$之间周期振动,其振动周期为
\begin{equation}
	T = \frac{2\sqrt{2}K(k)}{\omega_0\sqrt{u_3-u_1}}
\end{equation}
当$\theta$作为时间的函数求出之后,积分方程\eqref{chapter3:例-球面摆的方位角导数}可得$\phi(t)$。

需要注意的是,虽然球面摆$\theta$角的变化是周期的,但是由于在一个$\theta$的周期内,$\phi$角的改变量与$2\pi$的比值不一定是有理数,因此球面摆有可能永远也没有办法回到其初始位置。

\item $\alpha^2=f(\beta)$。在这种情况下多项式$G(u)$的根$u_1=u_2=u_*<0$,问题变为圆锥摆。角$\theta$在运动过程中是常数$\theta=\theta_*=\arccos u_*>\dfrac{\pi}{2}$,质点沿着平面$z=z_*=l\cos\theta_*<0$内半径为$l\sin\theta_*$的圆周运动。
\end{enumerate}
\end{example}

\section{非完整系统的运动方程}

\subsection{Lagrange乘子法求解线性非完整系统}

设非完整系统有$k$个完整约束和$k'$个线性不可积运动约束,由式\eqref{chapter2:完整约束的一般形式}和\eqref{chapter2:非完整约束的一般形式}给出:
\begin{subnumcases}{}
	f_j(\mbf{r}_1,\mbf{r}_2,\cdots,\mbf{r}_n,t)=0 \quad (j=1,2,\cdots,k) \label{chapter2:完整约束的一般形式-重写} \\
	\sum_{i=1}^n \mbf{A}_{ji} \cdot \mbf{v}_i + A_{j0} = 0 \quad (j=1,2,\cdots,k') \label{chapter2:非完整约束的一般形式-重写}
\end{subnumcases}
根据完整约束引入广义坐标$q_\alpha(\alpha = 1,2,\cdots,s=3n-k)$消去完整约束,利用广义坐标
可以将线性非完整约束\eqref{chapter2:非完整约束的一般形式-重写}表示为式\eqref{chapter2:用广义坐标表示的非完整约束}的形式:
\begin{align}
	\sum_{\alpha=1}^s B_{j\alpha}(\mbf{q},t) \dot{q}_\alpha + B_{j0}(\mbf{q},t) = 0 \quad (j=1,2,\cdots,k')
	\label{chapter2:用广义坐标表示的非完整约束-重写}
\end{align}
其中的系数$B_{j\alpha}$和$B_{j0}$根据式\eqref{chapter2:用广义坐标表示的非完整约束-系数}得到,即
\begin{equation}
	B_{j\alpha}(\mbf{q},t) = \sum_{i=1}^n \mbf{A}_{ji} \cdot \frac{\pl \mbf{r}_i}{\pl q_\alpha}, \quad B_{j0}(\mbf{q},t) = \sum_{i=1}^n \mbf{A}_{ji} \cdot \frac{\pl \mbf{r}_i}{\pl t} + A_{j0}
	\label{chapter2:用广义坐标表示的非完整约束-系数-重写}
\end{equation}
式\eqref{chapter2:用广义坐标表示的非完整约束-重写}可以用广义虚位移表示为式\eqref{chapter2:非完整约束的虚位移表示},即
\begin{align}
	\sum_{\alpha=1}^s B_{j\alpha}(\mbf{q},t) \delta q_\alpha = 0 \quad (j=1,2,\cdots,k')
	\label{chapter2:非完整约束的虚位移表示-重写}
\end{align}
根据广义坐标形式的动力学普遍方程\eqref{chapter2:广义坐标形式的动力学普遍方程}可有
\begin{equation}
	\sum_{\alpha=1}^s \left(\ddt\frac{\pl T}{\pl \dot{q}_\alpha} - \frac{\pl T}{\pl q_\alpha} - Q_\alpha\right) \delta q_\alpha = 0
	\label{chapter2:非完整系统的动力学普遍方程}
\end{equation}
但此处式\eqref{chapter2:非完整系统的动力学普遍方程}中的$s$个广义虚位移之间并不是相互独立的,它们之间由$k'$个运动约束方程\eqref{chapter2:非完整约束的虚位移表示-重写}相联系。引入$k'$个不定乘子$\lambda_j(t)$,将其分别与方程\eqref{chapter2:非完整约束的虚位移表示-重写}相乘,并与广义坐标形式的动力学普遍方程\eqref{chapter2:非完整系统的动力学普遍方程}相减可得
\begin{equation}
	\sum_{\alpha=1}^s \left(\ddt\frac{\pl T}{\pl \dot{q}_\alpha} - \frac{\pl T}{\pl q_\alpha} - Q_\alpha - \sum_{j=1}^{k'} \lambda_j B_{j\alpha}\right) \delta q_\alpha = 0
	\label{chapter2:非完整系统带有乘子的动力学普遍方程}
\end{equation}
在式\eqref{chapter2:非完整系统带有乘子的动力学普遍方程}中的$s$个广义虚位移中,只有$f=s-k'=3n-k-k'$个是相互独立的,而此处引入的$k'$个不定乘子是可以任意取值的,那么总是可以选择这样的不定乘子$\lambda_j(t)$,使得式\eqref{chapter2:非完整系统带有乘子的动力学普遍方程}中前$k'$个广义虚位移的系数为零,那么这样式\eqref{chapter2:非完整系统带有乘子的动力学普遍方程}就只剩下相互独立的$f=s-k'$个广义虚位移,它们的系数必须为零。综上可得非完整系统的运动微分方程为
\begin{equation}
\begin{cases}
	\ds \ddt \frac{\pl T}{\pl \dot{q}_\alpha} - \frac{\pl T}{\pl q_\alpha} = Q_\alpha + \sum_{j=1}^{k'} \lambda_j B_{j\alpha}& (\alpha = 1,2,\cdots,s) \\[1.5ex]
	\ds \sum_{\alpha=1}^s B_{j\alpha} \dot{q}_\alpha + B_{j0} = 0 & (j = 1,2,\cdots,k')
\end{cases}
\label{chapter2:非完整系统带有约束乘子的运动微分方程}
\end{equation}
如果系统有广义势$U=U(\mbf{q},\dot{\mbf{q}},t)$则有
\begin{equation}
	Q_\alpha = \ddt\frac{\pl U}{\pl \dot{q}_\alpha} - \frac{\pl U}{\pl q_\alpha}\quad (\alpha=1,2,\cdots,s)
\end{equation}
由此可将系统方程\eqref{chapter2:非完整系统带有约束乘子的运动微分方程}利用Lagrange函数改写为
\begin{equation}
\begin{cases}
	\displaystyle \frac{\mathrm{d}}{\mathrm{d} t} \frac{\pl L}{\pl \dot{q}_\alpha} - \frac{\pl L}{\pl q_\alpha} = \sum_{j=1}^{k'} \lambda_j B_{j\alpha} & (\alpha = 1,2,\cdots,s) \\[1.5ex]
	\displaystyle \sum_{\alpha=1}^s B_{j\alpha} \dot{q}_\alpha + B_{j0} = 0 & (j = 1,2,\cdots,k')
\end{cases}
\end{equation}

\begin{example}[直立圆盘在水平面上的纯滚动]\label{chapter3:example-非完整系例题1}
\begin{figure}[htb]
\centering
\begin{asy}
	size(200);
	//纯滚动直立圆盘
	real hd,hdd;
	pair O,x,y,z,i,j,k;
	hd = 131+25/60;
	hdd = 7+10/60;
	O = (0,0);
	x = 1.2*dir(90+hd);
	y = 1.5*dir(-hdd);
	z = 1.5*dir(90);
	draw(Label("$x$",EndPoint),O--x,Arrow);
	draw(Label("$y$",EndPoint),O--y,Arrow);
	draw(Label("$z$",EndPoint),O--z,Arrow);
	label("$O$",O,NW);
	i = (-sqrt(2)/4,-sqrt(14)/12);
	j = (sqrt(14)/4,-sqrt(2)/12);
	k = (0,2*sqrt(2)/3);
	
	real x0,y0,r,phi;
	pair P;
	pair planecur(real xx){
		real yy = y0+1*tan(xx-x0);
		return xx*i+yy*j;
	}
	pair dplanecur(real xx){
		real yy = 1/((cos(xx-x0))**2);
		return xx*i+yy*j;
	}
	pair wheel(real theta){
		real xx,yy,zz;
		xx = x0+r*sin(theta)*cos(phi);
		yy = y0+r*sin(theta)*sin(phi);
		zz = r+r*cos(theta);
		return xx*i+yy*j+zz*k;
	}
	real getphi(real xx){
		real yy = 1/((cos(xx-x0))**2);
		return atan(yy/xx);
	}
	x0 = 1;
	y0 = 0.7;
	r = 0.5;
	phi = getphi(x0);
	unfill(graph(wheel,0,2*pi)--cycle);
	draw(graph(wheel,0,2*pi),linewidth(0.8bp));
	draw(graph(planecur,x0-1,x0+1));
	P = planecur(x0);
	dot(P);
	draw((P-0.7*dplanecur(x0))--(P+dplanecur(x0)),dashed);
	draw(P--P+r*k--wheel(-1.2));
	label("$(x,y)$",P+r*k,E);
	label("$\theta$",P+r*k,WSW);
	label("$\phi$",P-0.7*dplanecur(x0),2*S);
	
	//draw(O--(1.7,0),invisible);
\end{asy}
\caption{例\theexample}
\label{第三章例8图}
\end{figure}
\end{example}
\begin{solution}
此非完整系统有完整约束
\begin{equation*}
	z-R = 0
\end{equation*}
以及非完整约束
\begin{equation*}
	\begin{cases}
		\dot{x} - R\dot{\theta} \cos \phi = 0 \\
		\dot{y} - R\dot{\theta} \sin \phi = 0
	\end{cases}
\end{equation*}
由此可将广义坐标取为$x,y,\theta,\phi$,系统的Lagrange函数为
\begin{equation*}
	L = T = \frac12 m(\dot{x}^2+\dot{y}^2) + \frac12 I_1 \dot{\theta}^2 + \frac12 I_2 \dot{\phi}^2
\end{equation*}
考虑到
\begin{align*}
	& B_{11} = 1,\quad B_{12} = 0,\quad B_{13} = -R\cos \phi,\quad \bar{A}_{14} = 0 \\
	& B_{21} = 0,\quad B_{22} = 1,\quad B_{23} = -R\sin \phi,\quad \bar{A}_{24} = 0
\end{align*}
由此可以写出系统方程为
\begin{subnumcases}{}
%\renewcommand{\theequation}{\theparentequation-\arabic{equation}}
	m\ddot{x} = \lambda_1 \label{方程第1式} \\
	m\ddot{y} = \lambda_2 \label{方程第2式} \\
	I_1 \ddot{\theta} = -R(\lambda_1 \cos \phi + \lambda_2 \sin \phi) \label{方程第3式} \\
	I_2 \ddot{\phi} = 0 \label{方程第4式} \\
	\dot{x} - R\dot{\theta} \cos \phi = 0 \label{方程第5式} \\
	\dot{y} - R\dot{\theta} \sin \phi = 0 \label{方程第6式} 
\end{subnumcases}
由式\eqref{方程第4式}可得
\begin{equation*}
	\phi = \omega_2 t + \phi_0
\end{equation*}
考虑$\text{式}\eqref{方程第1式} \times R\cos\phi + \text{式}\eqref{方程第2式} \times R\sin \phi + \text{式}\eqref{方程第3式}$可得
\begin{equation}
	mR(\ddot{x} \cos \phi+\ddot{y}\sin \phi) + I_1 \ddot{\theta} = 0
	\label{方程第7式}
\end{equation}
再考虑$\cos \phi \times \dfrac{\mathrm{d}}{\mathrm{d} t}\text{式}\eqref{方程第5式} + \sin \phi \times \dfrac{\mathrm{d}}{\mathrm{d} t}\text{式}\eqref{方程第6式}$可得
\begin{equation}
	\ddot{x} \cos \phi+\ddot{y}\sin \phi - R\ddot{\theta} = 0
	\label{方程第8式}
\end{equation}
由式\eqref{方程第7式}和式\eqref{方程第8式}可解得
\begin{equation*}
	\theta = \omega_1 t + \theta_0
\end{equation*}
于是根据式\eqref{方程第5式}和式\eqref{方程第6式}可得
\begin{align*}
	\begin{cases}
		\dot{x} = R\omega_1 \cos(\omega_2 t + \phi_0) \\
		\dot{y} = R\omega_1 \sin(\omega_2 t + \phi_0)
	\end{cases}
\end{align*}

下面分两种情况讨论圆盘的运动,
\begin{enumerate}
\item $\omega_2 = 0$,系统的解为
\begin{equation}
\begin{cases}
	x = R\omega_1 t \cos \phi_0 + x_0 \\
	y = R\omega_1 t \sin \phi_0 + y_0 \\
	\theta = \omega_1 t + \theta_0 \\
	\phi = \phi_0 \\
	\lambda_1 = 0 \\
	\lambda_2 = 0
\end{cases}
\end{equation}
可见,圆盘沿直线匀速滚动。运动过程自动保证约束条件,约束反力为零。
\item $\omega_2 \neq 0$,系统的解为
\begin{equation}
\begin{cases}
	\displaystyle x = \frac{R\omega_1}{\omega_2} \sin(\omega_2 t + \phi_0) - \frac{R\omega_1}{\omega_2} \sin \phi_0 + x_0 \\[1,5ex]
	\displaystyle y = -\frac{R\omega_1}{\omega_2} \cos(\omega_2 t + \phi_0) + \frac{R\omega_1}{\omega_2} \cos \phi_0 + y_0 \\
	\theta = \omega_1 t + \theta_0 \\
	\phi = \omega_2 t + \phi_0 \\
	\lambda_1 = -R\omega_1 \omega_2 \sin \phi \\
	\lambda_2 = R\omega_1 \omega_2 \cos \phi
\end{cases}
\end{equation}	
可见,圆盘沿圆周匀速滚动,纯滚动条件自动保证,约束反力垂直盘面,提供向心力。
\end{enumerate}
\end{solution}

% \subsection{沃罗涅茨方程}

% \subsection{卡普雷金方程}

\subsection{Appell方程}

Appell\footnote{Paul Appell,阿佩尔,法国数学家。}提出了不包含约束乘子的运动方程,既适用于完整系统,也适用于有线性不可积运动约束的非完整系统。

\subsubsection{伪坐标}

Appell方程是利用称为{\bf 伪坐标}的更一般形式的坐标来表示的。设$f$为系统的自由度,考虑广义速度的$f$个相互独立的线性组合:
\begin{equation}
	\dot{\pi}_\nu = \sum_{\alpha=1}^s C_{\nu\alpha} \dot{q}_\alpha \quad (\nu=1,2,\cdots,f)
	\label{chapter2:伪速度的定义}
\end{equation}
其中系数$C_{\nu\alpha}=C_{\nu\alpha}(\mbf{q},t)$。此处的$\dot{\pi}_\nu$是广义速度的某些线性组合,有完全确定的意义,但是符号$\pi_\nu$可能没有意义,因为式\eqref{chapter2:伪速度的定义}的右端可能不是关于广义坐标和时间的函数对时间的全导数。而符号$\ddot{\pi}_\nu$就用来表示式\eqref{chapter2:伪速度的定义}右端对时间的导数。我们称$\pi_\nu$为{\bf 伪坐标},而$\dot{\pi}_\nu$和$\ddot{\pi}_\nu$分别称为{\bf 伪速度}和{\bf 伪加速度}。

在定义伪速度时,需要选择系数$C_{\nu\alpha}$使得从$s=f+k'$个方程\eqref{chapter2:用广义坐标表示的非完整约束-重写}、\eqref{chapter2:伪速度的定义}中解$\dot{q}_\alpha(\alpha=1,2,\cdots,s)$时,系数行列式非零,求解之后可得
\begin{equation}
	\dot{q}_\alpha = \sum_{\nu=1}^f D_{\nu\alpha}\dot{\pi}_\nu+g_\alpha\quad (\alpha=1,2,\cdots,s)
	\label{chapter2:用伪速度表示广义速度}
\end{equation}
式中$D_{i\alpha}$和$g_\alpha$是$\mbf{q}$和$t$的函数。伪速度$\dot{\pi}_\nu$可以任意取值,当它们是给定的时,可以通过式\eqref{chapter2:用伪速度表示广义速度}求解出所有的广义速度。

根据式\eqref{chapter2:伪速度的定义},形式上引入
\begin{equation}
	\delta \pi_\nu = \sum_{\alpha=1}^s C_{\nu\alpha}\delta q_\alpha \quad (\nu=1,2,\cdots,f)
	\label{chapter2:伪虚位移的定义}
\end{equation}
量$\delta \pi_\nu$称为{\bf 伪虚位移},式\eqref{chapter2:伪虚位移的定义}是它的定义式。注意,由于式\eqref{chapter2:伪速度的定义}的右端不一定是全微分,所以伪虚位移并不是真正的虚位移。利用方程\eqref{chapter2:非完整约束的虚位移表示-重写}和式\eqref{chapter2:伪虚位移的定义}求得用$\delta\pi_\nu(\nu=1,2,\cdots,f)$表示的$\delta q_\alpha$:
\begin{equation}
	\delta q_\alpha = \sum_{\nu=1}^f D_{\nu\alpha}\delta \pi_\nu\quad (\alpha=1,2,\cdots,s)
	\label{chapter2:用伪虚位移表示广义虚位移}
\end{equation}
这里的$\delta\pi_\nu$可以任意取值。

\subsubsection{Appell方程}

为了得到Appell方程,首先需要将动力学普遍方程\eqref{chapter2:d'Alambert-Lagrange原理}用伪坐标表示出来。首先考虑动力学普遍方程中的第一项:
\begin{equation}
	\sum_{i=1}^n \mbf{F}_i\cdot \delta\mbf{r}_i = \sum_{\alpha=1}^s Q_\alpha\delta q_\alpha
\end{equation}
将式\eqref{chapter2:用伪虚位移表示广义虚位移}代入,可得
\begin{equation}
	\sum_{i=1}^n \mbf{F}_i\cdot \delta\mbf{r}_i = \sum_{\alpha=1}^s Q_\alpha \sum_{\nu=1}^f D_{\nu\alpha}\delta \pi_\nu = \sum_{\nu=1}^f \left(\sum_{\alpha=1}^s D_{\nu\alpha}Q_\alpha\right) \delta \pi_\nu = \sum_{\nu=1}^f \varPi_\nu \delta\pi_\nu
	\label{chapter2:伪坐标下的动力学普遍方程-第一部分}
\end{equation}
式中
\begin{equation}
	\varPi_\nu = \sum_{\alpha=1}^s D_{\nu\alpha}Q_\alpha\quad (\nu = 1,2,\cdots,f)
	\label{chapter3:对应于伪坐标的广义力}
\end{equation}
称为对应于伪坐标$\pi_\nu$的广义力。

现在考虑动力学普遍方程中的第二项,由于
\begin{equation}
	\delta\mbf{r}_i = \sum_{\alpha=1}^s \frac{\pl \mbf{r}_i}{\pl q_\alpha}\delta q_\alpha
	\label{chapter2:虚位移与广义虚位移之间的关系-重写}
\end{equation}
将式\eqref{chapter2:用伪虚位移表示广义虚位移}代入式\eqref{chapter2:虚位移与广义虚位移之间的关系-重写}中,可得
\begin{equation}
	\delta \mbf{r}_i = \sum_{\alpha=1}^s \frac{\pl \mbf{r}_i}{\pl q_\alpha} \sum_{\nu=1}^f D_{\nu\alpha}\delta\pi_\nu = \sum_{\nu=1}^f \left(\sum_{\alpha=1}^s \frac{\pl \mbf{r}_i}{\pl q_\alpha} D_{\nu\alpha}\right)\delta\pi_\nu = \sum_{\nu=1}^f \mbf{E}_{i\nu}\delta \pi_\nu
	\label{chapter2:用伪虚位移表示系统各质点的虚位移}
\end{equation}
式中
\begin{equation*}
	\mbf{E}_{i\nu} = \sum_{\alpha=1}^s\frac{\pl \mbf{r}_i}{\pl q_\alpha}D_{\nu\alpha}\quad (i=1,2,\cdots,n; \nu=1,2,\cdots,f)
\end{equation*}
此处考虑将式\eqref{chapter2:用伪虚位移表示系统各质点的虚位移}用加速度$\mbf{a}_i$来表示。将等式\eqref{chapter2:用伪速度表示广义速度}两端对时间求导,然后将得到的$\ddot{q}_\alpha$代入加速度与广义加速度、广义速度的关系式\eqref{chapter2:各质点加速度与广义速度和广义加速度的关系}中,可得
\begin{equation}
	\mbf{a}_i = \sum_{\nu=1}^f \left(\sum_{\alpha=1}^s \frac{\pl \mbf{r}_i}{\pl q_\alpha}D_{\nu\alpha}\right) \ddot{\pi}_\nu + \mbf{H}_i = \sum_{\nu=1}^f \mbf{E}_{i\nu}\ddot{\pi}_\nu+\mbf{H}_i\quad (i=1,2,\cdots,n)
\end{equation}
此处$\mbf{H}_i$不依赖于$\ddot{\pi}_\nu$,由此可得
\begin{equation}
	\mbf{E}_{i\nu} = \frac{\pl \mbf{a}_i}{\pl \ddot{\pi}_\nu}\quad (i=1,2,\cdots,n; \nu=1,2,\cdots,f)
	\label{chapter2:用伪虚位移表示系统各质点的虚位移-中间步骤}
\end{equation}
将式\eqref{chapter2:用伪虚位移表示系统各质点的虚位移-中间步骤}代入式\eqref{chapter2:用伪虚位移表示系统各质点的虚位移}中,可得虚位移$\delta \mbf{r}_i$可以用加速度和伪虚位移表示为:
\begin{equation}
	\delta \mbf{r}_i = \sum_{\nu=1}^f \frac{\pl \mbf{a}_i}{\pl \ddot{\pi}_\nu}\delta \pi_\nu
	\label{chapter2:用伪虚位移表示系统各质点的虚位移-另一种形式}
\end{equation}
由此,动力学普遍方程的第二项可以写为
\begin{align}
	\sum_{i=1}^n m_i\mbf{a}_i\cdot\delta\mbf{r}_i & = \sum_{i=1}^n m_i\mbf{a}_i \cdot \sum_{\nu=1}^f \frac{\pl \mbf{a}_i}{\pl \ddot{\pi}_\nu}\delta \pi_\nu = \sum_{\nu=1}^f \left(\sum_{i=1}^n m_i\mbf{a}_i\cdot \frac{\pl \mbf{a}_i}{\pl \ddot{\pi}_\nu}\right) \delta \pi_\nu
	\label{chapter2:Appell方程-中间过程1}
\end{align}
引入函数$S$为
\begin{equation}
	S = \frac12 \sum_{i=1}^n m_i\mbf{a}_i^2 
	\label{chapter2:加速度能的定义}
\end{equation}
称为{\bf 加速度能},在一般情况下,它是$q_1,\cdots,q_s, \dot{\pi}_1,\cdots,\dot{\pi}_f, \ddot{\pi}_1,\cdots,\ddot{\pi}_f$的函数。利用加速度能,可以将式\eqref{chapter2:Appell方程-中间过程1}表示为
\begin{equation}
	\sum_{i=1}^n m_i\mbf{a}_i\cdot\delta\mbf{r}_i = \sum_{\nu=1}^f \frac{\pl S}{\pl \ddot{\pi}_\nu} \delta\pi_\nu
	\label{chapter2:伪坐标下的动力学普遍方程-第二部分}
\end{equation}

根据式\eqref{chapter2:伪坐标下的动力学普遍方程-第一部分}和式\eqref{chapter2:伪坐标下的动力学普遍方程-第二部分},动力学普遍方程可以用伪坐标表示为
\begin{equation}
	\sum_{\nu=1}^f \left(\frac{\pl S}{\pl \ddot{\pi}_\nu} - \varPi_\nu\right) \delta\pi_\nu = 0
\end{equation}
因为伪虚位移$\delta\pi_\nu(\nu=1,2,\cdots,f)$可以任意取值,所以有
\begin{equation}
	\frac{\pl S}{\pl \ddot{\pi}_\nu} = \varPi_\nu \quad (\nu = 1,2,\cdots,f)
	\label{chapter2:Appell方程}
\end{equation}
方程\eqref{chapter2:Appell方程}称为Appell方程,它们需要与非完整约束方程\eqref{chapter2:用广义坐标表示的非完整约束-重写}和伪速度定义式\eqref{chapter2:伪速度的定义}联立求解。即非完整系的运动微分方程为
\begin{equation}
\begin{cases}
	\ds \frac{\pl S}{\pl \ddot{\pi}_\nu} = \varPi_\nu & (\nu = 1,2,\cdots,f) \\[1.5ex]
	\ds \sum_{\alpha=1}^s B_{j\alpha}(\mbf{q},t) \dot{q}_\alpha + B_{j0}(\mbf{q},t) = 0 & (j=1,2,\cdots,k') \\[1.5ex]
	\ds \dot{\pi}_\nu = \sum_{\alpha=1}^s C_{\nu\alpha} \dot{q}_\alpha & (\nu=1,2,\cdots,f)
\end{cases}
\end{equation}
共有$k'+2f=s+f$个方程,共同决定$s+f$个位置函数即$s$个广义坐标$q_1,q_2,\cdots, q_s$和$f$个伪速度$\dot{\pi}_1,\dot{\pi}_2,\cdots,\dot{\pi}_f$。

这里需要指出的一点是,Appell方程也可以应用于完整系。当其应用于完整系时,将伪速度取为广义速度,得到的方程只是Lagrange方程的另一种形式。

为了得到Appell方程,需要根据式\eqref{chapter2:加速度能的定义}计算加速度能,并根据式\eqref{chapter3:对应于伪坐标的广义力}确定与伪坐标对应的广义力,这是十分繁琐的过程。

\subsubsection{加速度能的计算·K\"onig定理的类比}

设$\mbf{a}_C$是系统质心的绝对加速度,$\mbf{a}_i$是质点$m_i$的绝对加速度,而$\mbf{a}_i'$是该质点相对质心的加速度,那么对于每一个质点都有
\begin{equation}
	\mbf{a}_i = \mbf{a}_i'+\mbf{a}_C
\end{equation}
将其代入加速度能的定义式\eqref{chapter2:加速度能的定义}中,可得
\begin{equation}
	S = \frac12 \sum_{i=1}^n m_i\mbf{a}_i^2 = \frac12 \sum_{i=1}^n m_i\mbf{a}_C^2 + \sum_{i=1}^n m_i\mbf{a}_i' \cdot \mbf{a}_C + \frac12 \sum_{i=1}^n m_i\mbf{a}_i'^2
\end{equation}
由于$\ds \sum_{i=1}^n m_i = M$以及$\sum_{i=1}^n m_i\mbf{a}_i' = \mbf{0}$,可有
\begin{equation}
	S = \frac12 M\mbf{a}_C^2 + \frac12 \sum_{i=1}^n m_i\mbf{a}_i'^2 = S_C+S'
	\label{chapter3:加速度能的柯尼希定理}
\end{equation}
即系统的加速度能等于位于质心的质量等于系统总质量的质点的加速度能与系统相对质心的加速度能之和。这个结论与动能的K\"onig定理(见式\eqref{Konig定理})具有相同的形式。

\subsubsection{定点运动刚体的加速度能}

刚体的纯滚动是最常见的非完整约束。纯滚动下,刚体与滚动平面接触点的绝对速度等于零,这可以表示为
\begin{equation}
	\mbf{v}_C-R(\mbf{\omega}\times \mbf{n}) = \mbf{0}
	\label{chapter3:刚体的纯滚动约束}
\end{equation}
式中$\mbf{v}_C$表示刚体质心的速度,$\mbf{n}$为滚动面在接触点处的法向单位矢量,$R$为刚体质心到接触点的距离。在很多情况下,角速度矢量并不能表示为某个函数对时间的全导数,因此纯滚动约束方程\eqref{chapter3:刚体的纯滚动约束}通常是不可积运动约束\footnote{需要指出的是,虽然约束\eqref{chapter3:刚体的纯滚动约束}是非完整的,但它仍然是一个理想约束。}。

将刚体的本系统记作$Oxyz$,坐标原点为刚体的固定点$O$,坐标轴分别为刚体对$O$点的三个惯性主轴。设$\mbf{\omega}$为刚体的角速度,在本系统中,其分量分别为$\omega_1,\omega_2,\omega_3$。根据式\eqref{刚体上任意一点的加速度}可得,刚体上任意一点的加速度为
\begin{equation}
	\mbf{a} = \dot{\mbf{\omega}} \times \mbf{r} + \mbf{\omega} \times \left(\mbf{\omega} \times \mbf{r}\right) = \dot{\mbf{\omega}} \times \mbf{r} + (\mbf{\omega}\cdot\mbf{r})\mbf{\omega} - \omega^2 \mbf{r}
\end{equation}
所以加速度在本系统中的三个分量分别为
\begin{equation}
\begin{cases}
	a_1 = -x(\omega_2^2+\omega_3^2)+y(\omega_2\omega_1-\dot{\omega}_3)+z(\omega_1\omega_3+\dot{\omega}_2) \\
	a_2 = -y(\omega_3^2+\omega_1^2)+z(\omega_3\omega_2-\dot{\omega}_1)+x(\omega_2\omega_1+\dot{\omega}_3) \\
	a_3 = -z(\omega_1^2+\omega_2^2)+x(\omega_1\omega_3-\dot{\omega}_2)+y(\omega_3\omega_2+\dot{\omega}_1) \\
\end{cases}
\label{chapter3:定点运动刚体各点加速度在本系统内的分量}
\end{equation}
加速度能在刚体上则可以表示为
\begin{equation*}
	S = \frac12 \int_V \rho (a_1^2+a_2^2+a_3^2)\mathd V
\end{equation*}
将式\eqref{chapter3:定点运动刚体各点加速度在本系统内的分量}代入并注意到,在主轴系中有
\begin{align*}
	& I_1 = \int_V \rho(y^2+z^2)\mathd V, \quad I_2 = \int_V \rho(z^2+x^2)\mathd V, \quad I_3 = \int_V \rho(x^2+y^2)\mathd V \\
	& \int_V \rho yz\mathd V = \int_V \rho zx\mathd V = \int_V \rho xy\mathd V = 0
\end{align*}
可得
\begin{equation}
	S = \frac12 (I_1\omega_1^2 + I_2\omega_2^2 + I_3\omega_3^2) + (I_3-I_2)\omega_2\omega_3\dot{\omega}_1 + (I_1-I_3)\omega_3\omega_1\dot{\omega}_2 + (I_2-I_1)\omega_1\omega_2\dot{\omega}_3
	\label{chapter3:定点运动刚体的加速度能}
\end{equation}

\begin{example}[利用Appell方程获得Euler动力学方程]
将伪速度取为角动量的三个分量\footnote{根据Euler运动学方程,这样定义出的伪速度确实是广义速度的线性组合,而且它们之间是线性无关的。},即$\dot{\pi}_1=\omega_1, \dot{\pi}_2=\omega_2, \dot{\pi}_3=\omega_3$。这里需要获得与伪坐标对应的广义力,记$\mbf{\pi} = \begin{pmatrix} \pi_1 \\ \pi_2 \\ \pi_3 \end{pmatrix}$,并考虑
\begin{equation*}
	\mathd \mbf{r} = \mbf{\omega}\times \mbf{r}\mathd t = \mathd \mbf{\pi} \times \mbf{r}
\end{equation*}
其中最后一个等号是形式上的,所以有
\begin{equation*}
	\int_V \mbf{f} \mathd V\cdot \mathd \mbf{r} = \int_V (\mbf{f}\cdot \mathd \mbf{r}) \mathd V = \int_V \mbf{f}\cdot (\mathd \mbf{\pi} \times \mbf{r}) \mathd V = \int_V \mathd \mbf{\pi} \cdot (\mbf{r}\times \mbf{f})\mathd V = \int_V \mbf{r}\times \mbf{f}\mathd V \cdot \mathd \mbf{\pi}
\end{equation*}
即有
\begin{equation*}
	\mbf{F} \cdot \mathd \mbf{r} = \mbf{M}_O \cdot \mathd \mbf{\pi}
\end{equation*}
根据等时变分运算与微分运算的相似性,可以直接得到
\begin{equation*}
	\mbf{F}\cdot \delta\mbf{r} = \mbf{M}_O\cdot \delta \mbf{\pi}
\end{equation*}
由此可得相应于伪坐标$\pi_i$的广义力即为
\begin{equation*}
	\varPi_1 = M_1,\quad \varPi_2 = M_2,\quad \varPi_3 = M_3
\end{equation*}
然后根据加速度能\eqref{chapter3:定点运动刚体的加速度能}和Appell方程\eqref{chapter2:Appell方程}可以直接得到Euler动力学方程。
\end{example}

\begin{example}[利用Appell方程重新求解例\ref{chapter3:example-非完整系例题1}]
系统的非完整约束可以表示为
\begin{equation*}
\begin{cases}
	\dot{x} = R\dot{\theta}\cos\phi \\
	\dot{y} = R\dot{\theta}\sin\phi
\end{cases}
\end{equation*}
圆盘的加速度能可以利用式\eqref{chapter3:加速度能的柯尼希定理}来计算,即
\begin{equation*}
	S = \frac12 m(\ddot{x}^2+\ddot{y}^2) + \frac12 (I_1\ddot{\theta}^2 + I_2\ddot{\phi}^2)
\end{equation*}
其中$I_1=\dfrac12 mR^2$为圆盘绕过其中心并与盘面垂直轴自转时的转动惯量,$I_2=\dfrac14mR^2$为圆盘绕过其中心的竖直轴旋转时的转动惯量。

将伪速度取为$\dot{\pi}_1 = \dot{\theta}, \dot{\pi}_2 = \dot{\phi}$,则有
\begin{equation*}
\begin{cases}
	\ddot{x} = R(\ddot{\theta}\cos\phi - \dot{\theta}\dot{\phi}\sin\phi) \\
	\ddot{y} = R(\ddot{\theta}\sin\phi + \dot{\theta}\dot{\phi}\cos\phi)
\end{cases}
\end{equation*}
由此可得加速度能为
\begin{equation*}
	S = \frac12 mR^2(\ddot{\theta}^2 + \dot{\theta}^2\dot{\phi}^2) + \frac12 \left(\frac12 mR^2\ddot{\theta}^2 + \dfrac14mR^2\ddot{\phi}^2\right) = \frac34mR^2\ddot{\theta}^2 + \frac18mR^2\ddot{\phi}^2 + \frac12 mR^2 \dot{\theta}^2\dot{\phi}^2
\end{equation*}
再考虑到圆盘所受的主动力只有重力,所有主动力都不做功,因此有$\varPi_1=\varPi_2=0$。所以,根据Appell方程\eqref{chapter2:Appell方程}可得
\begin{equation*}
\begin{cases}
	\ddot{\theta} = 0 \\
	\ddot{\phi} = 0
\end{cases}
\end{equation*}
这与例\ref{chapter3:example-非完整系例题1}所得到的结果相同。
\end{example}

\section{Poisson括号}

\subsection{Poisson括号的定义}

对两个系统状态函数$f = f(\mbf{q},\mbf{p},t)$和$g = g(\mbf{q},\mbf{p},t)$,定义
\begin{equation}
	[f,g] = \sum_{\alpha=1}^s \left(\frac{\pl f}{\pl q_\alpha} \frac{\pl g}{\pl p_\alpha} - \frac{\pl g}{\pl q_\alpha} \frac{\pl f}{\pl p_\alpha}\right) = \sum_{\alpha=1}^s \begin{vmatrix} \dfrac{\pl f}{\pl q_\alpha} & \dfrac{\pl f}{\pl p_\alpha} \\[1.5ex] \dfrac{\pl g}{\pl q_\alpha} & \dfrac{\pl g}{\pl p_\alpha} \end{vmatrix} = \sum_{\alpha=1}^s \frac{\pl (f,g)}{\pl (q_\alpha,p_\alpha)}
\end{equation}
称为{\heiti Poisson括号}。
\begin{equation}
	[q_\alpha,q_\beta] = 0,\quad [p_\alpha,p_\beta] = 0,\quad [q_\alpha,p_\beta] = \delta_{\alpha\beta}
	\label{基本Poisson括号}
\end{equation}
称为{\heiti 基本Poisson括号},这里$\delta_{\alpha\beta}$是Kronecker符号,定义为
\begin{equation*}
	\delta_{\alpha\beta} = \begin{cases} 1, & \alpha=\beta \\ 0, & \alpha\neq\beta \end{cases}
\end{equation*}
根据Poisson括号的定义,容易得到
\begin{equation}
	[q_\alpha, f] =\frac{\pl f}{\pl p_\alpha},\quad [p_\alpha,f] = -\frac{\pl f}{\pl q_\alpha}
\end{equation}

\subsection{Poisson括号的性质}

利用Poisson括号的定义和行列式的性质很容易得到Poisson括号的以下性质:
\begin{enumerate}
	\item 反对称性:
	\begin{equation}
		[f,g] = -[g,f]
		\label{chapter9:Poisson括号的反对称性}
	\end{equation}
	特别地,可有$[f,f] = 0$;
	\item 线性性:
	\begin{equation}
		[f,C_1 g+C_2 h] = C_1[f,g] + C_2[f,h],\quad \forall \, C_1,C_2 \in \mathbb{R}
		\label{chapter9:Poisson括号的线性性}
	\end{equation}
	\item 结合性:
	\begin{equation}
		[f,gh] = [f,g]h+g[f,h]
		\label{chapter9:Poisson括号的结合性}
	\end{equation}
	\item Leibniz性:
	\begin{equation}
		\frac{\pl}{\pl x} [f,g] = \left[\frac{\pl f}{\pl x},g\right] + \left[f,\frac{\pl g}{\pl x}\right],\quad x \in \{\mbf{q},\mbf{p},t\}
		\label{chapter9:Poisson括号的Leibniz性}
	\end{equation}
	\item Jacobi恒等式:
	\begin{equation}
		\big[f,[g,h]\big] + \big[g,[h,f]\big] + \big[h,[f,g]\big] = 0
		\label{chapter9:Poisson括号的Jacobi恒等式}
	\end{equation}
\end{enumerate}

\iffalse
\begin{example}
计算Poisson括号$[L_y,L_z], [L_z,L_x], [L_x,L_y]$和$[L_x,L^2], [L_y,L^2], [L_z,L^2]$,这里$L_x, L_y, L_z$是质点角动量矢量$\mbf{L}$的三个分量,$L^2$是质点角动量大小的平方。
\end{example}
\begin{solution}
首先,根据角动量的定义$\mbf{L}=\mbf{r}\times \mbf{p}$可得
\begin{equation*}
\begin{cases}
	L_x = yp_z-zp_y \\
	L_y = zp_x-xp_z \\
	L_z = xp_y-yp_x
\end{cases}
\end{equation*}
利用Poisson括号的性质\eqref{chapter9:Poisson括号的线性性}可得
\begin{align*}
	[L_y,L_z] & = [zp_x-xp_z,xp_y-yp_x] = [zp_x,xp_y]-[xp_z,xp_y]-[zp_x,yp_x]+[xp_z,yp_x]
\end{align*}
再利用性质\eqref{chapter9:Poisson括号的结合性}可将上式中的第一项改写为
\begin{equation*}
	[zp_x,xp_y] = xz[p_x,p_y] + x[z,p_y]p_x + z[p_x,x]p_y + [z,x]p_xp_y
\end{equation*}
根据基本Poisson括号\eqref{基本Poisson括号}可得,上式的四项中第三项等于$-1$,其余三项都等于零,所以有
\begin{equation*}
	[zp_x,xp_y] = -zp_y
\end{equation*}
同理可得
\begin{equation*}
	[xp_z,xp_y] = 0,\quad [zp_x,yp_x] = 0,\quad [xp_z,yp_x] = yp_z
\end{equation*}
所以
\begin{equation*}
	[L_y,L_z] = yp_z-zp_y = L_x
\end{equation*}
同理
\begin{equation*}
	[L_z,L_x] = L_y,\quad [L_x,L_y] = L_z
\end{equation*}

下面计算$[L_x,L^2]$:
\begin{align*}
	[L_x,L^2] & = [L_x,L_x^2+L_y^2+L_z^2] = [L_x,L_x^2] + [L_x,L_y^2] + [L_x,L_z^2] \\
	& = L_x[L_x,L_x] + [L_x,L_x]L_x + L_y[L_x,L_y] + [L_x,L_y]L_y + L_z[L_x,L_z] + [L_x,L_z]L_z \\
	& = 0+0+L_yL_z+L_zL_y-L_zL_y-L_yL_z = 0
\end{align*}
同理可得
\begin{equation*}
	[L_y,L^2] = 0,\quad [L_z,L^2] = 0
\end{equation*}
\end{solution}\fi

\subsection{正则方程的Poisson括号形式}

设有函数$f = f(\mbf{q},\mbf{p},t)$,考虑
\begin{equation}
	\frac{\mathrm{d} f}{\mathrm{d} t} = \frac{\pl f}{\pl t} + \sum_{\alpha=1}^s \left(\frac{\pl f}{\pl q_\alpha} \dot{q}_\alpha + \frac{\pl f}{\pl p_\alpha} \dot{p}_\alpha\right) = \frac{\pl f}{\pl t} + \sum_{\alpha=1}^s \left(\frac{\pl f}{\pl q_\alpha} \frac{\pl H}{\pl p_\alpha} - \frac{\pl f}{\pl p_\alpha} \frac{\pl H}{\pl q_\alpha}\right)
\end{equation}
此处利用了Hamilton正则方程,由此得到
\begin{equation}
	\frac{\mathrm{d} f}{\mathrm{d} t} = \frac{\pl f}{\pl t} + [f,H]
\end{equation}
如果函数$f$不显含时间,即$\dfrac{\pl f}{\pl t} = 0$,则有
\begin{equation}
	\frac{\mathrm{d} f}{\mathrm{d} t} = [f,H]
	\eqref{chapter3:不显含时间的力学量对时间导数的Poisson括号表示}
\end{equation}
分别将函数$f$取为$q_\alpha$和$p_\alpha$可得{\heiti 正则方程的Poisson括号形式}
\begin{equation}
	\begin{cases}
		\dif{q_\alpha}{t} = [q_\alpha,H] \\[1.5ex]
		\dif{p_\alpha}{t} = [p_\alpha,H]
	\end{cases}
\end{equation}
将函数$f$取为$H$则得到
\begin{equation}
	\dot{H} = \frac{\pl H}{\pl t}
\end{equation}

如果将$[f,H]$记作$\hat{H} f$,此处$\hat{H}$称为{\heiti 算子},它代表一种操作——即求该函数与Hamilton函数的Poisson括号。如果$f$不显含时间,可有
\begin{equation*}
	\frac{\mathrm{d} f}{\mathrm{d} t} = [f,H] = \hat{H}f
\end{equation*}
依次可有
\begin{align*}
	\frac{\mathrm{d}^2 f}{\mathrm{d} t^2} & = \frac{\mathrm{d}}{\mathrm{d} t}[f,H] = [\hat{H}f,H] = \hat{H} (\hat{H} f) = \hat{H}^2 f \\
	& \vdots \\
	\frac{\mathrm{d}^n f}{\mathrm{d} t^n} & = \frac{\mathrm{d}}{\mathrm{d} t}\left(\frac{\mathrm{d}^{n-1}f}{\mathrm{d} t^{n-1}}\right) = [\hat{H}^{n-1}f,H] = \hat{H} (\hat{H}^{n-1} f) = \hat{H}^n f
\end{align*}
如果不考虑收敛的问题,任何正则变量的函数可以用Taylor级数表示为
\begin{align*}
	f(\mbf{q}(t),\mbf{p}(t)) = \sum_{n=0}^\infty \frac{1}{n!} \frac{\mathrm{d}^n f}{\mathrm{d} t^n}(0) t^n = \sum_{n=0}^\infty \frac{t^n}{n!} \hat{H}^n f(0) = \mathrm{e}^{t\hat{H}} f(0)
\end{align*}
由此,可以得到运动的算子形式解。

\begin{example}
利用算子形式求解一维谐振子,其Hamilton函数为
\begin{equation*}
	H = \frac{p^2}{2m} + \frac12 m\omega^2 q^2
\end{equation*}
\end{example}
\begin{solution}
根据基本Poisson括号\eqref{基本Poisson括号}可得
\begin{equation*}
	[q,q] = 0,\quad [p,p] = 0,\quad [q,p] = 1
\end{equation*}
可得
\begin{align*}
	[q,H] & = \frac{1}{2m}[q,p^2] + \frac12 m\omega^2 [q,q^2] = \frac{1}{2m} \bigg([q,p]p+p[q,p]\bigg) = \frac{p}{m} \\
	[p,H] & = \frac{1}{2m}[p,p^2] + \frac12 m\omega^2 [p,q^2] = \frac12 m\omega^2 \bigg([p,q]q+q[p,q]\bigg) = -m\omega^2 q
\end{align*}
由此可以依次计算
\begin{align*}
	\hat{H}q & = [q,H] = \frac{p}{m} \\
	\hat{H}^2q & = \frac{1}{m}[p,H] = -\omega^2 q \\
	\hat{H}^3q & = -\omega^2[q,H] = -\omega^2\frac{p}{m} \\
	\hat{H}^4q & = -\frac{\omega^2}{m}[p,H] = (-\omega^2)^2 q \\
	& \vdots \\
	\hat{H}^{2m-1} q & = (-\omega^2)^{m-1} \frac{p}{m} \\
	\hat{H}^{2m} q & = (-\omega^2)^m q \\
	& \vdots 
\end{align*}
故有
\begin{align*}
	q(t) & = \left[\sum_{m=0}^\infty \frac{t^{2m}}{(2m)!} \hat{H}^{2m} + \sum_{m=1} \frac{t^{2m-1}}{(2m-1)!} \hat{H}^{2m-1}\right] q(0) \\
	& = \sum_{m=0}^\infty \frac{(-1)^m(\omega t)^{2m}}{(2m)!} q(0) + \sum_{m=1}^\infty \frac{(-1)^{m-1}(\omega t)^{2m-1}}{(2m-1)!} \frac{p(0)}{m\omega} \\
	& = q(0) \cos \omega t + \frac{p(0)}{m\omega} \sin \omega t
\end{align*}
\end{solution}

\subsection{Poisson定理·系统的可积性}

系统力学状态随时间演变时不变的力学量称为{\heiti 运动积分},即运动积分需要满足
\begin{equation}
	\frac{\mathrm{d} f}{\mathrm{d} t} = \frac{\pl f}{\pl t} + [f,H] = 0
\end{equation}

\begin{theorem}[Poisson定理]
两个运动积分的Poisson括号仍为运动积分,即如果有
\begin{equation}
	\dfrac{\mathrm{d} f}{\mathrm{d} t} = 0,\quad \dfrac{\mathrm{d} g}{\mathrm{d} t} = 0
\end{equation}
则有
\begin{equation}
	\frac{\mathrm{d}}{\mathrm{d} t} [f,g] = 0
\end{equation}
即$[f,g]$也是运动积分。
\end{theorem}
\begin{proof}
由于
\begin{equation*}
	\frac{\mathrm{d} f}{\mathrm{d} t} = \frac{\pl f}{\pl t} + [f,H] = 0
\end{equation*}
所以有$\dfrac{\pl f}{\pl t} = -[f,H]$,同理可有$\dfrac{\pl g}{\pl t} = -[g,H]$。由此可有
\begin{align*}
	\frac{\mathrm{d}}{\mathrm{d} t} [f,g] & = \frac{\pl}{\pl t}[f,g] + \big[[f,g],H\big] = \left[\frac{\pl f}{\pl t},g\right] + \left[f,\frac{\pl g}{\pl t}\right] + \big[[f,g],H\big] \\
	& = \big[-[f,H],g\big] + \big[f,-[g,H]\big] + \big[[f,g],H\big] \\
	& = -\big[f,[g,H]\big] - \big[g,[H,f]\big] - \big[H,[f,g]\big] = 0 \qedhere
\end{align*}
\end{proof}

\begin{example}[角动量的Poisson括号]\label{chapter9:角动量的Poisson括号}
角动量可以表示为
\begin{equation*}
	\mbf{L} = \mbf{r} \times \mbf{p} = \begin{vmatrix} \mbf{e}_1 & \mbf{e}_2 & \mbf{e}_3 \\ x & y & z \\ p_x & p_y & p_z \end{vmatrix}
\end{equation*}
所以
\begin{align*}
	[L_x,L_y] & = [yp_z-zp_y,zp_x-xp_z] = [yp_z,zp_x] + [zp_y,xp_z] \\
	& = y[p_z,z]p_x + x[z,p_z] p_y = xp_y-yp_x = L_z
\end{align*}
同理可有
\begin{equation*}
	[L_y,L_z] = L_x,\quad [L_z,L_x] = L_y
\end{equation*}
考虑
\begin{align*}
	[L_z,L^2] & = [L_z,L_x^2] + [L_z,L_y^2] + [L_z,L_z^2] \\
	& = [L_z,L_x]L_x + L_x[L_z,L_x] + [L_z,L_y]L_y + L_y[L_z,L_y] \\
	& = L_yL_x + L_xL_y - L_xL_y - L_yL_x = 0
\end{align*}
同理可有
\begin{equation*}
	[L_x,L^2] = [L_y,L^2] = [L_z,L^2] = 0
\end{equation*}
由此可得,若角动量任意两分量守恒,则第三分量亦守恒。角动量三分量变化不独立,这与两次异轴转动可合成绕第三轴转动的事实相照应。
\end{example}

对于有$s$个自由度的力学系统,如果可以找到$s$个相互独立的运动积分$\phi_1,\phi_2,\cdots,\phi_s$,且满足
\begin{equation}
	[\phi_\alpha,\phi_\beta] = 0\quad (\alpha,\beta = 1,2,\cdots,s)
\end{equation}
这样的系统称为{\bf 可积系统},此时称$\phi_1,\phi_2,\cdots,\phi_s$是相互{\bf 对合}的。这个结论是Liouville首先证明的,故也称为{\bf Liouville定理},相应地上述可积性称为{\bf Liouville可积性}。

Poisson定理似乎表明,总是可以根据两个已知的运动积分找到另一个运动积分。但是,很多情况下这并不能得到有意义的结果,得到的Poisson括号经常是常数或者是已知运动积分的函数。例如例\ref{chapter9:角动量的Poisson括号}说明,如果$L_z$和$L^2$都是运动积分,则根据Poisson定理,$[L_z,L^2]=0$也是运动积分,但是用$0$作为运动积分并没有意义。

如果希望从两个已知的运动积分得到更多运动积分,甚至得到可积系统的全部运动积分,需要最初的两个已知运动积分中至少有一个能够刻画出给定问题的局部特征,能够完全反映该问题的物理本质。如果最初的两个已知的运动积分都是由所有系统都成立的动力学定理获得的,则一般不能有效地应用Poisson定理。

\subsection{量子Poisson括号}

量子Poisson括号定义为
\begin{equation}
	[f,g]_{-} = fg - gf
\end{equation}
在量子力学中,将正则方程中的Possion括号$[f,g]$替换为$\dfrac{1}{\mathrm{i}\hbar} [f,g]_{-}$的过程,称为{\heiti 正则量子化}。力学量的乘积一般非对易,可用算符或矩阵表示。

基本Poisson括号表示为
\begin{equation}
	[q_i,p_j]_{-} = \mathrm{i} \hbar \delta_{ij},\quad [q_i,q_j]_{-} = 0,\quad [p_i,p_j]_{-} = 0,\quad i,j=1,2,3
\end{equation}
可有{\heiti Heisenberg方程}
\begin{equation}
	\frac{\mathrm{d} f}{\mathrm{d} t} = \frac{\pl f}{\pl t} + \frac{1}{\mathrm{i} \hbar} [f,H]_{-}
\end{equation}
取宏观极限
\begin{equation*}
	\lim_{\hbar \to 0} \frac{1}{\mathrm{i} \hbar} [f,g]_{-} = [f,g]
\end{equation*}
时,动力学过渡为经典力学。


%第四章——两体问题
\chapter{两体问题}

无外力的两质点体系在相互作用下的运动求解问题称为{\heiti 两体问题}。

两体中心力是自然界最普遍、最典型的力场之一。中心力问题研究肇始于行星运动的理论解释,对原子核式模型的确立起了关键性的作用。

\section{两体问题约化与中心力场}

\subsection{Lagrange函数及其分离变量}

在两体系统不受外力作用时,设两个质点之间的相互作用势表示为$V = V(\mbf{r}_1 - \mbf{r}_2)$,则系统的Lagrange函数为
\begin{equation}
	L = \frac12 m_1 \dot{\mbf{r}}_1^2 + \frac12 m_2 \dot{\mbf{r}}_2^2 - V(\mbf{r}_1 - \mbf{r}_2)
\end{equation}
可见此Lagrange函数无循环坐标,两粒子的运动相互耦合。
\begin{figure}[htb]
\centering
\begin{asy}
	texpreamble("\usepackage{xeCJK}");
	texpreamble("\setCJKmainfont{SimSun}");
	usepackage("amsmath");
	import graph;
	import math;
	size(200);
	//两体问题
	pair O,m1,m2,C;
	O = (0,0);
	m1 = (1,0.8);
	m2 = (-0.2,0.6);
	C = interp(m2,m1,0.35);
	label("$O$",O,S);
	draw(Label("$\boldsymbol{r}_1$",MidPoint,Relative(E),black),O--m1,blue,Arrow);
	label("$m_1$",m1,SE);
	draw(Label("$\boldsymbol{r}_2$",MidPoint,Relative(W),black),O--m2,blue,Arrow);
	label("$m_2$",m2,W);
	draw(Label("$\boldsymbol{R}$",MidPoint,Relative(W),black),O--C,red,Arrow);
	label("$C$",C,N);
	draw(Label("$\boldsymbol{r}$",MidPoint,Relative(W),black),m2--m1,red,Arrow);
	dot(C);
	//draw(O--m1+(0.2,0),invisible);
\end{asy}
\caption{两体问题}
\label{两体问题}
\end{figure}

考虑取质心坐标
\begin{equation}
	\mbf{R} = \frac{m_1\mbf{r}_1+m_2\mbf{r}_2}{m_1+m_2}
\end{equation}
和两个质点之间的相对位矢
\begin{equation}
	\mbf{r} = \mbf{r}_1 - \mbf{r}_2
\end{equation}
为广义坐标。记$M = m_1+m_2$为体系的{\heiti 总质量},$\mu = \dfrac{m_1m_2}{m_1+m_2}$为两体{\heiti 约化质量},则有
\begin{equation}
	\begin{cases}
		\displaystyle \mbf{r}_1 = \mbf{R} + \frac{m_2}{M}\mbf{r} = \mbf{R} + \frac{\mu}{m_1} \mbf{r} \\[1.5ex]
		\displaystyle \mbf{r}_2 = \mbf{R} - \frac{m_1}{M}\mbf{r} = \mbf{R} - \frac{\mu}{m_2} \mbf{r}
	\end{cases}
	\label{两体问题广义坐标变换}
\end{equation}
由此,系统的Lagrange方程可以表示为
\begin{align}
	L & = \frac12 m_1 \left(\dot{\mbf{R}}+\frac{\mu}{m_1}\dot{\mbf{r}}\right)^2 + \frac12 m_2 \left(\dot{\mbf{R}}-\frac{\mu}{m_2}\dot{\mbf{r}}\right)^2 - V(\mbf{r}) \nonumber \\
	& = \frac12 M\dot{\mbf{R}}^2 + \frac12 \mu \dot{\mbf{r}}^2 - V(\mbf{r})
\end{align}
由此可以将两体问题等效为两个单体问题——解耦:
\begin{enumerate}
	\item 整体运动:质量为$M$的质点的运动。Lagrange函数不显含$\mbf{R}$,故有$\dot{P} = \mbf{0}$,即$M$做惯性运动。
	\item 相对运动:质量为$\mu$的质点的运动。
	\begin{equation*}
		\mu \ddot{\mbf{r}} = -\frac{\pl V}{\pl \mbf{r}}
	\end{equation*}
	即$\mu$在势场$V$中的运动。
\end{enumerate}
系统的动能$T=T_C+T'$,其中$\displaystyle T_C = \frac12 M\dot{\mbf{R}}^2$为整体动能,$\displaystyle T' = \frac12 \mu\dot{\mbf{r}}^2$为两体内动能。系统的角动量$\mbf{l} = \mbf{l}_C+\mbf{l}'$,其中$\mbf{l}_C = \mbf{R} \times M\dot{\mbf{R}}$为轨道角动量,
\begin{equation*}
	\mbf{l}' = \mbf{r}'_1 \times m_1 \dot{\mbf{r}}'_1 + \mbf{r}'_2 \times m_2 \dot{\mbf{r}}'_2 = \mbf{r} \times \mu \dot{\mbf{r}}
\end{equation*}
为内秉角动量。

由于两体系统的整体运动即为惯性运动,因此后面的讨论集中于相对运动的求解。

\subsection{两体中心力场}

如果两体相互作用势能只与两个粒子的相对距离$r$有关,而和它们的相对方向无关,即$V(\mbf{r}) = V(r)$,这种势场称为{\heiti 中心势场}。在中心势场中,可有两个质点之间的相互作用力为\footnote{此处由于\begin{equation*} r = \sqrt{x^2+y^2+z^2} \end{equation*}可有\begin{equation*} \frac{\pl r}{\pl \mbf{r}} = \bnb r = \frac{\pl r}{\pl x}{\mbf{e}_1}+\frac{\pl r}{\pl y}{\mbf{e}_2}+\frac{\pl r}{\pl z}{\mbf{e}_3} = \frac{x}{r}\mbf{e}_1 + \frac{y}{r}\mbf{e}_2 + \frac{z}{r}\mbf{e}_3 = \frac{\mbf{r}}{r} \end{equation*}}
\begin{equation}
	\mbf{f}_{12} = -\frac{\pl V}{\pl \mbf{r}} = -V'(r) \frac{\pl r}{\pl \mbf{r}} = -V'(r) \frac{\mbf{r}}{r} = -\mbf{f}_{21}
\end{equation}
由此,内力做功为
\begin{equation}
	\mbf{f}_{12} \cdot \mathrm{d} \mbf{r}_1 + \mbf{f}_{21} \cdot \mathrm{d} \mbf{r}_2 = -V'(r) \mathrm{d} r = -\mathrm{d} V(r)
\end{equation}
故$\mu$的运动能量守恒。再考虑内力矩
\begin{equation}
	\dot{\mbf{l}}{}' = \mbf{M}' = \mbf{r} \times \left(-V'(r) \frac{\mbf{r}}{r}\right) = \mbf{0}
\end{equation}
故$\mu$的运动内秉角动量守恒,而且有$\mbf{r} \perp \mbf{l}'$,即$\mu$在垂直于内秉角动量的固定平面内运动。因此相对运动可用平面极坐标系来描述,即取$r,\theta$为广义坐标,则此时的Lagrange函数为
\begin{equation}
	L' = \frac12 \mu \left(\dot{r}^2 + r^2 \dot{\theta}^2\right) - V(r)
\end{equation}

\section{等效势、运动解与轨道方程}

\subsection{运动方程}

根据相对运动的Lagrange函数
\begin{equation*}
	L' = \frac12 \mu \left(\dot{r}^2 + r^2 \dot{\theta}^2\right) - V(r)
\end{equation*}
可得运动方程为
\begin{equation}
	\begin{cases}
		\displaystyle \mu \left(\ddot{r}-r\dot{\theta}^2\right) = -V'(r) = f(r) \\
		\displaystyle \mu \left(r\ddot{\theta} + 2\dot{r} \dot{\theta}\right) = 0
	\end{cases}
	\label{两体运动方程}
\end{equation}

\subsection{守恒定律}

$\theta$为循环坐标,故有$p_\theta$守恒,即
\begin{equation}
	p_\theta = \mu r^2 \dot{\theta} = l\,\text{(常数)}
	\label{内秉角动量守恒}
\end{equation}
此即为两体内秉角动量守恒。

Lagrange函数$L'$不显含时间,故
\begin{equation}
	H = \frac12 \mu \left(\dot{r}^2+r^2\dot{\theta}^2\right) + V(r) = E\,\text{(常数)}
	\label{两体内能守恒}
\end{equation}
即两体内能守恒。

\subsection{运动解}

由式\eqref{内秉角动量守恒}可得
\begin{equation*}
	\dot{\theta} = \frac{l}{\mu r^2}
\end{equation*}
代入式\eqref{两体内能守恒}可得
\begin{equation}
	\frac12 \mu \dot{r}^2 + \frac{l^2}{2\mu r^2} + V(r) = E
	\label{相对径向运动}
\end{equation}
令
\begin{equation}
	U(r) = \frac{l^2}{2\mu r^2} + V(r)
\end{equation}
称为{\heiti 等效势},则式\eqref{相对径向运动}化为
\begin{equation}
	\frac12 \mu \dot{r}^2 + U(r) = E
	\label{两体运动等效为一维运动}
\end{equation}
由此,径向运动等效为质点$\mu$在等效势场$U(r)$中的一维运动。等效势中的惯性离心势来自可遗角坐目标共轭动量效应。

在
\begin{equation}
	\frac{\mathrm{d} r}{\mathrm{d} t} = \pm \sqrt{\frac{2}{\mu}\big[E-U(r)\big]}
	\label{r对t的方程}
\end{equation}
两边积分可得
\begin{equation}
	t-t_0 = \pm \sqrt{\frac{\mu}{2}} \int_{r_0}^r \frac{\mathrm{d} r}{\sqrt{E-U(r)}}
	\label{t与r的关系}
\end{equation}
即有$r = r(t)$。再在
\begin{equation}
	\frac{\mathrm{d} \theta}{\mathrm{d} t} = \frac{l}{\mu r^2}
	\label{theta对t的方程}
\end{equation}
两边积分可得
\begin{equation}
	\theta - \theta_0 = \frac{l}{\mu} \int_{t_0}^t \frac{\mathrm{d} t}{\big[r(t)\big]^2}
	\label{chp4:theta与t的关系}
\end{equation}
即有$\theta = \theta(t)$。由此即得到了全部相对运动的解。

\subsection{轨道方程}

将式\eqref{r对t的方程}与式\eqref{theta对t的方程}相除,可得
\begin{equation}
	\frac{\mathrm{d} r}{\mathrm{d} \theta} = \pm \frac{\mu r^2}{l} \sqrt{\frac{2}{\mu}\big[E-U(r)\big]}
	\label{两体轨道方程(变量为r)}
\end{equation}
做变量代换$u=\dfrac{1}{r}$,可得两体问题轨道微分方程
\begin{equation}
	\frac{\mathrm{d} u}{\mathrm{d} \theta} = \mp \frac{1}{l} \sqrt{2\mu\left[E - U\left(\frac{1}{u}\right)\right]}
	\label{两体轨道方程}
\end{equation}
对于势场$V(r) = \dfrac{\alpha}{r} + \dfrac{\beta}{r^2}$和$V(r) = \alpha r^2$,式\eqref{两体轨道方程}有解析解。如果$r$的变化区间有两个边界$r_{\mathrm{min}}$和$r_{\mathrm{max}}$,则轨道位于两个圆$r = r_{\mathrm{min}}$和$r = r_{\mathrm{max}}$所限制的环形区域内,但有限运动的轨道并不一定是闭合的。在$r$从$r_{\mathrm{max}}$变化到$r_{\mathrm{min}}$再变回$r_{\mathrm{max}}$的这段时间内,矢径转过了一个角度$\Delta \theta$,根据式\eqref{两体轨道方程}有
\begin{equation}
	\Delta \theta = \mp 2\int_{\frac{1}{r_{\mathrm{max}}}}^{\frac{1}{r_{\mathrm{min}}}} \frac{l\mathrm{d} u}{\sqrt{2\mu\left[E-U\left(\dfrac{1}{u}\right)\right]}}
\end{equation}
如果$\Delta \theta$与$2\pi$的比值是有理数,轨道是闭合的。一般来说,在任意形式的$V(r)$情况下,轨道闭合的情况是罕见的。在任意形式的$V(r)$情况下,$\Delta \theta$与$2\pi$的比值并不是有理数,因此在一般情况下,有限运动的轨道并不是闭合的。在无限长的时间进程中,轨道无数次经过$r_{\max}$和$r_{\min}$的位置而填满由两个圆所限制的整个圆环,如图\ref{非闭合轨道}所示。只有$V(r) = \dfrac{\alpha}{r}$和$V(r) = \alpha r^2$两种形式的势场中,任意有限运动的轨道都是闭合的。

\begin{figure}[htb]
\centering
\begin{asy}
	texpreamble("\usepackage{xeCJK}");
	texpreamble("\setCJKmainfont{SimSun}");
	usepackage("amsmath");
	import graph;
	import math;
	size(300);
	//进动的非闭合轨道
	real p,e,omega,rmax,rmin;
	pair O;
	path g;
	O = (0,0);
	p = 1;
	e = 0.7;
	omega = 1-0.06;
	rmax = p/(1-e);
	rmin = p/(1+e);
	pair rotell(real theta){
		real r = p/(1+e*cos(omega*theta));
		return (-r*cos(theta),r*sin(theta));
	}
	g = graph(rotell,16*pi/omega,-1.25*pi,10000);
	draw(g);
	for(real rp=1;rp>0;rp=rp-0.06){
		add(arrow(g,invisible,FillDraw(black),Relative(rp)));
	}
	draw(scale(rmax)*unitcircle,dashed);
	draw(scale(rmin)*unitcircle,dashed);
	draw(rotell(15*pi/omega)--1.06*rotell(15*pi/omega));
	draw(rotell(13*pi/omega)--1.06*rotell(13*pi/omega));
	draw(Label("$\Delta \theta$",MidPoint,Relative(E)),arc(O,length(1.03*rotell(15*pi/omega)),degrees(rotell(15*pi/omega)),degrees(rotell(13*pi/omega))),Arrows);
	draw(Label("$r_{\min}$",MidPoint,Relative(E)),O--rmin*dir(-20),Arrow);
	draw(Label("$r_{\max}$",Relative(0.7),Relative(W)),O--rmax*dir(110),Arrow);
	draw(O--(rmax+0.01)*dir(90),invisible);
\end{asy}
\caption{非闭合轨道}
\label{非闭合轨道}
\end{figure}

也可通过系统的运动方程\eqref{两体运动方程}得到轨道满足的微分方程。考虑
\begin{equation*}
	\dot{r} = \dot{\theta} \frac{\mathrm{d} r}{\mathrm{d} \theta} = \frac{l}{\mu r^2} \frac{\mathrm{d} r}{\mathrm{d} \theta} = -\frac{l}{\mu} \frac{\mathrm{d} u}{\mathrm{d} \theta}
\end{equation*}
所以
\begin{equation*}
	\ddot{r} = \dot{\theta} \frac{\mathrm{d} \dot{r}}{\mathrm{d} \theta} = \frac{l}{\mu r^2} \frac{\mathrm{d}}{\mathrm{d} \theta} \left(-\frac{l}{\mu} \frac{\mathrm{d} u}{\mathrm{d} \theta}\right) = -\frac{l^2u^2}{\mu^2} \frac{\mathrm{d}^2 u}{\mathrm{d} \theta^2}
\end{equation*}
由此可得
\begin{equation*}
	-\frac{l^2u^2}{\mu} \frac{\mathrm{d}^2 u}{\mathrm{d} \theta^2} - \mu r\left(\frac{l}{\mu r^2}\right)^2 = f(r)
\end{equation*}
整理可得二阶常微分方程
\begin{equation}
	\frac{\mathrm{d}^2 u}{\mathrm{d} \theta^2} + u = -\frac{\mu}{l^2 u^2} f\left(\frac{1}{u}\right)
	\label{Binet方程}
\end{equation}
称为{\heiti Binet方程}。

\subsection{圆轨道的存在性与稳定性}

圆轨道存在,即径向运动处于静止状态,即可以表示为
\begin{equation*}
	\frac{\mathrm{d} U}{\mathrm{d} r} \bigg|_{r_m} = 0
\end{equation*}
稳定要求在圆轨道上等效势能取极小值,即
\begin{equation*}
	\frac{\mathrm{d}^2 U}{\mathrm{d} r^2} \bigg|_{r_m} > 0
\end{equation*}

\begin{example}
利用等效势定性分析质点在势场
\begin{equation*}
	V(r) = -\frac{\alpha}{r^n},\quad \alpha,n > 0
\end{equation*}
中的运动。
\end{example}
\begin{solution}
势场对应的力
\begin{equation*}
	f(r) = -V'(r) = -\frac{\alpha n}{r^{n+1}}<0
\end{equation*}
即势场为吸引势。则等效势为
\begin{equation*}
	U(r) = \frac{l^2}{2\mu r^2} - \frac{\alpha}{r^n}
\end{equation*}
径向运动方程
\begin{equation*}
	\frac12 \mu \dot{r}^2 + U(r) = E
\end{equation*}
需要满足$\dfrac12 \mu \dot{r}^2 \geqslant 0$,即
\begin{equation*}
	U(r) \leqslant E
\end{equation*}
上面的不等式即决定了质点径向运动的范围。计算
\begin{equation*}
	U'(r) = -\frac{l^2}{\mu r^3} + \frac{\alpha n}{r^{n+1}} = 0
\end{equation*}
可得$U(r)$的极值点为
\begin{equation*}
	r_m = \left(\frac{l^2}{n\alpha \mu}\right)^{\frac{1}{2-n}}
\end{equation*}
极值点存在要求$n \neq 2$,因此势场分为$0<n<2$、$n>2$和$n=2$三种情形。
\begin{enumerate}
	\item $0<n<2$。当$E=E_0=U_{\mathrm{min}}$时,质点沿圆轨道运动,由于是极小值轨道是稳定的。
	
	当$E=E_1<0$时,运动被限制在$r_{\mathrm{min}} \leqslant r \leqslant r_{\mathrm{max}}$的范围内,这种状态称为束缚态。按前面的讨论,束缚态的轨道不一定是闭合的。
	
	当$E=E_2=0$时,粒子的运动范围为$r \geqslant r_{\mathrm{min}}$,当$r \to +\infty$时,粒子的速度趋于$0$,轨迹趋于直线。
	
	当$E=E_3>0$时,粒子的运动范围同样为$r \geqslant r_{\mathrm{min}}$,但当$r \to +\infty$时,粒子的动能趋于$E$,轨迹趋于直线。这种状态称为散射态。

	\item $n>2$。当$E=E_0=U_{\mathrm{max}}$时,质点沿圆轨道运动,由于是极大值轨道是不稳定的。
	
	当$0<E=E_1<U_{\mathrm{max}}$时,粒子只能出现在$r = r_{\mathrm{min}}$的圆内或$r = r_{\mathrm{max}}$的圆外,在中间的环形区域内粒子是禁戒的。若初始时刻粒子在$r = r_{\mathrm{min}}$的圆内,则$\dot{r} < 0$时,粒子被吸引至力心;当$\dot{r}>0$时,粒子先飞到$r = r_{\mathrm{min}}$的圆周上,然后折回到力心。而如果初始时刻粒子在$r = r_{\mathrm{max}}$的圆外,则$\dot{r} < 0$时,粒子先飞到$r = r_{\mathrm{max}}$的圆周上,然后飞向无穷远处;当$\dot{r}>0$时,粒子直接飞翔无穷远处最终成为自由粒子。
	
	当$E=E_2=0$时或者$E=E_4<0$时,粒子的运动范围为$r = r_{\mathrm{max}}$的圆内,并最终被力心俘获。
	
	当$E=E_3>U_{\mathrm{max}}$时,粒子可在全空间运动。当$\dot{r}>0$时,粒子飞向无穷远最终变为自由粒子;当$\dot{r}<0$时,粒子飞向中心,最后被力心俘获。

\begin{figure}[htb]
\centering
\begin{minipage}[t]{0.45\textwidth}
\centering
\begin{asy}
	texpreamble("\usepackage{xeCJK}");
	texpreamble("\setCJKmainfont{SimSun}");
	usepackage("amsmath");
	import graph;
	import math;
	size(190);
	//0<n<2的情形
	real alpha,beta,n;
	picture tmp;
	real fr(real r){
		return beta/(r**2)-alpha/(r**n);
	}
	real f1r(real r){
		return beta/(r**2);
	}
	real f2r(real r){
		return -alpha/(r**n);
	}
	alpha = 0.8*4;
	beta = 1*4;
	n = 1;
	draw(tmp,xscale(0.4)*graph(fr,0.3,18,operator..),linewidth(1bp));
	draw(tmp,xscale(0.4)*graph(f1r,0.3,18,operator..),dashed);
	draw(tmp,xscale(0.4)*graph(f2r,0.3,18,operator..),dashed);
	xequals(tmp,0,black);
	yequals(tmp,0,black);
	real rm,r1,r2;
	rm = (2*beta/alpha/n)**(1/(2-n));
	draw(tmp,xscale(0.4)*((0,fr(rm))--(rm,fr(rm))--(rm,0)),red);
	r1 = (beta/alpha)**(1/(2-n))+0.225;
	r2 = 8;
	draw(tmp,xscale(0.4)*((0,fr(r2))--(r2,fr(r2))--(r2,0)),green);
	draw(tmp,xscale(0.4)*((r1,0)--(r1,fr(r1))),green);
	draw(tmp,xscale(0.4)*((0,fr(1))--(18,fr(1))),blue);
	draw(tmp,xscale(0.4)*((1,fr(1))--(1,0)),blue);
	path clp;
	clp = box((-0.1,-fr(0.7)),(7.2,fr(0.7)));
	clip(tmp,clp);
	add(tmp);
	label("$\dfrac{l^2}{2\mu r^2}$",xscale(0.4)*(2,f1r(2)),NE);
	label("$V$",xscale(0.4)*(2,f2r(2)),SE);
	label("$E_0$",(0,fr(rm)),SW,red);
	label("$E_1$",(0,fr(r2)),W,green);
	label("$E_2$",(0,0),NW);
	label("$E_3$",(0,fr(1)),W,blue);
	label("$r$",(7.2,0),E);
	label("$U$",(0,fr(0.7)),W);
\end{asy}
\caption{$0<n<2$的情形}
\label{0<n<2的情形}
\end{minipage}
\hspace{0.7cm}
\begin{minipage}[t]{0.45\textwidth}
\centering
\begin{asy}
	texpreamble("\usepackage{xeCJK}");
	texpreamble("\setCJKmainfont{SimSun}");
	usepackage("amsmath");
	import graph;
	import math;
	size(190);
	//n>2的情形
	real alpha,beta,n;
	picture tmp;
	real fr(real r){
		return beta/(r**2)-alpha/(r**n);
	}
	real f1r(real r){
		return beta/(r**2);
	}
	real f2r(real r){
		return -alpha/(r**n);
	}
	alpha = 0.8*4;
	beta = 1*4;
	n = 3;
	draw(tmp,xscale(0.8)*graph(fr,0.3,18,operator..),linewidth(1bp));
	draw(tmp,xscale(0.8)*graph(f1r,0.3,18,operator..),dashed);
	draw(tmp,xscale(0.8)*graph(f2r,0.3,18,operator..),dashed);
	xequals(tmp,0,black);
	yequals(tmp,0,black);
	real rm,r1,r2,r3,E3;
	rm = (2*beta/alpha/n)**(1/(2-n));
	draw(tmp,xscale(0.8)*((0,fr(rm))--(rm,fr(rm))--(rm,0)),red);
	r1 = (beta/alpha)**(1/(2-n))+0.069;
	r2 = 2.5;
	draw(tmp,xscale(0.8)*((0,fr(r2))--(18,fr(r2))),green);
	draw(tmp,xscale(0.8)*((r2,fr(r2))--(r2,0)),green);
	draw(tmp,xscale(0.8)*((r1,0)--(r1,fr(r1))),green);
	r3 = 0.7;
	draw(tmp,xscale(0.8)*((0,fr(r3))--(r3,fr(r3))--(r3,0)),orange);
	E3 = 1.8;
	draw(tmp,xscale(0.8)*((0,E3)--(18,E3)),blue);
	path clp;
	clp = box((-0.1,-fr(0.6)),(7.2,fr(0.6)));
	clip(tmp,clp);
	add(tmp);
	label("$\dfrac{l^2}{2\mu r^2}$",xscale(0.8)*(1.3,f1r(1.3)),NE);
	label("$V$",xscale(0.8)*(1.3,f2r(1.3)),SE);
	label("$E_0$",(0,fr(rm)),W,red);
	label("$E_1$",(0,fr(r2)),W,green);
	label("$E_2$",(0,0),W);
	label("$E_3$",(0,E3),W,blue);
	label("$E_4$",(0,fr(r3)),W,orange);
	label("$r$",(7.2,0),E);
	label("$U$",(0,abs(fr(0.6))),W);
	//draw((0,0)--(7.9,0),invisible);
\end{asy}
\caption{$n>2$的情形}
\label{n>2的情形}
\end{minipage}

\begin{minipage}[t]{0.45\textwidth}
\centering
\begin{asy}
	texpreamble("\usepackage{xeCJK}");
	texpreamble("\setCJKmainfont{SimSun}");
	usepackage("amsmath");
	import graph;
	import math;
	size(190);
	//n=2的情形1
	real alpha,beta,n;
	picture tmp;
	real fr(real r){
		return beta/(r**2)-alpha/(r**n);
	}
	real f1r(real r){
		return beta/(r**2);
	}
	real f2r(real r){
		return -alpha/(r**n);
	}
	alpha = 1.6*1.7;
	beta = 1*1.7;
	n = 2;
	draw(tmp,graph(fr,0.3,18,operator..),linewidth(1bp));
	draw(tmp,graph(f1r,0.3,18,operator..),dashed);
	draw(tmp,graph(f2r,0.3,18,operator..),dashed);
	xequals(tmp,0,black);
	yequals(tmp,0,black);
	real E3,r1;
	r1 = 1.5;
	draw(tmp,(r1,0)--(r1,fr(r1))--(0,fr(r1)),green);
	E3 = 1.1;
	draw(tmp,(0,E3)--(18,E3),blue);
	path clp;
	clp = box((-0.1,-fr(0.6)),(6,fr(0.6)));
	clip(tmp,clp);
	add(tmp);
	label("$\dfrac{l^2}{2\mu r^2}$",(1,f1r(1)),NE);
	label("$V$",(1.3,f2r(1.3)),SE);
	label("$E_1$",(0,fr(r1)),W,green);
	label("$E_2$",(0,0),W);
	label("$E_3$",(0,E3),W,blue);
	label("$r$",(6,0),E);
	label("$U$",(0,abs(fr(0.6))),W);
	//draw((0,0)--(6.9,0),invisible);
\end{asy}
\caption{$n=2$的情形($\alpha>\dfrac{l^2}{2\mu}$)}
\label{n=2的情形1}
\end{minipage}
\hspace{0.7cm}
\begin{minipage}[t]{0.45\textwidth}
\centering
\begin{asy}
	texpreamble("\usepackage{xeCJK}");
	texpreamble("\setCJKmainfont{SimSun}");
	usepackage("amsmath");
	import graph;
	import math;
	size(190);
	//n=2的情形2
	real alpha,beta,n;
	picture tmp;
	real fr(real r){
		return beta/(r**2)-alpha/(r**n);
	}
	real f1r(real r){
		return beta/(r**2);
	}
	real f2r(real r){
		return -alpha/(r**n);
	}
	alpha = 1*1.7;
	beta = 1.6*1.7;
	n = 2;
	draw(tmp,graph(fr,0.3,18,operator..),linewidth(1bp));
	draw(tmp,graph(f1r,0.3,18,operator..),dashed);
	draw(tmp,graph(f2r,0.3,18,operator..),dashed);
	xequals(tmp,0,black);
	yequals(tmp,0,black);
	real r1;
	r1 = 1;
	draw(tmp,(r1,0)--(r1,fr(r1))--(0,fr(r1)),green);
	path clp;
	clp = box((-0.1,-fr(0.6)),(6,fr(0.6)));
	clip(tmp,clp);
	add(tmp);
	label("$\dfrac{l^2}{2\mu r^2}$",(1.2,f1r(1.2)),NE);
	label("$V$",(1.3,f2r(1.3)),SE);
	label("$E_1$",(0,fr(r1)),W,green);
	label("$E_2$",(0,0),W);
	label("$r$",(6,0),E);
	label("$U$",(0,abs(fr(0.6))),W);
	//draw((0,0)--(6.9,0),invisible);
\end{asy}
\caption{$n=2$的情形($\alpha<\dfrac{l^2}{2\mu}$)}
\label{n=2的情形2}
\end{minipage}
\end{figure}
	\item $n=2$而且$\alpha>\dfrac{l^2}{2\mu}$。当$E=E_1<0$时,粒子只能出现在$r = r_{\mathrm{max}}$的圆内,并最终被力心俘获。
	
	当$E=E_2=0$时,粒子可在全空间运动,最终粒子将飞向无穷远处且趋于静止。
	
	当$E=E_3>0$时,粒子可在全空间运动,最终将飞向无穷远处最终称为自由粒子。
	\item $n=2$而且$\alpha<\dfrac{l^2}{2\mu}$。在这种情形下,只能$E>0$,此时粒子只能出现在$r = r_{\mathrm{max}}$的圆外,最终飞向无穷远而称为自由粒子。
\end{enumerate}
\end{solution}

\begin{example}
求势场
\begin{equation*}
	V(r) = \alpha r^n
\end{equation*}
存在稳定圆轨道的条件。
\end{example}
\begin{solution}
等效势为
\begin{equation*}
	U(r) = \frac{l^2}{2\mu r^2} + \alpha r^n
\end{equation*}
存在圆轨道要求
\begin{equation*}
	U'(r_m) = -\frac{l^2}{\mu r_m^3} + \alpha n r_m^{n-1} = 0
\end{equation*}
即有
\begin{equation*}
	\alpha n r_m^{n+2} = \frac{l^2}{\mu}
\end{equation*}
由此即要求$\alpha n>0$。此时
\begin{equation*}
	f(r) = -V'(r) = -\alpha n r^{n-1} < 0
\end{equation*}
即存在圆轨道要求势场为吸引力场。下面考虑稳定性,稳定性要求
\begin{equation*}
	U''(r_m) = \frac{1}{r_m^4} \left[(n-1)\alpha n r_m^{n+2} + \frac{3l^2}{\mu}\right] = (n+2) \frac{l^2}{\mu r_m^4} > 0
\end{equation*}
即圆轨道稳定要求$n>-2$。

综上,存在稳定圆轨道的条件为$\alpha>0,n>0$或者$\alpha<0,-2<n<0$。
\end{solution}

\section{距离反比势场与Kepler问题}

\begin{figure}[htb]
\centering
\begin{asy}
	texpreamble("\usepackage{xeCJK}");
	texpreamble("\setCJKmainfont{SimSun}");
	usepackage("amsmath");
	import graph;
	import math;
	size(300);
	//距离反比势场势能曲线图
	real alpha,beta,xmax,ymax;
	picture tmp;
	real fr(real r){
		return beta/(r**2)+alpha/r;
	}
	real f1r(real r){
		return beta/(r**2);
	}
	real f2r(real r){
		return alpha/r;
	}
	alpha = -2.5*4;
	beta = 2*4;
	ymax = abs(fr(0.53));
	draw(tmp,graph(fr,0.3,18,operator..),linewidth(1bp)+red,Label("引力等效势"));
	draw(tmp,graph(f2r,0.3,18,operator..),dashed+red,Label("引力势"));
	alpha = 2.5*4;
	draw(tmp,graph(fr,0.3,18,operator..),linewidth(1bp)+blue,Label("斥力等效势"));
	draw(tmp,graph(f2r,0.3,18,operator..),dashed+blue,Label("斥力势"));
	draw(tmp,graph(f1r,0.3,18,operator..),dashed,Label("离心势"));
	xequals(tmp,0,black);
	yequals(tmp,0,black);
	path clp;
	xmax = 15;
	clp = box((-0.1,-2/3*ymax),(xmax,ymax));
	clip(tmp,clp);
	add(tmp);
	label("$r$",(xmax,0),E);
	label("$U$",(0,ymax),W);
	fill(box((xmax+0.1,ymax-0.2),(xmax-7.4,ymax-5.8)),black);
	add(legend(),(xmax,ymax),SW,UnFill);
	//draw((0,0)--(xmax+0.9,0),invisible);
\end{asy}
\caption{距离反比势场势能曲线图}
\label{距离反比势场势能曲线图}
\end{figure}

距离反比势可以表示为
\begin{equation*}
	V(r) = \frac{\alpha}{r},\quad f(r) = -V'(r) = \frac{\alpha}{r^2}
\end{equation*}
此势场当$\alpha<0$时表现为引力场,当$\alpha>0$时表现为斥力场。其等效势为
\begin{equation}
	U(r) = \frac{l^2}{2\mu r^2} + \frac{\alpha}{r}
\end{equation}
对于引力的情形,在
\begin{equation}
	r_m = -\frac{l^2}{\mu \alpha}
\end{equation}
时,等效势有极小值
\begin{equation}
	U_{\mathrm{min}} = -\frac{\mu \alpha^2}{2l^2}
\end{equation}
对于斥力的情形,等效势没有极值,仍记
\begin{equation*}
	r_m = -\frac{l^2}{\mu \alpha}
\end{equation*}

当$\alpha<0$时(即引力情形)称为{\bf Kepler问题}。

\subsection{轨道方程}

对于Kepler问题,Binet方程\eqref{Binet方程}为
\begin{equation}
	\frac{\mathrm{d}^2 u}{\mathrm{d} \theta^2} + u = \frac{1}{r_m}
	\label{Kepler问题的轨道微分方程}
\end{equation}
方程\eqref{Kepler问题的轨道微分方程}的通解为
\begin{equation*}
	u = \frac{1}{r_m} + A\cos(\theta-\theta_0)
\end{equation*}
即
\begin{equation*}
	r = \frac{r_m}{1+Ar_m \cos (\theta-\theta_0)}
\end{equation*}
适当选取极轴的方向,可以使得$\theta_0=0$,此时轨道方程为
\begin{equation*}
	r = \frac{r_m}{1+Ar_m \cos \theta}
\end{equation*}
记
\begin{equation*}
	p = |r_m| = \frac{l^2}{\mu |\alpha|},\quad e = Ap
\end{equation*}
其中$p$为{\heiti 半通径},$e$为{\heiti 离心率},由此可得Kepler问题的轨道方程为
\begin{equation}
	r = \frac{p}{1+e\cos \theta}
	\label{Kepler问题的轨道方程}
\end{equation}
% 当$\alpha>0$时(斥力),轨道方程为\footnote{在斥力的极坐标方程中,需要允许极径取负值。}
% \begin{equation}
	% r = \frac{p}{e\cos \theta-1}
% \end{equation}
% 首先考虑引力的情形,
根据轨道方程可得质点与力心的最短距离为
\begin{equation*}
	r_{\mathrm{min}} = \frac{p}{1+e}
\end{equation*}
此点一般称为{\heiti 近日点}。在近日点有$\dot{r}\big|_{r=r_{\mathrm{min}}} = 0$,因此$r_{\mathrm{min}}$满足
\begin{equation*}
	\frac{l^2}{2\mu r_{\mathrm{min}}^2} + \frac{\alpha}{r_{\mathrm{min}}} = E
\end{equation*}
由此可得离心率与系统总能量的关系
\begin{equation*}
	e = \sqrt{1+\frac{2El^2}{\mu\alpha^2}} = \sqrt{1-\frac{E}{U_{\mathrm{min}}}}
\end{equation*}
% 对于斥力的情形,根据轨道方程可得质点与原点的最短距离为\footnote{斥力情形下,轨道为双曲线的右支,此时矢径取负值}
% \begin{equation*}
	% r_{\mathrm{min}} = -\frac{p}{1+e}
% \end{equation*}
% 此时同样有
% \begin{equation*}
	% \frac{l^2}{2\mu r_{\mathrm{min}}^2} + \frac{\alpha}{r_{\mathrm{min}}} = E
% \end{equation*}
% 由此可得斥力情形下离心率与系统总能量的关系
% \begin{equation*}
	% e = \sqrt{1+\frac{2El^2}{\mu\alpha^2}}
% \end{equation*}
% 综上,无论对于引力还是斥力场,都有
% \begin{equation}
	% p = |r_m| = \frac{l^2}{\mu |\alpha|},\quad e = \sqrt{1+\frac{2El^2}{\mu\alpha^2}}
% \end{equation}

\begin{figure}[htb]
\centering
\begin{asy}
	texpreamble("\usepackage{xeCJK}");
	texpreamble("\setCJKmainfont{SimSun}");
	usepackage("amsmath");
	import graph;
	import math;
	size(250);
	//各种情形下轨道的形状
	pair O;
	real p,e[],ee,xmax,ymax;
	picture tmp;
	O = (0,0);
	real r(real theta){
		return p/(1+ee*cos(theta));
	}
	real nr(real theta){
		return p/(ee*cos(theta)-1);
	}
	pair xy(real theta){
		return (r(theta)*cos(theta),r(theta)*sin(theta));
	}
	pair nxy(real theta){
		return (nr(theta)*cos(theta),nr(theta)*sin(theta));
	}
	p = 1;
	e[1] = 0.4;
	ee = e[1];
	draw(tmp,graph(xy,0,2*pi),green+linewidth(1bp));
	e[2] = 1;
	ee = e[2];
	draw(tmp,graph(xy,-0.7*pi,0.7*pi),linewidth(1bp));
	e[3] = 2.5;
	ee = e[3];
	draw(tmp,graph(xy,-0.6*pi,0.6*pi),blue+linewidth(1bp));
	//draw(tmp,graph(nxy,-0.34*pi,0.34*pi),purple+linewidth(1bp));
	draw(tmp,scale(p)*unitcircle,red+linewidth(1bp));
	draw(tmp,(-100,0)--(100,0));
	draw(tmp,(0,-100)--(0,100));
	label(tmp,"$p$",(0,p/2),W);
	dot(tmp,O);
	xmax = p/(1-e[1])*1.1;
	ymax = xmax;
	path clp = box((-xmax,-ymax),(xmax,ymax));
	clip(tmp,clp);
	add(tmp);
	ee = e[1];
	label("$U_{\mathrm{min}}<E<0$",xy(-0.7*pi),SW,green);
	label("$E=0$",(-0.7*xmax,ymax),N);
	label("$E>0$",(-0.25*xmax,ymax),N,blue);
	//label("$\alpha<0$",(-0.25*xmax,-ymax),S,blue);
	label("$E>0$",(0.7*xmax,ymax),N,purple);
	//label("$\alpha>0$",(0.7*xmax,-ymax),S,purple);
	label("$E=U_{\mathrm{min}}$",relpoint(scale(p)*unitcircle,0.06),E,red);
\end{asy}
\caption{各种情形下轨道的形状}
\label{各种情形下轨道的形状}
\end{figure}

%在斥力的情形下,$\alpha>0$,$E>0$,则有$e>1$,轨道为双曲线。

在引力的情形下,$\alpha<0$,按照体系能量的区别可以分为以下4种情况:
\begin{enumerate}
	\item 当$E>0$时,$e>1$,轨道为双曲线;
	\item 当$E=0$时,$e=1$,轨道为拋物线;
	\item 当$U_{\mathrm{min}}<E<0$时,$0<e<1$,轨道为椭圆;
	\item 当$E=U_{\mathrm{min}}$时,$e=0$,轨道为圆。
\end{enumerate}
各种情形下,轨道的形状如图\ref{各种情形下轨道的形状}所示。

当质点沿着轨道运动时,坐标对时间的依赖关系可以用式\eqref{t与r的关系}得到。它可以表示为下面所述的一种方便的参数形式。

首先研究椭圆轨道。对椭圆轨道,质点与力心的最大距离为
\begin{equation*}
	r_{\max} = \frac{p}{1-e}
\end{equation*}
此点一般称为{\bf 远日点}。因此可有椭圆的半长轴为
\begin{equation*}
	a = \frac12 \left(\frac{p}{1-e}+\frac{p}{1+e}\right) = \frac{\alpha}{2E} > 0
\end{equation*}
由此,根据式\eqref{t与r的关系}可得
\begin{align*}
	t-t_0 & = \sqrt{\frac{\mu}{2}} \int_{r_{\min}}^r \frac{\mathrm{d}r}{\sqrt{E-U(r)}} = \sqrt{\frac{\mu}{2|E|}} \int_{r_{\min}}^r \frac{r\mathrm{d}r}{\sqrt{-r^2-\dfrac{\alpha}{|E|}r-\dfrac{l^2}{2\mu|E|}}} \\
	& = \sqrt{\frac{\mu a}{|\alpha|}} \int_{r_{\min}}^r \frac{r\mathrm{d}r}{\sqrt{a^2e^2-(r-a)^2}}
\end{align*}
利用变换
\begin{equation*}
	r-a = -ae\cos \xi
\end{equation*}
可得
\begin{equation*}
	t-t_0 = \sqrt{\frac{\mu a^3}{|\alpha|}} \int_0^\xi (1-e\cos\xi) \mathrm{d}\xi = \sqrt{\frac{\mu a^3}{|\alpha|}} (\xi-e\sin \xi)
\end{equation*}
如果将近日点的时刻取为$t_0=0$,可得{\bf Kepler方程}
\begin{equation}
	t = \sqrt{\frac{\mu a^3}{|\alpha|}} (\xi-e\sin \xi)
	\label{Kepler方程}
\end{equation}
此时即可有关于到力心距离的参数方程
\begin{equation}
\begin{cases}
	r=a(1-e\cos \xi) \\
	t = \sqrt{\dfrac{\mu a^3}{|\alpha|}} (\xi-e\sin \xi)
\end{cases}
\end{equation}
再结合轨道方程\eqref{Kepler问题的轨道方程},可得轨道的参数方程
\begin{equation}
\begin{cases}
	x = r\cos \theta = a(\cos \xi-e) \\
	y = \sqrt{r^2-x^2} = a\sqrt{1-e^2}\sin \xi \\
	t = \sqrt{\dfrac{\mu a^3}{|\alpha|}} (\xi-e\sin \xi)
\end{cases}
\end{equation}
沿着椭圆轨道运动一整圈即对应着参数$\xi$从$0$到$2\pi$,其中参数$\xi$称为椭圆的{\bf 圆心角},其几何意义如图\ref{圆心角的几何意义}所示。

\begin{figure}[htb]
\centering
\begin{asy}
	texpreamble("\usepackage{xeCJK}");
	texpreamble("\setCJKmainfont{SimSun}");
	usepackage("amsmath");
	import graph;
	import math;
	size(250);
	//圆心角的几何意义
	real x,a,b,c,theta,r;
	a = 2;
	b = 1.6;
	x = 2.5;
	c = sqrt(a^2-b^2);
	draw((-x,0)--(x,0));
	draw((0,-x)--(0,x));
	draw(xscale(a)*yscale(b)*unitcircle,linewidth(0.8bp));
	draw(scale(a)*unitcircle);
	dot((c,0));
	theta = 30;
	draw((0,0)--a*dir(theta));
	pair P;
	P = intersectionpoint(xscale(a)*yscale(b)*unitcircle,a*dir(theta)--(a*dir(theta).x,0));
	draw((c,0)--P);
	draw(a*dir(theta)--(a*dir(theta).x,0));
	label("$p$",(P+(a*dir(theta).x,0))/2,E);
	r = 0.3;
	draw(Label("$\xi$",MidPoint,Relative(E)),arc((0,0),r,0,theta));
	r = 0.15;
	draw(Label("$\theta$",MidPoint,Relative(E)),arc((c,0),r,0,degrees(P-(c,0))));
	label("$a$",(a,0),SE);
	label("$b$",(0,b),SW);
	label("$ae$",(c,0),S);
\end{asy}
\caption{圆心角的几何意义}
\label{圆心角的几何意义}
\end{figure}

对于双曲线轨道,同样有双曲线的半轴长为
\begin{equation*}
	a = \frac{|\alpha|}{2E}
\end{equation*}
同样,根据式\eqref{t与r的关系}可得
\begin{align*}
	t-t_0 & = \sqrt{\frac{\mu}{2}} \int_{r_{\min}}^r \frac{\mathrm{d}r}{\sqrt{E-U(r)}} = \sqrt{\frac{\mu}{2E}} \int_{r_{\min}}^r \frac{r\mathrm{d}r}{\sqrt{-r^2-\dfrac{\alpha}{E}r-\dfrac{l^2}{2\mu E}}} \\
	& = \sqrt{\frac{\mu a}{|\alpha|}} \int_{r_{\min}}^r \frac{r\mathrm{d}r}{\sqrt{(r+a)^2-a^2e^2}}
\end{align*}
利用变换
\begin{equation*}
	r+a = ae\cosh \xi
\end{equation*}
可得
\begin{equation*}
	t-t_0 = \sqrt{\frac{\mu a^3}{|\alpha|}} \int_0^\xi (e\cosh\xi-1) \mathrm{d}\xi = \sqrt{\frac{\mu a^3}{|\alpha|}} (e\sinh \xi-\xi)
\end{equation*}
如果将近日点的时刻取为$t_0=0$,可得
\begin{equation}
	t = \sqrt{\frac{\mu a^3}{|\alpha|}} (e\sinh \xi-\xi)
\end{equation}
此时即可有关于到力心距离的参数方程
\begin{equation}
\begin{cases}
	r=a(e\cosh \xi-1) \\
	t = \sqrt{\dfrac{\mu a^3}{|\alpha|}} (e\sinh \xi-\xi)
\end{cases}
\end{equation}
再结合轨道方程\eqref{Kepler问题的轨道方程},可得轨道的参数方程
\begin{equation}
\begin{cases}
	x = r\cos \theta = a(e-\cosh \xi) \\
	y = \sqrt{r^2-x^2} = a\sqrt{e^2-1}\sinh \xi \\
	t = \sqrt{\dfrac{\mu a^3}{|\alpha|}} (e\sinh \xi-\xi)
\end{cases}
\end{equation}
其中参数$\xi$的取值范围为$(-\infty,+\infty)$,近日点即对应$\xi=0$。

对于抛物线轨道,根据式\eqref{t与r的关系}可得
\begin{align*}
	t-t_0 & = \sqrt{\frac{\mu}{2}} \int_{r_{\min}}^r \frac{\mathrm{d}r}{\sqrt{E-U(r)}} = \sqrt{\frac{\mu}{2|\alpha|}} \int_{r_{\min}}^r \frac{r\mathrm{d}r}{\sqrt{r-\dfrac{l^2}{2\mu |\alpha|}}} \\
	& = \sqrt{\frac{\mu}{2|\alpha|}} \int_{r_{\min}}^r \frac{r\mathrm{d}r}{\sqrt{r-\dfrac{p}{2}}}
\end{align*}
利用变换
\begin{equation*}
	r = \frac{p}{2}(1+\eta^2)
\end{equation*}
可得
\begin{equation*}
	t-t_0 = \sqrt{\frac{\mu p^3}{|\alpha|}} \int_0^\eta \frac12(1+\eta^2) \mathrm{d}\eta = \sqrt{\frac{\mu p^3}{|\alpha|}} \frac{\eta}{2}\left(1+\frac{\eta^3}{3}\right)
\end{equation*}
如果将近日点的时刻取为$t_0=0$,可得
\begin{equation}
	t = \sqrt{\frac{\mu p^3}{|\alpha|}} \frac{\eta}{2}\left(1+\frac{\eta^3}{3}\right)
\end{equation}
此时即可有关于到力心距离的参数方程
\begin{equation}
\begin{cases}
	r = \dfrac{p}{2}(1+\eta^2) \\
	t = \sqrt{\dfrac{\mu p^3}{|\alpha|}} \dfrac{\eta}{2}\left(1+\dfrac{\eta^3}{3}\right)
\end{cases}
\end{equation}
再结合轨道方程\eqref{Kepler问题的轨道方程},可得轨道的参数方程
\begin{equation}
\begin{cases}
	x = r\cos \theta = \dfrac{p}{2}(1-\eta^2) \\
	y = \sqrt{r^2-x^2} = p\eta \\
	t = \sqrt{\dfrac{\mu p^3}{|\alpha|}} \dfrac{\eta}{2}\left(1+\dfrac{\eta^3}{3}\right)
\end{cases}
\end{equation}
其中参数$\eta$的取值范围为$(-\infty,+\infty)$,近日点即对应$\eta=0$。

\subsection{Laplace-Runge-Lenz矢量}

在任意有心力场$V(r) = \dfrac{\alpha}{r}$中存在其特有的运动积分,称为{\bf Laplace-Runge-Lenz矢量}
\begin{equation}
	\mbf{A} = \mbf{v} \times \mbf{l} + \frac{\alpha \mbf{r}}{r}
\end{equation}
直接求其时间导数可得
\begin{align*}
	\frac{\mathrm{d}\mbf{A}}{\mathrm{d}t} & = \dot{\mbf{v}} \times \mbf{l} + \frac{\alpha \mbf{v}}{r} - \frac{\alpha \mbf{r}(\mbf{v}\cdot \mbf{r})}{r^3} \\
	& = \dot{\mbf{v}} \times (\mu \mbf{r} \times \mbf{v}) + \frac{\alpha \mbf{v}}{r} - \frac{\alpha \mbf{r}(\mbf{v}\cdot \mbf{r})}{r^3} \\
	& = \mu\mbf{r}(\mbf{v}\cdot\dot{\mbf{v}}) - \mu\mbf{v}(\mbf{r}\cdot \dot{\mbf{v}}) + \frac{\alpha \mbf{v}}{r} - \frac{\alpha \mbf{r}(\mbf{v}\cdot \mbf{r})}{r^3}
\end{align*}
将运动方程$\mu \dot{\mbf{v}} = \dfrac{\alpha\mbf{r}}{r^3}$代入,即可得
\begin{equation*}
	\frac{\mathrm{d}\mbf{A}}{\mathrm{d}t} = \mbf{0}
\end{equation*}
由此便验证了Laplace-Runge-Lenz矢量$\mbf{A}$是运动积分。

\begin{figure}[htb]
\centering
\begin{asy}
	texpreamble("\usepackage{xeCJK}");
	texpreamble("\setCJKmainfont{SimSun}");
	usepackage("amsmath");
	import graph;
	import math;
	size(350);
	//Laplace-Runge-Lenz矢量
	real p,e,alpha,A;
	p = 1;
	e = 0.8;
	pair ell(real theta){
		real r;
		r = p/(1+e*cos(theta));
		return (r*cos(theta),r*sin(theta));
	}
	alpha = 1;
	draw(graph(ell,0,2*pi,1000),linewidth(0.8bp));
	A = alpha*e;
	pair P[];
	P[0] = ell(0);
	P[1] = ell(2.2);
	P[2] = ell(pi);
	P[3] = ell(3/2*pi);
	draw(Label("$\boldsymbol{r}$",MidPoint,Relative(W)),(0,0)--P[0],Arrow);
	draw(Label("$\boldsymbol{A}$",MidPoint,S),P[0]--P[0]+A*dir(0),red,Arrow);
	draw(Label("$\dfrac{\alpha\boldsymbol{r}}{r}$",MidPoint,S),shift(A*dir(0))*(P[0]+alpha*dir(0)--P[0]),darkgreen,Arrow);
	draw(Label("$\boldsymbol{v}\times\boldsymbol{l}$",MidPoint,N),shift(0.09*dir(90))*(P[0]--P[0]+alpha*dir(0)+A*dir(0)),blue,Arrow);
	draw(Label("$\boldsymbol{r}$",MidPoint,Relative(W)),(0,0)--P[1],Arrow);
	draw(Label("$\boldsymbol{A}$",MidPoint,Relative(W)),shift(alpha*dir(2.2*180/pi))*(P[1]--P[1]+A*dir(0)),red,Arrow);
	draw(Label("$\dfrac{\alpha\boldsymbol{r}}{r}$",Relative(0.1),Relative(E)),P[1]+alpha*dir(2.2*180/pi)--P[1],darkgreen,Arrow);
	draw(Label("$\boldsymbol{v}\times\boldsymbol{l}$",MidPoint,Relative(E)),P[1]--P[1]+alpha*dir(2.2*180/pi)+A*dir(0),blue,Arrow);
	draw(Label("$\boldsymbol{r}$",MidPoint,Relative(E)),(0,0)--P[2],Arrow);
	draw(Label("$\boldsymbol{A}$",MidPoint,S),P[2]-A*dir(0)--P[2],red,Arrow);
	draw(Label("$\dfrac{\alpha\boldsymbol{r}}{r}$",MidPoint,N),shift(0.09*dir(90))*(P[2]-alpha*dir(0)--P[2]),darkgreen,Arrow);
	draw(Label("$\boldsymbol{v}\times\boldsymbol{l}$",MidPoint,S),P[2]-A*dir(0)--P[2]-alpha*dir(0),blue,Arrow);
	draw(Label("$\boldsymbol{r}$",MidPoint,Relative(E)),(0,0)--P[3],Arrow);
	draw(Label("$\boldsymbol{A}$",MidPoint,S),P[3]+alpha*dir(-90)--P[3]+alpha*dir(-90)+A*dir(0),red,Arrow);
	draw(Label("$\dfrac{\alpha\boldsymbol{r}}{r}$",Relative(0.3),Relative(W)),P[3]+alpha*dir(-90)--P[3],darkgreen,Arrow);
	draw(Label("$\boldsymbol{v}\times\boldsymbol{l}$",MidPoint,Relative(W)),P[3]--P[3]+alpha*dir(-90)+A*dir(0),blue,Arrow);
	dot((0,0));
\end{asy}
\caption{Laplace-Runge-Lenz矢量}
\label{Laplace-Runge-Lenz矢量}
\end{figure}

\subsection{Kepler行星运动定律}

\begin{itemize}
	\item {\heiti Kepler第一定律}:行星以太阳为焦点,沿椭圆轨道运动。椭圆轨道的半长轴半短轴分别为
	\begin{align}
		a & = \frac12 \left(\frac{p}{1+e}+\frac{p}{1-e}\right) = \frac{\alpha}{2E} \\
		b & = \frac{p}{\sqrt{1-e^2}} = \frac{l}{\sqrt{2\mu|E|}}
	\end{align}
	由此可得,能量取决于长轴,与形状无关;长轴相同,短轴长的轨道角动量大。
	\item {\heiti Kepler第二定律}:太阳与行星连线的扫面速度为常量。即
	\begin{equation*}
		\mathrm{d} \mbf{S} = \frac12 \mbf{r} \times \mathrm{d} \mbf{r}
	\end{equation*}
	因此
	\begin{equation}
		\frac{\mathrm{d} \mbf{S}}{\mathrm{d} t} = \frac12 \mbf{r} \times \mbf{v} = \frac{\mbf{L}}{2\mu} = \text{常矢量}
	\end{equation}
	\item {\heiti Kepler第三定律}:行星公转周期的平方正比于椭圆半长轴的立方。根据第二定律中求出的面积速度,可以求出行星公转的周期
	\begin{equation}
		T = \frac{\pi ab}{l/2\mu} = \pi |\alpha| \sqrt{\frac{\mu}{2|E|^3}} = \pi \sqrt{\frac{4\mu a^3}{|\alpha|}}
	\end{equation}
	由此即
	\begin{equation}
		\frac{T^2}{a^3} = \frac{4\pi^2 \mu}{|\alpha|}
	\end{equation}
\end{itemize}

\section{粒子弹性散射}

粒子散射是获取粒子相互作用信息,进而确定物质围观结构的实验手段之一。

散射前后,两粒子相距无限远,相互作用势可取为零。散射过程中不受外力,或外力可忽略。粒子间的相互作用为两体中心力。

\subsection{弹性散射}

散射过程中,粒子的内部状态不变,因而静能不变。散射前后,粒子的总动量与总动能守恒。

由于散射过程中不受外力作用,因此质心系满足
\begin{equation*}
	\dot{\mbf{V}} = \ddot{\mbf{R}} = \mbf{0}
\end{equation*}
质心系是惯性系。在质心系中,动量守恒和质量守恒可以分别表示为
\begin{subnumcases}{}
	m_1 \mbf{v}'_{1i} + m_2 \mbf{v}'_{2i} = m_1 \mbf{v}'_{1f} + m_2 \mbf{v}'_{2f} = \mbf{0} \label{第四章:弹性散射动量守恒} \\
	\dfrac12 m_1 \mbf{v}'^2_{1i} + \dfrac12 m_2 \mbf{v}'^2_{2i} = \dfrac12 m_1 \mbf{v}'^2_{1f} + \dfrac12 m_2 \mbf{v}'^2_{2f} \label{第四章:弹性散射能量守恒}
\end{subnumcases}
由动量守恒\eqref{第四章:弹性散射动量守恒}可得
\begin{equation*}
	m_1 \mbf{v}'_{1i} = - m_2 \mbf{v}'_{2i},\quad m_1 \mbf{v}'_{1f} = - m_2 \mbf{v}'_{2f}
\end{equation*}
由上式和动量守恒\eqref{第四章:弹性散射能量守恒}可得
\begin{equation*}
	v'_{1f} = v'_{1i},\quad v'_{2f} = v'_{2i}
\end{equation*}
即质心系中,两粒子动量大小相等,方向相反。弹性散射后,粒子的速率不变,运动方向偏转,偏转角度记作$\theta$,称为{\heiti 质心系散射角}。散射角由相互作用决定。由于在实验室坐标系中,速度满足
\begin{equation*}
	\mbf{v}_1 = \mbf{V} + \mbf{v}'_1,\quad \mbf{v}_2 = \mbf{V} + \mbf{v}'_2
\end{equation*}
因此,两个粒子的相对速度在质心系和实验室系中相同,记作
\begin{equation*}
	\mbf{v} = \mbf{v}_1 - \mbf{v}_2 = \mbf{v}'_1 - \mbf{v}'_2
\end{equation*}

\begin{figure}[htb]
\centering
\begin{asy}
	texpreamble("\usepackage{xeCJK}");
	texpreamble("\setCJKmainfont{SimSun}");
	usepackage("amsmath");
	import graph;
	import math;
	size(300);
	//质心坐标系中的散射
	pair O,O1;
	real theta,v1,v2,v;
	O = (0,0);
	v1 = 2;
	v2 = 1;
	theta = 40;
	draw(Label("$\boldsymbol{v}'_{1i}$",EndPoint,black),O--v1*dir(0),red,Arrow);
	draw(Label("$\boldsymbol{v}'_{1f}$",EndPoint,black),O--v1*dir(theta),red,Arrow);
	draw(Label("$\boldsymbol{v}'_{2i}$",EndPoint,black),O--v2*dir(180),blue,Arrow);
	draw(Label("$\boldsymbol{v}'_{2f}$",EndPoint,black),O--v2*dir(180+theta),blue,Arrow);
	draw(arc(O,v1,0,theta),dashed);
	draw(arc(O,v2,180,180+theta),dashed);
	label("$\theta$",O,5*dir(theta/2));
	picture tmp;
	O1 = (3.3,-0.3);
	v = 2.5;
	draw(tmp,Label("$\boldsymbol{v}'_i$",EndPoint,black),O--v*dir(0),Arrow);
	draw(tmp,Label("$\boldsymbol{v}'_f$",EndPoint,black),O--v*dir(theta),Arrow);
	draw(tmp,arc(O,v,0,theta),dashed);
	label(tmp,"$\theta$",O,5*dir(theta/2));
	add(shift(O1)*tmp);
	//draw((0,0)--(6.7,0),invisible);
\end{asy}
\caption{质心坐标系中的散射}
\label{质心坐标系中的散射}
\end{figure}

下面考虑靶粒子初始时刻静止的情形,即$\mbf{v}_{2i} = \mbf{0}$,此时$\mbf{v}'_{2i} = -\mbf{V}$,因此有
\begin{equation*}
	m_1 \mbf{v}'_{1i} = -m_2 \mbf{v}'_{2i} = m_2 \mbf{V}
\end{equation*}

\begin{figure}[htb]
\centering
\begin{asy}
	texpreamble("\usepackage{xeCJK}");
	texpreamble("\setCJKmainfont{SimSun}");
	usepackage("amsmath");
	import graph;
	import math;
	size(300);
	//静止靶粒子情形下的速度
	pair O;
	real V,v1,theta;
	O = (0,0);
	V = 1;
	v1 = 2.5;
	theta = 35;
	draw(Label("$\boldsymbol{v}'_{1f}$",Relative(0.7),Relative(E),black),O--v1*dir(theta),red,Arrow);
	draw(Label("$\boldsymbol{v}_{1f}$",Relative(0.7),Relative(W),black),V*dir(180)--v1*dir(theta),red,Arrow);
	draw(V*dir(180)--O,Arrow);
	label("$\boldsymbol{V}$",O,N);
	draw(Label("$\boldsymbol{v}'_{2f}$",Relative(0.7),Relative(W),black),O--V*dir(180+theta),blue,Arrow);
	draw(Label("$\boldsymbol{v}_{2f}$",Relative(0.7),Relative(E),black),V*dir(180)--V*dir(180+theta),blue,Arrow);
	draw(Label("$\boldsymbol{v}'_{1i}$",EndPoint,black),O--v1*dir(0),red,Arrow);
	draw(v1*dir(theta)--((v1*dir(theta)).x,0),dashed);
	draw(Label("$\theta$",MidPoint,Relative(E)),arc(O,0.3,0,theta),Arrow);
	draw(Label("$\mathnormal{\Theta}_1$",MidPoint,Relative(E)),arc(V*dir(180),0.3,0,degrees(v1*dir(theta)-V*dir(180))),Arrow);
	draw(Label("$\mathnormal{\Theta}_2$",MidPoint,Relative(W)),arc(V*dir(180),0.2,360,degrees(V*dir(180+theta)-V*dir(180))),Arrow);
	//draw(O--(3,0),invisible);
\end{asy}
\caption{静止靶粒子情形下的速度}
\label{静止靶粒子情形下的速度}
\end{figure}

速度合成如图\ref{静止靶粒子情形下的速度}所示,其中$\mathnormal{\Theta}_1$称为实验室系入射粒子{\heiti 散射角},$\mathnormal{\Theta}_2$称为实验室系靶粒子{\heiti 反冲角}。它们与质心系散射角的关系如下
\begin{equation}
	\begin{cases}
		\displaystyle \tan \mathnormal{\Theta}_1 = \frac{v'_{1i}\sin \theta}{V+v'_{1i} \cos \theta} = \frac{\sin \theta}{\frac{m_1}{m_2} + \cos \theta} \\[1.5ex]
		\displaystyle \mathnormal{\Theta}_2 = \frac{\pi - \theta}{2}
	\end{cases}
	\label{实验室系散射角和反冲角}
\end{equation}

\subsection{质心系散射角}

质心系散射角与碰撞前后粒子的相对速度偏转角相等。相对速度偏转角由入射粒子相对靶粒子的运动决定。

\begin{figure}[htb]
\centering
\begin{asy}
	texpreamble("\usepackage{xeCJK}");
	texpreamble("\setCJKmainfont{SimSun}");
	usepackage("amsmath");
	import graph;
	import math;
	size(350);
	//弹性散射的几何参数
	real p,e,theta0,b,l;
	picture tmp;
	real fr(real theta){
		return p/(e*cos(theta)-1);
	}
	pair xy(real theta){
		return (fr(theta)*cos(theta),fr(theta)*sin(theta));
	}
	p = 1;
	e = 2;
	theta0 = acos(1/e);
	b = p/(e*sqrt(e*e-1));
	l = 10;
	draw(tmp,rotate(180-theta0/pi*180)*graph(xy,-0.32*pi,0.32*pi),red+linewidth(1bp));
	draw(tmp,rotate(180-theta0/pi*180)*((p/(e*e-1),0)--(p/(e*e-1),0)+l*dir(theta0/pi*180)),dashed);
	draw(tmp,rotate(180-theta0/pi*180)*((p/(e*e-1),0)--(p/(e*e-1),0)+l*dir(-theta0/pi*180)),dashed);
	draw(tmp,rotate(180-theta0/pi*180)*((0,0)--p/(e-1)*dir(0)),dashed);
	draw(tmp,rotate(180-theta0/pi*180)*(l*dir(theta0/pi*180)--(0,0)--l*dir(-theta0/pi*180)),dashed);
	dot(tmp,(0,0));
	label("$\phi$",rotate(180-theta0/pi*180)*(p/(e*e-1),0),dir(180-theta0/2/pi*180));
	label("$\phi$",rotate(180-theta0/pi*180)*(p/(e*e-1),0),dir(180-theta0*3/2/pi*180));
	path clp;
	clp = (0,0)--l*dir(theta0/pi*180)--l*dir(theta0/pi*180)+2*b*dir(theta0/pi*180-90)--l*dir(0)--l*dir(-theta0/pi*180)+2*b*dir(-theta0/pi*180+90)--l*dir(-theta0/pi*180)--cycle;
	clp = rotate(180-theta0/pi*180)*shift((-0.1,0))*clp;
	clip(tmp,clp);
	draw(tmp,(0,0)--(0.5*l,0),dashed);
	label(tmp,"$\theta$",(0,0),2*dir((180-2*theta0/pi*180)/2));
	draw(tmp,Label("$r_{\mathrm{min}}$",BeginPoint,Relative(W)),rotate(180-theta0/pi*180)*(p/(e-1)*dir(0)+dir(0)--p/(e-1)*dir(0)),Arrow);
	draw(tmp,rotate(180-theta0/pi*180)*(-dir(0)--(0,0)),Arrow);
	draw(tmp,0.9*l*dir(180)-dir(90)--(0.9*l*dir(180)),Arrow);
	draw(tmp,Label("$b$",BeginPoint,Relative(W)),0.9*l*dir(180)+b*dir(90)+dir(90)--(0.9*l*dir(180)+b*dir(90)),Arrow);
	draw(tmp,Label("$\boldsymbol{v}_i$",MidPoint,Relative(E)),shift((0,b))*(l*dir(180)+dir(180)--l*dir(180)),Arrow);
	draw(tmp,Label("$\boldsymbol{v}_f$",MidPoint,Relative(E)),shift(b*dir(90+180-2*theta0/pi*180))*(l*dir(180-2*theta0/pi*180)--l*dir(180-2*theta0/pi*180)+dir(180-2*theta0/pi*180)),Arrow);
	label(tmp,"$O$",(0,0),SW);
	add(tmp);
	//draw((0,0)--(0.5*l+1.2,0),invisible);
\end{asy}
\caption{弹性散射的几何参数}
\label{弹性散射的几何参数}
\end{figure}

入射粒子初始速度与靶粒子之间的距离记作$b$,称为{\heiti 碰撞参数}或{\heiti 瞄准距离}。由于靶粒子初始是静止的,故系统的能量为
\begin{equation*}
	E = \frac12 \mu v^2
\end{equation*}
角动量为
\begin{equation*}
	l = \mu b v
\end{equation*}
因此角动量$l$可以用能量$E$和瞄准距离$b$来表示
\begin{equation*}
	l^2 = 2\mu b^2 E
\end{equation*}
根据式\eqref{两体运动等效为一维运动},可得
\begin{equation*}
	U(r_{\mathrm{min}}) = \frac{l^2}{2\mu r_{\mathrm{min}}^2} + V(r_{\mathrm{min}}) = E
\end{equation*}
即有
\begin{equation*}
	1- \frac{V(r_{\mathrm{min}})}{E} - \frac{b^2}{r_{\mathrm{min}}^2} = 0
\end{equation*}
据此可得$r_{\mathrm{min}}$。再由式\eqref{两体轨道方程(变量为r)}可得
\begin{equation}
	\phi = \int_{r_{\mathrm{min}}}^{+\infty} \frac{b\mathrm{d} r}{r^2 \sqrt{1-\dfrac{V(r)}{E} - \dfrac{b^2}{r^2}}}
\end{equation}
由此即可得质心系散射角
\begin{equation}
	\theta = \pi - 2\phi = \theta(b,E)
	\label{质心系散射角的通用关系式}
\end{equation}
具体关系决定于$V(r)$。

\subsection{散射截面}

散射实验一般是使得具有确定能量的粒子束均匀入射,然后对散射粒子的角分布进行测量以获得粒子之间作用势的信息。

\begin{figure}[htb]
\centering
\begin{asy}
	texpreamble("\usepackage{xeCJK}");
	texpreamble("\setCJKmainfont{SimSun}");
	usepackage("amsmath");
	import graph;
	import math;
	size(350);
	//散射截面
	pair O;
	real E0,p,e,theta0,theta1,theta2,b,l,delta,r1,r0,alpha,lb,lc,x0,bb;
	picture tmp;
	O = (0,0);
	real fr(real theta){
		return p/(e*cos(theta)-1);
	}
	pair xy(real theta){
		return (fr(theta)*cos(theta),fr(theta)*sin(theta));
	}
	r1 = 1.3;
	draw(tmp,scale(r1)*unitcircle);
	l = 7;
	lb = 6;
	lc = 5;
	delta = 0.03;
	alpha = 1/3;
	E0 = 1.5;
	b = 0.96;
	p = E0*b*b;
	e = sqrt(1+2*E0*E0*b*b);
	theta0 = acos(1/e);
	theta1 = theta0;
	draw(tmp,rotate(180-theta0/pi*180)*graph(xy,-theta0+delta*pi,theta0-delta*pi),red+linewidth(1bp));
	draw(tmp,O--l*dir(180-2*theta0/pi*180),red+dashed);
	draw(tmp,shift(lc*dir(180))*xscale(alpha)*arc(O,0.72*b,-90,90),dashed);
	draw(tmp,shift(lc*dir(180))*xscale(alpha)*arc(O,0.72*b,90,270));
	draw(tmp,Label("$b$",MidPoint,Relative(W)),lb*dir(180)--lb*dir(180)+0.72*b*dir(90),Arrow);
	x0 = (1-alpha**2)*r1*cos(pi-2*theta0);
	bb = sqrt(r1**2-x0**2/(1-alpha**2));
	draw(tmp,shift((x0,0))*xscale(alpha)*arc(O,bb,80,360-80));
	draw(tmp,shift((x0,0))*xscale(alpha)*arc(O,bb,-80,80),dashed);
	b = 1.2;
	p = E0*b*b;
	e = sqrt(1+2*E0*E0*b*b);
	theta0 = acos(1/e);
	theta2 = theta0;
	draw(tmp,rotate(180-theta0/pi*180)*graph(xy,-theta0+delta*pi,theta0-delta*pi),blue+linewidth(1bp));
	draw(tmp,O--l*dir(180-2*theta0/pi*180),blue+dashed);
	draw(tmp,shift(lc*dir(180))*xscale(alpha)*arc(O,0.75*b,-90,90),dashed);
	draw(tmp,shift(lc*dir(180))*xscale(alpha)*arc(O,0.75*b,90,270));
	draw(tmp,Label("$\mathrm{d} b$",BeginPoint,Relative(E)),lb*dir(180)+0.75*b*dir(90)+0.72*b*dir(90)--lb*dir(180)+0.75*b*dir(90),Arrow);
	x0 = (1-alpha**2)*r1*cos(pi-2*theta0);
	bb = sqrt(r1**2-x0**2/(1-alpha**2));
	draw(tmp,shift((x0,0))*xscale(alpha)*arc(O,bb,80,360-80));
	draw(tmp,shift((x0,0))*xscale(alpha)*arc(O,bb,-80,80),dashed);
	dot(tmp,O);
	label(tmp,"$O$",O,S);
	draw(tmp,l*dir(180)--0.5*l*dir(0));
	r0 = 1.9;
	draw(tmp,Label("$\theta$",MidPoint,Relative(E)),arc(O,r0,0,180-2*theta1/pi*180),Arrow);
	r0 = 2.4;
	draw(tmp,Label("$\mathrm{d}\theta$",MidPoint,Relative(E)),arc(O,r0,180-2*theta2/pi*180,180-2*theta1/pi*180),Arrow);
	path clp;
	clp = scale(l)*unitcircle;
	clip(tmp,clp);
	label(tmp,"$\mathrm{d} \omega = 2\pi \sin \theta\mathrm{d} \theta$",r1*dir(-47),SE);
	add(tmp);
\end{asy}
\caption{散射截面}
\label{散射截面}
\end{figure}

设入射粒子流密度为$I$,则入射粒子数按$b$的分布为
\begin{equation*}
	\mathrm{d} N = 2\pi I b \mathrm{d}b
\end{equation*}
则散射粒子按$\theta$的分布可以表示为
\begin{equation*}
	\mathrm{d} N = I \cdot 2\pi b(\theta)|b'(\theta)| \mathrm{d} \theta
\end{equation*}
对$\theta$的分布并不是十分方便,因此需要使用对立体角$\omega$的分布。在球体的情形下,立体角元$\mathrm{d} \omega$与$\mathrm{d} \theta$的关系为
\begin{equation*}
	\mathrm{d} \omega = 2\pi \sin \theta\mathrm{d} \theta
\end{equation*}
由此可有散射粒子按$\omega$的分布
\begin{equation*}
	\mathrm{d} N = I \frac{b(\theta)}{\sin \theta}|b'(\theta)| \mathrm{d} \omega
\end{equation*}
定义{\heiti 微分散射截面}为
\begin{equation}
	\mathrm{d} \sigma = \frac{\mathrm{d}N}{I} = 2\pi b \mathrm{d} b = \frac{b(\theta)}{\sin \theta} |b'(\theta)| \mathrm{d} \omega
	\label{微分散射截面}
\end{equation}
同样有{\heiti 单位立体角散射截面}为
\begin{equation}
	\sigma(\theta) = \frac{\mathrm{d} \sigma}{\mathrm{d} \omega} = \frac{b(\theta)}{\sin \theta}|b'(\theta)|
\end{equation}
根据式\eqref{质心系散射角的通用关系式}可知,上式中的$b = b(\theta,E)$决定于两体中心势的具体形式,与实验条件无关。

前面的讨论都基于质心系,在实验室系中,有
\begin{equation*}
	\mathrm{d} \mathnormal{\Sigma} = \frac{\mathrm{d}N}{I} = \mathnormal{\Sigma}_1(\mathnormal{\Theta}_1) \mathrm{d} \mathnormal{\Omega}_1, \quad \mathrm{d} \mathnormal{\Omega}_1 = 2\pi \sin \mathnormal{\Theta}_1 \mathrm{d} \mathnormal{\Theta}_1
\end{equation*}
所以有
\begin{equation*}
	\mathnormal{\Sigma}_1(\mathnormal{\Theta}_1) = \sigma(\theta) \frac{\mathrm{d} \omega}{\mathrm{d} \mathnormal{\Omega}_1} = \sigma(\theta) \frac{\mathrm{d} \cos \theta}{\mathrm{d} \cos \mathnormal{\Theta}_1}
\end{equation*}
其中$\mathnormal{\Theta}_1$满足
\begin{equation*}
	\tan \mathnormal{\Theta}_1 = \frac{\sin \theta}{\frac{m_1}{m_2} + \cos \theta}
\end{equation*}
在实验中,可实际测量的量为$\displaystyle \mathnormal{\Sigma}_1(\mathnormal{\Theta}_1)$。

{\heiti 总散射截面}可以计算为
\begin{equation}
	\sigma_t = \int_S \mathrm{d} \sigma = \int_0^\pi 2\pi \sigma(\theta) \sin \theta \mathrm{d} \theta = \int_0^{b_{\mathrm{max}}} 2\pi b\mathrm{d} b = \pi b_{\mathrm{max}}^2
\end{equation}
式中$b_{\mathrm{max}}$为散射得以发生的最大瞄准距离,即当$b>b_{\mathrm{max}}$时,入射粒子始终在相互作用力程之外,粒子将直线掠过。

\subsection{Coulomb势弹性散射}

Coulomb势可以表示为
\begin{equation*}
	V = \frac{\alpha}{r},\quad \alpha > 0
\end{equation*}
此时$r_{\mathrm{min}}$满足
\begin{equation*}
	1-  \frac{\alpha}{E r_{\mathrm{min}}} - \frac{b^2}{r_{\mathrm{min}}^2} = 0
\end{equation*}
由此可得
\begin{equation}
	\frac{1}{r_{\mathrm{min}}} = \frac{1}{b^2} \left[\sqrt{\left(\frac{\alpha}{2E}\right)^2+b^2} - \frac{\alpha}{2E}\right]
\end{equation}
即有
\begin{equation*}
	\phi = \int_{r_{\mathrm{min}}}^{+\infty} \frac{\dfrac{b}{r^2} \mathrm{d} r}{\sqrt{1-\dfrac{\alpha}{Er} - \dfrac{b^2}{r^2}}} = \left.\arccos \frac{\dfrac{b^2}{r}+\dfrac{\alpha}{2E}}{\sqrt{\left(\dfrac{\alpha}{2E}\right)^2+ b^2}}\right|_{r_{\mathrm{min}}}^{+\infty} = \arccos \frac{\dfrac{\alpha}{2E}}{\sqrt{\left(\dfrac{\alpha}{2E}\right)^2+b^2}}
\end{equation*}
所以有
\begin{equation*}
	\tan \phi = \frac{2Eb}{\alpha}
\end{equation*}
考虑到$\theta = \pi-2\phi$,可有
\begin{equation}
	b = \frac{\alpha}{2E}\cot \frac{\theta}{2}
\end{equation}
由此可得Coulomb势下的微分散射截面
\begin{equation}
	\sigma(\theta) = \frac{b(\theta)}{\sin \theta} |b'(\theta)| = \frac{\left(\dfrac{\alpha}{4E}\right)^2}{\sin^4 \dfrac{\theta}{2}}
	\label{Rutherford公式}
\end{equation}
式\eqref{Rutherford公式}称为{\heiti Rutherford公式}。总散射截面为
\begin{equation*}
	\sigma_t = \int_0^\pi 2\pi \sigma(\theta) \sin \theta \mathrm{d} \theta \to +\infty
\end{equation*}
此式说明Coulomb力为长程力。

\begin{example}[刚球势散射]
\label{刚球势散射}
求粒子在刚球势
\begin{equation*}
	V(r) = \begin{cases} +\infty, & r < a \\ 0, & r \geqslant a \end{cases}
\end{equation*}
中的散射截面。
\end{example}
\begin{solution}
\begin{figure}[htb]
\centering
\begin{asy}
	texpreamble("\usepackage{xeCJK}");
	texpreamble("\setCJKmainfont{SimSun}");
	usepackage("amsmath");
	import graph;
	import math;
	size(300);
	//刚球势散射
	pair O;
	real theta,phi,r,l,r0;
	O = (0,0);
	r = 1;
	l = 1.8;
	theta = 55;
	phi = (180+theta)/2;
	draw(shift(O)*scale(r)*unitcircle);
	draw(r*dir(phi)+0.8*l*dir(180)--r*dir(phi),red,Arrow);
	draw(r*dir(phi)--r*dir(phi)+0.8*l*dir(theta),red,Arrow);
	draw(O--2*r*dir(phi),dashed);
	draw(O--2*r*dir(theta),dashed);
	draw(l*dir(180)--l*dir(0),dashed);
	draw(Label("$b$",MidPoint,Relative(E)),r*dir(phi)--((r*dir(phi)).x,0),dashed);
	r0 = 0.2;
	draw(Label("$\theta$",MidPoint,Relative(E)),arc(O,r0,0,theta),Arrow);
	draw(Label("$\phi$",MidPoint,Relative(E)),arc(O,r0,phi,180),Arrow);
\end{asy}
\caption{例\theexample}
\label{第四章例3}
\end{figure}
对此问题,可直接根据几何关系求出瞄准距离$b$与散射角$\theta$之间的关系为
\begin{equation*}
	b = a\sin \phi = a \cos \frac{\theta}{2}
\end{equation*}
由此可得微分散射截面
\begin{equation*}
	\mathrm{d} \sigma = \mathrm{d} (\pi b^2) = \frac{\pi a^2}{2} \sin \theta \mathrm{d} \theta = \frac{a^2}{4} \mathrm{d} \omega
\end{equation*}
所以有
\begin{equation*}
	\sigma(\theta) = \frac{\mathrm{d} \sigma}{\mathrm{d} \omega} = \frac{a^2}{4}
\end{equation*}
总散射截面
\begin{equation*}
	\sigma_t = \frac{a^2}{4} \int \mathrm{d} \omega = \frac{a^2}{4} 4\pi = \pi a^2
\end{equation*}
\end{solution}

\begin{example}[阶跃中心势散射]
求粒子在势场
\begin{equation*}
	V = \begin{cases} V_0,& r<a \\ 0,& r\geqslant a \end{cases}
\end{equation*}
\end{example}
\begin{solution}
首先考虑$V_0>0$的情形。

当$E<V_0$时,粒子无法进入$r=a$的内部,结果等同于例\ref{刚球势散射}的刚球势散射。

当$E>V_0$时,根据能量守恒,可有粒子在$r=a$外部和内部满足如下关系式
\begin{equation*}
	\frac12 \mu v^2 = \frac12 \mu v'^2 + V_0 = E
\end{equation*}
由此可得
\begin{equation*}
	\frac{v'}{v} = \sqrt{\frac{E-V_0}{E}} =: n
\end{equation*}
其中$n$可称为折射率。再根据角动量守恒,可有粒子在$r=a$外部和内部满足如下关系式\footnote{此处考虑到粒子在常数势场中的运动轨迹必然为直线。}
\begin{equation*}
	\mu vb = \mu v'r_{\mathrm{min}}
\end{equation*}
由此可得
\begin{equation*}
	\frac{b}{r_{\mathrm{min}}} = \frac{\sin \alpha}{\sin \beta} = n
\end{equation*}
\begin{figure}[htb]
\centering
\begin{asy}
	texpreamble("\usepackage{xeCJK}");
	texpreamble("\setCJKmainfont{SimSun}");
	usepackage("amsmath");
	import graph;
	import math;
	size(300);
	//阶跃中心势散射
	pair O;
	real r,alpha,beta,theta,n,l,r0;
	r = 1;
	alpha = 30;
	beta = 55;
	n = 1.8;
	l = 1;
	theta = 2*beta-2*alpha;
	draw(scale(r)*unitcircle);
	draw(O--n*r*dir(180-alpha),dashed);
	draw(O--n*r*dir(2*beta-alpha),dashed);
	draw(r*dir(180-alpha)+l*dir(180)--r*dir(180-alpha),red,Arrow);
	draw(r*dir(180-alpha)--r*dir(2*beta-alpha),red,Arrow);
	draw(r*dir(2*beta-alpha)--r*dir(2*beta-alpha)+l*dir(theta),red,Arrow);
	draw(O--n*r*dir(theta),dashed);
	draw(n*r*dir(180)--n*r*dir(0),dashed);
	draw(Label("$b$",MidPoint,E),r*dir(180-alpha)--((r*dir(180-alpha)).x,0),Arrows);
	draw(Label("$r_{\mathrm{min}}$",Relative(0.6),Relative(E)),O--interp(r*dir(180-alpha),r*dir(2*beta-alpha),0.5),Arrows);
	r0 = 0.2;
	draw(Label("$\alpha$",MidPoint,Relative(W)),arc(r*dir(180-alpha),r0,180,180-alpha),Arrow);
	draw(Label("$\beta$",MidPoint,Relative(W)),arc(r*dir(180-alpha),r0,beta-alpha,-alpha),Arrow);
	draw(Label("$\theta$",MidPoint,Relative(E)),arc(O,r0,0,theta),Arrow);
\end{asy}
\caption{阶跃中心势散射}
\label{阶跃中心势散射}
\end{figure}
所以
\begin{equation*}
	\sin \alpha = \frac{b}{a},\quad \sin \beta = \frac{b}{na}
\end{equation*}
根据几何关系,可有$\theta=2(\beta-\alpha)$,所以
\begin{equation*}
	\cos \frac{\theta}{2} = \cos(\beta-\alpha) = \cos\beta \cos\alpha + \sin \beta \sin \alpha
\end{equation*}
所以有
\begin{equation*}
	\left(\cos \frac{\theta}{2}-\frac{b^2}{na^2}\right)^2 = \left(1-\frac{b^2}{a^2}\right)\left(1-\frac{b^2}{na^2}\right)
\end{equation*}
由此解得
\begin{equation*}
	b^2 = \frac{n^2a^2\sin^2 \dfrac{\theta}{2}}{1+n^2-2n\cos \dfrac{\theta}{2}}
\end{equation*}
根据
\begin{equation*}
	\mathrm{d} \sigma = \mathrm{d} (\pi b^2)% = \sigma(\theta) \mathrm{d} \omega
\end{equation*}
即可得到微分散射截面。
\end{solution}


%第五章——多自由度微振动与阻尼
\chapter{多自由度微振动与阻尼}

体系在稳定平衡位形附近的小幅振动即称为{\heiti 微振动}。

\section{微振动近似与求解}

\begin{example}[单摆]
对于单摆,自由度$s=1$。取摆角$\theta$为广义坐标,体系的动能为
\begin{equation*}
	T = \frac12 ml^2 \dot{\theta}^2
\end{equation*}
势能为
\begin{equation*}
	V = mgl(1-\cos \theta)
\end{equation*}

\begin{figure}[htb]
\centering
\begin{minipage}[t]{0.45\textwidth}
\begin{asy}
	size(190);
	//单摆
	pair O;
	real theta,l,r;
	O = (0,0);
	l = 1;
	theta = 30;
	draw(O--0.6*l*dir(-90),dashed);
	draw(Label("$l$",MidPoint,Relative(W)),O--l*dir(theta-90));
	r = 0.03;
	fill(shift(l*dir(theta-90))*scale(r)*unitcircle,black);
	label("$m$",l*dir(theta-90),2*W);
	r = 0.1;
	draw(Label("$\theta$",MidPoint,Relative(E)),arc(O,r,-90,theta-90),Arrow);
	draw(0.4*l*dir(180)--0.4*l*dir(0));
	int imax;
	pair dash,pace,P;
	imax = 20;
	dash = 0.04*l*dir(60);
	pace = (0.8*l/imax,0);
	P = 0.4*l*dir(180)+0.2*pace;
	for(int i=1;i<=imax;i=i+1){
		draw(P--P+dash);
		P = P+pace;
	}
	draw(O--(0,0.01),invisible);
\end{asy}
\caption{单摆}
\label{第五章单摆示意}
\end{minipage}
\hspace{0.7cm}
\begin{minipage}[t]{0.45\textwidth}
\begin{asy}
	size(190);
	//单摆
	pair O;
	real mgl;
	picture tmp;
	real V1(real theta){
		return 0.5*mgl*theta**2;
	}
	real V2(real theta){
		return mgl*(1-cos(theta));
	}
	O = (0,0);
	mgl = 3;
	draw(tmp,graph(V1,-pi,pi),dashed);
	draw(tmp,graph(V2,-pi,pi),red);
	path clp;
	clp = box((-pi,0),(pi,2*mgl));
	clip(tmp,clp);
	add(tmp);
	draw(Label("$\theta$",EndPoint),(-3.5,0)--(3.5,0));
	draw(Label("$V$",EndPoint),(0,0)--(0,2.2*mgl));
	draw(Label("$-\pi$",BeginPoint),(-pi,0)--(-pi,0.1));
	draw(Label("$\pi$",BeginPoint),(pi,0)--(pi,0.1));
	draw(Label("$mgl$",EndPoint),(-0.1,mgl)--(0.1,mgl));
	draw((-0.1,2*mgl)--(0.1,2*mgl));
	//draw(O--(4.3,0),invisible);
\end{asy}
\caption{单摆的势能曲线}
\label{第五章单摆的势能曲线示意}
\end{minipage}
\end{figure}

作线性近似可以将势能在平衡位形附近展开为
\begin{equation*}
	V = \frac12 mgl\theta^2 + o(\theta^2)
\end{equation*}
由此在线性近似下体系的Lagrange函数
\begin{equation*}
	L = \frac12 ml^2 \dot{\theta}^2 - \frac12 mgl\theta^2 
\end{equation*}
是广义坐标$\theta$和广义速度$\dot{\theta}$的二次型,进而体系的运动方程
\begin{equation*}
	ml^2\ddot{\theta} + mgl\theta = 0
\end{equation*}
是一个线性常微分方程,解为
\begin{equation*}
	\theta = \theta_m \cos \left(\sqrt{\frac{g}{l}} t + \phi_0\right)
\end{equation*}
即,单摆在小角度下可近似为谐振子。
\end{example}

\begin{example}[弹性耦合摆]
对于弹性耦合摆,此时自由度$s=2$。
\begin{figure}[htb]
\centering
\begin{asy}
	size(300);
	//弹性耦合摆
	pair O1,O2;
	real theta1,theta2,d,l,r;
	O1 = (0,0);
	l = 1;
	d = 0.8;
	O2 = (d,0);
	theta1 = 25;
	theta2 = 35;
	draw(O1--O1+0.6*l*dir(-90),dashed);
	draw(O2--O2+0.6*l*dir(-90),dashed);
	draw(Label("$l$",MidPoint,Relative(W)),O1--O1+l*dir(theta1-90));
	draw(Label("$l$",MidPoint,Relative(W)),O2--O2+l*dir(theta2-90));
	r = 0.03;
	fill(shift(O1+l*dir(theta1-90))*scale(r)*unitcircle,black);
	label("$m$",O1+l*dir(theta1-90),2*W);
	fill(shift(O2+l*dir(theta2-90))*scale(r)*unitcircle,black);
	label("$m$",O2+l*dir(theta2-90),2*E);
	r = 0.1;
	draw(Label("$\theta_1$",MidPoint,Relative(E)),arc(O1,r,-90,theta1-90),Arrow);
	draw(Label("$\theta_2$",MidPoint,Relative(E)),arc(O2,r,-90,theta2-90),Arrow);
	draw(Label("$d$",MidPoint,Relative(E)),O1+0.3*l*dir(180)--O2+0.3*l*dir(0),linewidth(1bp));
	draw(shift(O1+l*dir(theta1-90))*rotate(degrees(O2+l*dir(theta2-90)-O1-l*dir(theta1-90)))*scale(abs(O2+l*dir(theta2-90)-O1-l*dir(theta1-90))/10/pi)*graph(sin,0,10*pi));
	label("$k$",interp(O1+l*dir(theta1-90),O2+l*dir(theta2-90),0.5),2*S);
	//draw(O1--O2--O2+l*dir(theta2-90)+(0.2,0),invisible);
\end{asy}
\caption{弹性耦合摆}
\label{第五章弹性耦合摆}
\end{figure}

选两个摆角$\theta_1,\theta_2$为广义坐标,则体系的动能为
\begin{equation*}
	T = \frac12 ml^2 \left(\dot{\theta}_1^2+\dot{\theta}_2^2\right)
\end{equation*}
势能为
\begin{equation*}
	V = mgl(1-\cos \theta_1+ 1-\cos \theta_2) + \frac{k}{2} \left[\sqrt{(d+l\sin \theta_2-l\sin \theta_1)^2+(l\cos \theta_2-l\cos \theta_1)^2} - d\right]
\end{equation*}
作线性近似可以将势能在平衡位形处展开为
\begin{align*}
	V & = \frac12 mgl(\theta_1^2+\theta_2^2) + \frac12 kl^2 (\theta_2-\theta_1)^2 + o(\theta_1^2+\theta_2^2)
\end{align*}
取新的广义坐标$x_1 = l\theta_1,x_2 = l\theta_2$,可有体系在线性近似下的Lagrange函数为
\begin{equation*}
	L = \frac12 m(\dot{x}_1^2+\dot{x}_2^2) - \frac{mg}{2l} (x_1^2+x_2^2) - \frac12 k(x_1^2-2x_1x_2+x_2^2)
\end{equation*}
进而体系的运动方程为
\begin{equation*}
	\begin{cases}
		\displaystyle m\ddot{x}_1 + \left(k +\frac{mg}{l}\right) x_1 - kx_2 = 0 \\[1.5ex]
		\displaystyle m\ddot{x}_2 + \left(k +\frac{mg}{l}\right) x_2 - kx_1 = 0
	\end{cases}
\end{equation*}
定义$\mbf{x} = \begin{pmatrix} x_1 \\ x_2 \end{pmatrix},\ddot{\mbf{x}} = \begin{pmatrix} \ddot{x}_1 \\ \ddot{x}_2 \end{pmatrix}$,以及
\begin{equation*}
	\mbf{M} = \begin{pmatrix} m & 0 \\ 0 & m \end{pmatrix},\quad \mbf{K} = \begin{pmatrix} \dfrac{mg}{l}+k & -k \\[1.5ex] -k & \dfrac{mg}{l} + k \end{pmatrix}
\end{equation*}
其中矩阵$\mbf{M}$称为{\heiti 惯性矩阵},$\mbf{K}$称为{\heiti 弹性矩阵},由此体系的运动方程可以简单的表示为
\begin{equation*}
	\mbf{M} \ddot{\mbf{x}} + \mbf{K} \mbf{x} = \mbf{0}
\end{equation*}
考虑取特解
\begin{equation*}
	\mbf{x} = \mbf{a} \cos (\omega t+\phi)
\end{equation*}
则可有线性方程组
\begin{equation*}
	(\mbf{K} - \omega^2 \mbf{M}) \mbf{a} = \mbf{0}
\end{equation*}
或者
\begin{equation*}
	\begin{pmatrix} \dfrac{mg}{l}+k-m\omega^2 & -k \\[1.5ex] -k & \dfrac{mg}{l}+k-m\omega^2 \end{pmatrix} \begin{pmatrix} a_1 \\ a_2 \end{pmatrix} = \begin{pmatrix} 0 \\ 0 \end{pmatrix}
\end{equation*}
此为{\heiti 广义本征值问题}。存在振动解$\mbf{a} \neq \mbf{0}$的条件为
\begin{equation*}
	\begin{vmatrix} \dfrac{mg}{l} + k - m\omega^2 & -k \\[1.5ex] -k & \dfrac{mg}{l} + k - m\omega^2 \end{vmatrix} = 0
\end{equation*}
由此可以解得广义本征值
\begin{equation*}
	\omega_1 = \sqrt{\dfrac{g}{l}},\quad \omega_2 = \sqrt{\frac{g}{l} + \frac{2k}{m}}
\end{equation*}
以及其对应的广义本征矢
\begin{equation*}
	\mbf{a}_1 = \begin{pmatrix} 1 \\ 1 \end{pmatrix},\quad \mbf{a}_2 = \begin{pmatrix} 1 \\ -1 \end{pmatrix}
\end{equation*}
方程的通解即为以上两个线性无关的特解的线性组合
\begin{equation*}
	\begin{pmatrix} x_1 \\ x_2 \end{pmatrix} = \begin{pmatrix} 1 \\ 1 \end{pmatrix} C_1 \cos \left(\sqrt{\frac{g}{l}}t + \phi_1 \right) + \begin{pmatrix} 1 \\ -1 \end{pmatrix} C_1 \cos \left(\sqrt{\frac{g}{l} + \frac{2k}{m}}t + \phi_2 \right)
\end{equation*}
通解中的四个任意常数$C_1,C_2,\phi_1,\phi_2$可以由$x_1,x_2$以及$\dot{x}_1,\dot{x}_2$的初始值来决定。
\end{example}

现在在上述例子中考虑变量代换
\begin{equation*}
	\begin{pmatrix} x_1 \\ x_2 \end{pmatrix} = \begin{pmatrix} 1 & 1 \\ 1 & -1 \end{pmatrix} \begin{pmatrix} \xi_1 \\ \xi_2 \end{pmatrix}
\end{equation*}
可有
\begin{equation*}
	\begin{pmatrix} \xi_1 \\ \xi_2 \end{pmatrix} = \begin{pmatrix} C_1 \cos (\omega_1 t + \phi_1) \\ C_2 \cos (\omega_2 t + \phi_2) \end{pmatrix}
\end{equation*}
此处得到的两个相互独立的简谐振动称为{\heiti 简正振动},其频率为{\heiti 本征频率},简正振动对应的坐标$\xi_1,\xi_2$称为{\heiti 简正坐标}。在简正坐标下,体系的Lagrange函数$L = \dfrac12 \dot{\mbf{x}}^{\mathrm{T}} \mbf{M} \dot{\mbf{x}} - \dfrac12 \mbf{x}^{\mathrm{T}} \mbf{K} \mbf{x}$可以转化为
\begin{align*}
	L & = \frac12 \dot{\mbf{\xi}}^{\mathrm{T}} \mbf{S}^{\mathrm{T}} \mbf{M} \mbf{S} \dot{\mbf{\xi}} - \frac12 \mbf{\xi}^{\mathrm{T}} \mbf{S}^{\mathrm{T}} \mbf{K} \mbf{S} \mbf{x} = \frac12 \dot{\mbf{\xi}}^{\mathrm{T}} \mbf{M}' \dot{\mbf{\xi}} - \frac12 \mbf{\xi}^{\mathrm{T}} \mbf{K}' \mbf{x}
\end{align*}
其中
\begin{align*}
	\mbf{M}' & = \mbf{S}^{\mathrm{T}} \mbf{M} \mbf{S} = \begin{pmatrix} 1 & 1 \\ 1 & -1 \end{pmatrix}^{\mathrm{T}} \begin{pmatrix} m & 0 \\ 0 & m \end{pmatrix} \begin{pmatrix} 1 & 1 \\ 1 & -1 \end{pmatrix} = \begin{pmatrix} 2m & 0 \\ 0 & 2m \end{pmatrix} \\
	\mbf{K}' & = \mbf{S}^{\mathrm{T}} \mbf{K} \mbf{S} = \begin{pmatrix} 1 & 1 \\ 1 & -1 \end{pmatrix}^{\mathrm{T}} \begin{pmatrix} \dfrac{mg}{l}+k & -k \\[1.5ex] -k & \dfrac{mg}{l}+k \end{pmatrix} \begin{pmatrix} 1 & 1 \\ 1 & -1 \end{pmatrix} = \begin{pmatrix} \dfrac{2mg}{l} & 0 \\[1.5ex] 0 & \dfrac{2mg}{l}+4k \end{pmatrix}
\end{align*}
都为对角矩阵。因此,引入简正坐标,可以使$\mbf{M}$和$\mbf{K}$同时对角化,不同自由度的振动解耦,即简正振动模式。

\section{多自由度微振动体系}

\subsection{微振动近似}

考虑保守体系,其广义坐标取为$q_\alpha\,(\alpha=1,2,\cdots,s)$,不失一般性,规定其稳定平衡位形处有$q_{\alpha 0} = 0,V_0 = 0$,由此平衡条件可以表示为
\begin{equation*}
	\left(\frac{\pl V}{\pl q_\alpha}\right)_0 = 0,\quad \alpha = 1,2,\cdots,s
\end{equation*}
在平衡位形附近,可对势能$V$作Taylor展开,即
\begin{align*}
	V & = V_0 + \sum_{\alpha=1}^s \left(\dfrac{\pl V}{\pl q_\alpha}\right)_0 q_\alpha + \frac12 \sum_{\alpha,\beta=1}^s \left(\frac{\pl^2 V}{\pl q_\alpha \pl q_\beta}\right)_0 q_\alpha q_\beta + o(|\mbf{q}|^2) \\
	& = \frac12 \sum_{\alpha,\beta=1}^s \left(\frac{\pl^2 V}{\pl q_\alpha \pl q_\beta}\right)_0 q_\alpha q_\beta
\end{align*}
令$k_{\alpha\beta} = k_{\beta\alpha} = \left(\dfrac{\pl^2 V}{\pl q_\alpha \pl q_\beta}\right)_0$,则有
\begin{equation*}
	\mbf{K} = \begin{pmatrix} k_{11} & \cdots & k_{1s} \\ \vdots & & \vdots \\ k_{s1} & \cdots & k_{ss} \end{pmatrix}
\end{equation*}
因此势能可以近似为
\begin{equation}
	V = \frac12 \sum_{\alpha,\beta=1}^s k_{\alpha\beta}q_\alpha q_\beta = \frac12 \mbf{q}^{\mathrm{T}} \mbf{K} \mbf{q}
\end{equation}
考虑到$\mbf{q}_0$为稳定平衡位形,因此$V$在$\mbf{q}_0$处取极小值,即矩阵$\mbf{K}$是正定的\footnote{如非严格极小值,可以允许$\mbf{K}$为半正定。}。广义坐标变换可以表示为
\begin{equation*}
	\mbf{r}_i = \mbf{r}_i(\mbf{q}),\quad i=1,2,\cdots,n
\end{equation*}
由此可将速度展开至一阶,即
\begin{equation*}
	\dot{\mbf{r}}_i = \sum_{\alpha=1}^s \frac{\pl \mbf{r}_i}{\pl q_\alpha} \dot{q}_\alpha = \sum_{\alpha=1}^s \left(\frac{\pl \mbf{r}_i}{\pl q_\alpha}\right)_0 \dot{q}_\alpha + \cdots
\end{equation*}
由此,动能可表示为
\begin{equation*}
	T = \frac12 \sum_{i=1}^n m_i \dot{\mbf{r}}_i \cdot \dot{\mbf{r}}_i = \frac12 \sum_{\alpha,\beta=1}^s \left[\sum_{i=1}^n m_i \left(\frac{\pl \mbf{r}_i}{\pl q_\alpha}\right)_0 \cdot \left(\frac{\pl \mbf{r}_i}{\pl q_\beta}\right)_0 \right] \dot{q}_\alpha \dot{q}_\beta + \cdots
\end{equation*}
令$\displaystyle m_{\alpha\beta} = m_{\beta\alpha} = \sum_{i=1}^n m_i \left(\frac{\pl \mbf{r}_i}{\pl q_\alpha}\right)_0 \cdot \left(\frac{\pl \mbf{r}_i}{\pl q_\beta}\right)_0$,则可得惯性矩阵
\begin{equation*}
	\mbf{M} = \begin{pmatrix} m_{11} & \cdots & m_{1s} \\ \vdots & & \vdots \\ m_{s1} & \cdots & m_{ss} \end{pmatrix}
\end{equation*}
由此,动能被表示为广义速度的二次型
\begin{equation}
	T = \frac12 \sum_{\alpha,\beta=1}^s m_{\alpha\beta} \dot{q}_\alpha \dot{q}_\beta = \frac12 \dot{\mbf{q}}^{\mathrm{T}} \mbf{M} \dot{\mbf{q}}
\end{equation}
动能必然是非负的,而且只有$\dot{\mbf{q}}=\mbf{0}$时为零,故矩阵$\mbf{M}$正定。因此体系的Lagrange函数为
\begin{equation*}
	L = \frac12 \dot{\mbf{q}}^{\mathrm{T}} \mbf{M} \dot{\mbf{q}} - \frac12 \mbf{q}^{\mathrm{T}} \mbf{K} \mbf{q}
\end{equation*}
由此可得
\begin{align*}
	\frac{\pl L}{\pl \dot{q}_\alpha} & = \frac12 \sum_{\beta,\gamma=1}^s (m_{\beta\gamma}\delta_{\alpha\beta} \dot{q}_\gamma + m_{\beta\gamma}\dot{q}_\beta \delta_{\alpha\gamma}) = \frac12 \left(\sum_{\gamma=1}^s m_{\alpha\gamma} \dot{q}_\gamma + \sum_{\beta=1}^s m_{\beta\alpha} \dot{q}_\beta\right) \\
	& = \frac12 \sum_{\beta=1}^s (m_{\alpha\beta} + m_{\beta\alpha}) \dot{q}_\beta = \sum_{\beta=1}^s m_{\alpha\beta} \dot{q}_\beta \\
	\frac{\pl L}{\pl q_\alpha} & = -\sum_{\beta=1}^s k_{\alpha\beta} q_\beta
\end{align*}
由此得到体系的运动方程为
\begin{equation*}
	\sum_{\beta=1}^s (m_{\alpha\beta} \ddot{q}_\beta + k_{\alpha\beta} q_\beta) = 0,\quad \alpha=1,2,\cdots,s
\end{equation*}
用矩阵可以表示为
\begin{equation}
	\mbf{M} \ddot{\mbf{q}} + \mbf{K} \mbf{q} = \mbf{0}
	\label{多自由度微振动体系的运动方程}
\end{equation}
即,Lagrange函数在稳定平衡位形处取二阶近似,可得线性振动方程。

\subsection{运动求解}

作试解
\begin{equation*}
	\mbf{q} = \mbf{a} \cos (\omega t+\phi)
\end{equation*}
则体系的运动方程\eqref{多自由度微振动体系的运动方程}可化为
\begin{equation}
	(\mbf{K} - \omega^2 \mbf{M})\mbf{a} = \mbf{0}
	\label{多自由度微振动本征方程}
\end{equation}
式\eqref{多自由度微振动本征方程}称为体系的{\heiti 本征方程},体系有振动解要求本征方程\eqref{多自由度微振动本征方程}有非平凡解,即要求
\begin{equation}
	\det (\mbf{K} - \omega^2 \mbf{M}) = 0
	\label{久期方程或频率方程}
\end{equation}
式\eqref{久期方程或频率方程}称为体系的{\heiti 久期方程}或{\heiti 频率方程},它是$\omega^2$的$s$次方程。因为矩阵$\mbf{K}$至少是半正定的,矩阵$\mbf{M}$是正定的,所以方程\eqref{久期方程或频率方程}必然有$s$个非负的解,即有
\begin{equation*}
	\omega^2 = \omega_\mu^2,\quad \mu = 1,2,\cdots,s
\end{equation*}
然后即可通过线性方程组
\begin{equation*}
	(\mbf{K} - \omega_\mu^2 \mbf{M}) \mbf{a}_\mu = \mbf{0},\quad \mu = 1,2,\cdots,s
\end{equation*}
解出每个本征频率对应的本征矢量$\mbf{a}_\mu$。体系运动的通解即为上述$s$个特解的线性组合
\begin{equation}
	\mbf{q} = \sum_{\mu=1}^s \mbf{a}_\mu C_\mu \cos (\omega_\mu t + \phi_\mu)
	\label{多自由度微振动通解}
\end{equation}
此处$C_\mu$和$\phi_\mu$是积分常数,可以通过$q_\alpha$和$\dot{q}_\alpha$共$2s$个初始值确定\footnote{此处通解公式中没有考虑本征频率为零的情况,因为当本征频率为零时,其对应的不是振动解。}。

\subsection{简正坐标与简正振动}

将体系运动的通解\eqref{多自由度微振动通解}表示为
\begin{equation*}
	\mbf{q} = \begin{pmatrix} \mbf{a}_1 & \cdots & \mbf{a}_s \end{pmatrix} \begin{pmatrix} C_1 \cos (\omega_1 t + \phi_1) \\ \vdots \\ C_s \cos (\omega_s t + \phi_s) \end{pmatrix}
\end{equation*}
做变量代换
\begin{equation*}
	\mbf{q} = \mbf{A} \mbf{\xi}
\end{equation*}
其中矩阵
\begin{equation*}
	\mbf{A} = \begin{pmatrix} \mbf{a}_1 & \cdots & \mbf{a}_s \end{pmatrix} = \begin{pmatrix} a_{11} & \cdots & a_{1s} \\ \vdots & & \vdots \\ a_{s1} & \cdots & a_{ss} \end{pmatrix}
\end{equation*}
则体系的解可以用新变量表示为
\begin{equation}
	\mbf{\xi} = \begin{pmatrix} C_1 \cos (\omega_1 t + \phi_1) \\ \vdots \\ C_s \cos (\omega_s t + \phi_s) \end{pmatrix}
\end{equation}
或者
\begin{equation}
	\xi_\mu = C_\mu \cos (\omega_\mu t+ \phi_\mu)
\end{equation}
即,新坐标可以将体系的运动表示为相互独立的简谐振动,坐标$\mbf{\xi}$称为体系的{\heiti 简正坐标}。

\begin{theorem}[同时对角化]
设$n$阶方阵$\mbf{A},\mbf{B}$都是对称矩阵,其中$\mbf{A}$是正定矩阵,则唯一存在一个非奇异方阵$\mbf{P}$满足
\begin{equation*}
	\mbf{P}^{\mathrm{T}} \mbf{A} \mbf{P} = \mbf{I},\quad \mbf{P}^{\mathrm{T}} \mbf{B} \mbf{P} = \begin{pmatrix} \lambda_1 & & \\ & \ddots & \\ & & \lambda_n \end{pmatrix}
\end{equation*}
其中$\lambda_i$满足
\begin{equation*}
	\det(\mbf{B} - \lambda_i \mbf{A}) = 0,\quad i=1,2,\cdots,n
\end{equation*}
\end{theorem}

由上述定理,矩阵$\mbf{A} = \begin{pmatrix} \mbf{a}_1 & \cdots & \mbf{a}_s \end{pmatrix}$可以使得矩阵$\mbf{M}$和$\mbf{K}$同时对角化\footnote{此时得到的同时对角化是同时对角化定理的一个弱化形式,此时的同时对角化形式不是唯一的。},即
\begin{align*}
	\mbf{M}' & = \mbf{A}^{\mathrm{T}} \mbf{M} \mbf{A} = \begin{pmatrix} m_1 & & \\ & \ddots & \\ & & m_s \end{pmatrix} \\
	\mbf{K}' & = \mbf{A}^{\mathrm{T}} \mbf{K} \mbf{A} = \begin{pmatrix} k_1 & & \\ & \ddots & \\ & & k_s \end{pmatrix}
\end{align*}
其中$m_\alpha,k_\alpha$满足
\begin{equation*}
	\det\left(\mbf{K}-\frac{k_\alpha}{m_\alpha} \mbf{M}\right) = 0,\quad \alpha = 1,2,\cdots,s
\end{equation*}
即有
\begin{align*}
	T & = \frac12 \dot{\mbf{q}}^{\mathrm{T}} \mbf{M} \dot{\mbf{q}} = \frac12 \dot{\mbf{\xi}}^{\mathrm{T}} \mbf{A}^{\mathrm{T}} \mbf{M} \mbf{A} \dot{\mbf{\xi}} = \frac12 \dot{\mbf{\xi}}^{\mathrm{T}} \mbf{M}' \dot{\mbf{\xi}} = \frac12 \sum_{\alpha=1}^s m_\alpha \dot{\xi}_\alpha^2 \\
	V & = \frac12 \mbf{q}^{\mathrm{T}} \mbf{K} \mbf{q} = \frac12 \mbf{\xi}^{\mathrm{T}} \mbf{A}^{\mathrm{T}} \mbf{K} \mbf{A} \mbf{\xi} = \frac12 \mbf{\xi}^{\mathrm{T}} \mbf{K}' \mbf{\xi} = \frac12 \sum_{\alpha=1}^s k_\alpha \xi_\alpha^2
\end{align*}
即,简正坐标使$\mbf{M}$和$\mbf{K}$同时对角化。简正坐标对应的运动方程为
\begin{equation*}
	m_\mu \ddot{\xi}_\mu + k_\mu \xi_\mu = 0,\quad \mu=1,2,\cdots,s
\end{equation*}
以简正坐标表示的各自由度的振动相互独立。

\begin{example}[双摆]
双摆体系的自由度$s=2$,取两个摆角$\theta_1,\theta_2$为广义坐标,则两个摆球的坐标分别为
\begin{equation*}
\begin{array}{ll}
	x_1 = l\sin\theta_1,& y_1 = l\cos\theta_1 \\
	x_2 = l\sin\theta_1+l\sin\theta_2,& y_2 = l\cos\theta_1+l\cos\theta_2
\end{array}
\end{equation*}
体系的势能为
\begin{equation*}
	V = mg(2l-y_2)+mg(l-y_1) = mgl(3-2\cos\theta_1 - \cos \theta_2)
\end{equation*}
由此可得稳定平衡位形为$\theta_1=\theta_2=0$。

\begin{figure}[htb]
\centering
\begin{asy}
	size(250);
	//双摆
	pair O;
	real theta1,theta2,l,r;
	O = (0,0);
	l = 0.7;
	theta1 = 30;
	theta2 = 15;
	draw(O--0.6*l*dir(-90),dashed);
	draw(Label("$l$",MidPoint,Relative(W)),O--l*dir(theta1-90));
	draw(l*dir(theta1-90)--l*dir(theta1-90)+0.6*l*dir(-90),dashed);
	draw(Label("$l$",MidPoint,Relative(W)),l*dir(theta1-90)--l*dir(theta1-90)+l*dir(theta2-90));
	r = 0.03;
	fill(shift(l*dir(theta1-90))*scale(r)*unitcircle,black);
	fill(shift(l*dir(theta1-90)+l*dir(theta2-90))*scale(r)*unitcircle,black);
	label("$m$",l*dir(theta1-90),2*W);
	label("$m$",l*dir(theta1-90)+l*dir(theta2-90),2*W);
	r = 0.1;
	draw(Label("$\theta_1$",MidPoint,Relative(E)),arc(O,r,-90,theta1-90),Arrow);
	r = 0.2;
	draw(Label("$\theta_2$",MidPoint,Relative(E)),arc(l*dir(theta1-90),r,-90,theta2-90),Arrow);
	draw(0.4*l*dir(180)--0.4*l*dir(0));
	int imax;
	pair dash,pace,P;
	imax = 15;
	dash = 0.06*l*dir(60);
	pace = (0.8*l/imax,0);
	P = 0.4*l*dir(180)+0.2*pace;
	for(int i=1;i<=imax;i=i+1){
		draw(P--P+dash);
		P = P+pace;
	}
	//draw(O--(0.7,0.02),invisible);
\end{asy}
\caption{双摆}
\label{第五章双摆示意}
\end{figure}

两个摆球的速度分量分别为
\begin{equation*}
\begin{array}{ll}
	\dot{x}_1 = l \dot{\theta}_1\cos\theta_1,& \dot{y}_1 = -l \dot{\theta}_1\sin\theta_1 \\
	\dot{x}_2 = l \dot{\theta}_1\cos\theta_1+l \dot{\theta}_2\cos\theta_2,& \dot{y}_2 = -l \dot{\theta}_1\sin\theta_1-l \dot{\theta}_2\sin\theta_2
\end{array}
\end{equation*}
体系的动能为
\begin{equation*}
	T = \frac12 m(\dot{x}_1^2+\dot{y}_1^2) + \frac12 m(\dot{x}_2^2+\dot{y}_2^2) = \frac12 ml^2 \dot{\theta}_1^2 + \frac12 m\left[l^2 \dot{\theta}_1^2 + l^2 \dot{\theta}_2^2 + 2l^2 \dot{\theta}_1 \dot{\theta}_2 \cos (\theta_1 - \theta_2)\right]
\end{equation*}
将势能和动能在稳定平衡位形处展开至二阶,即
\begin{align*}
	V & = mgl\left[3-2\left(1-\frac12 \theta_1^2+o(\theta_1^2)\right)-\left(1-\frac12 \theta_2^2+o(\theta_2^2)\right)\right] = \frac12 mgl(2\theta_1^2 + \theta_2^2) + o(\theta_1^2+\theta_2^2) \\
	T & = \frac12 ml^2 \dot{\theta}_1^2 + \frac12 m\left(l^2 \dot{\theta}_1^2 + l^2 \dot{\theta}_2^2 + 2l^2 \dot{\theta}_1 \dot{\theta}_2\right) = \frac12 ml^2 \left(2\dot{\theta}_1^2+\dot{\theta}_2^2+ 2\dot{\theta}_1\dot{\theta}_2\right)
\end{align*}
由此即有弹性矩阵和惯性矩阵
\begin{equation*}
	\mbf{K} = \begin{pmatrix} 2mgl & 0 \\ 0 & mgl \end{pmatrix},\quad \mbf{M} = \begin{pmatrix} 2ml^2 & ml^2 \\ ml^2 & ml^2 \end{pmatrix}
\end{equation*}
则有久期方程
\begin{equation*}
	\begin{vmatrix}
		2mgl-2ml^2 \omega^2 & -ml^2\omega^2 \\
		-ml^2\omega^2 & mgl-ml^2\omega^2
	\end{vmatrix} = 0
\end{equation*}
解得频率
\begin{equation*}
	\omega_1^2 = (2-\sqrt{2})\frac{g}{l},\quad \omega_2^2 = (2+\sqrt{2})\frac{g}{l}
\end{equation*}
其对应的本征矢量为
\begin{equation*}
	\mbf{a}_1 = \begin{pmatrix} 1 \\ \sqrt{2} \end{pmatrix},\quad \mbf{a}_2 = \begin{pmatrix} 1 \\ -\sqrt{2} \end{pmatrix}
\end{equation*}
因此,体系的运动解为
\begin{equation*}
	\begin{pmatrix} \theta_1 \\ \theta_2 \end{pmatrix} = \begin{pmatrix} 1 \\ \sqrt{2} \end{pmatrix} C_1 \cos (\omega_1 t+\phi_1) + \begin{pmatrix} 1 \\ -\sqrt{2} \end{pmatrix} C_2 \cos (\omega_2 t+\phi_2)
\end{equation*}
体系的简正坐标$\xi_1,\xi_2$满足
\begin{equation*}
	\begin{pmatrix} \theta_1 \\ \theta_2 \end{pmatrix} = \begin{pmatrix} 1 & 1 \\ \sqrt{2} & -\sqrt{2} \end{pmatrix} \begin{pmatrix} \xi_1 \\ \xi_2 \end{pmatrix}
\end{equation*}
\end{example}

\section{阻尼运动}

\subsection{阻尼的一般性质}

物体在运动过程中收到的阻力$\mbf{f}$,通常可以分成以下四类:
\begin{enumerate}
	\item {\heiti 摩擦阻力}。固体和固体接触面间的阻力,即通常所说的摩擦力。
	\item {\heiti 粘滞阻力}。由流体的粘滞性产生的阻力。
	\item {\heiti 尾流阻力}。当物体穿过静止不动的介质,或者说,当流体流过静止不动的障碍物时,物体前后压力差产生的阻力。
	\item {\heiti 波阻力}。对于运动速度超过声速的物体,会形成自物体发射到周围介质中的波(激波),它要消耗物体的能量,由此产生的阻力即为波阻力。
\end{enumerate}

产生以上各种阻力的根源是十分复杂的,而且各不相同。阻力总是消耗运动物体的能量,使物体的能量从运动物体向周围介质散逸出去最终转化为热,因此阻力也常被称为{\heiti 耗散力}。不过不论是哪一种阻力,其方向恒与物体的运动方向相反,其大小与物体的形状和尺寸、周围介质的物性以及物体的运动速度有关,因此阻力总是可以表达为
\begin{equation}
	\mbf{f} = -cf(v) \frac{\dot{\mbf{r}}}{v}
\end{equation}
其中函数$f(v)$反映阻力随速度的变化关系,系数$c$则和物体的形状、大小、表面状况及介质的物性有关。

\subsection{恒力作用下的阻尼直线运动}

最简单的一种阻尼运动是无其它外力或仅受恒力作用的阻尼直线运动,其运动方程可以表示为
\begin{equation*}
	m\frac{\mathrm{d} v}{\mathrm{d} t} = F_0 - cf(v)
\end{equation*}
或者
\begin{equation}
	\frac{\mathrm{d} v}{\mathrm{d} t} = \alpha-2\beta f(v)
	\label{阻尼直线运动的微分方程}
\end{equation}
式中$\alpha=\dfrac{F_0}{m}$是由恒力$F_0$产生的加速度,$2\beta = \dfrac{c}{m}$称为阻尼因子。当$c$不变时,$\beta$也是一个常数,若$c$和速度有关,则$\beta$也是速度的函数,不论何种情形,式\eqref{阻尼直线运动的微分方程}的右端只是速度$v$的函数,对其进行分离变量,即有
\begin{equation}
	t = \int_{v_0}^v \frac{\mathrm{d} v}{\alpha-2\beta f(v)}
\end{equation}
再根据$v = \dfrac{\mathrm{d} x}{\mathrm{d} t}$,可有
\begin{equation}
	x = \int_{v_0}^v \frac{v\mathrm{d} v}{\alpha-2\beta f(v)}
\end{equation}
求出上面的两个积分并从中消去$v$即可得到物体的运动情况$x = x(t)$。

\subsection{一维阻尼振动}

取体系的平衡位置为广义坐标原点,并假定阻力是速度的一次函数\footnote{在微振动情况下,这个近似是可以接受的。},则物体的运动方程为
\begin{equation*}
	m\ddot{x} = -kx - c\dot{x}
\end{equation*}
引入$\omega_0^2 = \dfrac{k}{m}$,此方程可以表示为
\begin{equation}
	\ddot{x} + 2\beta \dot{x} + \omega_0^2 x = 0
	\label{一维阻尼振动微分方程}
\end{equation}
此处$\omega_0$即为没有阻尼时物体自由振动的角频率。方程\eqref{一维阻尼振动微分方程}的通解为
\begin{equation}
	x = C_1 \mathrm{e}^{\lambda_1 t} + C_2 \mathrm{e}^{\lambda_2 t}
	\label{一维阻尼振动的通解}
\end{equation}
式中$\lambda_1$和$\lambda_2$是特征方程
\begin{equation*}
	\lambda^2 + 2\beta \lambda + \omega_0^2 = 0
\end{equation*}
的两个根,即
\begin{equation}
	\lambda_{1,2} = -\beta \pm \sqrt{\beta^2-\omega_0^2}
\end{equation}
下面分三种情况讨论解\eqref{一维阻尼振动的通解}的性质:
\begin{enumerate}
	\item $\beta < \omega_0$。此时通解可以表示为
	\begin{equation}
		x = a\mathrm{e}^{-\beta t} \cos (\omega t-\phi)
	\end{equation}
	式中
	\begin{equation*}
		\omega = \sqrt{\omega_0^2 -\beta^2} ,\quad \phi = \arctan\left(\frac{\dot{x}_0}{x_0 \omega} + \frac{\beta}{\omega}\right),\quad a = x_0\left[1+\left(\frac{\dot{x}_0}{x_0\omega} + \frac{\beta}{\omega}\right)^2\right]
	\end{equation*}
	均为实常数,$x_0$和$\dot{x}_0$为物体的初位置和初速度。此时振动的图像如图\ref{一维阻尼振动情形1}所示。
	
\begin{figure}[htb]
\centering
\begin{minipage}[t]{0.45\textwidth}
\begin{asy}
	size(190);
	//一维阻尼振动情形1
	pair O;
	real tscale,tmax,xmax,omega0,beta,x0,v0,omega,phi,a;
	tmax = 10;
	xmax = 15;
	omega0 = 2;
	beta = 0.2;
	x0 = 4;
	v0 = 10;
	omega = sqrt(omega0**2-beta**2);
	phi = atan(v0/(x0*omega)+beta/omega);
	a = x0+x0*(v0/(x0*omega)+beta/omega)**2;
	real vibrate(real t){
		return a*exp(-beta*t)*cos(omega*t-phi);
	}
	real decay(real t){
		return a*exp(-beta*t);
	}
	tscale = 3;
	draw(Label("$t$",EndPoint),xscale(tscale)*(O--1.1*tmax*dir(0)),Arrow);
	draw(Label("$x$",EndPoint),1.1*xmax*dir(-90)--1.1*xmax*dir(90),Arrow);
	label("$O$",O,W);
	draw(xscale(tscale)*graph(vibrate,0,tmax));
	draw(Label("$\phantom{-}a\mathrm{e}^{-\beta t}$",MidPoint,Relative(W)),xscale(tscale)*graph(decay,0,tmax),dashed);
	draw(Label("$-a\mathrm{e}^{-\beta t}$",MidPoint,Relative(E)),yscale(-1)*xscale(tscale)*graph(decay,0,tmax),dashed);
\end{asy}
\caption{$\beta < \omega_0$}
\label{一维阻尼振动情形1}
\end{minipage}
\hspace{0.5cm}
\begin{minipage}[t]{0.45\textwidth}
\begin{asy}
	size(190);
	//一维阻尼振动情形2
	pair O;
	real x0,tmax,xmax,tscale,beta1,beta2,c1,c2;
	tmax = 10;
	xmax = 13;
	beta1 = 0.8;
	beta2 = 0.2;
	c1 = 1;
	c2 = 10;
	real decay(real t){
		return c1*exp(-beta1*t)+c2*exp(-beta2*t);
	}
	tscale = 2;
	draw(Label("$t$",EndPoint),xscale(tscale)*(O--1.1*tmax*dir(0)),Arrow);
	draw(Label("$x$",EndPoint),O--1.1*xmax*dir(90),Arrow);
	label("$O$",O,W);
	draw(xscale(tscale)*graph(decay,0,tmax));
\end{asy}
\caption{$\beta > \omega_0$}
\label{一维阻尼振动情形2}
\end{minipage}
\end{figure}

	\item $\beta > \omega_0$。这是$\lambda$的两个值都是实数,而且都是负数,式\eqref{一维阻尼振动的通解}可改写为
	\begin{equation}
		x = c_1 \mathrm{e}^{-(\beta-\sqrt{\beta^2-\omega_0^2})t} + c_2 \mathrm{e}^{-(\beta+\sqrt{\beta^2-\omega_0^2})t}
	\end{equation}
	这时$x$是$t$的单调减少函数,当$t \to \infty$时,$x$趋向平衡位置而没有振动,如图\ref{一维阻尼振动情形2}所示。这种运动类型称为{\heiti 非周期性衰减}。
	
	\item $\beta = \omega_0$。此时特征方程只有一个根$\lambda=-\beta$,此时方程\eqref{一维阻尼振动微分方程}的通解为
	\begin{equation}
		x = (c_1+c_2 t) \mathrm{e}^{-\beta t}
	\end{equation}
	这是非周期性衰减的特殊情况,虽然不一定是单调减少的运动,但同样没有振动的性质。
\end{enumerate}

现在讨论有阻尼时的强迫振动。在方程\eqref{一维阻尼振动微分方程}中添加强迫力$F\cos \omega_p t$可得强迫振动的运动方程
\begin{equation}
	\ddot{x} + 2\beta \dot{x} + \omega_0^2 x = \frac{F}{m} \cos \omega_p t
	\label{有阻尼的强迫振动运动方程}
\end{equation}
采用复数求解比较简便,即求解复变量微分方程
\begin{equation}
	\ddot{x} + 2\beta \dot{x} + \omega_0^2 x = \frac{F}{m} \mathrm{e}^{\mathrm{i} \omega_p t}
	\label{复数形式的有阻尼的强迫振动运动方程}
\end{equation}
这个方程的通解就是对应的齐次方程的通解$x_1$和任意一个特解$x_2$的和。此处$x_1$已经在自由阻尼振动中讨论过了,现在只需求一个特解$x_2$即可。由此,可令
\begin{equation}
	x_2 = B\mathrm{e}^{\mathrm{i}\omega_p t} = b\mathrm{e}^{\mathrm{i}(\omega_p t - \delta)}
	\label{有阻尼的强迫振动的一个特解}
\end{equation}
此处$B$是复常数,$b$和$-\delta$分别是$B$的模和幅角,它们都是实数。将式\eqref{有阻尼的强迫振动的一个特解}代入方程\eqref{复数形式的有阻尼的强迫振动运动方程}中,可得
\begin{equation*}
	B = \frac{F}{m(\omega_0^2-\omega_p^2 +2\mathrm{i}\beta \omega_p)}
\end{equation*}
因此有
\begin{equation}
\begin{cases}
	b = \dfrac{F}{m\sqrt{(\omega_0^2-\omega_p)^2+4\beta^2 \omega_p^2}} \\[1.5ex]
	\tan \delta = \dfrac{2\beta \omega_p}{\omega_0^2 - \omega_p^2}
\end{cases}
\label{强迫振动的特解参数}
\end{equation}
现在只讨论有振动的情形,即$\omega_0>\beta$的情形,此时方程\eqref{有阻尼的强迫振动运动方程}的通解为
\begin{equation}
	x = a\mathrm{e}^{-\beta t} \cos (\omega t-\phi) + b\cos(\omega_p t - \delta)
\end{equation}
其中第一项随时间以指数规律衰减,所以经过足够长时间后,留下的只是第二项
\begin{equation*}
	x = b\cos (\omega_p t-\delta)
\end{equation*}
这是一个简谐运动,其频率即为强迫力的频率$\omega_p$,振幅$b$由式\eqref{强迫振动的特解参数}决定。令
\begin{equation}
	\kappa = \frac{1}{\sqrt{\left(1-\dfrac{\omega_p^2}{\omega_0^2}\right)^2+\gamma^2 \dfrac{\omega_p^2}{\omega_0^2}}}
	\label{强迫振动的振幅放大因子}
\end{equation}
称为{\heiti 振幅的放大因子},其中$\gamma=\dfrac{2\beta}{\omega_0}$,则振幅$b$可以表示为
\begin{equation*}
	b = \kappa\frac{F}{k}
\end{equation*}
$\kappa$随阻尼因子$\gamma$和频率比值$\dfrac{\omega_p}{\omega_0}$的变化如图\ref{阻尼放大因子与频率比的关系}所示。

\begin{figure}[htb]
\centering
\begin{asy}
	size(300);
	//阻尼放大因子与频率比的关系
	picture tmp;
	pair O;
	real gamma,omega,xs,l;
	path clp;
	real kappa(real omega){
		return 1/(sqrt((1-omega**2)**2+(gamma*omega)**2));
	}
	pair mcurve(real gamma){
		return (sqrt((2-gamma**2)/2),2/(gamma*sqrt(4-gamma**2)));
	}
	xs = 2.5;
	draw(Label("$\dfrac{\omega_p}{\omega_0}$",EndPoint),xscale(xs)*(O--(2.75,0)),Arrow);
	draw(Label("$\kappa$",EndPoint),O--(0,5.5),Arrow);
	draw(tmp,graph(mcurve,0.2,0.6),dashed);
	gamma = 0.5;
	draw(tmp,graph(kappa,0,3),red+linewidth(1bp));
	gamma = 0.3;
	draw(tmp,graph(kappa,0,3),orange+linewidth(1bp));
	gamma = 0.2;
	draw(tmp,graph(kappa,0,3),green+linewidth(1bp));
	gamma = 0;
	draw(tmp,graph(kappa,0,3),blue+linewidth(1bp));
	clp = box((0,0),(2.5,5.2));
	clip(tmp,clp);
	//画刻度
	l = 0.05;
	for(int i=1;i<=10;i=i+1){
		draw(tmp,(0.25*i,l)--(0.25*i,0));
	}
	for(int i=1;i<=10;i=i+1){
		draw((0,0.5*i)--(l,0.5*i));
	}
	add(xscale(xs)*tmp);
	erase(tmp);
	label("$0.5$",(0,0.5*1),W);
	label("$1.0$",(0,0.5*2),W);
	label("$1.5$",(0,0.5*3),W);
	label("$2.0$",(0,0.5*4),W);
	label("$2.5$",(0,0.5*5),W);
	label("$3.0$",(0,0.5*6),W);
	label("$3.5$",(0,0.5*7),W);
	label("$4.0$",(0,0.5*8),W);
	label("$4.5$",(0,0.5*9),W);
	label("$5.0$",(0,0.5*10),W);
	label("$0$",xs*(0,0),S);
	label("$0.5$",xs*(0.5*1,0),S);
	label("$1.0$",xs*(0.5*2,0),S);
	label("$1.5$",xs*(0.5*3,0),S);
	label("$2.0$",xs*(0.5*4,0),S);
	label("$2.5$",xs*(0.5*5,0),S);
	real height,width,right,top;
	pair P;
	path legendbox;
	right = 2.5;
	top = 4.8;
	width = 0.55;
	height = 1.5;
	legendbox = xscale(xs)*box((right,top),(right-width,top-height));
	fill(shift(0.07,-0.07)*legendbox,black);
	unfill(legendbox);
	draw(legendbox);
	P = (right-width/2,top-1*height/8);
	label("$\gamma=0$",xscale(xs)*P,blue);
	P = (right-width/2,top-3*height/8);
	label("$\gamma=0.2$",xscale(xs)*P,green);
	P = (right-width/2,top-5*height/8);
	label("$\gamma=0.3$",xscale(xs)*P,orange);
	P = (right-width/2,top-7*height/8);
	label("$\gamma=0.5$",xscale(xs)*P,red);
\end{asy}
\caption{$\kappa$随阻尼因子$\gamma$和频率比值$\dfrac{\omega_p}{\omega_0}$的关系}
\label{阻尼放大因子与频率比的关系}
\end{figure}

强迫振动有相位差,不同频率下相位差$\delta$随频率比值$\dfrac{\omega_p}{\omega_0}$的关系如图\ref{强迫振动的相位差与频率比之间的关系}所示。

\begin{figure}[htb]
\centering
\begin{asy}
	size(300);
	//强迫振动的相位差与频率比之间的关系
	picture tmp;
	pair O;
	real gamma,xs,l;
	real delta(real omega){
		real de;
		if(omega==1) return pi/2;
		de = atan(gamma*omega/(1-omega**2));
		if(de<0) de=de+pi;
		return de;
	}
	xs = 2;
	draw(Label("$\dfrac{\omega_p}{\omega_0}$",EndPoint),xscale(xs)*(O--(2.75,0)),Arrow);
	draw(Label("$\delta$",EndPoint),O--(0,1.1*pi),Arrow);
	gamma = 1;
	draw(tmp,graph(delta,0,2.5),red+linewidth(1bp));
	gamma = 0.5;
	draw(tmp,graph(delta,0,2.5),orange+linewidth(1bp));
	gamma = 0.25;
	draw(tmp,graph(delta,0,2.5),green+linewidth(1bp));
	draw(tmp,O--(1,0)--(1,pi)--(2.5,pi),blue+linewidth(1bp));
	draw(tmp,(0,pi/2)--(1,pi/2),dashed);
	draw(tmp,(0,pi)--(1,pi),dashed);
	l = 0.05;
	for(int i=1;i<=5;i=i+1){
		draw(tmp,(0.5*i,l)--(0.5*i,0));
	}
	add(xscale(xs)*tmp);
	label("$\dfrac{\pi}{2}$",(0,pi/2),W);
	label("$\pi$",(0,pi),W);
	label("$0$",xs*(0,0),S);
	label("$0.5$",xs*(0.5*1,0),S);
	label("$1.0$",xs*(0.5*2,0),S);
	label("$1.5$",xs*(0.5*3,0),S);
	label("$2.0$",xs*(0.5*4,0),S);
	label("$2.5$",xs*(0.5*5,0),S);
	real height,width,right,top;
	pair P;
	path legendbox;
	right = 2.6;
	top = pi/2+0.5;
	width = 0.55;
	height = 1.2;
	legendbox = xscale(xs)*box((right,top),(right-width,top-height));
	fill(shift(0.06,-0.06)*legendbox,black);
	unfill(legendbox);
	draw(legendbox);
	P = (right-width/2,top-1*height/8);
	label("$\gamma=0$",xscale(xs)*P,blue);
	P = (right-width/2,top-3*height/8);
	label("$\gamma=0.25$",xscale(xs)*P,green);
	P = (right-width/2,top-5*height/8);
	label("$\gamma=0.5$",xscale(xs)*P,orange);
	P = (right-width/2,top-7*height/8);
	label("$\gamma=1$",xscale(xs)*P,red);
\end{asy}
\caption{相位差$\delta$随频率比值$\dfrac{\omega_p}{\omega_0}$的关系}
\label{强迫振动的相位差与频率比之间的关系}
\end{figure}

\subsection{耗散函数}

在质点系中,每个质点都收到与速度方向相反的阻尼力
\begin{equation*}
	\mbf{f}_i = -f_i(v_i) \frac{\dot{\mbf{r}}_i}{v_i},\quad i=1,2,\cdots,n
\end{equation*}
其中$f_i(v_i)>0$,则其对应的广义力为
\begin{equation*}
	f_\alpha = -\sum_{i=1}^n f_i(v_i) \frac{\dot{\mbf{r}}_i}{v_i} \cdot \frac{\pl \mbf{r}_i}{\pl q_\alpha} = -\sum_{i=1}^n f_i(v_i) \frac{\dot{\mbf{r}}_i}{v_i} \cdot \frac{\pl \dot{\mbf{r}}_i}{\pl \dot{q}_\alpha} = -\sum_{i=1}^n f_i(v_i) \frac{\pl v_i}{\pl \dot{q}_\alpha}
\end{equation*}
式中利用了经典Lagrange关系\eqref{经典Lagrange关系}。引入{\heiti 耗散函数}
\begin{equation}
	D(\dot{\mbf{r}}_1,\cdots,\dot{\mbf{r}}_n) = \sum_{i=1}^n \int_0^{v_i} f_i(v'_i) \mathrm{d} v'_i
\end{equation}
则广义阻尼力可以表示为
\begin{equation*}
	f_\alpha = -\frac{\pl D}{\pl \dot{q}_\alpha}
\end{equation*}
如果体系内其余主动力皆为有势力,则可有系统方程
\begin{equation*}
	\frac{\mathrm{d}}{\mathrm{d} t} \frac{\pl L}{\pl \dot{q}_\alpha} - \frac{\pl L}{\pl q_\alpha} = f_\alpha,\quad \alpha = 1,2,\cdots,s
\end{equation*}
用耗散函数可以表示为
\begin{equation}
	\frac{\mathrm{d}}{\mathrm{d} t} \frac{\pl L}{\pl \dot{q}_\alpha} - \frac{\pl L}{\pl q_\alpha} + \frac{\pl D}{\pl \dot{q}_\alpha} = 0,\quad \alpha=1,2,\cdots,s
	\label{阻尼系统的Lagrange方程}
\end{equation}

对于线性阻尼力$\mbf{f}_i = -c_i \dot{\mbf{r}}_i,\,\,i=1,2,\cdots,n$,其中$c_i > 0$,此时耗散函数可以表示为
\begin{equation*}
	D = \sum_{i=1}^n \frac12 c_i \dot{\mbf{r}}_i^2 %=: \sum_{m=0}^2 D_m(\mbf{q},\dot{\mbf{q}},t)
\end{equation*}
这种形式的耗散函数称为{\heiti Rayleigh耗散函数}。根据坐标转换关系
\begin{equation*}
	\mbf{r}_i = \mbf{r}_i(\mbf{q})
\end{equation*}
可有
\begin{align*}
	\dot{\mbf{r}}_i & = \sum_{\alpha=1}^s \frac{\pl \mbf{r}_i}{\pl q_\alpha} \dot{q}_\alpha \\
	\dot{\mbf{r}}_i^2 & = \sum_{\alpha,\beta=1}^s \frac{\pl \mbf{r}_i}{\pl q_\alpha} \cdot \frac{\pl \mbf{r}_i}{\pl q_\beta} \dot{q}_\alpha \dot{q}_\beta
\end{align*}
此时,可有
\begin{align}
	D & = \frac12 \sum_{i=1}^n \sum_{\alpha,\beta=1}^s c_i \frac{\pl \mbf{r}_i}{\pl q_\alpha} \cdot \frac{\pl \mbf{r}_i}{\pl q_\beta} \dot{q}_\alpha \dot{q}_\beta = \frac12 \sum_{\alpha,\beta=1}^s c_{\alpha\beta} \dot{q}_\alpha \dot{q}_\beta
	\label{用广义速度表示的Rayleigh耗散函数}
\end{align}
式中
\begin{equation*}
	c_{\alpha\beta} = \sum_{i=1}^n c_i \frac{\pl \mbf{r}_i}{\pl q_\alpha} \cdot \frac{\pl \mbf{r}_i}{\pl q_\beta}
\end{equation*}
此处$c_{\alpha\beta}$只与广义坐标$\mbf{q}$有关,而与广义速度$\dot{\mbf{q}}$无关,因此Rayleigh耗散函数$D$是广义速度$\dot{\mbf{q}}$的二次齐次函数。特别地,在微振动问题中,$T$和$V$都只保留至二阶小量,因此$D$也只要保留到二阶小量即可,在这种情况下,$c_{\alpha\beta}$可以看成常数。

耗散函数$D$具有能量变化率的量纲,它决定了力学体系能量的耗散率$-\dfrac{\mathrm{d}E}{\mathrm{d}t}$。对于Rayleigh耗散函数,当$c_{\alpha\beta}$为常数时,可有
\begin{align*}
	\frac{\mathrm{d} E}{\mathrm{d} t} & = \frac{\mathrm{d}}{\mathrm{d} t} \left(\sum_{\alpha=1}^s \dot{q}_\alpha \frac{\pl L}{\pl \dot{q}_\alpha} - L\right) = \sum_{\alpha=1}^s \dot{q}_\alpha \left(\frac{\mathrm{d}}{\mathrm{d} t} \frac{\pl L}{\pl \dot{q}_\alpha} - \frac{\pl L}{\pl q_\alpha}\right)
\end{align*}
考虑到式\eqref{阻尼系统的Lagrange方程},可有
\begin{equation*}
	\frac{\mathrm{d} E}{\mathrm{d} t} = -\sum_{\alpha=1}^s \dot{q}_\alpha \frac{\pl D}{\pl \dot{q}_\alpha}
\end{equation*}
由式\eqref{用广义速度表示的Rayleigh耗散函数}可知,Rayleigh耗散函数$D$是广义速度$\dot{\mbf{q}}$的二次齐次函数,根据齐次函数的Euler定理(定理\ref{齐次函数的Euler定理},第\pageref{齐次函数的Euler定理}页)可知
\begin{equation*}
	\sum_{\alpha=1}^s \dot{q}_\alpha \frac{\pl D}{\pl \dot{q}_\alpha} = 2D
\end{equation*}
由此即有
\begin{equation}
	-\frac{\mathrm{d} E}{\mathrm{d} t} = 2D
\end{equation}
即单位时间内体系内耗散的能量等于耗散函数的两倍。


%第六章——刚体
\chapter{刚体}

任意两点之间距离在运动过程中保持不变的体系即可称为{\heiti 刚体}。刚体是一个理想模型,在研究问题的过程中,形变可忽略的体系都可以视为刚体。

\section{刚体运动学}

\subsection{刚体的自由度}

首先将刚体看作多质点系,其中的质点满足约束
\begin{equation*}
	|\mbf{r}_i-\mbf{r}_j| = C_{ij}\,\text{(常数)},\quad i,j=1,2,\cdots,n\,(i\neq j)
\end{equation*}
因此,其中的任意不共线的三个质点的位置确定,其余各质点的位置也相应地确定。此三个质点之间有三个完整约束,即$n=3,k=3$,因此自由度$s=3n-k=6$。每增加一个质点,增加三个独立约束,因此自由度不变。

由此可得,仅有内部刚性约束而无外部附加约束的刚体的自由度$s=6$。

\subsection{刚体的运动}
\label{节:刚体的运动}
为了描述刚体的运动,需要引入三个坐标系:
\begin{itemize}
	\item {\heiti 空间系}:即实验室坐标系,为惯性系。
	\item {\heiti 本体系}:以刚体上任选的一点(称为{\heiti 基点})$O$为原点,并固定在刚体上的坐标系。刚体上各点相对本体系静止。
	\item {\heiti 平动系}:以基点$O$为原点,相对空间系平动的坐标系。
\end{itemize}

\begin{figure}[htb]
\centering
\begin{asy}
	texpreamble("\usepackage{xeCJK}");
	texpreamble("\setCJKmainfont{SimSun}");
	usepackage("amsmath");
	import graph;
	import math;
	size(250);
	//刚体的运动
	pair O,OO,r,i,j,k;
	picture tmp;
	O = (0,0);
	i = 0.5*dir(-135);
	j = dir(0);
	k = dir(90);
	draw(Label("$X_0$",EndPoint),O--i,Arrow);
	draw(Label("$Y_0$",EndPoint),O--j,Arrow);
	draw(Label("$Z_0$",EndPoint),O--k,Arrow);
	label("$O_0$",O,SSE);
	OO = (0.7,0.5);
	draw(tmp,Label("$X$",EndPoint,SE),O--i,Arrow);
	draw(tmp,Label("$Y$",EndPoint),O--j,Arrow);
	draw(tmp,Label("$Z$",EndPoint),O--k,Arrow);
	label(tmp,"$O$",O,SSE);
	add(shift(OO)*scale(0.9)*tmp);
	erase(tmp);
	real f(real x){
		return x**2-0.3;
	}
	draw(tmp,graph(f,-1.5,1.5),linewidth(1bp));
	draw(tmp,shift((0,f(1.5)))*yscale(1/3)*scale(1.5)*unitcircle,linewidth(1bp));
	add(shift(OO)*rotate(-25)*scale(0.4)*tmp);
	erase(tmp);
	draw(tmp,Label("$x$",EndPoint,S),rotate(-25)*(O--i),Arrow);
	draw(tmp,Label("$y$",EndPoint),rotate(-25)*(O--j),Arrow);
	draw(tmp,Label("$z$",EndPoint,W),rotate(-25)*(O--k),Arrow);
	add(shift(OO)*scale(0.9)*tmp);
	r = (-0.1,0.6);
	//dot(OO+r);
	draw(Label("$\boldsymbol{r}_O$",Relative(0.3),Relative(E),black),O--OO,red,Arrow);
	draw(Label("$\boldsymbol{r}'$",Relative(0.3),Relative(W),black),OO--OO+r,red,Arrow);
	draw(Label("$\boldsymbol{r}$",Relative(0.5),Relative(W),black),O--OO+r,red,Arrow);
	//draw(O--(1.8,1.55),invisible);
\end{asy}
\caption{刚体的运动}
\label{刚体的运动}
\end{figure}

则刚体在这三个坐标系中的运动为(见图\ref{刚体的运动}):
\begin{itemize}
	\item {\heiti 本体系}:刚体静止。
	\item {\heiti 平动系}:刚体做定点运动(基点固定)。
	\item {\heiti 空间系}:刚体随基点平动的同时,绕基点做定点运动。
\end{itemize}

如果基点在空间系内不动,则平动系与空间系合一,刚体无平动,只有定点运动。即刚体平动自由度$s_t = 3$,定点运动自由度$s_r=3$。

\subsection{定点运动的矩阵表示}

\begin{figure}[htb]
\centering
\begin{asy}
	texpreamble("\usepackage{xeCJK}");
	texpreamble("\setCJKmainfont{SimSun}");
	usepackage("amsmath");
	import graph;
	import math;
	size(250);
	//坐标系的定点旋转
	pair O,i,j,k;
	O = (0,0);
	i = 0.5*dir(-135);
	j = dir(0);
	k = dir(90);
	pair xyz(triple coff){
		real x,y,z,r,theta,phi;
		r = coff.x;
		theta = coff.y;
		phi = coff.z;
		x = r*sin(theta)*cos(phi);
		y = r*sin(theta)*sin(phi);
		z = r*cos(theta);
		return x*i+y*j+z*k;
	}
	pair t2t(triple xyz){
		return xyz.x*i+xyz.y*j+xyz.z*k;
	}
	draw(Label("$X_1$",EndPoint,S),O--i,Arrow);
	draw(Label("$X_2$",EndPoint),O--j,Arrow);
	draw(Label("$X_3$",EndPoint),O--k,Arrow);
	label("$O$",O,SSE);
	//dot(xyz((1,pi/4,pi/3)),red);
	draw(Label("$\boldsymbol{R}$",MidPoint,Relative(E),black),O--xyz((1,pi/4,pi/3)),red,Arrow);
	draw(Label("$x_3$",EndPoint,black),O--xyz((1,pi/6,pi/3)),blue,Arrow);
	draw(Label("$x_2$",EndPoint,black),O--xyz((1,pi/6+pi/2,pi/3)),blue,Arrow);
	draw(Label("$x_1$",EndPoint,black),O--xyz((1,pi/2,pi/3-pi/2)),blue,Arrow);
	//draw(O--(1.15,0),invisible);
\end{asy}
\caption{坐标系的定点旋转}
\label{坐标系的定点旋转}
\end{figure}

设有矢量$\mbf{R}$,则其在坐标系$OX_1X_2X_3$和$Ox_1x_2x_3$中可以分别表示为
\begin{equation*}
	\mbf{R} = \sum_{i=1}^3 X_i \mbf{E}_i = \sum_{i=1}^3 x_j \mbf{e}_j
\end{equation*}
同理可有
\begin{equation*}
	\mbf{E}_i = \sum_{j=1}^3 (\mbf{E}_i\cdot \mbf{e}_j) \mbf{e}_j = \sum_{j=1}^3 u_{ij} \mbf{e}_j
\end{equation*}
此处$u_{ij} = \mbf{E}_j\cdot \mbf{e}_j$。因此有
\begin{equation}
	X_i = \mbf{E}_i \cdot \mbf{R} = \sum_{j=1}^3 u_{ij} (\mbf{e}_j \cdot \mbf{R}) = \sum_{j=1}^3 u_{ij} x_j
	\label{坐标转换关系}
\end{equation}
式\eqref{坐标转换关系}用矩阵可以表示为
\begin{equation}
	\mbf{X} = \mbf{U}\mbf{x}
\end{equation}
其中
\begin{equation*}
	\mbf{X} = \begin{pmatrix} X_1 \\ X_2 \\ X_3 \end{pmatrix},\quad \mbf{x} = \begin{pmatrix} x_1 \\ x_2 \\ x_3 \end{pmatrix},\quad \mbf{U} = \begin{pmatrix} u_{11} & u_{12} & u_{13} \\ u_{21} & u_{22} & u_{23} \\ u_{31} & u_{32} & u_{33} \end{pmatrix}
\end{equation*}
坐标系的定点运动需要满足矢量的长度不变,即
\begin{equation*}
	\mbf{R}^{\mathrm{T}} \mbf{R} = \mbf{X}^{\mathrm{T}} \mbf{X} = \mbf{x}^{\mathrm{T}} \mbf{x}
\end{equation*}
所以有
\begin{equation*}
	\mbf{X}^{\mathrm{T}} \mbf{X} = \mbf{x}^{\mathrm{T}} \mbf{U}^{\mathrm{T}} \mbf{U} \mbf{x} = \mbf{x}^{\mathrm{T}} \mbf{x}
\end{equation*}
即
\begin{equation*}
	\mbf{U}^{\mathrm{T}} \mbf{U} = \mbf{I}
\end{equation*}
因此,系数矩阵$\mbf{U}$是正交矩阵,其独立矩阵元有3个。

对于刚体的定点运动,本体系内刚体的各点坐标不变,将某参考点的坐标记作$\mbf{x}$。设在$t_1,t_2$时刻,刚体上该参考点在平动系的坐标分别为$\mbf{X}_1,\mbf{X}_2$,则有
\begin{equation*}
	\mbf{X}_1 = \mbf{U}_1 \mbf{x},\quad \mbf{X}_2 = \mbf{U}_2 \mbf{x}
\end{equation*}
此处,$\mbf{U}_1,\mbf{U}_2$都是正交矩阵,所以有
\begin{equation*}
	\mbf{X}_2 = \mbf{U}_2 \mbf{U}_1^{-1} \mbf{X}_1 = (\mbf{U}_2 \mbf{U}_1^{\mathrm{T}}) \mbf{X}_1 = \mbf{U} \mbf{X}_1
\end{equation*}
其中矩阵$\mbf{U}$满足
\begin{equation*}
	\mbf{U}^{\mathrm{T}} \mbf{U} = \mbf{U}_1 \mbf{U}_2^{\mathrm{T}} \mbf{U}_2 \mbf{U}_1^{\mathrm{T}} = \mbf{I}
\end{equation*}
因此,矩阵$\mbf{U}$也是正交矩阵。由此可得,刚体的定点运动可用正交矩阵表示。

\iffalse
\begin{example}[有限角位移的非对易性]

\begin{figure}[htb]
\centering
\begin{asy}
	texpreamble("\usepackage{xeCJK}");
	texpreamble("\setCJKmainfont{SimSun}");
	usepackage("amsmath");
	import graph;
	import math;
	size(200);
	//刚体的定轴转动
	pair O,n,R,RR;
	real theta,l,r,r1,r2;
	path cir;
	picture tmp;
	O = (0,0);
	theta = 70;
	n = dir(theta);
	l = 2;
	r = 0.8;
	r1 = -0.2;
	r2 = -0.05;
	draw(Label("$\boldsymbol{n}$",EndPoint),O--O+l*n,Arrow);
	cir = scale(r)*unitcircle;
	draw(tmp,cir);
	R = relpoint(cir,r1);
	RR = relpoint(cir,r2);
	draw(tmp,R--O--RR);
	draw(tmp,R--RR,red,Arrow);
	draw(tmp,Label(rotate(90-theta)*"$\phi$",EndPoint,Relative(E)),arc(O,0.2,degrees(R),degrees(RR)),Arrow);
	add(shift(2/3*l*n)*rotate(theta-90)*yscale(1/3)*tmp);
	R = shift(2/3*l*n)*rotate(theta-90)*yscale(1/3)*R;
	RR = shift(2/3*l*n)*rotate(theta-90)*yscale(1/3)*RR;
	//draw(R--2/3*l*n--RR);
	//draw(R--RR,red,Arrow);
	draw(Label("$\boldsymbol{R}$",MidPoint,Relative(E)),O--R,Arrow);
	draw(Label("$\boldsymbol{R}'$",MidPoint,Relative(E)),O--RR,Arrow);
	//draw(O--(1.3,0),invisible);
\end{asy}
\caption{刚体的定轴转动}
\label{刚体的定轴转动}
\end{figure}

考虑刚体绕指向为$\mbf{n}$的转动轴旋转$\theta$,即{\heiti 定轴转动},如图\ref{刚体的定轴转动}所示。定轴转动是定点运动的特例,因此可用矩阵表示为
\begin{equation*}
	\mbf{X}' = \mbf{U}\mbf{X}
\end{equation*}
如果进行两次定轴转动(转轴指向可以不同),那么可有
\begin{equation*}
	\mbf{X}'' = \mbf{U}' \mbf{X}' = \mbf{U}' \mbf{U} \mbf{X}
\end{equation*}
由于矩阵的乘积具有不可交换性,因此有限角位移一般不具有对易性,因而不能用矢量表示。
\end{example}\fi

\begin{example}[有限定点运动的非对易性]

刚体的定点运动可用矩阵表示为
\begin{equation*}
	\mbf{X}' = \mbf{U}\mbf{X}
\end{equation*}
如果再进行一次定轴运动,那么可有
\begin{equation*}
	\mbf{X}'' = \mbf{U}' \mbf{X}' = \mbf{U}' \mbf{U} \mbf{X}
\end{equation*}
由于矩阵的乘积具有不可交换性,因此有限定点运动一般不具有对易性,从而有限定点转动不能用矢量表示。
\end{example}

\subsection{无限小定点运动的矢量表示}

有限定点运动可以由连续的无限小定点运动合成。考虑
\begin{equation*}
	\mbf{X}' = \mbf{U}\mbf{X}
\end{equation*}
满足
\begin{equation*}
	\mathrm{d} \mbf{X}' = \mbf{X}' - \mbf{X} = (\mbf{U}-\mbf{I}) \mbf{X}
\end{equation*}
是一阶无穷小量,即有
\begin{equation*}
	\mbf{U} = \mbf{I}+\mbf{u}
\end{equation*}
此处$\mbf{u}$为一阶无穷小量。$\mbf{U}$应为正交矩阵,即有
\begin{equation*}
	\mbf{I} = (\mbf{I}+\mbf{u}^{\mathrm{T}})(\mbf{I}+\mbf{u}) = \mbf{I} + \mbf{u}^{\mathrm{T}} + \mbf{u} + \mbf{u}^{\mathrm{T}} \mbf{u}
\end{equation*}
略去上式中的高阶无穷小量,可以得到
\begin{equation}
	\mbf{u}^{\mathrm{T}} = -\mbf{u}
\end{equation}
即无穷小定点运动的变换矩阵为反对称矩阵,将其记作
\begin{equation*}
	\mbf{u} = \begin{pmatrix} 0 & -\eps_3 & \eps_2 \\ \eps_3 & 0 & -\eps_1 \\ -\eps_2 & \eps_1 & 0 \end{pmatrix}
\end{equation*}
所以有
\begin{equation*}
	\mathrm{d} \mbf{X} = \mbf{u} \mbf{X} = \begin{pmatrix} \eps_2 X_3 - \eps_3 X_2 \\ \eps_3 X_1 - \eps_1 X_3 \\ \eps_1 X_2 - \eps_2 X_1 \end{pmatrix} = \mbf{\eps} \times \mbf{R}
\end{equation*}
此处$\mbf{\eps} = \begin{pmatrix} \eps_1 \\ \eps_2 \\ \eps_3 \end{pmatrix}$称为{\heiti 无限小角位移矢量}。

将无限小角位移矢量表示为$\mbf{\eps} = \eps \mbf{n}$,其中$\mbf{n}$是单位矢量,取位矢$\mbf{R} = R\mbf{n}$,则在此无限小定点运动下的位移为
\begin{equation*}
	\mathrm{d} \mbf{R} = \eps \mbf{n} \times R\mbf{n} = \mbf{0}
\end{equation*}
上式说明,此定点运动下,所有位于过定点方向为$\mbf{n}$的直线上的点都是固定的,即$\mbf{n}$为转轴。因此无限小定点运动是绕瞬时轴的无限小转动,定点运动即为{\heiti 定点转动}。

\subsection{刚体运动的矢量-矩阵描述}

\label{节:刚体运动的矢量-矩阵描述}
\ref{节:刚体的运动}节已指出,刚体的运动可由平动系相对于空间坐标系的平动与本体系相对于平动系的定点转动,即刚体中任意一点的位矢为
\begin{equation}
	\mbf{r} = \mbf{r}_O + \mbf{r}'
	\label{刚体中任意一点的位矢的矢量表示}
\end{equation}
设矢量$\mbf{r}'$在平动系中的坐标为$\mbf{X}$,在本体系中的坐标为$\mbf{x}$\footnote{需要注意到,平动系中的坐标$\mbf{X}$是与时间有关的,而本体系中的坐标$\mbf{x}$则与时间无关。},则它们之间通过旋转矩阵可以表示为
\begin{equation*}
	\mbf{X} = \mbf{U} \mbf{x}
\end{equation*}
此处矩阵$\mbf{U} = \mbf{U}(t)$是正交矩阵,即矩阵$\mbf{U}$满足
\begin{equation}
	\mbf{U}^{\mathrm{T}} \mbf{U} = \mbf{I}
	\label{正交矩阵的条件}
\end{equation}
记平动系的基为$\mbf{E}_1,\mbf{E}_2,\mbf{E}_3$,本体系的基为$\mbf{e}_1,\mbf{e}_2,\mbf{e}_3$\footnote{需要注意到,平动系中的基$\mbf{E}_1,\mbf{E}_2,\mbf{E}_3$是与时间无关的,而本体系中的基$\mbf{e}_1,\mbf{e}_2,\mbf{e}_3$则与时间有关。},则有
\begin{align*}
	\mbf{r}' & = \begin{pmatrix} \mbf{E}_1 & \mbf{E}_2 & \mbf{E}_3 \end{pmatrix} \begin{pmatrix} X_1 \\ X_2 \\ X_3 \end{pmatrix} = \mbf{E} \mbf{X} = \begin{pmatrix} \mbf{e}_1 & \mbf{e}_2 & \mbf{e}_3 \end{pmatrix} \begin{pmatrix} x_1 \\ x_2 \\ x_3 \end{pmatrix} = \mbf{e} \mbf{x}
\end{align*}
由此可以得到
\begin{equation*}
	\mbf{e} = \mbf{E} \mbf{U}
\end{equation*}
综上,式\eqref{刚体中任意一点的位矢的矢量表示}可以用矩阵表示为\footnote{在用矩阵表示矢量表达式时,如不同时乘以该坐标系基构成的矩阵,则容易造成谬误。}
\begin{align}
	\mbf{r} & = \mbf{r}_O + \mbf{E} \mbf{X} = \mbf{r}_O + \mbf{E} \mbf{U} \mbf{x}
	\label{刚体中任意一点的位矢}
\end{align}

\begin{figure}[htb]
\centering
\begin{asy}
	texpreamble("\usepackage{xeCJK}");
	texpreamble("\setCJKmainfont{SimSun}");
	usepackage("amsmath");
	import graph;
	import math;
	size(250);
	//刚体的运动
	pair O,OO,r,i,j,k;
	picture tmp;
	O = (0,0);
	i = 0.5*dir(-135);
	j = dir(0);
	k = dir(90);
	draw(Label("$X_0$",EndPoint),O--i,Arrow);
	draw(Label("$Y_0$",EndPoint),O--j,Arrow);
	draw(Label("$Z_0$",EndPoint),O--k,Arrow);
	label("$O_0$",O,SSE);
	OO = (0.7,0.5);
	draw(tmp,Label("$X$",EndPoint,SE),O--i,Arrow);
	draw(tmp,Label("$Y$",EndPoint),O--j,Arrow);
	draw(tmp,Label("$Z$",EndPoint),O--k,Arrow);
	label(tmp,"$O$",O,SSE);
	add(shift(OO)*scale(0.9)*tmp);
	erase(tmp);
	real f(real x){
		return x**2-0.3;
	}
	draw(tmp,graph(f,-1.5,1.5),linewidth(1bp));
	draw(tmp,shift((0,f(1.5)))*yscale(1/3)*scale(1.5)*unitcircle,linewidth(1bp));
	add(shift(OO)*rotate(-25)*scale(0.4)*tmp);
	erase(tmp);
	draw(tmp,Label("$x$",EndPoint,S),rotate(-25)*(O--i),Arrow);
	draw(tmp,Label("$y$",EndPoint),rotate(-25)*(O--j),Arrow);
	draw(tmp,Label("$z$",EndPoint,W),rotate(-25)*(O--k),Arrow);
	add(shift(OO)*scale(0.9)*tmp);
	r = (-0.1,0.6);
	//dot(OO+r);
	draw(Label("$\boldsymbol{r}_O$",Relative(0.3),Relative(E),black),O--OO,red,Arrow);
	draw(Label("$\boldsymbol{r}'$",Relative(0.3),Relative(W),black),OO--OO+r,red,Arrow);
	draw(Label("$\boldsymbol{r}$",Relative(0.5),Relative(W),black),O--OO+r,red,Arrow);
	//draw(O--(1.8,1.55),invisible);
\end{asy}
\caption{刚体的运动}
\label{刚体的运动2}
\end{figure}

\label{定点转动矩阵与基点的选取无关}考虑刚体上的另外一点$O'$,则有
\begin{equation*}
	\mbf{r}_{O'} = \mbf{r}_O + \mbf{E}\mbf{U}\mbf{x}_{O'}
\end{equation*}
于是可有
\begin{equation}
	\mbf{r} = \mbf{r}_O + \mbf{E}\mbf{U}\mbf{x} = \mbf{r}_{O'} - \mbf{E}\mbf{U} \mbf{x}_{O'} + \mbf{E}\mbf{U} \mbf{x} = \mbf{r}_{O'} + \mbf{E} \mbf{U} (\mbf{x}-\mbf{x}_{O'})
	\label{定点转动矩阵与基点的选取无关1}
\end{equation}
将$O'$取为基点,各坐标系坐标轴的方向不变,则同样有刚体中任意一点的位矢可以表示为
\begin{equation}
	\mbf{r} = \mbf{r}_{O'} + \mbf{E} \mbf{U}' \mbf{x}'
	\label{定点转动矩阵与基点的选取无关2}
\end{equation}
对比式\eqref{定点转动矩阵与基点的选取无关1}和\eqref{定点转动矩阵与基点的选取无关2},并考虑到$\mbf{x}'=\mbf{x}-\mbf{x}_{O'}$,可有
\begin{equation*}
	\mbf{U} =\mbf{U}'
\end{equation*}
即,{\heiti 刚体定点转动矩阵与基点的选取无关}。

\subsection{刚体上各点的速度与加速度}
\label{节:刚体上各点的速度与加速度}
将式\eqref{刚体中任意一点的位矢}两端对时间$t$求导数,得
\begin{equation*}
	\mbf{v} = \mbf{v}_O + \mbf{E} \dot{\mbf{U}} \mbf{x} = \mbf{v}_O + \mbf{e} \mbf{U}^{\mathrm{T}} \dot{\mbf{U}} \mbf{x}
\end{equation*}
将式\eqref{正交矩阵的条件}两端对时间$t$求导数,得
\begin{equation*}
	\dot{\mbf{U}}^{\mathrm{T}} \mbf{U} + \mbf{U}^{\mathrm{T}} \dot{\mbf{U}} = \left(\mbf{U}^{\mathrm{T}} \dot{\mbf{U}}\right)^{\mathrm{T}} + \mbf{U}^{\mathrm{T}} \dot{\mbf{U}} = \mbf{0}
\end{equation*}
即矩阵$\mbf{U}^{\mathrm{T}} \dot{\mbf{U}}$是反对称矩阵,因此可以将其表示为
\begin{equation*}
	\mbf{U}^{\mathrm{T}} \dot{\mbf{U}} = \begin{pmatrix} 0 & -\omega_3 & \omega_2 \\ \omega_3 & 0 & -\omega_1 \\ -\omega_2 & \omega_1 & 0 \end{pmatrix}
\end{equation*}
由此,记矢量$\mbf{\omega} = \omega_1 \mbf{e}_1 + \omega_2 \mbf{e}_2 + \omega_3 \mbf{e}_3$称为刚体的{\heiti 角速度},则有
\begin{equation*}
	\mbf{e} \mbf{U}^{\mathrm{T}} \dot{\mbf{U}} \mbf{x} = \begin{pmatrix} \mbf{e}_1 & \mbf{e}_2 & \mbf{e}_3 \end{pmatrix} \begin{pmatrix} \omega_2 x_3 - \omega_3 x_2 \\ \omega_3 x_1 - \omega_1 x_3 \\ \omega_1 x_2 - \omega_2 x_1 \end{pmatrix} = \mbf{\omega} \times \mbf{r}'
\end{equation*}
所以刚体上任意一点的速度可以表示为
\begin{equation}
	\mbf{v} = \mbf{v}_O + \mbf{\omega} \times \mbf{r}' = \mbf{v}_O + \mbf{\omega} \times (\mbf{r}-\mbf{r}_O)
	\label{刚体上任意一点的速度}
\end{equation}
另取基点$O'$,则当以$O$为基点时,$O'$的速度为
\begin{equation*}
	\mbf{v}_{O'} = \mbf{v}_O + \mbf{\omega} \times (\mbf{r}_{O'} - \mbf{r}_O)
\end{equation*}
于是,以$O'$为基点表示刚体上各点的速度可有
\begin{align*}
	\mbf{v} = \mbf{v}_O + \mbf{\omega} \times (\mbf{r}-\mbf{r}_O) = \mbf{v}_{O'} + \mbf{\omega} \times (\mbf{r}-\mbf{r}_{O'})
\end{align*}
上式说明{\heiti 刚体的角速度与基点的选取无关}。

下面考虑刚体上各点的加速度。首先定义{\heiti 角加速度}为
\begin{equation}
	\mbf{\alpha} = \frac{\mathrm{d} \mbf{\omega}}{\mathrm{d} t} = \dot{\mbf{\omega}}
\end{equation}
将式\eqref{刚体上任意一点的速度}两端对时间$t$求导数,得
\begin{equation*}
	\mbf{a} = \mbf{a}_O + \dot{\mbf{\omega}} \times (\mbf{r}-\mbf{r}_O) + \mbf{\omega} \times (\mbf{v}-\mbf{v}_O)
\end{equation*}
由此即有刚体上任意一点的加速度
\begin{equation}
	\mbf{a} = \mbf{a}_O + \mbf{\alpha} \times (\mbf{r}-\mbf{r}_O) + \mbf{\omega} \times \left[\mbf{\omega} \times (\mbf{r}-\mbf{r}_O)\right]
	\label{刚体上任意一点的加速度}
\end{equation}

\subsection{刚体的定轴转动}

\begin{figure}[htb]
\centering
\begin{minipage}[t]{0.45\textwidth}
\centering
\begin{asy}
	texpreamble("\usepackage{xeCJK}");
	texpreamble("\setCJKmainfont{SimSun}");
	usepackage("amsmath");
	usepackage("amssymb");
	size(200);
	import graph;
	import math;
	//刚体的定轴转动
	
	pair O,i,j,k,ii,jj,P;
	real a,r,alpha;
	O = (0,0);
	a = 1;
	alpha = 3;
	i = (-a/sqrt(1+alpha**2),-a/sqrt(1+alpha**2));
	j = (a,0);
	k = (0,a);
	draw(Label("$X$",EndPoint),O--2*i,Arrow);
	draw(Label("$Y$",EndPoint),O--2*j,Arrow);
	draw(Label("$Z$",EndPoint),O--2*k,Arrow);
	label("$O$",O,W);
	path cir;
	cir = yscale(1/alpha)*scale(a)*unitcircle;
	draw(cir,dashed);
	ii = relpoint(cir,-0.16);
	jj = relpoint(cir,0.13);
	draw(Label("$x$",EndPoint),O--2*ii,Arrow);
	draw(Label("$y$",EndPoint),O--2*jj,Arrow);
	dot(1.0*k);
	label("$O_1$",1.0*k,W);
	draw(Label("$z$",EndPoint,W),O--1.7*k,Arrow);
	r = 0.3;
	real d2tan(pair de){
		return atan(alpha*de.y/de.x)/pi*180;
	}
	P = a*cos(pi/3)*i+a*sin(pi/3)*j+1.2*k;
	draw(Label("$\boldsymbol{r}$",MidPoint,Relative(W)),O--P,red,Arrow);
	draw(P--(P.x,-a/alpha*sqrt(1-P.x*P.x/a/a))--O,red+dashed);
	draw(P--(P-(P.x,-a/alpha*sqrt(1-P.x*P.x/a/a))),red+dashed);
	label("$P$",P,N);
	draw(Label("$\phi$",MidPoint,Relative(E)),yscale(1/alpha)*arc(O,r,d2tan(i)-180,d2tan(ii)),Arrow);
	draw(Label("$\phi$",MidPoint,E),yscale(1/alpha)*arc(O,r,d2tan(j),d2tan(jj)),Arrow);
\end{asy}
\caption{刚体的定轴转动}
\label{刚体的定轴转动}
\end{minipage}
\hspace{0.5cm}
\begin{minipage}[t]{0.45\textwidth}
\centering
\begin{asy}
	texpreamble("\usepackage{xeCJK}");
	texpreamble("\setCJKmainfont{SimSun}");
	usepackage("amsmath");
	usepackage("amssymb");
	size(200);
	import graph;
	import math;
	//向心加速度和切向加速度
	pair O,OO,i,j,k,P,tau,n;
	real r,phi;
	O = (0,0);
	i = (-sqrt(2)/4,-sqrt(14)/12);
	j = (sqrt(14)/4,-sqrt(2)/12);
	k = (0,2*sqrt(2)/3);
	pair cir(real theta){
		real xx,yy;
		xx = r*cos(theta);
		yy = r*sin(theta);
		return xx*i+yy*j;
	}
	r = 1;
	draw(graph(cir,0,2*pi));
	draw(Label("$z$",EndPoint),O--1.5*k,Arrow);
	draw(Label("$d$",MidPoint,Relative(W)),O--(-1*j),dashed);
	OO = -1.5*k;
	label("$O$",OO,W);
	draw(Label("$\boldsymbol{\omega}$",EndPoint,W),OO--k,linewidth(0.6bp),Arrow);
	draw(Label("$\boldsymbol{\alpha}$",EndPoint,W),OO--(-0.7*k),linewidth(0.6bp),Arrow);
	phi = 65*pi/180;
	P = r*cos(phi)*i+r*sin(phi)*j;
	tau = -sin(phi)*i+cos(phi)*j;
	n = -0.5*P;
	label("$P$",P,SE);
	draw(O--P,dashed);
	draw(Label("$\boldsymbol{r}$",MidPoint,Relative(E)),OO--P,linewidth(0.6bp),Arrow);
	draw(Label("$\boldsymbol{a}_\tau$",EndPoint,Relative(E)),P--P+0.7*tau,linewidth(0.6bp),Arrow);
	draw(Label("$\boldsymbol{v}$",EndPoint),P--P+tau,linewidth(0.6bp),Arrow);
	draw(Label("$\boldsymbol{a}_n$",EndPoint,Relative(W)),P--P+n,linewidth(0.6bp),Arrow);
	draw(Label("$\boldsymbol{a}$",EndPoint,Relative(W)),P--P+n+0.7*tau,linewidth(0.6bp),Arrow);
	draw(P+0.7*tau--P+0.7*tau+n--P+n,dashed);
\end{asy}
\caption{向心加速度和切向加速度}
\label{向心加速度和切向加速度}
\end{minipage}
\end{figure}

设刚体上有两个不动点$O$和$O_1$,则过$O$和$O_1$的直线即为转轴,平动系的$OZ$轴和本体系的$Oz$都沿着转轴。刚体在平动系中的位形由$OX$轴和$Ox$轴之间的夹角$\phi$确定,刚体上不在转动轴上的点沿着以转轴为圆心的圆周运动,设此时刚体上任意一点的矢径在本体系中的坐标为$\mbf{x}$,在平动系的坐标为$\mbf{X}$,则有
\begin{equation*}
	\mbf{X} = \mbf{U} \mbf{x}
\end{equation*}
此处转动矩阵为
\begin{equation}
	\mbf{U} = \begin{pmatrix} \cos \phi & -\sin \phi & 0 \\ \sin \phi & \cos \phi & 0 \\ 0 & 0 & 1 \end{pmatrix}
	\label{定轴转动的转动矩阵}
\end{equation}
直接计算可以验证
\begin{equation*}
	\mbf{U}^{\mathrm{T}} \dot{\mbf{U}} = \begin{pmatrix} 0 & -\dot{\phi} & 0 \\ \dot{\phi} & 0 & 0 \\ 0 & 0 & 0 \end{pmatrix}
\end{equation*}
因此角速度和角加速度为
\begin{equation*}
	\mbf{\omega} = \begin{pmatrix} 0 \\ 0 \\ \dot{\phi} \end{pmatrix} = \dot{\phi} \mbf{e}_3,\quad \mbf{\alpha} = \dot{\mbf{\omega}} = \begin{pmatrix} 0 \\ 0 \\ \ddot{\phi} \end{pmatrix} = \ddot{\phi} \mbf{e}_3
\end{equation*}
即角速度和角加速度都沿着转轴方向。由此根据式\eqref{刚体上任意一点的速度}可得刚体上任意一点的速度
\begin{equation*}
	\mbf{v} = \mbf{\omega} \times \mbf{r}
\end{equation*}
根据式\eqref{刚体上任意一点的加速度}可得刚体上任意一点的加速度
\begin{equation*}
	\mbf{a} = \mbf{\alpha} \times \mbf{r} + \mbf{\omega} \times (\mbf{\omega} \times \mbf{r})
\end{equation*}
此处的加速度$\mbf{a}$可以分解为向心加速度$\mbf{a}_n = \mbf{\omega} \times (\mbf{\omega} \times \mbf{r})$和切向加速度$\mbf{a}_\tau = \mbf{\alpha} \times \mbf{r}$。

\begin{figure}[htb]
\centering
\begin{asy}
	texpreamble("\usepackage{xeCJK}");
	texpreamble("\setCJKmainfont{SimSun}");
	usepackage("amsmath");
	usepackage("amssymb");
	size(300);
	import graph;
	import math;
	//定轴转动公式
	real r,theta,a,h;
	r = 1;
	a = 0.5;
	theta = 20;
	h = 3;
	path p;
	p = rotate(-theta)*shift((0,h))*scale(r)*yscale(a)*unitcircle;
	draw(p,linewidth(0.8bp));
	draw(Label("$N$",EndPoint,Relative(W)),(0,0)--h*dir(90-theta),Arrow);
	draw(h*dir(90-theta)--(h+1)*dir(90-theta));
	draw(Label("$\boldsymbol{n}$",EndPoint,Relative(W)),(0,0)--dir(90-theta),Arrow);
	label("$(\boldsymbol{n}\cdot\boldsymbol{r})\boldsymbol{n}$",0.6*h*dir(90-theta),dir(180-theta));
	pair NP,P,Q;
	NP = h*dir(90-theta);
	P = relpoint(p,0);
	Q = relpoint(p,0.9);
	draw(Label("$P$",EndPoint),NP--P,Arrow);
	draw(Label("$Q$",EndPoint),NP--Q,Arrow);
	draw(Label("$\boldsymbol{r}$",Relative(0.7),Relative(E)),(0,0)--P,Arrow);
	draw(Label("$\boldsymbol{r}'$",Relative(0.7),Relative(W)),(0,0)--Q,Arrow);
	real R = 0.3;
	draw(Label("$\phi$",MidPoint,Relative(W)),arc(NP,R,degrees(P-NP),degrees(Q-NP)),Arrow);
	draw(Label("$V$",BeginPoint),NP+0.7*(P-NP)--Q,Arrow);
	picture pic;
	r = 1.3;
	path q;
	q = scale(r)*unitcircle;
	pair P,Q;
	draw(pic,q,linewidth(0.8bp));
	P = relpoint(q,0);
	Q = relpoint(q,0.86);
	draw(pic,Label("$P$",EndPoint),(0,0)--P,Arrow);
	draw(pic,Label("$Q$",EndPoint),(0,0)--Q,Arrow);
	draw(pic,Label("$\boldsymbol{r}\times\boldsymbol{n}$",MidPoint,Relative(E)),(0,0)--r*dir(-90),Arrow);
	draw(pic,Label("$\phi$",MidPoint,Relative(W)),arc((0,0),R,0,degrees(Q)-360),Arrow);
	draw(pic,Label("$V$",BeginPoint),(Q.x,0)--Q,Arrow);
	label(pic,"$N$",(0,0),NW);
	add(shift((2.6+r,r+0.5))*pic);
\end{asy}
\caption{定轴转动公式}
\label{定轴转动公式}
\end{figure}

设转轴方向为$\mbf{n}$($\mbf{n}$是单位向量),转动角为$\phi$(顺时针方向),则根据图\ref{定轴转动公式}中的几何关系可有
\begin{align*}
	\mbf{r}' & = \overrightarrow{ON} + \overrightarrow{NV} + \overrightarrow{VQ} \\
	& = (\mbf{n}\cdot \mbf{r})\mbf{n} + \big[\mbf{r}-(\mbf{n}\cdot\mbf{r})\mbf{n}\big] \cos \phi + (\mbf{r}\times \mbf{n})\sin \phi
\end{align*}
由此可得{\heiti Rodrigues公式}
\begin{equation}
	\mbf{r}' = \mbf{r}\cos \phi + (\mbf{n}\cdot \mbf{r})\mbf{n}(1-\cos \phi) + (\mbf{r}\times \mbf{n})\sin \phi
	\label{Rodrigues公式}
\end{equation}

\subsection{刚体定点转动的Euler角}

刚体定点转动的自由度$s_r = 3$,可用$3$个广义坐标确定本体系相对平动系的取向。

\begin{figure}[htb]
\centering
\begin{asy}
	texpreamble("\usepackage{xeCJK}");
	texpreamble("\setCJKmainfont{SimSun}");
	usepackage("amsmath");
	usepackage("amssymb");
	size(350);
	import graph;
	import math;
	//刚体定点转动的Euler角
	real covert(pair x,real a){
		if (x.x==0) return 90;
		else return atan(a*x.y/x.x)*180/pi;
	}
	picture elpic1,tmp;
	pair O1,O,x[],y[],z[];
	real a,alpha1,alpha2,beta,gamma;
	a = 2;
	O1 = (0,0);
	O = (0,0);
	x[1] = (-a/sqrt(5),-a/sqrt(5));
	y[1] = (a,0);
	z[1] = (0,a);
	//第一层
	draw(elpic1,Label("$X$",EndPoint),O--x[1],blue,Arrow);
	draw(elpic1,Label("$Y$",EndPoint),O--y[1],blue,Arrow);
	draw(elpic1,Label("$Z$",EndPoint),O--z[1],blue,Arrow);
	label(elpic1,"$O$",O,W);
	path cir[],clp;
	cir[1] = yscale(1/2)*scale(a)*unitcircle;
	draw(elpic1,cir[1],blue);
	alpha1 = 70;
	alpha2 = 10;
	real k;
	k = tan(-alpha1*pi/180);
	x[2] = (a/sqrt(1+4*k*k),a*k/sqrt(1+4*k*k));
	k = tan(alpha2*pi/180);
	y[2] = (a/sqrt(1+4*k*k),a*k/sqrt(1+4*k*k));
	z[2] = z[1];
	draw(elpic1,Label("$N$",EndPoint),-x[2]--x[2],Arrow);
	draw(elpic1,Label("$L$",EndPoint),O--y[2],Arrow);
	//第二层
	picture elpic2;
	beta = 35;
	x[3] = x[2];
	y[3] = rotate(beta)*y[1];
	z[3] = rotate(0.9*beta)*z[1];
	draw(elpic2,Label("$M$",EndPoint),O--y[3],Arrow);
	draw(elpic2,Label("$z$",EndPoint),O--z[3],red,Arrow);
	real k1,k2,ra,b;
	k1 = -tan(alpha1*pi/180);
	k2 = -tan((alpha1+beta)*pi/180);
	ra = sqrt((1+4*k1^2)*(1+k2^2)/(1+k1^2)-1);
	b = k2*a/ra;
	cir[2] = rotate(beta)*yscale(b)*xscale(a)*unitcircle;
	clp = 1.1*x[3]--1.1*x[3]+1.1*y[3]--(-1.1*x[3]+1.1*y[3])--(-1.1*x[3])--cycle;
	draw(tmp,cir[2],red);
	clip(tmp,clp);
	add(elpic2,tmp);
	erase(tmp);
	draw(tmp,cir[2],red+dashed);
	unfill(tmp,clp);
	add(elpic2,tmp);
	erase(tmp);
	//第一层被第二层遮住的部分
	unfill(elpic1,cir[2]);
	draw(tmp,Label("$X$",EndPoint),O--x[1],blue+dashed,Arrow);
	draw(tmp,Label("$Y$",EndPoint),O--y[1],blue+dashed,Arrow);
	draw(tmp,Label("$Z$",EndPoint),O--z[1],blue,Arrow);
	draw(tmp,Label("$N$",EndPoint),-x[2]--x[2],Arrow);
	draw(tmp,Label("$L$",EndPoint),O--y[2],dashed,Arrow);
	label(tmp,"$O$",O,W);
	draw(tmp,cir[1],blue+dashed);
	clip(tmp,cir[2]);
	clip(tmp,clp);
	add(elpic1,tmp);
	erase(tmp);
	draw(tmp,Label("$X$",EndPoint),O--x[1],blue,Arrow);
	draw(tmp,Label("$Y$",EndPoint),O--y[1],blue,Arrow);
	draw(tmp,Label("$Z$",EndPoint),O--z[1],blue,Arrow);
	draw(tmp,Label("$N$",EndPoint),-x[2]--x[2],Arrow);
	draw(tmp,Label("$L$",EndPoint),O--y[2],dashed,Arrow);
	draw(tmp,Label("$Z$",EndPoint),O--z[2],dashed,Arrow);
	label(tmp,"$O$",O,W);
	draw(tmp,cir[1],blue);
	clip(tmp,cir[2]);
	unfill(tmp,clp);
	add(elpic1,tmp);
	erase(tmp);
	//画自转
	x[4] = relpoint(cir[2],0.78);
	y[4] = relpoint(cir[2],0.05);
	z[4] = z[3];
	draw(elpic2,Label("$x$",EndPoint),O--x[4],red,Arrow);
	draw(elpic2,Label("$y$",EndPoint),O--y[4],red,Arrow);
	add(shift(O1)*elpic1);
	add(shift(O1)*elpic2);
	picture elpic4;
	//画第一层上的角度
	real theta1,theta2,r;
	r = 0.4;
	theta1 = covert(x[1],2)+180;
	theta2 = covert(x[2],2)+360;
	draw(elpic4,Label("$\phi$",MidPoint,Relative(E)),yscale(1/2)*arc(O,r,theta1,theta2),Arrow);
	theta1 = covert(y[1],2);
	theta2 = covert(y[2],2);
	r = 0.7;
	draw(elpic4,Label("$\phi$",MidPoint,E),yscale(1/2)*arc(O,r,theta1,theta2),Arrow);
	//画第二层上的角度
	r = 1.3;
	theta1 = covert(y[2],0.9);
	theta2 = covert(y[3],0.9);
	draw(elpic4,Label("$\theta$",MidPoint,Relative(E)),xscale(0.9)*arc(O,r,theta1,theta2),Arrow);
	theta1 = covert(z[2],0.9);
	theta2 = covert(z[3],0.9);
	draw(elpic4,Label("$\theta$",MidPoint,Relative(E)),xscale(0.9)*arc(O,r,theta1,180+theta2),Arrow);
	//画自转角
	r = 0.3;
	//theta1 = covert(x[2],a/b);
	//theta2 = covert(x[4],a/b);
	draw(elpic4,Label("$\psi$",MidPoint,Relative(E)),arc(O,r,degrees(x[3]),degrees(x[4])),Arrow);
	r = 0.4;
	draw(elpic4,Label("$\psi$",MidPoint,Relative(E)),arc(O,r,degrees(y[3]),degrees(y[4])),Arrow);
	add(shift(O1)*elpic4);
\end{asy}
\caption{刚体定点转动的Euler角}
\label{刚体定点转动的Euler角}
\end{figure}

如图\ref{刚体定点转动的Euler角}所示,其中$XOY$平面与$xOy$平面的交线$ON$称为{\heiti 节线}。三个Euler角分别为{\heiti 进动(precession)角}$\,\,\phi$,{\heiti 章动(nutation)角}$\,\,\theta$和{\heiti 自转(spin)角}$\,\,\psi$。

三个Euler角是相互独立的,可以任意取值。如果给定三个角的值,则唯一地确定了刚体定点转动位形,通常假设$0 \leqslant \phi < 2\pi, \,0 \leqslant \theta \leqslant \pi,\, 0 \leqslant \psi < 2\pi$。刚体的连续转动表现为Euler角随时间连续变化。

从平动系$OXYZ$到本体系$Oxyz$的转换可以通过下面的三个按顺序的转动实现:
\begin{enumerate}
	\item 绕$OZ$轴旋转$\phi$角,称为{\heiti 进动};
	\item 绕$ON$旋转$\theta$角,称为{\heiti 章动};
	\item 绕$Oz$轴旋转$\psi$角,称为{\heiti 自转}。
\end{enumerate}
如果从转轴的顶端看,所有的转动都是逆时针方向。即定点转动可以由上面的进动、章动和自转运动来合成。由于刚体的有限定轴转动是不可交换的,因此上面的三种转动的顺序也是不可交换的。

根据定轴转动的转动矩阵,可以得到进动、章动和自转的旋转变换矩阵分别为
\begin{equation*}
	\mbf{U}_1 = \begin{pmatrix} \cos \phi & -\sin \phi & 0 \\ \sin \phi & \cos \phi & 0 \\ 0 & 0 & 1 \end{pmatrix},\quad \mbf{U}_2 = \begin{pmatrix} 1 & 0 & 0 \\ 0 & \cos \theta & -\sin \theta \\ 0 & \sin \theta & \cos \theta \end{pmatrix},\quad \mbf{U}_3 = \begin{pmatrix} \cos \psi & -\sin \psi & 0 \\ \sin \psi & \cos \psi & 0 \\ 0 & 0 & 1 \end{pmatrix}
\end{equation*}
因此,定点转动的旋转变换矩阵为
\begin{align*}
	\mbf{U} & = \mbf{U}_1 \mbf{U}_2 \mbf{U}_3 \\
	& = \begin{pmatrix}
		\cos \phi \cos \psi - \sin \phi \cos \theta \sin \psi & -\cos \phi \sin \psi - \sin \phi \cos \theta \cos \psi & \sin \phi \sin \theta \\
		\sin \phi \cos \psi + \cos \phi \cos \theta \sin \psi & -\sin \phi \sin \psi + \cos \phi \cos \theta \cos \psi & -\cos \phi \sin \theta \\
		\sin \theta \sin \psi & \sin \theta \cos \psi & \cos \theta
	\end{pmatrix}
\end{align*}
直接求$\mbf{U}^{\mathrm{T}} \dot{\mbf{U}}$的运算十分复杂,考虑到$\mbf{U}_1,\mbf{U}_2,\mbf{U}_3$都是正交矩阵,因此有
\begin{equation*}
	\mbf{U}_1^{\mathrm{T}} \dot{\mbf{U}}_1 = \begin{pmatrix} 0 & -\dot{\phi} & 0 \\ \dot{\phi} & 0 & 0 \\ 0 & 0 & 0 \end{pmatrix},\quad \mbf{U}_2^{\mathrm{T}} \dot{\mbf{U}}_2 = \begin{pmatrix} 0 & 0 & 0 \\ 0 & 0 & -\dot{\theta} \\ 0 & \dot{\theta} & 0 \end{pmatrix},\quad \mbf{U}_3^{\mathrm{T}} \dot{\mbf{U}}_3 = \begin{pmatrix} 0 & -\dot{\psi} & 0 \\ \dot{\psi} & 0 & 0 \\ 0 & 0 & 0 \end{pmatrix}
\end{equation*}
故有
\begin{align*}
	\mbf{U}^{\mathrm{T}} \dot{\mbf{U}} & = \mbf{U}_3^{\mathrm{T}} \mbf{U}_2^{\mathrm{T}} \mbf{U}_1^{\mathrm{T}} (\dot{\mbf{U}}_1 \mbf{U}_2 \mbf{U}_3 + \mbf{U}_1 \dot{\mbf{U}}_2 \mbf{U}_3 + \mbf{U}_1 \mbf{U}_2 \dot{\mbf{U}}_3) \\
	& = \mbf{U}_3^{\mathrm{T}} \mbf{U}_2^{\mathrm{T}} \mbf{U}_1^{\mathrm{T}} \dot{\mbf{U}}_1 \mbf{U}_2 \mbf{U}_3 + \mbf{U}_3^{\mathrm{T}} \mbf{U}_2^{\mathrm{T}} \dot{\mbf{U}}_2 \mbf{U}_3 + \mbf{U}_1^{\mathrm{T}} \dot{\mbf{U}}_1 \\
	& = \begin{pmatrix}
		0 & -\dot{\phi} \cos \theta - \dot{\psi} & \dot{\phi} \sin \theta \cos \psi - \dot{\theta} \sin \psi \\
		\dot{\phi} \cos \theta + \dot{\psi} & 0 & -\dot{\phi} \sin \theta \sin \psi - \dot{\theta} \cos \psi \\
		- \dot{\phi} \sin \theta \cos \psi + \dot{\theta} \sin \psi & \dot{\phi} \sin \theta \sin \psi + \dot{\theta} \cos \psi & 0
	\end{pmatrix}
\end{align*}
因此有
\begin{equation}
	\begin{cases}
		\omega_1 = \dot{\phi} \sin \theta \sin \psi + \dot{\theta} \cos \psi \\
		\omega_2 = \dot{\phi} \sin \theta \cos \psi - \dot{\theta} \sin \psi \\
		\omega_3 = \dot{\phi} \cos \theta + \dot{\psi}
	\end{cases}
	\label{Euler运动学方程}
\end{equation}
式\eqref{Euler运动学方程}称为{\heiti Euler运动学方程},它给出了刚体在平动系中的角速度在本体系内的分量与广义坐标(即Euler角)、广义速度(即Euler角的导数)之间的关系。

\begin{example}
半径为$a$的圆盘垂直于地面作纯滚动,圆盘中心$C$以速率$v_C = \omega_1 R$沿着半径为$R$的圆周运动,求圆盘边缘上任意一点$P$的速度。
\end{example}
\begin{solution}
\begin{figure}[htb]
\centering
\begin{asy}
	texpreamble("\usepackage{xeCJK}");
	texpreamble("\setCJKmainfont{SimSun}");
	usepackage("amsmath");
	usepackage("amssymb");
	size(350);
	import graph;
	import math;
	//第六章例2图
	picture tmp;
	pair O,i,j,k;
	real R,a,phi;
	O = (0,0);
	i = (-sqrt(2)/4,-sqrt(14)/12);
	j = (sqrt(14)/4,-sqrt(2)/12);
	k = (0,2*sqrt(2)/3);
	R = 2.5;
	a = 1;
	phi = 5/6*pi;
	pair cir(real theta){
		return R*cos(theta)*i+R*sin(theta)*j;
	}
	pair pln(real theta){
		real x,y,z;
		x = R*cos(phi)+a*cos(theta)*sin(phi);
		y = R*sin(phi)-a*cos(theta)*cos(phi);
		z = a+a*sin(theta);
		return x*i+y*j+z*k;
	}
	real inner(pair v1,pair v2){
		return v1.x*v2.x+v1.y*v2.y;
	}
	path plate;
	draw(Label("$Z$",EndPoint),O--3*a*k,Arrow);
	draw(graph(cir,0,2*pi),dashed);
	plate = graph(pln,0,2*pi)--cycle;
	draw(plate,linewidth(0.8bp));
	pair OO,C,RC,P,yy,dpsi,dphi,omega;
	real x1,x2;
	OO = a*k;
	C = R*cos(phi)*i+R*sin(phi)*j+a*k;
	RC = R*cos(phi)*i+R*sin(phi)*j;
	P = pln(2/3*pi);
	yy = pln(2/3*pi-pi/2);
	dot(O);
	label("$O$",OO,SE);
	dot(C);
	label("$C$",C,S);
	draw(OO--C);
	//draw(Label("$y$",EndPoint),C--yy,Arrow);
	draw(Label("$z$",EndPoint),C--interp(OO,C,-0.8),Arrow);
	dot(P);
	label("$P$",P,NW);
	draw(Label("$a$",MidPoint,Relative(W)),C--P,dashed);
	draw(Label("$x$",EndPoint),OO--OO+(P-C),Arrow);
	draw(OO+(P-C)--P,dashed);
	draw(Label("$\boldsymbol{r}_P$",MidPoint,Relative(E),black),OO--P,blue,Arrow);
	dphi = k;
	dpsi = dir(OO-C);
	omega = dir(RC-OO);
	x1 = -(inner(dpsi,dpsi)*inner(omega,dphi)-inner(dphi,dpsi)*inner(omega,dpsi))/(inner(dphi,dphi)*inner(dpsi,dpsi)-inner(dphi,dpsi)**2);
	x2 = -(inner(dphi,dphi)*inner(omega,dpsi)-inner(dphi,dpsi)*inner(omega,dphi))/(inner(dphi,dphi)*inner(dpsi,dpsi)-inner(dphi,dpsi)**2);
	dphi = x1*dphi;
	dpsi = x2*dpsi;
	draw(RC--OO,dashed);
	draw(Label("$\dot{\boldsymbol{\psi}}$",EndPoint,Relative(W),black),OO--OO+dpsi,red,Arrow);
	draw(Label("$\dot{\boldsymbol{\phi}}$",EndPoint,Relative(E),black),OO--OO+dphi,red,Arrow);
	draw(Label("$\boldsymbol{\omega}$",EndPoint,black),OO--OO+dphi+dpsi,red,Arrow);
	draw(OO+dpsi--OO+dpsi+dphi--OO+dphi,dashed);
	draw(Label("$R$",MidPoint,Relative(E)),O--RC,dashed);
	label("$Q$",RC,dir(70));
	label("$O'$",O,W);
	draw(C--pln(0),dashed);
	a = 0.3;
	draw(Label("$\psi$",MidPoint,Relative(E)),shift((1-a)*k)*graph(pln,0,2/3*pi),Arrow);
	draw(Label("$X$",EndPoint),O--3*i,Arrow);
	draw(Label("$Y$",EndPoint),O--3*j,Arrow);
	R = 0.4;
	draw(Label("$\phi$",MidPoint,Relative(E)),graph(cir,0,phi),Arrow);
\end{asy}
\caption{例\theexample}
\label{第六章例2图}
\end{figure}

{\heiti 方法一}:如图\ref{第六章例2图}所示,圆盘在转动过程中,盘上各点和$O$点的相对距离保持不变,因此圆盘的运动可以看成是绕$O$点的定点转动。而纯滚动条件要求$Q$点的瞬时速度为零,因此$OQ$为瞬时转轴,即角速度沿此方向。由此可得圆盘绕$O$的角速度(进动)$\omega_1$与圆盘自转的角速度$\omega_2$之间需要满足
\begin{equation*}
	\frac{\omega_1}{\omega_2} = \frac{a}{R}
\end{equation*}
由此可有
\begin{equation*}
	\omega_2 = \frac{R}{a} \omega_1
\end{equation*}
由此即有
\begin{equation*}
	\mbf{\omega} = \omega_1 (\sin \psi \mbf{e}_x + \cos \psi \mbf{e}_y) - \omega_2 \mbf{e}_z = \omega_1 (\sin \psi \mbf{e}_x + \cos \psi \mbf{e}_y) + \frac{R}{a} \omega_1 \mbf{e}_z
\end{equation*}
再考虑到$\mbf{r}_P = a\mbf{e}_x+R\mbf{e}_z$,此处的$\mbf{e}_x,\mbf{e}_y,\mbf{e}_z$为本体系中的基\footnote{此处$y$轴的方向需通过右手规则来确定以保证本体系为右手系。}。综上,即有
\begin{equation*}
	\mbf{v}_P = \mbf{\omega} \times \mbf{r}_P = \omega_1 R \cos \psi \mbf{e}_x - \omega_1 R(1+\sin \psi) \mbf{e}_y + \omega_1 a \cos \psi \mbf{e}_z
\end{equation*}

\noindent {\heiti 方法二}:也可以用确定Euler角的方法通过Euler运动学方程直接得到角速度。进动角$\phi$即为$O'Q$与平动系(此时与空间系重合)$X$轴之间的夹角,章动角$\theta = \dfrac{\pi}{2}$恒定不变,自转角$\psi$即为圆盘上水平线到本体系$Ox$轴的角。据此,根据Euler运动学方程\eqref{Euler运动学方程}可得圆盘的角速度$\mbf{\omega}$在本体系中的坐标为
\begin{equation*}
	\begin{cases}
		\omega_1 = \dot{\phi} \sin \psi \\
		\omega_2 = \dot{\phi} \cos \psi \\
		\omega_3 = \dot{\psi}
	\end{cases}
\end{equation*}
即有
\begin{equation*}
	\mbf{\omega} = \dot{\phi} \sin \psi \mbf{e}_x + \dot{\phi} \cos \psi \mbf{e}_y + \dot{\psi} \mbf{e}_z
\end{equation*}
现在需要得到$\dot{\phi},\dot{\psi}$与$\omega_1$的关系,考虑$\mbf{r}_C = -R\mbf{e}_z$,则有
\begin{equation*}
	\mbf{v}_C = \mbf{\omega} \times \mbf{r}_C = -R\dot{\phi} \cos \psi \mbf{e}_x + R\dot{\phi} \sin \psi \mbf{e}_y
\end{equation*}
易知$\mbf{v}_C$的方向沿$Q$点处的切线方向,并可得$\dot{\phi}$,其大小为
\begin{equation*}
	R\dot{\phi} = \omega_1 R
\end{equation*}
即有$\dot{\phi} = \omega_1$。再考虑$\mbf{r}_Q = -a\sin \psi \mbf{e}_x - a\cos \psi \mbf{e}_y - R\mbf{e}_z$,则有
\begin{equation*}
	\mbf{v}_Q = \mbf{\omega} \times \mbf{r}_Q = (a\dot{\psi}-R\dot{\phi}) \cos \psi \mbf{e}_x - (a\dot{\psi}-R\dot{\phi}) \sin \psi \mbf{e}_y
\end{equation*}
纯滚动条件要求$\mbf{v}_Q = \mbf{0}$,即可得
\begin{equation*}
	\dot{\psi} = \frac{R}{a} \dot{\phi} = \frac{R}{a} \omega_1
\end{equation*}
由此可得圆盘边缘上任意一点的速度
\begin{equation*}
	\mbf{v}_P = \mbf{\omega} \times \mbf{r}_P = \omega_1 R \cos \psi \mbf{e}_x - \omega_1 R(1+\sin \psi) \mbf{e}_y + \omega_1 a \cos \psi \mbf{e}_z
\end{equation*}
\end{solution}

\section{惯量矩阵}

\subsection{刚体定点转动的角动量、动能与转动惯量}

在平动系中,刚体绕$O$点定点转动。设刚体任意一点的位矢为$\mbf{r}$,则其在平动系中的速度为$\dot{\mbf{r}} = \mbf{\omega} \times \mbf{r}$,由此角动量可计算为\footnote{此式可通过先将刚体分割为$n$块,作为质点系处理,其角动量为\begin{equation*} \mbf{L} = \sum_{i=1}^n \mbf{r}_i \times \Delta m_i \dot{\mbf{r}}_i = \sum_{i=1}^n \mbf{r}_i \times \rho_i \dot{\mbf{r}}_i \Delta V_i \end{equation*}然后令$n$趋于无穷,根据重积分的定义即得。}
\begin{align*}
	\mbf{L} = \int_V \mbf{r} \times \rho \dot{\mbf{r}} \mathrm{d} V = \int_V \rho \mbf{r} \times (\mbf{\omega} \times \mbf{r}) \mathrm{d} V = \int_V \rho \big[r^2 \mbf{\omega} - (\mbf{r} \cdot \mbf{\omega}) \mbf{r} \big] \mathrm{d} V
\end{align*}
此处积分遍及刚体所在的空间,$\rho$为刚体的密度。将上式向任意直角座标系投影,即
\begin{align*}
	L_1 & = \mbf{L} \cdot \mbf{e}_1 = \int_V \rho \big[r^2 \mbf{\omega} \cdot \mbf{e}_1 - (\mbf{r} \cdot \mbf{\omega}) (\mbf{r} \cdot \mbf{e}_1) \big] \mathrm{d} V \\
	& = \int_V \rho \big[(x^2+y^2+z^2) \omega_1 - (x \omega_1 + y\omega_2 + z\omega_3) x \big] \mathrm{d} V \\
	& = \int_V \rho \big[(y^2+z^2)\omega_1 - xy \omega_2 - xz \omega_3\big] \mathrm{d} V \\
	L_2 & = \int_V \rho \big[-xy \omega_1 + (x^2+z^2)\omega_2 - yz \omega_3\big] \mathrm{d} V \\
	L_3 & = \int_V \rho \big[-xz \omega_1 - yz \omega_2 + (x^2+y^2)\omega_3\big] \mathrm{d} V
\end{align*}
由此,角动量矢量的分量可以表示为
\begin{equation}
	\begin{pmatrix} L_1 \\ L_2 \\ L_3 \end{pmatrix} = \begin{pmatrix} I_{11} & I_{12} & I_{13} \\ I_{21} & I_{22} & I_{23} \\ I_{31} & I_{32} & I_{33} \end{pmatrix} \begin{pmatrix} \omega_1 \\ \omega_2 \\ \omega_3 \end{pmatrix}
\end{equation}
其中
\begin{equation}
	I_{ij} = \int_V \rho \left(\delta_{ij}\sum_{k=1}^3 x_k^2-x_ix_j\right) \mathrm{d}V,\quad x_1 \equiv x,\,x_2 \equiv y,\,x_3 \equiv z
\end{equation}
称为刚体对$O$的{\heiti 惯量元素},矩阵
\begin{equation}
	\mbf{I} = \begin{pmatrix} I_{11} & I_{12} & I_{13} \\ I_{21} & I_{22} & I_{23} \\ I_{31} & I_{32} & I_{33} \end{pmatrix}
\end{equation}
称为刚体对$O$的{\heiti 惯量矩阵}。由此,角动量可以表示为
\begin{equation}
	\mbf{L} = \mbf{I} \mbf{\omega}
	\label{惯量矩阵与角动量}
\end{equation}

刚体的动能可以表示为\footnote{对于刚体,角动量矢量是与空间位置无关的,否则刚体将发生变形。}
\begin{equation*}
	T = \frac12 \int_V \rho \dot{\mbf{r}} \cdot \dot{\mbf{r}} \mathrm{d} V = \frac12 \int_V \rho \dot{\mbf{r}} \cdot (\mbf{\omega} \times \mbf{r}) \mathrm{d} V = \frac12 \int_V \mbf{\omega} \cdot (\mbf{r} \times \rho \dot{\mbf{r}}) \mathrm{d} L = \frac12 \mbf{\omega} \cdot \mbf{L}
\end{equation*}
考虑到式\eqref{惯量矩阵与角动量}可有
\begin{equation}
	T = \frac12 \mbf{\omega}^{\mathrm{T}} \mbf{I} \mbf{\omega}
\end{equation}
动能的正定性表明惯量矩阵$\mbf{I}$是正定矩阵。再考虑到惯量矩阵的定义,可知惯量矩阵$\mbf{I}$是对称正定矩阵。

在本体系中,刚体静止,因此在本体系中惯量元素为常量。

\subsection{平行轴定理}

设对$O$点,惯量元素为
\begin{equation*}
	I_{ij} = \int_V \rho \left(\delta_{ij}\sum_{k=1}^3 x_k^2-x_ix_j\right) \mathrm{d} V
\end{equation*}
对质心$C$,惯量元素为
\begin{equation*}
	I'_{ij} = \int_V \rho \left(\delta_{ij}\sum_{k=1}^3 x'_k{}^2-x'_ix'_j\right) \mathrm{d} V
\end{equation*}
设刚体质心的坐标为$X_i$,则有
\begin{equation*}
	x_i = X_i + x'_i
\end{equation*}
再考虑到质心坐标满足\footnote{此式可通过先将刚体分割为$n$块,作为质点系处理,然后令$n$趋于无穷,根据重积分的定义即得。}
\begin{equation*}
	\int_V \rho x_i \mathrm{d} V = MX_i
\end{equation*}
由此可得{\heiti 平行轴定理}
\begin{equation}
	\mbf{I} = \mbf{I}_C + \mbf{I}'
\end{equation}
其中
\begin{equation}
	I_{Cij} = M\left(\delta_{ij}\sum_{k=1}^3 X_k^2-X_iX_j\right)
\end{equation}
即为总质量集中于质心的质点对$O$的惯量元素。

平行轴定理给出了对质心$C$点的惯量矩阵与对任意点$O$的惯量矩阵之间的关系。

\subsection{会聚轴定理}

角速度表示可以为$\mbf{\omega} = \omega\mbf{n}$,此处$\mbf{n}$为瞬时转轴方向单位矢量。由此可得角动量轴向投影分量为
\begin{equation*}
	L_n = \mbf{n} \cdot \mbf{L} = \int_V \rho \mbf{r} \times (\mbf{\omega} \times \mbf{r}) \cdot \mbf{n} \mathrm{d} V = \int_V \rho (\mbf{n} \times \mbf{r}) \cdot (\mbf{n} \times \mbf{r}) \omega \mathrm{d} V
\end{equation*}
即有
\begin{equation}
	L_n = I_n \omega
\end{equation}
此处
\begin{equation}
	I_n = \int_V \rho |\mbf{n} \times \mbf{r}|^2 \mathrm{d} V
\end{equation}
为对$\mbf{n}$轴的转动惯量,$|\mbf{n} \times \mbf{r}|$即为矢径$\mbf{r}$末端点到转轴的距离。

设在某坐标系下,$\mbf{n}$和$\mbf{L}$的分量可以表示为
\begin{equation*}
	\mbf{n} = \begin{pmatrix} n_1 \\ n_2 \\ n_3 \end{pmatrix},\quad \mbf{L} = \begin{pmatrix} L_1 \\ L_2 \\ L_3 \end{pmatrix}
\end{equation*}
因此有
\begin{equation*}
	L_n = \mbf{n}^{\mathrm{T}} \mbf{L} = \mbf{n}^{\mathrm{T}} \mbf{I} \mbf{\omega} = \mbf{n}^{\mathrm{T}} \mbf{I} \mbf{n} \omega
\end{equation*}
即有
\begin{equation}
	I_n = \mbf{n}^{\mathrm{T}} \mbf{I} \mbf{n}
	\label{会聚轴定理}
\end{equation}
式\eqref{会聚轴定理}表示的关系即称为{\heiti 会聚轴定理}。此时,刚体的动能也可以简单的表示为
\begin{equation}
	T = \frac12 \mbf{\omega}^{\mathrm{T}} \mbf{I} \mbf{\omega} = \frac12 I_n \omega^2
\end{equation}

对轴的转动惯量一般与取向有关,是各向异性的。

\subsection{惯量主轴与主轴系}

刚体的动能是标量,因此是坐标变换的不变量。在不同的坐标系中,角动量的坐标之间将满足
\begin{equation*}
	\mbf{\omega} = \mbf{U} \mbf{\omega}'
\end{equation*}
因此,动能的形式将变为
\begin{equation*}
	T = \frac12 \mbf{\omega}^{\mathrm{T}} \mbf{I} \mbf{\omega} = \frac12 \mbf{\omega}'^{\mathrm{T}} \mbf{U}^{\mathrm{T}} \mbf{I} \mbf{U} \mbf{\omega}' = T' = \frac12 \mbf{\omega}'^{\mathrm{T}} \mbf{I}' \mbf{\omega}'
\end{equation*}
因此,在不同的坐标系中,惯量矩阵将满足
\begin{equation*}
	\mbf{I}' = \mbf{U}^{\mathrm{T}} \mbf{I} \mbf{U}
\end{equation*}
即,在不同的坐标系中,惯量矩阵是不同的。而能够使得惯量矩阵$\mbf{I}$为对角矩阵的坐标系称为{\heiti 主轴系},其坐标轴称为{\heiti 惯量主轴},对相应主轴的转动惯量称为{\heiti 主惯量}。

今后,我们约定本体系一律取为主轴系。在主轴系中,惯量矩阵为
\begin{equation*}
	\mbf{I} = \begin{pmatrix} I_1 & & \\ & I_2 & \\ & & I_3 \end{pmatrix}
\end{equation*}
此时,角动量和动能分别为
\begin{equation*}
	\mbf{L} = \sum_{i=1}^3 I_i \omega_i \mbf{e}_i,\quad T = \frac12 \sum_{i=1}^3 I_i \omega_i^2
\end{equation*}
惯量矩阵的正定性保证主惯量都是正值,即$I_i>0,\,\,i=1,2,3$。

\subsection{主轴系的求法}

考虑角速度沿主轴方向,即$\mbf{\omega} = \omega \mbf{e}_j$时,则
\begin{equation*}
	\mbf{L} = \mbf{I} \mbf{\omega} = \mbf{I} \omega \mbf{e}_j = I_j \omega \mbf{e}_j
\end{equation*}
由此可得
\begin{equation*}
	\mbf{I} \mbf{e}_j = I_j \mbf{e}_j
\end{equation*}
即惯量主轴为惯量矩阵的本征矢量,相应的本征值为主惯量\footnote{实际上,根据矩阵正交相似对角化的条件,可以直接得出此结论。}。

\begin{example}
求边长为$a$质量为$M$的匀质立方体对顶点的惯量矩阵,并求出其惯量主轴和主惯量。

\begin{figure}[htb]
\centering
\begin{asy}
	texpreamble("\usepackage{xeCJK}");
	texpreamble("\setCJKmainfont{SimSun}");
	usepackage("amsmath");
	usepackage("amssymb");
	size(200);
	import graph;
	import math;
	//第六章例3图
	pair O,i,j,k;
	picture tmp;
	O = (0,0);
	i = (-sqrt(2)/4,-sqrt(14)/12);
	j = (sqrt(14)/4,-sqrt(2)/12);
	k = (0,2*sqrt(2)/3);
	draw(Label("$x$",EndPoint),O--1.5*i,Arrow);
	draw(Label("$y$",EndPoint),O--1.5*j,Arrow);
	draw(Label("$z$",EndPoint),O--1.5*k,Arrow);
	label("$O$",O,W);
	path clp;
	clp = i--i+j--j--j+k--k--i+k--cycle;
	unfill(clp);
	draw(tmp,Label("$x$",EndPoint),O--1.5*i,dashed,Arrow);
	draw(tmp,Label("$y$",EndPoint),O--1.5*j,dashed,Arrow);
	draw(tmp,Label("$z$",EndPoint),O--1.5*k,dashed,Arrow);
	label(tmp,"$O$",O,W);
	clip(tmp,clp);
	add(tmp);
	erase(tmp);
	draw(clp);
	draw(i+k--i+j+k--j+k);
	draw(i+j--i+j+k);
	//draw(O--(1.6,0),invisible);
\end{asy}
\caption{例\theexample}
\label{第六章例3图}
\end{figure}
\end{example}
\begin{solution}
根据惯量矩阵中惯量元素的定义
\begin{equation*}
	I_{ij} = \int_V \rho \left(\delta_{ij} \sum_{k=1}^3 x_k^2 - x_ix_j\right) \mathrm{d} V
\end{equation*}
可有
\begin{align*}
	I_{11} & = \frac{M}{a^3} \int_V (y^2+z^2) \mathrm{d}x\mathrm{d}y\mathrm{d}z = \frac{M}{a^3} \int_0^a \mathrm{d} x \int_0^a \mathrm{d} y \int_0^a (y^2+z^2) \mathrm{d} z = \frac23 Ma^2 \\
	I_{22} & = \frac{M}{a^3} \int_V (x^2+z^2) \mathrm{d}x\mathrm{d}y\mathrm{d}z = \frac{M}{a^3} \int_0^a \mathrm{d} x \int_0^a \mathrm{d} y \int_0^a (x^2+z^2) \mathrm{d} z = \frac23 Ma^2 \\
	I_{33} & = \frac{M}{a^3} \int_V (x^2+y^2) \mathrm{d}x\mathrm{d}y\mathrm{d}z = \frac{M}{a^3} \int_0^a \mathrm{d} x \int_0^a \mathrm{d} y \int_0^a (x^2+y^2) \mathrm{d} z = \frac23 Ma^2 \\
	I_{12} & = I_{21} = \frac{M}{a^3} \int_V (-xy) \mathrm{d}x\mathrm{d}y\mathrm{d}z = -\frac{M}{a^3} \int_0^a x \mathrm{d} x \int_0^a y \mathrm{d} y \int_0^a \mathrm{d} z = -\frac14 Ma^2 \\
	I_{23} & = I_{32} = \frac{M}{a^3} \int_V (-yz) \mathrm{d}x\mathrm{d}y\mathrm{d}z = -\frac{M}{a^3} \int_0^a \mathrm{d} x \int_0^a y \mathrm{d} y \int_0^a z \mathrm{d} z = -\frac14 Ma^2 \\
	I_{13} & = I_{31} = \frac{M}{a^3} \int_V (-xz) \mathrm{d}x\mathrm{d}y\mathrm{d}z = -\frac{M}{a^3} \int_0^a x \mathrm{d} x \int_0^a \mathrm{d} y \int_0^a z \mathrm{d} z = -\frac14 Ma^2
\end{align*}
因此,惯量矩阵为
\begin{equation*}
	\mbf{I} = \frac{Ma^2}{12} \begin{pmatrix} 8 & -3 & -3 \\ -3 & 8 & -3 \\ -3 & -3 & 8 \end{pmatrix}
\end{equation*}
设矩阵$\mbf{I}$的特征值为$\dfrac{Ma^2}{12} \lambda$,则有
\begin{equation*}
	\begin{vmatrix} 8-\lambda & -3 & -3 \\ -3 & 8-\lambda & -3 \\ -3 & -3 & 8-\lambda \end{vmatrix} = 0
\end{equation*}
解得$\lambda_1=\lambda_2=11,\lambda_3=2$,因此有主惯量
\begin{equation*}
	I_1 = I_2 = \frac{11}{12} Ma^2,\quad I_3 =\frac16 Ma^2
\end{equation*}
相应的本征矢量可取为\footnote{其中$\mbf{e}_1$和$\mbf{e}_2$的取法不唯一。}
\begin{equation*}
	\mbf{e}_1 = \frac{1}{\sqrt{6}} \begin{pmatrix} 1 \\ 1 \\ -2 \end{pmatrix},\quad \mbf{e}_2 = \frac{1}{\sqrt{2}} \begin{pmatrix} -1 \\ 1 \\ 0 \end{pmatrix},\quad \mbf{e}_3 = \frac{1}{\sqrt{3}} \begin{pmatrix} 1 \\ 1 \\ 1 \end{pmatrix}
\end{equation*}

\begin{figure}[htb]
\centering
\begin{asy}
	texpreamble("\usepackage{xeCJK}");
	texpreamble("\setCJKmainfont{SimSun}");
	usepackage("amsmath");
	usepackage("amssymb");
	size(300);
	import graph;
	import math;
	//第六章例3惯量主轴
	pair O,i,j,k;
	picture tmp;
	O = (0,0);
	i = (-sqrt(2)/4,-sqrt(14)/12);
	j = (sqrt(14)/4,-sqrt(2)/12);
	k = (0,2*sqrt(2)/3);
	draw(Label("$x$",EndPoint),O--1.5*i,Arrow);
	draw(Label("$y$",EndPoint),O--1.5*j,Arrow);
	draw(Label("$z$",EndPoint),O--1.5*k,Arrow);
	label("$O$",O,W);
	path clp;
	clp = i--i+j--j--j+k--k--i+k--cycle;
	unfill(clp);
	draw(tmp,Label("$x$",EndPoint),O--1.5*i,dashed,Arrow);
	draw(tmp,Label("$y$",EndPoint),O--1.5*j,dashed,Arrow);
	draw(tmp,Label("$z$",EndPoint),O--1.5*k,dashed,Arrow);
	label(tmp,"$O$",O,W);
	clip(tmp,clp);
	add(tmp);
	erase(tmp);
	draw(clp);
	draw(i+k--i+j+k--j+k);
	draw(i+j--i+j+k);
	draw(-i--(-i+k),dashed);
	draw(-i--(j-i),dashed);
	draw(i--(i-2*k)--(i+j-2*k)--(j-2*k)--j,dashed);
	draw(O--(-2*k)--(j-2*k),dashed);
	draw((-2*k)--(i-2*k),dashed);
	draw(i+j--(i+j-2*k),dashed);
	draw(j--(j-i)--(j-i+k)--j+k,dashed);
	draw(k--(k-i)--(k-i+j),dashed);
	draw(O--(i+j-2*k),dashed);
	draw(O--(j-i),dashed);
	draw(O--i+j+k,dashed);
	draw(Label("$x'$",EndPoint,Relative(E),black),O--(i+j-2*k)/sqrt(3),red,Arrow);
	draw(Label("$y'$",EndPoint,black),O--(-i+j),red,Arrow);
	draw(Label("$z'$",EndPoint,N,black),O--(i+j+k),red,Arrow);
	//draw(O--(1.7,0),invisible);
\end{asy}
\caption{例\theexample:立方体对顶点的惯量主轴}
\label{第六章例3惯量主轴}
\end{figure}

质心的坐标为
\begin{equation*}
	\mbf{X} = \frac{a}{2} \begin{pmatrix} 1 \\ 1 \\ 1 \end{pmatrix}
\end{equation*}
故对质量集中与质心对顶点的惯量元素为
\begin{equation*}
	I_{Cij} = M\left(\delta_{ij} \sum_{k=1}^3 X_k^2 -X_iX_j\right)
\end{equation*}
由此可得质量集中与质心对顶点的惯量矩阵为
\begin{equation*}
	\mbf{I}_C = \frac{Ma^2}{4} \begin{pmatrix} 2 & -1 & -1 \\ -1 & 2 & -1 \\ -1 & -1 & 2 \end{pmatrix}
\end{equation*}
故有对质心$C$的惯量矩阵为
\begin{equation*}
	\mbf{I}' = \mbf{I} - \mbf{I}_C = \frac{Ma^2}{6} \begin{pmatrix} 1 & 0 & 0 \\ 0 & 1 & 0 \\ 0 & 0 & 1 \end{pmatrix}
\end{equation*}
故对质心$C$可选任意三个相互正交的方向为惯量主轴,主惯量为
\begin{equation*}
	I_1 = I_2 = I_3 = \frac16 Ma^2
\end{equation*}
即,匀质立方体对质心的转动惯性是球对称的。
\end{solution}

\subsection{惯量椭球}

在刚体的转轴上取点$P$,使得其坐标满足
\begin{equation*}
	\mbf{x} = \frac{1}{\sqrt{I_n}} \mbf{n}
\end{equation*}
所以有
\begin{equation*}
	\mbf{x}^{\mathrm{T}} \mbf{I} \mbf{x} = \frac{1}{I_n} \mbf{n}^{\mathrm{T}} \mbf{I} \mbf{n} = 1
\end{equation*}
由于惯量矩阵$\mbf{I}$对称正定,因此点$P$的集合为中心在原点的椭球面,此椭球面称为{\heiti 惯量椭球}。它是刚体转动惯量的几何表示。

在主轴系中,惯量椭球为标准形式
\begin{equation*}
	I_1 x'^2 + I_2 y'^2 + I_3 z'^2 = 1
\end{equation*}
主轴是惯量椭球的对称轴。

当$I_1=I_2=I_3$时,惯量椭球退化为球,其主轴可以任选,物体呈转动各项同性。当$I_1=I_2\neq I_3$时,惯量椭球退化为旋转椭球面,两主轴可以在垂直于第三个主轴的平面内任选。

\begin{figure}[htb]
\centering
\begin{asy}
	texpreamble("\usepackage{xeCJK}");
	texpreamble("\setCJKmainfont{SimSun}");
	usepackage("amsmath");
	import graph;
	import math;
	size(250);
	//惯量椭球和主轴系
	pair O,i,j,k;
	picture tmp,clp1,clp2;
	O = (0,0);
	i = (-sqrt(2)/4,-sqrt(14)/12);
	j = (sqrt(14)/4,-sqrt(2)/12);
	k = (0,2*sqrt(2)/3);
	label("$O$",O,W);
	
	real a,b,alpha,beta;
	a = 3;
	b = 5;
	
	pair ell1(real t){
		real x,y,z;
		x = a*cos(t)*cos(alpha);
		y = b*sin(t);
		z = a*cos(t)*sin(alpha);
		return x*i+y*j+z*k;
	}
	pair ell2(real t){
		real x,y,z;
		x = a*cos(t);
		y = 0;
		z = a*sin(t);
		return x*i+y*j+z*k;
	}
	alpha = 0;
	beta = 20;
	draw(tmp,Label(rotate(-beta)*"$x'$",EndPoint,black),O--4.5*i,red,Arrow);
	draw(tmp,Label(rotate(-beta)*"$y'$",EndPoint,black),O--6*j,red,Arrow);
	draw(tmp,Label(rotate(-beta)*"$z'$",EndPoint,black),O--4.5*k,red,Arrow);
	path cir1,cir2;
	cir1 = graph(ell1,0,2*pi)--cycle;
	cir2 = graph(ell2,0,2*pi)--cycle;
	unfill(tmp,cir1);
	unfill(tmp,cir2);
	draw(clp1,Label(rotate(-beta)*"$x'$",EndPoint,black),O--4.5*i,red+dashed,Arrow);
	draw(clp1,Label(rotate(-beta)*"$y'$",EndPoint,black),O--6*j,red+dashed,Arrow);
	draw(clp1,Label(rotate(-beta)*"$z'$",EndPoint,black),O--4.5*k,red+dashed,Arrow);
	add(clp2,clp1);
	clip(clp1,cir1);
	clip(clp2,cir2);
	add(tmp,clp1);
	add(tmp,clp2);
	draw(tmp,cir1,dashed);
	draw(tmp,cir2,dashed);
	draw(tmp,xscale(0.97*b)*yscale(a)*unitcircle);
	add(rotate(beta)*tmp);
	erase(tmp);
	
	draw(Label("$x$",EndPoint,black),O--6*i,blue,Arrow);
	draw(Label("$y$",EndPoint,black),O--6*j,blue,Arrow);
	draw(Label("$z$",EndPoint,black),O--4.5*k,blue,Arrow);
\end{asy}
\caption{惯量椭球和主轴系(红色坐标系为主轴系)}
\label{惯量椭球和主轴系}
\end{figure}

惯量椭球的几何意义为:
\begin{enumerate}
	\item 给出对过定点的任意轴的转动惯量;
	\item 给出主轴系;
	\item 已知瞬时转动轴(即角速度$\mbf{\omega}$的方向),确定角动量方向。
\end{enumerate}

\begin{figure}[htb]
\centering
\begin{asy}
	texpreamble("\usepackage{xeCJK}");
	texpreamble("\setCJKmainfont{SimSun}");
	usepackage("amsmath");
	import graph;
	import math;
	size(250);
	//角速度的方向与角动量的方向
	pair O,a,b,L,omega;
	real rx,ry,theta,lL,lomega;
	path rect,cir;
	picture tmp1,tmp2;
	O = (0,0);
	a = (2,0);
	b = (0.7,0.8);
	L = (0,-0.9);
	omega = L+0.7*dir(190);
	lL = 2.2;
	lomega = lL/cos((degrees(omega)-degrees(L))*pi/180);
	rx = 1.5;
	ry = 1;
	theta = 50;
	rect = L+a+b--L+a-b--L-a-b--L-a+b--cycle;
	cir = rotate(theta)*xscale(rx)*yscale(ry)*unitcircle;
	dot(O);
	label("$O$",O,N);
	draw(cir);
	draw(tmp1,rect);
	unfill(tmp1,cir);
	add(tmp1);
	erase(tmp1);
	draw(tmp1,O--lL*L,blue+dashed,Arrow);
	draw(tmp1,O--lomega*omega,red+dashed,Arrow);
	add(tmp2,tmp1);
	clip(tmp1,rect);
	add(tmp1);
	clip(tmp2,cir);
	unfill(tmp2,rect);
	add(tmp2);
	erase(tmp1);
	erase(tmp2);
	draw(tmp1,Label("$\boldsymbol{L}$",EndPoint,black),O--lL*dir(L),blue,Arrow);
	draw(tmp1,Label("$\boldsymbol{\omega}$",EndPoint,black),O--lomega*dir(omega),red,Arrow);
	unfill(tmp1,rect);
	unfill(tmp1,cir);
	add(tmp1);
	erase(tmp1);
	dot(omega);
	dot(L);
	label("$N$",omega,N);
	label("$L$",L,NE);
	draw(L--omega,dashed);
\end{asy}
\caption{角速度的方向与角动量的方向}
\label{角速度的方向与角动量的方向}
\end{figure}

下面考虑已知角速度$\mbf{\omega}$的方向,来求解角动量$\mbf{L}$的方向。首先刚体的动能可以表示为
\begin{equation*}
	T = \frac12 I_n \omega^2
\end{equation*}
此处$I_n$为刚体绕瞬时转轴的转动惯量。则转轴与惯量椭球的交点$N$(称为{\heiti 极点})为
\begin{equation*}
	\mbf{x}_N = \frac{1}{\sqrt{I_n}} \frac{\mbf{\omega}}{\omega} = \frac{\mbf{\omega}}{\sqrt{2T}}
\end{equation*}
则极点处,惯量椭球的切平面法向矢量可以表示为
\begin{equation*}
	\mbf{n} = \bnb (\mbf{x}^{\mathrm{T}} \mbf{I} \mbf{x}-1)\big|_N = 2\mbf{I} \mbf{x}_N = \frac{2\mbf{I}\mbf{\omega}}{\sqrt{2T}} = \frac{2\mbf{L}}{\sqrt{2T}}
\end{equation*}
由此即可得角动量的方向沿惯量椭球过极点切平面的法向。

惯量椭球固定于刚体,其转动代表刚体的定点转动。

\section{Euler动力学方程}

\subsection{刚体运动方程}

当刚体在空间系中作定点转动时(此时平动系与空间系重合),取定点为基点$O$,根据角动量定理可有
\begin{equation*}
	\frac{\mathrm{d} \mbf{L}}{\mathrm{d} t} = \mbf{M}^{(e)}
\end{equation*}
此处$\mbf{M}^{(e)}$为刚体对$O$点的合外力矩,定义为\footnote{此式同样可以通过将刚体分割为$n$个质点,然后令$n$趋于无穷大根据重积分的定义获得。}
\begin{equation*}
	\mbf{M}^{(e)} = \int_V \mbf{r} \times \mbf{f}^{(e)} \mathrm{d} V
\end{equation*}
式中$\mbf{f}^{(e)}$为刚体所受的外力分布(单位体积刚体所受外力)。刚体的动能为
\begin{equation*}
	T = \frac12 \mbf{\omega} \cdot \mbf{L}
\end{equation*}
由于刚体的性质,刚体在运动过程中其内部任意两个质点间都没有相对位移,因此内约束力不做功。根据动能定理可有
\begin{equation*}
	\frac{\mathrm{d} T}{\mathrm{d} t} = \int_V \mbf{f}^{(e)} \cdot \dot{\mbf{r}} \mathrm{d} V = \int_V \mbf{f}^{(e)} \cdot (\mbf{\omega} \times \mbf{r}) \mathrm{d} V = \int_V \mbf{\omega} \cdot (\mbf{r} \times \mbf{f}^{(e)}) \mathrm{d} V = \mbf{\omega} \cdot \mbf{M}^{(e)}
\end{equation*}

对于刚体在空间系的一般运动,有如下定理。

\begin{theorem}[Chasles定理\footnote{中文可作“沙勒定理”或“夏莱定理”。}]
\label{Chasles定理}
刚体最一般的位移可以分解为随任选基点的平动位移和绕该基点的定点转动。
\end{theorem}

由此,可取质心为基点,刚体的运动则分解为随质心的平动与绕质心的定点转动,而且前面已说明刚体定点转动的角速度与基点的选取无关。对于刚体质心的平动运动,根据动量定理可有
\begin{equation}
	M \dot{\mbf{V}} = \mbf{F}^{(e)}
\end{equation}
式中$\mbf{F}^{(e)}$为刚体所受合外力,即
\begin{equation*}
	\mbf{F}^{(e)} = \int_V \mbf{f}^{(e)} \mathrm{d} V
\end{equation*}
再考虑到刚体在运动过程中其内部任意两个质点间都没有相对位移,根据动能定理可有
\begin{equation}
	\frac{\mathrm{d}}{\mathrm{d} t}\left(\frac12 MV^2\right) = \mbf{F}^{(e)} \cdot \mbf{V}
\end{equation}
对于刚体绕质心的定点转动,根据质心系的角动量定理可有
\begin{equation}
	\frac{\mathrm{d} \mbf{L}'}{\mathrm{d} t} = \mbf{M}'^{(e)}
\end{equation}
式中$\mbf{M}'^{(e)}$为对质心的合外力矩,即
\begin{equation*}
	\mbf{M}'^{(e)} = \int_V \mbf{r}' \times \mbf{f}^{(e)} \mathrm{d} V
\end{equation*}
根据质心系的动能定理可有
\begin{equation}
	\frac{\mathrm{d} T'}{\mathrm{d} t} = \frac{\mathrm{d}}{\mathrm{d} t} \left(\frac12 \mbf{\omega} \cdot \mbf{L}'\right) = \mbf{\omega} \cdot \mbf{M}'^{(e)}
\end{equation}

在空间系中,刚体对原点的角动量为
\begin{equation}
	\mbf{L} = \mbf{R} \times M \mbf{V} + \mbf{L}'
\end{equation}
在空间系中,刚体的动能为
\begin{equation}
	T = \frac12 MV^2 + T' = \frac12 \mbf{V} \cdot \mbf{P} + \frac12 \mbf{\omega} \cdot \mbf{L}'
\end{equation}

\subsection{Euler动力学方程}

刚体质心的运动规律与质点是相同的,现在只考虑刚体绕质心的定点转动。为了更具一般性,我们考虑刚体对任意基点的定点运动方程。即考虑角动量定理
\begin{equation}
	\frac{\mathrm{d} \mbf{L}}{\mathrm{d} t} = \mbf{M}
	\label{Euler动力学方程——角动量定理}
\end{equation}
的分量形式。式中$\mbf{L}$为刚体在平动系(可以不是质心系)中的角动量,$\mbf{M}$为刚体对基点的外力矩。在本体系中,刚体的惯量矩阵是常数矩阵,因此下面考虑求得方程\eqref{Euler动力学方程——角动量定理}在本体系中的分量形式。本体系的基矩阵$\mbf{e}$与平动系的基矩阵$\mbf{E}$之间满足
\begin{equation*}
	\mbf{e} = \mbf{E} \mbf{U}
\end{equation*}
在本体系中可有
\begin{equation*}
	\mbf{L} = \sum_{i=1}^3 L_i \mbf{e}_i = \begin{pmatrix} \mbf{e}_1 & \mbf{e}_2 & \mbf{e}_3 \end{pmatrix} \begin{pmatrix} L_1 \\ L_2 \\ L_3 \end{pmatrix} = \mbf{e} \bar{\mbf{L}}
\end{equation*}
式中$\bar{\mbf{L}} = \begin{pmatrix} L_1 \\ L_2 \\ L_3 \end{pmatrix}$。由此可有
\begin{align}
	\frac{\mathrm{d} \mbf{L}}{\mathrm{d} t} & = \mbf{e} \dot{\bar{\mbf{L}}} + \dot{\mbf{e}} \bar{\mbf{L}} = \mbf{e} \dot{\bar{\mbf{L}}} + \mbf{E} \dot{\mbf{U}} \bar{\mbf{L}} = \mbf{e} \dot{\bar{\mbf{L}}} + \mbf{e} \mbf{U}^{\mathrm{T}} \dot{\mbf{U}} \bar{\mbf{L}} \nonumber \\
	& = \frac{\tilde{\mathrm{d}} \mbf{L}}{\mathrm{d} t} + \mbf{\omega} \times \mbf{L}
	\label{Euler动力学方程——角动量定理的矢量形式}
\end{align}
式中$\displaystyle \frac{\tilde{\mathrm{d}} \mbf{L}}{\mathrm{d} t} = \sum_{i=1}^3 \dot{L}_i \mbf{e}_i$称为{\heiti 相对变化率},$\mbf{\omega} \times \mbf{L}$称为{\heiti 转动牵连变化率}。考虑到$\mbf{L} = \mbf{I} \mbf{\omega}$,可以将式\eqref{Euler动力学方程——角动量定理的矢量形式}写作分量形式。

如果本体系取为主轴系,则有
\begin{equation*}
	L_i = I_i \omega_i,\quad i = 1,2,3
\end{equation*}
此时可使式\eqref{Euler动力学方程——角动量定理的矢量形式}的分量形式大为简化,即
\begin{subnumcases}{\label{Euler动力学方程}}
	I_1 \dot{\omega}_1 - (I_2-I_3)\omega_2 \omega_3 = M_1 \label{Euler动力学方程-1}\\
	I_2 \dot{\omega}_2 - (I_3-I_1)\omega_3 \omega_1 = M_2 \label{Euler动力学方程-2}\\
	I_3 \dot{\omega}_3 - (I_1-I_2)\omega_1 \omega_2 = M_3 \label{Euler动力学方程-3}
\end{subnumcases}
式\eqref{Euler动力学方程}称为{\heiti Euler动力学方程},它是对角速度的一阶非线性常微分方程组。将Euler运动学方程\eqref{Euler运动学方程}代入,可得对Euler角的二阶非线性常微分方程组。

\begin{example}[偏斜的飞轮]
设有半径为$R$,质量为$m$的圆盘状飞轮安装在两个间距为$l$的轴承正中,但飞轮的法线方向与轴承之间有一夹角$\alpha$,求飞轮以$\omega$的角速度匀速绕轴旋转时,轴承上受到的额外负荷。
\end{example}
\begin{solution}
\begin{figure}[htb]
\centering
\begin{asy}
	texpreamble("\usepackage{xeCJK}");
	texpreamble("\setCJKmainfont{SimSun}");
	usepackage("amsmath");
	import graph;
	import math;
	size(250);
	//第六章例4图
	picture tmp;
	pair O,l,co,dash,P;
	real alpha,d,R,f,ff,F,omega,r;
	O = (0,0);
	l = (2,0);
	co = (1.8,0);
	d = 0.1;
	R = 1;
	f = 0.2;
	ff = 0.1;
	alpha = -20;
	F = 1.1;
	dash = 0.07*dir(60);
	omega = 1;
	draw(-l--l);
	draw(tmp,co+(-f,ff)--co+(f,ff));
	draw(tmp,co+(-f,-ff)--co+(f,-ff));
	for(int i=1;i<=5;i=i+1){
		P = co+(-f,ff)+(i-1)*2*f/4.5;
		draw(tmp,P--P+dash);
		P = co+(f,-ff)-(i-1)*2*f/5;
		draw(tmp,P--P-dash);
	}
	add(tmp);
	add(shift(-2*co)*tmp);
	erase(tmp);
	draw(Label("$\boldsymbol{F}_e$",EndPoint,black),co--co+F*dir(-90),red,Arrow);
	draw(Label("$\boldsymbol{F}_e$",EndPoint,black),-co--(-co+F*dir(90)),red,Arrow);
	label("$O$",O,NW);
	fill(rotate(alpha)*box((-d/2,-R),(d/2,R)),0.6white);
	draw(Label("$x$",EndPoint),O--1.3*dir(90+alpha),Arrow);
	draw(Label("$z$",EndPoint),O--1.3*dir(alpha),Arrow);
	draw(Label("$\boldsymbol{\omega}$",EndPoint,Relative(W)),O--omega*dir(0),red,Arrow);
	draw(omega*cos(alpha*pi/180)*dir(alpha)--omega*dir(0)--omega*sin(-alpha*pi/180)*dir(90+alpha),dashed);
	label("$R$",0.5*R*dir(alpha-90),dir(alpha+180));
	r = 0.4;
	draw(Label("$\alpha$",MidPoint,Relative(W)),arc(O,r,0,alpha),Arrow);
\end{asy}
\caption{例\theexample}
\label{第六章例4图}
\end{figure}

圆盘的主惯量为
\begin{equation*}
	I_1 = I_2 = \frac14 mR^2,\quad I_3 = \frac12 mR^2
\end{equation*}
在本体系中,角速度为
\begin{equation*}
	\omega_1 = \omega \sin \alpha,\quad \omega_2 = 0,\quad \omega_3 = \omega \cos \alpha
\end{equation*}
由于是匀速转动,可有
\begin{equation*}
	\dot{\omega}_1 = \dot{\omega}_2 = \dot{\omega}_3 = 0
\end{equation*}
Euler动力学方程为
\begin{equation*}
	\begin{cases}
		I_1 \dot{\omega}_1 - (I_2-I_3)\omega_2 \omega_3 = M_1 \\
		I_2 \dot{\omega}_2 - (I_3-I_1)\omega_3 \omega_1 = M_2 \\
		I_3 \dot{\omega}_3 - (I_1-I_2)\omega_1 \omega_2 = M_3
	\end{cases}
\end{equation*}
由此可得
\begin{equation*}
	\begin{cases}
		M_1 = 0 \\
		M_2 = -\dfrac18 mR^2 \omega^2 \sin 2\alpha \\ 
		M_3 = 0
	\end{cases}
\end{equation*}
因此轴承上受到的力(动反作用力)为
\begin{equation*}
	F_e = \frac{|M_2|}{l} = \frac{mR^2 \omega^2 \sin 2\alpha}{8l}
\end{equation*}
而轴承受到的静反作用力(重力负荷)为
\begin{equation*}
	F_g = \frac12 mg
\end{equation*}
当角速度$\omega$很大时,轴承受到的动反作用力可以远大于静反作用力。
\end{solution}

\section{无力矩陀螺(Euler陀螺)}

不受外力矩作用的定点转动刚体,又称为{\heiti Euler陀螺}。

\subsection{运动方程与守恒量}

Euler陀螺的合外力矩恒为零,即$\mbf{M} = \mbf{0}$,由Euler动力学方程\eqref{Euler动力学方程}可得陀螺的运动方程为
\begin{equation}
	\begin{cases}
		I_1 \dot{\omega}_1 - (I_2-I_3)\omega_2 \omega_3 = 0 \\
		I_2 \dot{\omega}_2 - (I_3-I_1)\omega_3 \omega_1 = 0 \\
		I_3 \dot{\omega}_3 - (I_1-I_2)\omega_1 \omega_2 = 0
	\end{cases}
	\label{Euler陀螺的运动方程}
\end{equation}
将\eqref{Euler陀螺的运动方程}各式分别乘以$I_1\omega_1$、$I_2\omega_2$、$I_3\omega_3$相加,得
\begin{equation*}
	I_1^2 \omega_1 \dot{\omega}_1 + I_2^2 \omega_2 \dot{\omega}_2 + I_3^2 \omega_3 \dot{\omega}_3 = 0
\end{equation*}
即有角动量守恒\footnote{也可同角动量定理得到。}
\begin{equation}
	\sum_{i=1}^3 I_i^2 \omega_i^2 = L^2 \quad\text{(常数)}
	\label{Euler陀螺角动量守恒}
\end{equation}
将\eqref{Euler陀螺的运动方程}各式分别乘以$\omega_1$、$\omega_2$、$\omega_3$相加,得
\begin{equation*}
	I_1 \omega_1 \dot{\omega}_1 + I_2 \omega_2 \dot{\omega}_2 + I_3 \omega_3 \dot{\omega}_3 = 0
\end{equation*}
即有能量守恒\footnote{也可同动能定理得到。}
\begin{equation}
	\frac12 \sum_{i=1}^3 I_i \omega_i^2 = T \quad\text{(常数)}
	\label{Euler陀螺能量守恒}
\end{equation}

\subsection{对称Euler陀螺}

至少有两个主惯量相等的刚体即可称为{\heiti 对称刚体}。不妨设有$I_1=I_2$,此时Euler动力学方程化为
\begin{equation}
	\begin{cases}
		I_1 \dot{\omega}_1 - (I_1-I_3)\omega_2 \omega_3 = 0 \\
		I_1 \dot{\omega}_2 - (I_3-I_1)\omega_3 \omega_1 = 0 \\
		I_3 \dot{\omega}_3 = 0
	\end{cases}
	\label{对称Euler陀螺的运动方程}
\end{equation}
由方程\eqref{对称Euler陀螺的运动方程}中的第三式可得
\begin{equation}
	\omega_3 = \varOmega \quad \text{(常数)}
	\label{对称Euler陀螺的解-1}
\end{equation}
记$n = \dfrac{I_3-I_1}{I_1} \varOmega$,则方程\eqref{对称Euler陀螺的运动方程}的第一二式化为
\begin{equation*}
	\begin{cases}
		\dot{\omega}_1 + n\omega_2 = 0 \\
		\dot{\omega}_2 - n\omega_1 = 0
	\end{cases}
\end{equation*}
从中消去$\omega_2$可得
\begin{equation*}
	\ddot{\omega}_1 + n^2\omega_1 = 0
\end{equation*}
由此解得
\begin{equation}
\begin{cases}
	\omega_1 = \omega_0 \cos(nt+\eps) \\
	\omega_2 = \omega_0 \sin(nt+\eps)
\end{cases}
\label{对称Euler陀螺的解-2}
\end{equation}
综上可得刚体的角速度为
\begin{equation}
	\mbf{\omega} = \omega_1 \mbf{e}_1 + \omega_2 \mbf{e}_2 + \omega_3 \mbf{e}_3 = \omega_0 \cos(nt+\eps) \mbf{e}_1 + \omega_0 \sin(nt+\eps) \mbf{e}_2 + \varOmega \mbf{e}_3
	\label{对称Euler陀螺的解-3}
\end{equation}
角速度的大小为
\begin{equation*}
	\omega = \sqrt{\omega_1^2 +\omega_2^2 +\omega_3^2} = \sqrt{\omega_0^2 + \varOmega^2}\quad \text{(常数)}
\end{equation*}
该角速度与主轴系$z$轴夹角是固定的$\alpha = \arctan \dfrac{\omega_0}{\varOmega}$。

对称Euler陀螺的角速度位于主轴系的一个半顶角为$\alpha$,$z$轴为对称轴的锥面(称为{\heiti 本体极锥})上,绕$z$轴以匀角速度$n$旋转。

\begin{figure}[htb]
\centering
\begin{minipage}[t]{0.45\textwidth}
\centering
\begin{asy}
	texpreamble("\usepackage{xeCJK}");
	texpreamble("\setCJKmainfont{SimSun}");
	usepackage("amsmath");
	usepackage("amssymb");
	size(200);
	import graph;
	import math;
	//对称Euler陀螺的角速度矢量
	picture tmp;
	pair O,i,j,k;
	real r,h,rr,pos;
	O = (0,0);
	i = (-sqrt(2)/4,-sqrt(14)/12);
	j = (sqrt(14)/4,-sqrt(2)/12);
	k = (0,2*sqrt(2)/3);
	pos = 1.6;
	draw(Label("$x$",EndPoint),O--2*i,Arrow);
	draw(Label("$y$",EndPoint),O--2*j,Arrow);
	draw(Label("$z$",EndPoint),O--2*k,Arrow);
	label("$O$",O,W);
	r = 1;
	h = 1.5;
	rr = 0.4;
	pair cir(real theta){
		return r*cos(theta)*i+r*sin(theta)*j+h*k;
	}
	pair bot(real theta){
		return rr*cos(theta)*i+rr*sin(theta)*j;
	}
	draw(graph(cir,0,2*pi),dashed);
	draw(O--cir(pos),dashed);
	draw(xscale(-1)*(O--cir(pos)),dashed);
	draw(Label("$\alpha$",MidPoint,Relative(W)),arc(O,rr,90,degrees(cir(pos))),Arrow);
	draw(Label("$\boldsymbol{\omega}$",EndPoint,black),O--cir(pi/3),red,Arrow);
	draw(Label("$\boldsymbol{\omega}_0$",EndPoint,black),O--cir(pi/3)-h*k,red,Arrow);
	draw(Label("$\boldsymbol{\varOmega}$",EndPoint,Relative(W),black),O--h*k,red,Arrow);
	draw(h*k--cir(pi/3)--cir(pi/3)-h*k,dashed);
	draw(Label("$nt+\varepsilon$",MidPoint,Relative(E)),graph(bot,0,pi/3),Arrow);
\end{asy}
\caption{对称Euler陀螺的角速度矢量}
\label{对称Euler陀螺的角速度矢量}
\end{minipage}
\hspace{0.5cm}
\begin{minipage}[t]{0.45\textwidth}
\centering
\begin{asy}
	texpreamble("\usepackage{xeCJK}");
	texpreamble("\setCJKmainfont{SimSun}");
	usepackage("amsmath");
	usepackage("amssymb");
	size(200);
	import graph;
	import math;
	//对称Euler陀螺的角动量矢量
	picture tmp;
	pair O,i,j,k;
	real r,h,rr,pos;
	O = (0,0);
	i = (-sqrt(2)/4,-sqrt(14)/12);
	j = (sqrt(14)/4,-sqrt(2)/12);
	k = (0,2*sqrt(2)/3);
	pos = 1.8;
	draw(Label("$x$",EndPoint),O--2*i,Arrow);
	draw(Label("$y$",EndPoint),O--2*j,Arrow);
	draw(Label("$z$",EndPoint),O--2*k,Arrow);
	label("$O$",O,W);
	r = 0.8;
	h = 1.6;
	rr = 0.4;
	pair cir(real theta){
		return r*cos(theta)*i+r*sin(theta)*j+h*k;
	}
	pair bot(real theta){
		return rr*cos(theta)*i+rr*sin(theta)*j;
	}
	draw(graph(cir,0,2*pi),dashed);
	draw(O--cir(pos),dashed);
	draw(xscale(-1)*(O--cir(pos)),dashed);
	draw(Label("$\theta$",MidPoint,Relative(W)),arc(O,rr,90,degrees(cir(pos))),Arrow);
	draw(Label("$Z$",EndPoint,black),O--1.4*cir(pi/3),Arrow);
	draw(Label("$\boldsymbol{L}$",EndPoint,NW,black),O--cir(pi/3),blue,Arrow);
	draw(Label("$I_1\boldsymbol{\omega}_0$",EndPoint,black),O--cir(pi/3)-h*k,blue,Arrow);
	draw(Label("$I_3\boldsymbol{\varOmega}$",EndPoint,Relative(W),black),O--h*k,blue,Arrow);
	draw(h*k--cir(pi/3)--cir(pi/3)-h*k,dashed);
	draw(Label("$nt+\varepsilon$",MidPoint,Relative(E)),graph(bot,0,pi/3),Arrow);
\end{asy}
\caption{对称Euler陀螺的角动量矢量}
\label{对称Euler陀螺的角动量矢量}
\end{minipage}
\end{figure}

下面考虑角动量矢量$\mbf{L}$。根据角动量守恒,$\mbf{L}$为常矢量,方向不变,可取平动系基矢量为$\mbf{E}_Z = \dfrac{\mbf{L}}{L}$。具体可有
\begin{equation*}
	\mbf{L} = I_1(\omega_1 \mbf{e}_1+\omega_2 \mbf{e}_2) + I_3 \omega_3 \mbf{e}_3 = I_1 \omega_0 \big[\cos(nt+\eps)\mbf{e}_1 + \sin(nt+\eps)\mbf{e}_2\big] + I_3 \varOmega \mbf{e}_3
\end{equation*}
由此可以看出矢量$\mbf{e}_3$、$\mbf{\omega}$和$\mbf{L}$共面,角速度的大小
\begin{equation*}
	L = \sqrt{I_1^2 \omega_1^2+ I_2^2 \omega_2^2+ I_3^2 \omega_3^2} = \sqrt{I_1^2 \omega_0^2 + I_3^2 \varOmega^2}\quad \text{(常数)}
\end{equation*}
章动角$\theta$满足
\begin{equation*}
	\tan \theta = \frac{I_1 \omega_0}{I_3 \varOmega} = \frac{I_1}{I_3} \tan \alpha \quad \text{(常数)}
\end{equation*}
即有$\dot{\theta} = 0$,无章动。

再由Euler运动学方程
\begin{equation*}
	\begin{cases}
		\omega_1 = \dot{\phi} \sin \theta \sin \psi \\
		\omega_2 = \dot{\phi} \sin \theta \cos \psi \\
		\omega_3 = \dot{\phi} \cos \theta + \dot{\psi}
	\end{cases}
\end{equation*}
由此可得自转角$\psi$满足
\begin{equation*}
	\tan \psi = \frac{\omega_1}{\omega_2} = \cot (nt+\eps)
\end{equation*}
即有自转角$\psi = -nt+ \dfrac{\pi}{2} - \eps$。由此可得自转角速度以及进动角速度
\begin{equation*}
	\begin{cases}
		\displaystyle \dot{\psi} = -n = \frac{I_1-I_3}{I_1} \varOmega \quad \text{(常数)} \\[1.5ex]
		\displaystyle \dot{\phi} = \frac{\omega_3 - \dot{\psi}}{\cos \theta} = \frac{L}{I_1}>0 \quad \text{(常数)}
	\end{cases}
\end{equation*}
即有第三主轴绕角速度矢量以匀角速度逆时针进动。章动角恒定,进动和自转角速度恒定的转动称为{\heiti 规则进动}。考虑到$\mbf{e}_z$、$\mbf{\omega}$和$\mbf{L}$共面,故角速度绕角动量矢量以匀角速度逆时针旋转,由于
\begin{equation*}
	\mbf{\omega} \cdot \frac{\mbf{L}}{L} = \frac{2T}{L}\quad \text{(常数)}
\end{equation*}
故角速度位于空间系锥面(称为{\heiti 空间极锥})上。角$\alpha$和$\theta$满足
\begin{equation*}
	\frac{\tan \theta}{\tan \alpha} = \frac{I_1}{I_3}
\end{equation*}
它们决定了空间极锥与本体极锥之间的相对位置。Euler陀螺可能的三种空间极锥与本体极锥之间的相对位置关系如图\ref{空间极锥与本体极锥情形1}、\ref{空间极锥与本体极锥情形2}和\ref{空间极锥与本体极锥情形3}所示。

\begin{figure}[htbp]
\centering
\begin{minipage}[t]{0.45\textwidth}
\centering
\begin{asy}
	texpreamble("\usepackage{xeCJK}");
	texpreamble("\setCJKmainfont{SimSun}");
	usepackage("amsmath");
	usepackage("amssymb");
	size(200);
	import graph;
	import math;
	//空间极锥与本体极锥情形1
	picture tmp;
	pair O,i,j,k;
	real r,r1,r2,h,h1,h2,theta,alpha,pos,L;
	O = (0,0);
	i = (-sqrt(2)/4,-sqrt(14)/12);
	j = (sqrt(14)/4,-sqrt(2)/12);
	k = (0,2*sqrt(2)/3);
	pos = 1.88;
	r1 = 0.5;
	r2 = 0.8;
	h1 = 1.6;
	h2 = sqrt(r1*r1+h1*h1-r2*r2);
	L = 2;
	pair cir(real theta){
		return r*cos(theta)*i+r*sin(theta)*j+h*k;
	}
	r = r1;
	h = h1;
	draw(graph(cir,0,2*pi),blue+dashed);
	draw(O--cir(pos),blue+dashed);
	draw(xscale(-1)*(O--cir(pos)),blue+dashed);
	theta = 90-degrees(cir(pos));
	r = r2;
	h = h2;
	pos = 1.80;
	draw(tmp,graph(cir,0,2*pi),red+dashed);
	draw(tmp,O--cir(pos),red+dashed);
	draw(tmp,xscale(-1)*(O--cir(pos)),red+dashed);
	alpha = 90-degrees(cir(pos));
	add(rotate(alpha+theta)*tmp);
	draw(Label("$\boldsymbol{L}$",EndPoint,black),O--L*dir(90),blue,Arrow);
	draw(Label("$\boldsymbol{e}_3$",EndPoint),O--L*dir(90+alpha+theta),Arrow);
	//画角速度
	real omega,phi,psi,ralpha,rtheta,r0;
	ralpha = alpha*pi/180;
	rtheta = theta*pi/180;
	omega = 1;
	phi = sin(ralpha)/sin(ralpha+rtheta)*omega;
	psi = sin(rtheta)/sin(ralpha+rtheta)*omega;
	draw(Label("$\boldsymbol{\omega}$",EndPoint,Relative(W),black),O--omega*dir(90+theta),red,Arrow);
	draw(Label("$\dot{\phi}$",EndPoint,Relative(E),black),O--phi*dir(90),red,Arrow);
	draw(Label("$\dot{\psi}$",EndPoint,Relative(W),black),O--psi*dir(90+alpha+theta),red,Arrow);
	draw(psi*dir(90+alpha+theta)--omega*dir(90+theta)--phi*dir(90),dashed);
	r0 = 0.3;
	draw(Label("$\alpha$",MidPoint,Relative(E)),arc(O,r0,90+theta,90+alpha+theta),Arrow);
	r0 = 0.3;
	draw(Label("$\theta$",MidPoint,Relative(E)),arc(O,r0,90,90+theta),Arrow);
\end{asy}
\caption{$I_1=I_2>I_3$,$\theta>\alpha$}
\label{空间极锥与本体极锥情形1}
\end{minipage}
\hspace{0.5cm}
\begin{minipage}[t]{0.45\textwidth}
\centering
\begin{asy}
	texpreamble("\usepackage{xeCJK}");
	texpreamble("\setCJKmainfont{SimSun}");
	usepackage("amsmath");
	usepackage("amssymb");
	size(200);
	import graph;
	import math;
	//空间极锥与本体极锥情形2
	picture tmp;
	pair O,i,j,k;
	real r,r1,r2,h,h1,h2,theta,alpha,pos,L;
	O = (0,0);
	i = (-sqrt(2)/4,-sqrt(14)/12);
	j = (sqrt(14)/4,-sqrt(2)/12);
	k = (0,2*sqrt(2)/3);
	pos = 1.88;
	r1 = 0.5;
	r2 = 0.8;
	h1 = 1.6;
	h2 = sqrt(r1*r1+h1*h1-r2*r2);
	L = 2;
	pair cir(real theta){
		return r*cos(theta)*i+r*sin(theta)*j+h*k;
	}
	theta = 0;
	r = r2;
	h = h2;
	pos = 1.80;
	draw(tmp,graph(cir,0,2*pi),red+dashed);
	draw(tmp,O--cir(pos),red+dashed);
	draw(tmp,xscale(-1)*(O--cir(pos)),red+dashed);
	alpha = 90-degrees(cir(pos));
	add(rotate(alpha+theta)*tmp);
	draw(Label("$\boldsymbol{L}$",EndPoint,black),O--L*dir(90),blue,Arrow);
	draw(Label("$\boldsymbol{e}_3$",EndPoint),O--L*dir(90+alpha+theta),Arrow);
	//画角速度
	real omega,r0;
	omega = 1;
	draw(Label("$\boldsymbol{\omega}$",EndPoint,Relative(E),black),O--omega*dir(90+theta),red,Arrow);
	label("$\dot{\phi}$",0.8*omega*dir(90),E);
	r0 = 0.3;
	draw(Label("$\theta$",MidPoint,Relative(E)),arc(O,r0,90,90+alpha),Arrow);
	r0 = 0.3;
	draw(Label("$\alpha$",MidPoint,Relative(E)),arc(O,r0,90+alpha,90+2*alpha),Arrow);
\end{asy}
\caption{$I_1=I_2=I_3$,$\theta=\alpha$}
\label{空间极锥与本体极锥情形2}
\end{minipage}
\\
\vspace{0.2cm}
\begin{minipage}{0.45\textwidth}
\centering
\begin{asy}
	texpreamble("\usepackage{xeCJK}");
	texpreamble("\setCJKmainfont{SimSun}");
	usepackage("amsmath");
	usepackage("amssymb");
	size(250);
	import graph;
	import math;
	//空间极锥与本体极锥情形3
	picture tmp;
	pair O,i,j,k;
	real r,r1,r2,h,h1,h2,theta,alpha,pos,L;
	O = (0,0);
	i = (-sqrt(2)/4,-sqrt(14)/12);
	j = (sqrt(14)/4,-sqrt(2)/12);
	k = (0,2*sqrt(2)/3);
	pos = 1.88;
	r1 = 0.5;
	r2 = 0.8;
	h1 = 1.6;
	h2 = sqrt(r1*r1+h1*h1-r2*r2);
	L = 2;
	pair cir(real theta){
		return r*cos(theta)*i+r*sin(theta)*j+h*k;
	}
	r = r1;
	h = h1;
	draw(graph(cir,0,2*pi),blue+dashed);
	draw(O--cir(pos),blue+dashed);
	draw(xscale(-1)*(O--cir(pos)),blue+dashed);
	theta = 90-degrees(cir(pos));
	r = r2;
	h = h2;
	pos = 1.80;
	draw(tmp,graph(cir,0,2*pi),red+dashed);
	draw(tmp,O--cir(pos),red+dashed);
	draw(tmp,xscale(-1)*(O--cir(pos)),red+dashed);
	alpha = 90-degrees(cir(pos));
	add(rotate(alpha-theta)*tmp);
	draw(Label("$\boldsymbol{L}$",EndPoint,black),O--L*dir(90),blue,Arrow);
	draw(Label("$\boldsymbol{e}_3$",EndPoint),O--L*dir(90+alpha-theta),Arrow);
	//画角速度
	real omega,phi,psi,ralpha,rtheta,r0;
	theta = -theta;
	ralpha = alpha*pi/180;
	rtheta = theta*pi/180;
	omega = 0.3;
	phi = sin(ralpha)/sin(ralpha+rtheta)*omega;
	psi = sin(rtheta)/sin(ralpha+rtheta)*omega;
	draw(Label("$\boldsymbol{\omega}$",EndPoint,Relative(E),black),O--omega*dir(90+theta),red,Arrow);
	draw(Label("$\dot{\phi}$",EndPoint,Relative(E),black),O--phi*dir(90),red,Arrow);
	draw(Label("$\dot{\psi}$",EndPoint,Relative(W),black),O--psi*dir(90+alpha+theta),red,Arrow);
	draw(psi*dir(90+alpha+theta)--omega*dir(90+theta)--phi*dir(90),dashed);
	r0 = 0.2;
	draw(Label("$\alpha$",MidPoint,Relative(E)),arc(O,r0,90+alpha+theta,90+2*alpha+theta),Arrow);
	r0 = 0.5;
	draw(Label("$\theta$",MidPoint,Relative(W)),arc(O,r0,90,90+theta),Arrow);
\end{asy}
\caption{$I_1=I_2<I_3$,$\theta<\alpha$}
\label{空间极锥与本体极锥情形3}
\end{minipage}
\end{figure}

由于本体极锥相对本体系静止,故本体极锥的运动即代表了刚体额运动,角速度线上的各点在平动系中瞬时静止,即为瞬时转轴方向。对称Euler陀螺运动的几何图像即为本体极锥贴着空间极锥作纯滚动。

\subsection{非对称Euler陀螺}

现在利用Euler动力学方程研究三个主惯量各不相等的非对称陀螺的自由转动。为方便起见,假定
\begin{equation*}
	I_3>I_2>I_1
\end{equation*}
由角动量守恒和能量守恒可得
\begin{subnumcases}{\label{非对称Euler陀螺的守恒律方程}}
	I_1^2 \omega_1^2+I_2^2 \omega_2^2+I_3^2 \omega_3^2 = L^2 \\
	I_1 \omega_1^2+I_2 \omega_2^2+I_3 \omega_3^2 = 2T
\end{subnumcases}
其中$L$和$T$为常数。这两个等式可以用$\mbf{L}$的三个分量表示为
\begin{subnumcases}{}
	L_1^2+L_2^2+L_3^2 = L^2 \label{非对称Euler陀螺-角动量守恒} \\
	\frac{L_1^2}{I_1}+\frac{L_2^2}{I_2}+\frac{L_3^2}{I_3} = 2T \label{非对称Euler陀螺-能量守恒} 
\end{subnumcases}
在以$L_1,L_2,L_3$为轴的坐标系中,方程\eqref{非对称Euler陀螺-能量守恒}是半轴分别
\begin{equation*}
	\sqrt{2I_1T},\sqrt{2I_2T},\sqrt{2I_3T}
\end{equation*}
的椭球面,而方程\eqref{非对称Euler陀螺-角动量守恒}则为半径为$L$的球面方程。当矢量$\mbf{L}$相对陀螺的主轴移动时,其端点将沿着这两个曲面的交线运动。交线存在的条件由不等式
\begin{equation}
	2I_1T<L^2<2I_3T
\end{equation}
给出,即表示球\eqref{非对称Euler陀螺-角动量守恒}的半径介于椭球\eqref{非对称Euler陀螺-能量守恒}的最长半轴和最短半轴之间。图\ref{不同情况下角动量矢量末端的轨迹}画出了椭球与不同半径的球面的一系列这样的交线。

\begin{figure}[htb]
\centering
\begin{asy}
	texpreamble("\usepackage{xeCJK}");
	texpreamble("\setCJKmainfont{SimSun}");
	usepackage("amsmath");
	import graph;
	import math;
	size(300);
	//
	pair O,i,j,k;
	real a,b,c,L;
	O = (0,0);
	i = (-sqrt(2)/4,-sqrt(14)/12);
	j = (sqrt(14)/4,-sqrt(2)/12);
	k = (0,2*sqrt(2)/3);
	a = 3;
	b = 2;
	c = 1;
	pair ell1(real t){
		return a*cos(t)*i+b*sin(t)*j;
	}
	pair ell2(real t){
		return b*cos(t)*j+c*sin(t)*k;
	}
	pair ell3(real t){
		return c*cos(t)*k+a*sin(t)*i;
	}
	real la,lb,theta;
	la = 2.175;
	lb = 1.315;
	theta = 10.5;
	draw(rotate(theta)*xscale(la)*yscale(lb)*unitcircle,linewidth(0.8bp));
	real tmp1,tmp2;
	tmp1 = -1.15;
	draw(graph(ell1,tmp1,tmp1+pi),linewidth(0.5bp));
	draw(graph(ell1,tmp1+pi,tmp1+2*pi),dashed);
	tmp1 = -0.5;
	draw(graph(ell2,tmp1,tmp1+pi),linewidth(0.5bp));
	draw(graph(ell2,tmp1+pi,tmp1+2*pi),dashed);
	tmp1 = -0.9;
	draw(graph(ell3,tmp1,tmp1+pi),linewidth(0.5bp));
	draw(graph(ell3,tmp1+pi,tmp1+2*pi),dashed);
	pair ell_cur1(real t){
		real x,y,z;
		x = a*cos(t)/sqrt((a^2-c^2)/(L^2-c^2));
		y = b*sin(t)/sqrt((b^2-c^2)/(L^2-c^2));
		z = sqrt(L^2-x^2-y^2);
		return x*i+y*j+z*k;
	}
	pair ell_cur2(real t){
		real x,y,z;
		x = a*cos(t)/sqrt((a^2-c^2)/(L^2-c^2));
		y = b*sin(t)/sqrt((b^2-c^2)/(L^2-c^2));
		z = -sqrt(L^2-x^2-y^2);
		return x*i+y*j+z*k;
	}
	L = 1.2;
	draw(graph(ell_cur1,0,2*pi),linewidth(0.8bp)+red);
	draw(graph(ell_cur2,0,2*pi),dashed+red);
	L = 1.5;
	draw(graph(ell_cur1,0,2*pi),linewidth(0.8bp)+0.7red+0.3blue);
	draw(graph(ell_cur2,0,2*pi),dashed+0.7red+0.3blue);
	L = 1.8;
	draw(graph(ell_cur1,0,2*pi),linewidth(0.8bp)+0.3red+0.7blue);
	draw(graph(ell_cur2,0,2*pi),dashed+0.3red+0.7blue);
	L = 2;
	tmp1 = -1.15;
	tmp2 = 2*pi-2.8;
	draw(graph(ell_cur1,tmp1,tmp2),linewidth(0.8bp)+blue);
	draw(graph(ell_cur1,tmp2,2*pi+tmp1),dashed+blue);
	draw(graph(ell_cur2,pi+tmp1,pi+tmp2),dashed+blue);
	draw(graph(ell_cur2,pi+tmp2,3*pi+tmp1),linewidth(0.8bp)+blue);
	pair ell_cur3(real t){
		real x,y,z;
		y = b*cos(t)/sqrt((a^2-b^2)/(a^2-L^2));
		z = c*sin(t)/sqrt((a^2-c^2)/(a^2-L^2));
		x = sqrt(L^2-y^2-z^2);
		return x*i+y*j+z*k;
	}
	pair ell_cur4(real t){
		real x,y,z;
		y = b*cos(t)/sqrt((a^2-b^2)/(a^2-L^2));
		z = c*sin(t)/sqrt((a^2-c^2)/(a^2-L^2));
		x = -sqrt(L^2-y^2-z^2);
		return x*i+y*j+z*k;
	}
	L = 2.2;
	draw(graph(ell_cur3,0,2*pi),linewidth(0.8bp)+0.8blue+0.2green);
	draw(graph(ell_cur4,0,2*pi),dashed+0.8blue+0.2green);
	L = 2.5;
	draw(graph(ell_cur3,0,2*pi),linewidth(0.8bp)+0.5blue+0.5green);
	draw(graph(ell_cur4,0,2*pi),dashed+0.5blue+0.5green);
	L = 2.8;
	draw(graph(ell_cur3,0,2*pi),linewidth(0.8bp)+green);
	draw(graph(ell_cur4,0,2*pi),dashed+green);
	// draw(O--a*i,dashed);
	// draw(O--b*j,dashed);
	// draw(O--c*k,dashed);
	draw(Label("$x_3$",EndPoint),a*i--(a+2)*i,Arrow);
	draw(Label("$x_2$",EndPoint),b*j--(b+0.6)*j,Arrow);
	draw(Label("$x_1$",EndPoint),c*k--(c+1)*k,Arrow);
\end{asy}
\caption{不同情况下角动量矢量末端的轨迹}
\label{不同情况下角动量矢量末端的轨迹}
\end{figure}

现在考虑在给定能量$T$时,$L$的变化引起矢量$\mbf{L}$的端点的轨迹性质的变化。当$L^2$略大于$2I_1T$时,球和椭球相交于椭球极点附近围绕$x_1$轴的两条很小的封闭曲线(图\ref{不同情况下角动量矢量末端的轨迹}中红色线)。当$L^2\to 2I_1T$时,两条曲线分别收缩到极点。随着$L^2$的增大,曲线也随之扩大,当$L^2=2I_2T$时,曲线变成两条平面曲线(椭圆,图\ref{不同情况下角动量矢量末端的轨迹}中蓝色线),并相交于椭球在$x_2$轴上的极点。$L^2$再继续增大,将再次出现两条分离的封闭曲线,分别围绕$x_3$轴的两个极点(图\ref{不同情况下角动量矢量末端的轨迹}中绿色线)。而当$L^2\to 2I_3T$时,这两条曲线再次收缩到两个极点。

首先,轨迹的封闭性意味着矢量$\mbf{L}$在本体系中的运动是周期性的。其次,在椭球不同极点附近的轨迹性质具有本质的区别。在$x_1$轴和$x_3$轴附近,轨迹完全位于相应的极点周围,而在$x_2$轴极点附近的轨迹将远离这个极点。这种差别即相应于陀螺绕三个惯量主轴转动时有不同的稳定性。绕$x_1$轴和$x_3$轴(对应于陀螺三个主惯量中的最小值和最大值)的转动是稳定的,即如果使陀螺稍微偏离这些状态时,陀螺将继续在初始状态附近运动。而绕$x_2$轴的转动是不稳定的,即任意小的偏离都足以引起陀螺远离其初始位置运动。关于稳定性的更多讨论请见%第\ref{Euler陀螺的转动稳定性}节。
下文。

为了得到$\mbf{\omega}$分量对时间的依赖关系,利用方程\eqref{非对称Euler陀螺的守恒律方程}将$\omega_1$和$\omega_3$用$\omega_2$表示,可得
\begin{subnumcases}{}
	\omega_1^2 = \frac{(2I_3T-L^2)-I_2(I_3-I_2)\omega_2^2}{I_1(I_3-I_1)} \\
	\omega_3^2 = \frac{(L^2-2I_1T)-I_2(I_2-I_1)\omega_2^2}{I_3(I_3-I_1)}
\end{subnumcases}
将上述两个关系代入Euler动力学方程\eqref{Euler动力学方程-2}中,即有
\begin{align}
	\frac{\mathrm{d}\omega_2}{\mathrm{d}t} & = \frac{\sqrt{\big[(2I_3T-L^2)-I_2(I_3-I_2)\omega_2^2\big] \big[(L^2-2I_1T)-I_2(I_2-I_1)\omega_2^2\big]}}{I_2\sqrt{I_1I_3}}
	\label{非对称Euler陀螺的运动方程}
\end{align}
方程\eqref{非对称Euler陀螺的运动方程}可以分离变量,将其积分即可得到关系$t(\omega_2)$。为了将其化为标准形式,明确起见,首先假设
\begin{equation*}
	L^2>2I_2T
\end{equation*}
作变量代换
\begin{equation}
	\tau = \sqrt{\frac{(I_3-I_2)(L^2-2I_1T)}{I_1I_2I_3}}t,\quad s = \sqrt{\frac{I_2(I_3-I_2)}{2I_3T-L^2}}\omega_2
\end{equation}
并引入参数
\begin{equation}
	k^2 = \frac{(I_2-I_1)(2I_3T-L^2)}{(I_3-I_2)(L^2-2I_1T)}<1
\end{equation}
可将方程\eqref{非对称Euler陀螺的运动方程}化为
\begin{equation*}
	\frac{\mathrm{d}s}{\mathrm{d}\tau} = \sqrt{(1-s^2)(1-k^2s^2)}
\end{equation*}
将其分离变量并积分,可得
\begin{equation}
	\tau = \int_0^s \frac{\mathrm{d}\xi}{\sqrt{(1-\xi^2)(1-k^2\xi^2)}} = K(s,k)
	\label{非对称Euler陀螺的运动方程的解1}
\end{equation}
此处选择$\omega_2=0$的时刻为时间起点。式\eqref{非对称Euler陀螺的运动方程的解1}中右端的函数为{\bf 第一类不完全椭圆积分}\footnote{关于椭圆积分和Jacobi椭圆函数的相关知识,请见第\ref{椭圆积分与Jacobi椭圆函数}节。}。求其反函数可得Jacobi椭圆函数
\begin{equation*}
	s = \sn \tau
\end{equation*}
即
\begin{equation}
	\omega_2 = \sqrt{\frac{2I_3T-L^2}{I_2(I_3-I_2)}}\sn \tau = \sqrt{\frac{2I_3T-L^2}{I_2(I_3-I_2)}}\sn \left(\sqrt{\frac{(I_3-I_2)(L^2-2I_1T)}{I_1I_2I_3}}t\right)
\end{equation}
由此可得如下公式
\begin{equation}
\begin{cases}
	\displaystyle \omega_1 = \sqrt{\frac{2I_3T-L^2}{I_1(I_3-I_1)}}\cn \tau = \sqrt{\frac{2I_3T-L^2}{I_1(I_3-I_1)}}\cn \left(\sqrt{\frac{(I_3-I_2)(L^2-2I_1T)}{I_1I_2I_3}}t\right) \\
	\displaystyle \omega_2 = \sqrt{\frac{2I_3T-L^2}{I_2(I_3-I_2)}}\sn \tau = \sqrt{\frac{2I_3T-L^2}{I_2(I_3-I_2)}}\sn \left(\sqrt{\frac{(I_3-I_2)(L^2-2I_1T)}{I_1I_2I_3}}t\right) \\
	\displaystyle \omega_3 = \sqrt{\frac{L^2-2I_1T}{I_3(I_3-I_1)}}\dn \tau = \sqrt{\frac{L^2-2I_1T}{I_3(I_3-I_1)}}\dn \left(\sqrt{\frac{(I_3-I_2)(L^2-2I_1T)}{I_1I_2I_3}}t\right)
\end{cases}
\label{非对称Euler陀螺的运动方程的解2}
\end{equation}
函数$\sn \tau$是周期的,对变量$\tau$的周期为$4K$,其中$K$可用第一类完全椭圆积分表示为
\begin{equation*}
	K = \int_0^1 \frac{\mathrm{d}s}{\sqrt{(1-s^2)(1-k^2s^2)}} = \int_0^{\frac{\pi}{2}} \frac{\mathrm{d}\phi}{\sqrt{1-k^2\sin^2\phi}}
\end{equation*}
因而,解\eqref{非对称Euler陀螺的运动方程的解2}对时间$t$的周期为
\begin{equation}
	T_0 = 4K\sqrt{\frac{I_1I_2I_3}{(I_3-I_2)(L^2-2I_1T)}}
\end{equation}
即经过时间$T_0$后,矢量$\mbf{\omega}$在本体系中回到原位置,然而在空间系中,陀螺自身则不见得会回到原位。

下面考虑陀螺在空间系中的绝对运动,即相对于固定坐标系$OXYZ$的运动。令$Z$轴沿着矢量$\mbf{L}$的方向,由于方向$Z$相对$x_1,x_2,x_3$轴的极角和方位角分别为$\theta$和$\dfrac{\pi}{2}-\psi$\footnote{可见图\ref{刚体定点转动的Euler角}。此处“极角”和“方位角”即为球坐标系
\begin{equation*}
\begin{cases}
	x = r\sin\theta\cos\phi\\
	y = r\sin\theta\sin\phi\\
	z = r\cos\theta
\end{cases}
\end{equation*}
中的$\theta$和$\phi$。},则矢量$\mbf{L}$沿$x_1,x_2,x_3$的分量分别为
\begin{subnumcases}{\label{非对称Euler陀螺的Euler角-1}}
	I_1\omega_1 = L\sin\theta\sin\psi \\
	I_2\omega_2 = L\sin\theta\cos\psi \\
	I_3\omega_3 = L\cos\theta
\end{subnumcases}
由此可得
\begin{equation}
	\cos \theta = \frac{I_3\omega_3}{L},\quad \tan\psi = \frac{I_1\omega_1}{I_2\omega_2}
	\label{非对称Euler陀螺的Euler角-2}
\end{equation}
将式\eqref{非对称Euler陀螺的运动方程的解2}的结果代入,即有
\begin{equation}
\begin{cases}
	\displaystyle \cos \theta = \sqrt{\frac{I_3(L^2-2I_1T)}{L^2(I_3-I_1)}}\dn \tau \\
	\displaystyle \tan \psi = \sqrt{\frac{I_1(I_3-I_2)}{I_2(I_3-I_1)}}\frac{\cn \tau}{\sn \tau}
\end{cases}
\end{equation}
由此获得角$\theta$和$\psi$随时间的关系,它们都是周期为$T_0$的函数。为了获得角$\phi$随时间的关系,根据Euler运动学方程\eqref{Euler运动学方程}可有
\begin{equation*}
\begin{cases}
	\omega_1 = \dot{\phi}\sin \theta\sin \psi+\dot{\theta}\cos \psi \\
	\omega_2 = \dot{\phi}\sin \theta\cos \psi-\dot{\theta}\sin \psi
\end{cases}
\end{equation*}
从中解得
\begin{equation*}
	\dot{\phi} = \frac{\omega_1\sin\psi+\omega_2\cos\psi}{\sin \theta}
\end{equation*}
将式\eqref{非对称Euler陀螺的Euler角-1}代入,即有
\begin{equation}
	\frac{\mathrm{d}\phi}{\mathrm{d}t} = L\frac{I_1\omega_1^2+I_2\omega_2^2}{L^2-I_3^2\omega_3^2} = L\frac{I_1\omega_1^2+I_2\omega_2^2}{I_1^2\omega_1^2+I_2^2\omega_2^2}
	\label{非对称Euler陀螺的Euler角-3}
\end{equation}
由此便可确定$\phi$与时间的关系$\phi(t)$,这个解可以用$\varTheta$-函数表示,而且是非周期的,因此,陀螺在经过$T_0$时间后,其角速度矢量$\mbf{\omega}$相对本体系回到原位,但陀螺本身在空间系中却不能回到它的初始位置。

当
\begin{equation*}
	L^2<2I_2T
\end{equation*}
时,作变量代换
\begin{equation}
	\tau = \sqrt{\frac{(I_2-I_1)(2I_3T-L^2)}{I_1I_2I_3}}t,\quad s = \sqrt{\frac{I_2(I_2-I_1)}{L^2-2I_1T}}\omega_2
\end{equation}
并引入参数
\begin{equation}
	k^2 = \frac{(I_3-I_2)(L^2-2I_1T)}{(I_2-I_1)(2I_3T-L^2)}<1
\end{equation}
可将方程\eqref{非对称Euler陀螺的运动方程}化为
\begin{equation*}
	\frac{\mathrm{d}s}{\mathrm{d}\tau} = \sqrt{(1-s^2)(1-k^2s^2)}
\end{equation*}
将其积分可得
\begin{equation*}
	s = \sn \tau
\end{equation*}
此时的解为
\begin{equation}
\begin{cases}
	\displaystyle \omega_1 = \sqrt{\frac{2I_3T-L^2}{I_1(I_3-I_1)}}\dn \tau = \sqrt{\frac{2I_3T-L^2}{I_1(I_3-I_1)}}\dn \left(\sqrt{\frac{(I_2-I_1)(2I_3T-L^2)}{I_1I_2I_3}}t\right) \\
	\displaystyle \omega_2 = \sqrt{\frac{L^2-2I_1T}{I_2(I_2-I_1)}}\sn \tau = \sqrt{\frac{L^2-2I_1T}{I_2(I_2-I_1)}}\sn \left(\sqrt{\frac{(I_2-I_1)(2I_3T-L^2)}{I_1I_2I_3}}t\right) \\
	\displaystyle \omega_3 = \sqrt{\frac{L^2-2I_1T}{I_3(I_3-I_1)}}\cn \tau = \sqrt{\frac{L^2-2I_1T}{I_3(I_3-I_1)}}\cn \left(\sqrt{\frac{(I_2-I_1)(2I_3T-L^2)}{I_1I_2I_3}}t\right)
\end{cases}
\label{非对称Euler陀螺的运动方程的解3}
\end{equation}

当$I_1=I_2$时,考虑当$I_1\to I_2$时,参数$k^2\to 0$,椭圆函数退化为三角函数,级
\begin{equation*}
	\sn \tau \to \sin \tau,\quad \cn\tau \to \cos \tau,\quad \dn \tau \to 1
\end{equation*}
于是,解\eqref{非对称Euler陀螺的运动方程的解2}就变回到式\eqref{对称Euler陀螺的解-1}和式\eqref{对称Euler陀螺的解-2}的结果。

当$L^2=2I_3T$时,有
\begin{equation*}
	\omega_1 = \omega_2 = 0,\quad \omega_3 = \sqrt{\frac{2T}{I_3}}\,\,\text{(常数)}
\end{equation*}
即矢量$\mbf{\omega}$的方向总是沿着对称轴$x_3$,即陀螺绕$x_3$轴匀速转动。类似地,当$L^2=2I_1T$时,有
\begin{equation*}
	\omega_1 = \sqrt{\frac{2T}{I_1}}\,\,\text{(常数)},\quad \omega_2 = \omega_3 = 0
\end{equation*}

当$L^2=2I_2T$时,式\eqref{非对称Euler陀螺的运动方程}变为
\begin{equation}
	\frac{\mathrm{d}\omega_2}{\mathrm{d}t} = \sqrt{\left(\frac{I_2}{I_1}-1\right)\left(1-\frac{I_2}{I_3}\right)}\left(\frac{2T}{I_2}-\omega_2^2\right)
	\label{Euler陀螺的稳定性-5}
\end{equation}
记$\varOmega = \sqrt{\dfrac{2T}{I_2}}$,作变量代换
\begin{equation}
	\tau = \varOmega\sqrt{\left(\frac{I_2}{I_1}-1\right)\left(1-\frac{I_2}{I_3}\right)} t,\quad s = \frac{\omega_2}{\varOmega}
	\label{Euler陀螺的稳定性-6}
\end{equation}
可将方程\eqref{Euler陀螺的稳定性-5}化为
\begin{equation*}
	\frac{\mathrm{d}s}{\mathrm{d}\tau} = 1-s^2
\end{equation*}
将其积分可得
\begin{equation}
	s = \tanh \tau
\end{equation}
由此可得
\begin{equation}
\begin{cases}
	\displaystyle \omega_1 = \sqrt{\frac{I_2(I_3-I_2)}{I_1(I_3-I_1)}}\varOmega \frac{1}{\cosh \tau} \\
	\displaystyle \omega_2 = \varOmega \tanh \tau \\
	\displaystyle \omega_3 = \sqrt{\frac{I_2(I_2-I_1)}{I_3(I_3-I_1)}}\varOmega \frac{1}{\cosh \tau}
	\label{Euler陀螺的稳定性-7}
\end{cases}
\end{equation}
其中$\displaystyle \varOmega = \sqrt{\frac{2T}{I_2}},\tau = \varOmega\sqrt{\left(\frac{I_2}{I_1}-1\right)\left(1-\frac{I_2}{I_3}\right)} t$。如果我们令$\mbf{L}$的方向为$Z$轴,$\theta$为$Z$轴与$x_2$轴之间的夹角,则式\eqref{非对称Euler陀螺的Euler角-2}和式\eqref{非对称Euler陀螺的Euler角-3}中只要对下标做置换$123\to 312$即有
\begin{equation}
\begin{cases}
	\displaystyle \cos\theta = \frac{I_2\omega_2}{L} \\[1.5ex]
	\displaystyle \tan\psi = \frac{I_3\omega_3}{I_1\omega_1} \\[1.5ex]
	\displaystyle \dot{\phi} = L\frac{I_1\omega_1^2+I_3\omega_3^2}{I_1^2\omega_1^2+I_3^2\omega_3^2}
\end{cases}
\label{非对称Euler陀螺自由转动特殊情形-1}
\end{equation}
将式\eqref{Euler陀螺的稳定性-7}带入即有
\begin{equation}
\begin{cases}
	\displaystyle \cos\theta = \tanh \tau \\
	\displaystyle \tan\psi = \sqrt{\frac{I_3(I_2-I_1)}{I_1(I_3-I_2)}} \\
	\displaystyle \phi = \varOmega t+\phi_0
\end{cases}
\end{equation}
由上面各式可以看出,当$t\to \infty$时,矢量$\mbf{\omega}$渐近地趋于$x_2$轴,同时$x_2$轴也渐近地趋于固定轴$Z$。

当刚体的角动量$\mbf{L}$与$x_1$轴有微小偏离时\footnote{此时即$\sqrt{2I_3T-L^2}$为一阶小量。},那么$L_2,L_3$是一阶小量。根据Euler动力学方程\eqref{Euler陀螺的运动方程}可得
\begin{subnumcases}{}
	I_1\dot{\omega}_1-(I_2-I_3)\omega_2\omega_3 = 0 \label{Euler陀螺的稳定性-1.1} \\
	I_2\dot{\omega}_2-(I_3-I_1)\omega_1\omega_3 = 0 \label{Euler陀螺的稳定性-1.2} \\
	I_3\dot{\omega}_3-(I_1-I_2)\omega_1\omega_2 = 0 \label{Euler陀螺的稳定性-1.3}
\end{subnumcases}
由于$L_2,L_3$是一阶小量,故$\omega_2,\omega_3$也是一阶小量,故由方程\eqref{Euler陀螺的稳定性-1.1}可得
\begin{equation*}
	I_1\dot{\omega}_1 = 0
\end{equation*}
即有
\begin{equation*}
	\omega_1 = \varOmega \quad \text{(常数)}
\end{equation*}
而式\eqref{Euler陀螺的稳定性-1.2}和\eqref{Euler陀螺的稳定性-1.3}则可写作
\begin{equation*}
\begin{cases}
	\dot{\omega}_2 = \dfrac{I_3-I_1}{I_2}\varOmega\omega_3 \\
	\dot{\omega}_3 = \dfrac{I_1-I_2}{I_3}\varOmega\omega_2
\end{cases}
\end{equation*}
由此可以解得
\begin{equation*}
\begin{cases}
	\displaystyle \omega_2 = \frac{La}{I_2} \sqrt{1-\frac{I_1}{I_2}} \sin (\omega_0t+\phi_0) \\
	\displaystyle \omega_3 = \frac{La}{I_2} \sqrt{1-\frac{I_1}{I_3}} \cos (\omega_0t+\phi_0)
\end{cases}
\end{equation*}
其中$a$是一个小常数(一阶小量),$\omega_0 = \sqrt{\left(1-\dfrac{I_1}{I_2}\right)\left(1-\dfrac{I_1}{I_3}\right)}\varOmega$。由于
\begin{equation*}
	\mbf{L} = I_1\omega_1\mbf{e}_1+I_2\omega_2\mbf{e}_2+I_3\omega_3\mbf{e}_3
\end{equation*}
则有
\begin{equation}
\begin{cases}
	\displaystyle L_1 = I_1\varOmega \approx L \\
	\displaystyle L_2 = La\sqrt{1-\frac{I_1}{I_2}} \sin (\omega_0t+\phi_0) \\
	\displaystyle L_3 = La\sqrt{1-\frac{I_1}{I_3}} \cos (\omega_0t+\phi_0)
\end{cases}
\label{Euler陀螺的稳定性-2}
\end{equation}
式\eqref{Euler陀螺的稳定性-2}表明,矢量$\mbf{L}$的端点,以圆频率$\omega_0$绕着$x_1$轴上的极点画出小椭圆(即图\ref{不同情况下角动量矢量末端的轨迹}中的红色线)。

对于刚体的角动量$\mbf{L}$与$x_3$轴有微小偏离的情形,根据类似的过程可以得到
\begin{equation}
\begin{cases}
	\displaystyle \omega_1 = \frac{La}{I_1}\sqrt{\frac{I_3}{I_2}-1} \sin (\omega_0t+\phi_0) \\
	\displaystyle \omega_2 = \frac{La}{I_2}\sqrt{\frac{I_3}{I_1}-1} \sin (\omega_0t+\phi_0) \\
	\displaystyle \omega_3 = \varOmega
\end{cases}
\label{Euler陀螺的稳定性-3}
\end{equation}
其中$a$是一个小常数(一阶小量),$\omega_0 = \sqrt{\left(\dfrac{I_3}{I_1}-1\right)\left(\dfrac{I_3}{I_2}-1\right)}\varOmega$,对角动量则有
\begin{equation}
\begin{cases}
	\displaystyle L_1 = La\sqrt{\frac{I_3}{I_2}-1} \sin (\omega_0t+\phi_0) \\
	\displaystyle L_2 = La\sqrt{\frac{I_3}{I_1}-1} \sin (\omega_0t+\phi_0) \\
	\displaystyle L_3 = I_3\varOmega \approx L
\end{cases}
\label{Euler陀螺的稳定性-4}
\end{equation}
式\eqref{Euler陀螺的稳定性-4}表明,矢量$\mbf{L}$的端点,以圆频率$\omega_0$绕着$x_3$轴上的极点画出小椭圆(即图\ref{不同情况下角动量矢量末端的轨迹}中的绿色线)。

以上的讨论表明,非对称Euler陀螺绕其主惯量最大或最小的轴自由旋转的运动是稳定的。

而对于刚体的角动量$\mbf{L}$与$x_2$轴有微小偏离的情形,精确到一阶小量可得
\begin{subnumcases}{}
	\dot{\omega}_1 = \frac{I_2-I_3}{I_1}\varOmega\omega_3 \\
	\dot{\omega}_3 = \frac{I_1-I_2}{I_3}\varOmega\omega_1
\end{subnumcases}
从中消去其中$\omega_1$或$\omega_3$可得
\begin{subnumcases}{}
	\ddot{\omega}_1 - \frac{(I_2-I_1)(I_3-I_2)}{I_1I_3}\varOmega^2\omega_1 = 0 \\
	\ddot{\omega}_3 - \frac{(I_2-I_1)(I_3-I_2)}{I_1I_3}\varOmega^2\omega_3 = 0
\end{subnumcases}
由于$I_1<I_2<I_3$,故有
\begin{equation*}
	- \frac{(I_2-I_1)(I_3-I_2)}{I_1I_3}\varOmega^2 < 0
\end{equation*}
因此,非对称Euler陀螺绕其主惯量取中间值的轴自由旋转的运动是不稳定的。

\subsection{Poinsot几何图像}

\begin{figure}[htb]
\centering
\begin{asy}
	texpreamble("\usepackage{xeCJK}");
	texpreamble("\setCJKmainfont{SimSun}");
	usepackage("amsmath");
	import graph;
	import math;
	size(250);
	//角速度的方向与角动量的方向
	pair O,a,b,L,omega;
	real rx,ry,theta,lL,lomega;
	path rect,cir;
	picture tmp1,tmp2;
	O = (0,0);
	a = (2,0);
	b = (0.7,0.8);
	L = (0,-0.9);
	omega = L+0.7*dir(190);
	lL = 2.2;
	lomega = lL/cos((degrees(omega)-degrees(L))*pi/180);
	rx = 1.5;
	ry = 1;
	theta = 50;
	rect = L+a+b--L+a-b--L-a-b--L-a+b--cycle;
	cir = rotate(theta)*xscale(rx)*yscale(ry)*unitcircle;
	dot(O);
	label("$O$",O,N);
	draw(cir);
	draw(tmp1,rect);
	unfill(tmp1,cir);
	add(tmp1);
	erase(tmp1);
	draw(tmp1,O--lL*L,blue+dashed,Arrow);
	draw(tmp1,O--lomega*omega,red+dashed,Arrow);
	add(tmp2,tmp1);
	clip(tmp1,rect);
	add(tmp1);
	clip(tmp2,cir);
	unfill(tmp2,rect);
	add(tmp2);
	erase(tmp1);
	erase(tmp2);
	draw(tmp1,O--lL*dir(L),blue,Arrow);
	draw(tmp1,Label("$\boldsymbol{\omega}$",EndPoint,S,black),O--lomega*dir(omega),red,Arrow);
	label(tmp1,"$\boldsymbol{L}$(常矢量)",lL*dir(L)+(0.6,0),S);
	unfill(tmp1,rect);
	unfill(tmp1,cir);
	add(tmp1);
	erase(tmp1);
	dot(omega);
	dot(L);
	label("$N$",omega,N);
	label("$L$",L,NE);
	draw(L--omega,dashed);
\end{asy}
\caption{Poinsot几何图像}
\label{Poinsot几何图像示意}
\end{figure}

根据惯量椭球的几何意义,角动量$\mbf{L}$沿角速度$\mbf{\omega}$与惯量椭球交点处切平面的法线方向。对于Euler陀螺,角动量守恒,即$\mbf{L}$为常矢量,因此在Euler陀螺运动的过程中,切平面在各个时刻是相互平行的。再考虑到惯量椭球中心到切平面的距离为
\begin{equation*}
	x_N = \mbf{x}_N \cdot \frac{\mbf{L}}{L} = \frac{\mbf{\omega}}{\sqrt{2T}} \cdot \frac{\mbf{L}}{L} = \frac{\sqrt{2T}}{L}\quad \text{(常数)}
\end{equation*}
因此,Euler陀螺在运动过程中,惯量椭球极点处的切平面固定,称为{\heiti Poinsot平面}。

于是得到Euler陀螺的Poinsot几何解释:惯量椭球随着刚体的运动在Poinsot平面上纯滚动,极点为瞬时接触点,其瞬时速度为零\footnote{由于极点在角速度矢量$\mbf{\omega}$方向上,故极点速度为零。}。

在刚体运动时,极点在惯性椭球上画出的曲线称为{\heiti 本体极迹},相应地极点在Poinsot平面上画出的曲线称为{\heiti 空间极迹}。显然,本体极迹是刚体定点运动中本体极锥的准线,而空间极迹是刚体定点运动的空间极锥的准线。

\iffalse
\subsection{Euler陀螺的转动稳定性}\label{Euler陀螺的转动稳定性}

如果Euler陀螺的初始角速度沿其任一主轴,即$\mbf{\omega} = \omega \mbf{e}_i$,则有
\begin{equation*}
	\mbf{L} = I_i \omega \mbf{e}_i = I_i \mbf{\omega}
\end{equation*}
此时瞬时转动轴与Poinsot平面垂直,空间极迹和本体极迹退化为一点,角速度(瞬时转动轴)不变,刚体做定轴转动。

下面考虑这种定轴转动在刚体角速度稍微偏离主轴时的稳定性。将初始转轴的方向取为$z$轴,当角速度方向稍微偏离$z$轴时,有$\omega_1,\omega_2 \ll 1$,根据Euler动力学方程\eqref{Euler动力学方程-3}可有
\begin{equation*}
	I_3 \dot{\omega}_3 +(I_1-I_2) \omega_1 \omega_2 = 0
\end{equation*}
即有$I_3 \dot{\omega}_3 \approx 0$,所以有
\begin{equation*}
	\omega_3 = \varOmega \quad \text{(常数)}
\end{equation*}
由此Euler动力学方程\eqref{Euler动力学方程-1}和\eqref{Euler动力学方程-2}可有
\begin{equation*}
	\begin{cases}
		I_1 \dot{\omega}_1 - (I_2-I_3)\varOmega \omega_2 = 0 \\
		I_2 \dot{\omega}_2 - (I_3-I_1)\varOmega \omega_1 = 0
	\end{cases}
\end{equation*}
分别消去$\omega_1$和$\omega_2$可得
\begin{subnumcases}{\label{Euler陀螺的转动稳定性}}
	\ddot{\omega}_1 + \frac{(I_1-I_3)(I_2-I_3)\varOmega^2}{I_1I_2} \omega_1 = 0 \\
	\ddot{\omega}_2 + \frac{(I_1-I_3)(I_2-I_3)\varOmega^2}{I_1I_2} \omega_2 = 0
\end{subnumcases}
当$I_1>I_3,\,I_2>I_3$或者$I_1<I_3,\,I_2<I_3$时,$\dfrac{(I_1-I_3)(I_2-I_3)\varOmega^2}{I_1I_2} > 0$,方程\eqref{Euler陀螺的转动稳定性}为两个谐振方程,角速度的另外两个分量在小范围内周期性变化,因而转动是稳定的。

综上,主惯量最大或最小的主轴为稳定转动轴,而主惯量取中间值的主轴为不稳定转动轴。\fi

\section{对称重陀螺(Lagrange陀螺)}

满足如下条件的定点转动刚体称为{\heiti 对称重陀螺},又称为{\heiti Lagrange陀螺}:
\begin{enumerate}
	\item 对定点的两个主惯量相等,即$I_1=I_2$;
	\item 质心在第三主轴上,且与定点不重合,即$l = OC \neq 0$;
	\item 刚体仅受重力矩作用。
\end{enumerate}

\begin{figure}[htb]
\centering
\begin{asy}
	texpreamble("\usepackage{xeCJK}");
	texpreamble("\setCJKmainfont{SimSun}");
	usepackage("amsmath");
	import graph;
	import math;
	size(250);
	//Lagrange陀螺的运动
	pair O,r,i,j,k,C;
	picture tmp;
	O = (0,0);
	i = 0.5*dir(-135);
	j = dir(0);
	k = dir(90);
	draw(tmp,Label("$X$",EndPoint,SE,black),O--i,blue,Arrow);
	draw(tmp,Label("$Y$",EndPoint,black),O--j,blue,Arrow);
	draw(tmp,Label("$Z$",EndPoint,black),O--1.6*k,blue,Arrow);
	label(tmp,"$O$",O,SSE);
	add(tmp);
	erase(tmp);
	real f(real x){
		return x**2;
	}
	draw(tmp,graph(f,-1.5,1.5),linewidth(1bp));
	draw(tmp,shift((0,f(1.5)))*yscale(1/3)*scale(1.5)*unitcircle,linewidth(1bp));
	add(rotate(25)*xscale(0.8)*scale(0.5)*tmp);
	erase(tmp);
	draw(tmp,Label("$x$",EndPoint,S,black),rotate(25)*(O--i),red,Arrow);
	draw(tmp,Label("$y$",EndPoint,black),rotate(25)*(O--j),red,Arrow);
	draw(tmp,Label("$z$",EndPoint,W,black),rotate(25)*(O--1.6*k),red,Arrow);
	add(tmp);
	real r;
	r = 0.3;
	C = rotate(25)*(0.8*k);
	draw(Label("$\theta$",MidPoint,Relative(E)),arc(O,r,90,90+25),Arrow);
	dot(C);
	label("$C$",C,E);
	draw(Label("$Mg$",EndPoint,W),C--C+0.8*dir(-90),Arrow);
\end{asy}
\caption{Lagrange陀螺的运动}
\label{Lagrange陀螺的运动}
\end{figure}

\subsection{Lagrange函数与守恒量}

刚体的动能和势能分别为
\begin{equation*}
	T = \frac12 \sum_{i=1}^3 I_i \omega_i^2,\quad V = Mgl\cos \theta
\end{equation*}
其中角速度$\omega_1,\omega_2,\omega_3$与广义坐标和广义速度的关系由Euler运动学方程\eqref{Euler运动学方程}给出
\begin{equation*}
\begin{cases}
	\omega_1 = \dot{\phi} \sin \theta \sin \psi + \dot{\theta} \cos \psi \\
	\omega_2 = \dot{\phi} \sin \theta \cos \psi - \dot{\theta} \sin \psi \\
	\omega_3 = \dot{\phi} \cos \theta + \dot{\psi}
\end{cases}
\end{equation*}
因此Lagrange陀螺的Lagrange函数\footnote{Lagrange函数和角动量矢量的符号都是$L$,但因为角动量为矢量,是黑体的$L$,因此不会产生混淆。}为
\begin{equation}
	L = T-V = \frac12 I_1 (\dot{\theta}^2 + \dot{\phi}^2 \sin^2 \theta) + \frac12 I_3 (\dot{\psi} + \dot{\phi} \cos \theta)^2 - Mgl\cos \theta
	\label{Lagrange陀螺的Lagrange函数}
\end{equation}
由$\dfrac{\pl L}{\pl \phi} = 0$可得
\begin{equation}
	p_\phi = \frac{\pl L}{\pl \dot{\phi}} = (I_1 \sin^2 \theta + I_3 \cos^2 \theta) \dot{\phi} + I_3 \dot{\psi} \cos \theta = L_Z \quad \text{(常数)}
	\label{Lagrange陀螺守恒量——Z轴角动量}
\end{equation}
由$\dfrac{\pl L}{\pl \psi} = 0$可得
\begin{equation}
	p_\psi = \frac{\pl L}{\pl \dot{\psi}} = I_3(\dot{\psi}+\dot{\phi}\cos \theta) = L_3 \quad \text{(常数)}
	\label{Lagrange陀螺守恒量——3轴角动量}
\end{equation}
由$\dfrac{\pl L}{\pl t} = 0$可得
\begin{equation}
	H = \frac12 I_1 (\dot{\theta}^2 + \dot{\phi}^2 \sin^2 \theta) + \frac12 I_3 (\dot{\psi} + \dot{\phi} \cos \theta)^2 + Mgl\cos \theta = E \quad \text{(常数)}
	\label{Lagrange陀螺守恒量——能量}
\end{equation}

\subsection{等效势与求解}

由式\eqref{Lagrange陀螺守恒量——Z轴角动量}和式\eqref{Lagrange陀螺守恒量——3轴角动量}中可以得到
\begin{subnumcases}{\label{Lagrange陀螺的进动角速度和自转角速度}}
	\dot{\phi} = \frac{L_Z - L_3 \cos \theta}{I_1 \sin^2 \theta} \label{Lagrange陀螺的进动角速度和自转角速度1} \\
	\dot{\psi} = \frac{L_3}{I_3} - \frac{(L_Z-L_3 \cos \theta)\cos \theta}{I_1 \sin^2 \theta} \label{Lagrange陀螺的进动角速度和自转角速度2}
\end{subnumcases}
将式\eqref{Lagrange陀螺的进动角速度和自转角速度1}和式\eqref{Lagrange陀螺的进动角速度和自转角速度2}代入式\eqref{Lagrange陀螺守恒量——能量}中,可得
\begin{equation}
	\frac12 I_1 \dot{\theta}^2 + Mgl\cos \theta + \frac{(L_Z-L_3 \cos \theta)^2}{2I_1\sin^2 \theta} + \frac{L_3^2}{2I_3} = E
\end{equation}
令
\begin{equation}
	U(\theta) = Mgl\cos \theta + \frac{(L_Z-L_3 \cos \theta)^2}{2I_1\sin^2 \theta} + \frac{L_3^2}{2I_3}
	\label{Lagrange陀螺等效势}
\end{equation}
为重力与惯性力合力矩的{\heiti 等效势},由此对$\theta$的微分方程为
\begin{equation}
	\frac12 I_1 \dot{\theta}^2 + U(\theta) = E
	\label{Lagrange陀螺章动角方程}
\end{equation}
因此,Lagrange陀螺的章动相当于在等效保守力矩作用下的定轴转动。

方程\eqref{Lagrange陀螺章动角方程}可以分离变量为
\begin{equation*}
	\int_{t_0}^t \mathrm{d} t = \pm \sqrt{\frac{I_1}{2}} \int_{\theta_0}^\theta \frac{\mathrm{d} \theta}{\sqrt{E-U(\theta)}}
\end{equation*}
由此可得$\theta = \theta(t)$,于是由\eqref{Lagrange陀螺的进动角速度和自转角速度}可有
\begin{equation*}
	\begin{cases}
		\displaystyle \phi-\phi_0 = \int_{t_0}^t \frac{L_Z - L_3 \cos \theta(t)}{I_1 \sin^2 \theta(t)} \mathrm{d} t \\[1.5ex]
		\displaystyle \psi-\psi_0 = \int_{t_0}^t \left[\frac{L_3}{I_3} - \frac{\big(L_Z-L_3\cos \theta(t) \big) \cos \theta(t)}{I_1 \sin^2 \theta(t)} \right] \mathrm{d} t
	\end{cases}
\end{equation*}

\subsection{定性图像}

对等效势
\begin{equation*}
	U(\theta) = Mgl\cos \theta + \frac{(L_Z-L_3 \cos \theta)^2}{2I_1 \sin^2 \theta} + \frac{L_3^2}{2I_3}
\end{equation*}
进行能图分析。

\begin{figure}[htb]
\centering
\begin{asy}
	texpreamble("\usepackage{xeCJK}");
	texpreamble("\setCJKmainfont{SimSun}");
	usepackage("amsmath");
	import graph;
	import math;
	size(250);
	//Lagrange陀螺等效势能曲线
	picture tmp;
	pair O;
	real mgl,LZ,L3,I1,I3,x,thetam,theta1,theta2,theta0,bot,top;
	path clp;
	real U(real theta){
		return mgl*cos(theta)+(LZ-L3*cos(theta))*(LZ-L3*cos(theta))/(2*I1*sin(theta)*sin(theta)) + L3*L3/(2*I3);
	}
	real dU(real theta){
		return -mgl*sin(theta)+(L3*(sin(theta))**2*(LZ-L3*cos(theta))-sin(theta)*cos(theta)*(LZ-L3*cos(theta))**2)/(I1*(sin(theta))**3);
	}
	real Utheta1(real theta){
		return mgl*cos(theta)+(LZ-L3*cos(theta))*(LZ-L3*cos(theta))/(2*I1*sin(theta)*sin(theta)) + L3*L3/(2*I3)-U(theta1);
	}
	real bis(real f(real),real x1,real x2,real eps){
		int imax;
		real a,b,fab,fa,fb;
		a = x1;
		b = x2;
		imax = 1000;
		for(int i=1;i<=imax;i=i+1){
			fa = f(a);
			fb = f(b);
			fab = f((a+b)/2);
			if(fab == 0) return (a+b)/2;
			if(fab*fa<0){
				b = (a+b)/2;
			}
			else {
				a = (a+b)/2;
			}
			if(abs(b-a)<eps){
				return (a+b)/2;
			}
		}
		return (a+b)/2;
	}
	mgl = 10;
	LZ = -.6;
	L3 = 2;
	I1 = 2;
	I3 = 1;
	x = 20;
	bot = -10;
	top = 60;
	draw(tmp,xscale(x)*graph(U,0.01,pi-0.01),linewidth(1bp));
	clp = box((0,bot),(x*pi,80));
	clip(tmp,clp);
	draw(tmp,clp);
	clp = box((-1,bot),(x*pi,top));
	clip(tmp,clp);
	add(tmp);
	thetam = bis(dU,2,3,1e-10)-0.04;
	draw((0,U(thetam))--(x*pi,U(thetam)),red);
	draw((x*thetam,bot)--(x*thetam,U(thetam)),dashed);
	theta1 = 1.7;
	theta2 = bis(Utheta1,2,3,1e-10);
	theta0 = acos(LZ/L3);
	draw((0,U(theta1))--(x*pi,U(theta1)),blue);
	draw((x*theta1,bot)--(x*theta1,U(theta1)),dashed);
	draw((x*theta2,bot)--(x*theta2,U(theta2)),dashed);
	//draw((0,U(theta0))--(x*pi,U(theta0)),dashed+green);
	//draw((x*theta0,bot)--(x*theta0,U(theta0)),dashed+green);
	label("$0$",(x*0,bot),S);
	label("$\theta_1$",(x*theta1,bot),S);
	label("$\theta_m$",(x*thetam,bot),S);
	label("$\theta_2$",(x*theta2,bot),S);
	//label("$\theta_0$",(x*theta0,bot),S);
	label("$\pi$",(x*pi,bot),S);
	label("$\theta$",(x*pi,bot),E);
	label("$E$",(0,U(theta1)),W);
	label("$U$",(0,top),N);
\end{asy}
\caption{Lagrange陀螺的等效势能曲线}
\label{Lagrange陀螺等效势能曲线}
\end{figure}

分以下几种情况考虑:
\begin{enumerate}
	\item $\left|\dfrac{L_Z}{L_3}\right| \neq 1$,而且$E>U_{\min}$。根据图\ref{Lagrange陀螺等效势能曲线},当$E>U_{\min}$时,章动角的取值范围为$\theta_1 \leqslant \theta \leqslant \theta_2$,此时章动为往复周期运动。进动角速度为
	\begin{equation*}
		\dot{\phi} = \frac{L_Z-L_3 \cos \theta}{I_1\sin^2 \theta}
	\end{equation*}
	\begin{enumerate}
		\item $\left|\dfrac{L_Z}{L_3}\right| < 1$。令$\theta_0 = \arccos \dfrac{L_Z}{L_3}$,则有
		\begin{equation*}
			U(\theta) = Mgl\cos \theta + \frac{L_3^2(\cos \theta_0-\cos \theta)^2}{2I_1\sin^2 \theta} + \frac{L_3^2}{2I_3}
		\end{equation*}
		以及
		\begin{equation*}
			\dot{\phi}(\theta_0) = 0,\quad U(\theta_0) = Mgl\cos \theta_0 + \frac{L_3^2}{2I_3}
		\end{equation*}
		即,如果$\theta_0$在章动的范围中,进动角速度将在运动中不断改变符号。实际章动的范围由不等式$E \geqslant U(\theta)$决定。
		
\begin{figure}[htb]
\centering
\begin{minipage}[t]{0.3\textwidth}
\centering
\begin{asy}
	texpreamble("\usepackage{xeCJK}");
	texpreamble("\setCJKmainfont{SimSun}");
	usepackage("amsmath");
	import graph;
	import math;
	size(150);
	//Lagrange陀螺进动1
	pair i,j,k;
	real r,theta0,thetam,w,theta;
	r = 1;
	theta0 = pi/4;
	thetam = pi/10;
	w = 9;
	i = (-sqrt(2)/4,-sqrt(14)/12);
	j = (sqrt(14)/4,-sqrt(2)/12);
	k = (0,2*sqrt(2)/3);
	pair spconvert(real r,real theta,real phi){
		return r*sin(theta)*cos(phi)*i+r*sin(theta)*sin(phi)*j+r*cos(theta)*k;
	}
	pair thetaphi(real phi){
		return spconvert(r,theta0-thetam*sin(w*r*sin(theta0)*phi),phi);
	}
	pair phicir(real phi){
		return spconvert(r,theta,phi);
	}
	draw(Label("$Z$",EndPoint),r*k--1.4*r*k,Arrow);
	theta = theta0-thetam;
	draw(graph(phicir,-1.8,2.57),dashed);
	label("$\theta_1$",phicir(2.57),NE);
	theta = theta0+thetam;
	draw(graph(phicir,-1.4,2.1),dashed);
	label("$\theta_2$",phicir(2.1),NE);
	draw(graph(thetaphi,-1.4,2.4,200),linewidth(0.8bp)+red);
	draw(scale(r)*unitcircle,linewidth(0.8bp));
\end{asy}
\label{Lagrange陀螺进动1}
\caption{$\theta_0<\theta_1<\theta_2$}
\end{minipage}
\hspace{0.2cm}
\begin{minipage}[t]{0.3\textwidth}
\centering
\begin{asy}
	texpreamble("\usepackage{xeCJK}");
	texpreamble("\setCJKmainfont{SimSun}");
	usepackage("amsmath");
	import graph;
	import math;
	size(150);
	//Lagrange陀螺进动2
	pair i,j,k;
	real r,theta0,thetam,w,theta;
	r = 1;
	theta0 = pi/4;
	thetam = pi/10;
	w = 7.5;
	i = (-sqrt(2)/4,-sqrt(14)/12);
	j = (sqrt(14)/4,-sqrt(2)/12);
	k = (0,2*sqrt(2)/3);
	pair spconvert(real r,real theta,real phi){
		return r*sin(theta)*cos(phi)*i+r*sin(theta)*sin(phi)*j+r*cos(theta)*k;
	}
	pair thetaphi(real t){
		real x,y;
		x = t-sin(t);
		y = cos(t);
		return spconvert(r,theta0-thetam*y,x/r/sin(theta0)/w-0.2);
	}
	pair phicir(real phi){
		return spconvert(r,theta,phi);
	}
	draw(Label("$Z$",EndPoint),r*k--1.4*r*k,Arrow);
	theta = theta0-thetam;
	draw(graph(phicir,-1.8,2.57),dashed);
	label("$\theta_1$",phicir(2.57),NE);
	theta = theta0+thetam;
	draw(graph(phicir,-1.4,2.1),dashed);
	label("$\theta_2$",phicir(2.1),NE);
	draw(graph(thetaphi,-8,14,200),linewidth(0.8bp)+red);
	draw(scale(r)*unitcircle,linewidth(0.8bp));
\end{asy}
\label{Lagrange陀螺进动2}
\caption{$\theta_0=\theta_1<\theta_2$}
\end{minipage}
\hspace{0.2cm}
\begin{minipage}[t]{0.3\textwidth}
\centering
\begin{asy}
	texpreamble("\usepackage{xeCJK}");
	texpreamble("\setCJKmainfont{SimSun}");
	usepackage("amsmath");
	import graph;
	import math;
	size(150);
	//Lagrange陀螺进动3
	pair i,j,k;
	real r,theta0,thetam,w,theta,d;
	r = 1;
	theta0 = pi/4;
	thetam = pi/10;
	w = 10;
	d = 2.5;
	i = (-sqrt(2)/4,-sqrt(14)/12);
	j = (sqrt(14)/4,-sqrt(2)/12);
	k = (0,2*sqrt(2)/3);
	pair spconvert(real r,real theta,real phi){
		return r*sin(theta)*cos(phi)*i+r*sin(theta)*sin(phi)*j+r*cos(theta)*k;
	}
	pair thetaphi(real t){
		real x,y;
		x = t-d*sin(t);
		y = cos(t);
		return spconvert(r,theta0-thetam*y,x/r/sin(theta0)/w-0.1);
	}
	pair phicir(real phi){
		return spconvert(r,theta,phi);
	}
	draw(Label("$Z$",EndPoint),r*k--1.4*r*k,Arrow);
	theta = theta0-thetam;
	draw(graph(phicir,-1.8,2.57),dashed);
	label("$\theta_1$",phicir(2.57),NE);
	theta = theta0+thetam;
	draw(graph(phicir,-1.4,2.1),dashed);
	label("$\theta_2$",phicir(2.1),NE);
	draw(graph(thetaphi,-9.3,15.5,200),linewidth(0.8bp)+red);
	draw(scale(r)*unitcircle,linewidth(0.8bp));
\end{asy}
\label{Lagrange陀螺进动3}
\caption{$\theta_1<\theta_0<\theta_2$}
\end{minipage}
\end{figure}
		
		\begin{enumerate}
			\item $U(\theta) \leqslant E < U(\theta_0)$。此时
			\begin{equation*}
				Mgl\cos \theta + \frac{L_3^2(\cos \theta_0-\cos \theta)^2}{2I_1\sin^2 \theta} + \frac{L_3^2}{2I_3} < Mgl\cos \theta_0 + \frac{L_3^2}{2I_3}
			\end{equation*}
			即有$\theta_0 < \theta_1 < \theta_2$。此时进动角速度在运动中不变号,进动方向不变,如图\ref{Lagrange陀螺进动1}所示。
			\item $U(\theta) \leqslant E = U(\theta_0)$。此时
			\begin{equation*}
				Mgl\cos \theta + \frac{L_3^2(\cos \theta_0-\cos \theta)^2}{2I_1\sin^2 \theta} + \frac{L_3^2}{2I_3} \leqslant Mgl\cos \theta_0 + \frac{L_3^2}{2I_3}
			\end{equation*}
			即有$\theta_0 = \theta_1 < \theta_2$。此时在最小章动角位置,进动角速度为零,如图\ref{Lagrange陀螺进动2}所示。
			\item $E>U(\theta_0)$。此时有$\theta_1 < \theta_0 < \theta_2$,进动角速度在运动中周期性改变符号,进动方向周期性改变,如图\ref{Lagrange陀螺进动3}所示。
		\end{enumerate}
		
		\item $\left|\dfrac{L_Z}{L_3}\right| > 1$。进动角速度
		\begin{equation*}
			\dot{\phi} = \frac{L_Z - L_3 \cos \theta}{I_1 \sin^2 \theta}
		\end{equation*}
		在章动角$\theta$取任何值时都不可能等于零,此时不存在进动角速度为零的章动角,进动方向始终不变。
	\end{enumerate}
	\item $E>U_{\min}$。此时$\theta = \theta_m$不变,无章动。因此进动角速度和自转角速度
	\begin{equation*}
		\dot{\phi} = \frac{L_Z-L_3\cos \theta_m}{I_1\sin^2 \theta_m},\quad \dot{\psi} = \frac{L_3}{I_3} - \dot{\phi}\cos \theta_m
	\end{equation*}
	都是常数,因此此时Lagrange陀螺为规则进动。此时有$U'(\theta_m) = 0$,由此可得$\theta_m$满足的方程为
	\begin{equation*}
		\frac{(L_Z-L_3\cos \theta_m)(L_Z\cos \theta_m-L_3)}{I_1 \sin^3 \theta_m} + Mgl\sin \theta_m = 0
	\end{equation*}
	由此,根据式\eqref{Lagrange陀螺守恒量——能量}可得
	\begin{equation*}
		\sin \theta_m (I_1 \dot{\phi}^2 \cos \theta_m - L_3 \dot{\phi} + Mgl) = 0
	\end{equation*}
	上式有解的条件为
	\begin{equation}
		L_3^2 \geqslant 4I_1 Mgl\cos \theta_m
		\label{Lagrange陀螺规则进动条件}
	\end{equation}
	式\eqref{Lagrange陀螺规则进动条件}即为Lagrange陀螺的规则进动条件。
	\item $\left|\dfrac{L_Z}{L_3}\right| = 1$。
	\begin{enumerate}
		\item $L_Z = L_3 \neq 0$。此时等效势能为
		\begin{equation*}
			U(\theta) = \frac{L_3^2}{2I_3} + Mgl + \frac{L_3^2}{2I_1} \tan^2 \frac{\theta}{2} - 2Mgl\sin^2 \frac{\theta}{2}
		\end{equation*}
		当$\theta=0$时,有$E = \dfrac{L_3^2}{2I_3}+Mgl$,因此在$\theta=0$附近,等效势能可以展开为
		\begin{equation*}
			U(\theta) = \frac{L_3^2}{2I_3} + Mgl + \frac14 \left(\frac{L_3^2}{2I_1} - 2Mgl\right) \theta^2 + \frac{1}{24} \left(\frac{L_3^2}{2I_1}+Mgl\right) \theta^4 + o(\theta^4)
		\end{equation*}
		因此,在$\theta = 0$点处,当
		\begin{enumerate}
			\item $L_3^2 > 4I_1Mgl$时,Lagrange陀螺作稳定直立转动;
			\item $L_3^2 = 4I_1Mgl$时,Lagrange陀螺作次稳定直立转动;
			\item $L_3^2 < 4I_1Mgl$时,Lagrange陀螺作不稳定直立转动。
		\end{enumerate}
	
\begin{figure}[htb]
\centering
\begin{asy}
	texpreamble("\usepackage{xeCJK}");
	texpreamble("\setCJKmainfont{SimSun}");
	usepackage("amsmath");
	import graph;
	import math;
	size(250);
	//直立转动Lagrange陀螺的等效势能曲线
	picture tmp;
	pair O,P;
	real a,b,top,bot,x,left,right,height;
	path clp;
	real U(real theta){
		return a*(tan(theta/2))**2-b*(sin(theta/2))**2;
	}
	O = (0,0);
	top = 70;
	bot = -15;
	x = 25;
	draw(tmp,O--(pi,0));
	a = 4;
	b = 2;
	draw(tmp,graph(U,0,3),green+linewidth(0.8bp));
	a = 2;
	b = 2;
	draw(tmp,graph(U,0,3),orange+linewidth(0.8bp));
	a = 0.1;
	b = 10;
	draw(tmp,graph(U,0,3.141),red+linewidth(0.8bp));
	clp = box((0,bot),(pi,top));
	clip(tmp,clp);
	draw(tmp,clp);
	top = 60;
	clp = box((-1,bot),(pi,top));
	clip(tmp,clp);
	add(xscale(x)*tmp);
	label("$0$",(0,bot),S);
	label("$\pi$",(x*pi,bot),S);
	label("$\theta$",(x*pi,bot),E);
	label("$U$",(0,top),N);
	label("$E$",O,W);
	left = 2;
	right = 28;
	height = 20;
	top = top-2;
	fill(shift((1,-1))*box((left,top),(right,top-height)),black);
	unfill(box((left,top),(right,top-height)));
	draw(box((left,top),(right,top-height)));
	P = ((left+right)/2,top-height/6);
	label("$L_3^2 > 4I_1 Mgl$",P,green);
	P = ((left+right)/2,top-3*height/6);
	label("$L_3^2 = 4I_1 Mgl$",P,orange);
	P = ((left+right)/2,top-5*height/6);
	label("$L_3^2 < 4I_1 Mgl$",P,red);
	//draw(O--(x*pi*1.1,0),invisible);
\end{asy}
\caption{直立转动Lagrange陀螺的等效势能曲线}
\label{直立转动Lagrange陀螺的等效势能曲线}
\end{figure}

		\item $L_Z = -L_3 \neq 0$。此时等效势能为
		\begin{equation*}
			U(\theta) = \frac{L_3^2}{2I_3} - Mgl + \frac{L_3^2}{2I_1} \cot^2 \frac{\theta}{2} + 2Mgl\cos^2 \frac{\theta}{2}
		\end{equation*}
		当$\theta=\pi$时,有$E = \dfrac{L_3^2}{2I_3}-Mgl$,因此在$\theta = \pi$附近,等效势能可以展开为
		\begin{equation*}
			U(\theta) = \frac{L_3^2}{2I_3} - Mgl + \frac14\left(\frac{L_3^2}{2I_1}+2Mgl\right) (\theta-\pi)^2 + o((\theta-\pi)^2)
		\end{equation*}
		此时Lagrange陀螺作稳定倒悬转动。
	\end{enumerate}
\end{enumerate}

\begin{example}
一陀螺由半径为$2r$的薄圆盘及通过圆盘中心$C$,并和盘面垂直的场为$r$的杆轴所组成,杆的质量可忽略不计。将杆的另一端$O$放在水平面上,使其作无滑动的转动。如起始时杆$OC$与铅直线的夹角为$\alpha$,起始时的总角速度为$\omega$,方向沿着$\alpha$角的平分线,证明经过
\begin{equation*}
	t = \int_0^\alpha \frac{\mathrm{d} \theta}{\omega\sqrt{1 - \cos^2 \dfrac{\alpha}{2} \sec^2 \dfrac{\theta}{2} + \dfrac{g}{r\omega^2}(\cos \alpha-\cos \theta)}}
\end{equation*}
后杆将直立起来,式中$\theta$为任意瞬时杆轴与铅直线的夹角。

\begin{figure}[htb]
\centering
\begin{asy}
	texpreamble("\usepackage{xeCJK}");
	texpreamble("\setCJKmainfont{SimSun}");
	usepackage("amsmath");
	import graph;
	import math;
	size(250);
	//第六章例6图
	pair O,dash,P;
	real alpha,r,d;
	O = (0,0);
	r = 1;
	d = 0.05;
	alpha = 30;
	fill(rotate(alpha)*box((r-d,-2*r),(r+d,2*r)),0.6white);
	draw(Label("$z$",EndPoint),O--2.5*r*dir(alpha),Arrow);
	draw(Label("$Z$",EndPoint),O--2.5*r*dir(90),Arrow);
	draw(Label("$\boldsymbol{\omega}_0$",EndPoint,black),O--2.5*r*dir(alpha+(90-alpha)/2),red,Arrow);
	draw((-r/2,0)--(r/2,0));
	dash = 2*d*dir(-120);
	for(int i=1;i<=6;i=i+1){
		P = (r/2-r/6*(i-1),0);
		draw(P--P+dash);
	}
	label("$O$",O,NW);
	label("$C$",r*dir(alpha),2*E);
	label("$r$",1/2*r*dir(alpha),dir(alpha-90));
	label("$2r$",r*dir(alpha)+r*dir(alpha-90),dir(alpha-180));
	draw(Label("$\dfrac{\alpha}{2}$",Relative(0.3),Relative(E)),arc(O,0.4*r,alpha,alpha+(90-alpha)/2),Arrow);
	draw(Label("$\dfrac{\alpha}{2}$",Relative(0.5),Relative(E)),arc(O,0.4*r,alpha+(90-alpha)/2,90),Arrow);
	//draw(O--(2.5,0),invisible);
\end{asy}
\caption{例\theexample}
\label{第六章例6图}
\end{figure}
\end{example}

\begin{solution}
圆盘对$C$的惯量矩阵为
\begin{equation*}
	\mbf{I}' = \begin{pmatrix} mr^2 & 0 & 0 \\ 0 & mr^2 & 0 \\ 0 & 0 & 2mr^2 \end{pmatrix}
\end{equation*}
质心对$O$点的坐标为$\mbf{X}_C = \begin{pmatrix} 0 \\ 0 \\ r \end{pmatrix}$,因此有质心对$O$的转动惯量为
\begin{equation*}
	\mbf{I}_C = \begin{pmatrix} mr^2 & 0 & 0 \\ 0 & mr^2 & 0 \\ 0 & 0 & 0 \end{pmatrix}
\end{equation*}
根据平行轴定理可得圆盘对$O$的转动惯量为
\begin{equation*}
	\mbf{I} = \mbf{I}_C + \mbf{I}' = \begin{pmatrix} 2mr^2 & 0 & 0 \\ 0 & 2mr^2 & 0 \\ 0 & 0 & 2mr^2 \end{pmatrix}
\end{equation*}
故圆盘为Lagrange陀螺。其转动惯量和动能分别为
\begin{equation*}
	\mbf{L} = \mbf{I} \mbf{\omega} = 2mr^2 \mbf{\omega},\quad T = \frac12 \mbf{\omega}\cdot \mbf{L} = mr^2 \omega^2
\end{equation*}
由此可得三个积分常数为
\begin{equation*}
	\begin{cases}
		L_Z = L_{Z0} = 2mr^2 \omega_{Z0} = 2mr^2 \omega \cos \dfrac{\alpha}{2} \\
		L_3 = L_{30} = 2mr^2 \omega_{30} = 2mr^2 \omega \cos \dfrac{\alpha}{2} \\
		E = E_0 = mr^2 \omega^2 + mgr\cos \alpha
	\end{cases}
\end{equation*}
此时,章动等效势为
\begin{equation*}
	U(\theta) = mr^2 \omega^2 \cos^2 \frac{\alpha}{2} \sec^2 \frac{\theta}{2} + mgr\cos \theta = mgr\left(\frac{r\omega^2}{g}\cos^2 \frac{\alpha}{2} \sec^2 \frac{\theta}{2} + \cos \theta\right)
\end{equation*}
章动方程为
\begin{equation*}
	\frac12 I_1 \dot{\theta}^2 + U(\theta) = E
\end{equation*}
初始时刻有
\begin{equation*}
	\frac12 I_1 \dot{\theta}_0^2 = E - U(\alpha) = 0
\end{equation*}
即初始章动角速度为零,章动处于极限角位置。下面按$\dfrac{r\omega^2}{g}$可能的取值分类讨论,每种情况对应的等效势能曲线如图\ref{第六章例6势能曲线}所示。

\begin{figure}[htb]
\centering
\begin{asy}
	texpreamble("\usepackage{xeCJK}");
	texpreamble("\setCJKmainfont{SimSun}");
	usepackage("amsmath");
	import graph;
	import math;
	size(400);
	//直立转动Lagrange陀螺的等效势能曲线
	picture tmp;
	pair O,P;
	real alpha,beta,top,bot,x,left,right,height;
	path clp;
	real U(real theta){
		return 10*(beta*(cos(alpha/2))**2*(sec(theta/2))**2+cos(theta));
	}
	O = (0,0);
	alpha = 1.3;
	beta = 0;
	top = 70;
	bot = U(pi);
	x = 20;
	draw(tmp,O--(pi,0));
	draw(tmp,(alpha,bot)--(alpha,top),dashed);
	beta = 0;
	draw(tmp,graph(U,0,pi),black+linewidth(1bp));
	draw(tmp,Label("$E_1$",BeginPoint),(0,U(alpha))--(pi,U(alpha)),black);
	beta = (cos(alpha/2))**2;
	draw(tmp,graph(U,0,3),red+linewidth(1bp));
	draw(tmp,Label("$E_2$",BeginPoint),(0,U(alpha))--(pi,U(alpha)),red);
	beta = 2*(cos(alpha/2))**2;
	draw(tmp,graph(U,0,3),orange+linewidth(1bp));
	draw(tmp,Label("$E_3$",BeginPoint),(0,U(alpha))--(pi,U(alpha)),orange);
	beta = (2*(cos(alpha/2))**2+2)/2;
	draw(tmp,graph(U,0,3),green+linewidth(1bp));
	draw(tmp,Label("$E_4$",BeginPoint),(0,U(alpha))--(pi,U(alpha)),green);
	beta = 2;
	draw(tmp,graph(U,0,3),blue+linewidth(1bp));
	draw(tmp,Label("$E_5$",BeginPoint),(0,U(alpha))--(pi,U(alpha)),blue);
	beta = (2+2*(sec(alpha/2))**2)/2;
	draw(tmp,graph(U,0,3),darkblue+linewidth(1bp));
	draw(tmp,Label("$E_6$",BeginPoint),(0,U(alpha))--(pi,U(alpha)),darkblue);
	beta = 2*(sec(alpha/2))**2;
	draw(tmp,graph(U,0,3),purple+linewidth(1bp));
	draw(tmp,Label("$E_7$",BeginPoint),(0,U(alpha))--(pi,U(alpha)),purple);
	beta = 3*(sec(alpha/2))**2;
	draw(tmp,graph(U,0,3),brown+linewidth(1bp));
	draw(tmp,Label("$E_8$",BeginPoint),(0,U(alpha))--(pi,U(alpha)),brown);
	clp = box((-10,bot),(pi,top));
	clip(tmp,clp);
	clp = box((0,bot),(pi,top));
	draw(tmp,clp);
	top = 60;
	clp = box((-1,bot),(pi,top));
	clip(tmp,clp);
	add(xscale(x)*tmp);
	label("$0$",(0,bot),S);
	label("$\pi$",(x*pi,bot),S);
	label("$\alpha$",(x*alpha,bot),S);
	label("$\theta$",(x*pi,bot),E);
	label("$U$",(0,top),N);
	label("$0$",O,W);
	left = x*pi+4;
	right = x*pi+30;
	height = 66;
	top = top-2;
	fill(shift((1,-1))*box((left,top),(right,top-height)),black);
	unfill(box((left,top),(right,top-height)));
	draw(box((left,top),(right,top-height)));
	P = ((left+right)/2,top-height+height/16);
	label("$\dfrac{r\omega^2}{g} = 0$",P,black);
	P = ((left+right)/2,top-height+3*height/16);
	label("$0 < \dfrac{r\omega^2}{g} < 2\cos^2 \dfrac{\alpha}{2}$",P,red);
	P = ((left+right)/2,top-height+5*height/16);
	label("$\dfrac{r\omega^2}{g} = 2\cos^2 \dfrac{\alpha}{2}$",P,orange);
	P = ((left+right)/2,top-height+7*height/16);
	label("$2\cos^2 \dfrac{\alpha}{2} < \dfrac{r\omega^2}{g} < 2$",P,green);
	P = ((left+right)/2,top-height+9*height/16);
	label("$\dfrac{r\omega^2}{g} = 2$",P,blue);
	P = ((left+right)/2,top-height+11*height/16);
	label("$2 < \dfrac{r\omega^2}{g} < 2\sec^2 \dfrac{\alpha}{2}$",P,darkblue);
	P = ((left+right)/2,top-height+13*height/16);
	label("$\dfrac{r\omega^2}{g} = 2\sec^2 \dfrac{\alpha}{2}$",P,purple);
	P = ((left+right)/2,top-height+15*height/16);
	label("$\dfrac{r\omega^2}{g} > 2\sec^2 \dfrac{\alpha}{2}$",P,brown);
	//draw(O--(right*1.05,0),invisible);
\end{asy}
\caption{例\theexample 势能曲线}
\label{第六章例6势能曲线}
\end{figure}

\begin{enumerate}
	\item $\dfrac{r\omega^2}{g} = 0$。此时$U(\theta) = mgr\cos \theta$,刚体为复摆,平衡位置在$\theta = \pi$,不可能直立。
	\item $0 < \dfrac{r\omega^2}{g} < 2\cos^2 \dfrac{\alpha}{2}$。此时$U'(\alpha)<0$,刚体进行初始向下的往复章动,不可能直立。
	\item $\dfrac{r\omega^2}{g} = 2\cos^2 \dfrac{\alpha}{2}$。此时$U'(\alpha) = 0$,刚体进行规则进动,不可能直立。
	\item $2\cos^2 \dfrac{\alpha}{2} < \dfrac{r\omega^2}{g} < 2$。此时$U'(\alpha) > 0,\,U(0) > E$,刚体进行初始向上的往复章动,不可能直立。
	\item $\dfrac{r\omega^2}{g} = 2$。此时$U(0) = E$,恰能直立,刚体将以无限小章动角速度趋于直立位置,所需时间为无穷大。
	\item $2 < \dfrac{r\omega^2}{g} < 2\sec^2 \dfrac{\alpha}{2}$。此时$U(0)<E,\,U''(0)<0$,章动角速度先增后减,渐趋匀速,故刚体可在有限时间之内直立起来。
	\item $\dfrac{r\omega^2}{g} = 2\sec^2 \dfrac{\alpha}{2}$。此时$U''(0) = 0,\,U^{(4)}(0) > 0$,章动角速度递增,较快趋于匀速,故刚体可在有限时间之内直立起来。
	\item $\dfrac{r\omega^2}{g} > 2\sec^2 \dfrac{\alpha}{2}$。此时$U''(0) > 0$,章动角速度递增,较慢趋于匀速,故刚体可在有限时间之内直立起来。
\end{enumerate}
综上,刚体能够直立的条件为
\begin{equation*}
	\frac{r\omega^2}{g} \geqslant 2
\end{equation*}
由章动方程$\dfrac12 I_1 \dot{\theta}^2 + U(\theta) = E$可得
\begin{equation*}
	\frac{\mathrm{d} \theta}{\mathrm{d} t} = -\sqrt{\frac{2(E-U(\theta))}{I_1}}
\end{equation*}
此处,由于直立过程中,章动角随时间减小,故开方取负。因此,从初始章动角位置至直立所需时间为
\begin{equation*}
	t = -\sqrt{\frac{I_1}{2}} \int_\alpha^0 \frac{\mathrm{d} \theta}{\sqrt{E-U(\theta)}} = \int_0^\alpha \frac{\mathrm{d} \theta}{\omega\sqrt{1 - \cos^2 \dfrac{\alpha}{2} \sec^2 \dfrac{\theta}{2} + \dfrac{g}{r\omega^2}(\cos \alpha-\cos \theta)}}
\end{equation*}
\end{solution}

\section{Legendre椭圆积分与Jacobi椭圆函数}\label{椭圆积分与Jacobi椭圆函数}

\subsection{Legendre椭圆积分}

定义函数\footnote{本节提到的三类椭圆积分在不同的文献上可能会有不同的形式。}
\begin{equation}
	F(k,x) = \int_0^x \frac{\mathrm{d}t}{\sqrt{(1-t^2)(1-k^2t^2)}} = \int_0^{\arcsin x} \frac{\mathrm{d}\phi}{\sqrt{1-k^2\sin^2\phi}}
	\label{第一类椭圆积分-定义}
\end{equation}
为{\bf 第一类椭圆积分}。当$x=1$时,记
\begin{equation}
	K(k) = \int_0^1 \frac{\mathrm{d}t}{\sqrt{(1-t^2)(1-k^2t^2)}} = \int_0^{\frac{\pi}{2}} \frac{\mathrm{d}\phi}{\sqrt{1-k^2\sin^2\phi}}
	\label{chp6:第一类完全椭圆积分-定义}
\end{equation}
称为{\bf 第一类完全椭圆积分}。

第一类椭圆积分可以用级数表示为
\begin{align}
	F(k,x) & = \int_0^{\arcsin x} \left[1+\sum_{n=1}^\infty \frac{(2n-1)!!}{(2n)!!}k^{2n}\sin^{2n}\phi\right]\mathrm{d}\phi \nonumber \\
	& = \arcsin x + \sum_{n=1}^\infty \frac{(2n-1)!!}{(2n)!!}k^{2n}\int_0^{\arcsin x} \sin^{2n}\phi \mathrm{d}\phi
	\label{chp6:第一类椭圆积分-级数表示}
\end{align}
由此可得第一类完全椭圆积分的级数表示为
\begin{align}
	K(k) & = \arcsin 1 + \sum_{n=1}^\infty \frac{(2n-1)!!}{(2n)!!}k^{2n}\int_0^{\frac{\pi}{2}} \sin^{2n}\phi \mathrm{d}\phi \nonumber \\
	& = \frac{\pi}{2}\left[1+\sum_{n=1}^\infty \left(\frac{(2n-1)!!}{(2n)!!}\right)^2k^{2n}\right]
	\label{chp6:第一类完全椭圆积分-级数表示}
\end{align}

定义函数
\begin{equation}
	E(k,x) = \int_0^x \sqrt{\frac{1-k^2t^2}{1-t^2}}\mathrm{d}t = \int_0^{\arcsin x} \sqrt{1-k^2\sin^2\phi}\mathrm{d}\phi
	\label{第二类椭圆积分-定义}
\end{equation}
为{\bf 第二类椭圆积分}。当$x=1$时,记
\begin{equation}
	E(k) = \int_0^1 \sqrt{\frac{1-k^2t^2}{1-t^2}}\mathrm{d}t = \int_0^{\frac{\pi}{2}} \sqrt{1-k^2\sin^2\phi}\mathrm{d}\phi
	\label{第二类完全椭圆积分-定义}
\end{equation}
称为{\bf 第二类完全椭圆积分}。

第二类椭圆积分可以用级数表示为
\begin{align}
	E(k,x) & = \int_0^{\arcsin x} \left[1-\sum_{n=1}^\infty \frac{1}{2n-1}\frac{(2n-1)!!}{(2n)!!}k^{2n}\sin^{2n}\phi\right]\mathrm{d}\phi \nonumber \\
	& = \arcsin x - \sum_{n=1}^\infty \frac{1}{2n-1}\frac{(2n-1)!!}{(2n)!!}k^{2n}\int_0^{\arcsin x} \sin^{2n}\phi \mathrm{d}\phi
	\label{第二类椭圆积分-级数表示}
\end{align}
由此可得第二类完全椭圆积分的级数表示为
\begin{align}
	E(k) & = \arcsin 1 - \sum_{n=1}^\infty \frac{1}{2n-1}\frac{(2n-1)!!}{(2n)!!}k^{2n}\int_0^{\frac{\pi}{2}} \sin^{2n}\phi \mathrm{d}\phi \nonumber \\
	& = \frac{\pi}{2}\left[1-\sum_{n=1}^\infty \frac{1}{2n-1}\left(\frac{(2n-1)!!}{(2n)!!}\right)^2k^{2n}\right]
	\label{chp6:第二类完全椭圆积分-级数表示}
\end{align}

定义函数
\begin{align}
	\varPi(h,k,x) & = \int_0^x \frac{\mathrm{d}t}{(1+ht^2)\sqrt{(1-t^2)(1-k^2t^2)}} \nonumber \\
	& = \int_0^{\arcsin x} \frac{\mathrm{d}\phi}{(1+h\sin^2\phi)\sqrt{1-k^2\sin^2\phi}}
\end{align}
为{\bf 第三类椭圆积分}。当$x=1$时,记
\begin{equation}
	\varPi(h,k) = \int_0^1 \frac{\mathrm{d}t}{(1+ht^2)\sqrt{(1-t^2)(1-k^2t^2)}} = \int_0^{\frac{\pi}{2}} \frac{\mathrm{d}\phi}{(1+h\sin^2\phi)\sqrt{1-k^2\sin^2\phi}}
\end{equation}
称为{\bf 第三类完全椭圆积分}。

\subsection{Jacobi椭圆函数}

第一类椭圆积分\eqref{第一类椭圆积分-定义}
\begin{equation*}
	x = \int_0^s \frac{\mathrm{d}t}{\sqrt{(1-t^2)(1-k^2t^2)}}
\end{equation*}
的反函数记作
\begin{equation}
	s = \sn x
\end{equation}
称为{\bf 椭圆正弦函数}。

令
\begin{equation}
	\sin \phi = \sn x
\end{equation}
则称$\phi$为{\bf 椭圆振幅函数},记作$\phi = \am x$。

定义
\begin{equation}
	\cn x = \cos \phi = \sqrt{1-\sn^2 x}
\end{equation}
称为{\bf 椭圆余弦函数}。定义
\begin{equation}
	\tn x = \tan \phi = \frac{\sn x}{\cn x} = \frac{\sn x}{\sqrt{1-\cn^2 x}}
\end{equation}
称为{\bf 椭圆正切函数}。定义
\begin{equation}
	\dn x = \sqrt{1-k^2\sn^2x}
\end{equation}
称为{\bf 椭圆幅度函数}。

当$k\to 0$时,根据式\eqref{第一类椭圆积分-定义}可得
\begin{equation*}
	F(0,s) = \int_0^s \frac{\mathrm{d}t}{\sqrt{1-t^2}} = \arcsin s
\end{equation*}
由此,当$k\to 0$时,有
\begin{equation}
	\sn x \to \sin x
\end{equation}
根据其余椭圆函数的定义,当$k\to 0$时,有
\begin{equation}
	\cn x \to \cos x,\quad \dn x \to 1
\end{equation}


%第七章——非惯性系
\chapter{非惯性系}

由Newton第一定律定义的坐标系称为{\heiti 惯性系},在惯性系中Newton第二定律成立。

对实际问题,有时在非惯性系中描述运动比较方便。为使非惯性系的运动方程与Newton方程形式一致,必须引入虚拟的{\heiti 惯性力}。

\section{非惯性系的Newton第二定律}

\subsection{不同参考系的速度、加速度变换}

设$S$和$S'$是两个不同的参考系,设$S$是惯性系,而$S'$系相对与$S$系作一般运动。此处类似于刚体的一般运动,$S$系相当于刚体运动中的空间系,而$S'$系即为固连于刚体的本体系。根据Chasles定理\ref{Chasles定理},$S'$系相对于$S$系的运动既可以分解为$S'$系原点$O'$相对于$S$系的平动,与$S'$系相对于原点同样在$O'$但坐标轴都与$S$系平行的参考系(相当于刚体运动中的平动系)的定点转动,如图\ref{惯性参考系与非惯性参考系}所示。

\begin{figure}[htb]
\centering
\begin{asy}
	size(300);
	//惯性参考系与非惯性参考系
	pair O,OO,r,i,j,k;
	picture tmp;
	O = (0,0);
	i = 0.5*dir(-135);
	j = dir(0);
	k = dir(90);
	draw(Label("$x$",EndPoint),O--i,Arrow);
	draw(Label("$y$",EndPoint),O--j,Arrow);
	draw(Label("$z$",EndPoint),O--k,Arrow);
	label("$O$",O,SSE);
	label("$S$",0.5*i+0.7*k);
	OO = (0.7,0.5);
	label(tmp,"$O'$",O,2*dir(45-25));
	draw(tmp,Label("$x'$",EndPoint,S),rotate(-25)*(O--i),Arrow);
	draw(tmp,Label("$y'$",EndPoint),rotate(-25)*(O--j),Arrow);
	draw(tmp,Label("$z'$",EndPoint,W),rotate(-25)*(O--k),Arrow);
	label(tmp,"$S'$",rotate(-25)*(0.2*j+0.7*k));
	add(shift(OO)*scale(0.9)*tmp);
	r = (-0.1,0.6);
	dot(OO+r);
	label("$P$",OO+r,N);
	draw(Label("$\boldsymbol{r}_0$",Relative(0.5),Relative(E),black),O--OO,red,Arrow);
	draw(Label("$\boldsymbol{r}'$",Relative(0.5),Relative(W),black),OO--OO+r,red,Arrow);
	draw(Label("$\boldsymbol{r}$",Relative(0.5),Relative(W),black),O--OO+r,red,Arrow);
	//draw(O--(1.7,0),invisible);
\end{asy}
\caption{惯性参考系与非惯性参考系}
\label{惯性参考系与非惯性参考系}
\end{figure}

设$S'$系相对$S$系的平动为$\mbf{r}_0(t)$,转动角速度为$\mbf{\omega}_0(t)$,首先考虑任意矢量$\mbf{A}$在$S'$系下的分量表示有
\begin{equation*}
	\mbf{A} = \sum_{i=1}^3 A'_i \mbf{e}'_i = \mbf{e}' \tilde{\mbf{A}}
\end{equation*}
式中,$\mbf{e}' = \begin{pmatrix} \mbf{e}'_1 & \mbf{e}'_2 & \mbf{e}'_3 \end{pmatrix}$为$S'$系的基矢量矩阵,$\mbf{e}'_1,\mbf{e}'_2,\mbf{e}'_3$为$S'$系的基矢量,$\tilde{\mbf{A}} = \begin{pmatrix} A'_1 \\ A'_2 \\ A'_3 \end{pmatrix}$为$\mbf{A}$在$S'$系中的坐标。根据与第\ref{节:刚体运动的矢量-矩阵描述}节类似的方法,可有
\begin{equation*}
	\mbf{A} = \mbf{e}' \tilde{\mbf{A}} = \mbf{e} \mbf{U} \tilde{\mbf{A}}
\end{equation*}
式中,$\mbf{e} = \begin{pmatrix} \mbf{e}_1 & \mbf{e}_2 & \mbf{e}_3 \end{pmatrix}$为$S$系的基矢量矩阵,$\mbf{e}_1,\mbf{e}_2,\mbf{e}_3$为$S$系的基矢量,$\mbf{U}$为$S$系与$S'$系基矢量之间的旋转变换矩阵,即
\begin{equation*}
	\mbf{e}' = \mbf{e} \mbf{U}
\end{equation*}
因此可有
\begin{equation*}
	\frac{\mathrm{d} \mbf{A}}{\mathrm{d} t} = \mbf{e}' \frac{\mathrm{d} \tilde{\mbf{A}}}{\mathrm{d} t} + \mbf{e} \dot{\mbf{U}} \tilde{\mbf{A}} = \mbf{e}' \frac{\mathrm{d} \tilde{\mbf{A}}}{\mathrm{d} t} + \mbf{e}' \mbf{U}^{\mathrm{T}} \dot{\mbf{U}} \tilde{\mbf{A}}
\end{equation*}
记$\displaystyle \frac{\mathrm{d} \tilde{\mbf{A}}}{\mathrm{d} t} = \sum_{i=1}^3 \dot{A}'_i \mbf{e}'_i = \frac{\tilde{\mathrm{d}} \mbf{A}}{\mathrm{d} t}$,再根据第\ref{节:刚体上各点的速度与加速度}节中的相关结论,有$\mbf{e}' \mbf{U}^{\mathrm{T}} \dot{\mbf{U}} \tilde{\mbf{A}} = \mbf{\omega}_0 \times \mbf{A}$。因此可有
\begin{equation}
	\frac{\mathrm{d} \mbf{A}}{\mathrm{d} t} = \frac{\tilde{\mathrm{d}} \mbf{A}}{\mathrm{d} t} + \mbf{\omega}_0 \times \mbf{A}
	\label{相对变化率与绝对变化率的关系}
\end{equation}
式中$\displaystyle \frac{\tilde{\mathrm{d}} \mbf{A}}{\mathrm{d} t} = \sum_{i=1}^3 \dot{A}'_i \mbf{e}'_i$称为{\heiti 相对变化率},$\mbf{\omega}_0 \times \mbf{A}$称为{\heiti 转动牵连变化率},与式\eqref{Euler动力学方程——角动量定理的矢量形式}的结果相同。

对于$S$系和$S'$系,首先有
\begin{equation*}
	\mbf{r} = \mbf{r}_0 + \mbf{r}'
\end{equation*}
对上式求时间$t$的导数,并对矢量$\mbf{r}'$应用式\eqref{相对变化率与绝对变化率的关系}可有
\begin{equation}
	\mbf{v} = \mbf{v}_0 + \mbf{\omega}_0 \times \mbf{r}' + \mbf{v}'
	\label{非惯性系中的速度}
\end{equation}
式中$\mbf{v}' = \dfrac{\tilde{\mathrm{d}} \mbf{r}'}{\mathrm{d} t}$称为{\heiti 相对速度},$\mbf{v}_0 = \dfrac{\mathrm{d} \mbf{r}_0}{\mathrm{d} t}$称为{\heiti 平动牵连速度},$\mbf{\omega}_0 \times \mbf{r}'$称为{\heiti 转动牵连速度}。

再对式\eqref{非惯性系中的速度}求时间$t$的导数,可有
\begin{align*}
	\mbf{a} = \mbf{a}_0 + \frac{\tilde{\mathrm{d}}}{\mathrm{d} t} (\mbf{\omega}_0 \times \mbf{r}') + \mbf{\omega_0} \times (\mbf{\omega}_0 \times \mbf{r}') + \mbf{a}' + \mbf{\omega}_0 \times \mbf{v}'
\end{align*}
考虑到
\begin{equation*}
	\frac{\mathrm{d} \mbf{\omega}_0}{\mathrm{d} t} = \frac{\tilde{\mathrm{d}} \mbf{\omega}_0}{\mathrm{d} t} + \mbf{\omega}_0 \times \mbf{\omega}_0 = \frac{\tilde{\mathrm{d}} \mbf{\omega}_0}{\mathrm{d} t}
\end{equation*}
由此可有
\begin{equation}
	\mbf{a} = \mbf{a}_0 + \dot{\mbf{\omega}}_0 \times \mbf{r}' + \mbf{\omega}_0 \times (\mbf{\omega}_0 \times \mbf{r}') + 2\mbf{\omega}_0 \times \mbf{v}' + \mbf{a}'
\end{equation}
式中$\mbf{a}' = \dfrac{\tilde{\mathrm{d}} \mbf{v}'}{\mathrm{d} t}$称为{\heiti 相对加速度},$\mbf{a}_0 = \dfrac{\mathrm{d} \mbf{v}_0}{\mathrm{d} t}$称为{\heiti 平动牵连加速度},$\dot{\mbf{\omega}}_0 \times \mbf{r}'$称为{\heiti 转动牵连加速度},$\mbf{\omega}_0 \times (\mbf{\omega}_0 \times \mbf{r}')$称为{\heiti 向轴加速度},$2\mbf{\omega}_0 \times \mbf{v}'$称为{\heiti Coriolis加速度}。

\subsection{非惯性系的运动方程与惯性力}

$S$系为惯性系,在其中Newton第二定律成立,即有
\begin{equation*}
	\mbf{F} = m \mbf{a}
\end{equation*}
将此方程在非惯性系$S'$系中投影,可有
\begin{equation*}
	\mbf{F} = m \big(\mbf{a}' + \mbf{a}_0 + \dot{\mbf{\omega}}_0 \times \mbf{r}' + \mbf{\omega}_0 \times (\mbf{\omega}_0 \times \mbf{r}') + 2\mbf{\omega}_0 \times \mbf{v}'\big)
\end{equation*}
为了使得非惯性系中Newton第二定律与惯性系中具有相同的形式,将上式写作
\begin{equation}
	\mbf{F}' = m \mbf{a}'
\end{equation}
式中
\begin{equation}
	\mbf{F}' = \mbf{F} - m\mbf{a}_0 - m\dot{\mbf{\omega}}_0 \times \mbf{r}' - m\mbf{\omega}_0 \times (\mbf{\omega}_0 \times \mbf{r}') - 2m\mbf{\omega}_0 \times \mbf{v}'
	\label{非惯性系中的力}
\end{equation}
此处式\eqref{非惯性系中的力}中除$\mbf{F}$项外的其它项即称为{\heiti 惯性力},其中$-m\mbf{a}_0$称为{\heiti 平动牵连惯性力},$-m\dot{\mbf{\omega}}_0 \times \mbf{r}'$称为{\heiti 转动牵连惯性力},$-m\mbf{\omega}_0 \times (\mbf{\omega}_0 \times \mbf{r}')$称为{\heiti 惯性离心力},$-2m\mbf{\omega}_0 \times \mbf{v}'$称为{\heiti Coriolis力}。

\section{非惯性系的Lagrange动力学}

\subsection{非惯性系质点运动的Lagrange函数}

在非惯性系中,将广义坐标取为非惯性系中的坐标$x',y',z'$,则有
\begin{equation*}
	\mbf{r}(x',y',z',t) = \mbf{r}_0(t) + \sum_{i=1}^3 x'_i \mbf{e}'_i(t)
\end{equation*}
因此可得
\begin{equation*}
	\dot{\mbf{r}} = \sum_{i=1}^3 \frac{\pl \mbf{r}}{\pl x'_i} \dot{x}'_i + \frac{\pl \mbf{r}}{\pl t} = \mbf{v}' + \mbf{v}_0(t) + \mbf{\omega}_0(t) \times \mbf{r}' = \mbf{v}_0(t) + \frac{\mathrm{d} \mbf{r}'}{\mathrm{d} t}
\end{equation*}
即有Lagrange函数为
\begin{align*}
	L & = T-U = \frac12 m\left[\left(\frac{\mathrm{d} \mbf{r}'}{\mathrm{d} t}\right)^2 + 2\mbf{v}_0 \cdot \frac{\mathrm{d} \mbf{r}'}{\mathrm{d} t} + \mbf{v}_0^2\right] - U(\mbf{r}',\mbf{v}',t) \\
	& = \frac12 m\left(\frac{\mathrm{d} \mbf{r}'}{\mathrm{d} t}\right)^2 + \frac12 m \mbf{v}_0^2 + \frac{\mathrm{d}}{\mathrm{d} t} (m\mbf{v}_0 \cdot \mbf{r}') -m \mbf{a}_0 \cdot \mbf{r}' - U(\mbf{r}',\mbf{v}',t)
\end{align*}
项$\dfrac12 m \mbf{v}_0^2$不含有广义坐标和广义速度,因此对Lagrange方程无贡献,可以略去。项$\dfrac{\mathrm{d}}{\mathrm{d} t} (m\mbf{v}_0 \cdot \mbf{r}')$是广义坐目标函数对时间的全导数,同样对Lagrange方程无贡献,也可以略去。由此,Lagrange函数简化为
\begin{equation*}
	L' = \frac12 m\left(\frac{\mathrm{d} \mbf{r}'}{\mathrm{d} t}\right)^2 -m \mbf{a}_0 \cdot \mbf{r}' - U(\mbf{r}',\mbf{v}',t)
\end{equation*}
由于$\dfrac{\mathrm{d} \mbf{r}'}{\mathrm{d} t} = \mbf{v}' + \mbf{\omega}_0(t) \times \mbf{r}'$,可有
\begin{align}
	L' & = \frac12 m \big( \mbf{v}'^2 + 2\mbf{v}' \cdot (\mbf{\omega}_0 \times \mbf{r}') + (\mbf{\omega}_0 \times \mbf{r}')^2 \big) - m\mbf{a}_0 \cdot \mbf{r}' - U(\mbf{r}',\mbf{v}',t) \nonumber \\
	& = \frac12 m\mbf{v}'^2 - \left(U + m\mbf{a}_0 \cdot \mbf{r}' - \frac12 m(\mbf{\omega}_0 \times \mbf{r}')^2 - m(\mbf{\omega}_0 \times \mbf{r}') \cdot \mbf{v}'\right)
	\label{非惯性系的Lagrange函数}
\end{align}
定义{\heiti 非惯性系动能}
\begin{equation}
	T' = \frac12 m \mbf{v}'^2
\end{equation}
和{\heiti 非惯性系等效势能}
\begin{equation}
	U' = U + m\mbf{a}_0 \cdot \mbf{r}' - \frac12 m(\mbf{\omega}_0 \times \mbf{r}')^2 - m(\mbf{\omega}_0 \times \mbf{r}') \cdot \mbf{v}'
\end{equation}
则在非惯性系中,Lagrange函数与惯性系具有相同的形式
\begin{equation*}
	L' = T' - U'
\end{equation*}
非惯性系等效势能中,不仅包括真实主动力的势,还包括虚拟惯性力的势。其中
\begin{equation*}
	U_1 = m\mbf{a}_0(t) \cdot \mbf{r}',\quad U_2 = -\frac12 m\big[\mbf{\omega}_0(t) \times \mbf{r}'\big]^2
\end{equation*}
是速度无关势,其对应的力分别为
\begin{align*}
	\mbf{F}'_1 & = -\frac{\pl U_1}{\pl \mbf{r}'} = -m\mbf{a}_0 \\
	\mbf{F}'_2 & = -\frac{\pl U_2}{\pl \mbf{r}'} = -m\mbf{\omega}_0 \times (\mbf{\omega}_0 \times \mbf{r}')
\end{align*}
即$U_1$为{\heiti 平动牵连惯性势},$U_2$为{\heiti 惯性离心势}。而
\begin{equation*}
	U_3 = -m(\mbf{\omega}_0 \times \mbf{r}') \cdot \mbf{v}' = -m(\mbf{v}' \times \mbf{\omega}_0) \cdot \mbf{r}'
\end{equation*}
是速度相关势,其对应的力为
\begin{equation*}
	\mbf{F}'_3 = \frac{\tilde{\mathrm{d}}}{\mathrm{d} t} \frac{\pl U_3}{\pl \mbf{v}'} - \frac{\pl U_3}{\pl \mbf{r}'} = -m\dot{\mbf{\omega}}_0 \times \mbf{r}' - 2m\mbf{\omega}_0 \times \mbf{v}'
\end{equation*}
即$U_3$为{\heiti 转动牵连-Coriolis惯性势}。

\subsection{非惯性系质点运动的Lagrange方程}

在非惯性系中,Lagrange函数与惯性系中的Lagrange函数具有相同的形式
\begin{equation*}
	L' = T'-U'
\end{equation*}
其中
\begin{equation*}
	T' = \frac12 m\mbf{v}'^2,\quad U' = U+m\mbf{a}_0 \cdot \mbf{r}' - \frac12 m (\mbf{\omega}_0 \times \mbf{r}')^2 - m(\mbf{\omega}_0 \times \mbf{r}') \cdot \mbf{v}'
\end{equation*}
因此,在非惯性系中,Lagrange方程也与惯性系中的Lagrange方程具有相同的形式,即
\begin{equation}
	\frac{\tilde{\mathrm{d}}}{\mathrm{d} t} \frac{\pl L'}{\pl \dot{x}'_i} - \frac{\pl L'}{\pl x'_i} = 0,\quad i=1,2,3
\end{equation}
或者简单地记作
\begin{equation}
	\frac{\tilde{\mathrm{d}}}{\mathrm{d} t} \frac{\pl L'}{\pl \mbf{v}'} - \frac{\pl L'}{\pl \mbf{r}'} = \mbf{0}
\end{equation}

\begin{example}
质量为$m$的小环$M$,套在半径为$R$的光滑圆圈上,并可沿着圆圈滑动。如圆圈在水平面内以角速度$\mbf{\omega}$绕圆圈上某点$O$转动,求小环沿圆周的运动微分方程。
\end{example}
\begin{solution}
\begin{figure}[htb]
\centering
\begin{asy}
	size(300);
	//第七章例1图
	pair O,O1,M;
	real theta,phi,r,x,y,rr;
	O = (0,0);
	theta = 50;
	phi = 35;
	r = 1;
	x = 2.1;
	y = 2.1;
	O1 = r*dir(theta);
	M = O1+r*dir(theta+phi);
	draw(Label("$x$",EndPoint),O--x*dir(0),Arrow);
	draw(Label("$y$",EndPoint),O--y*dir(90),Arrow);
	label("$O$",O,SW);
	draw(shift(O1)*scale(r)*unitcircle);
	draw(Label("$x'$",EndPoint),O--1.6*O1,Arrow);
	draw(Label("$y'$",EndPoint),O1--O1+0.6*r*dir(90+theta),Arrow);
	label("$O'$",O1,SE);
	dot(M);
	label("$M$",M,N);
	draw(O--M--O1);
	draw(Label("$\boldsymbol{e}_r$",EndPoint,Relative(E)),O1--O1+0.5*r*dir(theta+phi),Arrow);
	draw(Label("$\boldsymbol{e}_\theta$",EndPoint,Relative(W)),O1--O1+0.5*r*dir(90+theta+phi),Arrow);
	rr = 0.2;
	draw(Label("$\omega t$",MidPoint,Relative(E)),arc(O,rr,0,theta),Arrow);
	draw(Label("$\theta$",MidPoint,Relative(E)),arc(O1,rr,theta,theta+phi),Arrow);
	//draw(O--1.1*x*dir(0),invisible);
\end{asy}
\caption{例\theexample}
\label{第七章例1图}
\end{figure}

取两个坐标系$Oxy$和$O'x'y'$(如图\ref{第七章例1图}所示),则旋转坐标系$O'x'y'$相对于固定坐标系$Oxy$的角速度为$\mbf{\omega}$。取$\theta$作为小环$M$的广义坐标,因此在非惯性系$O'x'y'$中,体系的Lagrange方程为
\begin{equation*}
	\frac{\tilde{\mathrm{d}}}{\mathrm{d} t} \frac{\pl L'}{\pl \dot{\theta}} - \frac{\pl L'}{\pl \theta} = 0
\end{equation*}
体系的Lagrange函数即为式\eqref{非惯性系的Lagrange函数}
\begin{equation*}
	L' = \frac12 m\mbf{v}'^2 - \left(U + m\mbf{a}_0 \cdot \mbf{r}' - \frac12 m(\mbf{\omega}_0 \times \mbf{r}')^2 - m(\mbf{\omega}_0 \times \mbf{r}') \cdot \mbf{v}'\right)
\end{equation*}
其中,
\begin{align*}
	& \mbf{\omega}_0 = \mbf{\omega} = \omega \mbf{e}_3,\quad \mbf{r}' = R\mbf{e}_r \\
	& \mbf{v}' = \frac{\tilde{\mathrm{d}} \mbf{r}'}{\mathrm{d} t} = R\dot{\theta} \mbf{e}_\theta,\quad \mbf{r}_0 = R\mbf{e}_1 \\ 
	& \mbf{v}_0 = \frac{\mathrm{d} \mbf{r}_0}{\mathrm{d} t} = \frac{\mathrm{d} (R\mbf{e}_1)}{\mathrm{d} t} = \frac{\tilde{\mathrm{d}} (R\mbf{e}_1)}{\mathrm{d} t} + \mbf{\omega}_0 \times R\mbf{e}_1 = R\omega \mbf{e}_2 \\
	& \mbf{a}_0 = \frac{\mathrm{d} \mbf{v}_0}{\mathrm{d} t} = \frac{\tilde{\mathrm{d}} \mbf{v}_0}{\mathrm{d} t} + \mbf{\omega}_0 \times \mbf{v}_0 = -R\omega^2 \mbf{e}_1
\end{align*}
由此可有
\begin{align*}
	L' & = \frac12 m\mbf{v}'^2 - \left(U + m\mbf{a}_0 \cdot \mbf{r}' - \frac12 m(\mbf{\omega}_0 \times \mbf{r}')^2 - m(\mbf{\omega}_0 \times \mbf{r}') \cdot \mbf{v}'\right) \\
	& = \frac12 mR^2 \left(\dot{\theta}^2 + 2\omega \dot{\theta} + \omega^2 + 2\omega^2 \cos \theta\right)
\end{align*}
然后由非惯性系的Lagrange方程,即有小环$M$的运动微分方程为
\begin{equation*}
	\ddot{\theta} + \omega^2 \sin \theta = 0
\end{equation*}

如果不利用非惯性系,此题目也同样可解。小环$M$的坐标为
\begin{equation*}
	\begin{cases}
		x_M = R\cos \omega t + R\cos (\omega t+\theta) \\
		y_M = R\sin \omega t + R\sin (\omega t+\theta)
	\end{cases}
\end{equation*}
因此其速度为
\begin{equation*}
	\begin{cases}
		\dot{x}_M = -R\omega \sin \omega t - R(\omega + \dot{\theta}) \sin (\omega t+\theta) \\
		\dot{y}_M = R \omega \cos \omega t + R(\omega + \dot{\theta}) \cos (\omega t+\theta)
	\end{cases}
\end{equation*}
因此体系的Lagrange函数为
\begin{align*}
	L & = \frac12 (\dot{x}_M^2 + \dot{y}_M^2) = \frac12 mR^2 \left[\dot{\theta}^2 + 2\omega \dot{\theta} + 2\omega^2 + 2\omega(\omega+ \dot{\theta}) \cos \theta\right]
\end{align*}
根据惯性系中理想完整系的Lagrange方程\eqref{理想完整系的Lagrange方程}可得相同的运动微分方程
\begin{equation*}
	\ddot{\theta} + \omega^2 \sin \theta = 0
\end{equation*}
\end{solution}

\begin{example}
直管$AB$长为$l$,以匀角速度$\omega_0$绕固定点$O$在水平面内转动。固定点$O$与直管两端的连线互相垂直。管的内壁是光滑的。管内有一质点,开始时它在$A$处,相对管的速度为$v'$,方向指向$B$端。试求解质点的运动,并求它对管的压力和离管所需的时间。

\begin{figure}[htb]
\centering
\begin{asy}
	size(300);
	//第七章例2图
	picture tmp;
	real R1,R2,d,alpha,r;
	pair O,A,B;
	R1 = 1;
	R2 = 0.6;
	O = (0,0);
	A = (-R1,0);
	B = (0,R2);
	d = 0.02;
	alpha = degrees(B-A);
	draw(interp(A,O,-.1)--O--interp(O,B,1.1),dashed);
	draw(A+d*dir(-90+alpha)--B+d*dir(-90+alpha),linewidth(0.8bp));
	draw(A-d*dir(-90+alpha)--B-d*dir(-90+alpha),linewidth(0.8bp));
	label("$R_2$",A/2,S);
	label("$R_1$",B/2,E);
	r = 0.15;
	draw(Label("$\alpha$",MidPoint,Relative(E)),arc(A,r,0,alpha));
	label("$B$",A,S);
	label("$A$",B,E);
	real x,y;
	x = 0.7;
	y = 0.8;
	draw(Label("$x$",EndPoint),O--x*dir(alpha),Arrow);
	draw(Label("$y$",EndPoint),O--y*dir(alpha+90),Arrow);
	label("$O$",O,S);
	pair M;
	M = interp(A,B,0.56);
	fill(circle(M,d));
	draw(Label("$\boldsymbol{r}$",MidPoint,Relative(W)),O--M-d*dir(M),Arrow);
	r = 0.1;
	draw(Label("$\boldsymbol{\omega}_0$",Relative(0.75),Relative(E)),arc(O,r,0,70),Arrow);
	label("$h$",R1*Cos(alpha)/2*dir(alpha+90),dir(alpha));
\end{asy}
\caption{例\theexample}
\label{第七章例2图}
\end{figure}
\end{example}
\begin{solution}
取一绕$O$点以相同角速度$\omega_0$转动的参考系,如图\ref{第七章例2图}所示,此参考系为非惯性系。在此坐标系下,$t$时刻质点的位矢为
\begin{align*}
	\mbf{r}' = x\mbf{i}+h\mbf{j}
\end{align*}
则质点在非惯性系中的速度和加速度为
\begin{equation*}
	\mbf{v}' = \dot{x}\mbf{i},\quad \mbf{a}' = \ddot{x}\mbf{i}
\end{equation*}
在非惯性系中,质点受到支持力$\mbf{N} = N_y\mbf{j}+N_z\mbf{k}$,重力$m\mbf{g}=-mg\mbf{k}$以及惯性力
\begin{align*}
	\mbf{F}'_1 & = -m\mbf{a}_0 = \mbf{0} \\
	\mbf{F}'_2 & = -m\frac{\mathd\mbf{\omega}}{\mathd t} \times \mbf{r}' = \mbf{0} \\
	\mbf{F}'_3 & = -m\mbf{\omega} \times (\mbf{\omega} \times \mbf{r}') = m\omega_0^2\mbf{r}' = m\omega_0^2(\dot{x}\mbf{i}+h\mbf{j}) \\
	\mbf{F}'_c & = -2m\mbf{\omega} \times \mbf{v}' = -2m\omega_0\dot{x}\mbf{j}
\end{align*}
根据非惯性系中的Newton第二定律$m\mbf{a}'=\mbf{N}+m\mbf{g}+\mbf{F}'_1+\mbf{F}'_2+\mbf{F}'_3+\mbf{F}'_c$可得
\begin{equation*}
\begin{cases}
	m\ddot{x} = m\omega_0x \\
	0 = N_y+m\omega_0^2h-2m\omega_0\dot{x} \\
	0 = N_z-mg
\end{cases}
\end{equation*}
由第一式可解得
\begin{equation*}
	x = A\mathe^{\omega_0t}+B\mathe^{-\omega_0t}
\end{equation*}
由初始条件,$t=0$时,$x=R_1\sin\alpha=\dfrac{R_1^2}{l}, \dot{x}=-v'$,可得
\begin{equation*}
	A = \frac12 \left(\frac{R_1^2}{l} - \frac{v'}{\omega_0}\right),\quad B = \frac12 \left(\frac{R_1^2}{l} + \frac{v'}{\omega_0}\right)
\end{equation*}
由此可得
\begin{equation*}
	x = \frac12 \left(\frac{R_1^2}{l} - \frac{v'}{\omega_0}\right)\mathe^{\omega_0t}+\frac12 \left(\frac{R_1^2}{l} + \frac{v'}{\omega_0}\right)
\end{equation*}
以及
\begin{equation*}
	N_y = m\omega_0^2\left[\left(\frac{R_1^2}{l} - \frac{v'}{\omega_0}\right)\mathe^{\omega_0t}-\left(\frac{R_1^2}{l} + \frac{v'}{\omega_0}\right)\mathe^{-\omega_0t}-h\right],\quad N_z=mg
\end{equation*}

当质点由$B$端离开直管时,可有
\begin{equation*}
	\frac12 \left(\frac{R_1^2}{l} - \frac{v'}{\omega_0}\right)\mathe^{\omega_0t}+\frac12 \left(\frac{R_1^2}{l} + \frac{v'}{\omega_0}\right) = -\frac{R_2^2}{l}
\end{equation*}
可以解得
\begin{equation*}
	\mathe^{\omega_0t} = \frac{-\omega_0R_2^2\pm \sqrt{\omega_0^2R_2^4-\omega_0^2R_1^4+l^2v'^2}}{\omega_0R_1^2-lv'}
\end{equation*}
上式右端为正需要满足$\ds\frac{R_1^2}{l} - \frac{v'}{\omega_0} > 0$即$v'>\dfrac{\omega_0R_1^2}{l}$,因此离管所需的时间为
\begin{equation*}
	t = \frac{1}{\omega_0} \ln \frac{\omega_0R_2^2+\sqrt{\omega_0^2R_2^4-\omega_0^2R_1^4+l^2v'^2}}{lv'-\omega_0R_1^2}
\end{equation*}

当质点由$A$端离开直管时,可有
\begin{equation*}
	\frac12 \left(\frac{R_1^2}{l} - \frac{v'}{\omega_0}\right)\mathe^{\omega_0t}+\frac12 \left(\frac{R_1^2}{l} + \frac{v'}{\omega_0}\right) = \frac{R_1^2}{l}
\end{equation*}
可以解得
\begin{equation*}
	\mathe^{-\omega_0t} = \frac{\omega_0R_1^2\pm lv'}{\omega_0R_1^2-lv'}
\end{equation*}
上式右端为正需要满足$\omega_0R_1^2-lv'>0$即$v'<\dfrac{\omega_0R_1^2}{l}$,因此离管所需的时间为
\begin{equation*}
	t = \frac{1}{\omega_0}\ln \frac{\omega_0R_1^2+lv'}{\omega_0R_1^2-lv'}
\end{equation*}
\end{solution}

\section{地球自转的动力学效应}

地球不是一个严格的惯性系,地球自转的角速度为
\begin{equation*}
	\omega_0 = \frac{2\pi}{24 \times 3600}\si{\per\second} \approx \SI{7.29e-5}{\per\second}
\end{equation*}
地球的平均半径为$R = \SI{6.37e6}{\meter}$,因此由于地球自转所引起的向心加速度
\begin{equation*}
	a_t \leqslant \omega^2 R = \SI[per-mode=symbol]{0.0338}{\meter\per\second\squared} = \num{3e-3} g
\end{equation*}
式中$g$为重力加速度。因此在$g$起作用的问题中,如果计算精度达到$\num{e-3}$量级时,必须考虑重力加速度的这一修正。由于地球自转所引起的Coriolis加速度为
\begin{equation*}
	a_C \leqslant 2\omega_0 v
\end{equation*}
由此引起的线偏离和角偏离分别为
\begin{equation*}
	\Delta s \leqslant \frac12 a_C t^2 = \omega_0 vt,\quad \Delta \theta = \frac{\Delta s}{vt} \leqslant \omega_0 t
\end{equation*}
对于短时间的运动,由于$\omega \approx \SI{10e-5}{\per\second}$,这一效应可以忽略,但是对于长时间的运动,这一效应就不能忽略了。

地球公转的角速度是地球自转角速度的$\dfrac{1}{365}$,地球与太阳之间的距离是地球半径的$\num{2.5e4}$倍,因此地球公转的向心力与地球自转的向心力之比为
\begin{equation*}
	\left(\frac{1}{365}\right)^2 \times \num{2.5e4} \approx 0.2
\end{equation*}
Coriolis力大小之比为$\dfrac{1}{365} \approx 0.03$。因此,地球公转引起的非惯性效应比地球自转引起的非惯性效应要小一至两个数量级,一般可忽略。

\subsection{重力加速度$g$随纬度$\lambda$的变化}

在纬度为$\lambda$处,地面上质量为$m$的物体,在地面参考系中受到地球的引力$\mbf{F}$和惯性离心力$\mbf{F}_t$,实际观察到的重力$m\mbf{g}$是$\mbf{F}$和$\mbf{F}_t$的合力
\begin{equation*}
	m\mbf{g} = \mbf{F}+\mbf{F}_t
\end{equation*}
如果认为地球是均匀的刚性球体,则$\mbf{F}$的值各地相同,但$\mbf{F}_t$的值随纬度$\lambda$而改变,即
\begin{equation}
	F_t = mR\omega_0^2 \cos \lambda
	\label{地球上惯性离心力的大小}
\end{equation}
式中$R$为地球半径。因此,$m\mbf{g}$的大小和方向都随纬度$\lambda$而变化,大小的变化反映在重力加速度$g$随$\lambda$的变化,方向的变化反映在$m\mbf{g}$的方向(即铅直方向)和引力$\mbf{F}$方向(即地球半径的方向)之间的夹角$\alpha$随$\lambda$的变化。

\begin{figure}[htb]
\centering
\begin{minipage}[t]{0.45\textwidth}
\centering
\begin{asy}
	size(200);
	//重力引力离心力1
	pair O;
	real r,rr,lambda,F,Ft;
	O = (0,0);
	r = 1;
	lambda = 50;
	F = 0.7;
	Ft = 0.2;
	rr = 0.15;
	draw(shift(O)*scale(r)*unitcircle);
	draw(O--r*dir(lambda));
	draw(Label("$\boldsymbol{F}$",EndPoint,Relative(E)),r*dir(lambda)--(r-F)*dir(lambda),Arrow);
	draw(Label("$\boldsymbol{F}_t$",EndPoint),r*dir(lambda)--r*dir(lambda)+Ft*dir(0),Arrow);
	draw(Label("$m\boldsymbol{g}$",EndPoint,Relative(W)),r*dir(lambda)--(r-F)*dir(lambda)+Ft*dir(0),Arrow);
	draw((r-F)*dir(lambda)--(r-F)*dir(lambda)+Ft*dir(0)--r*dir(lambda)+Ft*dir(0),dashed);
	draw(Label("$\omega_0$",EndPoint),(0,-r)--(0,1.15*r));
	draw(O--(r,0));
	draw(Label("$\lambda$",MidPoint,Relative(E)),arc(O,rr,0,lambda),Arrow);
	draw(Label("$\alpha$",Relative(0.6),Relative(E)),arc(r*dir(lambda),1.8*rr,lambda+180,degrees(-F*dir(lambda)+Ft*dir(0))),Arrow);
	draw(shift((0,1.15*r))*yscale(1/2)*arc(O,rr,180+30,360-30),Arrow);
\end{asy}
\caption{惯性离心力对重力的影响}
\label{重力引力离心力1}
\end{minipage}
\hspace{0.5cm}
\begin{minipage}[t]{0.45\textwidth}
\centering
\begin{asy}
	size(200);
	//重力引力离心力2
	pair O;
	real rr,lambda,F,Ft;
	O = (0,0);
	lambda = 50;
	F = 0.7;
	Ft = 0.2;
	draw(Label("$\boldsymbol{F}$",Relative(0.6),Relative(E)),F*dir(lambda)--O,Arrow);
	draw(Label("$\boldsymbol{F}_t$",MidPoint,Relative(E)),O--Ft*dir(0),Arrow);
	draw(Label("$m\boldsymbol{g}$",MidPoint,Relative(W)),F*dir(lambda)--Ft*dir(0),Arrow);
	draw(O--(0,F*dir(lambda).y)--F*dir(lambda),dashed);
	draw(Ft*dir(0)--Ft*dir(0)+(0,F*dir(lambda).y),dashed);
	draw(Ft*dir(0)--Ft*dir(0)+Ft*Sin(lambda)*dir(90+lambda),dashed);
	rr = 0.05;
	draw(Label("$\lambda$",MidPoint,Relative(E)),arc(O,rr,0,lambda),Arrow);
	rr = 0.1;
	draw(Label("$\alpha$",MidPoint,Relative(E)),arc(F*dir(lambda),rr,180+lambda,degrees(Ft*dir(0)-F*dir(lambda))),Arrow);
	draw(O--F*dir(lambda)+(0,0.005),invisible);
\end{asy}
\caption{$m\mbf{g}$、$\mbf{F}$和$\mbf{F}_t$之间的关系}
\label{重力引力离心力2}
\end{minipage}
\end{figure}

根据图\ref{重力引力离心力2}可有
\begin{subnumcases}{}
	F_t \sin \lambda = mg\sin \alpha \label{重力与纬度关系1} \\
	F \sin \lambda = mg\sin (\lambda+ \alpha) \label{重力与纬度关系2} \\
	F = mg\cos \alpha + F_t \cos \alpha \label{重力与纬度关系3} 
\end{subnumcases}
将式\eqref{地球上惯性离心力的大小}代入式\eqref{重力与纬度关系1}中可得
\begin{equation}
	\sin \alpha = \frac{R\omega_0^2 \sin 2\lambda}{2g}
	\label{偏角与纬度和重力加速度的关系1}
\end{equation}
由此可以看出,当认为$g$的值不变时,$\alpha$的值在$\ang{45}$处最大,如果将$\lambda= \ang{45}$处实测的$g = \SI[per-mode=symbol]{9.8062}{\meter\per\second\squared}$的值和$\omega_0$及$R$代入,可得
\begin{equation*}
	\alpha_{\max} \approx \frac{\omega_0^2 R}{2g} = \ang{;6;}
\end{equation*}
所以偏角$\alpha$是很小的,当$\omega_0$作为一阶小量时,$\alpha$是二阶小量。在赤道处,$\alpha=\lambda=0$,由式\eqref{偏角与纬度和重力加速度的关系1}和式\eqref{重力与纬度关系3}可得
\begin{equation*}
	F = m(g_0 + R\omega_0^2)
\end{equation*}
式中$g_0$是赤道地区的重力加速度,再将上式代入式\eqref{重力与纬度关系2}中,可得
\begin{equation}
	g = \frac{(g_0+R\omega_0^2)\sin 2\lambda}{\sin \lambda \cos \alpha+\cos \lambda\sin \alpha}
	\label{偏角与纬度和重力加速度的关系2}
\end{equation}
联立式\eqref{偏角与纬度和重力加速度的关系1}和\eqref{偏角与纬度和重力加速度的关系2}可得$g$作为$\alpha$和$\lambda$很复杂的函数关系,下面作些近似以简化之。首先根据式\eqref{偏角与纬度和重力加速度的关系1}可有近似
\begin{equation}
	\sin \alpha \approx \alpha \approx \frac{R\omega_0^2}{2g_0} \sin 2\lambda
	\label{偏角与纬度和重力加速度的关系3}
\end{equation}
然后将式\eqref{偏角与纬度和重力加速度的关系3}代入式\eqref{偏角与纬度和重力加速度的关系2}中,并取$\cos \alpha \approx 1$,略去其中的二阶以上的小量,得
\begin{align}
	g & \approx \frac{(g_0+R\omega_0^2) \sin \lambda}{\sin \lambda + \cos \lambda \dfrac{R\omega_0^2}{2g_0}\sin 2\lambda} = \frac{g_0+R\omega_0^2}{1+\dfrac{R\omega_0^2 \cos^2 \lambda}{g_0}} \nonumber\\
	& \approx g_0\left(1+\frac{R \omega_0^2}{g_0} \sin^2 \lambda\right) \label{偏角与纬度和重力加速度的关系4}
\end{align}
式\eqref{偏角与纬度和重力加速度的关系4}和式\eqref{偏角与纬度和重力加速度的关系3}就是$g$和$\alpha$随$\lambda$改变的近似公式。如果再将地球半径随纬度的变化关系考虑进去,可以得到更精确的计算公式。

\subsection{落体偏东}\label{节:落体偏东}

考虑惯性力的情况下,地球表面质点的运动方程为
\begin{equation*}
	m\mbf{a}' = \mbf{F} - m\mbf{\omega}_0 \times (\mbf{\omega_0} \times \mbf{v}') - 2m\mbf{\omega_0} \times \mbf{v}'
\end{equation*}
由于地球自转角速度$\omega_0$是一个小量,略去上式中$\omega_0^2$项(惯性离心力),可得质点相对于地球的运动方程为
\begin{equation}
	m\mbf{a}' = \mbf{F} - 2m\mbf{\omega_0} \times \mbf{v}'
	\label{地球表面质点的运动微分方程}
\end{equation}

\begin{figure}[htb]
\centering
\begin{asy}
	size(250);
	//地球表面的坐标系
	pair O,i,j,k;
	real x,y,z,r;
	path clp;
	picture tmp;
	O = (0,0);
	x = 2.5;
	y = 1.5;
	z = 1.5;
	r = 1;
	i = (-sqrt(2)/4,-sqrt(14)/12);
	j = (sqrt(14)/4,-sqrt(2)/12);
	k = (0,2*sqrt(2)/3);
	draw(Label("$x$",EndPoint),O--x*i,Arrow);
	draw(Label("$y$",EndPoint),O--y*j,Arrow);
	draw(Label("$z$",EndPoint),O--z*k,Arrow);
	clp = r*i--r*j--r*k--cycle;
	unfill(clp);
	draw(tmp,Label("$x_0$",EndPoint),O--x*i,dashed,Arrow);
	draw(tmp,Label("$y_0$",EndPoint),O--y*j,dashed,Arrow);
	draw(tmp,Label("$z_0$",EndPoint),O--z*k,dashed,Arrow);
	label(tmp,"$O$",O,NW);
	clip(tmp,clp);
	add(tmp);
	erase(tmp);
	real phi,theta;
	pair vcir(real t){
		return r*sin(t)*cos(phi)*i+r*sin(t)*sin(phi)*j+r*cos(t)*k;
	}
	pair hcir(real t){
		return r*sin(theta)*cos(t)*i+r*sin(theta)*sin(t)*j+r*cos(theta)*k;
	}
	pair sphcoor(real R,real theta,real phi){
		return R*sin(theta)*cos(phi)*i+R*sin(theta)*sin(phi)*j+R*cos(theta)*k;
	}
	phi = 0;
	draw(graph(vcir,-0.4,pi-0.4));
	draw(graph(vcir,pi-0.4,2*pi-0.4),dashed);
	phi = pi/2;
	draw(graph(vcir,-0.9,pi-1));
	draw(graph(vcir,pi-1,2*pi-0.9),dashed);
	theta = pi/2;
	draw(graph(hcir,-1,pi-1.2));
	draw(graph(hcir,pi-1.2,2*pi-1),dashed);
	theta = pi/4;
	draw(graph(hcir,-1.3,pi-1.1),orange);
	draw(graph(hcir,pi-1.1,2*pi-1.3),dashed+orange);
	phi = pi/3;
	draw(graph(vcir,-0.5,pi-0.6),orange);
	draw(graph(vcir,pi-0.6,2*pi-0.5),dashed+orange);
	draw(scale(r)*unitcircle,linewidth(1bp));
	pair P,Q,er,et,ep;
	P = sphcoor(r,theta,phi);
	Q = sphcoor(r,pi/2,phi);
	er = 0.6*sphcoor(r,theta,phi);
	et = 0.6*sphcoor(r,theta+pi/2,phi);
	ep = 0.6*sphcoor(r,pi/2,phi+pi/2);
	draw(Label("$A$",EndPoint,ESE),O--P,dashed);
	draw(O--Q,dashed+orange);
	draw(Label("$z$",EndPoint,black),P--P+er,red,Arrow);
	draw(Label("$x$",EndPoint,Relative(W),black),P--P+et,red,Arrow);
	draw(Label("$y$",EndPoint,black),P--P+ep,red,Arrow);
	dot(P);
	r = 0.3;
	draw(Label("$\lambda$",MidPoint,Relative(E)),graph(vcir,pi/2,theta),Arrow);
\end{asy}
\caption{地球表面的坐标系}
\label{地球表面的坐标系}
\end{figure}

取固定在地球上的参考系$Axyz$,$x$轴向南,$y$轴向东,$z$轴垂直地面向上,如图\ref{地球表面的坐标系}所示。由于此处忽略了惯性离心力的作用,因此重力方向和$z$轴平行,而
\begin{equation*}
	\mbf{\omega}_0 = \omega_0 \mbf{e}_3 = \omega_0(-\cos \lambda \mbf{i} + \sin \lambda \mbf{k})
\end{equation*}
式中$\lambda$为$A$点的纬度。由此运动方程\eqref{地球表面质点的运动微分方程}在$Axyz$系中的分量方程可以表示为
\begin{equation}
\begin{cases}
	m\ddot{x} = F_x+2m\omega_0 \dot{y} \sin \lambda \\
	m\ddot{y} = F_y-2m\omega_0 (\dot{x} \sin \lambda + \dot{z} \cos \lambda) \\
	m\ddot{z} = F_z+2m\omega_0 \dot{y} \cos \lambda
\end{cases}
\label{地球表面质点的运动微分方程的分量形式}
\end{equation}

如果质点从有限高度$h$以初速度$v' = 0$自由下落,不考虑阻力等其它力的作用,重力$mg$看成常数,则对式\eqref{地球表面质点的运动微分方程的分量形式}积分可得
\begin{equation}
	\begin{cases}
		\dot{x} = 2\omega_0 y\sin \lambda \\
		\dot{y} = -2\omega_0 \big[x\sin \lambda+(z-h) \cos \lambda\big] \\
		\dot{z} = -gt + 2\omega_0 y \cos \lambda
	\end{cases}
	\label{地球表面的自由落体运动方程}
\end{equation}
将式\eqref{地球表面的自由落体运动方程}代入式\eqref{地球表面质点的运动微分方程的分量形式}的右端,并继续略去$\omega_0$的二阶项,得
\begin{equation*}
\begin{cases}
	\ddot{x} = 0 \\
	\ddot{y} = 2gt\omega_0\cos \lambda \\
	\ddot{z} = -g
\end{cases}
\end{equation*}
再对时间积分两次可得
\begin{equation}
\begin{cases}
	x = 0 \\[1.5ex]
	y = \dfrac13 gt^3 \omega_0 \cos \lambda \\[1.5ex]
	z = h - \dfrac12 gt^2
\end{cases}
\label{地球表面的自由落体的解}
\end{equation}
消去$t$可得落体的轨道方程为
\begin{equation*}
	y^2 = -\frac89 \frac{\omega_0^2 \cos^2 \lambda}{g} (z-h)^3
\end{equation*}
这是位于东西铅直平面内的半三次拋物线。当落体落到地面时,落体偏东的数值为
\begin{equation}
	y = \frac{2\sqrt{2}}{3} \omega_0 h \sqrt{\frac{h}{g}} \cos \lambda
\end{equation}

\subsection{地转偏向力}

%由于地球不是一个严格的惯性系,因此在地球表面附近运动的物体相当于受到了额外的Coriolis力。一般又将Coriolis力称为{\bf 地转偏向力}。

下面以地球表面炮弹平射轨道受地转偏向力影响为例来定量计算地转偏向力对长时间运动的影响。取固定在地球上的参考系$Axyz$,$x$轴向南,$y$轴向东,$z$轴垂直地面向上,如图\ref{地球表面的坐标系}所示。同样忽略惯性离心力,则质点相对于地球的运动方程将由式\eqref{地球表面质点的运动微分方程}来表示,其分量形式即为式\eqref{地球表面质点的运动微分方程的分量形式},其中$F_x=F_y=0, F_z=-mg$,则有
\begin{equation}
\begin{cases}
	\ddot{x} = 2\omega_0\dot{y}\sin\lambda \\
	\ddot{y} = -2\omega_0(\dot{x}\sin\lambda+\dot{z}\cos\lambda) \\
	\ddot{z} = -g+2\omega_0\dot{y}\cos\lambda
\end{cases}
\label{chapter7:地转偏向力方程分量形式}
\end{equation}
对式\eqref{chapter7:地转偏向力方程分量形式}积分,并利用初始条件,即$t=0$时,$x=y=z=0, \dot{x}=v_0\sin\theta, \dot{y}=v_0\cos\theta, \dot{z}=0$,其中$\theta$为初始速度与正东方向的夹角,则有
\begin{equation}
\begin{cases}
	\dot{x} = v_0\sin\theta + 2\omega_0y\sin\lambda \\
	\dot{y} = v_0\cos\theta - 2\omega_0(x\sin\lambda+z\cos\lambda) \\
	\dot{z} = -gt+2\omega_0y\cos\lambda
\end{cases}
\label{chapter7:地转偏向力中间结果1}
\end{equation}
将式\eqref{chapter7:地转偏向力中间结果1}代入式\eqref{chapter7:地转偏向力方程分量形式}中,并略去$\omega_0$的二阶项,可得
\begin{equation}
\begin{cases}
	\ddot{x} = 2\omega_0 v_0 \sin\lambda\cos\theta \\
	\ddot{y} = -2\omega_0v_0 \sin\lambda\sin\theta + 2g\omega_0 t\cos\lambda \\
	\ddot{z} = -g+2\omega_0v_0 \cos\lambda\cos\theta
\end{cases}
\label{chapter7:地转偏向力中间结果2}
\end{equation}
将式\eqref{chapter7:地转偏向力中间结果2}积分两次,可得
\begin{equation}
\begin{cases}
	x = v_0t\sin\theta + \omega_0v_0t^2\sin\lambda\cos\theta \\
	y = v_0t\cos\theta - \omega_0v_0t^2\sin\theta\sin\lambda + \dfrac13gt^3 \omega_0\cos\lambda \\
	z = -\dfrac12 gt^2 + \omega_0v_0t^2\cos\theta\cos\lambda
\end{cases}
\label{chapter7:地转偏向力结果1}
\end{equation}
式\eqref{chapter7:地转偏向力结果1}中,第二式的第三项即为第\ref{节:落体偏东}节中落体偏东的结果。因此可见,Coriolis力引起的偏离为
\begin{equation}
\begin{cases}
	\Delta x = \omega_0v_0t^2 \cos\theta\sin\lambda \\
	\Delta y = -\omega_0v_0t^2 \sin\theta\sin\lambda \\
	\Delta z = \omega_0v_0t^2 \cos\theta\cos\lambda
\end{cases}
\end{equation}
由此可以看出,在北半球$\lambda>0$,则$\Delta x>0,\Delta y<0$,炮弹向右偏离;而在南半球$\lambda<0$,则$\Delta x<0,\Delta y>0$,炮弹向左偏离。设炮弹通过的路程为$l$,则其水平偏移量为
\begin{equation}
	\Delta s = \sqrt{\Delta x^2 + \Delta y^2} = \omega_0v_0t^2\sin\lambda = \omega_0v_0\left(\frac{l}{v_0}\right)^2\sin\lambda = \frac{\omega_0l^2}{v_0}\sin\lambda
\end{equation}

实际上,本节的公式也可以适用于求解落体偏东,只需令$v_0=0$即可。

本节是利用逐次逼近的方法求解微分方程组\eqref{chapter7:地转偏向力中间结果2}的,而实际上,方程\eqref{chapter7:地转偏向力中间结果2}是可以严格求解的。

将方程\eqref{chapter7:地转偏向力中间结果2}中第一式乘以$\sin\lambda$加上第三式乘以$\cos\lambda$,可得
\begin{equation*}
\begin{cases}
	\ddot{x}\sin\lambda+\ddot{z}\cos\lambda = -g\cos\lambda + 2\omega_0 \dot{y} \\
	\ddot{y} = -2\omega_0(\dot{x}\sin\lambda+\dot{z}\cos\lambda)
\end{cases}
\end{equation*}
令$\xi = \dot{x}\sin\lambda+\dot{z}\cos\lambda, \eta = \dot{y}$,则有
\begin{equation}
\begin{cases}
	\dot{\xi} = -g\cos\lambda + 2\omega_0\eta \\
	\dot{\eta} = -2\omega_0\xi
\end{cases}
\label{chapter7:地转偏向力-严格求解1}
\end{equation}
在方程\eqref{chapter7:地转偏向力-严格求解1}中消去$\eta$可得方程
\begin{equation*}
	\ddot{\xi} + 4\omega_0^2 \xi = 0
\end{equation*}
由此可解得
\begin{equation}
\begin{cases}
	\xi = A\cos2\omega_0t+B\sin2\omega_0t \\
	\eta = \dfrac{g}{2\omega_0}\cos\lambda - A\sin 2\omega_0t+B\cos2\omega_0t
\end{cases}
\label{chapter7:地转偏向力-严格求解2}
\end{equation}
利用初始条件,即
\begin{equation*}
\begin{cases}
	\xi|_{t=0} = \dot{x}|_{t=0}\sin\lambda+\dot{z}|_{t=0}\cos\lambda = v_0\sin\theta\sin\lambda \\
	\eta|_{t=0} = \dot{y}|_{t=0} = v_0\cos\theta
\end{cases}
\end{equation*}
由此可得
\begin{equation}
\begin{cases}
	\ds \xi = v_0\sin\theta\sin\lambda \cos2\omega_0t+\left(v_0\cos\theta-\frac{g}{2\omega_0}\cos\lambda\right)\sin2\omega_0t \\
	\ds \eta = \dot{y} = -v_0\sin\theta\sin\lambda\sin2\omega_0t+v_0\cos\theta\cos2\omega_0t+\frac{g}{2\omega_0}\cos\lambda\left(1-\cos2\omega_0t\right)
\end{cases}
\label{chapter7:地转偏向力-严格求解3}
\end{equation}
将\eqref{chapter7:地转偏向力-严格求解3}的第二式代入方程\eqref{chapter7:地转偏向力中间结果2}的右端并积分,可得
\begin{equation}
\begin{cases}
	\ds \dot{x} = v_0\sin\theta+v_0\sin\theta\sin^2\lambda(\cos2\omega_0t-1) + v_0\cos\theta\sin\lambda\sin 2\omega_0t \\
	\ds \phantom{\dot{x} = }{} + g\sin\lambda\cos\lambda\left(t-\frac{1}{2\omega_0}\sin2\omega_0t\right) \\
	\ds \dot{z} = -gt + v_0\sin\theta\sin\lambda\cos\lambda(\cos2\omega_0t-1) + v_0\cos\theta\cos\lambda\sin2\omega_0t \\
	\ds \phantom{\dot{z} = }{} + g\cos^2\lambda\left(t-\frac{1}{2\omega_0}\sin2\omega_0t\right)
\end{cases}
\label{chapter7:地转偏向力-严格求解4}
\end{equation}
将式\eqref{chapter7:地转偏向力-严格求解4}和\eqref{chapter7:地转偏向力-严格求解3}的第二式再积分一次,即可得
\begin{equation}
\begin{cases}
	\ds x = v_0t\sin\theta+v_0\sin\theta\sin^2\lambda\left(\frac{1}{2\omega_0}\sin2\omega_0t-t\right) - \frac{v_0}{2\omega_0}\cos\theta\sin\lambda(\cos 2\omega_0t-1) \\
	\ds \phantom{\dot{x} = }{} + g\sin\lambda\cos\lambda\left(\frac12t^2-\frac{1}{4\omega_0^2}\cos2\omega_0t+\frac{1}{4\omega_0^2}\right) \\
	\ds y = \frac{v_0}{2\omega_0}\sin\theta\sin\lambda(\cos2\omega_0t-1) +\frac{g}{2\omega_0}\cos\lambda\left(t-\frac{1}{2\omega_0}\sin2\omega_0t\right) \\
	\ds \phantom{\dot{x} = }{} +\frac{v_0}{2\omega_0}\cos\theta\sin2\omega_0t \\
	\ds z = -\frac12gt^2 + g\cos^2\lambda\left(\frac12t^2-\frac{1}{4\omega_0^2}\cos2\omega_0t+\frac{1}{4\omega_0^2}\right) \\
	\ds \phantom{\dot{x} = }{}+v_0\sin\theta\sin\lambda\cos\lambda\left(\frac{1}{2\omega_0}\sin2\omega_0t-t\right) - \frac{v_0}{2\omega_0}\cos\theta\cos\lambda(\cos2\omega_0t-1)
\end{cases}
\label{chapter7:地转偏向力-严格求解-解}
\end{equation}
式\eqref{chapter7:地转偏向力-严格求解-解}即为方程\eqref{chapter7:地转偏向力中间结果2}的严格解,如果将其按$\omega_0$的幂级数展开,并保留至$\omega_0$的一阶项,将得到与式\eqref{chapter7:地转偏向力结果1}相同的结果。

\subsection{Foucault(傅科)摆}

假如在北极悬挂一单摆使之作微振动,在惯性系中的观察者看来摆的振动面始终在铅直平面内,但地球在以$\mbf{\omega}_0$的角速度自转,因此在非惯性系的地球上的观察者看来,摆的振动面应以$-\mbf{\omega}_0$的角速度转动。这是Foucault在1860年首先指出的,藉此可以直接证实地球是具有自转的。除了赤道之外的各地都可以观察到这种效应,下面对此问题作一近似分析。

\begin{figure}[htb]
\centering
\begin{asy}
	size(250);
	//Foucault摆示意图
	pair O,i,j,k;
	real x,y,z,zz,r,l;
	O = (0,0);
	i = 0.5*dir(-135);
	j = 1*dir(0);
	k = 1*dir(90);
	draw(Label("$x$",EndPoint),O--i,Arrow);
	draw(Label("$y$",EndPoint),O--j,Arrow);
	draw(Label("$z$",EndPoint,Relative(W)),O--k,Arrow);
	label("$A$",O,W);
	zz = 1.2;
	x = 0.5;
	y = 0.7;
	z = 0.4;
	r = 0.03;
	draw(k--zz*k);
	draw(Label("$l$",Relative(0.4),Relative(E)),zz*k--x*i+y*j+z*k);
	fill(shift(x*i+y*j+z*k)*scale(r)*unitcircle,black);
	label("$m$",x*i+y*j+z*k,2*E);
	draw(Label("$\boldsymbol{F}$",EndPoint,Relative(E)),x*i+y*j+z*k--interp(x*i+y*j+z*k,zz*k,0.3),Arrow);
	draw(x*i+y*j+z*k--x*i+y*j,dashed);
	draw(x*i--x*i+y*j--y*j,dashed);
	l = 0.4;
	draw(shift(zz*k)*(l*dir(180)--l*dir(0)));
	int imax;
	pair P,dash,pace;
	imax = 20;
	dash = 0.05*dir(60);
	pace = 2*l/imax*dir(0);
	P = zz*k+l*dir(180)+0.2*pace;
	for(int i=1;i<=imax;i=i+1){
		draw(P--P+dash);
		P = P+pace;
	}
\end{asy}
\caption{Foucault摆}
\label{Foucault摆示意图}
\end{figure}

设摆长为$l$,摆球的质量为$m$,摆杆的张力为$\mbf{F}$,取与节\ref{节:落体偏东}相同的坐标系$Axyz$,其中$A$点为摆球的平衡位置。于是,张力$\mbf{F}$的三个分量为
\begin{equation*}
\begin{cases}
	F_x = -\dfrac{x}{l} F \\[1.5ex]
	F_y = -\dfrac{y}{l} F \\[1.5ex]
	F_z = -\dfrac{l-z}{l} F
\end{cases}
\end{equation*}
代入式\eqref{地球表面质点的运动微分方程}可得摆球的运动方程为
\begin{equation}
\begin{cases}
	m\ddot{x} = -\dfrac{x}{l} F+2m\omega_0 \dot{y}\sin \lambda \\[1.5ex]
	m\ddot{y} = -\dfrac{y}{l} F-2m\omega_0(\dot{x}\sin \lambda+ \dot{z}\cos \lambda) \\[1.5ex]
	m\ddot{z} = \dfrac{l-z}{l}F-mg+2m\omega_0\dot{y}\cos \lambda
\end{cases}
\label{Foucault摆摆球的运动方程}
\end{equation}
以及一个约束方程
\begin{equation}
	x^2+y^2+(l-z)^2 = l^2
	\label{Foucault摆摆球的约束方程}
\end{equation}
通过解方程组\eqref{Foucault摆摆球的运动方程}和\eqref{Foucault摆摆球的约束方程},即可得出摆的运动规律。但这组方程很难严格求解,但考虑到微振动条件下,$x$、$y$和$\dot{x}$、$\dot{y}$都是小量,因而由式\eqref{Foucault摆摆球的约束方程}得
\begin{equation*}
	l-z = \sqrt{l^2-(x^2+y^2)} = l\left(1-\frac{x^2+y^2}{2l} + o(x^2+y^2)\right)
\end{equation*}
因此,$z$是二阶小量,$\dot{z}$和$\ddot{z}$也是二阶小量。在忽略二阶小量的情况下,由方程\eqref{Foucault摆摆球的运动方程}的第三式可得$F=mg$,代入另外两式中可得
\begin{equation*}
\begin{cases}
	\ddot{x} - 2\omega_0 \dot{y}\sin \lambda + \omega^2 x = 0 \\
	\ddot{y} + 2\omega_0 \dot{x}\sin \lambda + \omega^2 y = 0
\end{cases}
\end{equation*}
式中$\omega^2 =\dfrac{g}{l}$。将第二式乘以$\mathrm{i}$再与第一式相加,可得一个复变数方程
\begin{equation}
	\ddot{\xi} + \mathrm{i} 2\omega_0 \dot{\xi} \sin \lambda + \omega^2 \xi = 0
	\label{Foucault摆复变数微分方程}
\end{equation}
其中$\xi = x + \mathrm{i}y$,方程\eqref{Foucault摆复变数微分方程}的通解为
\begin{equation*}
	\xi = A\mathrm{e}^{n_1 t} + B\mathrm{e}^{n_2 t}
\end{equation*}
其中$A$和$B$为两个复常数,由初始条件决定。$n_1$和$n_2$为特征方程
\begin{equation*}
	n^2 + \mathrm{i} \big(2\omega_0 \sin \lambda \big) n + \omega^2 = 0
\end{equation*}
的两个根:
\begin{equation*}
\begin{cases}
	n_1 = -\mathrm{i} \omega_0 \sin \lambda + \mathrm{i} \sqrt{\omega_0^2 \sin^2 \lambda + \omega^2} \approx -\mathrm{i} \omega_0 \sin \lambda + \mathrm{i} \omega \\
	n_2 = -\mathrm{i} \omega_0 \sin \lambda - \mathrm{i} \sqrt{\omega_0^2 \sin^2 \lambda + \omega^2} \approx -\mathrm{i} \omega_0 \sin \lambda - \mathrm{i} \omega
\end{cases}
\end{equation*}
因此可将通解表示为
\begin{equation}
	\xi = \mathrm{e}^{-\mathrm{i}(\omega_0\sin \lambda) t} \big(A \mathrm{e}^{\mathrm{i} \omega t} + B \mathrm{e}^{-\mathrm{i} \omega t}\big)
	\label{Foucault摆通解}
\end{equation}
当地球没有自转,即$\omega_0 = 0$时,方程\eqref{Foucault摆摆球的运动方程}的解为
\begin{equation*}
	\xi = A \mathrm{e}^{\mathrm{i} \omega t} + B \mathrm{e}^{-\mathrm{i} \omega t} = (A+B) \cos \omega t + \mathrm{i}(A-B) \sin \omega t = x'+\mathrm{i} y'
\end{equation*}
所以
\begin{equation}
\begin{cases}
	x' = (A+B)\cos \omega t \\
	y' = (A-B)\sin \omega t
\end{cases}
\end{equation}

由此可知,当地球自转效应忽略时,摆球走一椭圆轨道,$x'$和$y'$分别为椭圆的两个主轴。当考虑地球自转时,即$\omega_0 \neq 0$时,由式\eqref{Foucault摆通解}可得
\begin{align*}
	\xi & = x+\mathrm{i} y = \big[\cos(\omega_0 t\sin \lambda)- \mathrm{i} \sin (\omega_0 t\sin \lambda)\big](x'+\mathrm{i} y') \\
	& = \big[x'\cos(\omega_0 t\sin \lambda) + y'\sin(\omega t\sin \lambda)\big] + \mathrm{i} \big[-x'\sin(\omega_0 t\sin \lambda) + y'\cos(\omega_0 t\sin \lambda)\big]
\end{align*}
所以
\begin{equation}
\begin{cases}
	x = x'\cos(\omega_0 t\sin \lambda) + y'\sin(\omega t\sin \lambda) \\
	y = -x'\sin(\omega_0 t\sin \lambda) + y'\cos(\omega_0 t\sin \lambda)
\end{cases}
\end{equation}
这表明,摆球在作周期为
\begin{equation*}
	T = \frac{2\pi}{\omega} = 2\pi \sqrt{\frac{l}{g}}
\end{equation*}
的椭圆轨道运动,同时此椭圆的轴$x'$和$y'$又以角速度$-\omega_0 \sin\lambda$绕$Oz$转动,亦即摆的振动面以角速度$-\omega_0 \sin \lambda$旋转,它们的几何关系如图\ref{Foucault摆振动面的旋转}所示。

\begin{figure}[htb]
\centering
\begin{asy}
	size(200);
	//Foucault摆振动面的旋转
	pair O,x,y;
	real a,b,theta,r;
	O = (0,0);
	x = (5,0);
	y = (0,4);
	a = 4;
	b = 3;
	r = 0.8;
	theta = -30;
	label("$O$",O,W);
	draw(Label("$x'$",EndPoint),rotate(theta)*(O--x),Arrow);
	draw(Label("$y'$",EndPoint),rotate(theta)*(O--y),Arrow);
	draw(rotate(theta)*xscale(a)*yscale(b)*unitcircle);
	draw(Label("$x$",EndPoint),O--x,Arrow);
	draw(Label("$y$",EndPoint),O--y,Arrow);
	draw(Label("$\omega_0 t\sin \lambda$",MidPoint,Relative(W)),arc(O,r,0,theta),Arrow);
\end{asy}
\caption{Foucault摆振动面的旋转}
\label{Foucault摆振动面的旋转}
\end{figure}


%附录——金尚年《经典力学》
\chapter{习题及解答}

\section{课后习题解答}

\subsection{牛顿力学}

\begin{question}[金尚年《理论力学》30页1.13]
质点$A$约束在光滑水平平台上运动,在此质点上系着一根长为$l$的轻绳,绳子穿过平台上的小孔,另一端垂直地挂着另一个质点$B$,如图\ref{理论力学:30页1.13}所示。
\begin{figure}[htb]
\centering
\begin{asy}
	size(200);
	//理论力学:30页1.13
	pair O,a,b,m,M;
	real r;
	O = (0,0);
	a = (1.5,0);
	b = 0.8*dir(60);
	m = dir(30);
	M = (0,-1.5);
	r = 0.05;
	draw((O+a+b)--(O-a+b)--(O-a-b)--(O+a-b)--cycle);
	draw(O--0.6*a,dashed);
	draw(Label("$r$",MidPoint,Relative(W)),O--m);
	draw(O--(0,-b.y),dashed);
	draw((0,-b.y)--M);
	draw(Label("$z$",EndPoint),O--(0,1),Arrow);
	fill(shift(m)*scale(r)*unitcircle,black);
	label("$m$",m,E);
	fill(shift(M)*scale(r)*unitcircle,black);
	label("$M$",M,W);
	label("$\theta$",0.5*dir(15));
	unfill(scale(r)*unitcircle);
	draw(scale(r)*unitcircle);
	label("$O$",O,W);
\end{asy}
\caption{题\thequestion}
\label{理论力学:30页1.13}
\end{figure}
\begin{enumerate}
	\item 问此动力学体系的动量、角动量、能量是否守恒,并解释之;
	\item 用质点系动量定理写出体系的运动微分方程;
	\item 若$t=0$时,质点$A$离$O$的距离为$a$,速度为$v_0 = \sqrt{\dfrac{9ag}{2}}$,其方向垂直于$OA$,且$m_A=m_B=m$,证明以后质点$A$离$O$点的距离始终在$a$和$3a$之间。
\end{enumerate}
\end{question}
\begin{solution}
\begin{enumerate}
\item 体系合外力非零,因此体系动量不守恒;体系对$O$点力矩为零,因此体系角动量守恒;体系非保守外力(绳子张力)不做功,其余外力(重力)为保守力,因此体系能量守恒。
\item 首先求得体系的动量
\begin{equation*}
	\mbf{p} = \mbf{p}_A + \mbf{p}_B = m_A(\dot{r} \mbf{e}_r + r\dot{\theta} \mbf{e}_\theta) + m_B \dot{z} \mbf{e}_z
\end{equation*}
因此
\begin{align*}
	\frac{\mathrm{d} \mbf{p}}{\mathrm{d} t} & = m_A(\ddot{r}-r\dot{\theta}^2) \mbf{e}_r + m_A(2\dot{r}\dot{\theta}+r\ddot{\theta}) \mbf{e}_\theta + m_B \ddot{z} \mbf{e}_z \\
	& = m_A(\ddot{r}-r\dot{\theta}^2) \mbf{e}_r + m_A(2\dot{r}\dot{\theta}+r\ddot{\theta}) \mbf{e}_\theta + m_B \ddot{r} \mbf{e}_z
\end{align*}
此处利用了
\begin{equation*}
	\dot{\mbf{e}}_r = \dot{\theta} \mbf{e}_\theta,\quad \dot{\mbf{e}}_\theta = -\dot{\theta} \mbf{e}_r
\end{equation*}
以及约束$r-z=\text{常数}$。体系受到的外力为
\begin{equation*}
	\mbf{F} = -F_T \mbf{e}_r + (F_T-m_Bg) \mbf{e}_z
\end{equation*}
因此根据质点系动量定理$\dfrac{\mathrm{d} \mbf{p}}{\mathrm{d} t} = \mbf{F}$可得体系的运动微分方程为
\begin{subnumcases}{}
	m_A(\ddot{r}-r\dot{\theta}^2) = -F_T \label{理论力学:1.13-1} \\
	m_A(2\dot{r}\dot{\theta} + r\ddot{\theta}) = 0 \label{理论力学:1.13-2} \\
	m_B\ddot{r} = F_T - m_Bg \label{理论力学:1.13-3}
\end{subnumcases}
\item 由式\eqref{理论力学:1.13-2}可得
\begin{equation*}
	2\dot{r}\dot{\theta} + r\ddot{\theta} = \frac{\mathrm{d}}{\mathrm{d} t}(r^2 \dot{\theta}) = 0
\end{equation*}
即有
\begin{equation}
	r^2 \dot{\theta} = l = av_0 = a\sqrt{\frac{9ag}{2}}
	\label{理论力学:1.13-4}
\end{equation}
再由式\eqref{理论力学:1.13-1}和式\eqref{理论力学:1.13-3}消去$F_T$可得
\begin{equation*}
	2\ddot{r} - r\dot{\theta}^2 + g = 0
\end{equation*}
再由式\eqref{理论力学:1.13-4}以及
\begin{equation*}
	\ddot{r} = \frac{\mathrm{d} \dot{r}}{\mathrm{d} r} \dot{r} = \frac12 \frac{\mathrm{d} \left(\dot{r}^2\right)}{\mathrm{d} r}
\end{equation*}
可得$\dot{r}^2$满足的微分方程为
\begin{equation*}
	\frac{\mathrm{d} \left(\dot{r}^2\right)}{\mathrm{d} r} - \frac{l^2}{r^3} + g = 0
\end{equation*}
分离变量可得
\begin{equation*}
	\mathrm{d} \left(\dot{r}^2\right) = \left(\frac{l^2}{r^3} - g\right) \mathrm{d} r
\end{equation*}
两端积分(注意到初始时刻$\dot{r}=0$)可得
\begin{equation*}
	\dot{r}^2 = \int_a^r \left(\frac{l^2}{r^3} - g\right) \mathrm{d} r = \frac{l^2}{2a^2} + ga - \frac{l^2}{2r^2} - gr = g \frac{(r-a)(r-3a)\left(r+\dfrac34 a\right)}{r^2}
\end{equation*}
运动范围可由
\begin{equation*}
	\dot{r}^2 = g \frac{(r-a)(r-3a)\left(r+\dfrac34 a\right)}{r^2} \geqslant 0
\end{equation*}
确定,由此可解得
\begin{equation*}
	a \leqslant r \leqslant 3a
\end{equation*}
即质点$A$离$O$点的距离始终在$a$和$3a$之间。
\end{enumerate}
\end{solution}

\begin{question}[金尚年《理论力学》30页1.20]
一质量为$m$的质点约束在对称轴为铅直线,半顶角为$\alpha$的圆锥形漏斗的内表面运动。漏斗的上底半径为$R_1$,下底半径为$R_2$。开始时质点具有水平方向的初速度$v_0$,离下底的高度为$h$(如图\ref{理论力学:30页1.20}所示),在运动过程中摩擦力可忽略不计。试问质点能飞出漏斗之外,落入漏斗下面和在漏斗内做圆周运动三种情况下$v_0$和$h$各赢满足什么关系?
\begin{figure}[htb]
\centering
\begin{asy}
	size(250);
	//理论力学:30页1.20
	picture tmp;
	pair O,i,j,k;
	real R1,R2,r,R,H,alpha,phi,r0,l;
	path clp;
	O = (0,0);
	i = (-sqrt(2)/4,-sqrt(14)/12);
	j = (sqrt(14)/4,-sqrt(2)/12);
	k = (0,2*sqrt(2)/3);
	pair cir(real theta){
		return r*cos(theta)*i+r*sin(theta)*j+r/Tan(alpha)*k;
	}
	alpha = 20;
	R2 = 0.5;
	R1 = 2;
	r = R2;
	draw(graph(cir,0,2*pi),linewidth(0.8bp));
	draw(tmp,graph(cir,0,2*pi),linewidth(0.8bp)+dashed);
	clp = box((-R1,r/Tan(alpha)),(R1,10*r/Tan(alpha)));
	unfill(clp);
	clip(tmp,clp);
	add(tmp);
	erase(tmp);
	r = R1;
	draw(graph(cir,0,2*pi),linewidth(0.8bp));
	pair slop(real t){
		return t*cos(phi)*i+t*sin(phi)*j+t/Tan(alpha)*k;
	}
	phi = atan(j.x/i.x)+asin((i.y*j.x-i.x*j.y)/(k.y*sqrt(i.x**2+j.x**2))*Tan(alpha))+pi;
	draw(graph(slop,R2,R1),linewidth(0.8bp));
	draw(graph(slop,0,R2),linewidth(0.8bp)+dashed);
	r0 = 0.1;
	R = interp(R2,R1,0.7);
	l = 0.5;
	fill(shift(slop(R)+r0*dir(degrees(slop(R))+90))*scale(r0)*unitcircle,black);
	draw(slop(R)--slop(R)+l*dir(0));
	draw(slop(R2)--((slop(R)+l*dir(0)).x,slop(R2).y));
	draw(Label("$h$",MidPoint,Relative(W)),slop(R)+0.6*l*dir(0)--((slop(R)+0.6*l*dir(0)).x,slop(R2).y),Arrows);
	r0 = 0.5;
	draw(Label("$\alpha$",MidPoint,Relative(E)),arc(O,r0,degrees(slop(R2)),90));
	phi = atan(j.x/i.x)-asin((i.y*j.x-i.x*j.y)/(k.y*sqrt(i.x**2+j.x**2))*Tan(alpha));
	draw(graph(slop,R2,R1),linewidth(0.8bp));
	draw(graph(slop,0,R2),linewidth(0.8bp)+dashed);
	H = 2;
	draw(slop(R2)--slop(R2)+H*dir(-90),linewidth(1bp));
	draw(xscale(-1)*(slop(R2)--slop(R2)+H*dir(-90)),linewidth(1bp));
	draw((0,slop(R2).y)+1.1*H*dir(-90)--1.1*R1/Tan(alpha)*dir(90),dashed);
	draw(Label("$R_1$",MidPoint,Relative(E)),R1/Tan(alpha)*k--slop(R1));
	draw(Label("$R_2$",MidPoint,Relative(E)),R2/Tan(alpha)*k--slop(R2));
	label("$O$",O,SW);
\end{asy}
\caption{题\thequestion}
\label{理论力学:30页1.20}
\end{figure}
\end{question}
\begin{solution}
考虑质点恰好不能飞出漏斗的情况,在漏斗边缘质点的速度将在水平面内。根据能量守恒,可有
\begin{equation*}
	mgH + \frac12 mv^2 = mgh + \frac12 mv_0^2
\end{equation*}
质点对$O$点的力矩方向恒在水平平面内,因此质点竖直方向角动量守恒,即有
\begin{equation*}
	R_1 v = (R_2+h \tan \alpha) v_0
\end{equation*}
再考虑到
\begin{equation*}
	H = (R_1 - R_2)\cot \alpha
\end{equation*}
由此可解得
\begin{equation*}
	v_0^2 = \frac{2g \big[(R_1-R_2)\cot \alpha - h\big]R_1^2}{R_1^2 - (R_2 + h\tan \alpha)^2}
\end{equation*}
故质点能飞出漏斗之外的条件为
\begin{equation*}
	v_0^2 \geqslant \frac{2g \big[(R_1-R_2)\cot \alpha - h\big]R_1^2}{R_1^2 - (R_2 + h\tan \alpha)^2}
\end{equation*}

同理,考虑质点恰好不能落入漏斗下面的情况,在漏斗边缘质点的速度将在水平面内。同样根据能量守恒,可有
\begin{equation*}
	\frac12 mv'^2 = mgh + \frac12 mv_0^2
\end{equation*}
根据角动量守恒,可有
\begin{equation*}
	R_2 v' = (R_2 + h\tan \alpha) v_0
\end{equation*}
由此可解得
\begin{equation*}
	v_0^2 = \frac{2ghR_2^2}{(R_2+h\tan \alpha)^2 - R_2^2}
\end{equation*}
故质点能落入漏斗下面的条件为
\begin{equation*}
	v_0^2 \leqslant \frac{2ghR_2^2}{(R_2+h\tan \alpha)^2 - R_2^2}
\end{equation*}

在漏斗内做圆周运动时,需满足
\begin{equation*}
\begin{cases}
	N\cos \alpha = m\dfrac{v_0^2}{R_2 + h\tan \alpha} \\[1.5ex]
	N\sin \alpha - mg = 0
\end{cases}
\end{equation*}
由此可得质点在漏斗内做圆周运动的条件为
\begin{equation*}
	v_0^2 = g(h+R_2 \cot \alpha)
\end{equation*}
\end{solution}

\begin{question}[金尚年《理论力学》31页1.23]
总长度为$L$的软链放在水平光滑的桌面上,此时长为$l$的一部分链条从桌上下垂,起始时链条是静止的,求当链条末端滑倒桌子边缘时,链的速度$v$和所需的时间。
\end{question}
\begin{solution}
设软链的线密度为$\rho$,根据能量守恒,可有
\begin{equation*}
	\frac12 \rho L v^2 - \rho L g \frac{L}{2} = -\rho l g \frac{l}{2}
\end{equation*}
由此可有
\begin{equation*}
	v = \sqrt{\frac{g(L^2-l^2)}{L}}
\end{equation*}
为求所需时间,在下垂长度为$x$时,有
\begin{equation*}
	\frac12 \rho L v^2 - \rho x g \frac{x}{2} = -\rho l g \frac{l}{2}
\end{equation*}
即有
\begin{equation*}
	v = \frac{\mathrm{d} x}{\mathrm{d} t} = \sqrt{\frac{g(x^2-l^2)}{L}}
\end{equation*}
所以有
\begin{equation*}
	\sqrt{\frac{L}{g}} \frac{\mathrm{d} x}{\sqrt{x^2-l^2}} = \mathrm{d} t
\end{equation*}
两端积分可得
\begin{equation*}
	t = \int_l^L \sqrt{\frac{L}{g}} \frac{\mathrm{d} x}{\sqrt{x^2-l^2}} = \sqrt{\frac{L}{g}} \ln \frac{L+\sqrt{L^2-l^2}}{l}
\end{equation*}
\end{solution}

\begin{question}[金尚年《理论力学》32页1.37]
质量相同的两个物体$A$和$B$用弹簧链接(如图\ref{理论力学:32页1.37}所示),垂直至于地面上,原来静止。现给物体$A$以瞬间冲量$I$,问$I$要多大才可使物体$B$跳起来?
\begin{figure}[htb]
\centering
\begin{asy}
	size(200);
	//理论力学:32页1.37
	picture dashpic;
	pair O,A,B,dash,P;
	real d1,d2,H,I,ldash;
	path pdash;
	guide spring(real d,real h,int nmax){
	guide temp,springunit;
	temp = O--(d/2,h/2);
	springunit = (d/2,-3*h/2)--(-d/2,-h/2)--(d/2,h/2);
	for(int i=1;i<nmax;i=i+1){
		temp = temp--shift(2*i*h*dir(90))*springunit;
	}
	temp = temp--(-d/2,nmax*2*h-h/2)--(0,nmax*2*h);
	return temp;
	}
	O = (0,0);
	d1 = 1;
	d2 = 0.5;
	B = (0,d2);
	draw(shift(B)*box((-d1,-d2),(d1,d2)),linewidth(0.8bp));
	H = 2.5;
	A = (0,H+3*d2);
	draw(shift(A)*box((-d1,-d2),(d1,d2)),linewidth(0.8bp));
	draw(Label("$k$",MidPoint,2*E),shift(B+d2*dir(90)+0.5*d1*dir(0))*spring(0.1,H/20,10));
	draw(Label("$k$",MidPoint,2*W),shift(B+d2*dir(90)-0.5*d1*dir(0))*spring(0.1,H/20,10));
	label("$A$",A);
	label("$B$",B);
	I = 1;
	draw(Label("$I$",BeginPoint),A+(0,I+d2)--A+(0,d2),Arrow);
	ldash = 10;
	pdash = ldash*dir(180)--ldash*dir(0);
	dash = ldash*dir(-135);
	draw(dashpic,pdash,linewidth(1bp));
	for(real r=0;r<=1;r=r+0.01){
		P = relpoint(pdash,r);
		draw(dashpic,P--P+dash);
	}
	clip(dashpic,box((-2*d1,-0.2),(2*d1,1)));
	add(shift(B+d2*dir(-90))*dashpic);
\end{asy}
\caption{题\thequestion}
\label{理论力学:32页1.37}
\end{figure}
\end{question}
\begin{solution}
以弹簧的平衡位置为重力势能零点,设初始弹簧形变量为$x_0$,则有
\begin{equation*}
	2kx_0 = m_Ag
\end{equation*}
设冲量$I$给物体$A$的初始速度为$v$,物体$B$在物体$A$运动到最高点时地面的支持力最小,因此考虑当物体$A$运动到最高点时,根据能量守恒可有
\begin{equation*}
	2\cdot \frac12 kx^2 + m_Agx = 2 \cdot \frac12 kx_0^2 - m_Agx_0 + \frac12 m_Av^2
\end{equation*}
可得
\begin{equation*}
	x = \sqrt{\frac{m_A}{2k}} v - x_0
\end{equation*}
此时如果物体$B$恰好跳起来,则有
\begin{equation*}
	2kx = m_B g
\end{equation*}
由此可得
\begin{equation*}
	v = \frac{(m_A+m_B)g}{\sqrt{2km_A}}
\end{equation*}
根据动量定理
\begin{equation*}
	I = m_Av = (m_A+m_B)g\sqrt{\frac{m_A}{2k}}
\end{equation*}
\end{solution}

\subsection{Lagrange力学}

\begin{question}[金尚年《理论力学》64页2.12]
质量为$m$的质点约束在光滑的旋转抛物面$x^2+y^2=az$的内壁运动,$z$轴为铅直轴。写出:
\begin{enumerate}
\item 质点的运动方程;
\item 质点作圆周运动所应满足的条件。
\end{enumerate}
\end{question}
\begin{solution}
质点的自由度为$2$,取柱坐标$r,\theta$作为广义坐标,则有$z = \dfrac{r^2}{a}$。
\begin{enumerate}
\item 质点的Lagrange函数为
\begin{align*}
	L & = \frac12 m \left(\dot{r}^2 + r^2 \dot{\theta}^2 + \dot{z}^2\right) - mgz = \frac12 m \left[\left(1+\frac{4r^2}{a^2}\right)\dot{r}^2 + r^2 \dot{\theta}^2\right] - \frac{mg}{a}r^2
\end{align*}
$\theta$是循环坐标,故有
\begin{equation*}
	p_\theta = \frac{\pl L}{\pl \dot{\theta}} = mr^2 \dot{\theta} = L\text{(常数)}
\end{equation*}
Lagrange函数不显含时间,故有广义能量守恒,即
\begin{equation*}
	\frac12 m \left[\left(1+\frac{4r^2}{a^2}\right)\dot{r}^2 + r^2 \dot{\theta}^2\right] + \frac{mg}{a}r^2 = E\text{(常数)}
\end{equation*}
因此质点的运动方程为
\begin{equation*}
\begin{cases}
	\displaystyle mr^2 \dot{\theta} = L \\
	\displaystyle \frac12 m \left[\left(1+\frac{4r^2}{a^2}\right)\dot{r}^2 + r^2 \dot{\theta}^2\right] + \frac{mg}{a}r^2 = E
\end{cases}
\end{equation*}
\item 在运动方程中消去$\dot{\theta}$可得
\begin{equation*}
	\frac12 m \left(1+\frac{4r^2}{a^2}\right)\dot{r}^2 + \frac{L^2}{2mr^2} + \frac{mg}{a}r^2 = E
\end{equation*}
即将此运动等效为一个一维运动,其等效势能为
\begin{equation*}
	V_{\mathrm{eff}} = \frac{L^2}{2mr^2} + \frac{mg}{a}r^2
\end{equation*}
质点做圆周运动时,要求初始时质点即处于平衡位置(势能最低的位置),即
\begin{equation*}
	\left.\frac{\mathrm{d} V}{\mathrm{d} r}\right|_{r=r_0} = -\frac{L^2}{mr^3_0} + \frac{2mg}{a}r_0 = 0
\end{equation*}
考虑到$L = mr_0 v_0$以及$r_0^2 = ah$,可得质点做圆周运动需要满足的条件为
\begin{equation*}
	v_0^2 = 2gh
\end{equation*}
\end{enumerate}
\end{solution}

\begin{question}[金尚年《理论力学》65页2.19]
由两个相同的重物$P$和用铰链链接着的四根长度为$l$的轻棒及一个质量可忽略的弹簧$k$所组成的力学体系,其结构如图\ref{理论力学:65页2.19}所示。体系处于铅直平面内,$O$点是固定的,当图中的$\theta = \ang{45}$时弹簧处于固有长度,求平衡时体系的位置。
\begin{figure}[htb]
\centering
\begin{asy}
	size(250);
	//理论力学:65页2.19
	picture spring(real r,real d,int nmax){
	picture springpic;
		real theta0;
		path clp;
		pair prolate(real theta){
			return (r*theta-d*sin(theta),-d*cos(theta));
		}
		real f(real theta){
			return r*theta-d*sin(theta);
		}
		real fprime(real theta){
			return r-d*cos(theta);
		}
		theta0 = newton(f,fprime,pi/2);
		draw(springpic,graph(prolate,theta0,nmax*2*pi-theta0,10000)..((nmax+0.3)*2*pi*r,0)..((nmax+1)*2*pi*r,0),linewidth(0.8bp));
		return springpic;
	}
	picture dashpic;
	pair O,A,P,dash;
	real l,theta,d1,d2,ldash,ddash,r0;
	path tre;
	O = (0,0);
	l = 1;
	theta = 50;
	A = 2*l*Cos(theta)*dir(90);
	draw(Label("$l$",MidPoint,Relative(E)),O--l*dir(90-theta));
	draw(Label("$l$",MidPoint,Relative(E)),l*dir(90-theta)--A);
	draw(Label("$l$",MidPoint,Relative(E)),A--l*dir(90+theta));
	draw(Label("$l$",MidPoint,Relative(E)),l*dir(90+theta)--O);
	d1 = 0.03;
	d2 = 0.12;
	unfill(shift(A)*box((-d1,-d2),(d1,d2)));
	draw(A+(-d1,d2-2*d1/Tan(theta))--A+(-d1,-d2)--A+(d1,-d2)--A+(d1,d2-2*d1/Tan(theta)),linewidth(1bp));
	add(shift(A+d2*dir(-90))*rotate(-90)*scale((length(A-O)-d2)/(21*2*pi*0.025))*spring(0.025,0.08,20));
	label("$k$",(O+A)/2,2*W);
	draw(O--A+2*d2*dir(90));
	dot(O,UnFill);
	label("$O$",O,S);
	ddash = 1;
	ldash = 1;
	dash = ldash*dir(-135);
	tre = ldash*dir(180)--ldash*dir(0);
	draw(dashpic,tre,linewidth(1bp));
	for(real r=0;r<=1;r=r+0.015){
		P = relpoint(tre,r);
		draw(dashpic,P--P+dash);
	}
	clip(dashpic,box((-3*d1,-d1),(3*d1,1)));
	add(shift(A+2*d2*dir(90))*rotate(180)*dashpic);
	erase(dashpic);
	dash = ldash*dir(45);
	tre = ldash*dir(-90)--ldash*dir(90);
	for(real r=0;r<=1;r=r+0.015){
		P = relpoint(tre,r);
		draw(dashpic,P--P+dash);
	}
	clip(dashpic,box((0,-d2),(3*d1,d2)));
	draw(dashpic,box((0,-d2),(3*d1,d2)),linewidth(1bp));
	add(shift(l*dir(90-theta))*dashpic);
	add(shift(l*dir(90+theta)+3*d1*dir(180))*dashpic);
	label("$P$",l*dir(90-theta)+d2*dir(-90)+1.5*d1*dir(0),S);
	label("$P$",l*dir(90+theta)+d2*dir(-90)+1.5*d1*dir(180),S);
	r0 = 0.1;
	draw(Label("$\theta$",Relative(0.4),Relative(E)),arc(O,r0,90-theta,90));
\end{asy}
\caption{题\thequestion}
\label{理论力学:65页2.19}
\end{figure}
\end{question}
\begin{solution}
以$\theta$为广义坐标,可以求得体系的势能为
\begin{equation*}
	V = 2Pl\cos \theta + \frac12 k\left(2l\cos \theta - \sqrt{2}l\right)^2 
\end{equation*}
平衡位置$\theta_0$需满足
\begin{equation*}
	\left.\frac{\mathrm{d} V}{\mathrm{d} \theta} \right|_{\theta=\theta_0} = 0
\end{equation*}
可解得
\begin{equation*}
	\theta_0 = \arccos\left(\frac{\sqrt{2}}{2} - \frac{P}{2kl}\right)
\end{equation*}
\end{solution}

\begin{question}[金尚年《理论力学》66页2.24]
一质点在螺旋面上运动,螺旋面的方程为
\begin{equation*}
	x = R\cos \phi,\quad y = R\sin \phi,\quad z = b\phi
\end{equation*}
此质点并受辐向斥力,其大小与质点到轴的距离成正比。写出此质点的的Lagrange函数和运动方程。
\end{question}
\begin{solution}
辐向斥力可以表示为
\begin{equation*}
	\mbf{F} = 2kR \mbf{e}_r
\end{equation*}
此力对应的势能可表示为
\begin{equation*}
	V = -kR^2 
\end{equation*}
质点的速度为
\begin{equation*}
	\mbf{v} = \begin{pmatrix} \dot{x} \\ \dot{y} \\ \dot{z} \end{pmatrix} = \begin{pmatrix} \dot{R} \cos \phi - R\dot{\phi} \sin \phi \\ \dot{R} \sin \phi + R\dot{\phi} \cos \phi \\ b \dot{\phi} \end{pmatrix}
\end{equation*}
因此,质点的Lagrange函数为
\begin{equation*}
	L = T - V = \frac12 m \left(\dot{R}^2 + R^2 \dot{\phi}^2 + b^2 \dot{\phi}^2\right) + kR^2 
\end{equation*}
Lagrange函数中不含坐标$\phi$,故$\phi$是循环坐标,即有
\begin{equation*}
	p_\phi = \frac{\pl L}{\pl \dot{\phi}} = m(R^2+b^2)\dot{\phi} = l_\phi\text{(常数)}
\end{equation*}
再由$\displaystyle \frac{\mathrm{d}}{\mathrm{d} t} \frac{\pl L}{\pl \dot{R}} - \frac{\pl L}{\pl R} = 0$可得
\begin{equation*}
	m\ddot{R} - mR\dot{\phi}^2 -2kR = 0
\end{equation*}
综上,质点的运动方程为
\begin{equation*}
\begin{cases}
	m(R^2+b^2)\dot{\phi} = l_\phi\text{(常数)} \\
	m\ddot{R} - mR\dot{\phi}^2 -2kR = 0
\end{cases}
\end{equation*}
\end{solution}

\subsection{Hamilton力学}

\begin{question}[金尚年《理论力学》272页8.1]
写出自由质点在柱坐标和球坐标中的Hamilton函数。
\end{question}
\begin{solution}
在球坐标系中,质点的速度可以表示为
\begin{equation*}
	\mbf{v} = \dot{r} \mbf{e}_r + r\dot{\theta} \mbf{e}_\theta + r \dot{\phi} \sin \theta \mbf{e}_\phi
\end{equation*}
因此自由质点的Lagrange函数为
\begin{equation*}
	L = \frac12 m(\dot{r}^2 + r^2 \dot{\theta}^2 + r^2 \dot{\phi}^2 \sin^2 \theta)
\end{equation*}
广义动量分别为
\begin{equation*}
	p_r = \frac{\pl L}{\pl \dot{r}} = m\dot{r},\quad p_\theta = \frac{\pl L}{\pl \dot{\theta}} = mr^2 \dot{\theta},\quad p_\phi = \frac{\pl L}{\pl \dot{\theta}} = mr^2 \dot{\phi} \sin^2 \theta
\end{equation*}
所以Hamilton函数为
\begin{equation*}
	H = \sum_{\alpha=1}^3 p_\alpha \dot{q}_\alpha - L = \frac{p_r^2}{2m} + \frac{p_\theta^2}{2mr^2} + \frac{p_\phi^2}{2mr^2\sin^2 \theta}
\end{equation*}
\end{solution}

\begin{question}[金尚年《理论力学》272页8.4]
假定在两个固定点$(x_1,y_1)$和$(x_2,y_2)$之间作一条曲线,并让它绕$y$轴旋转而形成一个旋转曲面,求使此旋转曲面的表面积为极小的曲线方程式。
\end{question}
\begin{solution}
设曲线的方程可以表示为$x=f(y)$,则旋转体的表面积可以表示为
\begin{equation*}
	S[f(y)] = \int_{y_1}^{y_2} 2\pi f(y) \sqrt{1+\big[f'(y)\big]^2} \mathrm{d} y
\end{equation*}
此旋转体表面积取极小值的条件为$\delta S = 0$,根据Euler-Lagrange方程可有
\begin{equation*}
	\frac{\mathrm{d}}{\mathrm{d} y} \frac{\pl F}{\pl f'} - \frac{\pl F}{\pl f} = 0
\end{equation*}
其中$\displaystyle F(f,f',y) = 2\pi f(y) \sqrt{1+\big[f'(y)\big]^2}$,由于$F(f,f',y)$中不含有自变量$y$,故Euler-Lagrange方程有初积分
\begin{equation*}
	F - f'\frac{\pl F}{\pl f'} = a\, \text{(常数)}
\end{equation*}
由此方程整理可得
\begin{equation*}
	f'(y) = \sqrt{\frac{f^2(y)}{a^2}-1} = \frac{1}{a} \sqrt{x^2-a^2}
\end{equation*}
分离变量可有
\begin{equation*}
	\frac{\mathrm{d} x}{\sqrt{x^2-a^2}} = \frac{1}{a} \mathrm{d} y
\end{equation*}
两端积分,可有
\begin{equation*}
	\arcosh \frac{x}{a} = \frac{1}{a}(y-b)
\end{equation*}
因此所求曲线的方程为
\begin{equation*}
	x = a\cosh \frac{y-b}{a}
\end{equation*}
式中$a,b$是积分常数,可由边界条件
\begin{equation*}
\begin{cases}
	\displaystyle x_1 = a\cosh \frac{y_1-b}{a} \\
	\displaystyle x_2 = a\cosh \frac{y_2-b}{a}
\end{cases}
\end{equation*}
确定。
\end{solution}

\begin{question}[金尚年《理论力学》273页8.8]
质量为$m$的质点在重力场中自由落下。试利用正则变换母函数$F_1(y,\theta) = -mg\left(\dfrac{g}{6} \theta^3 + y\theta\right)$写出新的Hamilton函数和正则方程,并求解。
\end{question}
\begin{solution}
此系统原本的Hamilton函数为
\begin{equation*}
	H = \frac{p^2}{2m}+mgy
\end{equation*}
根据第一类正则变换母函数的结论,可有
\begin{equation*}
\begin{cases}
	\ds p = \frac{\pl F_1}{\pl y} = -mg\theta \\
	\ds P_\theta = -\frac{\pl F_1}{\pl \theta} = \frac12 mg^2\theta^2 + mgy \\
	\ds K = H+\frac{\pl F_1}{\pl t} = \frac{p^2}{2m}+mgy
\end{cases}
\end{equation*}
利用前两式消去$K$表达式中的$p$和$y$即可得新的Hamilton函数为
\begin{equation*}
	K = \frac{p^2}{2m}+mgy = \frac{(mg\theta)^2}{2m}+P_\theta-\frac12 mg^2\theta^2 = P_\theta
\end{equation*}
因此,新的正则方程为
\begin{equation*}
\begin{cases}
	\ds \dot{\theta} = \frac{\pl K}{\pl P_\theta} = 1 \\
	\ds \dot{P}_\theta = -\frac{\pl K}{\pl \theta} = 0
\end{cases}
\end{equation*}
因此可有
\begin{equation*}
	\theta = t+\beta,\quad P_\theta = \gamma
\end{equation*}
其中$\beta, \gamma$为积分常数。再由变换关系可得原问题的解为
\begin{equation*}
	y = \frac{1}{mg}\left(P_\theta-\frac12 mg^2\theta^2\right) = \frac{\gamma}{mg} - \frac12 g(t+\beta)^2
\end{equation*}
\end{solution}

\begin{question}[金尚年《理论力学》273页8.10]
证明$q = \sqrt{\dfrac{2Q}{k}}\cos P,\,p = \sqrt{2kQ}\sin P$($k$为常数)是一正则变换。若原来的Hamilton函数为$H = \dfrac12 (p^2+k^2q^2)$,求用新正则变量表示的正则方程。
\end{question}
\begin{solution}
正则变换保持基本Poisson括号不变,因此考虑
\begin{equation*}
	[q,p] = \frac{\pl q}{\pl Q} \frac{\pl p}{\pl P} - \frac{\pl q}{\pl P} \frac{\pl p}{\pl Q} = 1
\end{equation*}
故此变换为正则变换。在此变换下,新的Hamilton函数满足
\begin{equation*}
	K = \frac12 \left[\left(\sqrt{2kQ}\sin P\right)^2 + k^2 \left(\sqrt{2kQ}\sin P\right)^2\right] = kQ
\end{equation*}
由此,新的正则方程为
\begin{equation*}
\begin{cases}
	\dot{Q} = \dfrac{\pl K}{\pl P} = 0 \\[1.5ex]
	\dot{P} = -\dfrac{\pl K}{\pl Q} = k
\end{cases}
\end{equation*}
\end{solution}

\begin{question}[{\it Classical Mechanics}(Goldstein) 10.3]
用Hamilton-Jacobi方法求解竖直平面内的斜抛运动。假设物体在$t=0$时刻由原点以速度$v_0$抛出,初始速度与水平方向夹角为$\alpha$。
\end{question}
\begin{solution}
此问题的Hamilton函数为
\begin{equation*}
	H = \frac{p_x^2}{2m}+\frac{p_y^2}{2m}+mgy
\end{equation*}
进而可获得其Hamilton-Jacobi方程为
\begin{equation*}
	\frac{\pl S}{\pl t} + \frac{1}{2m}\left(\frac{\pl S}{\pl x}\right)^2 + \frac{1}{2m}\left(\frac{\pl S}{\pl y}\right)^2 + mgy = 0
\end{equation*}
由于Hamilton函数不显含时间$t$和坐标$x$,因此可令
\begin{equation*}
	S(x,y,t) = -Et+\gamma x+S_2(y)
\end{equation*}
其中$\gamma$和$E$为积分常数。由此可得方程
\begin{equation*}
	\frac{\gamma^2}{2m} + \frac{1}{2m}\left(\frac{\mathd S_2}{\mathd y}\right)^2 + mgy = E
\end{equation*}
由此可以解得
\begin{equation*}
	\frac{\mathd S_2}{\mathd y} = \pm \sqrt{2mE-\gamma^2-2m^2gy}
\end{equation*}
假设开方前取正号,积分可得
\begin{equation*}
	S_2(y) = \int \sqrt{2mE-\gamma^2-2m^2gy} \mathd y = -\frac{1}{3m^2g}\left[2mE-\gamma^2-2m^2gy\right]^{\frac32}
\end{equation*}
由此即有
\begin{equation*}
	S = -Et + \gamma x - \frac{1}{3m^2g}\left[2mE-\gamma^2-2m^2gy\right]^{\frac32}
\end{equation*}
其中$\gamma$和$E$为积分常数。于是,此问题的解由如下方程决定:
\begin{equation*}
\begin{cases}
	\ds \beta_1 = \frac{\pl S}{\pl E} = -\frac{1}{mg} \sqrt{2mE-\gamma^2-2m^2gy} - t \\
	\ds \beta_2 = \frac{\pl S}{\pl \gamma} = x + \frac{\gamma}{m^2g}\sqrt{2mE-\gamma^2-2m^2gy}
\end{cases}
\end{equation*}
从上式中反解出$x$和$y$可得
\begin{equation*}
\begin{cases}
	\ds x = \beta_2 +\frac{\gamma}{m}(t+\beta_1) \\
	\ds y = \frac{E}{mg}-\frac{\gamma^2}{2m^2g} - \frac12 g(t+\beta_1)^2
\end{cases}
\end{equation*}
利用初始条件确定解中的4个积分常数$E, \gamma, \beta_1$和$\beta_2$:
\begin{equation*}
\begin{cases}
	\ds x(0) = \beta_2+\frac{\gamma\beta_1}{m} = 0 \\
	\ds y(0) = \frac{E}{mg}-\frac{\gamma^2}{2m^2g} - \frac12 g\beta_1^2 = 0 \\
	\ds \dot{x}(0) = \frac{\gamma}{m} = v_0\cos\alpha \\
	\ds \dot{y}(0) = -g\beta_1 = v_0\sin\alpha
\end{cases}
\end{equation*}
由此解得
\begin{equation*}
\begin{cases}
	\ds \gamma = mv_0\cos\alpha \\
	\ds E = \frac12 mv_0^2 \\
	\ds \beta_1 = -\frac{v_0}{g}\sin\alpha \\
	\ds \beta_2 = \frac{v_0^2}{g}\cos\alpha\sin\alpha
\end{cases}
\end{equation*}
最终可得此问题的解为
\begin{equation*}
\begin{cases}
	\ds x(t) = v_0t\cos\alpha \\
	\ds y(t) = v_0t\sin\alpha - \frac12 gt^2
\end{cases}
\end{equation*}
\end{solution}

\subsection{两体问题}

\begin{question}[金尚年《理论力学》98页3.6]
\label{理论力学:题3.6}
求粒子在中心力
\begin{equation*}
	F = -\frac{k}{r^2}+\frac{c}{r^3}
\end{equation*}
的作用下的轨道方程。
\end{question}
\begin{solution}
根据Binet方程\eqref{Binet方程}可有
\begin{equation*}
	\frac{\mathrm{d}^2 u}{\mathrm{d} \theta^2} + u = -\frac{m}{l^2 u^2} F\left(\frac{1}{u}\right) = -\frac{m}{l^2}(-k+cu)
\end{equation*}
即有
\begin{equation*}
	\frac{\mathrm{d}^2 u}{\mathrm{d} \theta^2} + \left(1+\frac{cm}{l^2}\right) u = \frac{km}{l^2}
\end{equation*}
此方程的解为
\begin{equation*}
	u = \frac{km}{l^2+cm} + A\cos \left(\sqrt{1+\frac{cm}{l^2}} \theta + \theta_0\right)
\end{equation*}
其中$A$和$\theta_0$为积分常数。选择极角的起始点,总可以使得$\theta_0=0$,由此有轨道方程
\begin{equation*}
	r = \frac{1}{\dfrac{km}{l^2+cm} + A\cos \left(\sqrt{1+\dfrac{cm}{l^2}} \theta\right)}
\end{equation*}
考虑中心力
\begin{equation*}
	F = -\frac{k}{r^2}+\frac{c}{r^3}
\end{equation*}
对应的势能为
\begin{equation*}
	V = -\frac{k}{r} + \frac{c}{2r^2}
\end{equation*}
近日点的距离为$r_{\min} = \dfrac{1}{\dfrac{km}{l^2+cm} + A}$,而在近日点有$\dot{r}\big|_{r=r_{\min}} = 0$,故有
\begin{equation*}
	U(r_{\min}) = \frac{l^2}{2m r_{\min}^2} + V(r_{\min}) = E
\end{equation*}
即
\begin{equation*}
	\frac{l^2+cm}{2m} u_{\min}^2 - ku_{\min} - E = 0
\end{equation*}
由此可得
\begin{equation*}
	u_{\min} = \frac{km+\sqrt{k^2m^2-2mE(l^2+cm)}}{l^2+cm}
\end{equation*}
所以有
\begin{equation*}
	A = u_{\min} - \frac{km}{l^2+cm} = \sqrt{\left(\frac{km}{l^2+cm}\right)^2 - \frac{2mE}{l^2+2m}}
\end{equation*}
由此即得轨道方程
\begin{equation*}
	r = \frac{\dfrac{l^2+cm}{km}}{1 + \sqrt{1-\dfrac{2E(l^2+cm)}{km}}\cos \left(\sqrt{\dfrac{l^2+cm}{l^2}} \theta\right)}
\end{equation*}
\end{solution}

\begin{question}[金尚年《理论力学》98页3.7]
求粒子在题\ref{理论力学:题3.6}中的势场中能作圆周运动的轨道半径,并讨论在此轨道上粒子受到微小扰动后的运动情况。
\end{question}
\begin{solution}
考虑粒子径向运动的等效势能
\begin{equation*}
	U(r) = \frac{l^2}{2mr^2} + V(r) = \frac{l^2+cm}{2mr^2} - \frac{k}{r}
\end{equation*}
粒子做圆周运动需要其半径$r_m$满足
\begin{equation*}
	\left.\frac{\mathrm{d} U}{\mathrm{d} r}\right|_{r=r_m} = -\frac{l^2-cm}{mr_m^3} - \frac{k}{r_m^2} = 0
\end{equation*}
即圆轨道半径为
\begin{equation*}
	r_m = \frac{l^2+2m}{km}
\end{equation*}
再考虑
\begin{equation*}
	\left.\frac{\mathrm{d}^2 U}{\mathrm{d} r^2}\right|_{r=r_m} = \frac{3(l^2+cm)}{mr_m^4} - \frac{2k}{r_m^3} = \frac{l^2+cm}{mr_m^4} > 0
\end{equation*}
故该圆轨道是稳定的。
\end{solution}

\begin{question}[金尚年《理论力学》98页3.16]
初速度为$v_\infty$的粒子受到势场$V=-\dfrac{\alpha}{r^2}$的散射,问在什么条件下粒子会被力心所俘获,并求俘获的总散射截面。
\end{question}
\begin{solution}
粒子在此势场中的等效势能为
\begin{equation*}
	U(r) = \frac{l^2}{2mr^2} + V(r) = \left(\frac{l^2}{2m}-\alpha\right) \frac{1}{r^2}
\end{equation*}
粒子能被力心俘获的条件即为$U(r) \leqslant 0$,即
\begin{equation*}
	\frac{l^2}{2m}-\alpha \leqslant 0
\end{equation*}
由于$l = \mu b v_\infty$,所以粒子能被力心俘获的条件为
\begin{equation*}
	b \leqslant \sqrt{\frac{2\alpha}{mv_\infty^2}}
\end{equation*}
总散射截面为
\begin{equation*}
	\sigma_t = \pi b_{\max} = \frac{2\pi\alpha}{mv_\infty^2}
\end{equation*}
\end{solution}

\begin{question}[金尚年《理论力学》99页3.20]
证明在中心势场$V=\dfrac{\alpha}{r}$中运动的粒子,其Laplace-Runge-Lenz矢量$\mbf{A} = \mbf{v} \times \mbf{L} + \alpha \dfrac{\mbf{r}}{r}$是守恒量,并讨论$\mbf{A}$的物理意义。式中$\mbf{v}$是粒子的速度,$\mbf{L}$是粒子的角动量。
\end{question}
\begin{solution}
考虑
\begin{align*}
	\frac{\mathrm{d} \mbf{A}}{\mathrm{d} t} & = \frac{\mathrm{d} \mbf{v}}{\mathrm{d} t} \times \mbf{L} + \alpha \frac{\mathrm{d} \mbf{e}_r}{\mathrm{d} t} = \frac{1}{m} \mbf{F} \times \mbf{L} + \alpha \dot{\theta} \mbf{e}_\theta
\end{align*}
式中利用了角动量守恒,即$\dfrac{\mathrm{d} \mbf{L}}{\mathrm{d} t} = \mbf{0}$。又有
\begin{align*}
	\mbf{F} & = \frac{\alpha}{r^2} \mbf{e}_r \\
	\mbf{L} & = mr^2 \dot{\theta} \mbf{e}_z
\end{align*}
故
\begin{equation*}
	\frac{\mathrm{d} \mbf{A}}{\mathrm{d} t} = \alpha \dot{\theta} \mbf{e}_r \times \mbf{e}_k + \alpha \dot{\theta} \mbf{e}_\theta = \mbf{0}
\end{equation*}
因此,Laplace-Runge-Lenz矢量$\mbf{A}$是守恒量。

当质点运动到极轴上时,设质点的速度$\mbf{v} = \mbf{v}_0 = v_0 \mbf{e}_2$,矢径$\mbf{r} = r_0 \mbf{e}_1$,故角动量矢量可以表示为$\mbf{L} = \mbf{r} \times \mbf{p} = mr_0 v_0 \mbf{e}_3$,此时Laplace-Runge-Lenz矢量为
\begin{equation*}
	\mbf{A} = \mbf{v} \times \mbf{L} + \alpha \mbf{e}_r = (mr_0 v_0^2 +\alpha)\mbf{e}_1
\end{equation*}
故Laplace-Runge-Lenz矢量沿极轴方向。再考虑
\begin{align*}
	A^2 & = \mbf{A} \cdot \mbf{A} = \left(\mbf{v} \times \mbf{L} + \alpha \mbf{e}_r\right) \cdot \left(\mbf{v} \times \mbf{L} + \alpha \mbf{e}_r\right) = v^2 L^2 + 2\alpha \mbf{e}_r \cdot (\mbf{v} \times \mbf{L}) + \alpha^2 \\
	& = v^2 L^2 + \frac{2\alpha}{mr} \mbf{L} \cdot (\mbf{r} \times \mbf{p}) + \alpha^2 = v^2 L^2 + \frac{2\alpha}{mr} L^2 + \alpha^2 = \frac{2}{m}\left(\frac12 mv^2 + \frac{\alpha}{r}\right) L^2 + \alpha^2 \\
	& = \frac{2EL^2}{m} + \alpha^2 = \alpha^2 e^2
\end{align*}
即Laplace-Runge-Lenz矢量的大小与离心率成正比,可定义{\heiti 离心率矢量}为
\begin{equation*}
	\mbf{e} = \frac{\mbf{A}}{\alpha} = \frac{1}{\alpha} \mbf{v} \times \mbf{L} + \frac{\mbf{r}}{r}
\end{equation*}
离心率矢量也是守恒量,其方向沿极轴,大小为轨道的离心率。
\end{solution}

\subsection{多自由度微振动与阻尼}

\begin{question}[金尚年《理论力学》207页6.1]
质量为$m$的质点,用一固有长度为$l$,劲度系数为$k$,质量可以忽略的弹簧悬挂于$O$点,质点约束在铅直平面内运动(如图\ref{理论力学:207页6.1}所示),求体系作微振动时的振动频率。
\begin{figure}[htb]
\centering
\begin{asy}
	size(200);
	//理论力学:207页6.1
	guide spring(real d,real h,int nmax){
		guide gspring;
		path springunit;
		gspring = (0,0)--(d,h/2);
		springunit = (d,h/2)--(-d,h/2+h)--(d,h/2+2*h);
		for(int i=0;i<nmax-1;i=i+1){
			gspring = gspring--shift(2*i*h*dir(90))*springunit;
		}
		gspring = gspring--shift(2*(nmax-1)*h*dir(90))*((-d,h/2+h)--(0,2*h));
		return gspring;
	}
	picture dashpic;
	pair O,P,dash;
	real theta,d,l1,ls,l,r,h;
	path tre;
	O = (0,0);
	theta = 30;
	d = 0.05;
	l1 = 0.6;
	ls = 1.2;
	l = 2.4;
	r = 0.03;
	draw(O--l*dir(-90),dashed);
	draw(O--l1*dir(-90+theta)--shift(l1*dir(-90+theta))*rotate(-180+theta)*spring(d,ls/10/2,10)--l*dir(-90+theta),linewidth(1bp));
	fill(shift(l*dir(-90+theta))*scale(r)*unitcircle,black);
	label("$m$",l*dir(-90+theta),2*S);
	label("$k$",(l1+ls/2)*dir(-90+theta),2*dir(theta));
	r = 0.25;
	draw(Label("$\theta$",MidPoint,Relative(E)),arc(O,r,-90,-90+theta));
	tre = l*dir(180)--l*dir(0);
	dash = l*dir(45);
	draw(dashpic,tre,linewidth(1bp));
	for(real rp=0;rp<=1;rp=rp+0.015){
		P = relpoint(tre,rp);
		draw(dashpic,P--P+dash);
	}
	h = 0.06;
	clip(dashpic,box((-l1,-1),(l1,h)));
	add(dashpic);
\end{asy}
\caption{题\thequestion}
\label{理论力学:207页6.1}
\end{figure}
\end{question}
\begin{solution}
此体系的自由度为$2$,取摆长$x$和摆动角$\theta$作为广义坐标,则体系的Lagrange函数可以表示为
\begin{equation*}
	L = \frac12 m \left(\dot{x}^2 + x^2 \dot{\theta}^2\right) + mgx\cos \theta - \frac12 k (x-l)^2
\end{equation*}
体系的平衡位置可以通过
\begin{equation*}
\begin{cases}
	\left.\dfrac{\pl V}{\pl x}\right|_{x=x_0} = 0 \\[1.5ex]
	\left.\dfrac{\pl V}{\pl \theta}\right|_{\theta=\theta_0} = 0
\end{cases}
\end{equation*}
来确定,即有
\begin{equation*}
	x_0 = l + \frac{mg}{k},\quad \theta_0 = 0
\end{equation*}
在平衡位置附近将Lagrange函数展开至两阶并记$q=x-x_0$,可有
\begin{align*}
	L & = \frac12 m \left(\dot{q}^2 + (q+x_0)^2 \dot{\theta}^2\right) + mg(q+x_0) \cos \theta - \frac12 k(q+x_0-l)^2 \\
	& = \frac12 m \dot{q}^2 + \frac12 m x_0^2 \dot{\theta}^2 - \frac12 kq^2 - \frac12 mgx_0 \theta^2 + mgx_0 - \frac{m^2 g^2}{2k}
\end{align*}
由此,体系微振动的运动方程为
\begin{equation*}
\begin{cases}
	m\ddot{q}+kq = 0 \\
	mx_0^2 \ddot{\theta} + mgx_0 \theta = 0
\end{cases}
\end{equation*}
因此,体系作微振动的频率为
\begin{equation*}
	\omega_1 = \sqrt{\frac{k}{m}},\quad \omega_2 = \sqrt{\frac{g}{x_0}} = \sqrt{\frac{g}{l + \dfrac{mg}{k}}}
\end{equation*}
\end{solution}

\begin{question}[金尚年《理论力学》208页6.6]
有一弹簧连着一个质量为$m_1$的滑块,滑块可沿光滑水平直线自由滑动,在滑块上系有一摆长为$l$,质量为$m_2$的单摆,求体系的振动频率。
\begin{figure}[htb]
\centering
\begin{minipage}[t]{0.6\textwidth}
\centering
\begin{asy}
	size(200);
	//理论力学:208页6.6
	guide spring(real d,real h,int nmax){
		guide gspring;
		path springunit;
		gspring = (0,0)--(d,h/2);
		springunit = (d,h/2)--(-d,h/2+h)--(d,h/2+2*h);
		for(int i=0;i<nmax-1;i=i+1){
			gspring = gspring--shift(2*i*h*dir(90))*springunit;
		}
		gspring = gspring--shift(2*(nmax-1)*h*dir(90))*((-d,h/2+h)--(0,2*h));
		return gspring;
	}
	picture dashpic;
	pair O,A,P,dash;
	real theta,d,l1,ls,l,r,h,d1,d2,D1,D2;
	path tre;
	O = (0,0);
	theta = 30;
	d = 0.1;
	l1 = 0.3;
	ls = 1.8;
	l = 2.2;
	d1 = 0.6;
	d2 = 0.6;
	D1 = 5;
	D2 = 2;
	A = d2*dir(90)+(2*l1+ls+d1)*dir(0);
	draw(O--D1*dir(0),linewidth(1bp));
	draw(O--D2*dir(90),linewidth(1bp));
	draw((0,d2)--shift(l1*dir(0)+(0,d2))*rotate(-90)*spring(d,ls/16,8)--(l1*2+ls)*dir(0)+(0,d2),linewidth(0.8bp));
	draw(shift(A)*box((-d1,-d2),(d1,d2)),linewidth(0.8bp));
	dot(A);
	draw(A--A+l*dir(-90),dashed);
	draw(Label("$l$",MidPoint,Relative(E)),A--A+l*dir(-90+theta),linewidth(0.8bp));
	r = 0.08;
	fill(shift(A+l*dir(-90+theta))*scale(r)*unitcircle,black);
	r = 0.3;
	draw(Label("$\theta$",MidPoint,Relative(E)),arc(A,r,-90,-90+theta));
	label("$k$",(l1+ls/2)*dir(0)+d2*dir(90),2*N);
	label("$m_1$",A+d2*dir(90),N);
	label("$m_2$",A+l*dir(-90+theta),2*E);
	tre = D1*dir(-90)--D1*dir(90);
	dash = D1*dir(-135);
	draw(dashpic,tre,linewidth(1bp));
	for(real rp=0;rp<=1;rp=rp+0.012){
		P = relpoint(tre,rp);
		draw(dashpic,P--P+dash);
	}
	h = 0.15;
	clip(dashpic,box((-h,0),(1,D2)));
	add(dashpic);
\end{asy}
\caption{题\thequestion}
\label{理论力学:208页6.6}
\end{minipage}
\hspace{0.5cm}
\begin{minipage}[t]{0.35\textwidth}
\centering
\begin{asy}
	size(120);
	//m2的速度合成图
	pair O;
	real theta,v1,v2;
	theta = 30;
	O = (0,0);
	v1 = 1.5;
	v2 = 1;
	draw(Label("$l\dot{\theta}$",MidPoint,Relative(E)),v2*dir(theta)--O,Arrow);
	draw(Label("$\dot{x}$",MidPoint,Relative(E)),O--v1*dir(0),Arrow);
	draw(Label("$\boldsymbol{v}_2$",MidPoint,Relative(W)),v2*dir(theta)--v1*dir(0),Arrow);
	draw(Label("$\theta$",MidPoint,Relative(E)),arc(O,0.2,0,theta));
\end{asy}
\caption{$m_2$的速度合成图}
\label{m2的速度合成图}
\end{minipage}
\end{figure}
\end{question}
\begin{solution}
取弹簧伸长量$x$和单摆的摆动角$\theta$为广义坐标,则根据相对运动的速度关系,可得$m_2$的速度如图\ref{m2的速度合成图}所示,因此有
\begin{equation*}
	v_2^2 = \dot{x}^2+l^2 \dot{\theta}^2 - 2l\dot{x} \dot{\theta} \cos \theta
\end{equation*}
因此,此体系的Lagrange函数为
\begin{equation*}
	L = \frac12 m_1 \dot{x}^2 + \frac12 m_2 \left(\dot{x}^2+l^2 \dot{\theta}^2 - 2l\dot{x} \dot{\theta} \cos \theta\right) - \frac12 kx^2 + m_2 gl \cos \theta
\end{equation*}
近似到二阶,可有
\begin{align*}
	L & = \frac12 m_1 \dot{x}^2 + \frac12 m_2 \left(\dot{x}^2+l^2 \dot{\theta}^2 - 2l\dot{x} \dot{\theta}\right) - \frac12 kx^2 + m_2 gl \left(1-\frac12 \theta^2\right) \\
	& = \frac12 (m_1+m_2) \dot{x}^2 - m_2l\dot{x} \dot{\theta} + \frac12 m_2l^2 \dot{\theta}^2 - \frac12 kx^2 - \frac12 m_2 gl \theta^2 + m_2 gl
\end{align*}
去掉Lagrange函数中对运动方程无任何影响的常数项,记$\mbf{q} = \begin{pmatrix} x \\ \theta \end{pmatrix}$,可将Lagrange函数表示为
\begin{equation*}
	L = \frac12 \dot{\mbf{q}}^{\mathrm{T}} \mbf{M} \dot{\mbf{q}} - \frac12 \mbf{q}^{\mathrm{T}} \mbf{K} \mbf{q}
\end{equation*}
其中
\begin{equation*}
	\mbf{M} = \begin{pmatrix} m_1+m_2 & -m_2l \\ -m_2l & m_2l^2 \end{pmatrix},\quad \mbf{K} = \begin{pmatrix} k & 0 \\ 0 & m_2gl \end{pmatrix}
\end{equation*}
体系的运动方程为
\begin{equation*}
	\mbf{M} \ddot{\mbf{q}} + \mbf{K} \mbf{q} = \mbf{0}
\end{equation*}
作试解$\mbf{q} = \mbf{a} \cos (\omega t + \phi)$代入方程中,可得
\begin{equation*}
	(\mbf{K} - \omega^2 \mbf{M}) \mbf{a} = \mbf{0}
\end{equation*}
此方程组有非平凡解的条件为
\begin{equation*}
	|\mbf{K} - \omega^2 \mbf{M}| = \begin{vmatrix} k-\omega^2(m_1+m_2) & \omega^2 m_2l \\ \omega^2 m_2l & m_2 gl - \omega^2 m_2l^2 \end{vmatrix} = 0
\end{equation*}
即体系的频率为方程
\begin{equation*}
	\omega^4-\left(\frac{k}{m_1} + \frac{m_1+m_2}{m_1} \frac{g}{l}\right) \omega^2 + \frac{kg}{m_1l} = 0
\end{equation*}
的两个正实根。
\end{solution}

\begin{question}[金尚年《理论力学》209页6.14]
设一体系的Lagrange函数为
\begin{equation*}
	L = \frac12 (\dot{x}^2+\dot{y}^2) - \frac12 \omega_0^2 (x^2+y^2) + \alpha xy
\end{equation*}
它表示两个固有频率为$\omega_0$的一维振子以相互作用$\alpha xy$耦合起来,求此体系的振动频率。
\end{question}
\begin{solution}
记$\mbf{q} = \begin{pmatrix} x \\ y \end{pmatrix}$,则体系的Lagrange函数可以表示为
\begin{equation*}
	L = \frac12 \dot{\mbf{q}}^{\mathrm{T}} \mbf{M} \dot{\mbf{\mbf{q}}} - \frac12 \mbf{q}^{\mathrm{T}} \mbf{K} \mbf{q}
\end{equation*}
其中
\begin{equation*}
	\mbf{M} = \begin{pmatrix} 1 & 0 \\ 0 & 1 \end{pmatrix},\quad \mbf{K} = \begin{pmatrix} \omega_0^2 & -\alpha \\ -\alpha & \omega_0^2 \end{pmatrix}
\end{equation*}
体系的频率满足久期方程
\begin{equation*}
	|\mbf{K}-\omega^2 \mbf{M}| = \begin{vmatrix} \omega_0^2-\omega^2 & -\alpha \\ -\alpha & \omega_0^2-\omega^2 \end{vmatrix} = 0
\end{equation*}
由此可解得体系的频率为
\begin{equation*}
	\omega_1 = \omega_0 \sqrt{1-\frac{\alpha}{\omega_0^2}},\quad \omega_2 = \omega_0 \sqrt{1+\frac{\alpha}{\omega_0^2}}
\end{equation*}
\end{solution}

\begin{question}[金尚年《理论力学》209页6.18]
设具有三个广义坐标$q_1,q_2,q_3$的力学体系产生微振动,其动能$T$和势能$V$可写作
\begin{align*}
	T & = \frac12 (\dot{q}_1^2 + \dot{q}_2^2 + \dot{q}_3^2) \\
	V & = -q_1q_2 - q_2q_3 + 2q_1^2 + \frac52 q_2^2 + 2q_3^2
\end{align*}
已知$t=0$时,$q_1 = q_{10},q_2 = q_{20},q_3 = q_{30},\dot{q}_1 = \dot{q}_2 = \dot{q}_3 = 0$,求此体系的运动。
\end{question}
\begin{solution}
记$\mbf{q}=\begin{pmatrix} q_1 \\ q_2 \\ q_3 \end{pmatrix}$,则体系的Lagrange函数为
\begin{align*}
	L & = \frac12 (\dot{q}_1^2 + \dot{q}_2^2 + \dot{q}_3^2) - q_1q_2 + q_2q_3 - 2q_1^2 - \frac52 q_2^2 - 2q_3^2 \\
	& = \frac12 \dot{\mbf{q}}^{\mathrm{T}} \mbf{M} \dot{\mbf{q}} - \frac12 \mbf{q}^{\mathrm{T}} \mbf{K} \mbf{q}
\end{align*}
其中
\begin{equation*}
	\mbf{M} = \begin{pmatrix} 1 & 0 & 0 \\ 0 & 1 & 0 \\ 0 & 0 & 1 \end{pmatrix},\quad \mbf{K} = \begin{pmatrix} 4 & -1 & 0 \\ -1 & 5 & -1 \\ 0 & -1 & 4 \end{pmatrix}
\end{equation*}
于是,体系的运动方程可以表示为
\begin{equation*}
	\mbf{M} \ddot{\mbf{q}} + \mbf{K} \mbf{q} = \mbf{0}
\end{equation*}
作试解$\mbf{q} = \mbf{a} \cos (\omega t+\phi)$,则方程可化为
\begin{equation*}
	(\mbf{K}-\omega^2 \mbf{M}) \mbf{a} = \mbf{0}
\end{equation*}
此方程具有非平凡解的条件为
\begin{equation*}
	|\mbf{K}-\omega^2 \mbf{M}| = \begin{vmatrix} 4-\omega^2 & -1 & 0 \\ -1 & 5-\omega^2 & -1 \\ 0 & -1 & 4-\omega^2 \end{vmatrix} = 0
\end{equation*}
由此解得
\begin{equation*}
	\omega_1 = \sqrt{3},\quad \omega_2 = 2,\quad \omega_3 = \sqrt{6}
\end{equation*}
其对应的振幅$\mbf{a}$分别为
\begin{equation*}
	\mbf{a}_1 = \begin{pmatrix} 1 \\ 1 \\ 1 \end{pmatrix},\quad \mbf{a}_2 = \begin{pmatrix} 1 \\ 0 \\ -1 \end{pmatrix},\quad \mbf{a}_3 = \begin{pmatrix} 1 \\ -2 \\ 1 \end{pmatrix}
\end{equation*}
因此,体系运动的通解为
\begin{align*}
	\mbf{q} & = C_1 \mbf{a}_1 \cos(\omega_1 t + \phi_1) + C_2 \mbf{a}_2 \cos(\omega_2 t + \phi_2) + C_3 \mbf{a}_3 \cos(\omega_3 t + \phi_3)
\end{align*}
速度为
\begin{align*}
	\dot{\mbf{q}} & = -\omega_1 C_1 \mbf{a}_1 \sin(\omega_1 t + \phi_1) - \omega_2 C_2 \mbf{a}_2 \sin(\omega_2 t + \phi_2) - \omega_3 C_3 \mbf{a}_3 \sin(\omega_3 t + \phi_3)
\end{align*}
由于$t=0$时,$\dot{\mbf{q}} = \mbf{0}$,即
\begin{equation*}
	-\omega_1 C_1 \mbf{a}_1 \sin \phi_1 - \omega_2 C_2 \mbf{a}_2 \sin \phi_2 - \omega_3 C_3 \mbf{a}_3 \sin \phi_3 = \mbf{0}
\end{equation*}
可得$\phi_1 = \phi_2 = \phi_3 = 0$。再根据$t=0$时,有$\mbf{q}= \mbf{q}_0$,可有
\begin{equation*}
	C_1 \mbf{a}_1 + C_2 \mbf{a}_2 + C_3 \mbf{a}_3 = \mbf{q}_0
\end{equation*}
解得
\begin{align*}
	\begin{pmatrix} C_1 \\ C_2 \\ C_3 \end{pmatrix} & = \begin{pmatrix} \mbf{a}_1 & \mbf{a}_2 & \mbf{a}_3 \end{pmatrix}^{-1} \mbf{q}_0 = \begin{pmatrix} 1 & 1 & 1 \\ 1 & 0 & -2 \\ 1 & -1 & 1 \end{pmatrix}^{-1} \begin{pmatrix} q_{10} \\ q_{20} \\ q_{30} \end{pmatrix} \\
	& = \begin{pmatrix} \dfrac13 q_{10} + \dfrac13 q_{20} + \dfrac13 q_{30} \\[1.5ex] \dfrac12 q_{10} - \dfrac12 q_{30} \\[1.5ex] \dfrac16 q_{10} - \dfrac13 q_{20} + \dfrac16 q_{30} \end{pmatrix}
\end{align*}
据此可得运动解为
\begin{equation*}
\begin{cases}
	\displaystyle q_1 = \dfrac13 \left(q_{10} + q_{20} + q_{30}\right) \cos \omega_1 t + \dfrac12\left(q_{10} - q_{30}\right) \cos \omega_2 t + \dfrac16 \left(q_{10} - 2q_{20} + q_{30}\right) \cos \omega_3 t \\[1.5ex]
	\displaystyle q_2 = \dfrac13 \left(q_{10} + q_{20} + q_{30}\right) \cos \omega_1 t + \dfrac13 \left(q_{10} - 2q_{20} + q_{30}\right) \cos \omega_3 t \\[1.5ex]
	\displaystyle q_3 = \dfrac13 \left(q_{10} + q_{20} + q_{30}\right) \cos \omega_1 t - \dfrac12 \left(q_{10} - q_{30}\right) \cos \omega_2 t + \dfrac16 \left(q_{10} - 2q_{20} + q_{30}\right) \cos \omega_3 t
\end{cases}
\end{equation*}
\end{solution}

\subsection{刚体}

\begin{question}[金尚年《理论力学》146页4.7]
高为$h$,顶角为$2\alpha$的圆锥在一平面上滚而不滑,如此锥体以等角速度$\omega$绕$OZ$轴转动,求此圆锥地面上最高点$A$的速度和加速度。
\begin{figure}[htb]
\centering
\begin{asy}
	size(300);
	//理论力学:146页4.7
	pair O,i,j,k,O1;
	real h,r,x,y,z,r0;
	O = (0,0);
	i = (-sqrt(2)/4,-sqrt(14)/12);
	j = (sqrt(14)/4,-sqrt(2)/12);
	k = (0,2*sqrt(2)/3);
	x = 1;
	y = 1.5;
	z = 1;
	h = 1.2;
	r = 0.3;
	draw(Label("$X$",EndPoint),O--x*i,Arrow);
	draw(Label("$Y$",EndPoint),O--y*j,Arrow);
	draw(Label("$Z$",EndPoint),O--z*k,Arrow);
	label("$O$",O,NW);
	pair cir(real theta){
		real xi,eta,x,y,z;
		xi = r*cos(theta);
		eta = r*sin(theta);
		x = xi;
		y = sqrt(r**2+h**2) - (r+eta)*r/sqrt(r**2+h**2);
		z = (r+eta)*h/sqrt(r**2+h**2);
		return x*i+y*j+z*k;
	}
	pair ang(real theta){
		real x,y;
		x = r0*cos(theta);
		y = r0*sin(theta);
		return x*i+y*j;
	}
	pair tritopair(triple P){
		return P.x*i+P.y*j+P.z*k;
	}
	path p;
	p = graph(cir,0,2*pi,200);
	draw(p,linewidth(1bp));
	//求路径p的最高点和最低点
	pair toppoint(path p,real fuzz=0.01){
		pair P,Q;
		P = relpoint(p,0);
		for(real rp=0;rp<=1;rp=rp+fuzz){
			Q = relpoint(p,rp);
			if(Q.y>P.y) P=Q;
		}
		return P;
	}
	pair botpoint(path p,real fuzz=0.01){
		pair P,Q;
		P = relpoint(p,0);
		for(real rp=0;rp<=1;rp=rp+fuzz){
			Q = relpoint(p,rp);
			if(Q.y<P.y) P=Q;
		}
		return P;
	}
	label("$A$",toppoint(p,0.001),N);
	draw(O--toppoint(p,0.001),linewidth(1bp));
	draw(O--botpoint(p,0.001),linewidth(1bp));
	O1 = tritopair((0,h*h/sqrt(h**2+r**2),h*r/sqrt(h**2+r**2)));
	draw(Label("$h$",MidPoint,Relative(W)),O--O1,dashed);
	draw(Label("$z$",EndPoint),O1--interp(O,O1,1.3),Arrow);
	draw(Label("$\alpha$",MidPoint,Relative(E)),arc(O,0.15,degrees(O1),degrees(toppoint(p,0.001))));
	draw(Label("$\theta$",MidPoint,Relative(W)),arc(O,0.3,90,degrees(O1)),Arrow);
	r0 = 0.2;
	draw(Label("$\phi$",MidPoint,Relative(E)),graph(ang,0,pi/2),Arrow);
	\end{asy}
\caption{题\thequestion}
\label{理论力学:146页4.7}
\end{figure}
\end{question}
\begin{solution}
首先有
\begin{equation*}
	\dot{\phi} = \omega,\quad \dot{\theta} = 0
\end{equation*}
根据纯滚动条件,可有
\begin{equation*}
	\phi \sqrt{h^2+r^2} = \psi r
\end{equation*}
即有
\begin{equation*}
	\dot{\psi} = \frac{\sqrt{h^2+r^2}}{r} \dot{\phi} = \frac{\sqrt{h^2+r^2}}{r} \omega = \frac{\omega}{\sin \alpha}
\end{equation*}
由此可得圆锥滚动的角速度
\begin{equation*}
	\mbf{\omega} = \dot{\phi} \mbf{e}_Z - \dot{\psi} \mbf{e}_z
\end{equation*}
根据几何关系,可得
\begin{equation*}
	\mbf{e}_z = \cos \phi \mbf{e}_X + \sin \phi \mbf{e}_Y + \cos \theta \mbf{e}_Z = \cos \phi \mbf{e}_X + \sin \phi \mbf{e}_Y + \sin \alpha \mbf{e}_Z
\end{equation*}
所以,可有
\begin{equation*}
	\mbf{\omega} = \omega \mbf{e}_Z - \frac{\omega}{\sin \alpha} \left(\cos \phi \mbf{e}_X + \sin \phi \mbf{e}_Y + \sin \alpha \mbf{e}_Z\right) = -\frac{\omega \cos \phi}{\sin \alpha} \mbf{e}_X - \frac{\omega \sin \phi}{\sin \alpha} \mbf{e}_Y
\end{equation*}
而
\begin{equation*}
	\mbf{r}_A = \frac{h}{\cos \alpha} \cos 2\alpha \left(\cos \phi \mbf{e}_X + \sin \phi \mbf{e}_Y\right) + \frac{h}{\cos \alpha} \sin 2\alpha \mbf{e}_Z
\end{equation*}
因此,$A$点的速度为
\begin{align*}
	\mbf{v}_A & = \mbf{\omega} \times \mbf{r}_A \\
	& = \left(-\frac{\omega \cos \phi}{\sin \alpha} \mbf{e}_X - \frac{\omega \sin \phi}{\sin \alpha} \mbf{e}_Y\right) \times \left(\frac{h}{\cos \alpha} \cos 2\alpha \left(\cos \phi \mbf{e}_X + \sin \phi \mbf{e}_Y\right) + \frac{h}{\cos \alpha} \sin 2\alpha \mbf{e}_Z\right) \\
	& = -\frac{2\omega h}{\cos \alpha} \left(\sin \phi \mbf{e}_X - \cos \phi \mbf{e}_Y\right)
\end{align*}
$A$点的加速度为
\begin{equation*}
	\mbf{a}_A = \dot{\mbf{v}}_A = -\frac{2\omega h}{\cos \alpha} \left(\dot{\phi} \cos \phi \mbf{e}_X + \dot{\phi} \sin \phi \mbf{e}_Y\right) = -\frac{2\omega^2 h}{\cos \alpha} \left(\cos \phi \mbf{e}_X + \sin \phi \mbf{e}_Y\right)
\end{equation*}
\end{solution}

\begin{question}[金尚年《理论力学》148页4.27]
均匀椭球绕自己的一个对称轴$AB$以角速度$\dot{\phi}$转动,并且$AB$轴又以角速度$\dot{\theta}$绕通过椭球中心并与$AB$垂直的$CD$轴转动,写出椭球的动能和角动量。
\begin{figure}[htb]
\centering
\begin{asy}
	size(300);
	//理论力学:148页4.27
	pair O,i,j,k;
	O = (0,0);
	i = (-sqrt(2)/4,-sqrt(14)/12);
	j = (sqrt(14)/4,-sqrt(2)/12);
	k = (0,2*sqrt(2)/3);
	real a,b,l,h,h1,h2,r;
	a = 1;
	b = 2;
	l = 3;
	h = 2;
	h1 = 2;
	h2 = 3;
	pair centercir(real theta){
		return a*cos(theta)*i+a*sin(theta)*k;
	}
	pair planecir(real theta){
		return a*cos(theta)*i+b*sin(theta)*j;
	}
	pair vang(real theta){
		return r*cos(theta)*i+h*j+r*sin(theta)*k;
	}
	pair hang(real theta){
		return r*cos(theta)*i+r*sin(theta)*j+h*k;
	}
	draw(xscale(b*0.975)*yscale(a)*unitcircle,linewidth(1bp));
	draw(graph(centercir,0,2*pi),dashed);
	draw(graph(planecir,0,2*pi),dashed);
	draw(Label("$D$",EndPoint,Relative(W)),-h*k--h1*k,dashed);
	draw(-h*k--(-h2*k));
	label("$C$",-h*k,SW);
	draw(-b*j--b*j,dashed);
	draw(b*j--l*j--l*j-h*k--(-l*j-h*k)--(-l*j)--(-b*j/.98),linewidth(1bp));
	label("$A$",-l*j,N);
	label("$B$",l*j,N);
	h = -2.5;
	r = 0.4;
	draw(Label("$\dot{\theta}$",EndPoint,Relative(E)),graph(hang,-0.5,1.1),Arrow);
	h = 2.7;
	draw(Label("$\dot{\phi}$",Relative(0.2),Relative(W)),graph(vang,-0.7,1.2),Arrow);
\end{asy}
\caption{题\thequestion}
\label{理论力学:148页4.27}
\end{figure}
\end{question}
\begin{solution}
建立主轴系,以$BA$的方向为$z$轴,$x$轴和$y$轴位于与$z$轴垂直的平面内。首先求出此主轴系中椭球的惯量矩阵,即
\begin{align*}
	I_1 & = \int_V \rho (y^2+z^2) \mathrm{d} V \\
	& = \rho \int_0^\pi \mathrm{d} \theta \int_0^{2\pi} \mathrm{d} \phi \int_0^1 (b^2r^2\sin^2 \phi \sin^2 \theta + c^2r^2 \cos^2 \theta) abc r^2 \sin \theta \mathrm{d} r \\
	& = \frac{4\pi}{3} abc \rho \frac{b^2+c^2}{5} = \frac{m}{5} (b^2+c^2) \\
	I_2 & = \int_V \rho (x^2+z^2) \mathrm{d} V \\
	& = \rho \int_0^\pi \mathrm{d} \theta \int_0^{2\pi} \mathrm{d} \phi \int_0^1 (a^2r^2\cos^2 \phi \sin^2 \theta + c^2r^2 \cos^2 \theta) abc r^2 \sin \theta \mathrm{d} r \\
	& = \frac{4\pi}{3} abc \rho \frac{a^2+c^2}{5} = \frac{m}{5} (a^2+c^2) \\
	I_3 & = \int_V \rho (x^2+y^2) \mathrm{d} V \\
	& = \rho \int_0^\pi \mathrm{d} \theta \int_0^{2\pi} \mathrm{d} \phi \int_0^1 (a^2r^2\cos^2 \phi \sin^2 \theta + b^2r^2 \sin^2 \phi \sin^2 \theta) abc r^2 \sin \theta \mathrm{d} r \\
	& = \frac{4\pi}{3} abc \rho \frac{a^2+b^2}{5} = \frac{m}{5} (a^2+b^2)
\end{align*}
在主轴系中,角速度为
\begin{equation*}
	\mbf{\omega} = \dot{\theta} \cos \phi \mbf{e}_x + \dot{\theta} \sin \phi \mbf{e}_y + \dot{\phi} \mbf{e}_z 
\end{equation*}
由此,动能和角速度分别为
\begin{align*}
	T & = \frac12 \mbf{\omega}^{\mathrm{T}} \mbf{L} \mbf{\omega} = \frac{m}{10} \left[(b^2+c^2) \dot{\theta}^2 \cos^2 \phi + (a^2+c^2) \dot{\theta}^2 \sin^2 \phi + (a^2+b^2) \dot{\phi}^2\right] \\
	\mbf{L} & = \mbf{L} \mbf{\omega} = \frac{m}{5} \left[(b^2+c^2) \dot{\theta} \cos \phi \mbf{e}_x + (a^2+c^2) \dot{\theta} \sin \phi \mbf{e}_y + (a^2+b^2) \dot{\phi} \mbf{e}_z \right]
\end{align*}
\end{solution}

\iffalse
\begin{question}[金尚年《理论力学》148页4.29]
对称Lagrange陀螺的轴位于铅直位置,陀螺以很大的角速度$\omega$作稳定的自转。今突然在离定点$d$处作用一与陀螺的垂直轴垂直的冲量$p$。证明在陀螺此后的运动中,最大章动角近似为$2\arctan \dfrac{pd}{I_3 \omega}$。
\end{question}
\begin{solution}

\end{solution}

作业:p146-149,4.29(Lagrange陀螺),4.31

\subsection{非惯性系}\fi


%附录——理论力学习题集
\section{理论力学习题集(部分)}

\subsection{平面力系}

\begin{question}[37页4.56]
如图\ref{37页4.56}所示,在两光滑斜面$OA$和$OB$之间放置两个相互接触的光滑匀质圆柱。中心为$C_1$的圆柱重$P_1 = \SI{10}{\newton}$,中心为$C_2$的圆柱中$P_2 = \SI{30}{\newton}$。设$\angle AOx_1 = \ang{60}$,$\angle BOx = \ang{30}$,求直线$C_1C_2$与水平线$xOx_1$的夹角$\phi$、两圆柱对鞋面的压力$N_1$和$N_2$,以及两圆柱之间的压力$N$。

\begin{figure}[htb]
\centering
\begin{asy}
	size(200);
	//37页4.56
	pair O,A,B,C1,C2;
	real r1,r2,l1,l,alpha,beta,delta;
	O = (0,0);
	l = 2;
	A = l*dir(30+90);
	B = l*dir(30);
	draw(Label("$A$",EndPoint),O--A,linewidth(0.8bp));
	draw(Label("$B$",EndPoint),O--B,linewidth(0.8bp));
	draw(Label("$x$",EndPoint),O--(B.x,0),dashed);
	draw(Label("$x_1$",EndPoint),O--(A.x,0),dashed);
	r1 = 0.4;
	l1 = 1.2;
	alpha = aTan(r1/l1);
	C1 = l1*dir(A)+rotate(-90)*r1*dir(A);
	draw(shift(C1)*scale(r1)*unitcircle);
	beta = 45-alpha;
	r2 = l1*Tan(beta);
	C2 = l1*dir(B)+rotate(90)*r2*dir(B);
	draw(shift(C2)*scale(r2)*unitcircle);
	pair CC1,CC2,CCt;
	CC1 = interp(C1,C2,-0.2);
	CC2 = interp(C1,C2,1.8);
	CCt = CC2+(2*r1+r2)*dir(180);
	draw(CC1--CC2--CCt,dashed);
	delta = 10;
	draw(Label("$\phi$",MidPoint,Relative(E)),arc(CC2,r1+r2,degrees(CC1-CC2)-delta,degrees(CC1-CC2)),Arrow);
	draw(arc(CC2,r1+r2,180+delta,180),Arrow);
	pair P,dash,pace;
	int imax;
	imax = 30;
	dash = 0.1*dir(-110);
	pace = (B-O)/imax;
	P = O;
	for(int i=1;i<=imax+1;i=i+1){
		draw(P--P+dash);
		P = P+pace;
	}
	pace = (A-O)/imax;
	P = O;
	for(int i=2;i<=imax+1;i=i+1){
		draw(P--P+dash);
		P = P+pace;
	}
	dot(C1,UnFill);
	label("$C_1$",C1,N);
	dot(C2,UnFill);
	label("$C_2$",C2,N);
	draw(Label("$\ang{30}$",MidPoint,Relative(E)),arc(O,0.5,0,degrees(B)),Arrows);
	draw(Label("$\ang{60}$",MidPoint,Relative(W)),arc(O,0.5,180,degrees(A)),Arrows);
\end{asy}
\caption{题\thequestion}
\label{37页4.56}
\end{figure}
\end{question}
\begin{solution}

\begin{figure}[htb]
\centering
\begin{minipage}[t]{0.45\textwidth}
\begin{asy}
	size(180);
	//37页4.56受力分析1
	pair O;
	real P,r;
	O = (0,0);
	P = 2;
	draw(Label("$\boldsymbol{N}_1$",EndPoint,Relative(E)),O--P*Sin(30)*dir(30),Arrow);
	draw(Label("$\boldsymbol{N}_2$",EndPoint,Relative(E)),P*Sin(30)*dir(30)--P*dir(90),Arrow);
	draw(Label("$\boldsymbol{P}_1+\boldsymbol{P}_2$",MidPoint,Relative(E)),P*dir(90)--O,Arrow);
	r = 0.2;
	draw(O--0.3*P*dir(0),dashed);
	draw(Label("$\ang{30}$",MidPoint,Relative(E)),arc(O,r,0,30),Arrow);
	draw(P*Sin(30)*dir(30)--P*Sin(30)*dir(30)+0.3*P*dir(180),dashed);
	draw(Label("$\ang{60}$",MidPoint,Relative(W)),arc(P*Sin(30)*dir(30),r,180,120),Arrow);
\end{asy}
\caption{题\thequestion 受力分析1}
\label{37页4.56受力分析1}
\end{minipage}
\hspace{0.7cm}
\begin{minipage}[t]{0.45\textwidth}
\begin{asy}
	size(180);
	//37页4.56受力分析2
	pair O;
	real P,NN,r,delta;
	O = (0,0);
	P = 2;
	NN = P*Sin(30)*Cos(30);
	draw(Label("$\boldsymbol{N}_1$",EndPoint,Relative(E)),O--P*Sin(30)*dir(30),Arrow);
	draw(Label("$\boldsymbol{N}$",EndPoint,Relative(E)),O--NN*dir(180-6.86989764584402),Arrow);
	draw(Label("$\boldsymbol{P}_1$",MidPoint,Relative(E)),O--0.25*P*dir(-90),Arrow);
	r = 0.2;
	draw(O--0.3*P*dir(0),dashed);
	draw(Label("$\ang{30}$",MidPoint,Relative(E)),arc(O,r,0,30),Arrow);
	draw(O--0.3*P*dir(180),dashed);
	r = 0.5;
	delta = 15;
	draw(Label("$\phi$",MidPoint,Relative(E)),arc(O,r,180-6.86989764584402-delta,180-6.86989764584402),Arrow);
	draw(arc(O,r,180+delta,180),Arrow);
\end{asy}
\caption{题\thequestion 受力分析2}
\label{37页4.56受力分析2}
\end{minipage}
\end{figure}

取两圆柱整体受力分析(如图\ref{37页4.56受力分析1}),可有
\begin{equation*}
\begin{cases}
	N_1 = (P_1+P_2)\sin \ang{30} = \dfrac12 (P_1+P_2) = \SI{20}{\newton} \\
	N_2 = (P_1+P_2)\sin \ang{60} = \dfrac{\sqrt{3}}{2} (P_1+P_2) = \SI{34.6}{\newton}
\end{cases}
\end{equation*}
然后取圆柱$C_1$受力分析(如图\ref{37页4.56受力分析2}),可有
\begin{equation*}
\begin{cases}
	N \cos \phi = N_1 \cos \ang{30} \\
	N \sin \phi + N_1 \sin \ang{30} = P_1
\end{cases}
\end{equation*}
可得$\phi = 0$以及$N= \SI{17.3}{\newton}$。
\end{solution}

\begin{question}[37页4.57]
如图\ref{37页4.57}所示,两个光滑匀质球$C_1$和$C_2$的半径分别为$R_1$和$R_2$,重量分别为$P_1$和$P_2$。这两球用绳子$AB$和$AD$挂在$A$点,且$AB = l_1,\,AD = l_2,\,l_1+R_1 = l_2 +R_2$,$\angle BAD = \alpha$。求绳子$AD$与水平面$AE$的夹角$\theta$、绳子的张力$T_1$和$T_2$,以及两球之间的压力。

\begin{figure}[htb]
\centering
\begin{asy}
	size(200);
	//37页4.57
	pair A,B,D,C1,C2;
	real r1,r2,l1,l2,l,alpha,theta,r;
	A = (0,0);
	l = 2;
	theta = -130;
	alpha = 50;
	C1 = l*dir(theta+alpha);
	C2 = l*dir(theta);
	r2 = 0.7;
	r1 = 2*l*Sin(alpha/2)-r2;
	l1 = l-r1;
	l2 = l-r2;
	draw(A--l1*dir(theta+alpha));
	
	draw(A--l2*dir(theta));
	draw(l1*dir(theta+alpha)--C1--C2--l2*dir(theta),dashed);
	draw(shift(C1)*scale(r1)*unitcircle);
	draw(shift(C2)*scale(r2)*unitcircle);
	draw(Label("$E$",EndPoint),A+0.6*l*dir(180)--A+0.6*l*dir(0),linewidth(0.8bp));
	pair P,dash,pace;
	int imax;
	imax = 30;
	dash = 0.1*dir(60);
	pace = 2*0.6*l*dir(0)/imax;
	P = A+0.6*l*dir(180)+0.2*pace;
	for(int i=1;i<=imax;i=i+1){
		draw(P--P+dash);
		P = P+pace;
	}
	dot(C1,UnFill);
	label("$C_1$",C1,S);
	dot(C2,UnFill);
	label("$C_2$",C2,S);
	label("$A$",A,3*N);
	label("$B$",l1*dir(theta+alpha),NW);
	label("$D$",l2*dir(theta),2*N);
	r = 0.5;
	draw(Label("$\alpha$",MidPoint,Relative(W)),arc(A,r,theta+alpha,theta),Arrows);
	r = 0.25;
	draw(Label("$\theta$",Relative(0.4),Relative(W)),arc(A,r,0,theta),Arrows);
\end{asy}
\caption{题\thequestion}
\label{37页4.57}
\end{figure}
\end{question}
\begin{solution}

\begin{figure}[htb]
\centering
\begin{minipage}[t]{0.45\textwidth}
\centering
\begin{asy}
	size(200);
	//37页4.57受力分析1
	pair O;
	real theta,alpha,P,T1,T2,l,r;
	O = (0,0);
	theta = 130;
	alpha = 50;
	P = 2;
	T1 = P/Sin(180-alpha)*Sin(theta-90);
	T2 = P/Sin(180-alpha)*Sin(alpha-theta+90);
	draw(Label("$\boldsymbol{T}_2$",EndPoint,Relative(E)),O--T2*dir(180-theta),Arrow);
	draw(Label("$\boldsymbol{T}_1$",EndPoint,Relative(E)),T2*dir(180-theta)--T2*dir(180-theta)+T1*dir(alpha+180-theta),Arrow);
	draw(Label("$\boldsymbol{P}_1+\boldsymbol{P}_2$",EndPoint,Relative(E)),T2*dir(180-theta)+T1*dir(alpha+180-theta)--O,Arrow);
	l = T2/Sin(theta-alpha)*Sin(alpha);
	r = 0.1;
	draw(l*dir(-theta)--O--l*dir(0),dashed);
	draw(T2*dir(180-theta)--T2*dir(180-theta)+l*dir(-theta+alpha),dashed);
	draw(Label("$\theta$",MidPoint,Relative(E)),arc(O,r,-theta,0),Arrows);
	r = 0.15;
	draw(Label("$\alpha$",MidPoint,Relative(E)),arc(T2*dir(180-theta),r,-theta,-theta+alpha),Arrows);
\end{asy}
\caption{题\thequestion 受力分析1}
\label{37页4.57受力分析1}
\end{minipage}
\hspace{0.7cm}
\begin{minipage}[t]{0.45\textwidth}
\centering
\begin{asy}
	size(200);
	//37页4.57受力分析2
	pair O;
	real theta,alpha,phi,P,T1,NN,l,r,delta;
	O = (0,0);
	theta = 130;
	alpha = 50;
	phi = theta-alpha/2-90;
	P = 2;
	T1 = P/Sin(theta-alpha-phi)*Sin(90+phi);
	NN = P/Sin(theta-alpha-phi)*Sin(90-theta+alpha);
	draw(Label("$\boldsymbol{N}$",EndPoint,Relative(E)),O--NN*dir(-phi),Arrow);
	draw(Label("$\boldsymbol{T}_1$",MidPoint,Relative(E)),NN*dir(-phi)--NN*dir(-phi)+T1*dir(180-theta+alpha),Arrow);
	draw(Label("$\boldsymbol{P}_1$",EndPoint,Relative(E)),NN*dir(-phi)+T1*dir(180-theta+alpha)--O,Arrow);
	l = NN/Sin(180-theta+alpha)*Sin(theta-alpha-phi);
	r = 0.1;
	draw(NN*dir(-phi)+T1*dir(180-theta+alpha)--NN*dir(-phi)+T1*dir(180-theta+alpha)+l*dir(0),dashed);
	draw(Label("$\theta-\alpha$",MidPoint,Relative(E)),arc(NN*dir(-phi)+T1*dir(180-theta+alpha),r,-theta+alpha,0),Arrows);
	delta = 15;
	draw(O--l*dir(0),dashed);
	r = 0.2;
	draw(Label("$\phi$",BeginPoint,E),arc(O,r,delta,0),Arrow);
	draw(arc(O,r,-phi-delta,-phi),Arrow);
	r = 0.15;
	//draw(Label("$\alpha$",MidPoint,Relative(E)),arc(T2*dir(180-theta),r,-theta,-theta+alpha),Arrows);
	
	dot(NN*dir(-phi)+T1*dir(180-theta+alpha)+(0,0.01),invisible);
\end{asy}
\caption{题\thequestion 受力分析2}
\label{37页4.57受力分析2}
\end{minipage}
\end{figure}

取两圆柱整体受力分析(如图\ref{37页4.57受力分析1}),根据正弦定理可有
\begin{equation*}
	\frac{T_1}{\sin\left(\theta-\dfrac{\pi}{2}\right)} = \frac{T_2}{\sin\left(\alpha-\theta+\dfrac{\pi}{2}\right)} = \frac{P_1+P_2}{\sin(\pi-\alpha)}
\end{equation*}
由此可得
\begin{subnumcases}{}
	T_1 = -\dfrac{\cos \theta}{\sin \alpha} (P_1+P_2) \label{4.57-1} \\
	T_2 = \dfrac{\cos(\theta-\alpha)}{\sin \alpha} (P_1+P_2) \label{4.57-2}
\end{subnumcases}
然后取圆柱$C_1$受力分析(如图\ref{37页4.57受力分析2}),记$\phi = \theta-\alpha-\dfrac{\pi-\alpha}{2} = \theta-\dfrac{\alpha}{2} - \dfrac{\pi}{2}$为$C_1C_2$与水平方向的夹角,则可有
\begin{equation*}
	\frac{P_1}{\sin(\theta-\alpha-\phi)} = \frac{T_1}{\sin\left(\dfrac{\pi}{2}+\phi\right)} = \frac{N}{\sin \left(\dfrac{\pi}{2} - \theta+\alpha\right)}
\end{equation*}
由此可得
\begin{subnumcases}{}
	T_1 = \dfrac{\sin \left(\theta-\dfrac{\alpha}{2}\right)}{\cos \dfrac{\alpha}{2}} P_1 \label{4.57-3} \\
	N = \dfrac{\cos(\theta-\alpha)}{\cos \dfrac{\alpha}{2}}P_1 \label{4.57-4}
\end{subnumcases}
由式\eqref{4.57-1}和式\eqref{4.57-3}可得
\begin{equation*}
	\tan \theta = -\frac{P_2+P_1\cos\alpha}{P_1\sin \alpha}
\end{equation*}
绳子的张力$T_1$和$T_2$以及两球之间的压力$N$前面已求出。
\end{solution}

\begin{question}[45页5.14]
如图\ref{45页5.14}所示,重$Q$的圆柱放在支座$A$和$B$上,这两支座关于圆柱中垂线对称。圆柱与支座之间的摩擦系数等于$\mu$。问:当切向力$T$多大时,圆柱开始转动?又当角$\theta$多大时该装置自锁?

\begin{figure}[htb]
\centering
\begin{asy}
	size(200);
	//45页5.14
	picture tmp;
	pair O,A,B,P,dash,pace;
	real R,r,l,d,dd,theta;
	int imax;
	path clp;
	O = (0,0);
	r = 1;
	l = 0.2;
	d = 0.15;
	theta = 50;
	A = (r+l)*dir(-theta-90);
	B = (r+l)*dir(theta-90);
	draw(tmp,A+1.5*d*dir(180-theta)--A-1.5*d*dir(180-theta),linewidth(0.8bp));
	draw(tmp,A+d*dir(180-theta)--A+d*dir(180-theta)+(l+r)*dir(90-theta)--A-d*dir(180-theta)+(l+r)*dir(90-theta)--A-d*dir(180-theta));
	imax = 8;
	dash = 0.5*dir(-90);
	pace = -2*1.5*d*dir(180-theta)/imax;
	P = A+1.5*d*dir(180-theta)-4*pace;
	for(int i=1;i<=2*imax;i=i+1){
		draw(tmp,P--P+dash);
		P = P+pace;
	}
	dd = 0.08;
	clp = A+1.5*d*dir(180-theta)-dd*dir(90-theta)--A+1.5*d*dir(180-theta)+(l+r)*dir(90-theta)--A-1.5*d*dir(180-theta)+(l+r)*dir(90-theta)--A-1.5*d*dir(180-theta)-dd*dir(90-theta)--cycle;
	clip(tmp,clp);
	add(tmp);
	add(xscale(-1)*tmp);
	unfill(shift(O)*scale(r)*unitcircle);
	draw(shift(O)*scale(r)*unitcircle);
	draw(O--1.1*A,dashed);
	draw(O--1.1*B,dashed);
	draw((r+l)*dir(180)--(r+l)*dir(0),dashed);
	draw(Label("$\boldsymbol{Q}$",EndPoint),O--(r+l)*dir(-90),Arrow);
	draw(Label("$\boldsymbol{T}$",BeginPoint,Relative(W)),r*dir(0)+0.9*(r+l)*dir(90)--r*dir(0),Arrow);
	R = 0.2;
	draw(Label("$\theta$",MidPoint,Relative(W)),arc(O,R,theta-90,-90),Arrows);
	draw(Label("$\theta$",MidPoint,Relative(W)),arc(O,R,-90,-90-theta),Arrows);
	label("$A$",A,3*W);
	label("$B$",B,3*E);
\end{asy}
\caption{题\thequestion}
\label{45页5.14}
\end{figure}
\end{question}
\begin{solution}
\begin{figure}[htb]
\centering
\begin{asy}
	size(200);
	//45页5.14受力分析
	picture tmp;
	pair O,A,B,P,dash,pace;
	real R,r,l,d,dd,theta;
	int imax;
	path clp;
	O = (0,0);
	r = 1;
	l = 0.2;
	d = 0.15;
	theta = 50;
	A = (r+l)*dir(-theta-90);
	B = (r+l)*dir(theta-90);
	draw(tmp,A+1.5*d*dir(180-theta)--A-1.5*d*dir(180-theta),linewidth(0.8bp));
	draw(tmp,A+d*dir(180-theta)--A+d*dir(180-theta)+(l+r)*dir(90-theta)--A-d*dir(180-theta)+(l+r)*dir(90-theta)--A-d*dir(180-theta));
	imax = 8;
	dash = 0.5*dir(-90);
	pace = -2*1.5*d*dir(180-theta)/imax;
	P = A+1.5*d*dir(180-theta)-4*pace;
	for(int i=1;i<=2*imax;i=i+1){
		draw(tmp,P--P+dash);
		P = P+pace;
	}
	dd = 0.08;
	clp = A+1.5*d*dir(180-theta)-dd*dir(90-theta)--A+1.5*d*dir(180-theta)+(l+r)*dir(90-theta)--A-1.5*d*dir(180-theta)+(l+r)*dir(90-theta)--A-1.5*d*dir(180-theta)-dd*dir(90-theta)--cycle;
	clip(tmp,clp);
	add(tmp);
	add(xscale(-1)*tmp);
	unfill(shift(O)*scale(r)*unitcircle);
	draw(shift(O)*scale(r)*unitcircle);
	draw(O--1.1*A,dashed);
	draw(O--1.1*B,dashed);
	draw((r+l)*dir(180)--(r+l)*dir(0),dashed);
	draw(Label("$\boldsymbol{Q}$",EndPoint),O--(r+l)*dir(-90),Arrow);
	draw(Label("$\boldsymbol{T}$",BeginPoint,Relative(W)),r*dir(0)+0.9*(r+l)*dir(90)--r*dir(0),Arrow);
	R = 0.2;
	draw(Label("$\theta$",MidPoint,Relative(W)),arc(O,R,theta-90,-90),Arrows);
	draw(Label("$\theta$",MidPoint,Relative(W)),arc(O,R,-90,-90-theta),Arrows);
	label("$A$",A,3*W);
	label("$B$",B,3*E);
	A = r*dir(-theta-90);
	B = r*dir(theta-90);
	draw(Label("$\boldsymbol{f}_1$",EndPoint),A--A+2*l*dir(-theta),Arrow);
	draw(Label("$\boldsymbol{N}_1$",EndPoint,Relative(W)),A--A+2.5*l*dir(90-theta),Arrow);
	draw(Label("$\boldsymbol{f}_2$",EndPoint),B--B+2*l*dir(theta),Arrow);
	draw(Label("$\boldsymbol{N}_2$",EndPoint,Relative(E)),B--B+2.5*l*dir(theta+90),Arrow);
\end{asy}
\caption{题\thequestion 受力分析}
\label{45页5.14受力分析}
\end{figure}

根据受力平衡和力矩平衡,可有
\begin{subnumcases}{}
	Q+T+f_1\sin \theta = N_1 \cos \theta + N_2 \cos \theta + f_2 \sin \theta \label{5.14-1} \\
	f_1 \cos \theta + f_2 \cos \theta + N_1 \sin \theta = N_2 \sin \theta \label{5.14-2} \\
	TR = f_1 R + f_2 R \label{5.14-3} \\
	f_1 = \mu N_1 \label{5.14-4} \\
	f_2 = \mu N_2 \label{5.14-5}
\end{subnumcases}
将式\eqref{5.14-1}和式\eqref{5.14-2}两端乘以$\mu$,并考虑到式\eqref{5.14-4}和式\eqref{5.14-5},可得
\begin{equation*}
\begin{cases}
	\displaystyle f_1 (\mu - \cos \theta + \mu \sin \theta) + f_2 (\mu - \cos \theta - \mu \sin \theta) = -\mu Q \\
	\displaystyle f_1 (\sin \theta + \mu \cos \theta) - f_2 (\sin \theta - \mu \cos \theta) = 0
\end{cases}
\end{equation*}
解此方程组可得
\begin{equation*}
\begin{cases}
	\displaystyle f_1 = \dfrac{\mu Q(\sin \theta - \mu \cos \theta)}{2\sin \theta \big[(1+\mu^2) \cos \theta - \mu\big]} \\[1.5ex]
	\displaystyle f_2 = \dfrac{\mu Q(\sin \theta + \mu \cos \theta)}{2\sin \theta \big[(1+\mu^2) \cos \theta - \mu\big]}
\end{cases}
\end{equation*}
再由式\eqref{5.14-3}可得
\begin{equation*}
	T = f_1 + f_2 = \dfrac{\mu Q}{(1+\mu^2) \cos \theta - \mu}
\end{equation*}
该体系不会自锁的条件为
\begin{equation*}
	T = \dfrac{\mu Q}{(1+\mu^2) \cos \theta - \mu} \geqslant 0
\end{equation*}
即
\begin{equation*}
	\theta \leqslant \arccos \frac{\mu}{1+\mu^2}
\end{equation*}
\end{solution}

\begin{question}[45页5.15]
如图\ref{45页5.15}所示,不计曲柄机构的滑块$A$和导轨之间的摩擦,也不计机构上所有铰链和轴承的摩擦,机构在图标位置能提起重物$Q$,试求力$P$的大小。如果滑块$A$和导轨之间的摩擦系数是$\mu$,为使重物$Q$不动,求力$P$的最大值和最小值。

\begin{figure}[htb]
\centering
\begin{minipage}[t]{0.5\textwidth}
\centering
\begin{asy}
	size(300);
	//45页5.15
	picture tmp,dashpic;
	pair O;
	real a,r,theta,l,da,d,ri,ro,tri;
	O = (0,0);
	a = 1;
	r = 1.6;
	l = 4;
	da = 0.4;
	d = 0.05;
	theta = 50;
	draw(shift(O)*scale(a)*unitcircle,linewidth(0.8bp));
	draw(tmp,O--r*dir(theta)--l*dir(90),dashed);
	draw(tmp,O--l*dir(90),dashed);
	draw(box(l*dir(90)-(da,da),l*dir(90)+(da,da)));
	draw(tmp,rotate(theta)*box((0,-d),(r,d)),linewidth(0.8bp));
	draw(tmp,shift(r*dir(theta))*rotate(degrees(l*dir(90)-r*dir(theta)))*box((0,-d),(length(l*dir(90)-r*dir(theta)),d)),linewidth(0.8bp));
	ri = 0.04;
	ro = 0.08;
	draw(tmp,r*dir(theta)--r*dir(theta)+1*r*dir(theta-90));
	unfill(tmp,shift(r*dir(theta))*scale(ro)*unitcircle);
	draw(tmp,shift(r*dir(theta))*scale(ro)*unitcircle);
	draw(tmp,shift(r*dir(theta))*scale(ri)*unitcircle);
	unfill(tmp,shift(l*dir(90))*scale(ro)*unitcircle);
	draw(tmp,shift(l*dir(90))*scale(ro)*unitcircle);
	draw(tmp,shift(l*dir(90))*scale(ri)*unitcircle);
	unfill(tmp,shift(O)*scale(a)*unitcircle);
	draw(tmp,Label("$a$",MidPoint,Relative(E)),O--a*dir(150),Arrow);
	draw(tmp,a*dir(180)--a*dir(0),dashed);
	draw(tmp,O--a*dir(90),dashed);
	draw(tmp,O--a*dir(theta),dashed);
	draw(tmp,O--1*r*dir(theta-90));
	tri = 0.35;
	draw(tmp,O--tri*a*dir(-60)--tri*a*dir(-120)--cycle);
	unfill(tmp,shift(O)*scale(ro)*unitcircle);
	draw(tmp,shift(O)*scale(ro)*unitcircle);
	draw(tmp,shift(O)*scale(ri)*unitcircle);
	add(tmp);
	erase(tmp);
	pair dash,P,pace;
	int imax;
	imax = 20;
	dash = 1*dir(120);
	pace = 4*da*dir(0)/imax;
	P = 2*da*dir(180);
	draw(dashpic,2*da*dir(180)--2*da*dir(0),linewidth(1bp));
	for(int i=1;i<=imax;i=i+1){
		draw(dashpic,P--P+dash);
		P = P+pace;
	}
	add(tmp,dashpic);
	clip(tmp,box((-0.3*a,-1),(0.3*a,0.15*a)));
	add(shift(tri*Sin(60)*dir(-90))*yscale(-1)*tmp);
	erase(tmp);
	add(tmp,dashpic);
	clip(tmp,box((-1.5*da,-1),(1.5*da,0.15*a)));
	add(shift(1.15*da*dir(180)+l*dir(90))*rotate(90)*tmp);
	add(shift(1.15*da*dir(0)+l*dir(90))*rotate(-90)*tmp);
	label("$A$",l*dir(90),E);
	draw(Label("$\boldsymbol{P}$",BeginPoint,Relative(E)),(l+da)*dir(90)+dir(90)--(l+da)*dir(90),Arrow);
	draw(a*dir(180)--a*dir(180)+1.2*a*dir(-90),linewidth(0.8bp));
	unfill(box(a*dir(180)+1.2*a*dir(-90)-(0.5*da,0.5*da),a*dir(180)+1.2*a*dir(-90)+(0.5*da,0.5*da)));
	draw(box(a*dir(180)+1.2*a*dir(-90)-(0.5*da,0.5*da),a*dir(180)+1.2*a*dir(-90)+(0.5*da,0.5*da)));
	draw(Label("$\boldsymbol{Q}$",EndPoint,Relative(W)),a*dir(180)+1.2*a*dir(-90)--a*dir(180)+2.2*a*dir(-90),Arrow);
	draw(Label("$r$",MidPoint,Relative(E)),0.8*r*dir(theta-90)--r*dir(theta)+0.8*r*dir(theta-90),Arrows);
	real rr = 0.5;
	draw(Label("$\theta$",MidPoint,Relative(E)),arc(O,rr,theta,90),Arrows);
	rr = 1.2;
	draw(Label("$\phi$",MidPoint,Relative(E)),arc(l*dir(90),rr,360-90,degrees(r*dir(theta)-l*dir(90))),Arrows);
\end{asy}
\caption{题\thequestion}
\label{45页5.15}
\end{minipage}
\hspace{0.7cm}
\begin{minipage}[t]{0.40\textwidth}
\centering
\begin{asy}
	size(150);
	//45页5.15受力分析
	pair O;
	real a,r,theta,l,da;
	O = (0,0);
	a = 1;
	r = 1.6;
	l = 4;
	da = 0.4;
	theta = 50;
	draw(box((-da,-da),(da,da)));
	draw(Label("$\boldsymbol{P}$",EndPoint),O--1.5*dir(-90),Arrow);
	draw(Label("$\boldsymbol{T}$",EndPoint),O--dir(l*dir(90)-r*dir(theta)),Arrow);
	draw(Label("$\boldsymbol{N}$",EndPoint),O--0.9*dir(0),Arrow);
	draw(Label("$\boldsymbol{f}$",EndPoint,Relative(W)),O--dir(-90),Arrow);
\end{asy}
\caption{题\thequestion 受力分析}
\label{45页5.15受力分析}
\end{minipage}
\end{figure}
\end{question}
\begin{solution}
根据转轮对其中心的力矩平衡,可有
\begin{equation}
	Tr\cos \left(\frac{\pi}{2}-\phi-\theta\right) = Qa
	\label{5.15-1}
\end{equation}
考虑滑块$A$的受力,如图\ref{45页5.15受力分析}所示,即有
\begin{equation}
\begin{cases}
	P+f=T\cos \phi \\
	T\sin \phi = N
\end{cases}
\label{5.15-2}
\end{equation}
当不计曲柄机构的滑块$A$和导轨之间的摩擦时,有$f=0$,此时可解得
\begin{equation*}
	P = \frac{Qa\cos \phi}{r\sin (\theta+\phi)}
\end{equation*}
当计及摩擦时,可得摩擦力$f$的取值范围为
\begin{equation}
	-\mu N \leqslant f \leqslant \mu N
	\label{5.15-3}
\end{equation}
由式\eqref{5.15-1}和式\eqref{5.15-2}解出$f$和$N$可有
\begin{equation*}
\begin{cases}
	f = \dfrac{Qa\cos \phi}{r\sin(\phi+\theta)}-P \\[1.5ex]
	N = \dfrac{Qa\sin \phi}{r\sin(\phi+\theta)}
\end{cases}
\end{equation*}
根据式\eqref{5.15-3}可有$P$的取值范围
\begin{equation*}
	\frac{Qa(\cos \phi-\mu\sin\phi)}{r\sin(\phi+\theta)} \leqslant P \leqslant \frac{Qa(\cos \phi+\mu\sin\phi)}{r\sin(\phi+\theta)}
\end{equation*}
\end{solution}

\begin{question}[45页5.16]
如图\ref{45页5.16}所示,沿不光滑曲面提升重为$P$的物体$B$过程中,依靠绳索$BAD$保持平衡。该曲面是四分之一圆柱面。曲面和重物间的摩擦系数是$\mu=\tan \phi$,其中$\phi$是摩擦角\footnote{全反力(即法向反力与摩擦力的合力)与接触面法线间夹角的最大值称为摩擦角。}。试以角$\alpha$的函数表示出绳索的张力$T$,为使绳索中的张力取极值,角$\alpha$必须满足什么条件?重物和滑轮的尺寸不计。

\begin{figure}[htb]
\centering
\begin{asy}
	size(300);
	//45页5.16
	picture tmp;
	pair O,A,AO,B,C,D,dash,P;
	real r,l1,l2,r0,rr,d,alpha;
	path rail,clp;
	O = (0,0);
	r = 1;
	l1 = 0.5;
	l2 = 0.5;
	rail = (l1+r)*dir(180)--arc(O,r,180,270)--r*dir(-90)+l2*dir(0);
	draw(tmp,rail,linewidth(1bp));
	dash = 0.05*dir(-120);
	for(real p=0;p<=1;p=p+0.012){
		P = relpoint(rail,p);
		draw(tmp,P--P+dash);
	}
	alpha = 30;
	r0 = 0.02;
	d = 0.1;
	AO = r*dir(180)+r0*dir(45);
	B = (r-r0)*dir(-90-alpha);
	draw(shift(B)*rotate(-alpha)*box((-d,-1.2*r0),(d,1.2*r0)));
	clip(rail--O--cycle);
	draw(r*dir(-90)--O--r*dir(-90-alpha),dashed);
	draw(shift(AO)*scale(r0)*unitcircle);
	rr = 0.2;
	draw(Label("$\alpha$",MidPoint,Relative(E)),arc(O,rr,-90-alpha,-90),Arrows);
	A = AO+r0*dir(degrees(B-AO)+aCos(r0/length(B-AO)));
	draw(tmp,(l1+r)*dir(180)+2*r0*dir(90)--arc(AO,r0,90,degrees(B-AO)+aCos(r0/length(B-AO)))--B,linewidth(0.8bp));
	clip(tmp,box((-r-0.8*l1,-2*r),(0.5*l2,r)));
	add(tmp);
	label("$D$",(-r-0.8*l1,2*r0),W);
	label("$A$",A,NE);
	label("$B$",B,2*dir(30));
	label("$C$",r*dir(-90),NE);
\end{asy}
\caption{题\thequestion}
\label{45页5.16}
\end{figure}
\end{question}
\begin{solution}
摩擦系数为$\mu=\tan \phi$,即摩擦角为$\phi$,因此摩擦力与支持力的合力与支持力方向的夹角最大为$\phi$。据此可作出矢量三角形关系如图\ref{45页5.16受力分析}所示。

\begin{figure}[htb]
\centering
\begin{asy}
	size(200);
	//45页5.16受力分析
	pair O;
	real alpha,phi,P,R,RR,r;
	alpha = 30;
	phi = 40;
	P = 1;
	R = P/Sin(45+phi+alpha/2)*Sin(45-alpha/2);
	RR = P/Sin(45+alpha/2)*Sin(45-alpha/2);
	r = 0.1;
	draw(Label("$\boldsymbol{P}$",EndPoint,Relative(E)),O--P*dir(-90),Arrow);
	draw(Label("$\boldsymbol{R}$",EndPoint,Relative(E)),P*dir(-90)--P*dir(-90)+R*dir(90-alpha-phi),Arrow);
	draw(Label("$\boldsymbol{T}$",Relative(0.7),Relative(E)),P*dir(-90)+R*dir(90-alpha-phi)--O,Arrow);
	draw(P*dir(-90)--P*dir(-90)+RR*dir(90-alpha),dashed);
	draw(Label("$\phi$",MidPoint,Relative(E)),arc(P*dir(-90),r,90-alpha-phi,90-alpha),Arrows);
	draw(Label("$\alpha$",MidPoint,Relative(E)),arc(P*dir(-90),r,90-alpha,90),Arrows);
	draw(P*dir(-90)--interp(P*dir(-90)+R*dir(90-alpha-phi),O,0.2),dashed,Arrow);
	draw(O--R*Sin(alpha+phi)*dir(0),dashed);
	draw(Label("$\dfrac{\pi}{4}+\dfrac{\alpha}{2}$",MidPoint,Relative(E)),arc(O,r,-45-alpha/2,0),Arrows);
\end{asy}
\caption{题\thequestion 的矢量三角形}
\label{45页5.16受力分析}
\end{figure}

在图示条件下,根据正弦定理可有
\begin{equation*}
	\frac{P}{\sin\left(\phi+\dfrac{\alpha}{2}+\dfrac{\pi}{4}\right)} = \frac{R}{\sin \left(\dfrac{\pi}{4}-\dfrac{\alpha}{2}\right)} = \frac{T}{\sin (\alpha+\phi)}
\end{equation*}
所以有
\begin{equation*}
	T = P\frac{\sin (\alpha+\phi)}{\sin\left(\phi+\dfrac{\alpha}{2}+\dfrac{\pi}{4}\right)}
\end{equation*}
考虑
\begin{equation*}
	\dfrac{\mathrm{d} T}{\mathrm{d} \alpha} = P \frac{\cos(\alpha+\phi)\sin \left(\phi+\dfrac{\alpha}{2}+\dfrac{\pi}{4}\right) - \dfrac12 \cos \left(\phi+\dfrac{\alpha}{2}+\dfrac{\pi}{4}\right) \sin (\alpha+\phi)}{\sin^2 \left(\phi+\dfrac{\alpha}{2}+\dfrac{\pi}{4}\right)} = 0
\end{equation*}
即当
\begin{equation*}
	\frac{\tan (\alpha+\phi)}{\tan \left(\phi+\dfrac{\alpha}{2}+\dfrac{\pi}{4}\right)} = 2
\end{equation*}
时,$T$取极值。
\end{solution}

\begin{question}[48页5.27]
如图\ref{48页5.27}所示,梯子$AB$靠在铅直墙上,下端搁在水平地板上。梯子与墙、地板之间的摩擦系数分别为$\mu_1,\mu_2$。梯子连同站在上面的人共重$P$,重力作用点$C$按比值$m:n$两分梯子的长度。在平衡状态下,求梯子与墙之间的最大夹角,并求当$\alpha$等于该值时墙和地板反力的法向分量$N_A$和$N_B$。

\begin{figure}[htb]
\centering
\begin{minipage}[t]{0.45\textwidth}
\centering
\begin{asy}
	size(200);
	//48页5.27
	picture tmp;
	pair O,dash,P,C;
	real l,alpha,rate,ddash,d;
	path wall;
	O = (0,0);
	l = 2;
	alpha = 35;
	rate = 0.4;
	wall = l*dir(90)--O--l*dir(0);
	dash = dir(-135);
	draw(tmp,wall,linewidth(1bp));
	for(real r=0;r<=1;r=r+0.01){
		P = relpoint(wall,r);
		draw(tmp,P--P+dash);
	}
	ddash = 0.05;
	clip(tmp,box((-ddash,-ddash),(1.1*l*Sin(alpha),1.1*l*Cos(alpha))));
	add(tmp);
	d = 0.05;
	draw(l*Sin(alpha)*dir(0)--l*Sin(alpha)*dir(0)+d*dir(alpha)--l*Cos(alpha)*dir(90)+d*dir(alpha)--l*Cos(alpha)*dir(90)--cycle,linewidth(0.8bp));
	C = interp(l*Cos(alpha)*dir(90)+0.5*d*dir(alpha),l*Sin(alpha)*dir(0)+0.5*d*dir(alpha),rate);
	dot(C);
	draw(Label("$\boldsymbol{P}$",EndPoint),C--C+0.7*dir(-90),Arrow);
	label("$A$",l*Cos(alpha)*dir(90),2*W);
	label("$B$",0.85*l*Sin(alpha)*dir(0),N);
	label("$C$",C,dir(alpha-180));
	draw(l*Sin(alpha)*dir(0)+d*dir(alpha)--l*Sin(alpha)*dir(0)+4*d*dir(alpha));
	draw(C+d*dir(alpha)--C+4*d*dir(alpha));
	draw(l*Cos(alpha)*dir(90)+d*dir(alpha)--l*Cos(alpha)*dir(90)+4*d*dir(alpha));
	draw(Label("$na$",MidPoint,Relative(E)),C+3*d*dir(alpha)--l*Cos(alpha)*dir(90)+3*d*dir(alpha),Arrows);
	draw(Label("$ma$",MidPoint,Relative(W)),C+3*d*dir(alpha)--l*Sin(alpha)*dir(0)+3*d*dir(alpha),Arrows);
	draw(Label("$\alpha$",MidPoint,Relative(E)),arc(l*Cos(alpha)*dir(90),0.3,-90,-90+alpha),Arrows);
\end{asy}
\caption{题\thequestion}
\label{48页5.27}
\end{minipage}
\hspace{0.5cm}
\begin{minipage}[t]{0.45\textwidth}
\centering
\begin{asy}
	size(200);
	//48页5.27受力分析
	picture tmp;
	pair O,dash,P,C;
	real l,alpha,rate,ddash,d;
	path wall;
	O = (0,0);
	l = 2;
	alpha = 35;
	rate = 0.4;
	wall = l*dir(90)--O--l*dir(0);
	dash = dir(-135);
	draw(tmp,wall,linewidth(1bp));
	for(real r=0;r<=1;r=r+0.01){
		P = relpoint(wall,r);
		draw(tmp,P--P+dash);
	}
	ddash = 0.05;
	clip(tmp,box((-ddash,-ddash),(1.1*l*Sin(alpha),1.1*l*Cos(alpha))));
	add(tmp);
	d = 0.05;
	draw(l*Sin(alpha)*dir(0)--l*Sin(alpha)*dir(0)+d*dir(alpha)--l*Cos(alpha)*dir(90)+d*dir(alpha)--l*Cos(alpha)*dir(90)--cycle,linewidth(0.8bp));
	C = interp(l*Cos(alpha)*dir(90)+0.5*d*dir(alpha),l*Sin(alpha)*dir(0)+0.5*d*dir(alpha),rate);
	dot(C);
	draw(Label("$\boldsymbol{P}$",EndPoint),C--C+0.7*dir(-90),Arrow);
	label("$A$",l*Cos(alpha)*dir(90),2*W);
	label("$B$",0.86*l*Sin(alpha)*dir(0),N);
	label("$C$",C,dir(alpha-180));
	draw(Label("$\alpha$",MidPoint,Relative(E)),arc(l*Cos(alpha)*dir(90),0.3,-90,-90+alpha),Arrows);
	draw(Label("$\boldsymbol{f}_A$",EndPoint,Relative(E)),l*Cos(alpha)*dir(90)--l*Cos(alpha)*dir(90)+0.4*dir(90),Arrow);
	draw(Label("$\boldsymbol{N}_A$",EndPoint),l*Cos(alpha)*dir(90)--l*Cos(alpha)*dir(90)+0.4*dir(0),Arrow);
	draw(Label("$\boldsymbol{f}_B$",EndPoint,Relative(E)),l*Sin(alpha)*dir(0)--l*Sin(alpha)*dir(0)+0.4*dir(180),Arrow);
	draw(Label("$\boldsymbol{N}_B$",EndPoint),l*Sin(alpha)*dir(0)--l*Sin(alpha)*dir(0)+0.6*dir(90),Arrow);
\end{asy}
\caption{题\thequestion 受力分析}
\label{48页5.27受力分析}
\end{minipage}
\end{figure}
\end{question}
\begin{solution}
在夹角最大时,梯子在该状态下恰不能滑动,因此摩擦力均取最大值,即有方程组
\begin{subnumcases}{}
	N_A - f_B = 0 \label{5.27-1} \\
	f_A + N_B - P = 0 \label{5.27-2} \\
	N_A na \cos \alpha + f_A na \sin \alpha + f_B ma \cos \alpha - N_B ma \sin \alpha = 0 \label{5.27-3} \\
	f_A = \mu_1 N_A \label{5.27-4} \\
	f_B = \mu_2 N_B \label{5.27-5}
\end{subnumcases}
由式\eqref{5.27-3}可得
\begin{equation*}
	\tan \alpha = \frac{nN_A+mf_B}{mN_B-nf_A}
\end{equation*}
由式\eqref{5.27-1}、式\eqref{5.27-4}和式\eqref{5.27-5}可得
\begin{equation*}
	f_B = N_A,\quad N_B = \frac{1}{\mu_2}N_A,\quad f_A = \mu_1 N_A
\end{equation*}
由此可有
\begin{equation*}
	\tan \alpha = \frac{(m+n)\mu_2}{m-n\mu_1\mu_2}
\end{equation*}
再考虑到式\eqref{5.27-2}即有
\begin{equation*}
	N_A = \frac{\mu_2 P}{1+\mu_1\mu_2},\quad N_B = \frac{P}{1+\mu_1\mu_2}
\end{equation*}
\end{solution}

\begin{question}[48页5.28]
重$P$的梯子$AB$靠在光滑的墙上,并搁置在不光滑的水平地面上。梯子与地板的摩擦系数为$\mu$。为使重$p$的人能沿梯子爬到顶端,求梯子与地板的夹角$\alpha$。

\begin{figure}[htb]
\centering
\begin{minipage}[t]{0.45\textwidth}
\centering
\begin{asy}
	size(200);
	//48页5.28
	picture tmp;
	pair O,dash,P,C;
	real l,alpha,rate,ddash,d;
	path wall;
	O = (0,0);
	l = 2;
	alpha = 35;
	rate = 0.5;
	wall = l*dir(90)--O--l*dir(0);
	dash = dir(-135);
	draw(tmp,wall,linewidth(1bp));
	for(real r=0;r<=1;r=r+0.01){
		P = relpoint(wall,r);
		draw(tmp,P--P+dash);
	}
	ddash = 0.05;
	clip(tmp,box((-ddash,-ddash),(1.1*l*Sin(alpha),1.1*l*Cos(alpha))));
	add(tmp);
	d = 0.05;
	draw(l*Sin(alpha)*dir(0)--l*Sin(alpha)*dir(0)+d*dir(alpha)--l*Cos(alpha)*dir(90)+d*dir(alpha)--l*Cos(alpha)*dir(90)--cycle,linewidth(0.8bp));
	//C = interp(l*Cos(alpha)*dir(90)+0.5*d*dir(alpha),l*Sin(alpha)*dir(0)+0.5*d*dir(alpha),rate);
	//dot(C);
	//draw(Label("$\boldsymbol{P}$",EndPoint),C--C+0.7*dir(-90),Arrow);
	label("$A$",l*Cos(alpha)*dir(90),2*W);
	label("$B$",1.1*l*Sin(alpha)*dir(0),N);
	//label("$C$",C,dir(alpha-180));
	draw(Label("$\alpha$",MidPoint,Relative(E)),arc(l*Sin(alpha)*dir(0),0.3,90+alpha,180),Arrows);
\end{asy}
\caption{题\thequestion}
\label{48页5.28}
\end{minipage}
\hspace{0.5cm}
\begin{minipage}[t]{0.45\textwidth}
\centering
\begin{asy}
	size(200);
	//48页5.28受力分析
	picture tmp;
	pair O,dash,P,C;
	real l,alpha,rate,ddash,d;
	path wall;
	O = (0,0);
	l = 2;
	alpha = 35;
	rate = 0.5;
	wall = l*dir(90)--O--l*dir(0);
	dash = dir(-135);
	draw(tmp,wall,linewidth(1bp));
	for(real r=0;r<=1;r=r+0.01){
		P = relpoint(wall,r);
		draw(tmp,P--P+dash);
	}
	ddash = 0.05;
	clip(tmp,box((-ddash,-ddash),(1.1*l*Sin(alpha),1.1*l*Cos(alpha))));
	add(tmp);
	d = 0.05;
	draw(l*Sin(alpha)*dir(0)--l*Sin(alpha)*dir(0)+d*dir(alpha)--l*Cos(alpha)*dir(90)+d*dir(alpha)--l*Cos(alpha)*dir(90)--cycle,linewidth(0.8bp));
	C = interp(l*Cos(alpha)*dir(90)+0.5*d*dir(alpha),l*Sin(alpha)*dir(0)+0.5*d*dir(alpha),rate);
	dot(C);
	draw(Label("$\boldsymbol{P}$",EndPoint),C--C+0.55*dir(-90),Arrow);
	label("$A$",l*Cos(alpha)*dir(90),2*W);
	label("$B$",1.1*l*Sin(alpha)*dir(0),N);
	//label("$C$",C,dir(alpha-180));
	draw(Label("$\alpha$",MidPoint,Relative(E)),arc(l*Sin(alpha)*dir(0),0.1,90+alpha,180),Arrows);
	draw(Label("$\boldsymbol{p}$",EndPoint,Relative(W)),l*Cos(alpha)*dir(90)--l*Cos(alpha)*dir(90)+0.4*dir(-90),Arrow);
	draw(Label("$\boldsymbol{N}_A$",EndPoint),l*Cos(alpha)*dir(90)--l*Cos(alpha)*dir(90)+0.4*dir(0),Arrow);
	draw(Label("$\boldsymbol{f}_B$",EndPoint,Relative(E)),l*Sin(alpha)*dir(0)--l*Sin(alpha)*dir(0)+0.4*dir(180),Arrow);
	draw(Label("$\boldsymbol{N}_B$",EndPoint),l*Sin(alpha)*dir(0)--l*Sin(alpha)*dir(0)+0.6*dir(90),Arrow);
\end{asy}
\caption{题\thequestion 受力分析}
\label{48页5.28受力分析}
\end{minipage}
\end{figure}
\end{question}
\begin{solution}
根据图\ref{48页5.28受力分析}所示的受力分析,可有
\begin{subnumcases}{}
	N_A - f_B = 0 \label{5.28-1} \\
	N_B - p - P = 0 \label{5.28-2} \\
	N_A l \sin \alpha + f_B l \sin \alpha - N_B l \cos \alpha - pl \cos \alpha = 0 \label{5.28-3} \\
	f_B \leqslant \mu N_B \label{5.28-4}
\end{subnumcases}
由式\eqref{5.27-1}和式\eqref{5.27-2}可得
\begin{equation*}
	N_A = f_B,\quad N_B = p+P
\end{equation*}
再根据式\eqref{5.27-3}可得
\begin{equation*}
	f_B = \frac12 (2p+P) \cot \alpha
\end{equation*}
由此根据式\eqref{5.27-4}可有
\begin{equation*}
	\frac12 (2p+P) \cot \alpha \leqslant \mu(p+P)
\end{equation*}
即有
\begin{equation*}
	\tan \alpha \geqslant \frac{2p+P}{2\mu(p+P)}
\end{equation*}
\end{solution}

%\iffalse
\begin{question}[49页5.32]
如图\ref{49页5.32}所示,匀质杆两端$A$和$B$可沿粗糙的圆周滑动。圆周在铅垂平面内,半径是$a$。杆到圆心的距离为$b$,杆与圆周间的摩擦系数是$\mu$。当杆平衡时,求直线$OC$与圆周的铅垂直径的夹角$\phi$。

\begin{figure}[htb]
\centering
\begin{asy}
	size(200);
	//49页5.32
	picture tmp;
	pair O;
	real r,phi,theta,d,rr;
	O = (0,0);
	r = 1;
	d = 0.05;
	rr = 0.2;
	phi = 30;
	theta = 50;
	draw(shift(O)*scale(r)*unitcircle);
	draw(r*dir(theta-phi-90)--r*dir(theta-phi-90)+d*dir(90-phi)--r*dir(-theta-phi-90)+d*dir(90-phi)--r*dir(-theta-phi-90)--cycle,linewidth(1bp));
	draw(1.1*r*dir(90)--1.1*r*dir(-90),dashed);
	draw(O--(r*Cos(theta)-d)*dir(-90-phi),dashed);
	label("$O$",O,E);
	label("$a$",0.5*r*dir(90),W);
	label("$b$",0.5*r*Cos(theta)*dir(-90-phi),dir(180-phi));
	label("$C$",r*Cos(theta)*dir(-90-phi),dir(-90-phi));
	label("$A$",r*dir(-theta-phi-90),W);
	label("$B$",r*dir(theta-phi-90),SE);
	draw(Label("$\phi$",MidPoint,Relative(E)),arc(O,rr,-90-phi,-90),Arrows);
\end{asy}
\caption{题\thequestion}
\label{49页5.32}
\end{figure}
\end{question}
\begin{solution}
\begin{figure}[htb]
\centering
\begin{asy}
	size(250);
	//49页5.32受力分析
	picture tmp;
	pair O;
	real r,phi,theta,d,rr;
	O = (0,0);
	r = 1;
	d = 0.05;
	rr = 0.2;
	phi = 30;
	theta = 50;
	draw(shift(O)*scale(r)*unitcircle);
	draw(r*dir(theta-phi-90)--r*dir(theta-phi-90)+d*dir(90-phi)--r*dir(-theta-phi-90)+d*dir(90-phi)--r*dir(-theta-phi-90)--cycle,linewidth(1bp));
	draw(1.1*r*dir(90)--1.1*r*dir(-90),dashed);
	draw(O--(r*Cos(theta)-d)*dir(-90-phi),dashed);
	label("$O$",O,E);
	label("$a$",0.5*r*dir(90),W);
	label("$b$",0.5*r*Cos(theta)*dir(-90-phi),dir(180-phi));
	label("$C$",r*Cos(theta)*dir(-90-phi),dir(-90-phi));
	label("$A$",r*dir(-theta-phi-90),W);
	label("$B$",r*dir(theta-phi-90),SE);
	draw(Label("$\phi$",MidPoint,Relative(E)),arc(O,rr,-90-phi,-90),Arrows);
	draw(Label("$\boldsymbol{P}$",EndPoint),(r*Cos(theta)-d/2)*dir(-90-phi)--(r*Cos(theta)-d/2)*dir(-90-phi)+0.8*dir(-90),Arrow);
	draw(Label("$\boldsymbol{N}_1$",EndPoint,Relative(W)),r*dir(-theta-phi-90)--r*dir(-theta-phi-90)-0.6*r*dir(-theta-phi-90),Arrow);
	draw(Label("$\boldsymbol{f}_1$",EndPoint),r*dir(-theta-phi-90)--r*dir(-theta-phi-90)-rotate(90)*0.4*r*dir(-theta-phi-90),Arrow);
	draw(Label("$\boldsymbol{N}_2$",EndPoint,Relative(E)),r*dir(theta-phi-90)--r*dir(theta-phi-90)-0.6*r*r*dir(theta-phi-90),Arrow);
	draw(Label("$\boldsymbol{f}_2$",EndPoint),r*dir(theta-phi-90)--r*dir(theta-phi-90)-rotate(90)*0.4*r*r*dir(theta-phi-90),Arrow);
\end{asy}
\caption{题\thequestion 受力分析}
\label{49页5.32受力分析}
\end{figure}

由图\ref{49页5.32受力分析}所示的受力分析,在其平衡时,可列出方程组
\begin{subnumcases}{}
	f_1\sin (\theta+\phi) - f_2 \sin(\theta-\phi) + N_1 \cos(\theta+\phi) + N_2\cos(\theta-\phi) = P \\
	f_1\cos (\theta+\phi) - f_2 \cos(\theta-\phi) - N_1 \sin(\theta+\phi) + N_2\sin(\theta-\phi) = 0 \\
	f_1 a + f_2 a = Pb\sin \phi \\
	f_1 \leqslant \mu N_1 \\
	f_2 \leqslant \mu N_2
\end{subnumcases}
竟然解不出。
\end{solution}%\fi

\subsection{空间力系}

\begin{question}[57页6.21]
如图\ref{57页6.21}所示,在直角坐标轴上离原点$O$为$l$的$A,B,C$三点各自系有细绳$AD=BD=CD=L$,这三根细绳的另一端结在点$D$,其坐标为
\begin{equation*}
	x=y=z=\dfrac13 \left(l-\sqrt{3L^2-2l^2}\right)
\end{equation*}
在点$D$挂有重物$Q$,设$\sqrt{\dfrac23}l < L < l$,求三根细绳的张力$T_A$,$T_B$和$T_C$。
\begin{figure}[htb]
\centering
\begin{asy}
	size(250);
	//57页6.21
	pair O,i,j,k;
	real l,L,x,y,z,d1,d2;
	O = (0,0);
	i = (-sqrt(2)/4,-sqrt(14)/12);
	j = (sqrt(14)/4,-sqrt(2)/12);
	k = (0,2*sqrt(2)/3);
	l = 1;
	L = 0.82;
	x = (l-sqrt(3*L**2-2*l**2))/3;
	y = x;
	z = x;
	d1 = 0.1;
	d2 = 0.08;
	draw(Label("$x$",EndPoint),O--1.4*i,Arrow);
	draw(Label("$y$",EndPoint),O--1.4*j,Arrow);
	draw(Label("$z$",EndPoint),O--1.4*k,Arrow);
	label("$O$",O,W);
	draw(i--x*i+y*j+z*k--k,linewidth(0.8bp));
	draw(x*i+y*j+z*k--j,linewidth(0.8bp));
	draw(x*i+y*j+z*k--x*i+y*j+z*k-0.5*l*k);
	unfill(box(x*i+y*j+z*k-0.5*l*k-(d1,d2),x*i+y*j+z*k-0.5*l*k+(d1,d2)));
	draw(box(x*i+y*j+z*k-0.5*l*k-(d1,d2),x*i+y*j+z*k-0.5*l*k+(d1,d2)));
	label("$Q$",x*i+y*j+z*k-0.5*l*k);
	label("$A$",i,E);
	label("$B$",j,NE);
	label("$C$",k,E);
	label("$D$",x*i+y*j+z*k,NE);
	label("$E$",x*i+y*j+z*k-0.5*l*k+d2*k,NE);
	dot(i,UnFill);
	dot(j,UnFill);
	dot(k,UnFill);
	dot(x*i+y*j+z*k,UnFill);
\end{asy}
\caption{题\thequestion}
\label{57页6.21}
\end{figure}
\end{question}
\begin{solution}
首先写出各点的坐标为
\begin{equation*}
	A = \begin{pmatrix} l \\ 0 \\ 0 \end{pmatrix},\quad B = \begin{pmatrix} 0 \\ l \\ 0 \end{pmatrix},\quad C = \begin{pmatrix} 0 \\ 0 \\ l \end{pmatrix},\quad D = \begin{pmatrix} \dfrac13 \left(l-\sqrt{3L^2-2l^2}\right) \\[1.5ex] \dfrac13 \left(l-\sqrt{3L^2-2l^2}\right) \\[1.5ex] \dfrac13 \left(l-\sqrt{3L^2-2l^2}\right) \end{pmatrix}
\end{equation*}
据此求出四条线的方向矢量
\begin{align*}
	& \mbf{l}_{DA} = \begin{pmatrix} \dfrac23 \dfrac{l}{L} + \dfrac13 \sqrt{3-2\left(\dfrac{l}{L}\right)^2} \\[1.5ex] -\dfrac13 \dfrac{l}{L} + \dfrac13 \sqrt{3-2\left(\dfrac{l}{L}\right)^2} \\[1.5ex] -\dfrac13 \dfrac{l}{L} + \dfrac13 \sqrt{3-2\left(\dfrac{l}{L}\right)^2} \end{pmatrix},\quad
	\mbf{l}_{DB} = \begin{pmatrix}-\dfrac13 \dfrac{l}{L} + \dfrac13 \sqrt{3-2\left(\dfrac{l}{L}\right)^2} \\[1.5ex] \dfrac23 \dfrac{l}{L} + \dfrac13 \sqrt{3-2\left(\dfrac{l}{L}\right)^2} \\[1.5ex] -\dfrac13 \dfrac{l}{L} + \dfrac13 \sqrt{3-2\left(\dfrac{l}{L}\right)^2} \end{pmatrix}, \\
	& \mbf{l}_{DC} = \begin{pmatrix} -\dfrac13 \dfrac{l}{L} + \dfrac13 \sqrt{3-2\left(\dfrac{l}{L}\right)^2} \\[1.5ex] -\dfrac13 \dfrac{l}{L} + \dfrac13 \sqrt{3-2\left(\dfrac{l}{L}\right)^2} \\[1.5ex] \dfrac23 \dfrac{l}{L} + \dfrac13 \sqrt{3-2\left(\dfrac{l}{L}\right)^2} \end{pmatrix},\quad
	\mbf{l}_{DE} = \begin{pmatrix} 0 \\ 0 \\ -1 \end{pmatrix}
\end{align*}
平衡方程为
\begin{equation*}
	\mbf{T}_A + \mbf{T}_B + \mbf{T}_C + \mbf{Q} = \mbf{0}
\end{equation*}
即
\begin{equation*}
	T_A\mbf{l}_{DA} + T_B\mbf{l}_{DB} + T_C\mbf{l}_{DC} + Q\mbf{l}_{DE} = \mbf{0}
\end{equation*}
写成矩阵形式为
\begin{equation*}
	\begin{pmatrix} 
		\dfrac23 \dfrac{l}{L} + \dfrac13 \sqrt{3-2\left(\dfrac{l}{L}\right)^2} & -\dfrac13 \dfrac{l}{L} + \dfrac13 \sqrt{3-2\left(\dfrac{l}{L}\right)^2} & -\dfrac13 \dfrac{l}{L} + \dfrac13 \sqrt{3-2\left(\dfrac{l}{L}\right)^2} \\[1.5ex]
		-\dfrac13 \dfrac{l}{L} + \dfrac13 \sqrt{3-2\left(\dfrac{l}{L}\right)^2} & \dfrac23 \dfrac{l}{L} + \dfrac13 \sqrt{3-2\left(\dfrac{l}{L}\right)^2} & -\dfrac13 \dfrac{l}{L} + \dfrac13 \sqrt{3-2\left(\dfrac{l}{L}\right)^2} \\[1.5ex]
		-\dfrac13 \dfrac{l}{L} + \dfrac13 \sqrt{3-2\left(\dfrac{l}{L}\right)^2} & -\dfrac13 \dfrac{l}{L} + \dfrac13 \sqrt{3-2\left(\dfrac{l}{L}\right)^2} & \dfrac23 \dfrac{l}{L} + \dfrac13 \sqrt{3-2\left(\dfrac{l}{L}\right)^2}
	\end{pmatrix} \begin{pmatrix} T_A \\ T_B \\ T_C \end{pmatrix} = \begin{pmatrix} 0 \\ 0 \\ Q \end{pmatrix}
\end{equation*}
可解得
\begin{equation*}
	\begin{pmatrix} T_A \\ T_B \\ T_C \end{pmatrix} = \begin{pmatrix} \dfrac{l-\sqrt{3L^2-2l^2}}{3l\sqrt{3L^2-2l^2}}LQ \\[1.5ex] \dfrac{l-\sqrt{3L^2-2l^2}}{3l\sqrt{3L^2-2l^2}}LQ \\[1.5ex] \dfrac{l+2\sqrt{3L^2-2l^2}}{3l\sqrt{3L^2-2l^2}}LQ \end{pmatrix}
\end{equation*}
\end{solution}

\subsection{点的运动学}

\begin{question}[86页11.4]
如图\ref{86页11.4}所示的曲柄连杆机构,曲柄$OA$以匀角速度$\omega$运动,求此机构的连杆中点$M$和滑块$B$的速度随时间的变化规律。已知长度$OA=OB=a$。

\begin{figure}[htb]
\centering
\begin{asy}
	size(250);
	//86页11.4
	picture tmp,dashpic;
	pair O,A,B,dash,P;
	real phi,a,x,y,ri,ro,tri,d;
	path tre;
	O = (0,0);
	phi = 35;
	a = 1;
	x = 2.3;
	y = 1;
	A = a*dir(phi);
	B = A+a*dir(-phi);
	ri = 0.018;
	ro = 0.035;
	tri = 0.1;
	tre = 0.5*x*dir(180)--0.5*x*dir(0);
	dash = a*dir(-60);
	draw(dashpic,tre,linewidth(1bp));
	for(real r=0;r<=1;r=r+0.02){
		P = relpoint(tre,r);
		draw(dashpic,P--P+dash);
	}
	d = 0.05;
	add(tmp,dashpic);
	clip(tmp,box((-1.5*tri,1),(1.5*tri,-d)));
	add(shift(tri*Sin(60)*dir(-90))*tmp);
	erase(tmp);
	add(tmp,dashpic);
	clip(tmp,box((-3*tri,1),(3*tri,-d)));
	add(shift(B)*tmp);
	erase(tmp);
	draw(Label("$x$",EndPoint),O--x*dir(0),Arrow);
	draw(Label("$y$",EndPoint),O--y*dir(90),Arrow);
	unfill(shift(B)*scale(1.2)*box((-tri,-d),(tri,d)));
	draw(shift(B)*scale(1.2)*box((-tri,-d),(tri,d)),linewidth(0.8bp));
	draw(O--A--B,linewidth(0.8bp));
	draw(O--tri*dir(-60)--tri*dir(-120)--cycle,linewidth(0.8bp));
	unfill(shift(O)*scale(ro)*unitcircle);
	draw(shift(O)*scale(ri)*unitcircle,linewidth(0.8bp));
	draw(shift(O)*scale(ro)*unitcircle,linewidth(0.8bp));
	unfill(shift(A)*scale(ro)*unitcircle);
	draw(shift(A)*scale(ri)*unitcircle,linewidth(0.8bp));
	draw(shift(A)*scale(ro)*unitcircle,linewidth(0.8bp));
	unfill(shift(B)*scale(ro)*unitcircle);
	draw(shift(B)*scale(ri)*unitcircle,linewidth(0.8bp));
	draw(shift(B)*scale(ro)*unitcircle,linewidth(0.8bp));
	label("$O$",O,NW);
	label("$A$",A,2*N);
	label("$B$",B,4*N);
	dot((A+B)/2);
	label("$M$",(A+B)/2,NE);
	draw(Label("$\phi=\omega t$",MidPoint,Relative(E)),arc(O,5*ro,0,phi),Arrow);
\end{asy}
\caption{题\thequestion}
\label{86页11.4}
\end{figure}
\end{question}
\begin{solution}
根据几何关系,可以求出$A$点、$B$点和$M$点的坐标
\begin{align*}
	& x_A = a \cos \omega t,\quad y_A = a \sin \omega t \\
	& x_B = 2a \cos \omega t,\quad y_B = 0 \\
	& x_M = \frac32 a \cos \omega t,\quad y_M = \frac12 a \sin \omega t
\end{align*}
由此可有$B$点和$M$点的速度
\begin{align*}
	& \dot{x}_B = -2a \omega \sin \omega t,\quad \dot{y}_B = 0 \\
	& \dot{x}_M = -\frac32 a \omega \sin \omega t,\quad \dot{y}_M = \frac12 a \omega \cos \omega t
\end{align*}
即
\begin{align*}
	& v_B = \sqrt{\dot{x}_B^2 + \dot{y}_B^2} = 2a \omega \sin \omega t \\
	& v_M = \sqrt{\dot{x}_M^2 + \dot{y}_M^2} = \frac12 a\omega \sqrt{8\sin^2 \omega t+1}
\end{align*}
\end{solution}

\begin{question}[91页12.19]
如图\ref{91页12.19}所示,在半径为$r$的铁丝圈上套有一个小环$M$,杆$OA$穿过小环并绕铁丝圈上的点$O$匀速转动,角速度为$\omega$。求小环$M$的速度$v$和加速度$a$。

\begin{figure}[htb]
\centering
\begin{asy}
	size(200);
	//91页12.19
	picture tmp;
	pair O,C,M;
	real phi,r,ro,ri;
	C = (0,0);
	phi = 55;
	r = 1;
	ri = 0.02;
	ro = 0.04;
	O = r*dir(180);
	M = r*dir(phi);
	draw(scale(r)*unitcircle,linewidth(0.8bp));
	draw(O--O+0.5*dir(90));
	draw(Label("$A$",EndPoint),O--interp(O,M,1.3),linewidth(0.8bp));
	unfill(shift(O)*scale(ro)*unitcircle);
	draw(shift(O)*scale(ri)*unitcircle,linewidth(0.8bp));
	draw(shift(O)*scale(ro)*unitcircle,linewidth(0.8bp));
	unfill(shift(M)*scale(ro)*unitcircle);
	draw(shift(M)*scale(ro)*unitcircle,linewidth(0.8bp));
	draw(tmp,arc(C,r,0,phi),linewidth(0.8bp));
	clip(tmp,shift(M)*scale(ro)*unitcircle);
	add(tmp);
	label("$O$",O,W);
	label("$M$",M,2*N);
	draw(Label("$\phi$",Relative(0.7),Relative(W)),arc(O,9*ro,90,degrees(M-O)),Arrow);
	draw(O--C--M,dashed);
	draw(Label("$2\phi$",MidPoint,Relative(E)),arc(C,3*ro,phi,180),Arrows);
\end{asy}
\caption{题\thequestion}
\label{91页12.19}
\end{figure}
\end{question}

\begin{solution}
以$O$为原点建立坐标系,可得$M$点的坐标
\begin{equation*}
\begin{cases}
	x_M = r+r\cos(\pi-2\phi) = r-r\cos 2\phi = r-r\cos 2\omega t \\
	y_M = r\sin(\pi-2\phi) = r\sin 2\phi = r\sin 2\omega t
\end{cases}
\end{equation*}
因此可有
\begin{equation*}
\begin{cases}
	\dot{x}_M = 2r\omega \sin 2\omega t \\
	\dot{y}_M = 2r\omega \cos 2\omega t
\end{cases},\quad 
\begin{cases}
	\ddot{x}_M = 4r\omega^2 \cos 2\phi \\
	\ddot{y}_M = -4r\omega^2 \sin 2\phi
\end{cases}
\end{equation*}
因此,可有
\begin{equation*}
	v = \sqrt{\dot{x}_M^2 + \dot{y}_M^2} = 2r\omega,\quad a = \sqrt{\ddot{x}_M^2 + \ddot{y}_M^2} = 4r\omega^2 
\end{equation*}
实际上,小环$M$在以$2\omega$的角速度做圆周运动。
\end{solution}

\begin{question}[91页12.31]
如图\ref{91页12.31}所示,杆$AB$的一段$A$以匀速$v_A$沿直线导轨$CD$移动,杆$AB$始终穿过一个与导轨$CD$相距为$a$的可转动套筒$O$。杆$AB$上点$M$至滑块$A$的距离是$b$。取$O$为极点,试用极坐标$r,\phi$表示点$M$的速度和加速度。

\begin{figure}[htb]
\centering
\begin{asy}
	size(250);
	//91页12.31
	picture dashpic;
	pair O,A,M,dash,P;
	real phi,a,b,l1,l2,d,dd,r,tri;
	path tre;
	O = (0,0);
	phi = 60;
	a = 1;
	b = 0.8;
	l1 = a/Cos(phi);
	l2 = 1.1;
	d = 0.05;
	dd = 1.5*d;
	r = 0.04;
	dash = dir(-135);
	tre = a*dir(90)--a*dir(-90);
	draw(dashpic,tre,linewidth(1bp));
	for(real r=0;r<=1;r=r+0.02){
		P = relpoint(tre,r);
		draw(dashpic,P--P+dash);
	}
	draw(Label("$x$",EndPoint),O--1.5*a*dir(0),Arrow);
	draw(Label("$D$",BeginPoint,Relative(W)),a*dir(0)+a*dir(-90)--a*dir(0)+1.3*l1*dir(90));
	label("$C$",a*dir(0)+1.3*l1*dir(90),W);
	draw(rotate(phi)*box((-l2,-d),(l1,d)),linewidth(0.8bp));
	draw(Label("$B$",EndPoint),l1*dir(phi)--1.05*l2*dir(phi-180),dashed);
	draw(O--0.4*a*dir(-90));
	M = (l1-b)*dir(phi);
	A = l1*dir(phi);
	draw(M--M+0.4*a*dir(phi+90));
	draw(A--A+0.4*a*dir(phi+90));
	draw(Label("$\boldsymbol{v}_A$",EndPoint,Relative(E)),A--A+0.5*dir(90),Arrow);
	tri = 0.23;
	draw(O--tri*dir(180-30)--tri*dir(180+30)--cycle,linewidth(0.8bp));
	clip(dashpic,box((-2*d,-tri),(1,tri)));
	add(shift(tri*Sin(60)*dir(180))*dashpic);
	unfill(rotate(phi)*box((-1.8*dd,-dd),(1.8*dd,dd)));
	draw(rotate(phi)*box((-1.8*dd,-dd),(1.8*dd,dd)),linewidth(0.8bp));
	draw(shift(O)*scale(r)*unitcircle,linewidth(0.8bp));
	unfill(shift(A)*rotate(90)*box((-1.8*dd,-dd),(1.8*dd,dd)));
	draw(shift(A)*rotate(90)*box((-1.8*dd,-dd),(1.8*dd,dd)),linewidth(0.8bp));
	draw(shift(A)*scale(r)*unitcircle,linewidth(0.8bp));
	draw(Label("$a$",MidPoint,Relative(E)),0.3*a*dir(-90)--a*dir(0)+0.3*a*dir(-90),Arrows);
	draw(Label("$b$",MidPoint,Relative(W)),M+0.3*a*dir(phi+90)--A+0.3*a*dir(phi+90),Arrows);
	label("$A$",A,2*E);
	dot(M);
	label("$M$",M,SE);
	label("$O$",O,2.5*dir(110));
	r = 0.4;
	draw(Label("$\phi$",MidPoint,Relative(E)),arc(O,r,0,phi),Arrow);
\end{asy}
\caption{题\thequestion}
\label{91页12.31}
\end{figure}
\end{question}
\begin{solution}
首先计算$A$点的坐标为
\begin{equation*}
	x_A = a,\quad y_A = a\tan \phi
\end{equation*}
由此有
\begin{equation*}
	v_A = \dot{y}_A = a\sec^2 \phi \dot{\phi}
\end{equation*}
所以
\begin{equation*}
	\dot{\phi} = \frac{v_A}{a} \cos^2 \phi
\end{equation*}
点$M$的极坐标$r = a\sec \phi - b$。因此,由式\eqref{第一章:柱坐标系中的速度}可得极坐标下的速度分量
\begin{equation*}
\begin{cases}
	v_r = \dot{r} = a \dot{\phi} \sec \phi \tan \phi = v_A \sin \phi \\
	v_\phi = r \dot{\phi} = \dfrac{v_A r}{a} \cos^2 \phi
\end{cases}
\end{equation*}
因此,$M$点的速度为
\begin{equation*}
	v = \sqrt{v_r^2+v_\phi^2} = \frac{v_A}{a}\sqrt{a^2 \sin^2 \phi + r^2 \cos^4 \phi}
\end{equation*}
由式\eqref{第一章:柱坐标系中的加速度}可得极坐标下的加速度分量
\begin{equation*}
\begin{cases}
	\displaystyle a_r = \ddot{r}-r\dot{\phi}^2 = v_A\dot{\phi} \cos \phi - r\dot{\phi}^2 = \frac{b v_A^2}{a^2} \cos^4 \phi \\[1.5ex]
	\displaystyle a_\phi = r\ddot{\phi} + 2\dot{r}\dot{\phi} = \frac{2bv_A^2}{a^2}\sin \phi \cos^3 \phi
\end{cases}
\end{equation*}
因此,$M$点的加速度为
\begin{equation*}
	a = \sqrt{a_r^2 + a_\phi^2} = \frac{b v_A^2}{a^2} \cos^3 \phi \sqrt{\cos^2\phi + 4\sin^2 \phi} = \frac{b v_A^2}{a^2} \cos^3 \phi \sqrt{1 + 3\sin^2 \phi}
\end{equation*}
\end{solution}

\subsection{刚体的运动学}

\begin{question}[100页14.8]
如图\ref{100页14.8}所示,半长轴和半转轴分别为$a$和$b$的椭圆齿轮副中,齿轮$I$的角速度$\omega_1$为常数,求此椭圆齿轮副的传动规律。已知两齿轮轴间的距离是$O_1O_2=2a$,$\phi$是两转动轴的联机与椭圆齿轮$I$的长轴之间的夹角。两转动轴各自通过椭圆的一个焦点。

\begin{figure}[htb]
\centering
\begin{asy}
	size(300);
	//100页14.8
	picture ell,tmp,dashpic;
	pair O,O1,O2,C,M,dash,P;
	real phi,theta,a,b,c,e,l,p,r1,tri,ri,ro,d;
	path tre;
	O = (0,0);
	phi = 35;
	a = 3;
	b = 1.5;
	c = sqrt(a**2-b**2);
	e = c/a;
	p = b**2/a;
	l = 2*a;
	tri = 0.3;
	ri = 0.05;
	ro = 0.1;
	d = 0.112;
	dash = dir(-135);
	tre = a*dir(180)--a*dir(0);
	draw(dashpic,tre,linewidth(1.5bp));
	for(real r=0;r<=1;r=r+0.015){
		P = relpoint(tre,r);
		draw(dashpic,P--P+dash);
	}
	O1 = (-c,0);
	draw(tmp,O--tri*dir(-60)--tri*dir(-120)--cycle,linewidth(0.8bp));
	clip(dashpic,box((-tri,-d),(tri,1)));
	add(tmp,shift(tri*Sin(60)*dir(-90))*dashpic);
	add(shift(O1)*rotate(180)*tmp);
	draw(ell,xscale(a)*yscale(b)*unitcircle,linewidth(0.8bp));
	draw(ell,1.1*a*dir(180)--1.1*a*dir(0),dashed);
	add(rotate(phi,O1)*ell);
	r1 = p/(1-e*Cos(phi));
	M = O1+r1*dir(0);
	C = O1+2c*dir(phi);
	draw(O1--M--C,dashed);
	unfill(shift(O1)*scale(ro)*unitcircle);
	draw(shift(O1)*scale(ro)*unitcircle,linewidth(0.8bp));
	draw(shift(O1)*scale(ri)*unitcircle,linewidth(0.8bp));
	unfill(shift(C)*scale(ri)*unitcircle);
	draw(shift(C)*scale(ri)*unitcircle,linewidth(0.8bp));
	draw(Label("$\omega_1$",MidPoint,Relative(W)),arc(O1,10*ro,85,phi+5),Arrow);
	draw(Label("$\phi$",MidPoint,Relative(W)),arc(O1,10*ro,phi,0),Arrows);
	label("$r_1$",O1+0.5*r1*dir(0),S);
	label("$I$",O1+c*dir(phi),dir(phi+90));
	label("$N_1$",O1+(1.1*a-c)*dir(phi-180),dir(phi-180));
	label("$M_1$",O1+(1.1*a+c)*dir(phi),dir(phi));
	label("$O_1$",O1,SE);
	label("$O'_1$",C,WNW);
	O2 = O1+(l,0);
	theta = aCos((1-p/(l-r1))/e);
	C = O2+2*c*dir(theta-180);
	add(shift(O2)*tmp);
	add(rotate(theta,O2)*shift((l-2*c)*dir(0))*ell);
	draw(O2--M--C,dashed);
	unfill(shift(O2)*scale(ro)*unitcircle);
	draw(shift(O2)*scale(ro)*unitcircle,linewidth(0.8bp));
	draw(shift(O2)*scale(ri)*unitcircle,linewidth(0.8bp));
	unfill(shift(C)*scale(ri)*unitcircle);
	draw(shift(C)*scale(ri)*unitcircle,linewidth(0.8bp));
	unfill(shift(M)*scale(ri)*unitcircle);
	draw(shift(M)*scale(ri)*unitcircle,linewidth(0.8bp));
	draw(Label("$\omega_2$",EndPoint),arc(O2,10*ro,180-15,180+theta+5),Arrow);
	label("$M$",M,SE);
	label("$r_2$",O2+0.5*(l-r1)*dir(180),N);
	label("$II$",O2+c*dir(theta-180),dir(theta-90));
	label("$N_2$",O2+(1.1*a+c)*dir(theta-180),dir(theta-180));
	label("$M_2$",O2+(1.1*a-c)*dir(theta),dir(theta));
	label("$O_2$",O2,NW);
	label("$O'_2$",C,ESE);
\end{asy}
\caption{题\thequestion}
\label{100页14.8}
\end{figure}
\end{question}
\begin{solution}
以$O_1$为极点,$O_1M_1$方向为极径建立极坐标系,则在此坐标系中,椭圆$I$的方程可以表示为
\begin{equation*}
	r_1 = \frac{p}{1-e\cos \xi}
\end{equation*}
其中$p = \dfrac{a^2-c^2}{a}$为半通径,$e=\dfrac{c}{a}$为离心率,$c=\sqrt{a^2-b^2}$为椭圆的焦距,$\xi$为极角。则椭圆的弧长可以表示为
\begin{equation*}
	s = \int_0^\phi \sqrt{\left(\frac{\mathrm{d} r_1}{\mathrm{d} \xi}\right)^2+r_1^2} \mathrm{d} \xi = \int_0^\phi p \frac{\sqrt{1-2e\cos\xi+e^2}}{(1-e\cos\xi)^2} \mathrm{d} \xi
\end{equation*}
由此可得$M$点的速度为
\begin{equation*}
	v_M = \dot{s} = p \frac{\sqrt{1-2e\cos\phi+e^2}}{(1-e\cos\phi)^2} \dot{\phi} = p \frac{\sqrt{1-2e\cos\phi+e^2}}{(1-e\cos\phi)^2} \omega_1
\end{equation*}
同样地,以$O_2$为极点,$O_2N_2$方向为极径建立极坐标系,则在此坐标系中,椭圆$II$的方程可以表示为
\begin{equation*}
	r_2 = \frac{p}{1-e\cos \xi}
\end{equation*}
记$O_2N_2$与$O_1O_2$的夹角为$\theta$,则在此坐标系中可得椭圆的弧长可以表示为
\begin{equation*}
	s = \int_{-\theta}^0 p \frac{\sqrt{1-2e\cos\xi+e^2}}{(1-e\cos\xi)^2} \mathrm{d} \xi = -\int_0^{-\theta} p \frac{\sqrt{1-2e\cos\xi+e^2}}{(1-e\cos\xi)^2} \mathrm{d} \xi
\end{equation*}
由此可得$M$点的速度为
\begin{equation*}
	v_M = \dot{s} = -p \frac{\sqrt{1-2e\cos\theta+e^2}}{(1-e\cos\theta)^2} (-\dot{\theta}) = p \frac{\sqrt{1-2e\cos\theta+e^2}}{(1-e\cos\theta)^2} \omega_2
\end{equation*}
由此可得椭圆齿轮副之间的传动规律为
\begin{equation*}
	p \frac{\sqrt{1-2e\cos\phi+e^2}}{(1-e\cos\phi)^2} \omega_1 = p \frac{\sqrt{1-2e\cos\theta+e^2}}{(1-e\cos\theta)^2} \omega_2
\end{equation*}
即
\begin{equation}
	\frac{\omega_1}{\omega_2} = \sqrt{\frac{1-2e\cos\phi+e^2}{1-2e\cos\theta+e^2}} \left(\frac{1-e\cos\theta}{1-e\cos\phi}\right)^2 = \sqrt{\frac{a^2-2ac\cos \phi + c^2}{a^2-2ac\cos \theta + c^2}} \left(\frac{a-c\cos\theta}{a-c\cos\phi}\right)^2
	\label{14.8-1}
\end{equation}
角度$\theta$和$\phi$的关系由$O_1O_2$的长度来确定,即
\begin{equation*}
	2a = r_1+r_2 = \frac{p}{1-e\cos \phi} + \frac{p}{1-e\cos \theta}
\end{equation*}
所以有
\begin{equation}
	\cos \theta = \frac{2ac-(a^2+c^2)\cos \phi}{a^2-2ac\cos \phi + c^2}
	\label{14.8-2}
\end{equation}
将式\eqref{14.8-2}代入传动规律\eqref{14.8-1}中,可得此椭圆齿轮副的传动关系为
\begin{equation*}
	\frac{\omega_1}{\omega_2} = \frac{a^2-c^2}{a^2-2ac\cos\phi+c^2}
\end{equation*}
\end{solution}

\begin{question}[102页14.13]
如图\ref{102页14.13}所示,偏心轮的直径$d=2r$,转轴$O$到轮心$C$的距离$OC=a$。轴$Ox$沿着推杆的方向,参考原点在转轴$O$上。求推杆的运动规律。

\begin{figure}[htb]
\centering
\begin{asy}
	size(250);
	//102页14.13
	picture dashpic;
	pair O,A,C,dash,P;
	real phi,theta,a,r,l,ll,tri,ri,ro,d;
	path tre;
	O = (0,0);
	phi = 60;
	a = 1;
	r = 1.4;
	l = 2.2;
	ll = 0.7*l;
	d = 0.07;
	tri = 0.3;
	ri = 0.05;
	ro = 0.1;
	dash = dir(-135);
	tre = a*dir(180)--a*dir(0);
	draw(dashpic,tre,linewidth(1bp));
	for(real r=0;r<=1;r=r+0.04){
		P = relpoint(tre,r);
		draw(dashpic,P--P+dash);
	}
	C = a*dir(90-phi);
	draw(Label("$x$",EndPoint),O--(2*a+l)*dir(90),dashed);
	draw(Label("$a$",MidPoint,Relative(E)),O--C);
	draw(shift(C)*scale(r)*unitcircle,linewidth(0.8bp));
	draw(O--tri*dir(-60)--tri*dir(-120)--cycle,linewidth(0.8bp));
	clip(dashpic,box((-tri,-1.6*d),(tri,1)));
	add(shift(tri*Sin(60)*dir(-90))*dashpic);
	theta = aSin(a/(r+ro)*Sin(phi));
	A = C+(r+ro)*dir(theta+90);
	draw(Label("$r$",MidPoint,Relative(W)),A--C,dashed);
	draw(shift(A)*box((-d,0),(d,l)),linewidth(0.8bp));
	unfill(shift(A)*scale(ro)*unitcircle);
	draw(shift(A)*scale(ro)*unitcircle,linewidth(0.8bp));
	draw(shift(A)*scale(ri)*unitcircle,linewidth(0.8bp));
	unfill(shift(C)*scale(ro)*unitcircle);
	draw(shift(C)*scale(ro)*unitcircle,linewidth(0.8bp));
	draw(shift(C)*scale(ri)*unitcircle,linewidth(0.8bp));
	unfill(shift(O)*scale(ro)*unitcircle);
	draw(shift(O)*scale(ro)*unitcircle,linewidth(0.8bp));
	draw(shift(O)*scale(ri)*unitcircle,linewidth(0.8bp));
	tri = 0.7*tri;
	clip(dashpic,box((-tri,-1.6*d),(tri,1)));
	draw(dashpic,(-tri,0)--(-tri,-1.6*d),linewidth(1bp));
	draw(dashpic,(tri,0)--(tri,-1.6*d),linewidth(1bp));
	add(shift(A+ll*dir(90)+1.5*d*dir(0))*rotate(90)*dashpic);
	add(shift(A+ll*dir(90)+1.5*d*dir(180))*rotate(-90)*dashpic);
	label("$O$",O,2*W);
	label("$C$",C,2*E);
	label("$A$",A,2*W);
	r = 0.4;
	draw(Label("$\phi$",MidPoint,Relative(E)),arc(O,r,90-phi,90),Arrows);
\end{asy}
\caption{题\thequestion}
\label{102页14.13}
\end{figure}
\end{question}
\begin{solution}
推杆的坐标可用$A$点的坐标$x$表示,记$\angle OAC = \theta$,则在$\triangle OAC$中根据正弦定理可有
\begin{equation*}
	\frac{a}{\sin\theta} = \frac{r}{\sin \phi}
\end{equation*}
即有
\begin{equation*}
	\sin \theta = \frac{a}{r} \sin \phi,\quad \cos \theta = \sqrt{1-\sin^2 \theta} = \sqrt{1-\frac{a^2}{r^2}\sin^2 \phi}
\end{equation*}
所以有
\begin{equation*}
	x = a\cos \phi + r\cos \theta = a\cos \phi + r\sqrt{1-\frac{a^2}{r^2}\sin^2 \phi} = a\cos \phi + \sqrt{r^2 - a^2 \sin^2 \phi}
\end{equation*}
\end{solution}

\begin{question}[106页15.6]
如图\ref{106页15.6}所示,沿直线导轨滑动的套筒$A$和$B$与长为$l$的连杆$AB$连接,套筒$A$以匀速$v_A$运动。设套筒$A$是从$O$点开始运动的,以$A$为基点,写出杆$AB$的运动方程。其中$\angle BOA = \pi-\alpha$。

\begin{figure}[htb]
\centering
\begin{asy}
	size(250);
	//106页15.6
	picture dashpic,tmp;
	pair O,A,B,dash,P;
	real alpha,phi,a,b,l,d,d1,d2,r,r0,dl1,dl2;
	path tre;
	O = (0,0);
	alpha = 50;
	a = 2;
	l = 3;
	b = sqrt(l**2-(a*Sin(alpha))**2)-a*Cos(alpha);
	phi = aSin(a/l*Sin(alpha));
	A = a*dir(180-alpha);
	B = b*dir(0);
	draw(Label("$x$",EndPoint,Relative(E)),(1.5*A.x,0)--2.3*B);
	draw(Label("$y$",EndPoint,Relative(E)),O--(0,1.5*A.y));
	draw(1.5*A--O--2*B,linewidth(1bp));
	d = 0.07;
	d1 = 0.2;
	d2 = 0.1;
	r = d;
	unfill(shift(A)*rotate(180-alpha)*box((-d1,-d2),(d1,d2)));
	draw(shift(A)*rotate(180-alpha)*box((-d1,-d2),(d1,d2)),linewidth(0.8bp));
	unfill(shift(B)*box((-d1,-d2),(d1,d2)));
	draw(shift(B)*box((-d1,-d2),(d1,d2)),linewidth(0.8bp));
	unfill(shift(B)*rotate(-phi)*box((-l,-d),(0,d)));
	draw(shift(B)*rotate(-phi)*box((-l,-d),(0,d)),linewidth(0.8bp));
	unfill(shift(A)*scale(r)*unitcircle);
	draw(shift(A)*scale(r)*unitcircle,linewidth(0.8bp));
	unfill(shift(B)*scale(r)*unitcircle);
	draw(shift(B)*scale(r)*unitcircle,linewidth(0.8bp));
	draw(A--interp(A,B,1.2),dashed);
	r0 = 0.4;
	draw(Label("$\phi$",MidPoint,Relative(W)),arc(B,r0,0,-phi),Arrows);
	r0 = 0.8;
	draw(Label("$\alpha$",MidPoint,Relative(E)),arc(O,r0,180-alpha,180),Arrows);

	dl2 = 0.18;
	dash = dl2*dir(-135);
	tre = O--l*dir(0);
	//draw(dashpic,tre,linewidth(1bp));
	for(real rp=0;rp<=1;rp=rp+0.02){
		P = relpoint(tre,rp);
		draw(dashpic,P--P+dash);
	}
	dl1 = 0.4*b;
	add(tmp,dashpic);
	clip(tmp,box((0,1),(dl1,-dl2)));
	add(shift(2*B-dl1*dir(0))*tmp);
	dl1 = 0.25*a;
	clip(tmp,box((0,1),(dl1,-dl2)));
	add(shift(1.5*A)*rotate(-alpha)*tmp);
	erase(tmp);
	erase(dashpic);
	tre = r0*dir(A)--O--r0*dir(B);
	for(real rp=0;rp<=1;rp=rp+0.04){
		P = relpoint(tre,rp);
		draw(dashpic,P--P+dash);
	}
	clip(dashpic,box(0.6*r0*dir(A)+2*dl2*dir(180),0.7*r0*dir(B)+dl2*dir(-90)));
	add(dashpic);
	label("$O$",O,2*S);
	label("$A$",A,2*NNE);
	label("$B$",B,2*NNE);
\end{asy}
\caption{题\thequestion}
\label{106页15.6}
\end{figure}
\end{question}
\begin{solution}
点$A$的坐标为
\begin{equation*}
	x_A = -v_A t \cos \alpha,\quad y_A = v_A t \sin \alpha
\end{equation*}
在$\triangle OAB$中应用正弦定理可得
\begin{equation*}
	\frac{l}{\sin(\pi-\alpha)} = \frac{v_A t}{\sin \phi}
\end{equation*}
由此可得
\begin{equation*}
	\phi = \arcsin\left(\frac{v_A t}{l} \sin \alpha\right)
\end{equation*}
\end{solution}

\subsection{质点动力学}

\begin{question}[209页31.28]
如图\ref{209页31.28}所示,质点$A$在重力作用下沿粗糙的螺旋面运动。螺旋面的轴$Oz$是铅垂的,方程为$z = a\phi+f(r)$。螺旋面与质点的摩擦系数为$k$,设$a$为常数。在什么条件下,质点在运动时,保持到轴$Oz$的距离$AB = r_0$不变,即质点将沿螺旋线运动?并求质点的速度。

\begin{figure}[htb]
\centering
\begin{asy}
	size(350);
	//209页31.28
	picture tmp;
	pair O,i,j,k,A,B,tau,n,dirtau,dirn,ctau;
	real a,h,h0,r,theta0,thetam;
	path clp1,clp2;
	O = (0,0);
	i = (-sqrt(2)/4,-sqrt(14)/12);
	j = (sqrt(14)/4,-sqrt(2)/12);
	k = (0,2*sqrt(2)/3);
	pair topcir(real theta){
		return r*cos(theta)*i+r*sin(theta)*j+h0*k;
	}
	pair botcir(real theta){
		return r*cos(theta)*i+r*sin(theta)*j;
	}
	pair helix(real theta){
		return r*cos(theta)*i+r*sin(theta)*j+(a*theta)*k;
	}
	r = 1;
	h = 3;
	h0 = h;
	draw(Label("$x$",EndPoint),O--2*r*i,Arrow);
	draw(Label("$y$",EndPoint),O--1.5*r*j,Arrow);
	draw(Label("$z$",EndPoint),O--1.2*h*k,Arrow);
	clp1 = graph(botcir,0,2*pi)--cycle;
	clp2 = box(r*dir(180),r*dir(0)+h*k);
	unfill(clp1);
	unfill(clp2);
	draw(tmp,O--2*r*i,dashed);
	draw(tmp,O--1.5*r*j,dashed);
	draw(tmp,O--1.2*h*k,dashed);
	clip(tmp,clp1);
	add(tmp);
	erase(tmp);
	draw(tmp,O--2*r*i,dashed);
	draw(tmp,O--1.5*r*j,dashed);
	draw(tmp,O--1.2*h*k,dashed);
	unfill(tmp,clp1);
	clip(tmp,clp2);
	add(tmp);
	erase(tmp);
	draw(graph(topcir,0,2*pi),linewidth(1bp));
	theta0 = 1.9;
	draw(graph(botcir,theta0,theta0+pi),dashed);
	draw(graph(botcir,theta0+pi,theta0+2*pi),linewidth(1bp));
	draw(r*dir(0)--r*dir(0)+h*k,linewidth(1bp));
	draw(r*dir(180)--r*dir(180)+h*k,linewidth(1bp));
	h0 = 1.5;
	draw(graph(topcir,0,2*pi),dashed);
	a = 0.208;
	thetam = atan(j.x/i.x)+pi;
	//draw(graph(helix,0,h/a),linewidth(1bp));
	draw(graph(helix,0,thetam),linewidth(1bp));
	draw(graph(helix,thetam,thetam+pi),dashed);
	thetam = thetam+pi;
	draw(graph(helix,thetam,thetam+pi),linewidth(1bp));
	thetam = thetam+pi;
	draw(graph(helix,thetam,thetam+pi),dashed);
	thetam = thetam+pi;
	draw(graph(helix,thetam,h/a),linewidth(1bp));
	theta0 = h0/a;
	A = helix(theta0);
	B = h0*k;
	dot(A);
	label("$A$",A,SE);
	label("$B$",B,W);
	draw(A--B,dashed);
	tau = -r*sin(theta0)*i+r*cos(theta0)*j+a*k;
	n = -r*cos(theta0)*i-r*sin(theta0)*j;
	dirtau = tau/(sqrt(r**2+a**2));
	dirn = n/r;
	ctau = -r*sin(theta0)*i+r*cos(theta0)*j;
	draw(Label("$\boldsymbol{\tau}$",EndPoint),A--A+0.8*tau,Arrow);
	draw(A--A+0.8*ctau,dashed);
	draw(Label("$\boldsymbol{n}$",EndPoint,SSW),A--A+0.8*n,Arrow);
	draw(Label("$\boldsymbol{N}$",EndPoint,ENE),A--A+0.6*dirtau+1.4*dirn,Arrow);
	draw(Label("$\boldsymbol{v}$",EndPoint),A--A-0.7*tau,Arrow);
	draw(A--A-h0*k--O,dashed);
	draw(Label("$\boldsymbol{g}$",EndPoint,Relative(W)),A--A-0.8*k,Arrow);
	r = 0.2;
	draw(Label("$\beta$",MidPoint,Relative(E)),arc(A,r,degrees(0.6*dirtau+1.4*dirn),degrees(n)));
	r = 0.3;
	draw(arc(A,r,degrees(ctau),degrees(tau)));
	draw(Label("$\alpha$",EndPoint),A+r*dir((degrees(ctau)+degrees(tau))/2)--A+r*dir((degrees(ctau)+degrees(tau))/2)+0.11*dir(-95));
	//draw(A+r*dir((degrees(0.6*dirtau+1.4*dirn)+degrees(n))/2)--A+r*dir((degrees(0.6*dirtau+1.4*dirn)+degrees(n))/2)+0.11*dir(80));
	draw(Label("$\phi$",MidPoint,Relative(E)),graph(botcir,0,theta0-2*pi),Arrows);
\end{asy}
\caption{题\thequestion}
\label{209页31.28}
\end{figure}
\end{question}
\begin{solution}
质点$A$运动的螺旋线的柱坐标方程为
\begin{equation*}
\begin{cases}
	r = r_0 \\
	z = a\phi+f(r)
\end{cases}
\end{equation*}
用直角坐标下的矢量可表示为
\begin{equation*}
	\mbf{r}(\phi) = \begin{pmatrix} r_0 \cos \phi \\ r_0 \sin \phi \\ a\phi + f(r_0) \end{pmatrix}
\end{equation*}
因此,螺旋线的单位切矢量为
\begin{equation*}
	\mbf{\tau}(\phi) = \frac{\mbf{r}'(\phi)}{|\mbf{r}'(\phi)|} = \frac{1}{\sqrt{a^2+r_0^2}} \begin{pmatrix} -r_0 \sin \phi \\ r_0 \cos \phi \\ a \end{pmatrix}
\end{equation*}
主法线单位矢量为
\begin{equation*}
	\mbf{n}(\phi) = \frac{\mbf{r}''(\phi)}{|\mbf{r}''(\phi)|} = \begin{pmatrix} -\cos \phi \\ -\sin \phi \\ 0 \end{pmatrix}
\end{equation*}
副法线单位矢量为
\begin{equation*}
	\mbf{b}(\phi) = \mbf{\tau}(\phi) \times \mbf{n}(\phi) = \frac{1}{\sqrt{a^2+r_0^2}} \begin{pmatrix} a\sin \phi \\ -a\cos \phi \\ r_0 \end{pmatrix}
\end{equation*}
曲率半径为
\begin{equation*}
	\rho(\phi) = \frac{|\mbf{r}'(\phi)|^3}{|\mbf{r}'(\phi) \times \mbf{r}''(\phi)|} = \frac{a^2+r_0^2}{r_0}
\end{equation*}

螺旋面用矢量可以表示为
\begin{equation*}
	\mbf{\varSigma}(r,\phi) = \begin{pmatrix} r\cos \phi \\ r\sin \phi \\ a\phi+f(r) \end{pmatrix}
\end{equation*}
由此可得螺旋面的法线单位矢量
\begin{equation*}
	\mbf{n}_\varSigma = \frac{\dfrac{\pl \mbf{\varSigma}}{\pl r} \times \dfrac{\pl \mbf{\varSigma}}{\pl \phi}}{\left|\dfrac{\pl \mbf{\varSigma}}{\pl r} \times \dfrac{\pl \mbf{\varSigma}}{\pl \phi}\right|} = \frac{1}{\sqrt{a^2+\big[(f'(r))^2+1\big] r^2}} \begin{pmatrix} a\sin\phi-rf'(r)\cos\phi \\ -a\cos\phi-rf'(r)\sin\phi \\ r \end{pmatrix}
\end{equation*}
直接计算可以得到
\begin{align*}
	& \mbf{n}_\varSigma \cdot \mbf{\tau} = 0,\quad \mbf{n}_\varSigma \cdot \mbf{n} = \frac{r_0 f'(r_0)}{\sqrt{a^2+ \big[(f'(r_0))^2+1\big] r_0^2}} = \cos \beta\\ 
	& \mbf{n}_\varSigma \cdot \mbf{b} = \frac{\sqrt{a^2+r_0^2}}{\sqrt{a^2+ \big[(f'(r_0))^2+1\big] r_0^2}} = \sin \beta 
\end{align*}
在$\mbf{\tau},\mbf{n},\mbf{b}$的方向上分解运动方程,记$\tan \alpha = \dfrac{a}{r_0}$,可以得到
\begin{subnumcases}{}
	kN - mg\sin \alpha = ma_\tau \label{31.28-1} \\
	N\cos \beta = m\frac{v^2}{\rho} \label{31.28-2} \\
	N\sin \beta - mg\cos \alpha = 0 \label{31.28-3}
\end{subnumcases}
由式\eqref{31.28-3}可有
\begin{equation*}
	N = \frac{\cos \alpha}{\sin \beta} mg
\end{equation*}
是常数值,因此由\eqref{31.28-2},速度的大小$v$也是常数值,故$a_\tau=0$。由此可得速度的大小为
\begin{equation*}
	v = \sqrt{\frac{\cos \alpha}{\tan \beta} \rho g} = \sqrt{gr_0f'(r_0)}
\end{equation*}
由式\eqref{31.28-1}可得质点沿螺旋线运动的条件为
\begin{equation*}
	\tan \alpha = \frac{k}{\sin \beta} = k\sqrt{\frac{a^2+\big[(f'(r_0))^2+1\big]r_0^2}{a^2+r_0^2}} = k\sqrt{1+(f'(r_0))^2 \cos^2 \alpha}
\end{equation*}
\end{solution}

\begin{question}[215页32.31]
如图\ref{215页32.31}所示,质量为$m$的物体$A$可在水平直线上移动,系有劲度系数为$k$的弹簧,弹簧的另一端固定在点$B$。当$\alpha=\alpha_0$时,弹簧没有变形。求物体微振动频率和周期。

\begin{figure}[htb]
\centering
\begin{asy}
	size(300);
	//215页32.31
	picture dashpic,tmp;
	pair O,A,B,P,dash;
	real alpha,l,lspring,lstart,d1,d2,r0,d,h,ldash;
	int nmax;
	path pdash;
	guide spring,springunit;
	//画阴影
	ldash = 10;
	pdash = ldash*dir(180)--ldash*dir(0);
	dash = ldash*dir(-135);
	draw(dashpic,pdash,linewidth(1bp));
	for(real r=0;r<=1;r=r+0.004){
		P = relpoint(pdash,r);
		draw(dashpic,P--P+dash);
	}
	//画弹簧
	d = 0.2;
	h = 0.1;
	nmax = 12;
	spring = O--(d/2,h/2);
	springunit = (d/2,-3*h/2)--(-d/2,-h/2)--(d/2,h/2);
	for(int i=1;i<nmax;i=i+1){
		spring = spring--shift(2*i*h*dir(90))*springunit;
	}
	spring = spring--(-d/2,nmax*2*h-h/2)--(0,nmax*2*h);
	//画小车和链接的弹簧
	O = (0,0);
	alpha = 35;
	l = 3;
	lstart = 0.2*l;
	lspring = 0.7*l;
	d1 = 0.8;
	d2 = 0.3;
	A = d1*dir(0);
	B = l*dir(180-alpha);
	draw(box(A-(d1,d2),A+(d1,d2)),linewidth(1bp));
	spring = O--shift(lstart*dir(180-alpha))*rotate(90-alpha)*scale(lspring/(nmax*2*h))*spring--B;
	draw(spring,linewidth(1bp));
	label("$A$",A);
	add(tmp,dashpic);
	h = 0.1;
	d = 0.25;
	clip(tmp,box((-d,-h),(d,1)));
	add(shift(B)*rotate(90-alpha)*yscale(-1)*tmp);
	erase(tmp);
	draw(O--l*dir(180),dashed);
	draw(Label("$\alpha$",MidPoint,Relative(W)),arc(O,0.6*lstart,180,180-alpha),Arrow);
	label("$B$",B,2*ENE);
	r0 = 0.07;
	draw(shift(A+(-d1/2,-d2-r0))*scale(r0)*unitcircle,linewidth(1bp));
	draw(shift(A+(d1/2,-d2-r0))*scale(r0)*unitcircle,linewidth(1bp));
	add(tmp,dashpic);
	d = 1.4*d1;
	clip(tmp,box((-d,-h),(d,1)));
	add(shift(A+(0,-d2-2*r0))*tmp);
	erase(tmp);
	draw(Label("$h$",MidPoint,Relative(W)),(B.x,0)--B,dashed);
	label("$O$",(B.x,0),S);
\end{asy}
\caption{题\thequestion}
\label{215页32.31}
\end{figure}
\end{question}
\begin{solution}
体系的广义坐标为$\alpha$,记$B$点的坐标为$(0,h)$,则小车的坐标可以表示为
\begin{equation*}
	x_A = h\cot \alpha
\end{equation*}
由此,体系的Lagrange函数可以表示为
\begin{align*}
	L & = \frac12 m \dot{x}_A^2 - \frac12 k \left(\frac{h}{\sin \alpha} - \frac{h}{\sin \alpha_0}\right)^2 = \frac{mh^2}{2\sin^4 \alpha} \dot{\alpha}^2 - \frac12 k \left(\frac{h}{\sin \alpha} - \frac{h}{\sin \alpha_0}\right)^2
\end{align*}
体系的平衡位置为$\alpha = \alpha_0$,将Lagrange函数在该点附近展开至两阶,可有
\begin{align*}
	L & = \frac{mh^2}{2\sin^4 \alpha_0} \dot{\alpha}^2 - \frac{kh^2 \cos^2 \alpha_0}{2 \sin^4 \alpha_0}(\alpha-\alpha_0)^2 
\end{align*}
由此,系统的运动方程为
\begin{equation*}
	\ddot{\alpha} + \frac{k\cos^2 \alpha_0}{m} (\alpha-\alpha_0) = 0
\end{equation*}
因此,体系微振动的角频率和周期分别为
\begin{equation*}
	\omega = \sqrt{\frac{k\cos^2 \alpha_0}{m}},\quad T = \frac{2\pi}{\omega} = 2\pi \sqrt{\frac{m}{k\cos^2 \alpha_0}}
\end{equation*}
\end{solution}

\begin{question}[216页32.34]
如图\ref{216页32.34}所示,质量为$m$的重物$M$固定在杆上。杆在$O$点用铰链固连,并用三根垂直弹簧与基础相连,三个弹簧的劲度系数分别是$k_1,k_2,k_3$,三弹簧在杆上的联结点到铰链的距离分别是$a_1,a_2,a_3$,重物在杆上的联结点到铰链的距离是$b$。杆平衡时沿水平方向。求重物微振动的频率。

\begin{figure}[htb]
\centering
\begin{asy}
	size(300);
	//216页32.34
	picture dashpic,tmp;
	pair O,A,B,P,dash,Ps,Pe;
	real alpha,a[],b,l[],ls,le,tri,ldash,ddash,ri,ro,M,d,h;
	path pdash;
	int nmax;
	//画阴影
	ldash = 10;
	pdash = ldash*dir(180)--ldash*dir(0);
	dash = ldash*dir(-135);
	draw(dashpic,pdash,linewidth(1bp));
	for(real r=0;r<=1;r=r+0.0025){
		P = relpoint(pdash,r);
		draw(dashpic,P--P+dash);
	}
	//画弹簧
	guide spring(real d,real h,int nmax){
		guide temp,springunit;
		temp = O--(d/2,h/2);
		springunit = (d/2,-3*h/2)--(-d/2,-h/2)--(d/2,h/2);
		for(int i=1;i<nmax;i=i+1){
			temp = temp--shift(2*i*h*dir(90))*springunit;
		}
		temp = temp--(-d/2,nmax*2*h-h/2)--(0,nmax*2*h);
		return temp;
	}
	O = (0,0);
	l[1] = 2;
	l[2] = 0.7;
	alpha = 15;
	a[1] = 1.9;
	a[2] = 1.4;
	a[3] = 0.9;
	ddash = 0.06;
	draw(l[1]*dir(180-alpha)--l[2]*dir(-alpha),linewidth(1bp));
	M = 0.05;
	fill(shift(l[2]*dir(-alpha))*scale(M)*unitcircle);
	label("$M$",l[2]*dir(-alpha),2*S);
	draw(O--0.6*l[2]*dir(90-alpha));
	draw(a[1]*dir(180-alpha)--a[1]*dir(180-alpha)+0.6*l[2]*dir(90-alpha));
	draw(Label("$a_1$",MidPoint,Relative(W)),a[1]*dir(180-alpha)+0.9*0.6*l[2]*dir(90-alpha)--0.9*0.6*l[2]*dir(90-alpha),Arrows);
	draw(a[2]*dir(180-alpha)--a[2]*dir(180-alpha)+0.4*l[2]*dir(90-alpha));
	draw(Label("$a_2$",MidPoint,Relative(W)),a[2]*dir(180-alpha)+0.9*0.4*l[2]*dir(90-alpha)--0.9*0.4*l[2]*dir(90-alpha),Arrows);
	draw(a[3]*dir(180-alpha)--a[3]*dir(180-alpha)+0.2*l[2]*dir(90-alpha));
	draw(Label("$a_3$",MidPoint,Relative(W)),a[3]*dir(180-alpha)+0.9*0.2*l[2]*dir(90-alpha)--0.9*0.2*l[2]*dir(90-alpha),Arrows);
	draw(l[2]*dir(-alpha)--l[2]*dir(-alpha)+0.4*l[2]*dir(90-alpha));
	draw(Label("$b$",MidPoint,Relative(E)),l[2]*dir(-alpha)+0.9*0.4*l[2]*dir(90-alpha)--0.9*0.4*l[2]*dir(90-alpha),Arrows);
	ro = 0.05;
	ri = 0.025;
	tri = 0.12;
	unfill(shift(O)*scale(ro)*unitcircle);
	draw(O--tri*dir(-60)--tri*dir(-120)--cycle,linewidth(1bp));
	unfill(shift(O)*scale(ri)*unitcircle);
	draw(shift(O)*arc(O,ro,-alpha,180-alpha),linewidth(1bp));
	draw(shift(O)*scale(ri)*unitcircle,linewidth(1bp));
	add(tmp,dashpic);
	clip(tmp,box((-1.3*tri,-ddash),(1.3*tri,1)));
	add(shift(tri*Sin(60)*dir(-90))*tmp);
	label("$O$",O,2*ENE);
	l[3] = 1;
	erase(tmp);
	add(tmp,dashpic);
	clip(tmp,box((-0.5*a[2],-ddash),(0.5*a[2],1)));
	add(shift(a[2]*dir(180)+l[3]*dir(-90))*tmp);
	unfill(shift(a[1]*dir(180-alpha))*scale(ro)*unitcircle);
	unfill(shift(a[2]*dir(180-alpha))*scale(ro)*unitcircle);
	unfill(shift(a[3]*dir(180-alpha))*scale(ro)*unitcircle);
	draw(shift(a[1]*dir(180-alpha))*arc(O,ro,-alpha,180-alpha),linewidth(1bp));
	draw(shift(a[2]*dir(180-alpha))*arc(O,ro,-alpha,180-alpha),linewidth(1bp));
	draw(shift(a[3]*dir(180-alpha))*arc(O,ro,-alpha,180-alpha),linewidth(1bp));
	Ps = a[1]*dir(180-alpha);
	Pe = a[1]*dir(180)+l[3]*dir(-90);
	ls = 0.1;
	le = 0.1;
	nmax = 8;
	h = (length(Ps-Pe)-ls-le)/2/nmax;
	d = 0.1;
	draw(Pe--shift(Pe+le*dir(Ps-Pe))*rotate(degrees(Ps-Pe)-90)*spring(d,h,nmax)--Ps,linewidth(1bp));
	label("$k_1$",(Pe+Ps)/2,2*W);
	Ps = a[2]*dir(180-alpha);
	Pe = a[2]*dir(180)+l[3]*dir(-90);
	ls = 0.1;
	le = 0.1;
	nmax = 6;
	h = (length(Ps-Pe)-ls-le)/2/nmax;
	d = 0.1;
	draw(Pe--shift(Pe+le*dir(Ps-Pe))*rotate(degrees(Ps-Pe)-90)*spring(d,h,nmax)--Ps,linewidth(1bp));
	label("$k_2$",(Pe+Ps)/2,2*W);
	Ps = a[3]*dir(180-alpha);
	Pe = a[3]*dir(180)+l[3]*dir(-90);
	ls = 0.1;
	le = 0.1;
	nmax = 14;
	h = (length(Ps-Pe)-ls-le)/2/nmax;
	d = 0.1;
	draw(Pe--shift(Pe+le*dir(Ps-Pe))*rotate(degrees(Ps-Pe)-90)*spring(d,h,nmax)--Ps,linewidth(1bp));
	label("$k_3$",(Pe+Ps)/2,2*W);
	unfill(shift(a[1]*dir(180-alpha))*scale(ri)*unitcircle);
	unfill(shift(a[2]*dir(180-alpha))*scale(ri)*unitcircle);
	unfill(shift(a[3]*dir(180-alpha))*scale(ri)*unitcircle);
	draw(shift(a[1]*dir(180-alpha))*scale(ri)*unitcircle,linewidth(1bp));
	draw(shift(a[2]*dir(180-alpha))*scale(ri)*unitcircle,linewidth(1bp));
	draw(shift(a[3]*dir(180-alpha))*scale(ri)*unitcircle,linewidth(1bp));
	unfill(shift(a[1]*dir(180)+l[3]*dir(-90))*scale(ri)*unitcircle);
	unfill(shift(a[2]*dir(180)+l[3]*dir(-90))*scale(ri)*unitcircle);
	unfill(shift(a[3]*dir(180)+l[3]*dir(-90))*scale(ri)*unitcircle);
	draw(shift(a[1]*dir(180)+l[3]*dir(-90))*scale(ri)*unitcircle,linewidth(1bp));
	draw(shift(a[2]*dir(180)+l[3]*dir(-90))*scale(ri)*unitcircle,linewidth(1bp));
	draw(shift(a[3]*dir(180)+l[3]*dir(-90))*scale(ri)*unitcircle,linewidth(1bp));
\end{asy}
\caption{题\thequestion}
\label{216页32.34}
\end{figure}
\end{question}
\begin{solution}
体系的广义坐标为杆与水平方向的夹角$\theta$,在此广义坐标下,系统的Lagrange函数为
\begin{align*}
	L = \frac12 m(b\dot{\theta})^2 - \left(\frac12 k_1 (a_1\sin \theta)^2 + \frac12 k_2 (a_2\sin \theta)^2 + \frac12 k_3 (a_3\sin \theta)^2\right)
\end{align*}
保留到二阶小量的情形下,系统的Lagrange函数为
\begin{align*}
	L = \frac12 mb^2 \dot{\theta}^2 - \frac12 \left(k_1 a_1^2 + k_2 a_2^2 + k_3 a_3^2\right) \theta^2
\end{align*}
此时系统的运动方程为
\begin{equation*}
	\ddot{\theta} + \frac{k_1 a_1^2 + k_2 a_2^2 + k_3 a_3^2}{b^2} \theta = 0
\end{equation*}
因此重物微振动的角频率为
\begin{equation*}
	\omega = \sqrt{\frac{k_1 a_1^2 + k_2 a_2^2 + k_3 a_3^2}{mb^2}}
\end{equation*}
\end{solution}


\backmatter
 
\titleformat{\chapter}[hang]{\hwyk\LARGE}
{}{0mm}{\hspace{-0.4cm}\mybackmatter}

\addcontentsline{toc}{chapter}{\listfigurename}{%
\let\oldnumberline\numberline%
\renewcommand{\numberline}{\figurename~\oldnumberline}%
\listoffigures
}

%参考文献
\begin{thebibliography}{9}
\addcontentsline{toc}{part}{\bibname}

\bibitem{金尚年等-理论力学-2004}
{金尚年, 马永利.}\, 理论力学(第二版). 北京: 高等教育出版社, 2004.

\bibitem{Landau-力学-2013}
{郎道, 栗弗席兹.}\, 力学(第五版). 北京: 高等教育出版社, 2013.

\bibitem{马尔契夫-理论力学-2012}
{马尔契夫.}\, 理论力学(第三版). 北京: 高等教育出版社, 2012.

\bibitem{密歇尔斯基-理论力学习题集-2013}
{密歇尔斯基.}\, 理论力学习题集(第50版). 北京: 高等教育出版社, 2013.

\bibitem{Landau-场论-2012}
{郎道, 栗弗席兹.}\, 场论(第八版). 北京: 高等教育出版社, 2012.

\bibitem{菲赫金哥尔茨-微积分学教程}
{菲赫金哥尔茨.}\, 微积分学教程(第八版). 北京: 高等教育出版社, 2009.

\bibitem{周培源-理论力学}
{周培源.}\, 理论力学. 北京: 科学出版社, 2015.

\end{thebibliography}


\end{document}
